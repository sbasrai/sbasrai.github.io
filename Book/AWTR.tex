\chapterimage{ChapterImageLaboratory.png} % Chapter heading image

\chapter{Background}


\begin{enumerate}
\item Wh
\end{enumerate}
Water makes the world go round and is a critical part of life as we know it. Without water, very little could survive. Water helps all living things grow and survive in their natural environments. From serving up a pitcher of water at a restaurant to watering crops on a massive farm, water is used constantly. However, not all areas of the world are abundant in clean water. Some places even face droughts. All people need access to clean water in order to survive. Dirty water can kill plants, animals and make people very sick.

People in this field determine the best water treatments for safe water. They also establish best processes for handling waste water. Without this role, people would have a hard time extracting and using water like they do today. While people can come from a variety of backgrounds in this field, many have a base in environmental engineering or mechanical engineering. They are proficient in math and science, two subjects significant for water treatment application. Environmental engineers directly help investigate current processes, find flaws and improve the system as a whole.

The processes behind advanced water treatment
There are several processes that make up advanced water treatment. Together, these processes make water a useful, abundant commodity. What are they?

Reverse osmosis
Reverse osmosis involves taking water from the ground and putting it through a process that removes all of the water’s minerals and deionizes it so that it is safe for people to drink. Without this critical process, people would not be able to extract the harmful materials found in natural water and could become sick or die as a result. This process is used in desalinization, which is when ocean water is turned into clean, fresh water. Reverse osmosis helps remove the salt from ocean water leaving behind clean water. The world is running out of fresh and natural resources to use, and 97 percent of the water on the planet is salt water. Reverse osmosis can also help recycle water to make it clean and safe again, and also is used in wastewater treatments.

Reverse osmosis is one of the processes that makes desalination (or removing salt from seawater) possible. Beyond that, reverse osmosis is used for recycling, wastewater treatment, and can even produce energy.

Water issues have become an extremely pressing global threat. With climate change come unprecedented environmental impacts: torrential flooding in some areas, droughts in others, rising and falling sea levels. Add to that the threat of overpopulation -- and the demand and pollution a swelling population brings -- and water becomes one of the paramount environmental issues to watch for in the next generation.

Water treatment plants and systems are now adapting reverse osmosis to address some of these concerns. In Perth, Australia (notably dry and arid, yet surrounded by sea), nearly 17 percent of the area's drinking water is desalinated sea water that comes from a reverse osmosis plant [source: The Economist]. Worldwide, there are now over 13,000 desalination plants in the world, according to the International Desalination Association.

But while knowing that reverse osmosis can convert seawater to drinking water is useful, what we really need to understand is how the heck the process occurs. Assuming that you have a fairly good grasp on the definition of "reverse," we better start by taking a look at how osmosis works before we put the two together.

Let's learn about osmosis by filtering through to the next page.

Contents
What's Osmosis, Anyway?
Osmosis Down, Flip it and Reverse it
Where is Reverse Osmosis Used?
Smaller-scale Applications of Reverse Osmosis
Disadvantages of Reverse Osmosis
What's Osmosis, Anyway?
Can't quite visualize osmosis in action? Here's a handy illustration to help.
Can't quite visualize osmosis in action? Here's a handy illustration to help.
Osmosis is the passage or diffusion of water or other solvents through a semipermeable membrane that blocks the passage of dissolved solutes [source: Encyclopedia Britannica].

What, you don't get it? No fear. Most of us don't, which is why there are countless explanations and analogies to clarify osmosis. We'll explore a few of those, but first let's break osmosis down to its parts to get a grasp on it.

First, we'll make our solution. We start with a boring old cup of water. To spice things up, we'll call water the "solvent" -- which is convenient, because that's what it is. To make our solvent a little tastier, we'll dissolve in some delicious sugar. The sugar is the solute. Just to keep track, we now have water (solvent) that we've dissolved sugar (solute) in, to make sugar water (our solution).

Now that we have our solution of sugar water, we'll grab a U-tube. This is not an internet video of kittens and monkeys hugging; a U-tube is a beaker, shaped in a u-shape. Right in the middle of the tube, imagine a bit of Gore-tex that cuts the U in half. Gore-tex is our "semipermeable membrane." Gore-tex is a thin plastic, dotted with a billion tiny little holes that allow water vapor to pass through, but liquid to stay out. (Saran wrap wouldn't let anything through, and a piece of cotton fabric would let just about anything.)

In one arm of the U-tube, we pour our sugar water mixture. On another we pour our plain old water. That's when the magic of osmosis begins, if you find the movement of water magical. The level of liquid in the sugar water arm will slowly rise, as the solvent (water) moves through the Gore-tex, to make both sides of the arm more equal in a sugar-to-water ratio.

But why does that happen? Simply put, because water wants to find equilibrium. And because the one side of the arm is crowded with sugar, pure water from the other side decides to move on over to make the concentration more equal or until the osmotic pressure (the pressure that happens as the molecules move) is reached.

So there you are; osmosis is when a solvent of low concentrated solute solution moves through a membrane to get to the higher concentrated solution, thus weakening it. You did it!

Now, after showing how it only makes sense for osmosis to work in one direction, let's throw that all out the window and reverse it. Walk backward to the next page to find out more.

Osmosis Down, Flip it and Reverse it
Freddie Mercury and David Bowie both recognized that being under pressure can burn a building down, split a family in two, put people on the street and also create a seriously catchy tune. One thing they left out? That pressure also makes reverse osmosis work.

So we learned that in osmosis, a lower-concentrate solution will filter its solvent to the higher concentrate solution. In reverse osmosis, we are (literally) just reversing the process, by making our solvent filter out of our high concentrate into the lower concentrate solution. So instead of creating a more equal balance of solvent and solute in both solutions, it is separating out solute from solvent.

But as we've explored, that isn't something that solutions really want to do. How do we make reverse osmosis occur? Just like Bowie and Freddie, we put the solution under pressure. Let's take saltwater as an example:

In reverse osmosis, we'd have ourselves a saltwater solution on one side of a tank and pure water on the other side, separated by a semi-permeable membrane. We would apply pressure to the saltwater side of the tank--enough to counteract the natural osmotic pressure from the pure water side, and then to push the saltwater through the filter. (This takes about 50-60 bars of pressure [source: Lenntech]. But because of the size of the salt molecules, only the smaller water molecules would make it to the other side, thus adding fresh water to the water side, and leaving the salt on the other.

And voila, you've seen reverse osmosis. To distill it (ha!): reverse osmosis takes place when pressure applied to a highly concentrated solute solution causes the solvent to pass through a membrane to the lower concentrated solution, leaving a higher concentration of solute on one side, and only solvent on the other.

It's great to be able to define reverse osmosis at dinner parties, but there are surprisingly interesting uses for reverse osmosis that might make more compelling conversation. Let's push our way through to the next page to learn more about what we can do with reverse osmosis.

Where is Reverse Osmosis Used?
These nifty blue guys are reverse osmosis cells, the semipermeable membranes that take the salt out of water in desalination plants.
These nifty blue guys are reverse osmosis cells, the semipermeable membranes that take the salt out of water in desalination plants.
ANDY SOTIRIOU/PHOTODISC/GETTY IMAGES
Unlike osmosis, we can't simply watch reverse osmosis happen in many everyday circumstances. It was only in the 1950s when researchers began exploring how to desalinate ocean water that reverse osmosis was brought up as a possibility. They found that applying pressure to the saltwater side could work to produce more fresh water, but the amount they created was extremely small and not useful on any practical scale. What changed?

A much more advanced filter, created by two UCLA scientists. The hand-cast membranes made from cellular acetate (a polymer used in photograph film) allowed larger quantities of water to move through much faster, and the first reverse osmosis desalination plant began running a small scale operation in Coalinga, California in 1965 [source: The Economist].

Which leads us to one of the most common uses of reverse osmosis we've already discussed: desalination of water. That includes large plants (there are over 100 countries using desalination) or smaller operations--for instance, the kind of filter you might take camping to ensure a healthy drinking supply [source: FDU].

Reverse osmosis is also one of the few ways that we can take certain minerals or chemicals out of a water supply. Some water sources have extremely high levels of natural fluoridation, which can lead to enamel fluorosis (mottled teeth), or the much more severe skeletal fluorosis (an actual bending of a person's bones or skeletal frame). Reverse osmosis can filter out fluoride, or other impurities, on a large scale in a way that a charcoal based filter (like the one most commonly found in homes) can't.

It's also used for recycling purposes; the chemicals used to treat metals for recycling creates harmful wastewater, and reverse osmosis can pull clean water out for better chemical disposal. But even more fun than recycling? Wastewater reverse-osmosis treatments, wherein wastewater goes through the process to create something drinkable. They've nicknamed it "toilet to tap" for a reason, and although it might give you pause, it's a promising ways for developing nations to produce drinkable water.

But reverse osmosis is used in other industries as well; maple syrup, in fact, is produced using osmosis to separate the sugary concentrate from water in sap. The dairy industry uses reverse osmosis filtration to concentrate whey and milk, and the wine industry has begun using it to filter out undesirable elements like some acids, smoke, or to control alcohol content. Reverse osmosis is used to create pure ethanol, free from contaminants.

One more fun thing about reverse osmosis is that the high pressure that makes reverse osmosis effective can actually recycle itself. High pressure pumps force water through, and the remaining salty water is shot out at an extremely high rate. If this off-shoot is put through a turbine or motor, the pressure can be reused to the pumps that initially force the water through, thus re-harvesting energy.

All this industrial jazz is great, but how does reverse osmosis technology affect you, the consumer, on a smaller scale? Find out on the next page.

Smaller-scale Applications of Reverse Osmosis
At some retail stores, you can buy jugs of reverse osmosis-treated water from vending machines.
At some retail stores, you can buy jugs of reverse osmosis-treated water from vending machines.
CARY HERZ/NEWS/GETTY IMAGES
Maybe you've decided that you'd like to get your hands on some delicious reverse osmosis water. Why don't you just pour some water into a reverse osmosis pitcher and enjoy a long, cool drink?

Well, it's not quite that simple. Because reverse osmosis requires a certain amount of pressure, you won't find a reverse osmosis filter pitcher. And if you do want reverse osmosis water running through your entire house, you are essentially committing to buying an entirely new water system. But if you just want reverse osmosis water for drinking or cooking, that doesn't mean you've committed to converting your basement to a mini-industrial reverse osmosis plant.

Your first smaller-scale option is an "under the counter" system. A reverse osmosis system is connected to the water supply under your sink, where the water passes through three to five filters to achieve purity. The filtered water is then stored in a storage tank (also under the sink). An entirely separate faucet is then installed on your sink, fed from the storage tank below. Expect to pay an average of \$200-500 for a system like this. And remember that you're probably doing the installation yourself, so you might want to be fairly confident in your fix-it skills.

Maybe you're little nervous about installing an entire faucet and water system (or perhaps nervous that your landlord might not be thrilled with your DIY resourcefulness). Renters and not-so-handy folks, rejoice. There are also reverse osmosis countertop filters, which allow you to hook up a small filtration system directly from your sink. Simply attach the "feed" line to the faucet, turn the faucet on, and the water is filtered through a small system that's small enough to cram next to the microwave. The purified water line can then be placed in a pitcher for easy, accessible purified water.

But they might not be ideal for everyone; keep in mind that the countertop systems can be quite slow due to lower-flow water faucets, and they'll cost around \$150 at least -- not to mention the cost of changing the filters (about \$30) every few months.

Let's shoot through to the next page to see some more of the drawbacks of reverse osmosis.

Disadvantages of Reverse Osmosis
So now we've seen some of the ways we can harness reverse osmosis to work for us. But does asking nature to reverse itself necessarily a good idea? There are a few issues that arise from using reverse osmosis, and we'll start with checking out what happens in desalination reverse osmosis.

After the water is filtered, you're left with lovely drinking water. But on the other hand, you have a whole mess of salt left to deal with. What do you do with the brine, which usually contains twice the amount of salt as seawater [source: The Economist]? Is it a problem to dump that brine back in the ocean? According to the Australian Centre for Water research, salinity seems to return to normal around 500 meters (about 1,600 feet) from the source [source: The Economist]. However, no one has yet gotten clear answers about if the metals and chemicals also trapped in the brine can cause an environmental impact.

Reverse osmosis systems, in general, are also not entirely self-sustaining. Water must be pretreated with chemicals, for instance, so nothing will clog the fine membrane. And the membrane itself is not entirely easy to deal with; it must be cleaned often, and can trap bacteria. A concern unique to the desalination plants is that small fish or marine life can be sucked into the system; adjusting intake pressures and velocities can usually prevent harm.

The biggest impediment of reverse osmosis filtration systems is the cost. For a developing nation, installing reverse osmosis systems is a fairly impractical possibility. Organizations like the WHO and UNICEF consider building reverse osmosis water treatment plans -- to remove toxins or provide a clean water supply -- part of their mission.

As for individual use, reverse osmosis systems can produce frustratingly little yield. A typical system will only be able to reuse about 5 to 15 percent of the water that's being pumped in, thus leaving up to 85 percent wastewater [source: NDSU].

Reverse osmosis -- and the ways it works and doesn't work -- can be a bit daunting. But if you're thirsty for more reverse osmosis information, go to the next page where you can find a lot more information.

Swim with the Fishes
Thinking you need some reverse osmosis water for your beloved guppies? You might want to think again. While reverse osmosis units can certainly filter out a lot of harmful impurities, you'll also need to add some essential minerals back in that get taken out in the process. Make sure to research what is taken out of water in reverse osmosis, and what minerals your fish need to thrive [source: Foster and Smith Aquatics].

Originally Published: May 8, 2008

Reverse Osmosis FAQ
How does reverse osmosis work?
Reverse osmosis takes place when you apply pressure to a highly concentrated solution, which causes the solvent to pass through a semipermeable membrane to the lower concentrated solution. This leaves behind a higher concentration of solute on one side, and pure solvent on the other.
What does reverse osmosis mean?
Reverse osmosis is a process that makes it possible to remove salt from seawater, which is also called desalination. It uses high pressure and a semipermeable membrane to filter salt and other impurities from water.
Is reverse osmosis water safe to drink?
Whether reverse osmosis water is safe to drink or not long-term is an ongoing debate. Some scientific studies claim that drinking reverse osmosis water can cause harm due to it being more acidic and voids of good minerals. However, it is being adapted by some countries for use in their water treatment plants in order to provide drinking water in areas where there is no or limited potable water.
What are some applications of reverse osmosis?
The process of reverse osmosis is primarily used to remove salt from seawater. It’s also used for recycling, wastewater treatment and for medical applications.
Lots More Information
Related Articles
How Reverse Osmosis Desalinators Work
How Desalination Works
How Water Filters Work
How Recycling Works
Exactly what happens if we run out of water?
What if everyone on Earth had easy access to clean water?
Can the sun's energy be used to clean water?
Why can't we convert salt water into drinking water?
Sources
Share:
Citation
More Awesome Stuff

Up Next
How Reverse Osmosis Desalinators Work
EXPLORE MORE
Special Offer on Antivirus Software From HowStuffWorks and TotalAV Security
Featured
Special Offer on Antivirus Software From HowStuffWorks and TotalAV Security
Try Our Crossword Puzzles!
Featured
Try Our Crossword Puzzles!
Can You Solve This Riddle?
Featured
Can You Solve This Riddle?
Really Rad Reverse Osmosis Quiz
The World
Really Rad Reverse Osmosis Quiz
How Desalination Works
Environmental Science
How Desalination Works
How does desalination work?
Environmental Science
How does desalination work?

You May Like
Is filtered water safer than tap water?
EXPLORE MORE
What If You Drink Saltwater?
Science Vs. Myth
What If You Drink Saltwater?
Why can't we convert salt water into drinking water?
Survival
Why can't we convert salt water into drinking water?
The Ultimate Fluoride Facts Quiz
The World
The Ultimate Fluoride Facts Quiz
Underwater Icicles Are Salty, Weird-Looking and Deadly
Environmental Science
Underwater Icicles Are Salty, Weird-Looking and Deadly
Could You Still Pass Your High School Biology Class?
The World
Could You Still Pass Your High School Biology Class?
The Ultimate Saving Money With a Water Filter Quiz
The World
The Ultimate Saving Money With a Water Filter Quiz

Keep Reading
What If Everyone on Earth Had Easy Access to Clean Water?
EXPLORE MORE
Advertisement


How Reverse Osmosis Desalinators Work
By: Sarah Winkler

&quot;Water, water, everywhere / nor any drop to drink&quot;. See pictures of ocean currents.
"Water, water, everywhere / nor any drop to drink". See pictures of ocean currents.
©ISTOCKPHOTO/MALIKETH
You're setting out for a backpacking expedition and packing up the important gear you'll need for your trip. You've got your compass, a map, some comfortable hiking boots, some snacks and an army knife. Seems like you have everything you'll need, right? Well, you're missing one important item that could save your life in a pinch: a way to purify water. Without water, you're susceptible to dehydration, hypothermia or altitude sickness. A water purification system like a filter or charcoal tablets could provide you with the purified water you'll need to survive in the outdoors.

But what if you need to do more than purify the water? What if your only available water sources are saltwater? (Or as Samuel Taylor Coleridge put it in his poem "The Rime of the Ancient Mariner": "Water, water, everywhere / nor any drop to drink.") Although seawater might look tempting, its high level of salt makes it unsuitable for human consumption. Average ocean seawater contains three times the salt content found in a person's bloodstream. If you drink seawater, you'll become even more dehydrated, which could lead to seizures, kidney failure or even brain damage and death [source: Seawater Facts].

If you're in the outdoors and the only available water is seawater, then you'll need to desalinate the water; that is, you'll need to reduce the salt content of the water. One way to desalinate water is through reverse osmosis with a reverse osmosis desalinator. This filtration process uses pressure to force water through a membrane. The solute (salt) remains on one side of the membrane, while the pure solvent (freshwater) passes to the other side. The solvent (in this case, water) moves from an area of high solute concentration to an area of low solute concentration. While osmosis was discovered as early as the 1700s, it wasn't until the 1960s that scientists were able to use the process to desalinate water [source: Water and Waste Digest]. As its name indicates, this process is the reverse of normal osmosis, in which a solvent moves with no added pressure from an area of low solute concentration to an area of high solute concentration. Not only does a reverse osmosis desalinator remove salt from water, but it also eliminates harmful bacteria and microorganisms.

On the next page, we'll take a look at the science of reverse osmosis desalinators.

Contents
The Science of Reverse Osmosis Desalinators
Desalination Process
Using Reverse Osmosis Desalinators
The Science of Reverse Osmosis Desalinators
To understand the science of reverse osmosis desalinators, you should first become acquainted with a few key terms:

Desalination: Desalination is simply the process of removing salt content from water. During this separation process, the dissolved salt in water is reduced to make the water usable. Although seawater is the largest source of water on our planet, it can't be used for drinking due to its high salt content. Desalination makes seawater fit for human consumption.

Osmosis: Osmosis is a natural phenomenon that affects a variety of biological functions in all forms of life. Osmosis does everything from allow plants to absorb nutrients from the soil to help the kidneys purify blood. An osmotic membrane, a membrane that allows water to pass through at higher levels than it does salt, allows for osmosis to occur. An osmotic membrane is semipermeable; that is, it allows some substances to pass through while others do not. Although pure water can flow freely in both directions, salt and other impurities can't pass through.

During osmosis, solvent water passes through a semi-permeable membrane toward a concentrated substance on the other side until the osmotic pressure across the membrane is equal (usually 350 pounds per square inch guage, or psig, freshwater/seawater) [source: Water and Waste Digest].

Reverse osmosis: Reverse osmosis is just like it sounds -- the exact opposite of osmosis. While in osmosis, solvent water passes through a membrane until the pressure across the membrane is equal, during reverse osmosis, a force with pressure greater than the osmotic pressure is needed to allow freshwater to pass through the membrane while salt is held back. The higher the pressure is above osmotic pressure, the more quickly freshwater will move across the membrane.

So, a reverse osmosis desalinator combines these processes to make saltwater drinkable. On the next page, we'll take a closer look at the reverse osmosis desalination process.

Desalination Process
During reverse osmosis, saltwater is forced through a semipermeable membrane that allows water molecules to pass through while all other impurities, including salt, are held back. Here's a look at the step-by-step process of reverse osmosis desalination:

To set up a reverse osmosis desalinator, you first need an intake pump at the source of the seawater.
Next, you need to create flow through the membrane. This will cause water to pass through the salted side of the membrane to the unsalted side. Pressure comes from a water column on the salted side of the membrane. This will both remove the natural osmotic pressure and create additional pressure on the water column, which will push the water through the membrane. Generally, to desalinate saltwater, you need to get the pressure up to about 50 to 60 bars [source: Lentech].
Feed water is then pumped into a closed container. As the water passes through the membrane, the remaining feed water and salt solution become more concentrated. To reduce the concentration of the remaining dissolved salts, some of the feed water and salt solution is taken out of the container because the dissolved salts in the feed water would continue to increase and thus require more energy to overwhelm the natural osmotic pressure.
Once water is flowing through the membrane, and the pressure is equal on both sides, the desalination process begins. After reverse osmosis has occurred, the water level will be higher on the side where salt was added. The difference in water level is caused by the addition of the salt and is called osmotic pressure; generally, the osmotic pressure of seawater is 26 bars. The quality of water is determined by the pressure, the concentration of salts in the feed water, and the salt permeation constant of the semi-permeable membrane. To improve the quality of the water, you can do a second pass of membrane.
Once the freshwater and saltwater are separated, the freshwater should be stabilized; that is, the pH should be tested to make sure it's fit for consumption.

On the next page, we'll take a look at how reverse osmosis desalinators are used.

 

Using Reverse Osmosis Desalinators
Reverse osmosis desalinators make seawater into delicious freshwater.
Reverse osmosis desalinators make seawater into delicious freshwater.
©ISTOCKPHOTO.COM/ROMAKOSHEL
Reverse osmosis desalinators can operate on both large and small scales. A backpacker or boating enthusiasts can purchase a reverse osmosis desalination system for personal use, or they can be acquired by larger industrial or community groups in need of freshwater.

Many communities in equatorial zones, arid environments and coastal areas are good candidates for reverse osmosis desalination systems because they generally have available seawater but lack freshwater. For example, places like California, Florida, the Caribbean, Central and South America, the Mediterranean, the Middle East and the Pacific Rim are areas in which reverse osmosis desalination could be a viable option for the production of freshwater on a large scale.

In comparison to two other desalination processes, distillation and freeze-thawing, reverse osmosis is the most cost effective and energy efficient. For example, while distillation require 30-186 horsepower of mechanical energy to remove one gallon (3.7 liters) of water from saline solution, reverse osmosis desalination only needs about 0.5-1.4 horsepower [source: Desalination: FAQ].

In addition to being energy efficient, reverse osmosis desalinators are also smaller in size than other desalination units. On a larger scale, they are also less costly to purchase and operate. Most desalinators are run by electricity; however, if electricity is not available or too expensive, you can also use a diesel or solar-powered desalinator. Although solar powered desalinators are initially more expensive to purchase, the energy savings may pay off in the end.

To take care of your reverse osmosis desalinator, make sure to keep an eye on the day-to-day operation of the system. Make sure to adjust the calibration and pumps for leaks or structural damage. The main problem that you can run into with reverse osmosis desalinators is fouling, when membrane pours become clogged. To prevent fouling, clean the unit every four months or so and replace filter elements about once every eight weeks.

Membrane filtration
Membrane filtration is a streamlined process that helps create clean drinking water. This process is often used to improve food quality, as it helps separate particles from water to create other beverages such as beer, milk and juice. There are four different types of membrane filtration, including nanofiltration, ultra-filtration, reverse osmosis and microfiltration. A different type of filtration process is used for different sized particles. The particles found in salt water are the smallest, so reverse osmosis is used. However, the particles in river water might be larger, so microfiltration is used. Though water filtration is used for a variety of reasons, one is to help create beverages and dairy products in the food industry. This process helps concentrate and purify a variety of foods, from beverages such as beer and vegetable juice to dairy products such as yogurt and cheese. This process is used in several stages of food and beverage development so these products are safe to be sold and used.

Water oxidation
Water oxidation is used to break down water into two elements- hydrogen and oxygen. The process separates the water back into its original elements so that it can be used for other things. People and other living organisms need oxygen to live, so this process can be used anywhere where oxygen is readily needed, such as filling up oxygen tanks. Given that climate change and air pollution are currently harming the environment, people are looking for other sources of fuel, one of which is hydrogen. This treatment process helps provide water and hydrogen where it’s needed to improve the environment as a whole.










\section{Water Treatment Plant Types} \index{Water Treatment Plant Types}
\begin{itemize}
\item Centralized Treatment System - The collection and drainage of wastewater, and sometimes stormwater, from a large, generally urban and suburban, area using an extensive network of pumps and piping for transport to a central location for treatment and reclamation, usually near the point of a convenient environmental discharge.

\item Decentralized Treatment System - Collection, treatment, and discharge/reuse of wastewater from individual homes, clusters of homes, isolated communities, industries, or institutional facilities, as well as from portions of existing communities at or near the point of wastewater generation.

\item Satellite Treatment System (aka "Scalping Plant") - Systems where wastewater in an upstream portion of the collection system is intercepted and diverted for treatment in a water reclamation facility located close to the point of reuse.  Satellite treatment systems generally do not have solids-processing facilities; solids removed during treatment are returned to the collection system for processing in a central treatment plant located downstream. 

\item Onsite Septic System - Septic systems are often found in older homes or in remote homes/communities that do not have a sewer collection system nearby.  A septic system is a relatively basic and one of the oldest forms of wastewater treatment.  It only requires a septic tank and a leach field (or drainage field).  Installed and maintained properly, they are a pretty reliable form of treatment.  When something goes wrong, it can be quite an expensive fix, particularly for a single homeowner.
\item Onsite Treatment System - Onsite treatment systems are not common in residential homes, but they do exist.  In general, these onsite treatment systems are more often seen for commercial or industrial businesses, particularly in remote areas, so the recycled water can be reused onsite.
\end{itemize}

\section{Water Recycling Treatment Process} \index{Water Recycling Treatment Process}
A typical wastewater treatment plant consists of:
\begin{itemize}
\item Preliminary Treatment - Removes large solids
\begin{itemize}
\item Coarse screens
\item Grit removal
\end{itemize}
\item Primary Treatment (1° Treatment) - Removes large solids (about 50-60\% of TSS and BOD removal occurs in primary treatment just by settling!)
\item Secondary Treatment (2° Treatment) - Removes organics (BOD, COD) and TSS.  This is the biological treatment process where the majority of nutrients (carbon, nitrogen, and phosphorus) are removed. 
Conventional activated sludge (CAS) has aeration to add dissolved oxygen into the treatment process for aerobic microbes to consume organics
\item Over 50\% of the treatment plant's total energy costs is attributed to the blowers for the aeration treatment process!
\item Secondary clarifier
\item Nutrient Removal (if needed) - Removes nitrogen (and phosphorus)
\item Anoxic Zone [without dissolved oxygen (O2)] is added in front of the aerobic zone for Nitrification / Denitrification (NdN)
\item If biological phosphorus removal is implemented, then an Anaerobic Zone [without dissolved oxygen (O2) or nitrates (NO3)] is added in front of the anoxic zone
The secondary effluent is often ready to be discharged to the nearest available water body for disposal as long as effluent meets NPDES permit requirements.
\end{itemize}
 
A water recycling plant does the same preliminary, primary, and secondary treatment processes and adds additional treatment processes to meet the desired end use for recycled water.
\begin{itemize}
\item Secondary/Tertiary Treatment - Membrane bioreactor (MBR) combines secondary biological treatment with tertiary membrane physical treatment to remove BOD, COD, TSS, and Turbidity
\item Tertiary Treatment (3° Treatment) - Removes TSS and Turbidity
\item Filtration [mono or dual-media, cloth, microfiltration (MF) or ultrafiltration (UF) membrane], coagulation/flocculation and tertiary clarifier, or recharge basin soil aquifer treatment (SAT)
\item Disinfection - Removes Pathogens
\item Chlorine [chlorine gas (Cl2), sodium hypochlorite or "bleach" (NaOCl), chloramine (NH2Cl)], ultraviolet (UV) disinfection, or ozone  (O3) disinfection; pasteurization (not common)
\item Advanced Treatment [Advanced Wastewater Treatment (AWT) or Full Advanced Treatment (FAT)] - Removes Salts and Trace Organics
\begin{itemize}
\item Reverse Osmosis (RO)
\item Ultraviolet-Advanced Oxidation Process (UV-AOP)
\item Biological Activated Carbon (BAC) filter
\item Mineral Stabilization
\end{itemize}
\end{itemize}







\section{Indirect Potable Reuse: Groundwater Replenishment - Surface Application}
\subsection{Requirements}
\begin(enumerate}
\item Wastewater Source Control:\\
A project sponsor shall ensure that the recycled municipal wastewater used for a GRRP shall be from a wastewater management agency that:
(a) administers an industrial pretreatment and pollutant source control program; and
(b) implements and maintains a source control program that includes, at a minimum;
(1) an assessment of the fate of Department-specified and Regional Board-specified chemicals and contaminants through the wastewater and recycled municipal wastewater treatment systems,
(2) chemical and contaminant source investigations and monitoring that focuses on Department-specified and Regional Board-specified chemicals and contaminants,
(3) an outreach program to industrial, commercial, and residential communities within the portions of the sewage collection agency's service area that flows into the water reclamation plant subsequently supplying the GRRP, for the purpose of managing and minimizing the discharge of chemicals and contaminants at the source, and
(4) a current inventory of chemicals and contaminants identified pursuant to this section, including new chemicals and contaminants resulting from new sources or changes to existing sources, that may be discharged into the wastewater collection system.

\item Pathogen Control:\\
(a) A project sponsor shall design and operate a GRRP such that the recycled municipal wastewater used as recharge water for a GRRP receives treatment that achieves at least 12-log enteric virus reduction, 10-log Giardia cyst reduction, and 10-log Cryptosporidium oocyst reduction. The treatment train shall consist of at least three separate treatment processes. Except as provided in subsection (c), for each pathogen (i.e., virus, Giardia cyst, or Cryptosporidium oocyst), a separate treatment process may be credited with no more than 6-log reduction, with at least three processes each being credited with no less than 1.0-log reduction.
(b) At a minimum, the recycled municipal wastewater applied at a GRRP shall receive treatment that meets:
(1) the definition of filtered wastewater, pursuant to section 60301.320; and
(2) the definition of disinfected tertiary recycled water, pursuant to section 60301.230.
(c) For each month retained underground as demonstrated in subsection (e), the recycled municipal wastewater or recharge water will be credited with 1-log virus reduction. A GRRP meeting subsections (b)(1) and (2) or providing advanced treatment in accordance with section 60320.201 for the entire flow of the recycled municipal wastewater used for groundwater replenishment, that also demonstrates at least six months retention underground pursuant to subsection (e), will be credited with 10-log Giardia cyst reduction and 10-log Cryptosporidium oocyst reduction.
(d) With the exception of log reduction credited pursuant to subsection (c), a project sponsor shall validate each of the treatment processes used to meet the requirements in subsection (a) for their log reduction by submitting a report for the Department's review and approval, or by using a challenge test approved by the Department, that provides evidence of the treatment process's ability to reliably and consistently achieve the log reduction. The report and/or challenge test shall be prepared by an engineer licensed in California with at least five years of experience, as a licensed engineer, in wastewater treatment and public water supply, including the evaluation of treatment processes for pathogen control. With the exception of retention time underground and a soil-aquifer treatment process, a project sponsor shall propose and include in its Operation Optimization Plan prepared pursuant to section 60320.122, on-going monitoring using the pathogenic microorganism of concern or a microbial, chemical, or physical surrogate parameter(s) that verifies the performance of each treatment process's ability to achieve its credited log reduction.
(e) To demonstrate the retention time underground in subsection (c), a tracer study utilizing an added tracer shall be implemented under hydraulic conditions representative of normal GRRP operations. The retention time shall be the time representing the difference from when the water with the tracer is applied at the GRRP to when either; two percent (2\%) of the initially introduced tracer concentration has reached the downgradient monitoring point, or ten percent (10\%) of the peak tracer unit value observed at the downgradient monitoring point reached the monitoring point. A project sponsor for a GRRP shall initiate the tracer study prior to the end of the third month of operation. A project sponsor for a GRRP permitted on or before June 18, 2014 that has not already performed such a tracer study shall complete a tracer study demonstrating the retention time underground. With Department approval, an intrinsic tracer may be used in lieu of an added tracer, with no more credit provided than the corresponding virus log reduction in column 2 of Table 60320.108.
(f) For the purpose of siting a GRRP location during project planning and until a GRRP's project sponsor has met the requirements of subsection (e), for each month of retention time estimated using the method in column 1, the recycled municipal wastewater or recharge water shall be credited with no more than the corresponding virus log reduction in column 2 of Table 60320.108.
Table 60320.108
Column 1
Column 2
Method used to estimate the retention time to the nearest downgradient drinking water well
Virus Log Reduction Credit per Month
Tracer study utilizing an added tracer.1
1.0 log
Tracer study utilizing an intrinsic tracer.1
0.67 log
Numerical modeling consisting of calibrated finite element or finite difference models using validated and verified computer codes used for simulating groundwater flow.
0.50 log
Analytical modeling using existing academically-accepted equations such as Darcy's Law to estimate groundwater flow conditions based on simplifying aquifer assumptions.
0.25 log
1 The retention time shall be the time representing the difference from when the water with the tracer is applied at the GRRP to when either; two percent (2\%) of the initially introduced tracer concentration has reached the downgradient monitoring point, or ten percent (10\%) of the peak tracer unit value observed at the downgradient monitoring point reached the monitoring point.
(g) A project sponsor shall obtain Department approval for the protocol(s) to be used to establish the retention times in subsections (e) and (f).
(h) Based on changes in hydrogeological or climatic conditions since the most recent demonstration, the Department may require a GRRP's project sponsor to demonstrate that the underground retention times required in this section are being met.
(i) If a pathogen reduction in subsection (a) is not met based on the ongoing monitoring required pursuant to subsection (d), within 24 hours of being aware a project sponsor shall immediately investigate the cause and initiate corrective actions. The project sponsor shall immediately notify the Department and Regional Board if the GRRP fails to meet the pathogen reduction criteria longer than 4 consecutive hours, or more than a total of 8 hours during any 7-day period. Failures of shorter duration shall be reported to the Regional Board by a project sponsor no later than 10 days after the month in which the failure occurred.
(j) If the effectiveness of a treatment train's ability to reduce enteric virus is less than 10-logs, or Giardia cyst or Cryptosporidium oocyst reduction is less than 8-logs, a project sponsor shall immediately notify the Department and Regional Board, and discontinue application of recycled municipal wastewater at the GRRP, unless directed otherwise by the Department or Regional Board.

\item Nitrogen Compounds Control:\\
o demonstrate control of the nitrogen compounds, a project sponsor shall:
(1) Each week, at least three days apart as specified in the GRRP's Operation Optimization Plan, collect at least two total nitrogen samples (grab or 24-hour composite) representative of the recycled municipal wastewater or recharge water applied throughout the spreading area. Samples may be collected before or after surface application;
(2) Have the samples collected pursuant to paragraph (1) analyzed for total nitrogen, with the laboratory being required by a project sponsor to complete each analysis within 72 hours and have the result reported to a project sponsor within the same 72 hours if the result of any single sample exceeds 10 mg/L;
(3) If the average of the results of two consecutive samples collected pursuant to paragraph (1) exceeds 10 mg/L total nitrogen;
(A) take a confirmation sample and notify the Department and the Regional Board within 48 hours of being notified of the results by the laboratory,
(B) investigate the cause for the exceedances and take actions to reduce the total nitrogen concentrations to ensure continued or future exceedances do not occur, and
(C) initiate additional monitoring for nitrogen compounds as described in the GRRP's Operation Optimization Plan, including locations in the groundwater basin and spreading area, to identify elevated concentrations and determine whether such elevated concentrations exceed or may lead to an exceedance of a nitrogen-based MCL; and
(4) If the average of the results of four consecutive samples collected pursuant to paragraph (1) exceeds 10 mg/L total nitrogen, suspend the surface application of recycled municipal wastewater. Surface application shall not resume until corrective actions have been taken and at least two consecutive total nitrogen sampling results are less than 10 mg/L.
(b) As determined by the Department and based on a GRRP's operation, including but not limited to the time the spreading area is out of service and utilization of a denitrification process, a project sponsor shall initiate additional monitoring for nitrogen compounds to identify elevated concentrations in the groundwater and determine whether such elevated concentrations exceed or may lead to an exceedance of a nitrogen-based MCL.
(c) Following Department and Regional Board approval, a project sponsor may initiate reduced monitoring frequencies for total nitrogen. A project sponsor may apply to the Department and Regional Board for reduced monitoring frequencies for total nitrogen if, for the most recent 24 months:
(1) the average of all results did not exceed 5 mg/L total nitrogen; and
(2) the average of a result and its confirmation sample (taken within 24 hours of receipt of the initial result) did not exceed 10 mg/L total nitrogen.
(d) If the results of reduced monitoring conducted as approved pursuant to subsection (c) exceed the total nitrogen concentration criteria in subsection (c), a project sponsor shall revert to the monitoring frequencies for total nitrogen prior to implementation of the reduced frequencies. Reduced frequency monitoring shall not resume unless the requirements of subsection (c) are met.

\item Other Regulated Contaminants:\\
 Each quarter, as specified in the GRRP's Operation Optimization Plan, a project sponsor shall collect samples (grab or 24-hour composite) representative of the applied recycled municipal wastewater and have the samples analyzed for:
(1) the inorganic chemicals in Table 64431-A, except for nitrogen compounds;
(2) the radionuclide chemicals in Tables 64442 and 64443;
(3) the organic chemicals in Table 64444-A;
(4) the disinfection byproducts in Table 64533-A; and
(5) lead and copper.
(b) Recharge water (including recharge water after surface application) may be monitored in lieu of recycled municipal wastewater to satisfy the monitoring requirements in subsection (a)(4) if the fraction of recycled municipal wastewater in the recharge water is equal to or greater than the average fraction of recycled municipal wastewater in the recharge water applied over the quarter. If the fraction of recycled municipal wastewater in the recharge water being monitored is less than the average fraction of recycled municipal wastewater in the recharge water applied over the quarter, the reported value shall be adjusted to exclude the effects of dilution.
(c) Each year, the GRRP's project sponsor shall collect at least one representative sample (grab or 24-hour composite) of the recycled municipal wastewater or recharge water and have the sample(s) analyzed for the secondary drinking water contaminants in Tables 64449-A and 64449-B.
(d) If a result of the monitoring performed pursuant to subsection (a) exceeds a contaminant's MCL or action level (for lead and copper), a project sponsor shall collect another sample within 72 hours of notification of the result and then have it analyzed for the contaminant as confirmation.
(1) For a contaminant whose compliance with its MCL or action level is not based on a running annual average, if the average of the initial and confirmation sample exceeds the contaminant's MCL or action level, or the confirmation sample is not collected and analyzed pursuant to this subsection, the GRRP's project sponsor shall notify the Department and Regional Board within 24 hours and initiate weekly monitoring until four consecutive weekly results are below the contaminant's MCL or action level. If the running four-week average exceeds the contaminant's MCL or action level, the GRRP's project sponsor shall notify the Department and Regional Board within 24 hours and, if directed by the Department or Regional Board, suspend application of the recycled municipal wastewater.
(2) For a contaminant whose compliance with its MCL is based on a running annual average, if the average of the initial and confirmation sample exceeds the contaminant's MCL, or a confirmation sample is not collected and analyzed pursuant to this subsection, the GRRP shall initiate weekly monitoring for the contaminant until the running four-week average no longer exceeds the contaminant's MCL.
(A) If the running four-week average exceeds the contaminant's MCL, a project sponsor shall describe the reason(s) for the exceedance and provide a schedule for completion of corrective actions in a report submitted to the Department and Regional Board no later than 45 days following the quarter in which the exceedance occurred.
(B) If the running four-week average exceeds the contaminant's MCL for sixteen consecutive weeks, a project sponsor shall notify the Department and Regional Board within 48 hours of knowledge of the exceedance and, if directed by the Department or Regional Board, suspend application of the recycled municipal wastewater.
(e) If the annual average of the results of the monitoring performed pursuant to subsection (c) exceeds a contaminant's secondary MCL in Table 64449-A or the upper limit in Table 64449-B, a project sponsor shall initiate quarterly monitoring of the recycled municipal wastewater for the contaminant and, if the running annual average of quarterly-averaged results exceeds a contaminant's secondary MCL or upper limit, describe the reason(s) for the exceedance and any corrective actions taken in a report submitted to Regional Board no later than 45 days following the quarter in which the exceedance occurred, with a copy concurrently provided to the Department. The annual monitoring in subsection (c) may resume if the running annual average of quarterly results does not exceed a contaminant's secondary MCL or upper limit.
(f) If four consecutive quarterly results for asbestos are below the detection limit in Table 64432-A for asbestos, monitoring for asbestos may be reduced to one sample every three years. Quarterly monitoring shall resume if asbestos is detected.

\end{enumerate}




