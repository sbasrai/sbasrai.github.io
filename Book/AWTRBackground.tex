\chapterimage{ChapterImageLaboratory.png} % Chapter heading image

\chapter{Background}
\begin{itemize}
\item In the United States, various sources discharge nearly 340 billion gallons of water per day,3 including municipal wastewater, industry process water and cooling water,agriculture runoff and return flows, oil and gas produced wastewater, and stormwater (including rainwater capture).

\item About 322 billion gallons per day is withdrawn in the USA from surface water and ground water sources.

\item While reclaimed water cannot be used to meet all needs, there is great opportunity to increase water reuse to enhance the availability and effective use of water resources.

\item Examples of reuse applications include agricultureand irrigation, potable and non-potable water supplies,groundwater storage and recharge, industrial processes,onsite non-potable use, saltwater intrusion barriers, and environmental restoration.
\end{itemize}

 potable water reuse applications include indirect potable reuse (groundwater replenishment and reservoir water augmentation). 
 
 22 defines the following approved potable uses:
\begin{itemize}
\item Indirect potable reuse (IPR)
\begin{itemize}
\item Groundwater replenishment: the planned use of recycled municipal wastewater that is operated for the purpose of replenishing a groundwater basin designated as a source of municipal and domestic water supply.
\begin{itemize}
\item Surface (spreading) application the application of recharge water to a spreading area for infiltration resulting in the recharge of a groundwater basin or aquifer.
\item Subsurface application the application of recharge water to a groundwater basin(s) by a means other than surface application.
\end{itemize}
\item Reservoir water augmentation: the planned use of recycled municipal wastewater into a surface water reservoir used as a source of domestic drinking water supply.
\end{itemize}
\end{itemize}

\section{Benefits of Water Reuse}
\subsection{Benefits}
Benefits include:
\begin{itemize}
\item improved agricultural production
\item reduced energy consumption associated with production
\item treatment, and distribution of water
\item significant environmental benefits, such as reduced nutrient loads to receiving waters due to reuse of the treated wastewater.
\end{itemize}

\subsection{Drivers}
\begin{itemize}
\item Increases in population and a dependency on high-water-demand agriculture
\item Increasing urbanization; all of these factors and others are effecting 
\item land use changes that exacerbate water supply challenges. 
\item sea level rise and increasing intensity and variability of local climate patterns are predicted to alter hydrologic and ecosystem dynamics and composition
\end{itemize}
Other drivers:
\begin{itemize}
\item Energy efficiency
The water-energy nexus recognizes that water and energy are mutually dependent—energy production requires large volumes of water, and water infrastructure requires large amounts of energy

Water reuse is a critical factor in slowing the compound loop of increased water and energy use witnessed in the water-energy nexus. A frequently-cited definition of sustainability comes from a 1987 report by the Bruntland Commission: “Sustainable development is development that meets the needs of the present without compromising the ability of future generations to meet their own needs” (WCED, 1987). Therefore, sustainable water management can be defined as water resource management that meets the needs of present and future generations.

\end{itemize}
\begin{itemize}
\item increasing need to meet potable water supply demands
\item other urban demands (e.g., landscape irrigation, commercial, and industrial needs)
\item increased agricultural demands due to greater incorporation of animal and dairy products into the diet
\item increase demands on water for food production (Pimentel and Pimentel, 2003). 
\begin{itemize}
\item Potable reuse – Recycled or reclaimed water that is safe for drinking.  There are two types of potable reuse:
\begin{itemize}
\item Indirect Potable Reuse (IPR) - The planned incorporation of reclaimed water into a raw water supply such as in potable water storage reservoirs or a groundwater aquifer, resulting in mixing and assimilation, thus providing an environmental buffer.
\item Direct Potable Reuse (DPR) - The introduction of highly treated reclaimed water either directly into the potable water supply distribution system downstream of a water treatment plant, or into the raw water supply immediately upstream of a water treatment plant.
\end{itemize}
\item While the terms IPR and DPR are still used, there is a transition in terminology to be more specific about the type of potable reuse:
\begin{itemize}
\item Groundwater augmentation (GWA) - Advanced treated water is recharged/injected into the groundwater basin used as a water supply.
\item Surface (reservoir) water augmentation (SWA) - Advanced treated water (+) is discharged into a surface reservoir used as raw water supply. 
\item Raw water augmentation (RWA) - Advanced treated water (++) is discharged directly upstream of a drinking water treatment plant.
\item Treated water augmentation (TWA) - Advanced treated water (+++) is discharged downstream of the drinking water treatment plant directly into the potable water distribution system.
\end{itemize}
\end{itemize}
\end{itemize}