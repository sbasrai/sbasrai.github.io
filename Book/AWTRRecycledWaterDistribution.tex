\chapterimage{ChapterImageLaboratory.png} % Chapter heading image

\chapter{Recycled Water Distribution}


\section{Recycled Water General Requirements} \index{Recycle Water General Requirements}
\subsection{Recycled Water Use Constraints} \index{Recycle Water Use Constraints}
(a) No irrigation with disinfected tertiary recycled water shall take place within 50 feet of any domestic water supply well unless all of the following conditions have been met:
(1) A geological investigation demonstrates that an aquitard exists at the well between the uppermost aquifer being drawn from and the ground surface.
(2) The well contains an annular seal that extends from the surface into the aquitard.
(3) The well is housed to prevent any recycled water spray from coming into contact with the wellhead facilities.
(4) The ground surface immediately around the wellhead is contoured to allow surface water to drain away from the well.
(5) The owner of the well approves of the elimination of the buffer zone requirement.
(b) No impoundment of disinfected tertiary recycled water shall occur within 100 feet of any domestic water supply well.
(c) No irrigation with, or impoundment of, disinfected secondary-2.2 or disinfected secondary-23 recycled water shall take place within 100 feet of any domestic water supply well.
(d) No irrigation with, or impoundment of, undisinfected secondary recycled water shall take place within 150 feet of any domestic water supply well.
(e) Any use of recycled water shall comply with the following:
(1) Any irrigation runoff shall be confined to the recycled water use area, unless the runoff does not pose a public health threat and is authorized by the regulatory agency.
(2) Spray, mist, or runoff shall not enter dwellings, designated outdoor eating areas, or food handling facilities.
(3) Drinking water fountains shall be protected against contact with recycled water spray, mist, or runoff.
(f) No spray irrigation of any recycled water, other than disinfected tertiary recycled water, shall take place within 100 feet of a residence or a place where public exposure could be similar to that of a park, playground, or school yard.
(g) All use areas where recycled water is used that are accessible to the public shall be posted with signs that are visible to the public, in a size no less than 4 inches high by 8 inches wide, that include the following wording: “RECYCLED WATER - DO NOT DRINK”. Each sign shall display an international symbol similar to that shown in figure 60310-A. The Department may accept alternative signage and wording, or an educational program, provided the applicant demonstrates to the Department that the alternative approach will assure an equivalent degree of public notification.
(h) Except as allowed under section 7604 of title 17, California Code of Regulations, no physical connection shall be made or allowed to exist between any recycled water system and any separate system conveying potable water.
(i) Except for use in a cemetery that complies with the requirements of section 8118 of the Health and Safety Code, the portions of the recycled water piping system that are in areas subject to access by the general public shall not include any hose bibs. Only quick couplers that differ from those used on the potable water system shall be used on the portions of the recycled water piping system in areas subject to public access.
Credits

\subsection{Recycled Water Dual Water System Delivery} \index{Recycle Water Dual Water System Delivery}
No person other than a recycled water agency shall deliver recycled water to a dual-plumbed facility.
(b) Except as allowed pursuant to section 13553(d) of the Water Code, a recycled water agency shall not deliver recycled water for any internal use to any individually-owned residential units including free-standing structures, multiplexes, or condominiums.
(c) No recycled water agency shall deliver recycled water for internal use except for fire suppression systems, to any facility that produces or processes food products or beverages. For purposes of this Subsection, cafeterias or snack bars in a facility whose primary function does not involve the production or processing of foods or beverages are not considered facilities that produce or process foods or beverages.
(d) No recycled water agency shall deliver recycled water to a facility using a dual plumbed system unless the report required pursuant to section 13522.5 of the Water Code, and which meets the requirements set forth in section 60314, has been submitted to, and approved by, the regulatory agency.
Credits

\subsection{Recycled Water Dual Water System Design Requirements} \index{Recycle Water Dual Water System Design Requirements}

The public water supply shall not be used as a backup or supplemental source of water for a dual-plumbed recycled water system unless the connection between the two systems is protected by an air gap separation which complies with the requirements of sections 7602(a) and 7603(a) of title 17, California Code of Regulations, and the approval of the public water system has been obtained.


\subsection{Recycled Water Dual Water System Operation Requirements} \index{Recycle Water Dual Water System Operation Requirements}
(a) Prior to the initial operation of the dual-plumbed recycled water system and annually thereafter, the Recycled Water Agency shall ensure that the dual plumbed system within each facility and use area is inspected for possible cross connections with the potable water system. The recycled water system shall also be tested for possible cross connections at least once every four years. The testing shall be conducted in accordance with the method described in the report submitted pursuant to section 60314. The inspections and the testing shall be performed by a cross connection control specialist certified by the California-Nevada section of the American Water Works Association or an organization with equivalent certification requirements. A written report documenting the result of the inspection or testing for the prior year shall be submitted to the department within 30 days following completion of the inspection or testing.
(b) The recycled water agency shall notify the department of any incidence of backflow from the dual-plumbed recycled water system into the potable water system within 24 hours of the discovery of the incident.
(c) Any backflow prevention device installed to protect the public water system serving the dual-plumbed recycled water system shall be inspected and maintained in accordance with section 7605 of Title 17, California Code of Regulations.