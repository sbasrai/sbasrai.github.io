\chapterimage{ChapterImageLaboratory.png} % Chapter heading image

\chapter{Sources and Uses of Recycled Water}

\section{Uses of Recycled Water} \index{Used of Recycled Water}
\begin{enumerate}
\item Irrigation:\\
(a) Recycled water used for the surface irrigation of the following shall be a disinfected tertiary recycled water, except that for filtration pursuant to Section 60301.320(a) coagulation need not be used as part of the treatment process provided that the filter effluent turbidity does not exceed 2 NTU, the turbidity of the influent to the filters is continuously measured, the influent turbidity does not exceed 5 NTU for more than 15 minutes and never exceeds 10 NTU, and that there is the capability to automatically activate chemical addition or divert the wastewater should the filter influent turbidity exceed 5 NTU for more than 15 minutes:
(1) Food crops, including all edible root crops, where the recycled water comes into contact with the edible portion of the crop,
(2) Parks and playgrounds,
(3) School yards,
(4) Residential landscaping,
(5) Unrestricted access golf courses, and
(6) Any other irrigation use not specified in this section and not prohibited by other sections of the California Code of Regulations.
(b) Recycled water used for the surface irrigation of food crops where the edible portion is produced above ground and not contacted by the recycled water shall be at least disinfected secondary-2.2 recycled water.
(c) Recycled water used for the surface irrigation of the following shall be at least disinfected secondary-23 recycled water:
(1) Cemeteries,
(2) Freeway landscaping,
(3) Restricted access golf courses,
(4) Ornamental nursery stock and sod farms where access by the general public is not restricted,
(5) Pasture for animals producing milk for human consumption, and
(6) Any nonedible vegetation where access is controlled so that the irrigated area cannot be used as if it were part of a park, playground or school yard
(d) Recycled wastewater used for the surface irrigation of the following shall be at least undisinfected secondary recycled water:
(1) Orchards where the recycled water does not come into contact with the edible portion of the crop,
(2) Vineyards where the recycled water does not come into contact with the edible portion of the crop,
(3) Non food-bearing trees (Christmas tree farms are included in this category provided no irrigation with recycled water occurs for a period of 14 days prior to harvesting or allowing access by the general public),
(4) Fodder and fiber crops and pasture for animals not producing milk for human consumption,
(5) Seed crops not eaten by humans,
(6) Food crops that must undergo commercial pathogen-destroying processing before being consumed by humans, and
(7) Ornamental nursery stock and sod farms provided no irrigation with recycled water occurs for a period of 14 days prior to harvesting, retail sale, or allowing access by the general public.
(e) No recycled water used for irrigation, or soil that has been irrigated with recycled water, shall come into contact with the edible portion of food crops eaten raw by humans unless the recycled water complies with subsection (a).
\item Impoundments:\\
(a) Except as provided in subsection (b), recycled water used as a source of water supply for nonrestricted recreational impoundments shall be disinfected tertiary recycled water that has been subjected to conventional treatment.
(b) Disinfected tertiary recycled water that has not received conventional treatment may be used for nonrestricted recreational impoundments provided the recycled water is monitored for the presence of pathogenic organisms in accordance with the following:
(1) During the first 12 months of operation and use the recycled water shall be sampled and analyzed monthly for Giardia, enteric viruses, and Cryptosporidium. Following the first 12 months of use, the recycled water shall be sampled and analyzed quarterly for Giardia, enteric viruses, and Cryptosporidium. The ongoing monitoring may be discontinued after the first two years of operation with the approval of the department. This monitoring shall be in addition to the monitoring set forth in section 60321.
(2) The samples shall be taken at a point following disinfection and prior to the point where the recycled water enters the use impoundment. The samples shall be analyzed by an approved laboratory and the results submitted quarterly to the regulatory agency.
(c) The total coliform bacteria concentrations in recycled water used for nonrestricted recreational impoundments, measured at a point between the disinfection process and the point of entry to the use impoundment, shall comply with the criteria specified in section 60301.230 (b) for disinfected tertiary recycled water.
(d) Recycled water used as a source of supply for restricted recreational impoundments and for any publicly accessible impoundments at fish hatcheries shall be at least disinfected secondary-2.2 recycled water.
(e) Recycled water used as a source of supply for landscape impoundments that do not utilize decorative fountains shall be at least disinfected secondary-23 recycled water.

\item Cooling:\\
(a) Recycled water used for industrial or commercial cooling or air conditioning that involves the use of a cooling tower, evaporative condenser, spraying or any mechanism that creates a mist shall be a disinfected tertiary recycled water.
(b) Use of recycled water for industrial or commercial cooling or air conditioning that does not involve the use of a cooling tower, evaporative condenser, spraying, or any mechanism that creates a mist shall be at least disinfected secondary-23 recycled water.
(c) Whenever a cooling system, using recycled water in conjunction with an air conditioning facility, utilizes a cooling tower or otherwise creates a mist that could come into contact with employees or members of the public, the cooling system shall comply with the following:
(1) A drift eliminator shall be used whenever the cooling system is in operation.
(2) A chlorine, or other, biocide shall be used to treat the cooling system recirculating water to minimize the growth of Legionella and other micro-organisms.

\item Other Uses:\\
 Recycled water used for the following shall be disinfected tertiary recycled water, except that for filtration being provided pursuant to Section 60301.320(a) coagulation need not be used as part of the treatment process provided that the filter effluent turbidity does not exceed 2 NTU, the turbidity of the influent to the filters is continuously measured, the influent turbidity does not exceed 5 NTU for more than 15 minutes and never exceeds 10 NTU, and that there is the capability to automatically activate chemical addition or divert the wastewater should the filter influent turbidity exceed 5 NTU for more than 15 minutes:
(1) Flushing toilets and urinals,
(2) Priming drain traps,
(3) Industrial process water that may come into contact with workers,
(4) Structural fire fighting,
(5) Decorative fountains,
(6) Commercial laundries,
(7) Consolidation of backfill around potable water pipelines,
(8) Artificial snow making for commercial outdoor use, and
(9) Commercial car washes, including hand washes if the recycled water is not heated, where the general public is excluded from the washing process.
(b) Recycled water used for the following uses shall be at least disinfected secondary-23 recycled water:
(1) Industrial boiler feed,
(2) Nonstructural fire fighting,
(3) Backfill consolidation around nonpotable piping,
(4) Soil compaction,
(5) Mixing concrete,
(6) Dust control on roads and streets,
(7) Cleaning roads, sidewalks and outdoor work areas and
(8) Industrial process water that will not come into contact with workers.
(c) Recycled water used for flushing sanitary sewers shall be at least undisinfected secondary recycled water.
\end{enumerate}


De facto reuse: A situation where reuse of treated wastewater is, in fact, practiced but is not officially recognized (e.g., a drinking water supply intake located downstream from a wastewater treatment plant [WWTP] discharge point).

Direct potable reuse (DPR): The introduction of reclaimed water (with or without retention in an engineered storage buffer) directly into a drinking water treatment plant, either collocated or remote from the advanced wastewater treatment system.

Indirect potable reuse (IPR): Augmentation of a drinking water source (surface or groundwater) with reclaimed water followed by an environmental buffer that precedes drinking water treatment.

Nonpotable reuse: All water reuse applications that do not involve potable reuse.
Potable reuse: Planned augmentation of a drinking water supply with reclaimed water.

Reclaimed water: Municipal wastewater that has been treated to meet specific water quality criteria with the intent of being used for a range of purposes. The term recycled water is synonymous with reclaimed water.

Water reclamation: The act of treating municipal wastewater to make it acceptable for reuse.
Water reuse: The use of treated municipal wastewater (reclaimed water). Other alternate sources of water, including graywater and stormwater, are discussed in Chapter 2.

Wastewater: Used water discharged from homes, business, industry, and agricultural facilities.