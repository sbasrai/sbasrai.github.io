\chapterimage{ChapterImageLaboratory.png} % Chapter heading image

\chapter{Treatment Methods}

\section{Ways of Producing Recycled Water} \index{Ways to Produce Recycle Water}

\begin{enumerate}

\item Recycled I - Disinfected Secondary\\

Disinfected secondary effluent can effectively treat COD, BOD, and Pathogens.  However, it does not have a significant impact on any remaining constituents.  This is the lowest level of water reclamation/recycling and has no filtration step.
\begin{itemize}
\item Organics - COD, BOD, TOC
\item Nutrients - TKN, NH3, NO3, NO2, TP, PO4
\item Salts - Na, K, Ca, Mg, Cl, SO4, HCO3
\item Trace Constituents - Boron, Miscellaneous Metals, Hormones, EDCs, PCPs, etc.
\item Pathogens
\end{itemize}

According to Title 22, §60301.225. Disinfected secondary-23 recycled water:\\

"Disinfected secondary-23 recycled water" means recycled water that has been oxidized and disinfected so that the median concentration of total coliform bacteria in the disinfected effluent does not exceed a most probable number (MPN) of 23 per 100 milliliters utilizing the bacteriological results of the last 7 days for which analyses have been completed, and the number of total coliform bacteria does not exceed an MPN of 240 per 100 milliliters in more than one sample in any 30 day period.
 
\item Recycled II - Tertiary with No Nutrient Removal\\

Disinfected tertiary treatment with no nutrient removal can effectively treat COD, BOD, and Pathogens, but does not have a significant impact on the remaining constituents.  This is similar to disinfected secondary.  However, it does have a filtration step, which captures smaller solids particles to reduce turbidity and TSS.  In removing these smaller solids, it may provide better treatment of certain constituents that may be attached or adhered to these solids, such as pathogens.  This is the second lowest level of water reclamation/recycling.
\begin{itemize}
\item Organics - COD, BOD, TOC
\item Nutrients - TKN, NH3, NO3, NO2, TP, PO4
\item Salts - Na, K, Ca, Mg, Cl, SO4, HCO3
\item Trace Constituents - Boron, Miscellaneous Metals, Hormones, EDCs, PCPs, etc.
\item Pathogens
\end{itemize}

Tertiary treatment no nutrient removal with filtration has certain other considerations:
\begin{itemize}
\item Conventional secondary effluent has a relatively low solids retention time (SRT), which essentially translates to smaller floc (clusters of solids) formation.  The smaller the floc, the more difficult it is to filter out the solids; therefore, some solids are more likely pass through the filter.
\item If chlorine is used as the disinfection treatment process, then the ammonia (NH3) can combine with chlorine to form chloramines (NH2Cl, NHCl2), or), which can provide assistance as a disinfectant and can provide longer-lasting disinfection.  
\item However, if the NH3 concentration is low, then there can be challenges with chlorine demand.
\end{itemize}
According to Title 22, §60301.230. Disinfected tertiary recycled water:
\begin{enumerate}
\item The filtered wastewater has been disinfected by either:
\begin{enumerate}
\item A chlorine disinfection process following filtration that provides a CT (the product of total chlorine residual and modal contact time measured at the same point) value of not less than 450 milligram-minutes per liter at all times with a modal contact time of at least 90 minutes, based on peak dry weather design flow; or
\item A disinfection process that, when combined with the filtration process, has been demonstrated to inactivate and/or remove 99.999 percent of the plaque forming units of F-specific bacteriophage MS2, or polio virus in the wastewater.  A virus that is at least as resistant to disinfection as polio virus may be used for purposes of the demonstration.
\end{enumerate}
\item The median concentration of total coliform bacteria measured in the disinfected effluent does not exceed an MPN of 2.2 per 100 milliliters utilizing the bacteriological results of the last seven days for which analyses have been completed and the number of total coliform bacteria does not exceed an MPN of 23 per 100 milliliters in more than one sample in any 30 day period.  No sample shall exceed an MPN of 240 total coliform bacteria per 100 milliliters.
\end{enumerate}
According to Title 22, §60301.320. Filtered wastewater:
"Filtered wastewater" means an oxidized wastewater that meets the criteria in subsection (a) or (b):
\begin{enumerate}
\item Has been coagulated and passed through natural undisturbed soils or a bed of filter media pursuant to the following:
\begin{enumerate}
\item At a rate that does not exceed 5 gallons per minute per square foot of surface area in mono, dual or mixed media gravity, upflow or pressure filtration systems, or does not exceed 2 gallons per minute per square foot of surface area in traveling bridge automatic backwash filters; and
\item So that the turbidity of the filtered wastewater does not exceed any of the following:
\end{enumerate}
\begin{enumerate}
\item An average of 2 NTU within a 24-hour period;
\item 5 NTU more than 5 percent of the time within a 24-hour period; and
\item 10 NTU at any time.
\end{enumerate}
\item Has been passed through a microfiltration, ultrafiltration, nanofiltration, or reverse osmosis membrane so that the turbidity of the filtered wastewater does not exceed any of the following:
\begin{enumerate}
\item 0.2 NTU more than 5 percent of the time within a 24-hour period; and
\item 0.5 NTU at any time.
\end{enumerate}
\end{enumerate}

\item Recycled III - Tertiary with Nutrient Removal\\

Disinfected tertiary treatment with nutrient removal can effectively treat the same constituents as the previous types of recycled water (COD, BOD, and Pathogens), but it also treats nutrients (TKN, NH3, NO3, NO2)!  Other remaining constituents are not significantly impacted.
\begin{itemize}
\item Organics - COD, BOD, TOC
\item Nutrients - TKN, NH3, NO3, NO2, TP, PO4
\item Salts - Na, K, Ca, Mg, Cl, SO4, HCO3
\item Trace Constituents - Boron, Miscellaneous Metals, Hormones, EDCs, PCPs, etc.
\item Pathogens
\end{itemize}

Note that in the wastewater treatment industry, the term "nutrient removal" is often used even when only talking about nitrogen removal  (TKN, NH3, NO3, NO2) and does not generally include phosphorus (TP, PO4).  This is primarily because it is relatively common for treatment plants to have effluent nitrogen limits, but not phosphorus limits.  High levels of nitrogen or phosphorus can potentially cause issues to the receiving water body.  Many water agencies closely follow developing rules and regulations to anticipate the likelihood of receiving stricter nutrient (N or P) limits in their NPDES/WDR permits from the RWQCB.  If a facility does remove nitrogen and phosphorus, then it is often referred to as "nutrient removal and phosphorus removal" or "enhanced biological phosphorus removal (EBPR)."

In 2002, the Elsinore Valley Municipal Water District (EVMWD) received a permit to discharge recycled water from its Regional Wastewater Treatment Plant into Lake Elsinore as part of a 2-year pilot project to research the effects of recycled water.  Lake Elsinore is a natural recreational lake and is subject to high losses due to evaporation (about 14,000 AFY).  The permit required the recycled water to meet a total nitrogen (TN) limit of 3 mg/L and total phosphorus (TP) limit of 0.5 mg/L.  About 2,000 AF of recycled water was discharged into Lake Elsinore for the remaining six months of 2002.  This is the first time that recycled water was released into a recreational lake in California!  This practice continues on today!  
 
\item Recycled IV - Tertiary with Ultrafiltration (UF) or Microfiltration (MF) and Nutrient Removal\\

Disinfected tertiary treatment with ultrafiltration (UF) or microfiltration (MF) membranes and nutrient removal can effectively treat the same constituents as the previous types of recycled water (COD, BOD, Pathogens, TKN, NH3, NO3, NO2) and to a much improved water quality.  This is because previous types of recycled water referred to filtration, which is often a type of monomedia or dual media consisting of sand and/or anthracite (i.e. coal) or a cloth media filter.  While this does provide effective filtration, it is typically in the particulate filtration range to capture particle sizes of 1 micron (micrometer or µm) or larger.  Additionally, channels can develop in media filters whereby allowing even larger particle sizes to pass through. UF or MF refers to filtration via membranes, which have a more consistent pore size (0.01 to 0.1 µm) that limits the size of particles that can pass through the filter.  By removing smaller solids particles via UF or MF membranes, the water quality produced is significantly improved and disinfection efficiency is increased to remove more pathogens.  Other remaining constituents (salts and trace) are not significantly impacted.
\begin{itemize}
\item Organics - COD, BOD, TOC
\item Nutrients - TKN, NH3, NO3, NO2, TP, PO4
\item Salts - Na, K, Ca, Mg, Cl, SO4, HCO3
\item Trace Constituents - Boron, Miscellaneous Metals, Hormones, EDCs, PCPs, etc.
\item Pathogens
\end{itemize}

This type of recycled water is generally produced from more advanced treatment technologies, such as a membrane bioreactor (MBR).  MBR uses both biological processes and membrane technology to treat and filter water.  Organic matter (and potentially nutrients) are removed using biological treatment processes, then membranes filter out the microscopic particles and microorganisms to produce permeate (i.e. water that permeated through the membrane) for downstream disinfection.
 
\item Recycled V - Tertiary with Advanced Oxidation Process (AOP)\\

Disinfected tertiary treatment with advanced oxidation processes (AOP) can effectively treat the same constituents as the previous types of recycled water (COD, BOD, Pathogens, TKN, NH3, NO3, NO2) and it can also treat potentially harmful dissolved organics (TOC) and some trace constituents (boron, miscellaneous metals, hormones, EDCs, PCPs, etc.).  Trace constituents are somewhat difficult to treat considering that they are in the parts per billion (µg/L) range as opposed to typical conventional constituents that are in the parts per million (mg/L) range.
  
\begin{itemize}
\item Organics - COD, BOD, TOC
\item Nutrients - TKN, NH3, NO3, NO2, TP, PO4
\item Salts - Na, K, Ca, Mg, Cl, SO4, HCO3
\item Trace Constituents - Boron, Miscellaneous Metals, Hormones, EDCs, PCPs, etc.
\item Pathogens
\end{itemize}

AOP treats trace constituents by introducing or generating a strong oxidant into the flow that will break down (i.e., break apart) chemicals and dissolved organic material so they are no longer harmful and disinfect water.  Biological Activated Carbon (BAC) filters can in conjunction with AOP to remove dissolved organic material whereby organics stick to the filter and beneficial bacteria, which thrive in a high oxygen environment and live on the filters, eat the organics.
 
\item Recycled VI - Tertiary with AOP and Desalination\\

Disinfected tertiary treatment with AOP and desalination is the highest level of treatment and can effectively treat all typical constituents found in wastewater. 
\begin{itemize}
\item Organics - COD, BOD, TOC
\item Nutrients - TKN, NH3, NO3, NO2, TP, PO4
\item Salts - Na, K, Ca, Mg, Cl, SO4, HCO3
\item Trace Constituents - Boron, Miscellaneous Metals, Hormones, EDCs, PCPs, etc.
\item Pathogens
\end{itemize}

The Orange County Water District (OCWD), West Basin Municipal Water District (WBMWD), and Water Replenishment District (WRD) are a few local Southern California water agencies that produce this type of advanced treated recycled water for groundwater augmentation (GWA) (which is also called groundwater replenishment or groundwater recharge) and/or seawater intrusion barrier.  This water can either be applied to surface recharge basins or injected into the groundwater aquifer via injection wells.

The advanced treatment process at OCWD's Groundwater Replenishment System (GWRS) includes MF, RO, AOP, UV, and mineral stabilization.  Later on in the course, we'll take virtual tour of their treatment process and how they use the advanced treated water, which is actually cleaner than your drinking water! 



\end{enumerate}