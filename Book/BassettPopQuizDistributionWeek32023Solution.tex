\documentclass{article}

%\usepackage[english]{babel}%

\usepackage{graphicx}

\usepackage{tabulary}

\usepackage{tabularx}

\usepackage[table,xcdraw]{xcolor}

\usepackage{pdflscape}

\usepackage{float}

\usepackage{lastpage}

\usepackage{multirow}

\usepackage{xcolor}

\usepackage{cancel}

\usepackage{amsmath}

\usepackage[table]{xcolor}

\usepackage{fixltx2e}

\usepackage[T1]{fontenc}

\usepackage[utf8]{inputenc}

\usepackage{ifthen}

\usepackage{fancyhdr}

\usepackage[document]{ragged2e}

\usepackage[margin=1in,top=1.2in,headheight=57pt,headsep=0.1in]
{geometry}

\usepackage{ifthen}

\usepackage{fancyhdr}

\everymath{\displaystyle}

\usepackage[document]{ragged2e}

\usepackage{fancyhdr}

\usepackage{mathabx}

\usepackage[shortlabels]{enumitem}
\usepackage{tikz}
\usepackage{mwe}
\usetikzlibrary{calc}
\usetikzlibrary{shapes.multipart, shapes.geometric, arrows}
\usetikzlibrary{calc, decorations.markings}
\usetikzlibrary{arrows.meta}
\usetikzlibrary{shapes,snakes}
\usetikzlibrary{quotes,angles, positioning}

\everymath{\displaystyle}

\linespread{2}%controls the spacing between lines. Bigger fractions means crowded lines%

%\pagestyle{fancy}

%\usepackage[margin=1 in, top=1in, includefoot]{geometry}

%\everymath{\displaystyle}

\linespread{1.3}%controls the spacing between lines. Bigger fractions means crowded lines%

%\pagestyle{fancy}

\pagestyle{fancy}

\setlength{\headheight}{56.2pt}

\usepackage{soul}

 \graphicspath{ {./images/} }

\chead{\ifthenelse{\value{page}=1}{\includegraphics[scale=0.3]{BassettCTCLogo}\\ \textbf \textbf Pop Quiz 1/24/2023}}

\rhead{\ifthenelse{\value{page}=1}{Solution}{}}

\lhead{\ifthenelse{\value{page}=1}{Water Distribution - January 2023}{Pop Quiz 1/24/2023}}

\rfoot{\ifthenelse{\value{page}=1}{}{}}

 

\cfoot{}

\lfoot{Page \thepage\ of \pageref{LastPage}}

\renewcommand{\headrulewidth}{2pt}

\renewcommand{\footrulewidth}{1pt}

\begin{document}

 


\begin{enumerate}

\item An operator mixes 40 lb of lime in a 100-gal tank containing 80 gal of water. What is the percent of lime in the slurry?\\
\vspace{0.3cm}
Solution:\\
\vspace{0.3cm}
Method 1: Using lbs formula: $\mathrm{mg/l}=\dfrac{40}{\dfrac{80}{1,000,000}*8.34}=59,952=60,000mg/l=\boxed{6\%}$

\vspace{0.3cm}

Method 2: Using unit conversion: $\dfrac{40 \enspace  \mathrm{lbs}}{80 \enspace\mathrm{gal}}*\dfrac{454 \enspace  \mathrm{gms}}{\mathrm{lb}}*\dfrac{1000 \enspace  \mathrm{mg}}{\mathrm{gm}}*\dfrac{\mathrm{gal}}{3.785 \enspace  \mathrm{l}}=60,000mg/l=\boxed{6\%}$

\item Find the detention time in minutes for a clarifier that has a diameter of 152 ft and a water depth of 8.22 ft, if the flow rate is 6.8 mgd.
\vspace{0.4cm}

Solution: 
$
\mathrm{DT}=\dfrac{Volume}{Flow}=\dfrac{0.785*152^2*8.22\enspace  \mathrm{ft^3}}{\dfrac{6.8*1,000,000 \enspace  \mathrm{gal}}{\mathrm{day}}*\dfrac{ft^3}{7.48 \enspace  \mathrm{gal}}*\dfrac{\mathrm{day}}{1,440 \enspace  \mathrm{min}}}=\boxed{236 \mathrm{~min}}
$

\vspace{0.3cm}

\item Water is flowing at a velocity of 1.70 fps in a 10-in. diameter pipe. If the pipe changes from the 10-in. to a 6-in. pipe, what will the velocity be in the 6-in. pipe?

\vspace{0.3cm}
Solution:\\
\vspace{0.3cm}
Method 1: Finding the Q given the velocity in the 10 in. pipe and then using that Q to find velocity in the 6 in. pipe.  Note:  The Q remains the same, only the velocity will change as the pipe becomes smaller.\\
\vspace{0.3cm}
Step A - Finding the flow rate:\\
\vspace{0.3cm}
$\mathrm{Q}=\mathrm{V}*\mathrm{A}= \dfrac{1.70 \enspace \mathrm{ft}}{\mathrm{sec}}*0.785*(\dfrac{10}{12})^2 \enspace \mathrm{ft}^2=0.9267 \enspace \mathrm{ft}^3/\mathrm{sec}$\\
\vspace{0.3cm}
Step B - Finding the veocity through the 6 in. pipe:\\
\vspace{0.3cm}
$\mathrm{Q}=\mathrm{V}*\mathrm{A} \implies \mathrm{V}= \dfrac{\mathrm{Q}}{\mathrm{A}} \implies  \dfrac{0.9267 \enspace \mathrm{ft}^3/sec}{0.785*\Big(\dfrac{6}{12}\Big)^2 \mathrm{ft}^2}=\boxed{4.7 \enspace \mathrm{ft/sec}}$\\
\vspace{0.3cm}
Method 2: Using proportions method.\\
\vspace{0.3cm}
As Q is constant, $\mathrm{V} \propto \dfrac{1}{\mathrm{A}} \propto \dfrac{1}{\mathrm{D}^2} \implies V*\mathrm{D}^2=constant$\\
\vspace{0.3cm}
$\implies \mathrm{V}_{10}*\mathrm{D_{10}}^2=  V_{6}*\mathrm{D}_{6}^2 \implies V_{6} = \dfrac{\mathrm{V}_{10}*\mathrm{D_{10}}^2}{\mathrm{D}_{6}^2} = \dfrac{1.70*10^2}{6^2}=\boxed{4.7 \enspace \mathrm{ft/sec}}$



\end{enumerate}
\end{document}