\chapterimage{QuizCover} % Chapter heading image

\chapter{Activated Sludge Assessment}
% \textbf{Multiple Choice}

\section*{Activated Sludge Assessment}
\begin{enumerate}
\item What is the activated sludge floc made of\\

\item What properties of activated sludge floc are key to the effectiveness of the activated sludge process\\

\item List the activated sludge process control parameters (names/description)\\

\item List the key design differences between a rectangular and circular primary clarifier\\

\item Why is it important to ensure having a "good" microbiological composition of the activated sludge process (5 points)\\

\item What is/are the main factor/s that control the microbial population\\

\item MLVSS represents the \rule{1.5cm}{0.3mm}  fraction of the MLSS\\

\item Optimal range of SVI is between \rule{1.5cm}{0.3mm} to \rule{1.5cm}{0.3mm} \\

\item Straggler floc is associated with [type1] sludge while pin-floc is associated with \rule{1.5cm}{0.3mm}  sludge\\
Bulking and foaming is due to\rule{1.5cm}{0.3mm} bacteria\\

\item Portion of the activated sludge floc settled in the clarifier that is returned to the front of the aeration basin to seed the incoming primary effluent is called [name]\\

\item Which one of the following statements is TRUE regarding the various modifications of the activated sludge process\\
a. MCRT of 5 to 10 days is typical for extended aeration \\
*b. Typical hydraulic detention times in the contact tank of the contact stabilization process need only be 0.5 to 1.0 hour. \\
c. F to M ratios of 0.03 to .1 are appropriate for the step-aeration mode of the activated sludge process \\
d. Pure oxygen activated sludge floc often has a large population of rotifers. \\
e. Step feed -aeration involves decreasing the air being fed along the length of the aeration tank. \\

\item What is the significance/importance of measuring OUR and SOUR and what are their respective units of measurement\\
Correct Answer(s): \\

\item List the advantages and disadvantages of the constant RAS flow control \\
Correct Answer(s): \\

\item Name and describe the two RAS control approaches \\

Answer the following related to activated sludge floc:\\

\item What is the activated sludge floc made of (3 points)\\

\item Activated sludge is an anaerobic process \\
a. True \\
*b. False \\

\item  Secondary treatment is mainly to remove the organic content of the wastewater \\
*a. True \\
b. False \\

\item  The contents of an aeration tank utilized in activated sludge treatment is referred to as mixed liquor. \\
*a. True \\
b. False \\

\item  In conventional activated sludge plants, six to eight hours of aeration detention time is used for acceptable plant operation. \\
*a. True \\
b. False \\

\item  Bulking occurs in primary clarifiers and is associated with improper scum removal. \\
a. True \\
*b. False \\

\item  Contact stabilization is a modification of the conventional activated sludge system. \\
*a. True \\
b. False \\

\item  In secondary settling tanks, the sludge pumping considerations would be the same as in primary settling tanks. \\
a. True \\
*b. False \\

\item  The main function of a launder in a secondary clarifier is to prevent scum and other floatables from leaving with the effluent flow \\
a. T\\
@Incorrect.  Launder collects and conveys the effluent flow.  Effluent baffles prevent scum and other floatables from leaving with the effluent flow \\
*b. F\\
@Correct.  Launder collects and conveys the effluent flow.  Effluent baffles prevent scum and other floatables from leaving with the effluent flow \\

\item  Excessive filamentous bacteria in activated sludge is typically controlled by bleach addition to RAS \\
*a. True \\
b. False \\

\item  SVI is a measure of the sludge volume that needs to be wasted \\
a. True \\
*b. False \\

\item  pH has little effect on the activated sludge plant \\
a. True \\
*b. False \\

\item  Bulking is caused by excessive filamentous bacteria \\
*a. True \\
b. False \\


\item  Extended aeration involves operating the activated sludge process at a high F: M ratio \\
a. True \\
*b. False \\

\item  In conventional secondary wastewater treatment processes, aerobic decomposition of solids will occur. \\
*a. True \\
b. False \\

\item  In the activated sludge process, the wastewater oxygen demand may be separated into two categories:  carbonaceous and nitrogenous \\
*a. True \\
b. False \\

\item  MCRT refers to the average number of days that a “cell” remains in an activated sludge system. \\
*a. True \\
b. False \\

\item  In activated sludge treatment a young sludge age is marked by a low F:M ratio \\
a. True \\
*b. False \\

\item  The "M" in the F:M ratio is the mass of mixed liquor suspended solids in the aeration basin \\
a. True \\
*b. False \\

\item  The SVI test is used for establishing amount of sludge to be wasted\\
a. True \\
*b. False \\

\item  A WAS or RAS flow change of 25\% in one day will have little impact on the activated sludge treatment process\\
a. True \\
*b. False \\

\item  Rotifers are the dominant microorganisms in a young activated sludge \\
a. True \\
*b. False \\

\item The F in the F to M ratio refers to the pounds of mixed liquor volatile suspended solids under aeration in an activated sludge plant. \\
a. True \\
*b. False \\

\item  The use of F:M ratio for controlling the activated sludge process implies the need for higher mass of microorganisms to treat a stream with a higher BOD \\
*a. True \\
b. False \\

\item  The white billowing foam commonly seen during the startup of the activated sludge plant is caused by low F:M ratio \\
a. True \\
*b. False \\

\item  When an activated sludge plant is first started, one should expect to see foaming \\
*a. True \\
b. False \\

\item MLVSS represents the \rule{1.5cm}{0.3mm}  fraction of the MLSS \\

\item Optimal range of SVI is between \rule{1.5cm}{0.3mm} \rule{1.5cm}{0.3mm} \\

\item Straggler floc is associated with [\rule{1.5cm}{0.3mm} sludge while pin-floc is associated with \rule{1.5cm}{0.3mm} sludge \\
\item Bulking and foaming is due to \rule{1.5cm}{0.3mm}  bacteria \\

\item  Portion of the activated sludge floc settled in the clarifier that is returned to the front of the aeration basin to seed the incoming primary effluent is called [name] \\

\item The basic objective in the activated sludge process is to maintain balanced conditions in the aeration basin, this balance is called: \\
a. Endogenous respiration \\
*b. Food/microorganism ratio \\
c. Equilibrium status \\
d. Mass balance ratio \\

\item In the activated sludge treatment process, there are several control methods. One method is to maintain a BOD:MLVSS ratio. This is commonly referred to as: \\
a. MCRT. \\
b. SA. \\
c. SA:SDI. \\
*d. F:M. \\
e. TS:SRT \\

\item In calculating the detention time in an aeration tank, which one factor would not be considered? \\
a. tank volume \\
b. RAS flow \\
c. plant flow \\
*d. MLSS concentration \\
e. none of the above \\

\item The BOD loading rate divided by the quantity of microorganisms present in the biological reactors (aeration tanks) is known as: \\
a. organic loading \\
b. toxicity \\
c. hydraulic loading \\
*d. food to microorganism ration F:M \\

\item An activated sludge process that has a desired F/M ratio of 0.05 and a sludge age of 30 days is what type of activated sludge process modification? \\
*a. Extended aeration \\
b. Conventional \\
c. Complete mix \\
d. Oxidation ditch \\

\item Two major operational difficulties which sometimes occur in activated sludge secondary clarifiers are: \\
*a. Low D.O. and algae growth \\
b. Short circuiting and scum accumulation \\
c. Rising sludge and bulking sludge \\
d. Long detention time and short MCRT. \\

\item A thick, scummy, dark tan foam on the surface of an activated sludge aeration tank is an indication of: \\
*a. Aeration tank is underloaded (high MLSS. \\
b. Aeration tank is overloaded (low MLSS. \\
c. Excess grease in raw wastewater \\
d. Excess phosphates (detergents. in raw wastewater \\

\item A good quality of activated sludge is shown by: \\
a. Black color and very small particle size \\
b. Finely dispersed milky white particles \\
c. A chocolate brown MLSS that does not settle well in the jar test \\
d. A sludge that settles in one minute in the jar test \\
*e. A chocolate color which settles out in 20-30 minutes with a D.O. of 2.0 \\

\item An aerobic treatment process is one that requires the presence of: \\
a. Ozone \\
b. organic oxygen \\
c. no oxygen \\
d. combined oxygen \\
*e. dissolved oxygen \\

\item An increasing F/M ratio and decreasing MCRT indicates \\
*a. Excessive solids wasting causing a decrease in solids inventory \\
b. Inadequate solids wasting causing an increase in the solids inventory \\
c. Decreased hydraulic load increasing the sludge detention time \\
d. Operation is normal \\

\item A rapid and significant increase in filamentous organisms in the mixed liquor may be expected to: \\
a. Result in a far better effluent because of the great amount of surface area for absorption \\
b. Plug up the return sludge pumps because the filaments hang upon valves and gaskets in the sludge line \\
c. Lead to much denser return sludge because the filaments would tend to strain the dispersed cells of ordinary organisms out of the effluent \\
*d. Cause bulking of the sludge solids to the point that some solids might be swept out along with an otherwise clear liquid phase and result in turbid, poor quality effluent \\
e. Lead to a much lower F/M ratio because the filaments are so totally insoluble. \\

\item In the activated sludge treatment process, there are many control methods. One method is to maintain a constant BODs:MLVSS ratio. sludge treatment process.  This is commonly referred to as: \\
a. MCRT \\
b. SA \\
c. SA: SDI \\
*d. F:M \\
e. TS:SRT \\

\item The SVI of activated sludge is defined as: \\
a. the volume of settled mixed liquor after 30 minutes or settling· \\
b. the weight in grams of 200 ml of settled activated sludge \\
*c. the volume in ml of 1 gram of activated sludge after 30 minutes of settling \\
d. the total volume of MLSS in the aeration tank \\
e. the volume of settled sludge in the secondary clarifier \\

\item The amount of air required in the operation of an activated sludge aeration tank is independent of the: \\
a. temperature \\
b. flow \\
c. detention time \\
d. organic loading \\
*e. none of the above \\

\item The successful operation of an activated sludge plant requires the maintenance of proper solids concentration in the system.  One major limiting factor is: \\
a. mixed liquor tank volume \\
b. effluent flow \\
*c. air supply \\
d. chlorine demand \\
e. none of the above \\

\item The main difference between primary and secondary clarifiers is the: \\
a. overall dimensions \\
b. type of outlet weirs \\
*c. density of sludge \\
d. detention period \\
e. flow distribution \\

\item Given the data below, what is the most likely cause of the extended aeration facility problem?\\
DATA: DO level high\\
Blower normal\\
Wastewater characteristics normal\\
Drop pipe air control valves open\\
Surface turbulence high \\
a. Air relief valve stuck shut \\
*b. Blower speed too fast \\
c. Blower speed too slow \\
d. Drop pipe air control valves not open far enough \\

\item What test is used to determine the organic matter found in the mixed liquor? \\
a. COD \\
b. MLSS \\
*c. MLVSS \\
d. TOC \\

\item Fixed porous plate diffusers can be cleaned by scrubbing with \\
a. Detergent \\
b. A strong acid solution \\
*c. A strong chlorine solution \\
d. A weak sodium hydroxide solution \\

\item A 30 minute settleability test MLSS sample should be collected: \\
a. At the primary clarifier effluent \\
b. In the return sludge line \\
c. Where the return sludge mixes with the aeration basin contents \\
d. At the aeration basin influent \\
*e. At the aeration basin outlet \\

\item A consulting engineer has recommended addition of a roughing filter and intermediate clarifier between your primary clarifier and aeration basin to better handle increasing industrial loads. This addition would: \\
a. Be the best form of flow equalization available \\
b. Remove most of the fixed dissolved solids \\
c. Reduce drastically the fine dissolved matter \\
d. Cost a lot and do nothing \\
*e. Reduce the organic load on the aeration basin \\


\item An activated sludge process that has a desired F/M ratio of 0.05 and a sludge age of 30 days is what type of activated sludge process modification? \\
*a. Extended aeration \\
b. Conventional \\
c. Complete mix \\
d. Oxidation ditch \\

\item Two major operational difficulties which sometimes occur in activated sludge secondary clarifiers are: \\
*a. Low D.O. and algae growth \\
b. Short circuiting and scum accumulation \\
c. Rising sludge and bulking sludge \\
d. Long detention time and short MCRT. \\

\item Two major operational difficulties which sometimes occur in activated sludge secondary clarifiers are: \\
*a. Low D.O. and algae growth \\
b. Short circuiting and scum accumulation \\
c. Rising sludge and bulking sludge \\
d. Long detention time and short MCRT. \\


\item Possible techniques for controlling filamentous organisms in an activated sludge process include: \\
*a. Dosage of return sludge with a disinfectant such as chlorine or hypochlorite \\
b. Lower DO levels in aeration bans so filamentous organisms cannot breathe or respire \\
c. Lower F/M level to starve filamentous organisms \\
d. Stop wasting to allow activated sludge bugs to gain control \\

\item Possible techniques for controlling filamentous organisms in an activated sludge process include: \\
*a. Dosage of return sludge with a disinfectant such as chlorine or hypochlorite \\
b. Lower DO levels in aeration bans so filamentous organisms cannot breathe or respire \\
c. Lower F/M level to starve filamentous organisms \\
d. Stop wasting to allow activated sludge bugs to gain control \\

\item An activated sludge process that has a desired F/M ratio of 0.05 and a sludge age of 30 days is what type of activated sludge process modification? \\
*a. Extended aeration \\
b. Conventional \\
c. Complete mix \\
d. Oxidation ditch \\

\item An aerobic treatment process is one that requires the presence of: \\
a. Ozone \\
b. organic oxygen \\
c. no oxygen \\
d. combined oxygen \\
*e. dissolved oxygen \\

\item An increasing F/M ratio and decreasing MCRT indicates \\
*a. Excessive solids wasting causing a decrease in solids inventory \\
b. Inadequate solids wasting causing an increase in the solids inventory \\
c. Decreased hydraulic load increasing the sludge detention time \\
d. Operation is normal \\

\item A rapid and significant increase in filamentous organisms in the mixed liquor may be expected to: \\
a. Result in a far better effluent because of the great amount of surface area for absorption \\
b. Plug up the return sludge pumps because the filaments hang upon valves and gaskets in the sludge line \\
c. Lead to much denser return sludge because the filaments would tend to strain the dispersed cells of ordinary organisms out of the effluent \\
*d. Cause bulking of the sludge solids to the point that some solids might be swept out along with an otherwise clear liquid phase and result in turbid, poor quality effluent \\
e. Lead to a much lower F/M ratio because the filaments are so totally insoluble. \\

\item During severe cold weather operation of an activated sludge plant biological activity and clarifier sludge settling is reduced. White of the following might help? \\
*a. Increase the MLSS \\
b. Decrease the MLSS \\
c. Increase the D.0. \\
d. Decrease the D.0. \\
e. Add ammonia \\

\item Excess white foam in an aeration basin can be corrected by \\
a. Decreasing the aeration rate \\
b. Decreasing detention time \\
*c. Increasing the MLSS \\
d. Decreasing the MLSS \\
e. Increasing aeration rate \\

\item Given the following data, what is the most likely cause of the activated sludge problem?\\
DATA: The aeration tanks in an activated sludge plant have maintained a stable white foam with a brownish tint less than one inch thick. ·\\
BOD removals have been at their normal high efficiency.\\
Settling of the activated sludge in the secondary clarifiers has been good - as is normal.\\
Air supplied to the system has been a normal 30,000 cfm, with a consistent DO of 2.5 mg/L.\\
MLSS has been maintained at 2,500 mg/L - normal.\\
Gradually during your shift the DO has risen to 5.0 mg/L. \\
a. A toxic substance has affected the activated sludge. \\
b. BOD loading on the aeration system has increased. \\
c. Increased BOD loading has caused a corresponding increase in activated sludge activity. \\
*d. No change. \\

\item Given the following data, what is the most likely cause of the secondary sedimentation tank problem?\\
DATA: Sludge depth in tank too high.\\
Tank effluent turbid.\\
Tank effluent requiring above normal chlorine dosage.  Sweeparms in tank bottom operating.\\
Return activated sludge flow to aeration tank low.\\
Controls on return activated sludge pump on automatic.\\
Control sensors for return sludge operating normally. \\
a. Accuracy of sludge depth measurement. \\
b. Return activated sludge pump worn, needing repair. \\
*c. Speed of sweeparms travel. \\
d. Sweep arm overload tripped. \\

\item How many gallons of paint will be required to paint the walls of a 40 ft long x 65 ft wide x 20 ft high tank if the paint coverage is 150 sq. ft per gallon.  Note:  We are painting walls only.  Disregard the floor and roof areas. \\
*a. 28 gallons \\
b. 63 gallons \\
c. 35 gallons \\
d. 56 gallons \\

\item If there is an insufficient supply of air or oxygen being introduced into the aeration tank of an extended aeration plant, the liquid in the tank will likely \\
a. Contain a very fine light brown floc. \\
b. Contain very small air bubbles. \\
*c. Have a black or blackish appearance and an offensive odor. \\
d. Have a dishwater appearance and a greasy odor. \\

\item If the return sludge pump does not function the effect on other unit processes will be to: \\
a. Tum the aeration basin influent dark \\
b. Increase chlorine residual \\
*c. Increase effluent suspended solids \\
d. All the above \\
e. None of the above. \\

\item If the sludge depth in a secondary sedimentation tank is too high, what will happen? \\
a. Decreased turbidity in effluent. \\
b. Return activated sludge will have lower oxygen demand. \\
c. Settleable solids from aeration tank will increase. \\
*d. Sludge may become septic. \\

\item If you must waste sludge from an activated sludge plant the maximum rate is: \\
*a. 20 \% per day \\
b. 40 \% per day \\
c. 60 \% per day \\
d. 75\% per day \\
e. 100 \% perday \\

\item In an activated sludge system, what is perhaps the most important parameter affecting biological activity? \\
a. pH. \\
b. Alkalinity. \\
*c. Dissolved oxygen. \\
d. Temperature. \\

\item Mean cell residence time (MCRT. represents the theoretical time that a microorganism stays in the  activated sludge system. The typical values for most activated sludge processes are: \\
a. 3 - 30 days \\
*b. 3 - 15 days \\
c. 5 - 15 days \\
d. 5 - 20 days \\

\item Nocardia is associated with a particular type of brown, viscous scum or foam on the surface of the activated sludge aeration tank. One operational strategy that has been somewhat successful in reducing the severity of this foam is: \\
a. to increase the plant's MCRT. \\
*b. to decrease the plant's mixed liquor concentration. \\
c. to operate at an F:M ratio of less than 0.025. \\
d. to spray the foam with fine mist water sprays. \\
e. to increase the luxury DO concentration at the end of the aeration tank. \\

\item One limitation in using constant mixed liquor volatile suspended solids (MLVSS) or mixed liquor total suspended solids (MLTSS) as the control methodology for activated sludge treatment is \\
*a. In practice it is not possible to operate at a constant MLTSS or MLVSS. \\
b. It is based on consistency of raw waste load which seldom exists. \\
c. Most facilities don't have the lab equipment necessary to determine MLVSS. \\
d. None of the above. \\

\item Define Nocardia \& problems associated with it.   List 5 methods of controlling Nocardia.\\

Response:\\
\begin{enumerate}[label=\alph*]
\item \textit{Define Nocardia \& problems associated with it.}
\begin{itemize}
\item Nocardia is a type of filamentous organism which on overgrowth in the mixed liquor is the cause of foaming during aeration in the activated sludge process.  Nocardia proliferation is associated with the following three causes in combination: (1) high grease and oil; (2) longer sludge age; and (3) low oxygen conditions or septicity.\\ 
\noindent Nocardia develops a persistent, viscous brown foam scum.  If not skimmed properly, this can cause an increase in both suspended solids and BOD and cause scum spill over on to the catwalks and make them sticky and slimy.  Additionally, Nocardia from the secondary sludge could cause digester foaming. 
\end{itemize}
\item \textit{List 5 methods of controlling Nocardia.}
\begin{enumerate}
\item Reduce sludge age by increasing wasting rate.  This would also effectively reduce MLSS concentration in the aeration tank and increase F:M.
\item Mechanically skim and remove the foam.
\item Enforce industrial waste control program and manage the collections cleaning procedures to minimize the amount of grease and oil in the influent wastewater.
\item Chlorinate the RAS stream or the mixed liquor return or by spraying directly into the aeration tank.
\item Use of mannich polymer to enhance settling and removal of the Nocardia filaments
\end{enumerate}
\end{enumerate}

\pagebreak
\item Clearly show how SOUR \& OUR are different.  List two possible causes for a sudden 40\% decrease in OUR during the last 12 hours of operation.  What lab test might confirm or refute the two causes of this sudden decrease in OUR.
Response:\\
\begin{enumerate}[label=\alph*]
\item \textit{Clearly show how SOUR \& OUR are different.}
\begin{itemize}
\item Oxygen Uptake Rate (OUR) involves measurement of the amount of oxygen used up by the microorganisms in the mixed liquor using a DO probe and is expressed in unit time of mg/L-hr (ppm O$_2$ consumed per hour). By knowing the OUR, we can establish the activity of the microorganisms in the aeration tank and know if they are consuming the oxygen provided for removing organic matter.  For conventional activated sludge process the typical OUR values range from 10 to 30 mg/L-hr\\
\vspace{0.2cm}
Specific Oxygen Uptake Rate (SOUR) provides the OUR information based on the concentration of microorganisms present. SOUR is obtained by dividing OUR with MLVSS.\\
\vspace{0.2cm}
The value is indicated and measured in terms of unit of $\frac{\frac{mg}{l-hr}O_2}{\frac{mg}{l}MLVSS}*1000\frac{mg}{gm}$. \\
\vspace{0.2cm}
Optimal range of SOUR is usually between 8 to 20.\\
\end{itemize}
\item \textit{List two possible causes for a sudden 40\% decrease in OUR during the last 12 hours of operation.}
\begin{enumerate}
\item Presence of toxic substance in the mixed liquor inhibiting the normal biological activity in the aeration basin
\item Excessive loss of biomass due to  inadvertent wasting 

\end{enumerate}
\item \textit{What lab test might confirm or refute the two causes of this sudden decrease in OUR.}
\begin{enumerate}
\item Run a MLSS test to see if adequate biomass is present
\item Run a COD test to ensure adequate amount of organics (food) is present\\
(If there is sufficient biomass and food present, the only reason why the OUR would have declined would be because of toxic substance entering the system)
 
\end{enumerate}
\end{enumerate}

\item The control and calculation of RAS flow rates are important considerations in the
Operation of activated sludge wastewater treatment plants. Most commonly the operator
Either sets the return rate (Qr) at a constant flow or as a constant percentage of flow.\\

(a) Identify two effects that each of these approaches will have on plant operations.\\
Assume that a normal diurnal variation in both wastewater flow and strength.\\

(b) The so-called “solids balance approach” (shown below) may be used to
Mathematically estimate Qr. What two assumptions are made in deriving this formula?\\

[Qr + Q] x MLSS Conc. = Qr x RAS Conc.\\

(c) Using the information given below and the “solids balance” equation, calculate SVI
and Qr comment on these values:\\

Q…………………..2.0 MGD\\
MLSS……...……2350 mg/L\\
RAS………….….7350 mg/L\\
SV30………………320 mLs/g\\

\end{enumerate}

