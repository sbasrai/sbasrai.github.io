\chapterimage{QuizCover} % Chapter heading image

\chapter{Biosolids Regulations Assessment}
% \textbf{Multiple Choice}

\section*{Biosolids Regulations Assessment}
\begin{enumerate}

\item  Grit and screenings from the preliminary treatment are also considered as biosolids \\

*a. True \\
b. False \\

\item  A true statement regarding the term “biosolids” is: \\

a. The term is mandated for user by public law 92-500. \\
b. The term was developed by US EPA to define all biologically toxic precipitates. \\
*c. The term is recommended by WEF for “a primarily organic solids product, produced by wastewater treatment processes, that can be beneficially recycled”. \\
d. The term is used by the California Water Resources Control Board to include “all insoluble matter derived from living aquatic organisms. \\

\item  When you spread sludge on agricultural land, the annual application rate of cadmium in the sludge should be less than 2 lbs/acre/year If your sludge contains 30 mg cadmium per kilogram of solids and your plant produces 950,000 lbs per year of dry solids, how many acres do you need? \\

a. 3 acres \\
*b. 14 acres \\
c. 19 acres \\
d. 27 acres \\
\newpage
\item Essay type question:\\
\begin{enumerate}
\item Define vector and pathogens.
\item Class A \& B landfill requirements: what you need to do to get your anaerobic digester to produce class B sludge to prepare for landfill.
\item What unit processes can produce class A \& B sludge?
\end{enumerate}

Response:\\
\begin{enumerate}[label=\alph*]
\item \textit{Define vector and pathogens.}
\begin{itemize}
\item Pathogens are disease causing organisms such as bacteria, viruses and parasites 
\item Vectors are organisms such as rodents and insects that can carry disease by carrying and transferring pathogens
\end{itemize}
\item \textit{Class A \& B landfill requirements: what you need to do to get your anaerobic digester to produce class B sludge to prepare for landfill.}\\
For the digested sludge to qualify as Class B sludge it needs to meet the following related to 40 CFR Part 503:
\begin{itemize}
\item Meet the Minimum Concentration Standards of the Pollution Concentration Standards
\item Meet the digestion sludge detention time-temperature requirements
\item Meet the Vector Reduction Standards
\end{itemize}
\item \textit{What unit processes can produce class A \& B sludge?}\\
\textbf{For Class A sludge:}
\begin{enumerate}
\item Composting
\item Heat drying
\item Heat treatment
\item Thermophillic aerobic digestion
\item Beta ray irradiation
\item Gamma ray radiation
\item Pasteurization
\end{enumerate}
\pagebreak
\textbf{For Class B sludge:}
\begin{enumerate}
\item Aerobic digestion
\item Air drying
\item Anaerobic digestion
\item Composting
\item Lime stabilization
\end{enumerate}
\end{enumerate}


\end{enumerate}