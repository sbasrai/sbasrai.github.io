\chapterimage{QuizCover} % Chapter heading image

\chapter{Collections Assessment}
% \textbf{Multiple Choice}

\section*{Collections Assessment}
\begin{enumerate}

\item  Hydrogen sulfide gas in a moist atmosphere can result in corrosion of concrete structures \\

*a. True \\
b. False \\

\item  As a rule of thumb, the velocity in a gravity sewer should be at least 2 cubic feet per second to ensure that the solids do not settle out in the sewer lines\\

a. True \\
*b. False \\

\item  A combined sewer implies that it carries domestic and industrial wastes only \\

a. True \\
*b. False \\

\item  Hydrogen sulfide gas in a moist atmosphere can result in corrosion of concrete structures \\

*a. True \\
b. False \\

\item  Inflow is storm water entering into the sewer system \\

*a. True \\
b. False \\

\item  A combined wastewater collection system handles only domestic waste and industrial waste. \\

a. True \\
*b. False \\

\item  Fats, oils and grease accumulation in sewers can cause sewer overflows \\

*a. True \\
b. False \\

\item  Storms could potentially cause sewer overflows in a Combined Sewer system  \\

*a. True \\
b. False \\

\item  Groundwater entering sewer collection pipes through cracks and defective pipe joints is termed as inflow \\

a. True \\
*b. False \\

\item  A Combined sewer system is the one that brings the combined domestic and industrial wastewater flow to the treatment plant \\

a. True \\
*b. False \\

\item  Infiltration is when groundwater enters the sewage collections systems \\

*a. True \\
b. False \\

\item  Infiltration is when groundwater enters the sewage collections systems \\

*a. True \\
b. False \\

\item  As a rule of thumb, the velocity in a gravity sewer should be at least 2 cubic feet per second to ensure that the solids do not settle out in the sewer lines\\

a. True \\
*b. False \\

\item  A combined sewer implies that it carries domestic and industrial wastes only \\

a. True \\
*b. False \\

\item  Hydrogen sulfide gas in a moist atmosphere can result in corrosion of concrete structures \\

*a. True \\
b. False \\

\item  Inflow is storm water entering into the sewer system \\

*a. True \\
b. False \\



\item  A lateral is the largest sewer line which brings the wastewater to the treatment plant \\

a. True \\
*b. False \\

\item  Hydrogen sulfide gas in a moist atmosphere can result in corrosion of concrete structures \\

*a. True \\
b. False \\

\item  As a rule of thumb, the velocity in a gravity sewer should be at least 2 cubic feet per second to ensure that the solids do not settle out in the sewer lines\\

a. True \\
*b. False \\

\item  A combined sewer implies that it carries domestic and industrial wastes only \\

a. True \\
*b. False \\

\item  Hydrogen sulfide gas in a moist atmosphere can result in corrosion of concrete structures \\

*a. True \\
b. False \\

\item  Inflow is storm water entering into the sewer system \\

*a. True \\
b. False \\

\item  A combined wastewater collection system handles only domestic waste and industrial waste. \\

a. True \\
*b. False \\

\item  Fats, oils and grease accumulation in sewers can cause sewer overflows \\

*a. True \\
b. False \\

\item  Storms could potentially cause sewer overflows in a Combined Sewer system  \\

*a. True \\
b. False \\

\item  Groundwater entering sewer collection pipes through cracks and defective pipe joints is termed as inflow \\

a. True \\
*b. False \\

\item  A Combined sewer system is the one that brings the combined domestic and industrial wastewater flow to the treatment plant \\

a. True \\
*b. False \\

\item  Infiltration is when groundwater enters the sewage collections systems \\

*a. True \\
b. False \\

\item Define Infiltration and Inflow.  Discuss the impact of inflow and infiltration on the wastewater treatment plant:


\item  In wastewater collections what is a gravity system:

\item  A sewer system designed to transport only wastewater from homes, industries, institutions and businesses is called: \\

a. Combined sewer \\
*b. Sanitary sewer \\
c. Service sewer \\
d. Building sewer \\

\item  Velocity of sewers is usually expressed as: \\

*a. Feet per second \\
b. Gallon per minute \\
c. MGD \\
d. mg/l \\
e. sq. ft \\

\item  Which of the following statements is not true regarding a wastewater collection system: \\

*a. A sewer is designed to allow the waste to flow at a rate of approximately 1 ft/sec. \\
b. Grease can be serious problem in a collection system. \\
c. Inflow and infiltration are frequently problems in older collection systems. \\
d. High concentrations of hydrogen sulfide in a sewer can lead to corrosion of concrete. \\
e. Scouring can be a problem if wastewater is flowing too fast. \\

\item  Which of the following statements is not true regarding a wastewater collection system: \\

*a. A sewer is designed to allow the waste to flow at a rate of approximately 1 ft/sec. \\
b. Grease can be serious problem in a collection system. \\
c. Inflow and infiltration are frequently problems in older collection systems. \\
d. High concentrations of hydrogen sulfide in a sewer can lead to corrosion of concrete. \\
e. Scouring can be a problem if wastewater is flowing too fast. \\

\item  A sewer system designed to transport only wastewater from homes, industries, institutions and businesses is called: \\

a. Combined sewer \\
*b. Sanitary sewer \\
c. Service sewer \\
d. Building sewer \\

\item  Velocity of sewers is usually expressed as: \\

*a. Feet per second \\
b. Gallon per minute \\
c. MGD \\
d. mg/l \\
e. sq. ft \\

\item  Velocity of flow in sewers is usually expressed in terms of \\

*a. Feet per second \\
b. Gallon per minute \\
c. MGD \\
d. Milligrams per liter \\
e. Square feet \\

\item  Velocity of flow in sewers is usually expressed in terms of \\

*a. Feet per second \\
b. Gallon per minute \\
c. MGD \\
d. Milligrams per liter \\
e. Square feet \\

\item Infiltration is caused by:
*a. Cracked pipes
b. Improper CCTV operation
c. Poor ventilation
d. All of the above



\item Define infiltration \& inflow, compatible \& non-compatible. List 4 major concerns.  List 2 corrections.


\end{enumerate}