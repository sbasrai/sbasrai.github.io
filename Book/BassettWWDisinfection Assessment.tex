\chapterimage{QuizCover} % Chapter heading image

\chapter{Disinfection Assessment}
% \textbf{Multiple Choice}

\section*{Disinfection Assessment}
\begin{enumerate}
\item All 100- and 150-pound chlorine cylinders should be restrained or safety-chained to sturdy supports even when empty. Except when actually being moved to or from storage. \\
*a. True \\
b. False \\
\item \item Chlorine demand is a good indicator of effective disinfection \\
a. True \\
*b. False \\
\item Chlorine demand is a good indicator of effective disinfection \\
a. True \\
*b. False \\
\item Chlorine demand is defined as the amount of chlorine remaining in the waste water at the end of a specific contact period. \\
a. True \\
*b. False \\
\item Chlorine disinfection is more effective at higher pH \\
a. True \\
*b. False \\
\item Chlorine dosage is the difference between the amount of chlorine added to wastewater and the amount of residual chlorine remaining after a given contact time. \\
a. True \\
*b. False \\
\item Chlorine feed rate and chlorine residual are both usually expressed in units of ppm. \\
*a. True \\
b. False \\
\item Chlorine is used to sterilize wastewater in order to insure protection of public health. \\
a. True \\
*b. False \\
\item Chlorine demand is a good indicator of effective disinfection \\
a. True \\
*b. False \\
\item Vents in a chlorine storage room are located at the ground level as chlorine is lighter than air \\
a. True \\
*b. False \\
\item Chlorine demand is defined as the amount of chlorine remaining in the waste water at the end of a specific contact period. \\
a. True \\
*b. False \\
\item Chlorine dosage is the difference between the amount of chlorine added to wastewater and the amount of residual chlorine remaining after a given contact time. \\
a. True \\
*b. False \\
\item Hypochlorite solution is a more effective disinfectant than gas chlorine and is much less expensive. \\
a. True \\
*b. False \\
\item All 100- and 150-pound chlorine cylinders should be restrained or safety-chained to sturdy supports even when empty. Except when actually being moved to or from storage. \\
*a. True \\
b. False \\
\item Vents in a chlorine storage room are located at the ground level as chlorine is lighter than air \\
a. True \\
*b. False \\
\item Chlorine demand is defined as the amount of chlorine remaining in the waste water at the end of a specific contact period. \\
a. True \\
*b. False \\
\item Chlorine dosage is the difference between the amount of chlorine added to wastewater and the amount of residual chlorine remaining after a given contact time. \\
a. True \\
*b. False \\
\item Hypochlorite solution is a more effective disinfectant than gas chlorine and is much less expensive. \\
a. True \\
*b. False \\
\item All 100- and 150-pound chlorine cylinders should be restrained or safety-chained to sturdy supports even when empty. Except when actually being moved to or from storage. \\
*a. True \\
b. False \\
\item Chlorine dosage is the difference between the amount of chlorine added to wastewater and the amount of residual chlorine remaining after a given contact time. \\
a. True \\
*b. False \\
\item Before any new or repaired chlorine piping system is placed into operation, it should be pressure tested with water to detect any possible leaks before chlorine gas or liquid is used. \\
a. True \\
*b. False \\
\item To produce a savings in chlorination cost and power consumption, chlorine feed rates for odor control should be determined at low flow conditions. \\
*a. True \\
b. False \\
\item Chlorine is used to sterilize wastewater in order to insure protection of public health. \\
a. True \\
*b. False \\
\item The purpose of disinfection with chlorine is to destroy pathogenic organisms. \\
*a. True \\
b. False 
\item Oxidation pond effluents are easily disinfected with chlorine because of the large amount of algae present \\
a. True \\
*b. False 
\item The chlorine demand of wastewater equals the chlorine dosage plus the residual. \\
a. True \\
*b. False \\
\item Post chlorination is employed primarily for odor control and BOD reduction. \\
a. True \\
*b. False \\
\item The safety plugs on a one-ton chlorine tank are designed to soften or melt at temperatures in excess of 158°F. \\
*a. True \\
b. False \\
\item When moving a chlorine cylinder a short distance, for example 10 feet, it is not necessary to replace the protective cap. \\
a. True \\
*b. False \\
\item A leaking chlorine cylinder cannot be transported. \\
*a. True \\
b. False \\
\item Canister-type gas masks are not recommended for use when working ·on chlorine leaks, because they do not supply oxygen to the wearer. \\
*a. True \\
b. False \\
\item Canister-type gas masks may be used to attend chlorine gas leaks at concentrations in the range of two to five percent. \\
a. True \\
*b. False \\
\item Hypochlorite solution is a more effective disinfectant than gas chlorine and is much less expensive. \\
a. True \\
*b. False \\
\item Objective of the disinfection process is to sterilize the wastewater \\
a. True \\
*b. False \\
\item Caution needs to be exercised when handling chlorine due to its explosive properties \\
a. True \\
*b. False \\
\item It is important that the (PPE) including self-contained breathing apparatus, for handling leaks from chlorine cylinders is located inside the chlorine storage building to ensure easy access \\
a. True \\
*b. False \\
\item Ammonia solution is used for finding leaks in chlorine cylinders \\
*a. True \\
b. False \\
\item Sodium hypochlorite is more effective and less expensive than chlorine gas \\
a. True \\
*b. False \\
\item Sodium hypochlorite is more effective and less expensive than chlorine gas \\
a. True \\
*b. False \\
\item Vents in a chlorine storage room are located at the ground level as chlorine is lighter than air \\
a. True \\
*b. False \\
\item All 100- and 150-pound chlorine cylinders should be restrained or safety-chained to sturdy supports even when empty. Except when actually being moved to or from storage. \\
*a. True \\
b. False \\
\item Chlorine demand is a good indicator of effective disinfection \\
a. True \\
*b. False \\
\item Chlorine demand is a good indicator of effective disinfection \\
a. True \\
*b. False \\
\item Chlorine demand is defined as the amount of chlorine remaining in the waste water at the end of a specific contact period. \\
a. True \\
*b. False \\
\item Chlorine disinfection is more effective at higher pH \\
a. True \\
*b. False \\
\item Chlorine dosage is the difference between the amount of chlorine added to wastewater and the amount of residual chlorine remaining after a given contact time. \\
a. True \\
*b. False \\
\item Chlorine feed rate and chlorine residual are both usually expressed in units of ppm. \\
*a. True \\
b. False \\
\item Chlorine is used to sterilize wastewater in order to insure protection of public health. \\
a. True \\
*b. False \\
\item Vents in a chlorine storage room are located at the ground level as chlorine is lighter than air \\
a. True \\
*b. False \\
\item Vents in a chlorine storage room are located at the ground level as chlorine is lighter than air \\
a. True \\
*b. False \\
\item Chlorine dosage is the difference between the amount of chlorine added to wastewater and the amount of residual chlorine remaining after a given contact time. \\
a. True \\
*b. False \\
\item Hypochlorite solution is a more effective disinfectant than gas chlorine and is much less expensive. \\
a. True \\
*b. False \\
\item All 100- and 150-pound chlorine cylinders should be restrained or safety-chained to sturdy supports even when empty. Except when actually being moved to or from storage. \\
*a. True \\
b. False \\
\item Hypochlorite solution is a more effective disinfectant than gas chlorine and is much less expensive. \\
a. True \\
*b. False \\
\item Sodium hypochlorite is more effective and less expensive than chlorine gas \\
a. True \\
*b. False \\
\item The chlorine dose rate plus the chlorine residual equals the chlorine demand. \\
a. True \\
*b. False \\
\item Vents in a chlorine storage room are located at the ground level as chlorine is lighter than air \\
a. True \\
*b. False \\
\item Chlorine demand is defined as the amount of chlorine remaining in the waste water at the end of a specific contact period.\\
a. True \\
*b. False \\
\item Chlorine dosage is the difference between the amount of chlorine added to wastewater and the amount of residual chlorine remaining after a given contact time.\\
a. True \\
*b. False \\
\item Hypochlorite solution is a more effective disinfectant than gas chlorine and is much less expensive.\\
a. True \\
*b. False \\
\item All 100- and 150-pound chlorine cylinders should be restrained or safety-chained to sturdy supports even when empty. Except when actually being moved to or from storage.\\
*a. True \\
b. False \\
\item Chlorine demand is a good indicator of effective disinfection\\
a. True \\
*b. False \\
\item A chlorine cylinder valve is thought to be leaking. If ammonia vapor is passed over the valve, the presence of a leak is indicated by \\
a. A hissing noise. \\
*b. A white cloud. \\
c. An odor of hydrogen sulfide. \\
d. Red smoke \\
\item A chlorine residual is often maintained in a plant effluent: \\
a. to keep the chlorinator working. \\
b. for control of fluctuation of wastewater flow. \\
c. for testing purposes. \\
*d. to protect the bacteriological quality of the receiving water. \\
e. None of the above. \\
\item Acids should never be added to chlorine solutions as they \\
*a. Cause chlorine gas to be released. \\
b. Corrode or "eat away" the solution tank. \\
c. Decrease the disinfecting properties of chlorine. \\
d. Result in the formation of a chloride precipitate.
,, \\
\item An amperometric titrater is used to measure \\
a. Alkalinity. \\
*b. Chlorine residual. \\
c. Conductivity. \\
d. COD. \\
\item An operator should never enter a room containing a high concentration of chlorine gas without \\
a. Staying low on the floor. \\
b. Holding breath and have help standing by. \\
*c. Having self-contained air or oxygen supply and help standing by. \\
d. Covering nose and mouth with a wet handkerchief. \\
\item As water temperatures decrease, the disinfecting action of chlorine \\
*a. Decreases. \\
b. Increases. \\
c. Remains the same. \\
\item At a wastewater treatment plant. The amount of chlorine used in a day from a cylinder or tank that is in service is normally determined by: \\
a. knowing both the pressure and temperature of the cylinder pressure gauges. \\
b. rotameter readings. \\
*c. weighing of the cylinder or tank \\
d. the chlorine residual test . \\
e. None of the above. \\
\item At what level should the exhaust be drawn off in a chlorination room? \\
a. At chest height. \\
b. At least two feet above the height of the chlorine cylinder. \\
c. Near the ceiling. \\
*d. Near the floor. \\
\item Chloramines are \\
*a. Combined chlorine. \\
b. Enzymes. \\
c. Found in polluted air. \\
d. Free chlorine. \\
\item Chlorine gas \\
a. Is lighter than air. \\
*b. Is heavier than air. \\
c. Is pink in color. \\
d. Will liquify at 70 degrees F. \\
\item Chlorine gas is \\
a. Colorless. \\
*b. Heavier than air. \\
c. Non-toxic. \\
d. Odorless. \\
\item Chlorine is: \\
a. Colorless \\
b. Explosive \\
*c. Toxic \\
d. All of the above \\
\item Chlorine is being applied at a constant dose rate of 24 mg/L to a partially nitrified activated sludge effluent having a pH of 6.8 and a temperature of 67F.  Ammonia-nitrogen is found to range from 2 to 3 mg/L in this effluent. Disinfection in this effluent might be difficult because: \\
a. a temperature of 70' F or higher is necessary in order to achieve effective disinfection. \\
b. chloramines are present most of the time. \\
c. the chlorine dose rate is too low. \\
d. the pH of this effluent will limit the effectiveness of free chlorine. \\
*e. the ratio of chlorine to ammonia-nitrogen may make it difficult at times to maintain adequate chlorine residual. \\
\item Chlorine is being fed at the rate of 75 pounds per day. Plant flow is 1.2 MGD. The chlorine residual is measured and found to be 2.6 mg/L Calculate chlorine demand. \\
*a. 4.9 mg/L \\
b. 5.7 mg/L \\
c. 7.5 mg/L \\
d. 8.3 mg/L \\
\item Chlorine is used to \\
*a. Disinfect. \\
b. Prevent corrosion. \\
c. Raise the pH. \\
d. Stabilize organics. \\
\item Chlorine residual may be determined using the reagent \\
*a. Diethyl-p-phenylenediamine (DPD). \\
b. Ethylendiamine tetraacetic acid (EDTA). \\
c. Polychlorinated biphenyls (PCB). \\
d. Sodium thiosulfate (Na2S203) \\
\item "Chlorine residual" refers to: \\
a. the amount of chlorine remaining in the ton cylinder after use. \\
b. the amount of chlorine consumed during disinfection. \\
*c. the chlorine remaining after disinfection. \\
d. the chlorine that displays no disinfection power. \\
e. the residue left after the evaporation of chlorine gas. \\
\item The effectiveness of chlorine disinfection is measured by: \\
a. the chlorine demand \\
b. the chlorine dosage \\
c. the total chlorine residual \\
*d. the coliform concentration of the effluent \\
\item The amount of chlorine used per day from a 1 ton chlorine cylinder is normally determined by: \\
a. Pressure gauges. \\
b. Rotometers. \\
*c. Weighings. \\
d. Chlorine residuals. \\
e. Ammonia equivalents. \\
\item One liter of liquid chlorine can evaporate and produce how many liters of chlorine gas? \\
a. 100 \\
b. 250 \\
*c. 460 \\
d. 490 \\
\item Which of the following discharges would in general, require the lowest chlorine dosage to ensure adequate disinfection? \\
a. Primary plant effluent \\
b. Activated sludge plant effluent \\
c. Trickling filter plant effluent \\
*d. Sand filter effluent \\
e. Stabilization pond effluent \\
\item Which of the following are factors that may influence the effectiveness of chlorine? \\
a. Chlorine dose rate \\
b. Contact time \\
c. Suspended solids concentration of the wastewater being disinfected. \\
*d. Only (a) and (b) \\
e. (a), (b), and (c) \\
\item The fundamental purpose of disinfection is to: \\
a. Destroy fecal coliform bacteria \\
b. Destroy all bacteria \\
*c. Destroy pathogenic organisms \\
d. Protect downstream users from waterborne diseases \\
\item In the application of chlorine for disinfection, which of the following is not normally an operational consideration? \\
a. Mixing \\
b. Contact time \\
*c. DO \\
d. pH \\
e. None of the above \\
\item "Chlorine residual" refers to: \\
a. the amount of chlorine remaining in the ton cylinder after use. \\
b. the amount of chlorine consumed during disinfection. \\
*c. the chlorine remaining after disinfection. \\
d. the chlorine that displays no disinfection power. \\
e. the residue left after the evaporation of chlorine gas. \\
\item In the application of chlorine for disinfection, which of the following is not normally an operational consideration? \\
a. Mixing \\
b. Contact time \\
*c. DO \\
d. pH \\
e. None of the above \\
\item A chlorine residual is often maintained in a plant effluent: \\
a. to keep the chlorinator working. \\
b. for control of fluctuation of wastewater flow. \\
c. for testing purposes. \\
*d. to protect the bacteriological quality of the receiving water. \\
e. None of the above. \\
\item At a wastewater treatment plant. The amount of chlorine used in a day from a cylinder or tank that is in service is normally determined by: \\
a. knowing both the pressure and temperature of the cylinder pressure gauges. \\
b. rotameter readings. \\
*c. weighing of the cylinder or tank \\
d. the chlorine residual test . \\
e. None of the above. \\
\item The fundamental purpose of disinfection is to: \\
a. Destroy fecal coliform bacteria \\
b. Destroy all bacteria \\
*c. Destroy pathogenic organisms \\
d. Protect downstream users from waterborne diseases \\
\item One liter of liquid chlorine can evaporate and produce how many liters of chlorine gas? \\
a. 100 \\
b. 250 \\
*c. 460 \\
d. 490 \\
\item Identify the incorrect statement regarding disinfection. \\
a. When chlorine is added to water it forms acids, which tend to lower the pH of the wastewater effluent \\
b. HTH is a dry form of calcium hypochlorite \\
c. Appropriate doses of chlorine may be used to control odors, control filamentous bulking in activated sludge mixed liquor, or reduce BOD5 of wastewater \\
*d. Hypochlorite’s are sometimes used in place of chlorine because they are more effective and less costly \\
\item Hypochlorite solution is used in effluent disinfection because: \\
a. Chlorine residual determination is more stable and accurate in hypochlorite \\
b. Hypochlorite residuals are more resistant to nitrite interference \\
*c. Chlorine gas is more hazardous to store and handle \\
d. Hypochlorite solution is easier and less costly to ship than gas chlorine \\
e. Chlorine causes too many problems with disinfection efficiency \\
\item Chlorine is: \\
a. Colorless \\
b. Explosive \\
*c. Toxic \\
d. All of the above \\
\item The effectiveness of chlorine disinfection is measured by: \\
a. the chlorine demand \\
b. the chlorine dosage \\
c. the total chlorine residual \\
*d. the coliform concentration of the effluent \\
\item The amount of chlorine used per day from a 1 ton chlorine cylinder is normally determined by: \\
a. Pressure gauges. \\
b. Rotometers. \\
*c. Weighings. \\
d. Chlorine residuals. \\
e. Ammonia equivalents. \\
\item Which of the following discharges would in general, require the lowest chlorine dosage to ensure adequate disinfection? \\
a. Primary plant effluent \\
b. Activated sludge plant effluent \\
c. Trickling filter plant effluent \\
*d. Sand filter effluent \\
e. Stabilization pond effluent \\
\item Which of the following are factors that may influence the effectiveness of chlorine? \\
a. Chlorine dose rate \\
b. Contact time \\
c. Suspended solids concentration of the wastewater being disinfected. \\
*d. Only (a) and (b) \\
e. (a), (b), and (c) \\
\item The fundamental purpose of disinfection is to: \\
a. Destroy fecal coliform bacteria \\
b. Destroy all bacteria \\
*c. Destroy pathogenic organisms \\
d. Protect downstream users from waterborne diseases \\
\item In the application of chlorine for disinfection, which of the following is not normally an operational consideration? \\
a. Mixing \\
b. Contact time \\
*c. DO \\
d. pH \\
e. None of the above \\
\item "Chlorine residual" refers to: \\
a. the amount of chlorine remaining in the ton cylinder after use. \\
b. the amount of chlorine consumed during disinfection. \\
*c. the chlorine remaining after disinfection. \\
d. the chlorine that displays no disinfection power. \\
e. the residue left after the evaporation of chlorine gas. \\
\item In the application of chlorine for disinfection, which of the following is not normally an operational consideration? \\
a. Mixing \\
b. Contact time \\
*c. DO \\
d. pH \\
e. None of the above \\
\item A chlorine residual is often maintained in a plant effluent: \\
a. to keep the chlorinator working. \\
b. for control of fluctuation of wastewater flow. \\
c. for testing purposes. \\
*d. to protect the bacteriological quality of the receiving water. \\
e. None of the above. \\
\item At a wastewater treatment plant. The amount of chlorine used in a day from a cylinder or tank that is in service is normally determined by: \\
a. knowing both the pressure and temperature of the cylinder pressure gauges. \\
b. rotameter readings. \\
*c. weighing of the cylinder or tank \\
d. the chlorine residual test . \\
e. None of the above. \\
\item The fundamental purpose of disinfection is to: \\
a. Destroy fecal coliform bacteria \\
b. Destroy all bacteria \\
*c. Destroy pathogenic organisms \\
d. Protect downstream users from waterborne diseases \\
\item One liter of liquid chlorine can evaporate and produce how many liters of chlorine gas? \\
a. 100 \\
b. 250 \\
*c. 460 \\
d. 490 \\
\item Identify the incorrect statement regarding disinfection. \\
a. When chlorine is added to water it forms acids, which tend to lower the pH of the wastewater effluent \\
b. HTH is a dry form of calcium hypochlorite \\
c. Appropriate doses of chlorine may be used to control odors, control filamentous bulking in activated sludge mixed liquor, or reduce BOD5 of wastewater \\
*d. Hypochlorite’s are sometimes used in place of chlorine because they are more effective and less costly \\
\item Hypochlorite solution is used in effluent disinfection because: \\
a. Chlorine residual determination is more stable and accurate in hypochlorite \\
b. Hypochlorite residuals are more resistant to nitrite interference \\
*c. Chlorine gas is more hazardous to store and handle \\
d. Hypochlorite solution is easier and less costly to ship than gas chlorine \\
e. Chlorine causes too many problems with disinfection efficiency \\
\item Chlorine is: \\
a. Colorless \\
b. Explosive \\
*c. Toxic \\
d. All of the above \\
\item The difference between the amount of chlorine added and the amount of chlorine remaining after the contact period is referred to as: \\
*a. the chlorine demand \\
b. free chlorine residual \\
c. total chlorine residual \\
d. combined chlorine residual \\
e. free available chlorine \\
\item The ultimate measure of the effectiveness of chlorination in disinfection is: \\
a. the measurement of chlorine dosage \\
b. meeting the chlorine demand of the wastewater \\
c. the establishment of a chlorine demand \\
*d. effective reduction of the coliform count \\
e. none of the above \\
\item A chlorine cylinder valve is thought to be leaking. If ammonia vapor is passed over the valve, the presence of a leak is indicated by \\
a. A hissing noise. \\
*b. A white cloud. \\
c. An odor of hydrogen sulfide. \\
d. Red smoke \\
\item A chlorine residual is often maintained in a plant effluent: \\
a. to keep the chlorinator working. \\
b. for control of fluctuation of wastewater flow. \\
c. for testing purposes. \\
*d. to protect the bacteriological quality of the receiving water. \\
e. None of the above. \\
\item Acids should never be added to chlorine solutions as they \\
*a. Cause chlorine gas to be released. \\
b. Corrode or "eat away" the solution tank. \\
c. Decrease the disinfecting properties of chlorine. \\
d. Result in the formation of a chloride precipitate.
,, \\
\item An amperometric titrater is used to measure \\
a. Alkalinity. \\
*b. Chlorine residual. \\
c. Conductivity. \\
d. COD. \\
\item An operator should never enter a room containing a high concentration of chlorine gas without \\
a. Staying low on the floor. \\
b. Holding breath and have help standing by. \\
*c. Having self-contained air or oxygen supply and help standing by. \\
d. Covering nose and mouth with a wet handkerchief. \\
\item As water temperatures decrease, the disinfecting action of chlorine \\
*a. Decreases. \\
b. Increases. \\
c. Remains the same. \\
\item At a wastewater treatment plant. The amount of chlorine used in a day from a cylinder or tank that is in service is normally determined by: \\
a. knowing both the pressure and temperature of the cylinder pressure gauges. \\
b. rotameter readings. \\
*c. weighing of the cylinder or tank \\
d. the chlorine residual test . \\
e. None of the above. \\
\item At what level should the exhaust be drawn off in a chlorination room? \\
a. At chest height. \\
b. At least two feet above the height of the chlorine cylinder. \\
c. Near the ceiling. \\
*d. Near the floor. \\
\item Chloramines are \\
*a. Combined chlorine. \\
b. Enzymes. \\
c. Found in polluted air. \\
d. Free chlorine. \\
\item Chlorine gas \\
a. Is lighter than air. \\
*b. Is heavier than air. \\
c. Is pink in color. \\
d. Will liquify at 70 degrees F. \\
\item Chlorine gas is \\
a. Colorless. \\
*b. Heavier than air. \\
c. Non-toxic. \\
d. Odorless. \\
\item Chlorine is: \\
a. Colorless \\
b. Explosive \\
*c. Toxic \\
d. All of the above \\
\item Chlorine is being applied at a constant dose rate of 24 mg/L to a partially nitrified activated sludge effluent having a pH of 6.8 and a temperature of 67F.  Ammonia-nitrogen is found to range from 2 to 3 mg/L in this effluent. Disinfection in this effluent might be difficult because: \\
a. a temperature of 70' F or higher is necessary in order to achieve effective disinfection. \\
b. chloramines are present most of the time. \\
c. the chlorine dose rate is too low. \\
d. the pH of this effluent will limit the effectiveness of free chlorine. \\
*e. the ratio of chlorine to ammonia-nitrogen may make it difficult at times to maintain adequate chlorine residual. \\
\item Chlorine is being fed at the rate of 75 pounds per day. Plant flow is 1.2 MGD. The chlorine residual is measured and found to be 2.6 mg/L Calculate chlorine demand. \\
*a. 4.9 mg/L \\
b. 5.7 mg/L \\
c. 7.5 mg/L \\
d. 8.3 mg/L \\
\item Chlorine is used to \\
*a. Disinfect. \\
b. Prevent corrosion. \\
c. Raise the pH. \\
d. Stabilize organics. \\
\item Chlorine residual may be determined using the reagent \\
*a. Diethyl-p-phenylenediamine (DPD). \\
b. Ethylendiamine tetraacetic acid (EDTA). \\
c. Polychlorinated biphenyls (PCB). \\
d. Sodium thiosulfate (Na2S203) \\
\item "Chlorine residual" refers to: \\
a. the amount of chlorine remaining in the ton cylinder after use. \\
b. the amount of chlorine consumed during disinfection. \\
*c. the chlorine remaining after disinfection. \\
d. the chlorine that displays no disinfection power. \\
e. the residue left after the evaporation of chlorine gas. \\
\item The effectiveness of chlorine disinfection is measured by: \\
a. the chlorine demand \\
b. the chlorine dosage \\
c. the total chlorine residual \\
*d. the coliform concentration of the effluent \\
\item The amount of chlorine used per day from a 1 ton chlorine cylinder is normally determined by: \\
a. Pressure gauges. \\
b. Rotometers. \\
*c. Weighings. \\
d. Chlorine residuals. \\
e. Ammonia equivalents. \\
\item Which of the following discharges would in general, require the lowest chlorine dosage to ensure adequate disinfection? \\
a. Primary plant effluent \\
b. Activated sludge plant effluent \\
c. Trickling filter plant effluent \\
*d. Sand filter effluent \\
e. Stabilization pond effluent \\
\item Which of the following are factors that may influence the effectiveness of chlorine? \\
a. Chlorine dose rate \\
b. Contact time \\
c. Suspended solids concentration of the wastewater being disinfected. \\
*d. Only (a) and (b) \\
e. (a), (b), and (c) \\
\item The fundamental purpose of disinfection is to: \\
a. Destroy fecal coliform bacteria \\
b. Destroy all bacteria \\
*c. Destroy pathogenic organisms \\
d. Protect downstream users from waterborne diseases \\
\item In the application of chlorine for disinfection, which of the following is not normally an operational consideration? \\
a. Mixing \\
b. Contact time \\
*c. DO \\
d. pH \\
e. None of the above \\
\item "Chlorine residual" refers to: \\
a. the amount of chlorine remaining in the ton cylinder after use. \\
b. the amount of chlorine consumed during disinfection. \\
*c. the chlorine remaining after disinfection. \\
d. the chlorine that displays no disinfection power. \\
e. the residue left after the evaporation of chlorine gas. \\
\item In the application of chlorine for disinfection, which of the following is not normally an operational consideration? \\
a. Mixing \\
b. Contact time \\
*c. DO \\
d. pH \\
e. None of the above \\
\item A chlorine residual is often maintained in a plant effluent: \\
a. to keep the chlorinator working. \\
b. for control of fluctuation of wastewater flow. \\
c. for testing purposes. \\
*d. to protect the bacteriological quality of the receiving water. \\
e. None of the above. \\
\item At a wastewater treatment plant. The amount of chlorine used in a day from a cylinder or tank that is in service is normally determined by: \\
a. knowing both the pressure and temperature of the cylinder pressure gauges. \\
b. rotameter readings. \\
*c. weighing of the cylinder or tank \\
d. the chlorine residual test . \\
e. None of the above. \\
\item The fundamental purpose of disinfection is to: \\
a. Destroy fecal coliform bacteria \\
b. Destroy all bacteria \\
*c. Destroy pathogenic organisms \\
d. Protect downstream users from waterborne diseases \\
\item One liter of liquid chlorine can evaporate and produce how many liters of chlorine gas? \\
a. 100 \\
b. 250 \\
*c. 460 \\
d. 490 \\
\item Identify the incorrect statement regarding disinfection. \\
a. When chlorine is added to water it forms acids, which tend to lower the pH of the wastewater effluent \\
b. HTH is a dry form of calcium hypochlorite \\
c. Appropriate doses of chlorine may be used to control odors, control filamentous bulking in activated sludge mixed liquor, or reduce BOD5 of wastewater \\
*d. Hypochlorite’s are sometimes used in place of chlorine because they are more effective and less costly \\
\item Hypochlorite solution is used in effluent disinfection because: \\
a. Chlorine residual determination is more stable and accurate in hypochlorite \\
b. Hypochlorite residuals are more resistant to nitrite interference \\
*c. Chlorine gas is more hazardous to store and handle \\
d. Hypochlorite solution is easier and less costly to ship than gas chlorine \\
e. Chlorine causes too many problems with disinfection efficiency \\
\item Chlorine is: \\
a. Colorless \\
b. Explosive \\
*c. Toxic \\
d. All of the above \\
\item Exhaust from a chlorinator room should be taken from \\
a. Anywhere--the location is not important. \\
*b. At floor level. \\
c. Close to the entrance. \\
d. In the ceiling. \\
\item How many pounds of chlorine gas is necessary to treat 4,000,000 gallons of wastewater at a dosage of 2 mg/L? \\
a. 61 lbs. \\
b. 65 lbs. \\
*c. 67 lbs. \\
d. 69 lbs. \\
\item How much chlorine is needed to provide 10 mg/L dosage for a flow of 2 MGD? \\
a. 75 lbs. \\
b. 83 lbs. \\
c. 130 lbs. \\
*d. 167 lbs. \\
\item Hypochlorite solution is used in effluent disinfection because: \\
a. Chlorine residual determination is more stable and accurate in hypochlorite \\
b. Hypochlorite residuals are more resistant to nitrite interference \\
*c. Chlorine gas is more hazardous to store and handle \\
d. Hypochlorite solution is easier and less costly to ship than gas chlorine \\
e. Chlorine causes too many problems with disinfection efficiency \\
\item Identify the incorrect statement regarding disinfection. \\
a. When chlorine is added to water it forms acids, which tend to lower the pH of the wastewater effluent \\
b. HTH is a dry form of calcium hypochlorite \\
c. Appropriate doses of chlorine may be used to control odors, control filamentous bulking in activated sludge mixed liquor, or reduce BOD5 of wastewater \\
*d. Hypochlorite’s are sometimes used in place of chlorine because they are more effective and less costly \\
\item If a chlorinator is connected to the bottom valve of a half full one-ton cylinder, the chlorinator is withdrawing \\
a. Chlorine gas. \\
*b. Liquid chlorine. \\
c. Liquid or gas chlorine, depending upon temperature. \\
d. Nothing, since there is only one connection. \\
\item If you encounter a liquid chlorine leak in a one-ton cylinder, what action will immediately help to reduce the effect of the leak? \\
a. Spray it with water. \\
b. Apply ice. to leaking area. \\
c. Spray it with ammonia solution. \\
*d. Rotate cylinder so leak is in uppermost position. \\
\item In the application of chlorine for disinfection, which of the following is not normally an operational consideration? \\
a. Mixing \\
b. Contact time \\
*c. DO \\
d. pH \\
e. None of the above \\
\item One liter of liquid chlorine can evaporate and produce how many liters of chlorine gas? \\
a. 100 \\
b. 250 \\
*c. 460 \\
d. 490 \\
\item The amount of chlorine used per day from a 1 ton chlorine cylinder is normally determined by: \\
a. Pressure gauges. \\
b. Rotometers. \\
*c. Weighings. \\
d. Chlorine residuals. \\
e. Ammonia equivalents. \\
\item The amperometric titration method is used to measure: \\
a. Alkalinity. \\
*b. Chlorine residual. \\
c. pH. \\
d. Total hardness. \\
\item The fundamental purpose of disinfection is to: \\
a. Destroy fecal coliform bacteria \\
b. Destroy all bacteria \\
*c. Destroy pathogenic organisms \\
d. Protect downstream users from waterborne diseases \\
\item The measure of the effectiveness of chlorine in disinfection is: \\
a. The chlorine demand. \\
b. The chlorine dosage. \\
*c. The chlorine residual. \\
d. The amount of chloramine formed. \\
e. The final effluent coliform concentration. \\
\item When replacing piping in your chlorinator, be sure to \\
a. Replace with steel lines only. \\
b. Replace with CIP only. \\
*c. Replace with corrosion resistant piping only. \\
d. No special reqirement. \\
\item Which of the following discharges would, in general, require the lowest chlorine dosage to ensure adequate disinfection? \\
a. Primary plant effluent \\
b. Activated sludge plant effluent \\
c. Trickling filter plant effluent \\
*d. Sand filter effluent \\
e. Stabilization pond effluent \\
\item Which of the following are factors that may influence the effectiveness of chlorine? \\
a. Chlorine dose rate \\
b. Contact time \\
c. Suspended solids concentration of the wastewater being disinfected. \\
*d. Only (a) and (b) \\
e. (a), (b), and (c) \\
\item Which of the following are safe procedures for handling chlorine cylinders? \\
a. Keep cylinders close to direct heat to prevent freezing. - \\
*b. Protective cap should always be in place when moving a cylinder. \\
c. Roll cylinders in a horizontal position. \\
d. Store cylinders on their sides. \\
\item Which of the following discharges would in general, require the lowest chlorine dosage to ensure adequate disinfection? \\
a. Primary plant effluent \\
b. Activated sludge plant effluent \\
c. Trickling filter plant effluent \\
*d. Sand filter effluent \\
e. Stabilization pond effluent \\



\end{enumerate}

