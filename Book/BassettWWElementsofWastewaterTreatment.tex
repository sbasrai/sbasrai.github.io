\chapterimage{ElementsofTreatmentImg.png} % Chapter heading image
\chapter{Elements of Wastewater Treatment}

Wastewater cycle is a part of the water cycle where the water consumed as part of the normal human and industrial activity is returned back to the environment after treatment.

Wastewater cycle comprises of the following sequential elements:
\begin{enumerate}
\item Generation
\item Collection
\item Treatment
\item Disposal or reuse
\end{enumerate}

\section{Generation}\index{Generation}

Wastewater originates from domestic, industrial, commercial or agricultural activities. The characteristics of wastewater vary depending on the source. Types of wastewater include: 
\begin{itemize}
\item \hl{Domestic Sewage:}  wastewater derived principally from dwellings, business buildings, institutions, and \\
\item \hl{Industrial Sewage:}  liquid waste from industrial processes\\
\end{itemize}
Typical per person generation of wastewater in the USA is about 70-100 gallons per day

\section{Collections}\index{Collections}

\begin{itemize}
\item Wastewater is collected from its point of origin - home, businesses, industries etc. and conveyed via sewer lines to a centralized wastewater treatment facility.  
\item When the rainwater drainage is made part of the sewer system, the system is termed as \hl{Combined System}.  
\item The system where the sewage is conveyed separately from the stormwater flows is termed as \hl{Separated System}.  
\item In the Separated System, the Sanitary Sewers convey the wastewater and the Stormwater Sewer conveys the storm water flows.  
\item For the Combined System, rainstorms pose the threat of overwhelming the sewers and the treatment plant
\end{itemize}  
\begin{center}
\includegraphics[scale=0.45]{SeperatedSystem1} \hspace{1 cm} \includegraphics[scale=0.45]{CombinedSystem1}
\end{center}
			\hspace{2.6cm} Separated System \hspace{3.2cm} \parbox{\textwidth}{Combined System}\\

\section{Treatment}\index{Treatment}
\subsection{Liquid Phase Treatment}\index{Liquid Phase Treatment}
\begin{itemize}
\item Wastewater treatment can involve physical, chemical or biological processes or combinations of these processes depending on the required outflow standards. 
\item Wastewater treatment typically involves a series of steps with increasing level of treatment:
\begin{itemize}
\item \hl{Preliminary}:  The preliminary process removes large/coarse solids which include rocks, tree branches, grit and other debris present in wastewater.
\item \hl{Primary}:  The primary process is also a physical process where the separable wastewater solids - solids that float and solids that can settle, are removed.  
\item \hl{Secondary}:  Secondary treatment is a biological treatment process where microorganisms consume the organic matter present in the wastewater. 
\item \hl{Tertiary or Advanced Treatment}:  The tertiary/advanced treatment processes improve the quality of treated water beyond the secondary treatment level.  This process may include nutrient removal and disinfection.
\end{itemize}

\subsection{Treatment of Wastewater Solids}\index{Treatment of Wastewater Solids}
\begin{itemize}
\item Solids are a byproduct of wastewater treatment.  
\item Screenings and grit removed as part of the preliminary treatment is typically disposed in a landfill.
\item Sludge generated from the wastewater treatment processes -  settled solids and scum from primary and secondary treatment processes needs to be treated prior to disposal or reuse to comply with wastewater solids - biosolids regulations.
\end{itemize}

\vspace{0.5cm}
Typical solids treatment is comprised of the following three sequential steps:
\begin{enumerate}
\item Sludge thickening
\item Sludge stabilization
\item Sludge dewatering
\end{enumerate}
\vspace{0.5cm}
\subsubsection{Sludge Thickening}\index{Sludge thickening}
Sludge thickening improves performance of sludge stabilization process and provides capital and operational cost savings due to a lower volume of sludge
\subsubsection{Sludge Stabilization}\index{Sludge Stabilization}
Sludge stabilization process produces solids (biosolids) that meet Part 503 rule requirements. 
\subsubsection{Sludge Dewatering}\index{Sludge Dewatering}
Solids stabilized using digestion process has only a small percentage by weight of solids -less than 5\%.  It therefore becomes necessary to dewater the stabilized sludge prior to hauling off-site for final disposal. 
\vspace{0.5cm}
A generalized layout/process sequencing in a wastewater treatment plant is shown below:
\begin{center}
\includegraphics[scale=0.6]{TreatmentFlow}
\end{center}
Individual wastewater treatment processes involve different process options or sequences which are illustrated in the graphic below:
\begin{center}
\includegraphics[scale=0.42]{Treatment}
\end{center}
\end{itemize}

\section{Disposal or Reuse}\index{Disposal or Reuse}

\begin{itemize}
\item Wastewater treatment processes can be designed to \hl{dispose} the treated water where the water is reintroduced to the environment or for \hl{reuse} where the treated water is \hl{reclaimed} or \hl{recycled} - for various purposes including irrigation, industrial use or for potable use.
\item Water disposal methods include:\\
\begin{itemize}
\item \hl{Surface water discharge}
\item \hl{Subsurface discharge}
\end{itemize}
\item Water reuse methods include:\\
\begin{itemize}
\item Potable water reuse
\begin{itemize}
\item \hl{Indirect potable reuse:}  Here the treated water is blended with groundwater or surface water and then reclaimed and treated further 
for drinking (potable) water use
\item \hl{Direct potable reuse:}  Here the treated wastewater is subjected to advanced treatment and introduced directly into a municipal water supply system
\end{itemize}
\item Water reclamation for irrigation or industrial use\\
\item Land application for beneficial use\\
\end{itemize}
\item Solids generated from the wastewater treatment process may be removed and disposed to a landfill or subject to further treatment which may allow for energy recovery - from the organic solids and for beneficial reuse due to its plant nutrient content.\\
\end{itemize}

