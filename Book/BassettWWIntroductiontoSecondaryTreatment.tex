% \documentclass{article}
% %\usepackage[english]{babel}%
% \usepackage{graphicx}
% \usepackage{tabulary}
% \usepackage{tabularx}
% \usepackage[normalem]{ulem}
% \usepackage{cancel}
% \usepackage{tikz} 
% \usepackage{pdflscape}
% \usepackage{colortbl}
% \usepackage{lastpage}
% \usepackage{multirow}
% \usepackage{enumerate}
% \usepackage[shortlabels]{enumitem}
% \usepackage{color,soul}
% \usepackage{pdflscape}
% \usepackage{hyperref}
% %\usepackage[table]{xcolor}
% \usepackage{rotating}
% \usepackage{amsmath}
% \usepackage{fixltx2e}
% \usepackage{framed}
% \usepackage{mdframed}
% \usepackage[T1]{fontenc}
% \usepackage[utf8]{inputenc}
% \usepackage{textcomp}
% \usepackage{siunitx}
% \usepackage{ifthen}
% \usepackage{fancyhdr}
% \usepackage{gensymb}
% \usepackage{newunicodechar}
% \usepackage[document]{ragged2e}
% \usepackage[margin=1in,top=1.1in,headheight=57pt,headsep=0.1in]
% {geometry}
% \usepackage{ifthen}
% \usepackage{fancyhdr}
% \everymath{\displaystyle}
% \usepackage[document]{ragged2e}
% \usepackage{fancyhdr}
% \everymath{\displaystyle}
% \usepackage{empheq}

% \usepackage[most]{tcolorbox}

% \usepackage{booktabs} % Required for nicer horizontal rules in tables


% \usepackage{enumitem}

% %\usepackage[table,xcdraw]{xcolor}
% \usetikzlibrary{arrows}
% \linespread{2}%controls the spacing between lines. Bigger fractions means crowded lines%
% %\pagestyle{fancy}
% %\usepackage[margin=1 in, top=1in, includefoot]{geometry}
% %\everymath{\displaystyle}
% \linespread{1.3}%controls the spacing between lines. Bigger fractions means crowded lines%
% %\pagestyle{fancy}
% \pagestyle{fancy}
% \setlength{\headheight}{56.2pt}

% \definecolor{myblue}{rgb}{.8, .8, 1}
% \newcommand*\mybluebox[1]{%
% \colorbox{myblue}{\hspace{1em}#1\hspace{1em}}}

% \chead{\ifthenelse{\value{page}=1}{\includegraphics[scale=0.3]{SCC}\\ \textbf \textbf Wastewater Constituents Analysis \& Laboratory Methods}}
% \rhead{\ifthenelse{\value{page}=1}{}{}}
% \lhead{\ifthenelse{\value{page}=1}{}{Wastewater Constituents Analysis \& Laboratory Methods}}
% \rfoot{\ifthenelse{\value{page}=1}{Module 1: WATR 048 - Spring 2019}{Module 1: WATR 048 - Spring 2019}}

% \lfoot{Shabbir Basrai}
% \cfoot{Page \thepage\ of \pageref{LastPage}}
% \renewcommand{\headrulewidth}{2pt}
% \renewcommand{\footrulewidth}{1pt}
% \begin{document}
% %\begin{empheq}[box=\mybluebox]{align}
% %a&=b\\
% %E&=mc^2 + \int_a^a x\, dx
% %\end{empheq}

% \newlist{steps}{enumerate}{1} % Defines "Steps" for enumerate as Step 1, Step 2 etc.
% \setlist[steps, 1]{label = Step \arabic*:} % Defines "Steps" for enumerate as Step 1, Step 2 etc.

% \setlist{nolistsep} % Reduce spacing between bullet points and numbered lists


%_______________________________________________________________________________________________________________________________________%
\chapterimage{IntroductionSecondaryTreatmentImage.jpg} % Chapter heading image

\chapter{Introduction to Secondary Treatment}
\begin{itemize}
\item While preliminary and primary treatment processes are designed primarily to remove solids from wastewater, secondary treatment is for the removal of organics.
\item Secondary treatment involves:
\begin{itemize}
\item biological conversion of the dissolved and suspended organics in wastewater into biomass, and
\item physical settling (separation) process where the solids including the biomass formed during secondary treatment is separated and removed from the treated wastewater.
\end{itemize}

\item With the removal of gross solids in the preliminary treatment followed by removable of settleable solids in the primary clarifiers and the removal of dissolved and suspended organics in the secondary treatment processes, the wastewater is considered treated.
\item Secondary treated wastewater is typically disposed or treated further for reuse or disposal (depending upon the end use/application and the NPDES permit stipulations).
\item The solids (biomass) removed from the secondary treatment is typically mixed with the solids from primary treatment and stabilized using a solids treatment process like sludge digestion prior to its disposal.
\end{itemize}
\vspace{1cm}

\textbf{Secondary treatment process incorporates one of the following three approaches:}


\section{Fixed film system}\index{Fixed Film System}	

\begin{itemize}
\item Here the microorganisms responsible for the treatment, grow on substrates such
as rocks, sand or plastic.
\item When the wastewater is spread over the substrate, the microorganisms up-take the organics present in the wastewater
\item Example of this secondary treatment process include trickling filters and rotating biological contactors\\
\end{itemize}

\section{Suspended Growth System}\index{Suspended Growth System}
\begin{itemize}
\item In this type of secondary treatment, the microbes are suspended in the
wastewater flow being treated. 
\item Air or oxygen is supplied to maintain an aerobic environment and to keep the microorganisms in suspension. 
\item Example of this secondary treatment approach include the activated sludge treatment process 
\end{itemize}

\section{Pond System}\index{Pond System}
Similar to the suspended growth, stabilization ponds are large man made bodies of water which treat wastewater using mainly natural processes including sunlight, algae and microorganisms.

