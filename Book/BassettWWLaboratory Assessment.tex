\chapterimage{QuizCover} % Chapter heading image

\chapter{Constituents Properties and Analysis Assessment}
% \textbf{Multiple Choice}

\section*{Constituents Properties and Analysis Assessment}

\begin{enumerate}

\item Positive ORP indicates the presence of an {\underline{\hspace{1cm}}} environment \\

\item {\underline{\hspace{1cm}}} is the predominant microorganism responsible for biological wastewater treatment processes \\

\item MPN is a statistics based method to estimate the concentration of viable bacteria is wastewater and it stands for {\underline{\hspace{1cm}}} \\

\item Describe the three types of composite sampling and explain in your own words on how the sampling would be conducted for each of those methods\\

\item Alkalinity helps in \rule{1.5cm}{0.3mm}control when organic acids are formed as a result of microbiological activity\\

\item Nitrogen in wastewater is typically present as \rule{1.5cm}{0.3mm} and \rule{1.5cm}{0.3mm}\\

\item Volatile solids is determined by incinerating the solids at \rule{1.5cm}{0.3mm} deg. C in a \rule{1.5cm}{0.3mm}

\item  What would be the expected BOD concentration in the wastewater if per capita (person) generation is 0.15lb BOD per day per person and each person produces 80 gallons per day?\\
(Hint: The answer is going to be in mg BOD/l. So multiply lb per day per person by the inverse of dallons per day - so that the unit is lbs per gallon and then convert that to mg/l)\\


\item  Describe the three types of composite sampling and explain in your own words on how the sampling would be conducted for each of those methods \\


\item  tBOD = \rule{.5cm}{0.3mm}BOD + \rule{0.5cm}{0.3mm}BOD

\item  BOD stands for \\

\item  Alkalinity helps in {Nderline{hspace{1cm}}} control when organic acids are formed as a result of microbiological activity \\

\item  Nitrogen in wastewater is typically present as {\underline{\hspace{1cm}}} and {\underline{\hspace{1cm}}} \\


\item  Volatile solids is determined by incinerating the solids at {\underline{\hspace{1cm}}} [temp] deg. C in a {\underline{\hspace{1cm}}} [equipment] 

\item How does one estimate the wastewater sample size used for BOD testing \\

\item Explain how is the nBOD quantified \\

\item Explain why coliforms and enterococcus are used for wastewater bacteriological testing \\

\item What is MPN and why is it used. \\

\item Explain the steps involved in the MTF method\\

\item What is the difference between volatile solids (VS) and volatile suspended solids (VSS) \\

\item Solids that can be used as food by microorganisms \\

\item Test method for determining total ammonium and organic nitrogen content: \rule{0.5cm}{0.3mm}\\

\item Inorganic solids are also called \rule{0.5cm}{0.3mm}\\

\item Oxygen in an anoxic environment is present as {\underline{\hspace{1cm}}} \\

\item Conductivity is the measure of {\underline{\hspace{1cm}}} solids \\

\item Negative ORP signifies {\underline{\hspace{1cm}}} environment \\

\item Alkalinity helps in \rule{0.5cm}{0.3mm} control when organic acids are formed as a result of microbiological activity \\

\item  An Imhoff cone is often used to measure the effectiveness of primary sedimentation. \\

a. True \\
*b. False 

\item  At a primarily domestic wastewater treatment plant, the influent wastewater BOD is always greater than its chemical oxygen demand. \\

a. True \\
*b. False 

\item  Dissolved oxygen in wastewater usually is referred to as combined oxygen. \\

a. True \\
*b. False 

\item  Domestic wastewater generally contains only about 0.1% of solids. \\

*a. True \\
b. False 

\item  Grab or composite samples may be used interchangeable, whichever is most convenient and safest for all laboratory tests. \\

a. True \\
*b. False 

\item  Wastewater with a pH of 12.0 to 14.0 would be too "acidic" for biological treatment. \\

*a. True \\
b. False 

\item  Pre-aeration improves settleability in primary clarifier \\

*a. True \\
b. False 

\item  Total solids are made up of dissolved and suspended solids both of which contain organic and inorganic matter \\

*a. True \\
b. False 

\item  pH value less than 7 indicates an alkaline or basic condition \\

a. True \\
*b. False 

\item  Conductivity is a very useful test for assessing sea water intrusion into sewer lines \\

*a. True \\
b. False 


\item  Sample obtained for wastewater bacteriological testing is typically a composite sample \\

a. True \\
*b. False 

\item  Samples collected for the analysis of COD, BOD, and pH should be acidified for preservation. \\

a. True \\
*b. False 

\item  The MPN test is used to measure pathogen concentrations in wastewater. \\

a. True \\
*b. False 

\item  The total solids in wastewater would be the combination of the fixed solids and the settleable solids. \\

a. True \\
*b. False 

\item  Wastewater with a pH of 2.0 to 4.0 would be too "basic" for biological treatment. \\

a. True \\
*b. False 

\item  Match the measurement units for each of the following: 

\item  The MPN test measures the number of pathogens in a wastewater sample \\

a. True \\
*b. False 

\item  The values of organic matter present in wastewater as measured by BOD and COD tests are typically almost identical \\

a. True \\
*b. False 

\item  In order to obtain valid results in coliform testing, the sample must be dechlorinated at the time of its collection. \\

*a. True \\
b. False 

\item  A flow of 100 gallons per capita per day is often used for estimating flow into a wastewater treatment plant. \\

*a. True \\
b. False 

\item  An Imhoff cone is used to measure settleable solids in units of mg/l. \\

a. True \\
*b. False 

\item  A standardized method exists for the measurement of floatable solids. \\

a. True \\
*b. False 

\item  In the test for coliform bacteria, samples are incubated for 5 days at 20 deg. C. \\

a. True \\
*b. False 

\item  Coliform group organisms are found only in wastewater. \\

a. True \\
*b. False 

\item  Where highly colored samples are involved, the determination of pH by use of a pH meter rather than by color-comparison is\\
preferred. \\

*a. True \\
b. False 

\item  What comes into a treatment plant must go out. This is the basis of the solids balance concept. \\

*a. True \\
b. False 

\item  Receiving water measurements are used to determine the effect of the plant's waste discharge on the receiving waters; therefore, it is necessary to measure both stream and plant\\
effluent characteristics. \\

*a. True \\
b. False 

\item  There are two types of samples that may be collected. One is called an integrated sample, and the other is referred to as a composite. \\

a. True \\
*b. False 

\item  Coliform testing is typically conducted on a grab sample \\

*a. True \\
b. False 

\item  Typical domestic wastewater BOD content is about 2000mg/l \\

a. True \\
*b. False 

\item  BOD is a measure of the organic content of wastewater \\

*a. True \\
b. False 

\item  ORP measurements can be used for controlling the disinfection process \\

*a. True \\
b. False 

\item  MPN is used for enumerating wastewater bacteria and it stands for Measured Pathogen Number \\

a. True \\
*b. False 

\item  For measuring dissolved oxygen in wastewater, it is advisable to use a composite sample \\

*a. True \\
b. False 

\item  A sample to be used for pH measurements should be preserved by the addition of an acid prior to its analysis in the laboratory \\

a. True \\
*b. False 

\item  BOD is a measure of the organic content of wastewater and it stands for Biological Oxidation Demand \\

a. True \\
*b. False 

\item  Conductivity is measured in the units of millivolts using an electrochemical probe \\

a. True \\
*b. False 

\item  Conductivity measurements provide an indirect way to measure total solids present in wastewater \\

a. True \\
*b. False 

\item  Typical wastewater TSS concentrations are in the 2,000-2,500 mg/l range \\

a. True \\
*b. False


\item  A 24-hr flow proportioned sample may be collected for a fecal coliform test\\

a. True \\
*b. False 

\item  Wastewater with a pH of 12.0 to 14.0 would be too "acidic" for biological treatment.\\

*a. True \\
b. False 

\item  The laboratory measurement of volatile solids is a fair approximation of the organic content of the wastewater.\\

*a. True \\
b. False 

\item  BOD5 and SS are both used to measure the strength of wastewater\\

*a. True \\
b. False 

\item  Pre-aeration improves settleability in primary clarifier\\

*a. True \\
b. False 

\item  pH value less than 7 indicates an alkaline or basic condition\\

a. True \\
*b. False 

\newpage
\item A BOD test is run on a secondary effluent.  The data for this test are given below.  Assuming that the average BOD test result for an “inhibitor” on this same secondary effluent was 22 mg/L (cBOD), calculate the oxygen demand caused by nitrification.  What percentage of the total BOD is the nitrogenous oxygen demand?  On April 04, 2009 Exam
  \begin{flalign*}
      Sample Volume, ml     && 50   &&  75  && Blank\\
      \hline
      Initial DO, mg/l      && 9.0  &&  8.9 && 9.2\\
      Final DO, mg/l      && 3.5  &&  1.2 && 9.0
  \end{flalign*}

\begin{itemize}
\item Solution:
  \begin{flalign*}
      Difference, mg/l      &&5.5   &&7.7 &&9.0\\
      tBOD, mg/l          && 5.5*300/50=33 && 7.7*300/75 = 30.8\\
  \end{flalign*}
Average: $\frac{(33 + 30.8)}{2} = 31.9\frac{mg}{l}$\\
$tBOD = cBOD + nBOD \implies 31.9=22 \enspace + \enspace nBOD \implies nBOD=9.9$\\
$\% nBOD = 9.9/31.9 *100 = \boxed{31\%}$
\end{itemize}
\pagebreak
\item BOD tests are run on the final effluent from an activated sludge plant with and without the use of a "nitrification inhibitor". Three hundred milliliter bottles (300 ml) are used in these tests. The raw data for these tests are presented below.  What is the average NITROGENOUS BOD (NBOD)? Exam on April 04, 2009\\
BOD Test without "inhibitor" (tBOD)\\
\begin{flalign*}
      Sample Volume, ml     && 30   &&  60  && Blank\\
      \hline
      Initial DO, mg/l      && 9.0  &&  8.7 && 9.1\\
      Final DO, mg/l      && 5.1  &&  1.2 && 9.0
  \end{flalign*}
BOD Test with "inhibitor" added (cBOD)\\
\begin{flalign*}
      Sample Volume, ml     && 30   &&  60  && Blank\\
      \hline
      Initial DO, mg/l      && 9.0  &&  8.7 && 9.1\\
      Final DO, mg/l      && 6.5  &&  3.5 && 9.0
  \end{flalign*}
\begin{itemize}
\item Solution:
  \begin{flalign*}
      tBOD \\
      Diff., mg/l       &&3.9   &&7.5 &&0.1\\
      tBOD, mg/l          && 3.9*300/30=39 && 7.5*300/60 = 37.5\\
      cBOD \\
      Diff., mg/l       &&2.5   &&5.2 &&0.1\\
      tBOD, mg/l          && 2.5*300/30=25 && 5.2*300/60 = 26\\
  \end{flalign*}

Average:\\
tBOD: $\frac{(39 + 37.5)}{2} = 38.3\frac{mg}{l}$\\
cBOD: $\frac{(25 + 26)}{2} = 25.5\frac{mg}{l}$\\
$tBOD = cBOD + nBOD \implies 38.3=25.5 \enspace + \enspace nBOD \implies nBOD=\boxed{12.8 \frac{mg}{l}}$\\

\end{itemize}
\pagebreak
\item Calculate the TSS of the secondary effluent given the following:\\
\begin{table}[!htbp]
\fontsize{11}{9}
\centering
\begin{tabular}{p{5cm}  p{2cm}}
\hline
\hline
Sample volume& 41 ml \\ [0.2ex] 
\hline
Tare weight of filter \enspace sample & 1.4604 gm \\ 
\hline
Filter + dried residue & 1.4722 gm\\ 
\hline
\hline
\end{tabular}
\end{table}

\item The technician quickly pours 41 ml of well mixed influent into the filter funnel.  The tare weight of the filter is 1.4604 gm. After rinsing, drying, cooling, and weighing, the first dry weight is 1.4722 gm. The filter is returned to the oven and dried, cooled, and weighed. The second dry weight is 1.4700 gm.  Calculate the TSS.

\begin{itemize}
\item Solution\\
$\frac{(1.4722-1.4604)gm \enspace TSS}{41ml}*1000\frac{mg}{gm}*\frac{1000ml}{l}=\boxed{288\frac{mg}{l}}$
\end{itemize}
\newpage
\item BOD tests are run on the final effluent from an activated sludge plant with and without the use of a "nitrification inhibitor". Three hundred milliliter bottles (300 ml) are used in these tests. The raw data for these tests are presented below.  What \textbf{percentage of the average total BOD is the average nBOD}? (10 points)\\
\begin{flalign*}
      Sample Volume, ml     && 10   &&  20  && 30 && 40 && Blank\\
      \hline
      Initial DO, mg/l      && 9.0  &&  8.9 && 8.8  && 9.1 && 9.1\\
      Final DO, mg/l      && 6.9  &&  4.8 && 2.5 && 1.1 && 9.0
  \end{flalign*}
BOD Test with "inhibitor" added (cBOD)\\
\begin{flalign*}
      Sample Volume, ml     && 10   &&  20  && 30 &&  40 && Blank\\
      \hline
      Initial DO, mg/l      && 8.9  &&  8.9  && 9.0 && 9.0 && 9.1\\
      Final DO, mg/l      && 7.5  &&  6.2  && 5.0  && 3.3 && 9.0
  \end{flalign*}

Solution:
  \begin{flalign*}
      tBOD Diff., mg/l      &&2.1   &&4.1 &&6.3 &&8\\
      tBOD, mg/l          && 2.1*300/10=63.0 && 4.1*300/20 = 61.5 && 6.3*300/30 = 63.0 && 8.0*300/40 = 60.0 \\
\hline
      cBOD Diff., mg/l      &&1.4   &&2.7 &&4.0 &&5.7\\
      cBOD, mg/l          && [Discard - diff. <2] && 2.7*300/20 = 40.5 && 4.0*300/30 = 40 && 5.7*300/40 = 42.8\\
  \end{flalign*}

$tBOD (avg) = 63+61.5+63+60=61.9 \hspace{1cm} cBOD (avg) = 40.5+40+42.75=41.1$\\
nBOD = tBOD - cBOD $\implies$ nBOD = 61.9-41.1=20.8 $\implies$ nBOD(\%)=20.8/61.9*100=$\boxed{33.6\%}$
\newpage
\item BOD tests are run on the final effluent from an activated sludge plant with and without the use of a "nitrification inhibitor". Three hundred milliliter bottles (300 ml) are used in these tests. The raw data for these tests are presented below.  What is the average NITROGENOUS BOD (NBOD)? Exam on April 04, 2009\\
BOD Test without "inhibitor" (tBOD)\\
\begin{flalign*}
      Sample Volume, ml     && 30   &&  60  && Blank\\
      \hline
      Initial DO, mg/l      && 9.0  &&  8.7 && 9.1\\
      Final DO, mg/l      && 5.1  &&  1.2 && 9.0
  \end{flalign*}
BOD Test with "inhibitor" added (cBOD)\\
\begin{flalign*}
      Sample Volume, ml     && 30   &&  60  && Blank\\
      \hline
      Initial DO, mg/l      && 9.0  &&  8.7 && 9.1\\
      Final DO, mg/l      && 6.5  &&  3.5 && 9.0
  \end{flalign*}

Solution:
  \begin{flalign*}
      tBOD \\
      Diff., mg/l       &&3.9   &&7.5 &&0.1\\
      tBOD, mg/l          && 3.9*300/30=39 && 7.5*300/60 = 37.5\\
      cBOD \\
      Diff., mg/l       &&2.5   &&5.2 &&0.1\\
      tBOD, mg/l          && 2.5*300/30=25 && 5.2*300/60 = 26\\
  \end{flalign*}

Average:\\
tBOD: $\frac{(39 + 37.5)}{2} = 38.3\frac{mg}{l}$\\
cBOD: $\frac{(25 + 26)}{2} = 25.5\frac{mg}{l}$\\
$tBOD = cBOD + nBOD \implies 38.3=25.5 \enspace + \enspace nBOD \implies nBOD=\boxed{12.8 \frac{mg}{l}}$\\

\pagebreak

\item Calculate percent total solids and percent volatile solids of a sludge sample given the following data:\\
\begin{tabular}{m {5 cm} m {0.5 cm} m  {3.5 cm}}
Weight of dish &=&  104.55 gms\\
Weight of dish and wet sludge &= & 199.95 gms\\
Weight of dish and dry sludge &= & 108.34 gms\\
Weight of dish and ash &= & 106.37 gms
\end{tabular}\\
(Answer: TS = 3.97\%.  VS = 52\%)\\
\vspace{0.2cm}
Solution:\\
\vspace{0.2cm}
Weight of dish=104.55 gms\\
Weight of dish and wet sludge=199.95 gms\\
Weight of dish and ash = 106.37 gms\\
\vspace{0.2cm}
$ \implies Weight \enspace of \enspace sludge=199.95-104.55=95.40 \enspace gms$\\
$\implies Weight \enspace of \enspace dry \enspace sludge \enspace (solids)=108.34-104.55=3.79 \enspace gms$\\
$\implies Weight \enspace of \enspace volatile \enspace solids=108.34-106.37=1.97 \enspace gms$\\
\vspace{0.2cm}
$Total \enspace solids (TS\%)=\frac{gms \enspace solids}{100 \enspace gms \enspace sludge}=\frac{3.79}{95.40} \enspace \frac{gms \enspace solids}{\cancel{gms \enspace sludge}}*\frac{100 \cancel{\enspace gms \enspace sludge}}{100 \enspace gms \enspace sludge}=\boxed{3.97\%}$\\
\vspace{0.2cm}
$Total \enspace volatile \enspace solids (VS\%) =\frac{1.97}{3.79} \enspace \frac{gms \enspace volatile \enspace solids}{\cancel{gms \enspace total \enspace solids}}*\frac{100 \cancel{\enspace gms \enspace total \enspace solids}}{100 \enspace gms \enspace total \enspace solids}=\boxed{52.0\%}$\\
\newpage

\item What is percent volatile solids of a sludge sample given the following data:\\
\begin{tabular}{m {5 cm} m {0.5 cm} m  {3.5 cm}}
Weight of dish &=&  130.69 gms\\
Weight of dish and wet sludge &= & 249.94 gms\\
Weight of dish and dry sludge &= & 134.74 gms\\
Weight of dish and ash &= & 132.05 gms
\end{tabular}\\
\vspace{0.2cm}
Solution:\\
\vspace{0.2cm}
Weight of dish=130.69 gms\\
Weight of dish and wet sludge=249.94 gms\\
Weight of dish and ash = 132.05 gms\\
\vspace{0.2cm}
$ \implies Weight \enspace of \enspace sludge=249.94-130.69=119.25 \enspace gms$\\
$\implies Weight \enspace of \enspace dry \enspace sludge \enspace (solids)=134.74-130.69=4.05 \enspace gms$\\
$\implies Weight \enspace of \enspace volatile \enspace solids=134.74-132.05=2.69 \enspace gms$\\
\vspace{0.2cm}
$Total \enspace solids (TS\%)=\frac{gms \enspace solids}{100 \enspace gms \enspace sludge}=\frac{4.05}{119.25} \enspace \frac{gms \enspace solids}{\cancel{gms \enspace sludge}}*\frac{100 \cancel{\enspace gms \enspace sludge}}{100 \enspace gms \enspace sludge}=\boxed{3.4\%}$\\
\vspace{0.2cm}
$Total \enspace volatile \enspace solids (VS\%) =\frac{2.69}{4.05} \enspace \frac{gms \enspace volatile \enspace solids}{\cancel{gms \enspace total \enspace solids}}*\frac{100 \cancel{\enspace gms \enspace total \enspace solids}}{100 \enspace gms \enspace total \enspace solids}=\boxed{66.4\%}$\\
\newpage
\item Products that are non-biodegradable will have {\underline{\hspace{1cm}}} as compared with biodegradable products \\

a. Same BOD \\
*b. A lower BOD \\
c. A higher BOD \\
d. There is no relationship between BOD and biodegradability 

\item Coliform bacteria are \\

a. Algae \\
b. Coagulant aids \\
*c. Indicators \\
d. Sequestering agents 



\item What is percent volatile solids of a sludge sample given the following data:\\
Weight of dish = 130.69 gms\\
Weight of dish and wet sludge = 249.94 gms\\
Weight of dish and dry sludge = 134.74 gms\\
Weight of dish and ash = 132.05 gms \\

*a. 38.3\% \\
b. 66.4\% \\
c. 42.6\% \\
d. 58.0\% 

\item What is percent volatile solids of a sludge sample given the following data:\\
Weight of dish = 130.69 gms\\
Weight of dish and wet sludge = 249.94 gms\\
Weight of dish and dry sludge = 134.74 gms\\
Weight of dish and ash = 132.05 gms\\

*a. 38.3\% \\
b. 66.4\% \\
c. 42.6\% \\
d. 58.0\% 

\item A device called an Imhoff cone is commonly used to measure settleable solids in: \\

a. \% \\
*b. mL/L \\
c. mg/L \\
d. ppm \\
e. SVI units 

\item An aerobic treatment process is one that requires the presence of: \\

a. Ozone \\
b. organic oxygen \\
c. no oxygen \\
d. combined oxygen \\
*e. dissolved oxygen 

\item A pH probe: \\

a. Can be used to measure ORP in chlorine disinfection. \\
b. Is often used to measure hydrogen production in wet wells. \\
c. Measures, in millivolts, the difference between oxidants like chlorine and reductants such as organic matter. \\
*d. Measures hydrogen ion activity in wastewater. \\
e. Sends a 4-20 mA signal directly to a chlorine controller. 

\item Sludge solids in wastewater has an average specific gravity of 1.2; this means they are \\

a. 12\% heavier than water \\
*b. 20\% heavier than water \\
c. 2\% lighter than water \\
d. 20\% lighter than water 

\item How should the pH electrode be stored when not in use: \\

a. In a strong acid solution \\
b. In a strong caustic solution \\
c. In a safe place in a drawer \\
*d. In distilled water \\
e. In a detergent 

\item In the normal Winkler test: \\

a. A snow white precipitate forms in direct proportion to the nitrate concentration \\
b. A brownish flocculant precipitate is evidence that D.O. is absent \\
c. An endpoint is reached when a dark blue color changes to black \\
d. The muffle furnace must be in excess of 500 deg. C before incubation \\
*e. A snow white precipitate forms if DO is absent 

\item Organisms in wastewater that are not harmful to humans but are indicators of diseases are: \\

a. Pathogens \\
b. Viruses \\
*c. Coliform \\
d. Bacteria 

\item The typical range of suspended solids in domestic influent wastewater is: \\

*a. 100-300 mg/L \\
b. 400-600 mg/L \\
c. 700-900 mg/L \\
d. 1000-12000 mg/L 

\item Which of the flowing statement(s) is/are true with regards to BOD\\
i) BOD test results are suitable for quickly establishing process efficiencies\\
ii) BOD value is always greater than the COD value of the same wastewater sample\\
iii) BOD is expressed in mg/ L or in ppm\\
iv) BOD is the measure of organic strength\\
v) BOD stands for biological oxygen demand \\

a. i) \& ii) \\
b. i), ii) \& iv) \\
c. i), iv) \& v) \\
*d. iii) \& iv) \\
e. iii), iv) \& v) 

\item An amperometric titrater is used to measure \\

a. Alkalinity \\
*b. Chlorine residual \\
c. Conductivity \\
d. COD. 

\item Which of these pH readings indicates an acidic wastewater? \\

*a. 3 \\
b. 7 \\
c. 9 \\
d. 12 

\item Which of the following conditions will probably cause the greatest change in pH? \\

a. Buffering sample \\
*b. Exposing sample to atmosphere \\
c. Fixing sample \\
d. Refrigerating sample 

\item {Nderline{hspace{1cm}}} matter in wastewater, is normally composed of grit, sand and silt. \\

a. Colloidal \\
*b. Inorganic \\
c. Organic \\
d. Volatile 

\item Which of the following characteristics would be least helpful to an operator assessing the organic loading on his plant? \\

a. solids concentration \\
b. BOD \\
c. COD \\
*d. pH \\
e. nitrogen content 

\item Pathogens \\

*a. Are bacteria or virus that cause disease. \\
b. Are bacteria which do not occur in water. \\
c. Can obtain their food supply without help. \\
d. Are not harmful to man. 

\item Regarding the total coliform test which one of the following statements is TRUE? \\

a. If less than 2 coliforms per 100 mL are found in a secondary effluent, we can be assured that we have destroyed all viruses and coliforms. \\
b. Total coliforms are monitored in wastewater because they are generally considered to be disease causing organisms. \\
c. Sodium thiosulfate should be added drop-wise after collection of a coliform sample to destroy residual chlorine. \\
*d. The multiple tube fermentation method for measuring total coliforms is only a statistical estimate of the coliform organism concentration. 

\item Results from the Multiple-Tube Fermentation Technique for members of the Total Coliform Group are expressed as \\

a. DPD. \\
b. MF. \\
c. MGD. \\
*d. MPN. 

\item The advantage in the use of coliform organisms as an indicator lies in the following fact: \\

a. They are found everywhere and they grow in common bacterial media \\
b. They are found everywhere and are readily killed by chlorine \\
*c. They are predominant bacteria associated with intestinal discharges and grow on\\
nutrient agar forming characteristic colonies 

\item The purpose of adding sodium thiosulfate to a microbiological sample bottle is to \\

a. Extend the allowable holding time from 6 to 30 hours. \\
b. React with nitrates that interfere with the MPN test. \\
*c. Remove any chlorine residual present. \\
d. To ensure sterilization of sample bottle. 

\item The solids in a raw wastewater sample may be classified in several different categories. However, one may say that all the solids in a wastewater sample may be expressed as the sum of: \\

a. The total of settleable solids + total dissolved solids. \\
*b. The total dissolved solids + total suspended solids. \\
c. The total of settleable solids + colloidal solids + total suspended solids 

\item The term MPN is used in reference to: \\

a. The mass of phosphorus and nitrogen per unit of carbon \\
*b. The number of coliforms per unit volume of sample that is most likely to have caused the observed results in a multiple tube test \\
c. The number of fecal stetococci per unit volume of sample that is most likely to have caused the observed results in the multiple tube test \\
d. The result of membrane filter test \\
e. The standard plate count result 

\item The volatile portion of suspended solids contained in normal domestic wastewater could be expected to be in the range of \\

a. Less than 10 % • \\
b. 25-50%. \\
*c. 70-80%. \\
d. 90-100%. 

\item What disease is not considered to be normally conveyed or transmitted by untreated wastewater? \\

a. Amoebic dysentery. \\
b. Hepatitis. \\
*c. Malaria. \\
d. Chlorea. 

\item What test is not run on the influent? \\

a. BOD. \\
*b. Fecal coliform. \\
c. Suspended solids. \\
d. pH. 

\item Which one of the following statements is TRUE regarding the standard BOD5 test? \\

a. Some NPDES permits specify that only CBOD is to be reported on a wastewater plant's final effluent. The letters CBOD refer to complete BOD (i.e the TOTAL BOD). \\
b. Phenylarsineoxide (PAD) is typically added to destroy nitrifying bacteria when running this test. \\
c. Dechlorinated secondary effluents need not be seeded when set up for the BODs test. \\
d. Nitrate ions interfere with this test \\
*e. The BOD measured includes the nitrogenous BOD (nBOD).results due to the oxygen demand exerted by certain bacteria as they oxidize ammonia to nitrate ions. 

\item A composite sample will provide a(an) \\

a. Even color. \\
b. High pH. \\
c. Low solids sample. \\
*d. Representative sample. 

\item Flow proportionate composite samples are collected because: \\

a. The waste characteristics are continually changing \\
b. The flow is continually changing \\
*c. The flow and waste characteristics are continually changing \\
d. This requires less time than grab samples \\
e. All of the above 

\item Over a four-year period, the totalizing hour meter on an instrument air compressor had the following readings at the end of each year: 1st year - 9,763; 2nd year - 13,258; 3rd year - 20,071; and 4th year - 23,714 How many hours does the meter show the compressor ran during \\the third year? \\

a. 349.5 hours \\
b. 364.3 hours \\
*c. 681.3 hours \\
d. 830.2 hours 

\item Grab sample is always collected for which of the following test \\

a. BOD \\
b. TSS \\
*c. Coliform \\
d. COD \\
e. None of the above 

\item Samples collected over several hours during the day and combined are known as: \\

*a. Composite samples. \\
b. Grab samples. \\
c. Deep samples. \\
d. Periodic samples. 

\item Grab samples are considered to be representative of the \\

a. Average daily condition at the sample location \\
b. Average daily condition in the system \\
c. System conditions for the two hours before and after the sample was taken \\
*d. System condition at the time of the sample 

\item Characteristics that should be measured immediately after the sample is collected are: \\

a. Velocity and dissolved solids \\
*b. Temperature, pH and DO \\
c. TSS and BOD \\
d. Hardness and alkalinity 

\item A stilling well on the effluent side of a wastewater facility is: \\

a. a chlorine injection concentrator \\
b. an automatic sampler for coliform counts \\
*c. a structure containing the float for a flow measuring device \\
d. dry well side of a pump station \\
e. a none of the above 

\item The advantages of automatic sampling equipment are: \\

a. elimination of human error inherent in manual sampl ing \\
b. reduction of personnel requirements and cost \\
c. allows for more frequent sampling \\
d. collection of more representative samples \\
*e. all of the above 

\item In collecting a sample for a chlorine residual determination of the final effluent, the most suitable sampling point is: \\

a. at the site of chlorine injection \\
b. at the entrance to the chlorine contact chamber \\
c. at the midpoint of the chlorine contact chamber \\
d. at the exit side of the chlorinator \\
*e. at the point of effluent discharge 

\item The recommended minimum portion which should be collected for testing in sampling wastewater is: (Assume a grab sample) \\

a. 10 ml \\
b. 50 ml \\
c. 10 0 ml \\
d. 500 ml \\
*e. 1,000 ml 

\item A composite sample will give a(n) \\

a. Even color \\
b. High pH \\
c. Instantaneous sample \\
*d. Representative sample 

\item Chlorine residual may be determined using the reagent \\

*a. Diethyl-p-phenylenediamine (DPD) \\
b. Ethylendiamine tetraacetic acid (EDTA) \\
c. Polychlorinated biphenyls (PCB) \\
d. Sodium thiosulfate (Na2S203) 

\item BOD5 is the most common method to quantify the organic content in wastewater. Another method used is: \\

a. Chemical oxygen demand \\
b. Volatile solids analysis \\
c. Total organic carbon \\
*d. Any of the above 

\item The laboratory test results on domestic raw sewage were: COD = 320 mg/1 BOD = 475 mg/1. The best interpretation would be: \\

a. the raw wastewater had a high grease content \\
b. the raw wastewater was septic \\
*c. the laboratory results, as reported, were in error \\
d. the sample was held too long before analysis \\
e. the glass fiber filters used to run the test were contaminated 

\item TKN is the measure of: \\

a. Ammonia/Ammonium+Inorganic Nitrogen \\
*b. Ammonia/Ammonium+Organic Nitrogen \\
c. Nitrate+Nitrite+Organic Nitrogen \\
d. Organic Nitrogen+Inorganic Nitrogen 

\item Volatile solids concentration of sludge with 6% solids containing 70% volatile matter is: \\

*a. 42,000mg/l \\
b. 70,000mg/l \\
c. 42mg/l \\
d. 700mg/l 

\item Wastewater solids can be categorized as: \\

a. Suspended+Fixed \\
b. Dissolved+Volatile \\
*c. Volatile+Fixed \\
d. Volatile+Settleable 

\item Non-settleable solids are composed of: \\

a. Volatile +Dissolved \\
b. Dissolved+Settleable \\
c. Floatable + Suspended \\
*d. Collodial + Floatable 

\item Sludge with a specific gravity of 1.1 will weigh: \\

a. 1.1lbs/gal \\
b. 8.34lbs/gal \\
*c. 9.17lbs/gal \\
d. 7.58lbs/gal 



\item  A composite sample will provide a(an) \\

a. Even color. \\
b. High pH. \\
c. Low solids sample. \\
*d. Representative sample. 

\item  Bacteria which cause diseases in man are generally called: \\

a. Mesophilic \\
b. Facultative \\
*c. Pathogenic \\
d. Coliforms 

\item  How should the pH electrode be stored when not in use: \\

a. In a strong acid solution \\
b. In a strong caustic solution \\
c. In a safe place in a drawer \\
*d. In distilled water \\
e. In a detergent 

\item  In the normal Winkler test: \\

a. A snow white precipitate forms in direct proportion to the nitrate concentration \\
b. A brownish flocculant precipitate is evidence that D.O. is absent \\
c. An endpoint is reached when a dark blue color changes to black \\
d. The muffle furnace must be in excess of 500 deg. C before incubation \\
*e. A snow white precipitate forms if DO is absent 

\item  The typical range of suspended solids in domestic influent wastewater is: \\

*a. 100-300 mg/L \\
b. 400-600 mg/L \\
c. 700-900 mg/L \\
d. 1000-12000 mg/L 

\item  Which of the flowing statement(s) is/are true with regards to BOD\\
1) BOD test results are suitable for quickly establishing process efficiencies\\
2) BOD value is always greater than the COD value of the same wastewater sample\\
3) BOD is expressed in mg/ L or in ppm\\
4) BOD is the measure of organic strength\\
5) BOD stands for biological oxygen demand\\

a. 1) and 2) \\
b. 1), 2) \& 4) \\
c. 1), 4) \& 5) \\
*d. 3) \& 4) \\
e. 3), 4) and 5) 

\item  {Nderline{hspace{1cm}}} matter in wastewater, is normally composed of grit, sand and silt. \\

a. Colloidal \\
*b. Inorganic \\
c. Organic \\
d. Volatile 

\item  Pathogens \\

*a. Are bacteria or virus that cause disease. \\
b. Are bacteria which do not occur in water. \\
c. Can obtain their food supply without help. \\
d. Are not harmful to man. 

\item  Regarding the total coliform test which one of the following statements is TRUE? \\

a. If less than 2 coliforms per 100 mL are found in a secondary effluent, we can be assured that we have destroyed all viruses and coliforms. \\
b. Total coliforms are monitored in wastewater because they are generally considered to be disease causing organisms. \\
c. Sodium thiosulfate should be added drop-wise after collection of a coliform sample to destroy residual chlorine. \\
*d. The multiple tube fermentation method for measuring total coliforms is only a statistical estimate of the coliform organism concentration. 

\item  Results from the Multiple-Tube Fermentation Technique for members of the Total Coliform Group are expressed as \\

a. DPD. \\
b. MF. \\
c. MGD. \\
*d. MPN. 

\item  The advantage in the use of coliform organisms as an indicator lies in the following fact: \\

a. They are found everywhere and they grow in common bacterial media \\
b. They are found everywhere and are readily killed by chlorine \\
*c. They are predominant bacteria associated with intestinal discharges and grow on\\
nutrient agar forming characteristic colonies 

\item  The purpose of adding sodium thiosulfate to a microbiological sample bottle is to \\

a. Extend the allowable holding time from 6 to 30 hours. \\
b. React with nitrates that interfere with the MPN test. \\
*c. Remove any chlorine residual present. \\
d. To ensure sterilization of sample bottle. 

\item  The term MPN is used in reference to: \\

a. The mass of phosphorus and nitrogen per unit of carbon \\
*b. The number of coliforms per unit volume of sample that is most likely to have caused the observed results in a multiple tube test \\
c. The number of fecal stetococci per unit volume of sample that is most likely to have caused the observed results in the multiple tube test \\
d. The result of membrane filter test \\
e. The standard plate count result 

\item Describe total coliform \& fecal coliform bacteria.  What is the difference between the two?  Describe the test procedures for both.


Response:\\
\begin{enumerate}[label=\alph*]
\item \textit{Describe total coliform \& fecal coliform bacteria.}
\begin{itemize}
\item Coliform bacteria are a broad group of bacteria found in soil, water and other environments.
\item Fecal coliform are coliforms which originate in the intestines of warm-blooded animals
\end{itemize}
\item \textit{What is the difference between the two? }\\
While total coliforms are found widely in different environments, fecal coliforms are typically found in the intestines of warm blooded animals and are indicators of fecal contamination.
\item \textit{Describe the test procedures for both.}\\
The Multiple Tube Fermentation method to estimate the quantity of both these microorganisms includes the following steps:
\begin{itemize}
\item Conducting a presumptive test by inoculating the sample in a set of 15 tubes containing Lauryl Tryptose Broth each containing an inverted Durham tube followed by incubation and observation of a positive result for each tube indicated by turbidity and presence of gas bubble.
\item Conducting a confirmed test by innoculating each of the positive samples from the presumptive test into a tube containing BGB broth - for Total Coliform and EC Broth - for Fecal Coliform
\item Conducting a completed test by streaking and incubating an agar plate with positives from the confirmed test followed by innoculating and incubating an agar slant and nutrient broth with colonies from the agar plate.


\end{itemize}
\end{enumerate}
\item What is ORP
\vspace{0.3cm}
Response:\\
\vspace{0.3cm}
Oxidation is a chemical reaction in which electrons are gained by oxidizing agent and lost by the substance being oxidized.  Electrons cause bonds to be broken in the organic matter thus destroying it.\\
\vspace{0.5cm}
Oxidation potential is the direct measure in millivolts of demand for chlorine; more chlorine increases oxidants potential more organic matter lowers oxidation potential.\\
Using ORP probe reduces amount of chlorine used for disinfection.  Chlorine by-products are toxic cause problems in effluent toxicity test.\\
\vspace{0.5cm}
ORP (oxidation red\vspace{0.5cm}uction potential) probes have been found to be effective in precisely controlling chlorine disinfection as well as de-chlorination.   An ORP probe measures directly the oxidation potential and compares (his measured potential to a set point (typically 540 mV).  A 4-20 milliamp controller can be used to adjust the chlorine feed to use precisely the amount of chlorine needed for disinfection. 


\item  When chlorine is added to a wastewater effluent it is said to act as an oxidizing agent.  Define oxidation.  Explain how oxidation might be effective in disinfection of a wastewater effluent.\\
\vspace{0.3cm}
Response:\\
\vspace{0.3cm}
Oxidation is a chemical reaction in which electrons are gained by the oxidizing agent and lost by the substance being oxidized.\\
These lost electrons cause bonds to be broken in the organic matter (bacteria) and thus destroy it.
\\
\item Define oxidation potential.   Explain how (his potential changes in relationship to the addition of chlorine or the increase in organic material or bacteria in the wastewater.  
\\
\vspace{0.3cm}
Response:\\
\vspace{0.3cm}Oxidation potential is a direct measure, in millivolts, of the potential for electron transfer to the oxidizing agent (chlorine).  \\
More chlorine increases oxidation potential.  \\
More organic material or bacteria lowers the oxidation potential.\\

\item One WWTP found that it was better able to meet its effluent toxicity limit when using an ORP probe.  Identify and briefly explain one possible reason for this.\\
\vspace{0.3cm}
Response:\\
\vspace{0.3cm}
Using an ORP probe reduces significantly the amount of chlorine being used for disinfection.  \\
Chlorine by-products are often toxic and thus cause problems in the effluent toxicity test (bioassay test).  \\
Reducing the chlorine used may make it easier to meet the toxicity limit.\\


\item An ORP probe must be place several minutes downstream of the point where chlorine is added in a chlorine contact tank.  Why?  Identify the data and conversion factor(s) necessary to calculate where an ORP probe is to be placed.  Assume that at peak dry flow (he probe is to be placed 5 minutes downstream of the point where chlorine is added.  Show the steps necessary to do this calculation.\\
\vspace{0.3cm}
Response:\\
\vspace{0.3cm}
Reactions with ammonia-nitrogen are not instantaneous.  Thus, sometime between the point of addition of chlorine and the site of the ORP probe to allow these reactions to be complete.
Data needed to calculate position of the probe: Peak Flow (MGD), the conversion factor = MGD = 1.547 cu. ft. / sec., the width of the chlorine contact chamber, the depth of the water in the chlorine contact chamber (so that the cross-sectional area of the flow can be calculated), 60 sec = 1 minute.\\

1.  Find the velocity of flow (ft/sec): Q at peak x 1.547 cu. ft./sec divided by cross-sectional area in ft2.
2.  Convert velocity (ft/sec) to ft/min = ft/sec x 60 sec/min = ft/min
3.  Multiply ft/min x 5 min to find ft.









\end{enumerate}





