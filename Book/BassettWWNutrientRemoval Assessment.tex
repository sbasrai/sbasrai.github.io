\chapterimage{QuizCover} % Chapter heading image

\chapter{Nutrient Removal Assessment}
% \textbf{Multiple Choice}

\section*{Nutrient Removal Assessment}
\begin{enumerate}
\item Nitrification is conversion of {\underline{\hspace{1cm}}} to {\underline{\hspace{1cm}}}

\item Nitrosomonas are {\underline{\hspace{1cm}}} (type) bacteria \\

Correct Answer(s):\\
a. Autotrophic\\

\item Denitrification requires {\underline{\hspace{1cm}}} condition \\

Correct Answer(s):\\
a. anoxic

\item Presence of adequate quantity of {\underline{\hspace{1cm}}} is critical for phosphorous removal \\

Correct Answer(s):\\
a. vfa

\item In nitrification, {\underline{\hspace{1cm}}} parts of oxygen required per part of ammonium nitrogen \\

Correct Answer(s):\\
a. 4.57 

\item In domestic water {\underline{\hspace{1cm}}} % of nitrogen is present as ammonia/ammonium \\

Correct Answer(s):\\
a. 40 

\item Nitrification is conversion of {\underline{\hspace{1cm}}} to {\underline{\hspace{1cm}}}

\item In domestic water {\underline{\hspace{1cm}}} \% of nitrogen is present as ammonia/ammonium \\

Correct Answer(s):\\
a. 40 

\item The approximate pH for good nitrification is closest to \\

a. 6.5 \\
b. 7.0 \\
*c. 8.5 \\
d. 10.0 

\item Which of the following biological processes can produce alkalinity? \\

a. Carbonaceous BOD removal \\
*b. Denitrification \\
c. Nitrification \\
d. Phosphorus removal by chemical addition with ferrous chloride 

\item In a typical biological nitrification system, an oxygen demand of {\underline{\hspace{1cm}}} parts oxygen per part of ammonia oxidized is exerted in the nitrification process. \\

a. 2.37 \\
b. 5.47 \\
c. 27 \\
*d. 4.57 

\item The biological nitrification process is carried out by bacteria that convert ammonia nitrogen to nitrate nitrogen. the two (2) specific groups of bacteria that perform this conversion are: \\

*a. Nitrosomonas and nitrobacter \\
b. Flagellates and swimming ciliates \\
c. Flagellates and crawling ciliates \\
d. Crawling ciliates and swimming ciliates 

\item Biological denitrification requires {\underline{\hspace{1cm}}} conditions. \\

a. Aerobic \\
*b. Anoxic \\
c. Anaerobic \\
d. Facultative 

\item Denitrification is an activated sludge plant involves: \\

a. Oxidation of ammonia to nitrate. \\
b. High concentrations of D.O. in the mixed liquor as it settle in the secondary clarifier. \\
c. Biological oxidation of nitrate to nitric oxide. \\
*d. Biological reduction of nitrate to nitrogen gas. \\
e. Poor compaction of mixed liquor as it settles in the secondary clarifier. 

\item In a nitrifying activated sludge plant, it is important to maintain adequate D.O. in the mixed liquor as it settles in the final clarifier to prevent: \\

a. Death of organisms in the floc that “eat” organic materials. \\
*b. Denitrification. \\
c. Bulking. \\
d. Growth of filamentous organisms. \\
e. Death of filamentous organisms. 

\item Nitrification in activated sludge does not involve: \\

a. The oxidation of ammonia to nitrite. \\
*b. Removal of oxygen from nitrate to form nitrogen gas. \\
c. The biological oxidation of nitrite to nitrate. \\
d. Higher concentration of D.O. in the aeration basin. \\
e. Control of alkalinity to stabilize pH. 

\item Limits on the concentration total NH3 - N are frequently placed on secondary effluents discharged into sensitive receiving waters. An important consideration in the setting of these limits by the RWQCB would be: \\

a. The concentration of algal cells in the receiving water. \\
*b. The pH \& temperature of the receiving water. \\
c. The salinity of the receiving water. \\
d. The population of nitrosomonas bacteria in the final effluent. \\
e. The concentration of copper ions in the final effluent because these ions form a non-toxic complex with the ammonia nitrogen. 

\item Many NPDES permits limit the amount of ammonia nitrogen that can be discharged into a stream because: \\

a. Ammonia exerts an oxygen demand in the stream \\
b. Ammonia can be toxic to aquatic life \\
c. Ammonia reacts with chlorine which can interfere with disinfection \\
*d. Any of these 

\item Possible techniques for controlling filamentous organisms in an activated sludge process include: \\

*a. Dosage of return sludge with a disinfectant such as chlorine or hypochlorite \\
b. Lower DO levels in aeration bans so filamentous organisms cannot breathe or respire \\
c. Lower F/M level to starve filamentous organisms \\
d. Stop wasting to allow activated sludge bugs to gain control 

\item Nitrogen and phosphorous are biologically significant in waste water because: \\

a. Nitrogen is reduced to pure nitrogen gas when phosphorous is present as a catalyst and this leads to nitrogen super saturation and rapid fish kills. \\
b. Nitrogen in the form of nitrates will reduce sodium thiosulfate and thus cause low results in the D.O. test while phosphates will oxidize the iodine to iodide and cause high results. \\
*c. Nitrogen and phosphorous are essential nutrients that can enable aquatic plant growth to become troublesome. \\
d. Nitrogen pentaphosphate is toxic to the organisms that metabolize. 

\item What are the bacteria that remove ammonia in the activated sludge process \\

a. Filamentous bacteria \\
*b. Nitrosomonas \\
c. Stalked ciliates \\
d. E. Coli 

\item The loading on a nitrification system consists of ammonia plus organic nitrogen and is typically referred to as total nitrogen. \\

a. True \\
*b. False 

\item Chlorine is being applied at a constant dose rate of 24 mg/L to a partially nitrified activated sludge effluent having a pH of 6.8 and a temperature of 67 deg. F. Ammonia nitrogen is found to range from 2 to 3 mg/L in this effluent. Disinfection in this effluent might be difficult because:\\

a. A temperature of 70 deg. F or higher is necessary in order to achieve effective disinfection. \\
b. Chloramines are present most of the time. \\
c. The chlorine dose rate is too low. \\
d. The pH of this effluent will limit the effectiveness of free chlorine. \\
*e. The ratio of chlorine to ammonia nitrogen may make it difficult at times to maintain adequate chlorine residual. 


\item During breakpoint chlorination:\\ 

a. A ratio of 5 parts of chlorine to 1 part of ammonia nitrogen is adequate to reach the breakpoint. \\
b. Ammonia nitrogen is reduced to nitrogen dioxide. \\
*c. As the breakpoint is approached, additional chlorine causes the chlorine residual to decrease. \\
d. Significant concentrations of dichloramine remain after the breakpoint is passed.


\item In the breakpoint chlorination method of ammonia nitrogen removal, parts of chlorine are required for each part of ammonia removed. \\

a. 3.0 \\
b. 6.7 \\
*c. 7.6 \\
d. 2.0 

\newpage
\item  The correct statement regarding the effect of nitrogen in chlorine disinfection is: \\

a. The reaction of ammonia with chlorine increases pH. \\
b. The reaction of ammonia with chlorine decreases pH. \\
c. The reaction of nitrite with free chlorine produces chloride. \\
*d. The reaction of ammonia with free chlorine produces chloramines. \\
e. The reaction of nitrate with free chlorine produces nitrogen gas. 

\item You are the superintendent of a 100,000 gpd conventional activated sludge plant which discharges into a shallow bay.  Your NPDES permit currently sets a discharge limit of 5 mg/L for total ammonia-nitrogen in your effluent. However, the Regional Water Quality Control Board at the request of the Department of Fish and Game will soon revise your plant’s discharge requirements and have even lower total ammonia limits so that certain species of fish may be re-introduced into the bay. The proposed maximum total ammonia-nitrogen concentration limit for  November through March 15th - 2.0 mg/L and 1.0	mg/L for the period March 16th through October 31st Answer the following questions:


\begin{enumerate}[1.]
\item Limits on the concentration total NH$_3$ - N are frequently placed on secondary effluents discharged into sensitive receiving waters.  An important consideration in the setting of these limits by the RWQCB would be:
\begin{enumerate}[a]
\item The concentration of algal cells in the receiving water.
\item The pH \& temperature of the receiving water.
\item The salinity of the receiving water.
\item The population of nitrosomonas bacteria in the final effluent.
\item The concentration of copper ions in the final effluent because these ions form a non-toxic complex with the ammonia nitrogen.
\end{enumerate}



\item Many NPDES permits limit the amount of ammonia nitrogen that can be discharged into a stream because:
\begin{enumerate}[a]
\item Ammonia exerts an oxygen demand in the stream
\item Ammonia can be toxic to aquatic life
\item Ammonia reacts with chlorine which can interfere with disinfection
\item Any of these
\end{enumerate}
\item Impacts of new ammonia-nitrogen limits on plant operations would include:
\begin{enumerate}[a]
\item Process change to remove ammonia further: nitrification – denitrification
\item Increased MCRT, higher activated sludge power demands and Lower sludge yields
\item Potential impacts to disinfection strategy related to breakpoint chlorination
\item Options [a] and [b]
\item Options [a], [b] and [c]
\end{enumerate}

\end{enumerate}

\newpage
 
\item You are the operations supervisor of a wastewater treatment plant that consists of the following unit processes: primary clarifiers, activated sludge aeration tanks, secondary clarifiers, a DAF thickener used to thicken WAS, and anaerobic digesters. Anaerobic digester gas is used, when it is available in sufficient amounts, to fuel engines that drive blowers supplying air to the activated sludge aeration tanks. Waste heat from these engines is sufficient to heat the digesters. When a sufficient amount of gas is not available, then electric motors are used to drive blowers. The regional water quality control board will soon impose a limit of 1.0 mg/L on the discharge of ammonia-nitrogen. In order to meet this limit, your plant will be modified to achieve complete nitrification.\\
Answer the following questions:

\begin{enumerate}[1.]
\item Demand Charge is measured in:
\begin{enumerate}[a).]
\item \$/kWh
\item \$/Hp
\item \$/kW
\item \$/day
\end{enumerate}
\item Most significant impact on energy consumption related to changes in mode of operation in order to meet the new permit requirement will be due to:
\begin{enumerate}[a).]
\item Increased digester gas production
\item Additional sampling and testing requirements
\item Associated with meeting higher F:M requirements for nitrification
\item Higher aeration air requirements
\end{enumerate}
\item Which of the following is true for Demand Charges:
\begin{enumerate}[a).]
\item Applies to residential customers also.
\item Based upon how much electricity used and the rate at which it is consumed
\item It is calculated based on 15-minute interval data
\item Options a) and c)
\item Options b) and c)
\item Options a), b) and c)
\end{enumerate}
\item Digester gas production is expected to:
\begin{enumerate}[a).]
\item Increase.
\item Decrease
\item Remain the same
\end{enumerate}
\item More power will need to be purchased because of:
\begin{enumerate}[a).]
\item Increased power demands related to denitrification
\item Meet power demand associated with higher F:M ratio requirements
\item Additional sludge pumping to digesters
\item Additional RAS pumping requirement
\end{enumerate}
\end{enumerate}
\newpage
\item Part A:  You are the chief plant operator at a 3 MGD conventional activated sludge plant that discharges into a nearby creek. Currently you operate this plant at an MCRT of between 4 to 5 days. Your plant’s maximum daily total coliform limit is 23 MPN/100 ml. Chlorine disinfection to achieve a total chlorine residual of 5 mg/L followed by dechlorination with sulfur dioxide had been used to meet your effluent total cofifom, limit. Recently, at the request of the state department of fish and game, the RWQCB revised your plant’s NPDES pemiit in order to make the creek more habitable for certain species of fish.  Your plant will now be required to meet a final effluent total ammonia-nitrogen (N-NH$_3$) limit of 2 mg/L during the winter months (November 1- March 1) and 1 mg/L during the rest of the year. Your plant’s total coliform limit was not revised.\\

Answer the following questions:
\begin{enumerate}
\item Define  total ammonia-nitrogen?
\item Why are such low limits necessary? Why are there different limits in winter and summer?
\item Identify and briefly discuss three significant impacts that these new limits will have on plant operations
\end{enumerate}

Part B:  Prior to this revision of your NPDES permit, your plant rarely had a problem meeting its total colifom, limit. Within days of meeting the revised ammonia-nitrogen limit, you’ve experienced frequent violations of your total coliform limit. Plant records show that maintaining a total chlorine residual of 5 mg/L has been difficult even though more chlorine has been used. Likewise, chlorine use has increased significantly. You’ve also noticed that at times the effluent in the chlorine contact chamber is "crystal clear" almost like a swimming pool. Plant records show no significant changes in plant flow or influent characteristics (i.e.  BOD, TSS, pH, etc.).  \begin{enumerate}
\item How do these observations help explain why you are having total colifom, violations?
\item Identify and briefly discuss one step you might take to prevent these coliform violations.
\end{enumerate}

Response:\\
Part A:\\
\begin{enumerate}[label=\alph*]
\item \textit{Define total ammonia-nitrogen}
\begin{itemize}
\item Total ammonia nitrogen is the total amount of nitrogen in the forms of NH$_3$ and NH$_4^+$in the wastewater.
\end{itemize}
\item \textit{Why are such low limits necessary? Why are there different limits in winter and summer?}
\begin{itemize}
\item Presence of nitrogen in wastewater effluent will promote plant \& algae growth causing eutrophication (oxygen depletion) of the water body in which the effluent is discharge impacting aquatic life
\item Additionally, ammonia-nitrogen is toxic to aquatic life
\item Toxicity of ammonia is affected by temperature, fish breeding times and susceptibility of juvenile fishes to ammonia– thus the different limits during winter and summer
\end{itemize}
\item \textit{Identify and briefly discuss three significant impacts that these new limits will have on plant operations}
\begin{itemize}
\item Process changes to remove ammonia further through implementation of nitrification – denitrification as part of the activated sludge process
\item Potential impact to the disinfection process required to meet the coliform limit - breakpoint chlorination\\
Associated impacts include:  
\begin{itemize}
\item Increase MCRT – increased RAS pumping
\item Lower sludge yields – less biosolids
\item Additional oxygen requirements – more energy consumption – higher power costs
\item Potential need for alkalinity and cBOD supplements for facilitating nitrification/denitrification
\item Less digester gas production
\end{itemize}
\end{itemize}	
\end{enumerate}
Part B:\\
\begin{enumerate}[label=\alph*]
\item \textit{How do these observations help explain why you are having total colifom, violations?}
\begin{itemize}
\item Potential impact of nitrite accumulation on chlorine disinfection
\item 5 ppm of chlorine used up for each part of nitrite present
\item Chlorine being consumed by nitrite and is not available for disinfection
\end{itemize}
\item \textit{Identify and briefly discuss one step you might take to prevent these coliform violations.}
\begin{itemize}
\item Measure nitrite levels at the effluent end of the activated sludge reactors
\item Increase DO for the nitrification step to facilitate the oxidation of nitrite to nitrate 
\end{itemize}
\end{enumerate}

\newpage
\item 
Define nitrification.  Name four factors that effect nitrification and discuss ranges, requirements and what lab tests are used.\\
Response:\\

Response:\\
\begin{enumerate}[label=\alph*]
\item \textit{Define nitrification}
\begin{itemize}
\item In wastewater treatment, nitrification is used for the removal of ammonia nitrogen - the predominant form of nitrogen in wastewater.  Nitrification involves a two step biological process where in the first step ammonia NH$_3$ is biologically oxidized by ammonia oxidizing bacteria such as nitrosomonas to nitrite (NO$_2$), followed by the oxidation of NO$_2$ to nitrate (NO$_3$) by ntirite oxidizing bacteria such as nitrobacter.
\end{itemize}
\item \textit{Name four factors that effect nitrification and discuss ranges, requirements and what lab tests are used.}
\begin{enumerate}
\item Dissolved oxygen

\noindent 4.5 parts of O$_2$ are needed for every part of NH$_4^{\enspace +}N$ (nBOD) to be degraded. In order for nitrification to occur, dissolved oxygen levels of 1.0 to 4.0 mg/L are usually maintained in the aeration tanks.  Maximum nitrification occurs at DO levels of about 3.0 mg/L.  Dissolved oxygen (DO)is measured using a DO probe.  

\item Alkalinity and pH

\noindent Presence of adequate alkalinity in the mixed liquor is critical as the nitrification process consumes alkalinity.  Alkalinity provides a buffer for pH change.  If the alkalinity is reduced beyond certain levels, further formation of acidic metabolic byproducts of bacterial activity may lead the pH to decline inhibiting bacterial activity. \textbf{7.14 parts of alkalinity are required for each part of ammonia removed.}    Nitrification rates are rapidly depressed as the pH is reduced below 7.0. pH levels of 7.5 to 8.5 are considered optimal. An alkalinity of 60 mg/L in the secondary treatment reactor is generally required to ensure adequate buffering.\\

Alkalinity is measured in the laboratory by titrating the sample with an acid until a specific pH is reached.  Alkalinity is reported as mg/l of calcium carbonate.

 
\item MCRT, F/M, or Sludge Age

\noindent For nitrification to occur the activated sludge treatment process needs to be operated at higher MCRT/sludge age (low F:M) as the reproductive rates of nitrfiers is low.  An MCRT of greater than 8 days is typically considered essential for nitrification to occur.


\item Wastewater temperature

\noindent Nitrification is inhibited at lower wastewater temperatures in wastewater treatment plants. To achieve the same level of nitrification, longer detention time may be needed in the winter versus the summer months since the activity drops significantly.  Lower temperature effects on Nitrification may be partially mitigated by increasing MLVSS and MCRT.  The optimal temperature range for nitrification is between 60$^{\circ}$  to 95$^{\circ}$  degrees F. Below 40$^{\circ}$  nitrification will probably not occur.

\item Inhibition to nitrification by toxic compounds

\noindent Many compounds can be toxic to nitrifiers- Nitrosomonas and Nitrobacter.  These include unionized ammonia, heavy metals, solvents and cyanide.  
\end{enumerate}
\end{enumerate}

\newpage
\item What is meant by un-ionized ammonia and total ammonia nitrogen?  Why is it important to set a low limit on un-ionized ammonia-nitrogen?



Response:\\
\begin{enumerate}[label=\alph*]
\item \textit{What is meant by un-ionized ammonia and total ammonia nitrogen?}\\
Ammonia is the predominant form of nitrogen in wastewater.  Ammonia (NH$_3$) can exist as ammonia itself which is the unionized form or it could change to its ionized form - ammonium (NH$_4^+$) by absorbing a proton (H$^+$).   Total ammonia nitrogen is the sum of the concentrations of the unionized ammonia and the ammonium ions present.  These two forms of nitrogen can rapidly change from one to the other depending on pH and temperature.
\item \textit{Why is it important to set a low limit on un-ionized ammonia-nitrogen?}\\
\begin{enumerate}
\item Nitrogen in any of its forms, if present in wastewater effluent discharge, promote growth of plant and algal matter in the receiving waters causing destruction of the normal aquatic life mainly due to oxygen depletion - eutrophication.  Ammonia is sought by nitrifying bacteria and is converted to nitrate at the expense of the oxygen present in the water body.\\

\item Additionally, ammonia specially in its unionized form is particularly toxic to aquatic life
\end{enumerate}
\end{enumerate}
\pagebreak
\item Define and explain importance of:
\begin{enumerate}
\item Breakpoint chlorination
\item Chlorine demand
\item Chlorine residual
\item Chloromines
\item Rotometer
\end{enumerate}
Response:\\
\begin{enumerate}[label=\alph*]
\item \textit{Breakpoint chlorination}\\
When disinfecting with chlorine, as chlorine is added to wastewater, the presence of inorganic and organic substances including ammonia in the wastewater, will exert a demand for chlorine as chlorine is a strong oxidizing agent.  This consumption of chlorine does not allow the chlorine to be present as free chlorine - the strongest form of chlorine disinfectant.  Breakpoint chlorination is the point where the demand for chlorine has been fully satisfied and any further addition of chlorine will show a proportional increase in free chlorine residual.


\item \textit{Chlorine demand}\\
The amount of chlorine used up as part of the reaction of chlorine with the inorganic and organic substances present in wastewater is referred to as the chlorine demand.

\item \textit{Chlorine residual}\\
Chlorine residual is the sum of free chlorine and combined chlorine and it represents the amount of chlorine available for disinfection.

\item \textit{Chloramines}\\
Chloramines which include monochloramine, dichloramine and trichloramine are products of the reaction of chlorine with ammonia.

\item \textit{Rotometer}\\
Rotometer is a flow measurement device most commonly used for measuring the flow of chlorine gas for disinfection.
\end{enumerate}
\end{enumerate}