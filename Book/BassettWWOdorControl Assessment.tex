\chapterimage{QuizCover} % Chapter heading image

\chapter{Odor Control Assessment}
% \textbf{Multiple Choice}

\section*{Odor Control Assessment}


\begin{enumerate}
\item Three systems used for vapor phase odor control in wastewater treatment 

\item Explain how bioxide works to control H2S odor 

\item Why do we need to control H2S in wastewater/wastewater treatment 

\item Explain how bioxide (sodium nitrate) works in controlling odors 

\item In your own words how is a biofilter different from a carbon scrubber

\item Why is the control of hydrogen sulfide important in wastewater treatment 

\item Explain how bioxide works to control H2S odor 

\item Why do we need to control H2S in wastewater/wastewater treatment 

\item  What are the characteristics of hydrogen sulfide 

\item  Explain how bioxide (sodium nitrate) works in controlling odors 

\item  In your own words how is a biofilter different from a carbon scrubber

\item  Why is the control of hydrogen sulfide important in wastewater treatment 

\item  Three systems used for vapor phase odor control in wastewater treatment 

\item  Hydrogen sulfide removal from the foul air can be accomplished by scrubbing with an alkaline solution \\

*a. True \\
b. False \\


\item Odor control of hydrogen sulfide can be accomplished by the use of which of the following agents? \\

a. hydrogen peroxide \\
b. chlorine \\
c. ozone \\
*d. all of the above \\
e. none of the above 

\item Ferric chloride helps in odor control by:\\

a. Oxidizing the odor constituents\\
b. Destruction of microorganisms responsible for odors \\
*c. Precipitating hydrogen sulfide \\
d. Raising the pH of the wastewater \\

\item Use of caustic soda in odor scrubbers is used for controlling:\\

*a. Hydrogen sulfide\\
b. Ammonia \\
c. Fouling\\
d. Organic compounds\\

\item Caustic soda is used in odor scrubbers for controlling:\\

*a. Hydrogen sulfide\\
b. Ammonia \\
c. Fouling\\
d. Organic compounds\\


\item Hydrogen sulfide control in the collection systems by caustic soda dosing is accomplished by: \\

*a. pH control\\
b. Chemical reaction\\
c. Oxidation \\
d. Biological control\\


\item High sulfide concentrations (either gaseous or dissolved) often cause problems in a wastewater treatment plant’s influent structure. Pre-chlorination of a plant influent is routinely practiced to control sulfides and odors. At your plant the piping that supplies chlorine for pre-chlorination needs replacing. It is developing cracks and is corroded. Rather than replace this piping immediately, you are directed by your supervisor to identify alternate means of controlling these influent sulfides. 
Do the following:
\begin{enumerate}
\item Identify THREE alternative methods of controlling sulfides.
\item Briefly explain how each of these methods identified in (1) is able to control sulfides (e.g chlorine destroys sulfides by chemical oxidization of these sulfides. 
\item From the three alternative methods of controlling sulfides select the one you feel is the best substitute for chlorine gas and briefly state why you think it is best.
\end{enumerate}

Response:\\
\begin{enumerate}[label=\alph*]
\item \textit{Identify THREE alternative methods of controlling sulfides.}
\begin{itemize}
\item pH control – Caustic/magnesium hydroxide addition in the collection system.
\item Foul air treatment using chemical scrubbers – capture and treat the foul air
\item Chemical precipitation using iron salts
\end{itemize}
\item \textit{Briefly explain how each of these methods identified in (1) is able to control sulfides (e.g chlorine destroys sulfides by chemical oxidization of these sulfides.}
\begin{itemize}
\item pH control:  Keeps the H$_2$S in the liquid phase.  Caustic would also help in controlling/removing the slime layer which is responsible for odor/H$_2$S generation.
\item Foul air treatment using chemical or biological scrubbers – capture and treat the foul air – for chemical scrubbers - methods used could be using an oxidizing agent (peroxide or beach) or alkaline pH (using caustic) recirculation water.
\item Chemical precipitation using iron salts – iron salts chemically remove the hydrogen sulfide by forming iron sulfide precipitate.
\end{itemize}
\item \textit{From the three alternative methods of controlling sulfides select the one you feel is the best substitute for chlorine gas and briefly state why you think it is best.}
\begin{itemize}
\item Process changes to remove ammonia further through implementation of nitrification – denitrification as part of the activated sludge process
\item Iron sulfide – good control over anticipated sulfide levels in the plant
\end{itemize}	
\end{enumerate}

\item Why is H$_2$S control important?  Describe the methods for controlling odor in both, in the sewer collection systems and in the plant.\\
Response:\\
\begin{enumerate}[label=\alph*]
\item \textit{Why is H$_2$S control important?}
\begin{itemize}
\item safety
\item preventing public nuisance, and
\item  corrosion prevention.\\
\end{itemize}
\item \textit{Describe the methods for controlling odor in both, in the sewer collection systems and in the plant.}\\
\textbf{In sewer collection system:}
\begin{enumerate}
\item pH adjustment by dosing with alkaline chemicals such as caustic soda or magnesium hydroxide
\item By injecting oxygen or air to prevent the formation of H$_2$S
\item By dosing with a iron salt to precipitate the H$_2$S 
\item By adding sodium nitrate (Bioxide) to facilitate biodegradation of H$_2$S
\end{enumerate}
\textbf{In the plant:}\\
By capturing and treating the foul air using the following methods:
\begin{enumerate}
\item Packed tower chemical scrubber using one of the following:
\begin{itemize}
\item Oxidizing chemicals such as hydrogen peroxide or bleach
\item Chemicals like caustic soda which increase the pH of the recirculation water
\end{itemize}
\item Use of biological removal based systems such as:
\begin{itemize}
\item Biofilter
\item Biotrickling filter
\end{itemize}
\item Use of adsorbtive material such as activated carbon
\end{enumerate}
\end{enumerate}


\item What is UV radiation disinfection?  Discuss maintenance and operational issues.  Advantages and disadvantages.  What is the dosage measured in and what does the equation (MW-sec/cm2) mean?

\end{enumerate}








