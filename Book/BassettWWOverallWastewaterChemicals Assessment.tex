\chapterimage{QuizCover} % Chapter heading image

\chapter{Wastewater Chemicals Assessment}
% \textbf{Multiple Choice}

\section*{Wastewater Chemicals Assessment}


\begin{enumerate}
\item What is mannich polymer and what are its drawbacks.  

\item  Chemical requirements for the conditioning of sludge are normally based upon laboratory-scale "jar tests" which determine the volume of chemical solution required for floc formation. 

*a. True \\
b. False 

\item  Cationic polymers are high-molecular-weight organic compounds carrying a negative charge

a. True \\
*b. False

\item Anionic polymer is used for:

a. Thickening solids in a gravity thickener \\
b. Flocculating solids for dewatering \\
c. For odor control \\
*d. For enhancing solids and BOD removal in the primaries 


\item Alum is frequently used along with an anionic polymer when dewatering anaerobically digested sludge using a belt press. 

a. Cationic polymers are high molecular weight organic compounds carrying a negative charge. \\
b. A dry polymer is always a better choice for application in centrifuges than any liquid polymer solution. \\
*c. Because of its viscosity, a Mannich polymer may be difficult to pump. \\
d. All liquid polymer solutions are harmless and need not require the examination of their MSDS. 

\item Either alum; ferric chloride; or lime may be used to remove solids from a secondary effluent. Which one of, the following statements is TRUE regarding these chemicals: 

a. Typical dose rates for alum when it is applied for the removal of phosphorus from a secondary effluent are 1 to 10 mg/L. \\
b. Hydrated lime needs to be "slaked" prior to use. \\
*c. The safety precautions for handling liquid ferric chloride are the same as those for handling an acid. \\
d. All of these chemicals raise the pH of the wastewater to which they are applied. 

\item Ferric chloride helps in odor control by:

a. Oxidizing the odor constituents \\
b. Destruction of microorganisms responsible for odors \\
*c. Precipitating hydrogen sulfide \\
d. Raising the pH of the wastewater 


\item Flocculation is best accomplished by 

a. Decreasing alkalinity. \\
*b. Gentle agitation. \\
c. Increased sunlight. \\

\item  Sodium hydroxide, (caustic soda) when used in wastewater: \\
Is typically applied at 1 - 10 mg/l when used to precipitate phosphorus in primary sedimentation systems. 

a. Should be treated as an acid with regard to safe handling. \\
b. Should be immediately diluted to 10\% upon receiving. \\
*c. Raises the pH of the wastewater to which it is added. \\
d. Is added to filtered effluent to improve de-chlorination with sulfur dioxide. 

\item  Either alum; ferric chloride; or lime may be used to remove solids from a secondary effluent. Which one of, the following statements is TRUE regarding these chemicals: 

a. Typical dose rates for alum when it is applied for the removal of phosphorus from a secondary effluent are 1 to 10 mg/L. \\
b. Hydrated lime needs to be "slaked" prior to use. \\
*c. The safety precautions for handling liquid ferric chloride are the same as those for handling an acid. \\
d. All of these chemicals raise the pH of the wastewater to which they are applied. 

\item  Flocculation is best accomplished by 

a. Decreasing alkalinity. \\
*b. Gentle agitation. \\
c. Increased sunlight. \\
d. Rapid mixing 

\item  Flocculation is best accomplished by: 

a. Decreasing alkalinity \\
*b. Gentle agitation \\
c. Increased sunlight. 

\item  If a chemical costs S30 per ton, how much will it cost per year to treat a flow of 1.5 MGD if the average dose is 18 mg/L? 

a. \$803. \\
b. \$110. \\
*c. \$233. \\
d. \$506.

Identify the correct statement regarding polymers. 

\item  Sodium hydroxide, (caustic soda) when used in wastewater: \\
Is typically applied at 1 - 10 mg/l when used to precipitate phosphorus in primary sedimentation systems. 

a. Should be treated as an acid with regard to safe handling. \\
b. Should be immediately diluted to 10\% upon receiving. \\
*c. Raises the pH of the wastewater to which it is added. \\
d. Is added to filtered effluent to improve de-chlorination with sulfur dioxide. 

\item  Which one of the following statement is TRUE regarding polymers? 

a. Alum is frequently used along with an anionic polymer when dewatering anaerobically digested sludge using a belt press. \\
b. Cationic polymers are high-molecular-weight organic compound carrying a negative charge. \\
c. A dry polymer is always a better choice for application in centrifuges than any liquid polymer solution. \\
*d. Because of its high pH, Mannich polymers may cause scale formation. \\
e. All. liquid polymer solutions are harmless and need not require the examination of its MSDS sheet. 

\item  Which one of the following statement is TRUE regarding polymers? 

a. Cationic polymers are high molecular weight organic compound carrying a negative charge. \\
b. A dry polymer is always a better choice for application in centrifuges than any liquid polymer solution. \\
*c. Because of its high pH, Mannich polymers may cause scale formation. \\
d. All liquid polymer solutions are harmless and need not require the examination of its MSDS sheet. \\
e. Alum is frequently used along with an anionic polymer when dewatering anaerobically digested sludge using a belt press. 


\end{enumerate}