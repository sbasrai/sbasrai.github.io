\chapterimage{Chapter17.jpg} % Chapter heading image

\chapter{Wastewater Chemicals}

\section{Wastewater treatment chemicals - by use/category}\index{Wastewater treatment chemicals - by use/category}

\setlength{\arrayrulewidth}{0.1mm}
\setlength{\tabcolsep}{8 pt}
\renewcommand{\arraystretch}{1.3}
%{\rowcolors{2}{green!60!yellow!50}{green!30!yellow!40}
\begin{tabular}{ |p{5.5cm}|p{4.5cm}|p{5cm}|  }
\hline
% \multicolumn{3}{|c|}{\textbf{WASTEWATER TREATMENT CHEMICALS - BY USE/CATEGORY}} \\
% \hline
%\thead{A Head} & \thead{A Second \\ Head} & \thead{A Third \\ Head} \\
%\hline%

\hspace{0.5cm}USE/CATEGORY & \hspace{1.2 cm} PROCESS & \hspace{1.2 cm} CHEMICALS USED \\
\hline
pH Control/ \newline Alkalinity Supplement & Odor Control \newline Secondary Treatment \newline Digestion & Caustic soda \newline Magnesium hydroxide \newline Calcium oxide \newline Ammonia \newline Sodium carbonate \newline Muriatic acid\\
\hline
Oxidant & Odor Control \newline Disinfection & Chlorine \newline Sodium hypochlorite (NaOCl) \newline Calcium hypochlorite (HTH) \newline Hydrogen Peroxide\\
\hline
Advance Primary Treatment/ \newline Chemically Enhanced Primary Treatment (CEPT)/Phys-Chem & Primary Treatment & Ferric Chloride \newline Anionic Polymer\\
\hline
Filament Control & Secondary  & Bleach \newline Cationic Polymer\\
\hline
Phosphorous Removal  & Primary Treatment \newline Secondary Treatment & Iron Salts \newline Alum (Precipitant)\\
\hline
Nitrogen Removal  & Breakpoint Chlorination & Chlrorine \newline Sodium Hypochlorite\\
Dechlorination  & Disinfection & Sodium bisulfite  \newline Sulfur dioxide   \\
\hline
Flocculation/Solids Separation & Sludge Dewatering \newline Sludge Thickening & Cationic Polymer \\
\hline
Descaling & Odor Control Scrubber & Muriatic Acid \\
\hline
\end{tabular}

\newpage
\section{Wastewater treatment chemicals - by use/category}\index{Wastewater treatment chemicals - by use/category}
%{\rowcolors{2}{green!60!yellow!50}{green!30!yellow!40}
\begin{tabular}{ |p{4cm}|p{4.5cm}|p{6.5cm}|  }
\hline
% \multicolumn{3}{|c|}{\textbf{WASTEWATER TREATMENT CHEMICALS - BY PROCESS}} \\
% \hline
%\thead{A Head} & \thead{A Second \\ Head} & \thead{A Third \\ Head} \\
%\hline%

\hspace{1 cm}PROCESS & \hspace{1.2 cm} ACTION & \hspace{1.2 cm} CHEMICAL USED (ROLE) \\
\hline
Collections & Odor Control & Caustic Soda (pH control) \newline Magnesium Hydroxide (pH control) \newline Hydrogen Peroxide (Oxidant) \newline Sodium Nitrate (Biological Degradation)\newline Iron Salts (Precipitant)\\
\hline
Primary & CEPT & Ferric Chloride (Coagulant) \newline Anionic Polymer (Flocculant) \\
\hline
Secondary    &Filament Control \newline \textsf{} \newline WAS Thickening & Bleach \newline Cationic Polymer \newline Cationic Polymer (Flocculant)\\
\hline
Nutrient Removal & Phosphorous Removal \newline \textsl{} \newline
Alkalinity Supplementation & Iron Salts (Precipitant) \newline Alum (Precipitant)\newline Calcium Oxide \newline Ammonia \newline Sodium Carbonate \\
\hline
Tertiary Treatment & Disinfection  \newline 
Dechlorination & Chlorine/Bleach \newline Sodium Bisulfite  \newline Sulfur Dioxide   \\
\hline
Dewatering & Flocculation & Cationic Polymer \\
\hline
Plant Odor Control & Foul Air Scrubbing & Hydrogen Peroxide (Oxidant) \newline Bleach (Oxidant) \newline Caustic Soda (pH Control) \newline Muriatic Acid (pH Control \& Scrubber Descaling)\\
\hline
Anaerobic Digestion & Hydrogen Sulfide Control \newline Alkalinity Supplementation & Iron Salts (Precipitant) \newline Calcium Oxide \newline Ammonia \newline Sodium Carbonate \\
\hline
\end{tabular}


\subsection{Polymers in wastewater treatment}\index{Polymers in wastewater treatment} 
        	\begin{itemize}
        		\item Polymer use in wastewater treatment includes:
        			\begin{itemize}
        				\item For enhancing primary removal efficiencies
        				\item For sludge thickening - to increase the solids content of the sludge feed to the digester
        				\item For solids dewatering - to reduce the digested solids hauling cost and to make the final solids product more manageable
        				\item For filament control in activated sludge treatment
        			\end{itemize}
        	\end{itemize}
  

\vspace{0.6cm}
Both anionic and cationic polymers used in wastewater treatment are available in the following forms: 
\begin{enumerate}
\item Dry Polymers:  These are available in granular, flake or bead form and have an active polymer as high as 95\%.  Prior to use, the dry polymers have to be dissolved in water using specialized mixing units 
\item Emulsion Polymer:  This water soluble version consists of water droplets dispersed in oil.  They have 25\% to 50\% active polymer content and require a specialized system to disperse it in water prior to use.
\item Solution polymers:  These are water soluble polymers in water. These polymers are relatively easy to put into dilute solution.  However, the lower active polymer content increases the shipping cost of this type of polymer.
\item Cationic polymer is also available as a low cost solution type polymer - Mannich Polymer (mannich is a type of chemical reaction involving formalydehyde which is used for making this polymer).  However, it has certain drawbacks which include: 
\begin{enumerate}
\item Presence of formaldehyde which lends its offensive odor
\item Higher viscosity which imposes operational challenges related to its use, and
\item High pH which leads to formation of hardness deposits in the associated piping and equipment.
\end{enumerate}
\end{enumerate}

The polymers use is primarily a function of the process stream.  Each system is different and there are no hard and fast rules regarding which products will work and therefore jar tests and pilot tests are conducted as part of the product selection process.\\




\newpage
\section*{Chapter Assessment}
\begin{tcolorbox}[breakable, enhanced,
colframe=blue!25,
colback=blue!10,
coltitle=blue!20!black,  
title= Chapter Assessment]

\begin{enumerate}
\item Anionic polymer is used for:

a. Thickening solids in a gravity thickener \\
b. Flocculating solids for dewatering \\
c. For odor control \\
d. For enhancing solids and BOD removal in the primaries 


\item Alum is frequently used along with an anionic polymer when dewatering anaerobically digested sludge using a belt press. 

a. Cationic polymers are high molecular weight organic compounds carrying a negative charge. \\
b. A dry polymer is always a better choice for application in centrifuges than any liquid polymer solution. \\
c. Because of its viscosity, a Mannich polymer may be difficult to pump. \\
d. All liquid polymer solutions are harmless and need not require the examination of their MSDS. 

\item Either alum; ferric chloride; or lime may be used to remove solids from a secondary effluent. Which one of, the following statements is TRUE regarding these chemicals: 

a. Typical dose rates for alum when it is applied for the removal of phosphorus from a secondary effluent are 1 to 10 mg/L. \\
b. Hydrated lime needs to be "slaked" prior to use. \\
c. The safety precautions for handling liquid ferric chloride are the same as those for handling an acid. \\
d. All of these chemicals raise the pH of the wastewater to which they are applied. 

\item Ferric chloride helps in odor control by:

a. Oxidizing the odor constituents \\
b. Destruction of microorganisms responsible for odors \\
c. Precipitating hydrogen sulfide \\
d. Raising the pH of the wastewater 


\item Flocculation is best accomplished by 

a. Decreasing alkalinity. \\
b. Gentle agitation. \\
c. Increased sunlight. \\

\item  Sodium hydroxide, (caustic soda) when used in wastewater: \\
Is typically applied at 1 - 10 mg/l when used to precipitate phosphorus in primary sedimentation systems. 

a. Should be treated as an acid with regard to safe handling. \\
b. Should be immediately diluted to 10\% upon receiving. \\
c. Raises the pH of the wastewater to which it is added. \\
d. Is added to filtered effluent to improve de-chlorination with sulfur dioxide. 

\end{enumerate}
\end{tcolorbox}

