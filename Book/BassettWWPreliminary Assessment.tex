\chapterimage{QuizCover} % Chapter heading image

\chapter{Preliminary Treatment Assessment}
% \textbf{Multiple Choice}

\section*{Preliminary Treatment Assessment}


\begin{enumerate}

\item A weir can be also be used for measuring flows\\

*a. True \\
b. False 
\vspace{0.4cm}
\item The process of pre-aeration in no way influences the degree of settling in a primary clarifier\\
\vspace{0.4cm}
a. True \\
*b. False 
\vspace{0.4cm}
\item Septic sludge has a low pH\\

*a. True \\
b. False 
\vspace{0.4cm}
\item A Parshall flume measures the velocity of the influent flow\\

a. True \\
*b. False 
\vspace{0.4cm}
\item  A barminutor frequently operates automatically. \\

*a. True \\
b. False 

\vspace{0.4cm}
\item  A grit chamber with a faster flow velocity than recommended may allow appreciable organic matter to collect in the grit. \\

a. True \\
*b. False 

\vspace{0.4cm}
\item  A grit chamber with a slower flow velocity than recommended may allow appreciable organics to settle out with the grit. \\

*a. True \\
b. False 

\vspace{0.4cm}
\item  A Parshall Flume is a device used to divide the incoming flow for equal distribution to a plant having more than one primary clarifier. \\

a. True \\
*b. False 

\vspace{0.4cm}
\item  A Parshall flume measures the velocity of the influent flow \\

a. True \\
*b. False 

\vspace{0.4cm}
\item  A properly operated grit chamber will normally increase the solids loading to the primary clarifier \\

*a. True \\
b. False 

\vspace{0.4cm}
\item  A properly operating grit chamber should yield grit that is high in fixed solids (inorganic material) and low in volatile solids. \\

*a. True \\
b. False 

\vspace{0.4cm}
\item  At most treatment plants preliminary treatment is used to protect pumping equipment and to facilitate subsequent treatment processes. \\

*a. True \\
b. False 

\vspace{0.4cm}
\item  A Venturi meter measures the amount of electricity used and should be read when there is a high electrical demand. \\

a. True \\
*b. False 

\vspace{0.4cm}
\item  A weir can be used as a flow measuring device \\

*a. True \\
b. False 

\vspace{0.4cm}
\item  Hydrogen sulfide gas in a moist atmosphere can result in corrosion of concrete structures \\

*a. True \\
b. False 

\vspace{0.4cm}
\item  Inefficient grit removal would tend to cause a decrease in percent volatile solids in raw primary sludge \\

*a. True \\
b. False 

\vspace{0.4cm}
\item  pH value less than 7 indicates an alkaline or basic condition \\

a. True \\
*b. False 

\vspace{0.4cm}
\item  Poor grit removal would affect all of the following: pumps and other mechanical equipment, anaerobic digestion, percent volatile and percent total solids in the raw sludge. \\

*a. True \\
b. False 

\vspace{0.4cm}
\item  Pre-aeration improves settleability in primary clarifier \\

*a. True \\
b. False 

\vspace{0.4cm}
\item  Pre-aeration of raw wastewater may cause better solids separation and removals\\
in the primary clarifier. \\

*a. True \\
b. False 

\vspace{0.4cm}
\item  Presence of hydrogen sulfide cannot always be detected by its characteristic odor \\

*a. True \\
b. False 

\vspace{0.4cm}
\item  In a typical treatment facility, it is necessary to have flow meters on the influent and effluent to detect the loss in plant flow. \\

a. True \\
*b. False 

\vspace{0.4cm}
\item  A Venturi meter is a reliable device for measuring flows of either treated or untreated wastewater. \\

*a. True \\
b. False 

\vspace{0.4cm}
\item  A flow measurement device such as a constant differential meter, or the more common name, rotameter, is used for the measurement of liquids or gases. \\

*a. True \\
b. False 

\vspace{0.4cm}
\item  It is often necessary to pre-chlorinate wastewater to prevent odors. When this is done, it is not necessary to satisfy a chlorine demand or expect a chlorine residual. \\

*a. True \\
b. False 

\vspace{0.4cm}
\item  Inefficient grit removal would tend to cause an increase in the percent volatile solids in raw sludge. \\

a. True \\
*b. False 

\vspace{0.4cm}
\item  Grit is composed mostly of inorganic material and organic material that is not easily biodegradable \\

*a. True \\
b. False 

\vspace{0.4cm}
\item  A weir can be also be used for measuring flows \\

*a. True \\
b. False 

\vspace{0.4cm}
\item  {Nderline{hspace{1cm}}} is used for controlling the flow velocity in a horizontal grit channel \\

Correct Answer(s):\\
a. True\\
b. False 

\vspace{0.4cm}
\item  Septic sludge has a low pH \\

*a. True \\
b. False 

\vspace{0.4cm}
\item  Barminutors and comminutors are devices which cut up or shred material which is normally found in raw wastewater. \\

*a. True \\
b. False 

\vspace{0.4cm}
\item  Barmunitor and comminutor are devices used for cutting up or shredding material normally present in raw wastewater \\

*a. True \\
b. False 

\vspace{0.4cm}
\item  Coliform testing is typically conducted on a grab sample \\

a. True \\
*b. False 

\vspace{0.4cm}
\item  Comminutors cut up or shred large objects normally found In raw wastewater and remove them from the wastewater flow. \\

a. True \\
*b. False 

\vspace{0.4cm}
\item  Conductivity is a very useful test for assessing sea water intrusion into sewer lines \\

*a. True \\
b. False 

\vspace{0.4cm}
\item  Fresh wastewater is characterized by a blackish color, foul and unpleasant odors with floating materials and suspended solids. \\

a. True \\
*b. False 

\vspace{0.4cm}
\item  Hydrogen sulfide in addition to creating an odor nuisance can be an explosion hazard when mixed with air in certain concentrations. \\

*a. True \\
b. False 

\vspace{0.4cm}
\item  Inflow is storm water entering into the sewer system \\

*a. True \\
b. False 

\vspace{0.4cm}
\item  Percent efficiency of total solids or BOD removal is calculated using the following formula: (In-Out*100)/(In-(In*Out)) \\

a. True \\
*b. False 

\vspace{0.4cm}
\item  Poor grit removal would affect anaerobic digester operation \\

*a. True \\
b. False 

\vspace{0.4cm}
\item  Pre-chlorination is frequently used to disinfect raw wastewater. \\

a. True \\
*b. False\\
@Prechlorination is primarily for reducing septicity 

\vspace{0.4cm}
\item  Solids removed from preliminary treatment are typically treated in an anaerobic digester \\

a. True \\
*b. False 

\vspace{0.4cm}
\item  The function of a comminutor is to shred rags, paper, wood, and other large wastewater solids and remove the from the flow. \\

a. True \\
*b. False 

\vspace{0.4cm}
\item  The laboratory measurement of volatile solids is a fair approximation of the organic content of the wastewater. \\

*a. True \\
b. False 

\vspace{0.4cm}
\item  The size and nature of solids in the wastewater is of no significant concern to the wastewater treatment plant operator. \\

a. True \\
*b. False 

\vspace{0.4cm}
\item  The velocity of wastewater flowing through a long channel type of grit chamber may be controlled by a proportional weir. \\

*a. True \\
b. False 

\vspace{0.4cm}
\item  Total solids are made up of dissolved and suspended solids both of which contain organic and inorganic matter \\

*a. True \\
b. False 

\vspace{0.4cm}
\item  Typical domestic wastewater BOD content is about 2000mg/l \\

a. True \\
*b. False 

\vspace{0.4cm}
\item  Wastewater with a pH of 12.0 to 14.0 would be too "acidic" for biological treatment \\

a. True \\
*b. False


\vspace{0.4cm}
\item {\underline{\hspace{1cm}}} matter in wastewater, is normally composed of grit, sand and silt.\\

a. Colloidal \\
*b. Inorganic \\
c. Organic \\
d. Volatile 

\vspace{0.4cm}
\item {\underline{\hspace{1cm}}} is used for controlling the flow velocity in a horizontal grit channel\\

a. Magmeter\\
*b. Proportional weir \\
c. Parshall flume \\
d. V-notch weir 


\vspace{0.4cm}
\item A {\underline{\hspace{1cm}}} is used in the wastewater treatment plant to remove debris such as large rocks, branches, pieces of lumber, leaves, paper, tree roots, etc. from the influent. \\

a. Comminutor \\
*b. Bar Screen \\
c. Belt filter \\
d. Grit chamber 

\vspace{0.4cm}
\item A pH probe: \\

a. Can be used to measure ORP in chlorine disinfection. \\
b. Is often used to measure hydrogen production in wet wells. \\
c. Measures, in millivolts, the difference between oxidants like chlorine and reductants such as organic matter. \\
*d. Measures hydrogen ion activity in wastewater. \\
e. Sends a 4-20 mA signal directly to a chlorine controller. 

\vspace{0.4cm}
\item A sewer system designed to transport only wastewater from homes, industries, institutions and businesses is called: \\

a. Combined sewer \\
*b. Sanitary sewer \\
c. Service sewer \\
d. Building sewer 

\vspace{0.4cm}
\item Carryover of grit from the grit chamber may indicate the need to: \\

*a. Increase rate of settled grit removal from the grit chamber.. \\
b. Decrease the operational depth of the channel. \\
c. Increase the flow to the primary clarifier. \\
d. Increase the air input to an aerated grit chamber. 

\vspace{0.4cm}
\item Characteristics that should be measured immediately after the sample is collected are: \\

a. Velocity and dissolved solids \\
*b. Temperature, pH and DO \\
c. TSS and BOD \\
d. Hardness and alkalinity 

\vspace{0.4cm}
\item Flow proportionate composite samples are collected because: \\

a. The waste characteristics are continually changing \\
b. The flow is continually changing \\
*c. The flow and waste characteristics are continually changing \\
d. This requires less time than grab samples \\
e. All of the above 

\vspace{0.4cm}
\item Grab samples are considered to be representative of the \\

a. Average daily condition at the sample location \\
b. Average daily condition in the system \\
c. System conditions for the two hours before and after the sample was taken \\
*d. System condition at the time of the sample 

\vspace{0.4cm}
\item Grit is composed mostly of which of the following substances? \\

a. Grease \\
b. Colloidal solids \\
c. Rubber goods \\
*d. Inorganics \\
e. Plastics 

\vspace{0.4cm}
\item Organisms in wastewater that are not harmful to humans but are indicators of diseases are: \\

a. Pathogens \\
b. Viruses \\
*c. Coliform \\
d. Bacteria 

\vspace{0.4cm}
\item Which of the following pollutants would be removed to the extent in an efficiently operating grit chamber? \\

a. Egg shells \\
b. Seeds \\
c. Oils and grease \\
*d. Sand \\
e. Fixed solids 

\vspace{0.4cm}
\item Proportional weirs usually are located at: \\

a. Immediately after the barscreens \\
b. Primary clarifiers \\
c. Aerobic digester scum boxes \\
*d. Grit chambers \\
e. Inside the Parshall flume 

\vspace{0.4cm}
\item Which of the following process units is not usually considered to be a preliminary treatment unit? \\

a. Grit chamber \\
b. Bar screen \\
c. Comminutor \\
d. Bar rack \\
*e. A meniscus 

\vspace{0.4cm}
\item  Which of the following would be included in the pretreatment unit? \\

a. preaeration \\
b. grit removal \\
c. screening \\
d. comminutor \\
*e. all of the above 

\vspace{0.4cm}
\item  As grit accumulates in a recetangular channel or chamber, the velocity of the influent wastewater: \\

*a. increases \\
b. decreases \\
c. remains constant \\
d. behaves independent of inflow \\
e. none of the above 

\vspace{0.4cm}
\item  Grit usually contains some organic matter which decomposes and creates odors. To facilitate disposal without nuisance, the organic matter is removed by washing methods. Commonly used is: \\

a. aeration \\
b. elutriation \\
c. vacuum filtration \\
d. wet oxidation \\
*e. none of the above 

\vspace{0.4cm}
\item  The flow through rate for grit channels is usually: \\

a. 20 seconds to 1.0 minute \\
b. 2 feet per second \\
*c. 1 foot per second \\
d. 30 days, depending on temperature \\
e. none of the above 

\vspace{0.4cm}
\item  Given the following data, what is the most likely cause of the mechanically cleaned bar-screen problem?\\
DATA: Normal dry weather flow\\
Motor running but unit not operating\\
Drive chain excessively tight\\
Water differential across screen above 6 inches\\
Controls on automatic.\\
Pin sheared or automatic clutch tripped. \\

a. Bubble tube malfunction \\
b. Flow too high \\
*c. Rocks lodged in screen \\
d. Suction channel level too high 

\vspace{0.4cm}
\item  In which of the following types of wastewater would settling occur most rapidly? \\

*a. Cold wastewater \\
b. Old (stale) wastewater \\
c. Septic wastewater \\
d. Strong wastewater 

\vspace{0.4cm}
\item  Given the following data, whatis the most likely cause of\\
the grit separator problem?\\
DATA: ·Lower than normal flow of water and grit from apex\\
High separation chamber pressure \\

*a. Grit pump suction clogged. \\
b. Partial obstruction near apex \\
c. Stick lodged in separation chamber \\
d. Vortex finder worn 

\vspace{0.4cm}
\item  What is the most likely cause of an aerated grit chamber problem if the grit pump is in automatic mode and running, the suction valve is wide open, the pressure on discharge line is low and erratic, and less than normal water and grit are discharging from discharge line? \\

a. Discharge check valve partially plugged. \\
b. Grit classifier partially plugged. \\
*c. Grit pump suction line partially plugged. \\
d. Malfunctioning air supply to grit chamber causing pump to get air bound. 

\vspace{0.4cm}
\item  Usually, preliminary treatment includes removal of most of the: \\

a. Pathogenic bacteria. \\
b. Biodegradable organics. \\
c. Settleable solids. \\
d. Dissolved solids. \\
*e. Grit. 

\vspace{0.4cm}
\item  A spray nozzle on the mechanically cleaned screen has become plugged. To ensure your safety, prior to entering the screen housing to repair the nozzle, you should \\

a. Leave note on breaker panel of repair being made \\
b. Request assistance for repair • \\
*c. Turn off and lock motor control \\
d. Turn off local control switch 

\vspace{0.4cm}
\item  An aerobic treatment process is one that requires the presence of: \\

a. Ozone \\
b. organic oxygen \\
c. no oxygen \\
d. combined oxygen \\
*e. dissolved oxygen 

\vspace{0.4cm}
\item  Which of the following should not normally be a significant part of grit \\

a. Sand. \\
b. Rocks \\
c. Eggshells. \\
*d. Fecal Matter. 

\vspace{0.4cm}
\item  Samples collected over several hours during the day and combined are known as: \\

*a. Composite samples. \\
b. Grab samples. \\
c. Deep samples. \\
d. Periodic samples. 

\vspace{0.4cm}
\item  Which of the following is not a characteristic of hydrogen sulfide? \\

a. Foul odors \\
*b. Lighter than air \\
c. Toxic \\
d. Corrosiveness \\
e. Explosiveness 

\vspace{0.4cm}
\item  The following device is used to measure the flow of wastewater: \\

a. Comminutor. \\
b. Comparator. \\
*c. Parshall flume. \\
d. Sluice gate. 

\vspace{0.4cm}
\item A device called an Imhoff cone is commonly used to measure settleable solids in: \\

a. Percent \\
*b. mL/L \\
c. mg/L \\
d. ppm \\
e. SVI units 


\vspace{0.4cm}
\item  The proper operation of an aerated grit removal process will: \\

a. Cause material with a specific gravity of greater than 1.0 to settle. \\
b. Cause sand and other non-organics to settle and keep organic material in suspension. \\
c. Help to freshen stale or septic wastewater. \\
*d. b and c 

\vspace{0.4cm}
\item  Velocity of sewers is usually expressed as: \\

*a. Feet per second \\
b. Gallon per minute \\
c. MGD \\
d. mg/l \\
e. sq. ft 

\vspace{0.4cm}
\item  Which of the following statements is not true regarding a wastewater collection system: \\

*a. A sewer is designed to allow the waste to flow at a rate of approximately 1 ft/sec. \\
b. Grease can be serious problem in a collection system. \\
c. Inflow and infiltration are frequently problems in older collection systems. \\
d. High concentrations of hydrogen sulfide in a sewer can lead to corrosion of concrete. \\
e. Scouring can be a problem if wastewater is flowing too fast. 

\vspace{0.4cm}
\item In order for the grit to settle in a non-aerated grit chamber, the average velocity should be kept near:\\

*a. 1 ft/sec . \\
b. 2 ft/sec. \\
c. 5 ft/sec. \\
d. 1 ft/min. \\
e. 2 ft/min. 

\end{enumerate}