\chapterimage{QuizCover} % Chapter heading image

\chapter{Primary Treatment Assessment}
% \textbf{Multiple Choice}

\section*{Primary Treatment Assessment}

\begin{enumerate}
\item  An Imhoff cone is often used to measure the effectiveness of primary sedimentation. \\

a. True \\
*b. False \\


\item  An inadequate detention time in a primary clarifier would result in increased solids pumping to digesters \\

a. True \\
*b. False \\


\item  A well-operated primary clarifier, will remove between 90 and 95 percent of the influent settleable solids. \\

*a. True \\
b. False \\


\item  Baffles ahead of outlet weirs in primary tanks serve as a physical barrier to inhibit the loss of floating solids and grease. \\

*a. True \\
b. False \\


\item  Baffles upstream of outlet weirs in primary tanks serve as a physical barrier to inhibit the loss of floating solids and grease. \\

*a. True \\
b. False \\


\item  Detention time is the time required to fill a tank at a given flow. \\

*a. True \\
b. False \\


\item  Gas bubbles rising to the surface in a primary clarifier are normally an indication that the sludge in the unit is undergoing proper settling \\

a. True \\
*b. False \\


\item  Hydraulic loading to a clarifier is expressed as gallons/day. \\

a. True \\
*b. False \\


\item  It is common for primary clarifiers to remove 80-85\% of the BODand90-95\% of the TSS. \\

a. True \\
*b. False \\


\item  Operators may use sludge blanket levels in primary clarifiers to adjust sludge pumping rates. \\

*a. True \\
b. False \\


\item  The weir overflow rate of a primary clarifier is expressed as gal/day/ft2 (gallons per day per square feet) \\

a. True \\
*b. False \\


\item  Primary treatment is a process which allows substances that will settle or float to be separated from the wastewater being treated. \\

*a. True \\
b. False \\


\item  A well operating primary clarifier will generally remove nearly all of the influent suspened solids. \\

a. True \\
*b. False \\


\item  Sludges from primary clarifiers are usual ly more dense than sludges from secondary clarifiers. \\

*a. True \\
b. False \\


\item  The generally accepted range of BOD removal in a well operating primary clarifier would be 40 - 60\% . \\

a. True \\
*b. False \\


\item  Wooden flights in a rectangular clarifier usually are equipped with wearing shoes. \\

*a. True \\
b. False \\


\item  The typical percentage of BOD which should be removed in primary clarifiers is in the range of about 90\%-99\% \\

a. True \\
*b. False \\


\item  If the collector mechanism on a circular clarifier stalls, it would be advisable to reverse the motor and back up the collector. \\

a. True \\
*b. False \\


\item  Raw sludge drawn from a primary sedimentation tank normally would contain 1,000 mg/l - 3,000 mg/1 of settleable solids. \\

a. True \\
*b. False \\


\item  When pumping primary clarifier scum to a holding tank, there should be no additional problems in cold weather. \\

a. True \\
*b. False \\


\item  If a shear pin for a particular installation is constantly bending, it would be proper and advisable to replace it with a pin of greater shear strength. \\

a. True \\
*b. False \\


\item  Septic sludge generally has a low pH. \\

*a. True \\
b. False \\


\item  Short circuiting in a primary clarifier occurs mainly because of the effluent weirs being plugged or not level \\

*a. True \\
b. False \\


\item  Wear shoes are found in circular clarifiers \\

a. True \\
*b. False \\


\item  Hydraulic loading is the number of gallons flowing each day through one cubic foot volume of the clarifier \\

a. True \\
*b. False \\


\item  Clarifier detention time is the time required to fill the clarifier at a given flow rate \\

*a. True \\
b. False \\


\item  Main objective of primary treatment is to remove organics present in wastewater \\

a. True \\
*b. False \\


\item  Raw sludge pumped from a primary clarifier generally contains about 3\% to 6\% total solids. \\

*a. True \\
b. False \\


\item  The main function of launders in a rectangular sedimentation tank is to prevent scum and other floatables from leaving with the primary effluent. \\

*a. True \\
b. False \\


\item  The solids in primary sludge contain both volatile and fixed parts. \\

*a. True \\
b. False \\


\item  Uneven weirs (heights) may result In short circuiting in a primary clarifier \\

*a. True \\
b. False \\


\item  “V” -notch weirs attached to a primary effluent launder measure the flow rate. \\

*a. True \\
b. False \\


\item  A well-operated primary clarifier, will remove between 90 and 95 percent of the influent settleable solids. \\

*a. True \\
b. False \\


\item  A well operating primary clarifier will generally remove nearly all of the influent settleable solids. \\

*a. True \\
b. False \\


\item  In a well-designed, efficiently operated wastewater treatment facility, most of the colloidal and dissolved matter in wastewater will be removed in the primary clarifiers \\

a. True \\
*b. False \\


\item  In some primary clarifiers water or aIr sprays may be used to push scum toward the scum removal point. \\

*a. True \\
b. False \\


\item  The main purpose of primary treatment is to remove the materials that might damage downstream equipment and/or the pumps. \\

a. True \\
*b. False \\


\item  The volume of sludge to be removed from a primary settling tank maybe estimated by measuring the primary influent and effluent settleable solids and wastewater flow. \\

*a. True \\
b. False \\

\item  A device called an Imhoff cone is commonly used to measure settleable solids in: \\

a. Percent \\
*b. mL/L \\
c. mg/L \\
d. ppm \\
e. SVI units \\


\item  An efficient primary clarifier is expected to remove what percent of the influent settleable solids? \\

a. 2 to 6 \% \\
b. 20 to 40 \% \\
c. 40 to 50 \% \\
d. 60 to 75 \% \\
*e. 90 \% or better \\


\item  A primary clarifier does not have adequate detention time. Which of the following would result? \\

a. Decreased organic leading on secondary unit \\
b. Overloading of collector or flight drive motor \\
c. Increased solids pumped to digester \\
*d. Low BOD removal \\
e. Reduced suspended solids in aeration tank. \\


\item  If the sludge depth in a secondary sedimentation tank is too high, what will happen? \\

a. Decreased turbidity in effluent. \\
b. Return activated sludge will have lower oxygen demand \\
c. Settleable solids from aeration tank will increase \\
*d. Sludge may become septic \\


\item  Odors associated with septic sludge are usually caused by \\

a. A neutral pH \\
b. Inorganic matter \\
c. Short retention time in collection system \\
d. Active aerobic bacteria \\
*e. Active anaerobic bacteria \\


\item  The main objectives of primary sedimentation are to remove: \\

a. Finely divided particles and dissolved organics \\
b. TDS and colloidal solids \\
c. BOD, COD, and SS \\
*d. Settleable solids and floatables \\
e. Detritus and volatile organics \\


\item  The withdrawal of sludge from a primary clarifier should be slow in order to: \\

a. conserve electricity \\
*b. prevent pulling too much water with the sludge \\
c. keep the BOD stabilized \\
d. not disturb the bacteria in the clarifier \\
e. protect the pump \\


\item  The withdrawal of sludge from a clarifier should be slow in order to: \\

a. protect the pump. \\
b. conserve electricity. \\
*c. prevent the pulling of too much water with the sludge. \\
d. not disturb the bacteria in the digester. \\
e. avoid breaking the water seal in the digester. \\


\item  Which of the following terms refers to a hydraulic condition, typically indicated by billowing solids flowing over the effluent weir, where a portion of the flow through a clarifier experiences a much shorter detention time than the rest of the wastewater in the tank? \\

a. surging \\
*b. shortcircuiting \\
c. overload \\
d. dispersion \\


\item  When wastewater enters a rectangular clarifier, it is often evenly dispersed across the entire width of the tank by means of: \\

*a. a baffle. \\
b. a proportional weir. \\
c. a submerged launder. \\
d. a dome diffuser. \\
e. a hydraulic pump. \\


\item  Rectangular primary clarifiers have wooden or plastic flights, which are equipped with: \\

a. shear pins. \\
b. friction skirts. \\
*c. wear shoes. \\
d. grease nipples. \\
e. sludge hinges. \\


\item  The main function of an inlet baffle in a settling tank is to: \\

*a. Reduce velocity and disperse the flow \\
b. Increase velocity to prevent excessive settling near the inlet \\
c. Remove scum from the wastewater \\
d. Protect the scrapping mechanism from damage by excessive velocities. \\


\item  If short circuiting occurs in a clarifier, the operator should \\

*a. Identify the cause \\
b. Change fuses. \\
c. Increase the sludge drawoff \\
d. Restart the pump. \\


\item  The expected range of BOD removal in a well operated primary clarifier is: \\

a. 10 to 20\% \\
*b. 25 to 40\% \\
c. 40 to 60\% \\
d. 60 to 80\% \\


\item  A treatment plant process using sedimentation after screening and grit removal and before aerobic treatment is known as: \\

a. preliminary treatment. \\
*b. primary treatment. \\
c. secondary treatment. \\
d. RBC treatment. \\
e. advanced wastewater treatment. \\


\item  Sludge gasification in a primary clarifier may be the result of: \\

a. hydraulic overloading \\
b. low influent BOD concentrations \\
*c. infrequent pumping of sludge \\
d. too short of detention time \\


\item  The withdrawal of sludge from a primary clarifier should be slow in order to: \\

a. conserve electricity \\
*b. prevent pulling too much water with the sludge \\
c. keep the BOD stabilized \\
d. not disturb the bacteria in the clarifier \\
e. protect the pump \\


\item  Primary sedimentation will remove most of the: \\

a. BOD and suspended solids. \\
b. settleable solids and suspended solids. \\
c. grit and raw sludge. \\
*d. floatable and settleable solids. \\
e. suspended solids and pathogens. \\


\item  The time it takes for a unit volume of wastewater to pass entirely through a primary clarifier is called: \\

*a. detention time \\
b. hydraulic loading rate \\
c. overflow time \\
d. weir loading rate \\


\item  Primary treatment units are designed to remove settleable solids from the wastewater stream by: \\

a. biological treatment \\
b. chemical addition \\
c. biofiltration \\
*d. gravity sedimentation \\
e. comminuting devices \\


\item  Which one of the following factors would have least influence on the settleability of solids in a clarifier: \\

a. detention time \\
b. flow velocity \\
c. short-circuiting \\
d. temperature \\
*e. soluble BOD \\


\item  The least acceptable method of handling scum and skimmings from a primary clarifier is: \\

a. burning \\
b. burial \\
c. pumping to a digester \\
*d. discharging into the effluent \\
e. hauling to a sanitary landfill \\


\item  Given the following data, what is the most likely cause of the primary sedimentation tank problem?
DATA: Raw sludge pumps run 10 minutes in each hour
Raw sludge has 3\% total solids at start of pumping cycle, 2\% total solids at end.
No sludge accumulation on tank floor
Slight sludge accumulation in sludge hopper \\

a. Raw sludge pumping duration too short \\
*b. Raw sludge pumping duration too long \\
c. Raw sludge total solids too high \\
d. Sludge collectors running too long \\


\item  Raw sludge pumping cycles that are too short will \\

*a. Cause wastestream going to secondary treatment to turn dark. \\
b. Decrease return and waste sludge suspended solids. \\
c. Increase amount of grit in raw sludge. \\
d. Increase rags in raw sludge. \\


\item  Raw sludge should be removed from a primary settling tank at any plant \\

a. At least hourly. \\
b. Not more often than once a week. \\
*c. At least once a day. \\
d. Whenever sludge rises to the surface. \\

\newpage
\item You are the operations supervisor at a modern 27 MGD (average dry weather flow) conventional activated sludge wastewater treatment plant. During a recent plant expansion, four (4) new primary sedimentation tanks (160 ft.  long x 25 ft.  wide x 12 ft.  deep) were built.  These tanks have been tested and are ready for service. These new tanks are to be put on line while six old sedimentation tanks are to be taken out of service so that they may be inspected and repaired. The primary influent channel and other mechanical equipment (e.g. pumps, collector mechanisms, flights, etc.) will also be inspected and repaired. It is estimated that the total down time will be one month or more. Because of the design of the primary influent channel, all six of the old sedimentation tanks must be taken out of service at one time.

The table below summarizes the calculated surface loading rates and detention times with the original and when only four tanks are on line.\\


\begin{flalign*}
FLOW \enspace CONDITION && SURFACE \enspace LOADING && \enspace DETENTION \enspace TIME\\
\hline
Ave. \enspace Daily \enspace Flow && 1688 \enspace gpd/ft^2 && 1.3 \enspace hours\\
Peak \enspace Flow && 2531 \enspace gpd/ft^2 && 0.85 \enspace hours
\end{flalign*}


Note: Peak flow is 1.5 times the average daily flow.\\

Write a memo to your shift supervisors in which you identify and briefly discuss FIVE (5) significant concerns in regard to a plan of action (POA) for removal and inspection of the old sedimentation tanks. State all assumptions.\\
\pagebreak
Response:\\
\textbf{Assessing the clarifier surface loading rate for the new clarifiers:}\\

Surface area of new tanks$=160ft*25ft=4,000 ft^2/tank=16,000 ft^2$ total surface area \enspace - four new clarifiers 

\vspace{0.1cm}
Clarifier surface loading - average dry flow condition:\\
\vspace{0.2cm}
$\Big(\frac{gpd}{ft^2}\Big) =\frac{Clarifier \enspace influent 			\enspace flow (gpd)}{Clarifier \enspace surface \enspace area (ft^2)}\implies \frac{27*10^6gpd}{16,000ft^2}=1,688 \enspace \frac{gpd}{ft^2}$\\
\vspace{0.2cm}
\textbf{Assessing the clarifier detention time for the new clarifiers:}\\

Volume of new tanks$=160ft*25ft*12ft=48,000 ft^3/tank=288,000 ft^3$ total volume \enspace - four new clarifiers 

\vspace{0.2cm}
Clarifier detention time  - average dry flow condition:\\
\vspace{0.2cm}
$(hrs) =\frac{Clarifier \enspace volume (ft^3)}{flow (\frac{ft^3}{hr})}\implies \frac{288,000ft^3}{27*10^6\frac{gal}{day}*\frac{day}{24hrs}*\frac{ft^3}{7.48gal}}=1.91 \enspace hrs$\\
\vspace{0.3cm}
Both, detention time \& the surface loading rate of the new clarifiers is comparable to the existing ones - so no concern from that perspective.\\
\begin{enumerate}[label=\roman*]
\item Ensure the effluent weirs in the new clarifiers are level to ensure there is no short circuiting
\item Ensure that the new clarifier mechanical equipment and its associated control systems are tested to ensure reliable operations prior to the switch.
\item For the inspection/repair of the old tanks, ensure lockout-tagout and proper confined entry procedures are followed
\end{enumerate}








\newpage

\item Chemical enhanced primary treatment (CEPT) is used at a number of conventional activated sludge planes. CEPT, also called advanced primary treatment, typically involves the application of liquid ferric chloride and an anion polymer ahead of the primary sedimentation tank. Although the long-term addition of chemicals to a wastewater flow can be costly, the benefits of CEPT may outweigh its cost.\\

Answer the fol- lowing question about advanced primary treatment:
\begin{enumerate}
\item Identify two operational considerations in order optimize CEPT.
\item Identify and explain two ways in which CEPT could result in energy saving in a convention activated sludge plant that uses cogeneration
\item Identify two additional benefits to anaerobic digester operations that may result from the addition of ferric chloride. Briefly explain how ferric chloride is able to yield these benefits.
\end{enumerate}

Response:\\
\begin{enumerate}[label=\alph*]
\item \textit{Identify two operational considerations in order optimize CEPT.}
\begin{itemize}
\item Optimal ferric chloride and polymer dosage
\item Adequate mixing time and energy for ensuring proper coagulation by ferric chloride 
\item Ensure the polymer is gently folded into the coagulated wastewater just prior to entering the clarifier.
\end{itemize}
\item \textit{Identify and explain two ways in which CEPT could result in energy saving in a convention activated sludge plant that uses cogeneration.}
\begin{itemize}
\item By increasing BOD removal in the primary clarifier, the influent BOD loading to the secondary treatment is lower reducing the oxygen requirements thus resulting in energy savings.
\item The organic material in the sludge feed to the digester has a higher proportion of primary sludge which is more easily digestable thus producing more digester gas for cogeneration.
\end{itemize}
\item \textit{Identify two additional benefits to anaerobic digester operations that may result from the addition of ferric chloride. Briefly explain how ferric chloride is able to yield these benefits.}
\begin{itemize}
\item The residual ferric chloride in the primary sludge feed allows for the control of H$_2$S concentration in the digester gas
\item Ferric chloride also helps in minimizing struvite forming potential in the digester due to its ability to remove phosphate, one of the key elements of struvite.
\end{itemize}	
\end{enumerate}
\newpage

\item Recently your activated sludge plant has acquired a two 2-meter belt press to dewater anaerobically digester sludge.

\begin{enumerate}
\item List four (4) factors, other than belt speed, which generally affect the belt press performance.
\item The belt press is running. You observe what one of your operator’s calls washout. Define "washout." 
\item Identify two possible cause of washout.
\item It has been said that the "the ideal belt speed is the slowest the operator can maintain without causing washout." Explain why this is so.
\end{enumerate}
Response:\\
\begin{enumerate}[label=\alph*]
\item \textit{List four (4) factors, other than belt speed, which generally affect the belt press performance.}
\begin{itemize}
\item Primary sludge dewaters better than secondary sludge.  Digested sludge with a higher proportion of primary will produce drier (higher solids content) dewatered biosolids.
\item Optimal dosage and adequate mixing of polymer with the sludge. 
\item Cleanliness of the belts.
\item Adequate and even hydraulic pressure to ensure water is squeezed out from the sludge.
\end{itemize}
\item \textit{Identify two possible cause of washout.}
\begin{itemize}
\item Inadequate polymer dosing.
\item Hydraulic overloading.
\item Blinded gravity zone belt.
\item Slow belt speed
\end{itemize}
\item \textit{It has been said that the "the ideal belt speed is the slowest the operator can maintain without causing washout." Explain why this is so.}\\
As the belt speed is slowed, the water from the sludge will have more time to drain producing a drier cake.  However, due to the slow belt speed, solids will build up on the belt preventing the water from draining.  Washout occurs when the belt speed is slowed to a point when there is so much standing water that it flows over the sides of the belt.
	
\end{enumerate}


\end{enumerate}
