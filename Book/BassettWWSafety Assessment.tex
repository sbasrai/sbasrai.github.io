\chapterimage{QuizCover} % Chapter heading image

\chapter{Constituents Properties and Analysis Assessment}
% \textbf{Multiple Choice}

\section*{Constituents Properties and Analysis Assessment}

\begin{enumerate}

\item Hazardous conditions in a manhole, wetwell or other similar type of structure may not be always be detected by the presence of odor because:\\

a. some toxic gases have no detectable odor \\
b. the environment may be oxygen deficient \\
c. some gases may deaden the sense of smelll \\
d. not all explosive gases have detectable odor \\
*e. all of the above \\

\item An operator should not enter an enclosed structure if the percentage of oxygen in the air is less than:\\

*a. 19.5\% \\
b. 23.5\% \\
c. 27.2\% \\
d. 32.5\% \\

\item Digester gas containing 60% methane by volume will likely explode when exposed to a spark or flame.\\

a. True \\
*b. False \\

\item Prior to working in a drained anaerobic digester, confined space entry permits must be prepared.\\

*a. True \\
b. False \\

\item Hydrogen sulfide gives off an odor similar to \\

a. Ammonia. \\
b. Chlorine gas \\
*c. Rotten eggs \\
d. Decayed wood. \\

\item Which of the following is not a characteristic of hydrogen sulfide? \\

a. Foul odors \\
*b. Lighter than air \\
c. Toxic \\
d. Corrosiveness \\
e. Explosiveness \\

\item If you come upon a co-worker who is not breathing, you should immediately \\

a. Apply cold compresses to the worker's forehead \\
b. Check for bleeding \\
*c. Call for help \\
d. Start CPR \\

\item What is the first, immediate, action you should take if concentrated acid is spilled on the floor? \\

a. Pour sodium nitrate and wash with warm water \\
b. Run to a shower and wash yourself thoroughly \\
*c. Sound the alarm \\
d. Wash with water and neutralize with sodium bicarbonate (baking soda) \\

\item Key steps in a safety program would not include \\

a. Controlling work habits \\
*b. Injury records for the operator \\
c. Locating hazards \\
d. Medical insurance for all employees \\

\item Three waterborne diseases are \\

a. Mumps, measles, colds \\
b. Scarlet fever, pneumonia, hay fever \\
*c. Typhoid fever, dysentery, cholera. \\
d. TB, diptheria, chickenpox \\

\item Improper handling, storing or preparing solutions of chemicals can cause \\

a. Burns \\
b. Explosions \\
c. Loss of eye sight \\
*d. All of the above \\

\item What disease is not considered to be normally conveyed or transmitted by untreated wastewater? \\

a. Amoebic dysentery \\
b. Hepatitis \\
*c. Malaria. \\
d. Chlorea. \\

\item Pathogens \\

*a. Are bacteria or virus that cause disease \\
b. Are bacteria which do not occur in water \\
c. Can obtain their food supply without help \\
d. Are not harmful to man \\

\item Employee hazards include \\

a. Noxious or toxic gases or vapors \\
b. Oxygen deficiency \\
c. Physical injuries \\
*d. All of the above \\

\item What is the proper slope of a ladder?\\
*a. Every 4 feet up the ladder is 1 foot out from the wall.\\
b.  Every 5 feet up the ladder is 1 foot out from the wall.\\
c.  Every 6 feet up the ladder is 1 foot out from the wall.\\
d.  Every 7 feet up the ladder is 1 foot out from the wall.


\item What is the safe oxygen level for entering a confined space?
a.  14 to 16 ppm.\\
b.  17 to 19 ppm.\\
*c. 20 to 22 ppm.\\
d.  23 to 25 ppm.

\item Cluttered work areas can cause accidents. Keep work areas clean. When you are finished with tools, put them:\\
a.  On the table.\\
b.  Under the table.\\
c.  On your supervisor’s desk.\\
*d. In the tool cabinet.


\item What type of tools are recommended to perform maintenance on an anaerobic digester?\\
*a. Brass.\\
b.  Stainless steel.\\
c.  Carbon steel.\\
d.  None of the above.

\item Before entering a permit-required confined space, you must:\\
a.  Check the atmosphere with a calibrated gas detector.\\
b.  Make notification that personnel are entering the space.\\
c.  Lock out and tag out all equipment.\\
*d. All of the above.


\item When working on a chemical feed pump, what of the following is not required?\\
a.  Nitrile gloves.\\
b.  Safety glasses.\\
*c. Leather work gloves.\\
d.  Full face shield.

\item When making a sulfuric acid dilution, the appropriate method is:\\
a.  Add the water to the acid.\\
*b. Add the acid to the water.\\
c.  Add both at the same time.\\
d.  None of the above.


\item When aluminum sulfate mixes with water, a very $\rule{2cm}{0.15mm}$ combination occurs.\\
a.  Noxious.
*b. Slippery.
c.  Colorful.
d.  Tacky.


\item Operators working with any form of lime are exposed to a number of hazards. Goggles, approved respiratory protection, emergency eyewash and deluge shower are necessary safety precautions. What else may be kept on hand to help flush eyes in case of severe exposure?\\
a.  Inert absorbent materials.\\
b.  A mild solution of acetic acid.\\
*c. A mild solution of boric acid.\\
d.  None of the above.

\item A mixture of 85\%-95\% atmospheric air in combination of 5\%-15\% methane creates which of the following?\\
*a. An explosive condition\\
b. Struvite\\
c. Excess pressure\\
d. Increased BTU

\item The first step the maintenance staff should take in properly locking and tagging out a piece of equipment is to $\rule{2cm}{0.15mm}$.\\
*a. Alert the operator on duty.
b.  Turn the equipment off at the motor control center (MCC).\\
c.  Pull the switch on the electrical panel to “OFF.”\\
d.  Fill out the tags.\\


\item When working in an area with two or more floor coverings, be sure that they are always $\rule{2cm}{0.15mm}$.\\
a.  Overlapping one another.\\
*b. Secured together.\\
c.  Separated from one another.\\
d.  At the entrances and exits only.

\item When manually lifting any object, be sure to $\rule{2cm}{0.15mm}$.\\
a.  Hold it at arm’s length.\\
b.  Keep your back bent and hold it low.\\
*c. Keep it close to your body and use leg strength.\\
d.  Keep your knees locked and bend at the waist.


\item Oxygen deficiency becomes a concern when the oxygen level in a confined space is less than $\rule{2cm}{0.15mm}$.\\
*a. 19.5\%.\\
b.  22.5\%.\\
c.  25.5\%.\\
d.  28.5\%.

\item Which of the following provides safety information for potentially hazardous or toxic materials?\\
a.  EPA.\\
b.  OSHA.\\
c.  CFR.\\
*d. SDS.

\item The threshold limit value concentration for chlorine vapor is $\rule{2cm}{0.15mm}$.
a.  0.1 ppm.\\
b.  0.3 ppm.\\
*c. 0.5 ppm.\\
d.  1.0 ppm.

\item When working in confined spaces where flammable gases may be present, use only tools made of $\rule{2cm}{0.15mm}$.
a.  Stainless steel.\\
b.  Lead.\\
c.  Iron.\\
*d. Beryllium.

\item Presence of hydrogen sulfide cannot always be detected by its characteristic odor \\

*a. True \\
b. False \\

\item The quantities and dosing requirements for a particular wastewater chemical can be found in the SDS \\

a. True \\
*b. False \\

\item Hydrogen sulfide in addition to creating an odor nuisance can be an explosion hazard when mixed with air in certain concentrations. \\

*a. True \\
b. False \\

\item The lower explosive limit for methane is 40\% \\

a. True \\
*b. False \\

\item What is the proper slope of a ladder?
\begin{enumerate}
\item Every 4 feet up the ladder is 1 foot out from the wall.
\item Every 5 feet up the ladder is 1 foot out from the wall.
c. Every 6 feet up the ladder is 1 foot out from the wall.
d. Every 7 feet up the ladder is 1 foot out from the wall.
\end{enumerate}

Hydrogen sulfide at 130 ppm smells most like:
\begin{enumerate}

\item Degreaser.
\item Rotten eggs.
\item Bleach.
\item Nothing.
\end{enumerate}

\item What is the safe oxygen level for entering a confined space?
\begin{enumerate}
\item 14 to 16 ppm.
\item 17 to 19 ppm.
\item 20 to 22 ppm.
\item 23 to 25 ppm.
\end{enumerate}

\item Cluttered work areas can cause accidents. Keep work areas clean. When you are finished with tools, put them:
\begin{enumerate}
\item On the table.
\item Under the table.
\item On your supervisor’s desk.
\item In the tool cabinet.
\end{enumerate}

\item What type of tools are recommended to perform maintenance on an anaerobic digester?
\begin{enumerate}
\item Brass.
\item Stainless steel.
\item Carbon steel.
\item None of the above.
\end{enumerate}

\item Before entering a permit-required confined space, you must:
\begin{enumerate}
\item Check the atmosphere with a calibrated gas detector.
\item Make notification that personnel are entering the space.
\item Lock out and tag out all equipment.
\item All of the above.
\end{enumerate}

\item When working on a chemical feed pump, what of the following is not required?
\begin{enumerate}
\item Nitrile gloves.
\item Safety glasses.
\item Leather work gloves.
\item Full face shield.
\end{enumerate}

\item When making a sulfuric acid dilution, the appropriate
method is:
\begin{enumerate}
\item Add the water to the acid.
\item Add the acid to the water.
\item Add both at the same time.
\item None of the above.
\end{enumerate}

\item When aluminum sulfate mixes with water, a very $\rule{2cm}{0.15mm}$ combination occurs.
\begin{enumerate}
\item Noxious.
\item Slippery.
\item Colorful.
\item Tacky.
\end{enumerate}

\item Operators working with any form of lime are exposed to a number of hazards. Goggles, approved respiratory protection, emergency eyewash and deluge shower are necessary safety precautions. What else may be kept on hand to help flush eyes in case of severe exposure?
\begin{enumerate}
\item Inert absorbent materials.
\item A mild solution of acetic acid.
\item A mild solution of boric acid.
\item None of the above.

\item What safety hazard is created by a polymer spill?
\begin{enumerate}
\item explosive condition 
\item fire
\item slippery surface 
\item toxic gas
\item skin burns
\end{enumerate}
\item When is it safe to enter a manhole?
\begin{enumerate}
\item after a hazardous condition has been identified
\item after ventilation equipment has been turned on
\item when wearing an SCBA and having a back-up standing by
\item when no hazardous condition exists
\end{enumerate}

\end{enumerate}








\end{enumerate}





