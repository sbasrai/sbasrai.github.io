\chapterimage{Week7Safety.png} 
\chapter{Safety}



\section{Wastewater Treatment Hazards}\index{Wastewater Treatment Hazards}
There are many hazards encountered in wastewater treatment operations.  The hazards and their respective mitigation measures are as follows:\\
\subsection{Hazardous Chemicals}\index{Hazardous Chemicals}
\begin{itemize}
\item Hazardous chemicals are used throughout wastewater treatment plants and in collection systems. 
\item To understand the dangers of these chemicals and to take adequate steps OSHA requires that the chemical manufacturer, distributor, or importer provide Safety Data Sheets (SDSs) (formerly MSDSs or Material Safety Data Sheets) for each hazardous chemical to downstream users to communicate information on hazards related to that particular chemical or product.
\item Employers must ensure that the SDSs are available and readily accessible to employees for all hazardous chemicals in their workplace.
\item The SDS includes information such as the properties of each chemical; the physical, health, and environmental health hazards; protective measures; and safety precautions for handling, storing, and transporting the chemical.\\
\end{itemize}


\section{Hazard Control Approaches}\index{Hazard Control Approaches}
\begin{itemize}
\item Treatment plant operators face a variety of workplace safety hazards ranging from physical injuries to hazardous material exposure.  
\item The three principal approaches for hazard control are:
\begin{enumerate}
\item Engineering Controls:  Engineering controls include incorporating safety elements during engineering design and include considerations related to selecting a less hazardous alternatives, establishing  physical barriers and elements related to ventilation.
\item Administrative Controls:  Administrative controls are used to improve safety within the workplace by putting in place policies and rules that reduce the occupational risk faced by workers via altering the way their work is performed.  These include: housekeeping, materials handling and transfer procedures, training, providing facilities to support personal hygiene practices and medical surveillance.
\item Personal Protective Equipment: Personal protective equipment (PPE)is equipment worn to minimize exposure to hazards that cause serious workplace injuries and illnesses.  PPE includes respiratory protection and protective clothing - hard hats, safety
glasses/goggles, work gloves, chemical resistant gloves and safety shoes.

\end{enumerate}
\item The Occupational Safety and Health Administration \colorbox{pink}{OSHA}, a. Federal Government agency is responsible for establishing and enforcing workplace safety and health regulations.
\end{itemize}


\section{Wastewater Treatment Hazards}\index{Water Treatment Hazards}
\subsection{Hazardous Gasses}\index{Hazardous Gasses}
\begin{itemize}
\item A summary of the properties and effects of hazardous gases found in wastewater operations is provided in the table below.
\item To safeguard against the potential impacts of these gases, employees are required to follow practices including donning appropriate Personal Protective Equipment (PPE) and utilizing respiratory protection\\
\end{itemize}
\begin{figure}[H]
\begin{center}

\includegraphics[scale=0.4]{SafetyHazardousGases}\\ 
\caption{Hazardous gasses in wastewater treatment operations}
\end{center}

\end{figure}
\newpage
\subsection{Hazardous Chemicals}\index{Hazardous Chemicals}
\begin{itemize}
\item Hazardous chemicals are used throughout wastewater operations.
\item To understand the dangers of these chemicals and to take adequate steps OSHA requires that the chemical manufacturer, distributor, or importer provide Safety Data Sheets (SDSs) (formerly MSDSs or Material Safety Data Sheets) for each hazardous chemical to downstream users to communicate information on hazards related to that particular chemical or product.
\item Employers must ensure that the SDSs are available and readily accessible to employees for all hazardous chemicals in their workplace.
\item The SDS includes information such as the properties of each chemical; the physical, health, and environmental health hazards; protective measures; and safety precautions for handling, storing, and transporting the chemical.\\
\end{itemize}


\subsubsection{Effects of Chemical Hazards}\index{Effects of Chemical Hazards}

Hazardous chemicals can cause:
\begin{itemize}
\item Headaches, rashes and burns
\item Respiratory problems
\item Lung and liver damage
\item Reproductive damage
\item Cancer
\item Death
\end{itemize} 


\subsubsection{Protection from chemicals}\index{Protection from chemicals}
Chemical manufacturers, distributors, or importers are required to provide Safety Data Sheets (SDSs) (formerly MSDSs or Material Safety Data Sheets) for each hazardous chemical to downstream users to communicate information on these hazards.
\begin{itemize}
\item Reading and understanding the associated Safety Data Sheets
\item Wearing appropriate Personal Protective Equipment
\item Implementing safe work practices by:
\begin{itemize}
\item \textbf{NEVER:} Eating, drinking or smoking when working with hazardous chemicals, and washing or storing PPE with personal clothing.
\item \textbf{ALWAYS:}Washing hands, arms and face with soap and water after use, ensuring integrity of PPE before and after use and also very importantly ensuring  readiness to deal with chemical exposure or spill.
\end{itemize}
\end{itemize}
\newpage
\hspace{0pt}
\vfill
\begin{center}
Safety Data Sheets Content - OSHA Guidance Document
\end{center}
\vfill
\hspace{0pt}
\pagebreak
\hspace{0pt}
\vfill
\begin{center}
\end{center}
\vfill
\hspace{0pt}
\pagebreak



\includepdf[pages=-]{oshasds.pdf}
\newpage
\hspace{0pt}
\vfill
\begin{center}
Sample SDS
\end{center}
\vfill
\hspace{0pt}
\pagebreak
\hspace{0pt}
\vfill
\begin{center}
\end{center}
\vfill
\hspace{0pt}
\pagebreak
\includepdf[pages=-]{anhydammoniasds.pdf}

\subsection{Falls}\index{Falls}
\begin{itemize}
\item Falls are one of the leading causes of injuries and deaths on the job.  Fall protection is a combination of methods and devices used to protect workers from falling off, onto, or through working levels.
\item Working on top of tanks, walkways roofs, and other elevated surfaces more than 4 feet or when working above open tanks or hazardous machinery requires the provision of fall protection.
\item Fall protection methods and devices are typically divided into two categories: those that prevent falls and those that arrest falls. 
\item Examples of fall protection methods and devices include rails, guards, guardrails, barriers, fall-arrest systems, safety nets, hole covers, and various work practices and procedures.
\end{itemize}
\begin{center}
\includegraphics[scale=0.8]{SafetyFallProtection1}\hspace{1cm} \includegraphics[scale=0.8]{SafetyFallProtection2}\\
\end{center}
\subsection{Noise}\index{Noise}
\begin{itemize}
\item Noise as a hazard is sound that is especially loud or impacting. 
\item A wastewater treatment plant has equipment that produces high noise levels both continuously and intermittently. 
\item As such, it is important to be aware of this hazard and to take preventive steps to reduce exposure to damaging noise levels by wearing effective hearing protection and to minimize the duration of the exposure to the noise.
\end{itemize}

\subsection{Electrical Hazards}\index{Electrical Hazards}
\begin{itemize}
\item Ordinary 120-V electricity can be fatal; most wastewater facility electrical systems operate at 120 to 4000 V or more.  
\item All voltages should be considered dangerous and potentially life threatening.  
\item Safe working rules and practices that should be followed when working on electrical systems
\item Before working on an electrical system, perform a job hazard analysis to determine any potential hazards and methods of abating those hazards
\end{itemize}

\subsection{Trenching \& Excavation Hazards}\index{Trenching \& Excavation Hazards}
\begin{itemize}
\item Trench collapses, or cave-ins, pose the greatest risk to workers’ lives.
\item Potential hazards related to trenching include: falls, falling loads, hazardous atmospheres, and incidents involving mobile equipment including vehicular traffic.
\item A good rule of thumb is to slope the sidewalls at least one and one half foot back for every one foot in depth on both sides of the trench unless the soils have been classified as to the type and the options provided in the OSHA Standard have been selected by a competent person.
\item For excavations greater than 4 feet, potential for hazardous atmospheres must be considered. If the environment has the potential for a hardous atmosphere, adequate precautions need to be taken to prevent worker exposure to those conditions.
\item Provision for access and egress for personnel as well as equipment is required. Ramps for access/egress must be designed by a competent person and be capable of handling the intended loads.
\item A stairway, ladder or ramp must be provided within 25 lateral feet of employees in trench excavations greater than 4 feet.  Ladders should extend a minimum distance of 3 feet past the edge they rest against but not more than 4 feet. 
\item Trenches 5 feet deep or greater require a protective system unless the excavation
is made entirely in stable rock. 
\item Trenches 20 feet deep or greater require that the protective system be designed
by a registered professional engineer.
\item Types of protective systems:
\begin{itemize}
\item \textbf{Benching} - excavating the sides of an excavation to form one or a series of horizontal levels or steps.
\item \textbf{Sloping} - involves cutting back the trench wall at an angle inclined away from the excavation.  A good rule of thumb is to slope the sides of the excavation to an angle not steeper
than 1½:1 (for every foot of depth, the trench must be excavated back 1\small{1/2} feet)
excavated back 1½ feet). A slope of this gradation is safe for any type of soil.
\begin{figure}[H]
\begin{center}
\includegraphics[scale=0.6]{TrenchSlope}
\caption{Trench Slope}
\end{center}
\end{figure}
\item Sloping and benching consists of the removal of the trench wall at a specific slope based on the type of soil being excavated.
\item \textbf{Sheeting} - wooden sheets or metal plates are placed against the side of the trench to hold back the walls. Uprights placed vertically along the face of the trench wall are used to support the sheeting. Stringers or wales are placed horizontally along the uprights in which trench braces are attached to prevent cave-in.
\item \textbf{Shoring} - requires installing aluminum hydraulic or other types of supports designed to support the walls of a trench.
\item \textbf{Shielding} -  uses a two-sided, braced box sometimes referred to as a drag shield or trench box, which is open at the top, bottom and ends.
\end{itemize}
\begin{figure}[H]
\begin{center}
\includegraphics[scale=0.6]{TrenchProtection}
\caption{Trench Protection Systems}
\end{center}
\end{figure}
\item OSHA standards require, before any worker entry, that employers have a competent person
inspect trenches daily and as conditions change to ensure elimination of excavation hazards. 
\item Key elements related to ensuring safety:
\begin{itemize}
\item Ensure that there's a safe way to enter and exit
\item Ensure trenches have cave-in protection
\item Test for atmospheric hazards such as low oxygen, hazardous fumes and toxic gases when > 4 feet deep.
\item Inspect trenches at the start of each shift and following a rainstorm or
other water intrusion.
\item Keep excavated soil (spoils) and other materials at least 2 feet from trench edges.
\item Never enter a trench unless it has been properly inspected by a competent person
\end{itemize}

\end{itemize}

\subsection{Rotating and Moving Equipment}\index{Rotating and Moving Equipment}

\begin{itemize}
\item All rotating and moving equipment should be guarded. 
\item The best method for preventing machinery-related injuries is through use of equipment guards enforced through engineering and administrative controls.   
\item The best way to prevent this type of injury is to install point-of-operation guards that prevent contact with nip points, pinch points, rotating parts, flying chips, and sparks.
\end{itemize}
\begin{center}
\includegraphics[scale=0.6]{SafetyMachineGuarding}\\
\end{center}

\subsection{Heat Stress}\index{Heat Stress}
\begin{itemize}
\item Heat stress falls into two categories: heat illness and heat stroke. 
\item Both are serious conditions and should not be taken lightly. 
\item Heat stress can result from: 
\begin{itemize}
\item High temperature and humidity, dehydration from low fluid consumption
\item Direct sun exposure (with no shade) or extreme heat, 
\item Limited air movement (no breeze or wind), 
\item Physical exertion, Use of bulky protective clothing and equipment, 
\item Poor physical condition or ongoing health problems, 
\item Some medications
\item Pregnancy
\end{itemize}
\end{itemize} 

\subsection{Material Handling Ergonomics}\index{Material Handling Ergonomics}
\begin{itemize}
\item Wastewater operators are potentially subject to risk of musculoskeletal injuries associated with handling heavy or unwieldy objects including tools and supplies as part of their daily work routine.
\item The risk and severity of these injuries can be mitigated through utilizing proper ergonomic techniques which include:
\begin{itemize}
\item Use mechanical means (e.g. hand trucks, pushcarts, etc.) when possible for heavier or awkward loads.
\item It is easier and safer to push than to pull.
\item Keep loads as close to the body as possible and do not twist while lifting, carrying, or setting down a load. Nose, shoulders, hips, and toes should all be facing the same direction.
\item Minimize reaching.
\item As a general rule, bend at the knees, not the hips.
\item Get help when needed. Do not lift or carry things you don’t feel comfortable with, no matter how light the load.
\item Plan ahead for all parts of the lift: lifting, carrying, and setting down.
\item Use personal protective equipment where needed, such as gloves with good grip and steel-toed boots where appropriate.
\item Implement rest breaks and job rotation for frequent and/or heavy lifting.
\end{itemize}
\end{itemize}



\section{Safety Practices}\index{Safety Practices}


\subsection{Lockout - Tagout (LOTO)}\index{Lockout - Tagout (LOTO)}

When conducting routine inspections, repairs and maintenance activities, requires meeting the mandates of \textbf{Occupational Safety  Hazard Administration(OSHAs) Lock-Out/Tag-Out (LOTO) program}\\
which is designed to prevent injury or fatalities.  It involves preventing an equipment from accidentally starting up and release of all stored energy.  Hazardous energy sources include: 
\begin{itemize}
\item Electrical 
\item Mechanical
\item Hydraulic
\item Pneumatic 
\item Chemical 
\item Thermal  
\item Other energy
\end{itemize}

The LOTO involves established and documented procedures specific to an equipment or machinery.  It typically comprises of:\\
\begin{itemize}
\item Notifying affected employees
\item Stopping and isolating the equipment
\item Releasing stored energy
\item Verification of the isolation and de-energization
\item Placing lock-out devices which use a positive means such as a lock, either key or combination type, to hold an energy isolating device in the safe position and prevent the energizing of a machine or equipment
\item Appropriately tagging the devices to indicate its non-operation and that it may not be operated until the tagout device is removed
\end{itemize}

\subsection{Personal Protective Equipment (PPE)}\index{Personal Protective Equipment (PPE)}
Employees depend on personal protective equipment to protect themselves from hazards and perform daily duties. PPE includes but is not limited to safety glasses, face shields, hard hats, gloves, foot protection, and durable and disposable chemical-protective clothing. Respirators and fall protection might also be required. However, respirators and fall protection fall under separate OSHA standards. \\

\subsection{Confined Space Entry}\index{Confined Space Entry}
OSHA defines a confined space as an area that:
\begin{itemize} 
\item is large enough and so configured that an employee's body can enter and perform assigned work
\item has limited or restricted means for entry or exit; and
\item is not designed for continuous employee occupancy.
\end{itemize}
Potentially dangerous conditions which can exist in confined spaces include: 
\begin{itemize}
\item Oxygen level: Some gasses are heavier than air and so will fill up a confined space, which forces oxygen out.  The oxygen concentration must not fall below 19.5\% at any time.  In plants where pure oxygen is used there is a potential hazard due to high the oxygen concentration.  Oxygen concentration greater than 23\% increases the risk of ignition and fire
\item Explosive conditions:  Many gasses are explosive when present in certain ratios with oxygen. These ratios are defined by the upper explosive limit(UEL) and the lower explosive limit (LEL).  The minimum concentration of a particular combustible gas or vapor necessary to support its combustion in air is defined as the Lower Explosive Limit (LEL) for that gas. Below this level, the mixture is too “lean” to burn. The maximum concentration of a gas
or vapor that will burn in air is defined as the Upper Explosive Limit (UEL). Above this level, the mixture is too “rich” to burn.  The range between the LEL and UEL is known as the flammable range for that gas or vapor.  
\item Toxic conditions:  This condition could potentially exist due to the presence of gasses such as carbon dioxide, chlorine and hydrogen sulfide.
\item Engulfment by liquids
\item Electrical hazards
\item Mechanical Hazards associated with equipment including mixers
\end{itemize}

A permit-required confined space is defined as a confined space that:
\begin{itemize} 
\item contains or has a potential to contain a hazardous atmosphere
\item contains a material that potentially could engulf an entrant
\item has an internal configuration that could trap or asphyxiate an entrant through inwardly converging walls or a floor that slopes downward and tapers to a smaller cross-section
\item contains any serious safety or health hazard
\end{itemize}


\newpage
\section*{Chapter Assessment}
\begin{tcolorbox}[breakable, enhanced,
colframe=blue!25,
colback=blue!10,
coltitle=blue!20!black,  
title= Chapter Assessment]

\begin{enumerate}
\item Presence of hydrogen sulfide cannot always be detected by its characteristic odor \\

a. True \\
b. False \\

\item The quantities and dosing requirements for a particular wastewater chemical can be found in the SDS \\

a. True \\
b. False \\

\item Hydrogen sulfide in addition to creating an odor nuisance can be an explosion hazard when mixed with air in certain concentrations. \\

a. True \\
b. False \\

\item The lower explosive limit for methane is 40\% \\

a. True \\
b. False \\

Hydrogen sulfide at 130 ppm smells most like:
\begin{enumerate}

\item Degreaser.
\item Rotten eggs.
\item Bleach.
\item Nothing.
\end{enumerate}

\item What is the safe oxygen level for entering a confined space?
\begin{enumerate}
\item 14 to 16 ppm.
\item 17 to 19 ppm.
\item 20 to 22 ppm.
\item 23 to 25 ppm.
\end{enumerate}

\item Cluttered work areas can cause accidents. Keep work areas clean. When you are finished with tools, put them:
\begin{enumerate}
\item On the table.
\item Under the table.
\item On your supervisor’s desk.
\item In the tool cabinet.
\end{enumerate}

\item What type of tools are recommended to perform maintenance on an anaerobic digester?
\begin{enumerate}
\item Brass.
\item Stainless steel.
\item Carbon steel.
\item None of the above.
\end{enumerate}

\item When is it safe to enter a manhole?
\begin{enumerate}
\item after a hazardous condition has been identified
\item after ventilation equipment has been turned on
\item when wearing an SCBA and having a back-up standing by
\item when no hazardous condition exists
\end{enumerate}



\end{enumerate}
\end{tcolorbox}