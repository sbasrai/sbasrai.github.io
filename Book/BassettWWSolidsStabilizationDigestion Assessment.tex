\chapterimage{QuizCover} % Chapter heading image

\chapter{Solids Stabilization Digestion Assessment}
% \textbf{Multiple Choice}

\section*{Solids Stabilization Digestion Assessment}

\begin{enumerate}

\item  A change in the-pH-of digesting sludge in anaerobic digester is the best early warning indicator of potential digester upset. \\

a. True \\
*b. False \\

\item  A good maintenance program should be established for all flame arresters to ensure they are all set at the recommended "pop-off' pressures. \\

*a. True \\
b. False \\

\item  A healthy anaerobic digester should have a carbon dioxide concentration of more than 40\%. \\

a. True \\
*b. False \\

\item  A high rate anaerobic digester is always heated and mixed. \\

*a. True
b.False \\

\item  Anaerobically digested sludge produces gas as a by-product. The gas produced is of little or no value. \\

a. True \\
*b. False \\

\item  Digester gas containing 60\% methane by volume will likely explode when exposed to a spark or flame. \\

a. True \\
*b. False \\

\item  Digester gas containing 60\% methane by volume will likely explode when exposed to a spark or flame \\

*a. True \\
b. False \\

\item  In an anaerobic digester, the [1] is destroyed producing [2].  The digester gas production is typically in the range of [3] cubic feet of digester gas per lb of volatile matter destroyed. \\

\item  Gas production in an anaerobic digester results from the destruction of fixed solids \\

a. True \\
*b. False \\

\item  A well operating digester will have a CO2 concentration of greater than 60\% \\

a. True \\
*b. False \\

\item  Volatile solids removal efficiency of a digester is most commonly monitored utilizing the BOD test\\

a. True \\
*b. False \\

\item  Volatile solids removal efficiency of a digester is most commonly monitored utilizing the BOD test\\

a. True \\
*b. False \\

\item  The range of volatile material in raw sludge may be as high as 60\% - 80\% while digested sludge normally will be below 20\%. \\

a. True \\
*b. False \\

\item  Foaming in digesters is often due to rapid acid digestion is caused by adding too much raw sludge during too short a time period. \\

*a. True \\
b. False \\

\item  Aerobic digestion produces methane gas that provides energy for other operations. \\

a. True \\
*b. False \\

\item  A good maintenance program should be established for all flame arresters to ensure they are all set at the recommended "pop-off" pressures. \\

a. True \\
*b. False \\

\item  A flame arrester in the gas line between a waste gas burner and an anaerobic digester contains plates which should periodically be inspected and cleaned. \\

*a. True \\
b. False \\

\item  The methane forming bacteria in an anaerobic digester reproduce more rapidly than the acid forming bacteria. \\

a. True \\
*b. False \\

\item  The main function of a secondary digester in normal operation is to provide a storage space for seed sludge. \\

a. True \\
*b. False \\

\item  Supernatant liquor is withdrawn from the lowest possible level in an anaerobic digester. \\

a. True \\
*b. False \\

\item  The pH of digested sludge in a healthy anaerobic digester will be near 7.0 \\

*a. True \\
b. False \\

\item  Volatile solids reduction is a measure of the effectiveness of an anaerobic digester. \\

*a. True \\
b. False \\

\item  A properly operated anaerobic, digester functions best within a pH range of 5.4 to 5.8. \\

a. True \\
*b. False \\

\item  In computing anaerobic digester loadings, it is necessary to take into account the solids lost in the supernatant system. \\

a. True \\
*b. False \\

\item  In conventional secondary wastewater treatment processes, aerobic decomposition of solids will occur. \\

*a. True \\
b. False \\

\item  Aerobic digesters are operated under the principle of extended aeration from the activated sludge process. \\

*a. True \\
b. False \\

\item  After pumping raw sludge to a digester, it is necessary to seal both ends of the line to prevent back siphonage. \\

a. True \\
*b. False \\

\item  Propeller mixers are located mainly on floating covers of anaerobic digesters rather than on fixed covers. \\

a. True \\
*b. False \\

\item  A gallon of water weighs 8.34 pounds.  A gallon of digested sludge, due to its high volatile content, will weigh much less. \\

a. True \\
*b. False \\

\item  The range of volatile material in raw sludge may be as high as 60\% - 80\% while digested sludge normally will be below 20\%. \\

a. True \\
*b. False \\

\item  Foaming in digesters is often due to rapid acid digestion is caused by adding too much raw sludge during too short a time period. \\

*a. True \\
b. False \\

\item  Aerobic digestion produces methane gas that provides energy for other operations. \\

a. True \\
*b. False \\

\item  A good maintenance program should be established for all flame arresters to ensure they are all set at the recommended "pop-off" pressures. \\

a. True \\
*b. False \\

\item  A flame arrester in the gas line between a waste gas burner and an anaerobic digester contains plates which should periodically be inspected and cleaned. \\

*a. True \\
b. False \\

\item  The methane forming bacteria in an anaerobic digester reproduce more rapidly than the acid forming bacteria. \\

a. True \\
*b. False \\

\item  The main function of a secondary digester in normal operation is to provide a storage space for seed sludge. \\

a. True \\
*b. False \\

\item  Supernatant liquor is withdrawn from the lowest possible level in an anaerobic digester. \\

a. True \\
*b. False \\

\item  The pH of digested sludge in a healthy anaerobic digester will be near 7.0 \\

*a. True \\
b. False \\

\item  Volatile solids reduction is a measure of the effectiveness of an anaerobic digester. \\

*a. True \\
b. False \\

\item  A properly operated anaerobic, digester functions best within a pH range of 5.4 to 5.8. \\

a. True \\
*b. False \\

\item  In computing anaerobic digester loadings, it is necessary to take into account the solids lost in the supernatant system. \\

a. True \\
*b. False \\

\item  In conventional secondary wastewater treatment processes, aerobic decomposition of solids will occur. \\

*a. True \\
b. False \\

\item  Aerobic digesters are operated under the principle of extended aeration from the activated sludge process. \\

*a. True \\
b. False \\

\item  After pumping raw sludge to a digester, it is necessary to seal both ends of the line to prevent back siphonage. \\

a. True \\
*b. False \\

\item  A gallon of water weighs 8.34 pounds.  A gallon of digested sludge, due to its high volatile content, will weigh much less. \\

a. True \\
*b. False \\

\item  Gas production in an anaerobic digester results from the destruction of fixed solids in the raw sludge. \\

a. True \\
*b. False \\

\item  "High rate" and Low rate" when used in referring to an anaerobic digester may refer to the rate of volatile solids loading. \\

*a. True \\
b. False \\

\item  In computing anaerobic digester loadings, it is necessary to take into account the solids lost in the supernatant system. \\

a. True \\
*b. False \\

\item  Increased concentrations of volatile acids and decreased alkalinity are the first measurable changes that take place when the process of anaerobic digestion is becoming upset. \\

*a. True \\
b. False \\

\item  A change in the-pH-of digesting sludge in anaerobic digester is the best early warning indicator of potential digester upset. \\

a. True \\
*b. False \\

\item  A good maintenance program should be established for all flame arrestors to ensure they are all set at the recommended "pop-off' pressures. \\

*a. True \\
b. False \\

\item  A healthy anaerobic digester should have a carbon dioxide concentration of more than 40\%. \\

a. True \\
*b. False \\

\item  A high rate anaerobic digester is always heated and mixed. \\

*a. True \\
b. False \\

\item  Anaerobically digested sludge produces gas as a by-product. The gas produced is of little or no value. \\

a. True \\
*b. False \\

\item  Digester gas containing 60\% methane by volume will likely explode when exposed to a spark or flame. \\

a. True \\
*b. False \\

\item  Volatile solids removal efficiency of a digester is most commonly monitored utilizing the BOD test\\

a. True \\
*b. False \\

\item  Gas production in an anaerobic digester results from the destruction of fixed solids in the raw sludge. \\

a. True \\
*b. False \\

\item  "High rate" and "Low rate" when used in referring to an anaerobic digester may refer to the rate of volatile solids loading. \\

*a. True \\
b. False \\

\item  In computing anaerobic digester loadings, it is necessary to take into account the solids lost in the supernatant system. \\

a. True \\
*b. False \\

\item  Increased concentrations of volatile acids and decreased alkalinity are the first measurable changes that take place when the process of anaerobic digestion is becoming upset. \\

*a. True \\
b. False \\

\item  In a two-stage anaerobic digestion system, it is in the secondary tank where most of the volatile solids destruction occurs \\

a. True \\
*b. False \\

\item  The lower explosive limit for methane is 40\% \\

a. True \\
*b. False \\

\item  Volatile solids removal efficiency of a digester is most commonly monitored utilizing the BOD test\\

a. True \\
*b. False \\

\item  Sludge digestion is a process by which principally the inorganic matter in sludge is gasified and/or converted to a more stable form by the biological process. \\

a. True \\
*b. False \\

\item  The determination of the pH on the anaerobic digester is one of the best early warning systems of digester upset. \\

a. True \\
*b. False \\

\item  The gas produced as part of the anaerobic digestion is of little or no value \\

a. True \\
*b. False \\

\item  The main function of a secondary digester in normal operation is to provide a storage space for seed sludge. \\

a. True \\
*b. False \\

\item  The pH of digested sludge in a healthy anaerobic digester will be near 7.0. \\

*a. True \\
b. False \\

\item  Volatile solids reduction is a measure of the effectiveness of an anaerobic digester. \\

*a. True \\
b. False \\

\item  Volatile solids removal efficiency of a digester is most commonly monitored utilizing the BOD test \\

a. True \\
*b. False \\

\item  Identify the incorrect statement regarding anaerobic digestion. \\

a. An anaerobic digester with pH of 7.05, an alkalinity of 2,900 mg/l and a volatile acid concentration of 250 mg/l is probably operating normally. \\
*b. Sodium bicarbonate may be used in place of lime to neutralize a sour anaerobic digester. \\
c. When adding lime to a sour anaerobic digester, it is important to add an excess of this chemical to act as a reservoir of alkalinity. \\
d. Ferrous sulfate may be added to an anaerobic digester to reduce hydrogen sulfide concentration where air quality is of concern. \\
e. Gas production from anaerobic digester may be expressed as cubic feet of gas produced per pound of volatile matter added per day. \\

\item  One action that may be taken to improve the health of an anaerobic digester that is going sour would be to: \\

a. Add small doses of lime daily to maintain the digester pH above 7.0. \\
*b. Add ferrous sulfate to reduce the concentration of hydrogen sulfide in the digester. \\
c. Add seed sludge from a healthy primary digester. \\
d. Pump raw sludge to this digester more frequently so that this sludge has a higher pH. \\
e. Increase the temperature of the digester to favor a population increase of the methane formers. \\

\item  Identify the incorrect statement regarding operation of an anaerobic digester. \\

*a. A healthy anaerobic digester would generally have volatile acids in the range of 50 mg/l to 300 mg/l. \\
b. The best strategy for pumping raw sludge to an anaerobic digester is to pump it once or twice a day so that the thickest possible raw sludge may be pumped. \\
c. An anaerobic digester operating at an alkalinity of 3200 mg/l should be able to tolerate a volatile acid concentration of 250 mg/l. \\
d. The change in pH is not a reliable indicator of the changing characteristics of the digesting sludge because the alkalinity in an anaerobic digester acts as a buffer. \\
e. The higher the volatile solid content of the primary sludge being fed to the digester, the higher is the expected volatile reduction. \\

\item  Identify the true statement about anaerobic digesters: \\

a. Carbon dioxide and methane in digester gas should be 65\% and 35\%, respectively. \\
*b. Increasing carbon dioxide readings indicate possible organic overload. \\
c. Decreasing mixing will always improve recovery of a sour digester. \\
d. Increasing the frequency and decreasing the amount of sludge pumping will not improve digester performance. \\
e. Reducing the ratio of primary to waste activated sludge will improve gas production. \\

\item  All of the following are normal operating guidelines for a healthy anaerobic digester except for: \\

a. A mesophilic digester operating at 93°F to 98°F. \\
b. Methane gas in the range of approximately 62\% to 70\%. \\
c. Carbon dioxide gas in the range of 30\% to 38\%. \\
*d. Organic loading to a high rate digester of 0.15 to 0.2 pounds (Lb) volatile Solids per day per ft3 of digester capacity. \\
e. Bicarbonate alkalinity in the range of 1500 to 1800. \\

\item  When adding anhydrous ammonia to a sour primary anaerobic digester, the\rule{1.5cm}{0.3cm} is an important consideration because if it is too high, digester "poisoning" may result: \\

a. Concentration of hydrogen sulfide in digester gas. \\
*b. pH in digesting sludge. \\
c. Concentration of free copper ions in digesting sludge \\
d. Ferrous and ferric ion concentration in digesting sludge. \\
e. Dissolved sulfide concentration. \\

\item  Which one of the following parameter ranges is most appropriate for a healthy anaerobic digester? \\

a. Volatile solid reduction in the range of 60 to 70\%. \\
b. Digester gas production of 7 to 10 ft3 of gas produced per day per pound of volatile solids destroyed. \\
*c. Alkalinity in the range of 2500 to 3500 mg/L. \\
d. Hydrogen sulfide in range of 1\% to 2\% by volume. \\
e. Volatile acids in range of 500 to 750 mg/L. \\

\item Identify the incorrect statement regarding operation of an anaerobic digester. 

a. A healthy anaerobic digester would generally have volatile acids in the range of 50 mg/l to 300 mg/l. 

*b. The best strategy for pumping raw sludge to an anaerobic digester is to pump it once or twice a day so that the thickest possible raw sludge may be pumped. 

c. An anaerobic digester operating at an alkalinity of 3200 mg/l should be able to tolerate a volatile acid concentration of 250 mg/l. 

d. The change in pH is not a reliable indicator of the changing characteristics of the digesting sludge because the alkalinity in an anaerobic digester acts as a buffer. 

e. The higher the volatile solid content of the primary sludge being fed to the digester, the higher is the expected volatile reduction. 

\item Lab data from your 100,000 gallon primary anaerobic digester, which receives primary sludge only, is shown below. Using this data :


\begin{flalign*}
Date	&& pH	&&Alkalinity(mg/L)	&&Vol.Acids(mg/L)	&&CO_2(\%)\\
\hline
9/02	&&7.10	&&3,200	&&280	&&35.5\\
9/09	&&7.00	&&3,020	&&320	&&36.0\\
9/16	&&6.90	&&2,800	&&400	&&37.7\\
9/17	&&6.85	&&2,720	&&450	&&38.2
\end{flalign*}




\begin{flalign*}
Date	&&RawSludge(TS\%)	&&Raw \enspace Sludge(VS\%)	&&Digested \enspace Sludge(VS\%)\\
\hline
9/02	&&5.4	&&65.5	&&56.0\\
9/09	&&5.0	&&66.7	&&53.8\\
9/16	&&4.9	&&65.9	&&54.2
\end{flalign*}

\begin{enumerate}
\item Calculate the average volatile solids reduction. Compare your calculated value to generally accepted ranges for a healthy anaerobic digester. Comment.\\
\item Compare the other data to expected ranges.

\item Is this digester experiencing an operational problem ? If so, what is the problem. Name three steps that may be taken to mitigate the problem.
\item Should slake lime be added ? Why or why not ?
\end{enumerate}
\vspace{1cm}




\item As part of an ongoing training program at your wastewater plant you are assigned the task of preparing a lecture for OITs on the subject of "buffer" in an anaerobic digester. Answer the following questions in preparation for your presentation.  
\begin{enumerate}[a.]
\item What is "buffer" and why is it important in an anaerobic digester?
\item What lab test is used to measure "buffer" in an anaerobic digester? Explain how this test is run
\item Most of the buffer in an anaerobic digester is due to the presence of
\end{enumerate}




\end{enumerate}