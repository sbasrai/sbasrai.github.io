\chapterimage{QuizCover} % Chapter heading image

\chapter{Stabilization Ponds Assessment}
% \textbf{Multiple Choice}

\section*{Stabilization Ponds Assessment}



\begin{enumerate}

\item  Organic loading to a stabilization pond is always calculated by knowing the pounds of BOD applied per 1.000 cubic feet of pond volume per day.\\


a. True \\

*b. False \\


\item  An abundance of aquatic plant growth in and around the edge of a facultative pond provides more surface area for biologic activity and increases treatment capacity.\\


a. True \\

*b. False \\


\item  Facultative ponds are anaerobic on the bottom and aerobic near the surface.\\


*a. True \\

b. False \\


\item  The best time of the year to initiate operation of a new facultative pond is during the coldest months of the year.\\


a. True \\

*b. False \\


\item  Ponds that contain an aerobic top layer and an anaerobic bottom layer are called facultative ponds.\\


*a. True \\

b. False \\


\item  In a facultative pond, the bottom layer of the pond will generally contain more dissolved oxygen than the surface layer.\\


a. True \\

*b. False \\


\item  It is recommended to start up a pond in the coldest months of the year in order to avoid odor problems and to take advantage of the increased solubility of oxygen in cold water.\\


a. True \\

*b. False \\


\item  A one acre wastewater treatment pond operated at a depth of 3 feet holds approximately 1 million gallons.\\


*a. True \\

b. False \\


\item  Algae is primarily responsible for the effective working of an anaerobic pond\\


a. True \\

*b. False \\


\item  Tertiary (polishing) ponds typically receive raw (untreated wastewater) as the feed stream\\


a. True \\

*b. False \\


\item  Facultative ponds are anaerobic on the bottom and aerobic near the surface.\\


*a. True \\

b. False \\


\item  In a facultative pond, a drop in the pH will normally be accompanied by a rise in the dissolved oxygen.\\


*a. True \\

b. False \\


\item  Algae is primarily responsible for the effective working of an anaerobic pond 

a. True \\
*b. False 


\item  Algae produce oxygen in a facultative pond. 

*a. True\\
b.False 


\item  A one acre wastewater treatment pond operated at a depth of 3 feet holds approximately 1 million gallons. 

*a. True \\
b. False 


\item  In a facultative pond, aerobic bacteria produce oxygen that is consumed by algae. 

a. True \\
*b. False 


\item  In a facultative pond, the pH falls when carbon dioxide produced by the aerobic bacteria is consumed by the algae 

a. True \\
*b. False 


\item  In a facultative pond, the algae utilizes the oxygen given off as a by-product of bacterial respiration 

a. True \\
*b. False 


\item  In a facultative pond, the bottom layer of the pond will generally contain more dissolved oxygen than the surface layer. 

a. True \\
*b. False 


\item  In a facultative pond, a drop in the pH will normally be accompanied by a rise in the dissolved oxygen. 

*a. True \\
b. False 


\item  In a facultative pond, the bottom layer of the pond will generally contain more dissolved oxygen than the surface layer. 

a. True \\
*b. False 


\item  In a facultative wastewater treatment pond that Is operating normally, a rise In DO will be accompanied by a drop in pH. 

a. True \\
*b. False 


\item  It is recommended to start up a pond in the coldest months of the year in order to avoid odor problems and to take advantage of the increased solubility of oxygen in cold water. 

a. True \\
*b. False 


\item  Operational depth of a wastewater pond is not used for calculating its organic loading 

*a. True \\
b. False 


\item  Ponds that contain an aerobic top layer and an anaerobic bottom layer are called facultative ponds. 

*a. True \\
b. False 


\item  In a facultative pond, the bottom layer of the pond will generally contain more dissolved oxygen than the surface layer. 

a. True \\
*b. False 


\item  It is recommended to start up a pond in the coldest months of the year in order to avoid odor problems and to take advantage of the increased solubility of oxygen in cold water. 

a. True \\
*b. False 


\item  A one acre wastewater treatment pond operated at a depth of 3 feet holds approximately 1 million gallons. 

*a. True \\
b. False 


\item  Algae is primarily responsible for the effective working of an anaerobic pond 

a. True \\
*b. False 


\item  Tertiary (polishing) ponds typically receive raw (untreated wastewater) as the feed stream 

a. True \\
*b. False \\

\item  The best time of the year to initiate operation of a new facultative pond is during the coldest months of the year. 

a. True \\
*b. False 


\item  Ponds that contain an aerobic top layer and an anaerobic bottom layer are called facultative ponds. 

*a. True \\
b. False 


\item  In a facultative pond, the bottom layer of the pond will generally contain more dissolved oxygen than the surface layer. 

a. True \\
*b. False 


\item  It is recommended to start up a pond in the coldest months of the year in order to avoid odor problems and to take advantage of the increased solubility of oxygen in cold water. 

a. True \\
*b. False 


\item  A one acre wastewater treatment pond operated at a depth of 3 feet holds approximately 1 million gallons. 

*a. True \\
b. False 


\item  A wastewater pond may have a detention time of 30 to 120 days. 

*a. True \\
b. False 


\item  Sludge deposits are usually in evidence at the effluent end of a waste treatment pond rather than the influent end. 

a. True \\
*b. False 


\item  It is good practice to start waste treatment ponds during winter because of the lower level of biological activity. 

a. True \\
*b. False 


\item  Anaerobic ponds require higher. minimum temperatures than aerobic ponds to achieve satisfactory treatment, 

*a. True \\
b. False 


\item  The formula for calculating percent BOD removal in pond systems is: BOD removal, % = (in-out)/in x 100% 

*a. True \\
b. False 


\item  The organic loading on a wastewater lagoon is higher than on a polishing pond. 

*a. True \\
b. False 


\item  Algae produce oxygen in a facultative pond. 

*a. True \\
b. False 


\item  A disadvantage of excessive scum on an oxidation pond is that it will interfere with photosynthesis. 

*a. True \\
b. False 


\item  Ranges of treatment efficiencies that can be expected from ponds vary more than most other treatment processes. 

*a. True \\
b. False 


\item  At night, algae in a facultative pond consume oxygen. 

*a. True \\
b. False 


\item  In wastewater treatment ponds, DO levels very rarely exceed 15 mg/l at the pond surface. 

a. True \\
*b. False 


\item  Organic loading to a wastewater treatment pond is expressed in units of pounds of VSS per acre per day. 

a. True \\
*b. False 


\item  Tertiary (polishing) ponds typically receive raw (untreated wastewater) as the feed stream 

a. True \\
*b. False 


\item  Ponds that contain an aerobic top layer and an anaerobic bottom layer are called facultative ponds. 

*a. True \\
b. False 


\item  Ponds that contain an aerobic top layer and an anaerobic bottom layer are called facultative ponds. 

*a. True \\
b. False 


\item  In a facultative pond, a drop in the pH will normally be accompanied by a rise in the dissolved oxygen. 

*a. True \\
b. False 


\item  Algae is primarily responsible for the effective working of an anaerobic pond 

a. True \\
*b. False 


\item  Algae is primarily responsible for the effective working of an anaerobic pond 

a. True \\
*b. False 


\item  Algae produce oxygen in a facultative pond. 

*a. True\\
b.False 


\item  A one acre wastewater treatment pond operated at a depth of 3 feet holds approximately 1 million gallons. 

*a. True \\
b. False 


\item  In a facultative pond, aerobic bacteria produce oxygen that is consumed by algae. 

a. True \\
*b. False 


\item  In a facultative pond, the pH falls when carbon dioxide produced by the aerobic bacteria is consumed by the algae 

a. True \\
*b. False 


\item  In a facultative pond, the algae utilizes the oxygen given off as a by-product of bacterial respiration 

a. True \\
*b. False 


\item  In a facultative pond, the bottom layer of the pond will generally contain more dissolved oxygen than the surface layer. 

a. True \\
*b. False 


\item  In a facultative pond, a drop in the pH will normally be accompanied by a rise in the dissolved oxygen. 

*a. True \\
b. False 


\item  In a facultative pond, the bottom layer of the pond will generally contain more dissolved oxygen than the surface layer. 

a. True \\
*b. False 


\item  In a facultative pond, a drop in the pH will normally be accompanied by a rise in the dissolved oxygen. 

a. True \\
*b. False 


\item  In a facultative wastewater treatment pond that Is operating normally, a rise In DO will be accompanied by a drop in pH. 

a. True \\
*b. False 


\item  It is recommended to start up a pond in the coldest months of the year in order to avoid odor problems and to take advantage of the increased solubility of oxygen in cold water.

a. True \\
*b. False 


\item  It is recommended to start up a pond in the coldest months of the year in order to avoid odor problems and to take advantage of the increased solubility of oxygen in cold water. 

a. True \\
*b. False 


\item  Operational depth of a wastewater pond is not used for calculating its organic loading 

*a. True \\
b. False 


\item  The best time of the year to initiate operation of a new facultative pond is during the coldest months of the year. 

a. True \\
*b. False 


\item  Ponds that contain an aerobic top layer and an anaerobic bottom layer are called facultative ponds. 

*a. True \\
b. False 


\item  In a facultative pond, the bottom layer of the pond will generally contain more dissolved oxygen than the surface layer. 

a. True \\
*b. False 


\item  It is recommended to start up a pond in the coldest months of the year in order to avoid odor problems and to take advantage of the increased solubility of oxygen in cold water. 

a. True \\
*b. False 


\item  A one acre wastewater treatment pond operated at a depth of 3 feet holds approximately 1 million gallons. 

*a. True \\
b. False 


\item  Algae is primarily responsible for the effective working of an anaerobic pond 

a. True \\
*b. False 


\item  Tertiary (polishing) ponds typically receive raw (untreated wastewater) as the feed stream 

a. True \\
*b. False 


\item  Ponds that contain an aerobic top layer and an anaerobic bottom layer are called facultative ponds. 

*a. True \\
b. False 


\item  Ponds that contain an aerobic top layer and an anaerobic bottom layer are called facultative ponds. 

*a. True \\
b. False 


\item  In a facultative pond, a drop in the pH will normally be accompanied by a rise in the dissolved oxygen. 

*a. True \\
b. False 


\item  Ranges of treatment efficiencies that can be expected from ponds vary more than most other treatment processes. 

*a. True \\
b. False 


\item  Tertiary (polishing) ponds typically receive raw (untreated wastewater) as the feed stream 

a. True \\
*b. False 


\item  The best time of the year to initiate operation of a new facultative pond is during the coldest months of the year. 

a. True \\
*b. False 


\item  The color of the algae at the surface of a facultative pond can be an indication of its operational condition. 

*a. True \\
b. False 


\item  The flow to a wastewater treatment pond may be expressed in units of acre -feet. 

a. True \\
*b. False 


\item  Usually facultative ponds are mixed with mechanical aerators at night and on cloudy days to disperse food and bacteria evenly. 

a. True \\
*b. False 


\item  Wastewater treatment ponds that receive untreated raw wastewater are called stabilization ponds. 

*a. True \\
b. False 


\item  When a facultative pond develops a green appearance, it should be treated with copper sulfate to kill off the algae. 

a. True \\
*b. False 


\item  When operating ponds in series, the accumulation of solids in the first pond may become a problem after long periods of use. 

*a. True \\
b. False \\

\item  Organic loading to a stabilization pond is always calculated by knowing the pounds of BOD applied per 1.000 cubic feet of pond volume per day.

a. True \\
*b. False 


\item  An abundance of aquatic plant growth in and around the edge of a facultative pond provides more surface area for biologic activity and increases treatment capacity.

a. True \\
*b. False 


\item  Facultative ponds are anaerobic on the bottom and aerobic near the surface.

*a. True \\
b. False 


\item  The best time of the year to initiate operation of a new facultative pond is during the coldest months of the year.

a. True \\
*b. False 


\item  Ponds that contain an aerobic top layer and an anaerobic bottom layer are called facultative ponds.

*a. True \\
b. False 


\item  In a facultative pond, the bottom layer of the pond will generally contain more dissolved oxygen than the surface layer.

a. True \\
*b. False 


\item  It is recommended to start up a pond in the coldest months of the year in order to avoid odor problems and to take advantage of the increased solubility of oxygen in cold water.

a. True \\
*b. False 


\item  A one acre wastewater treatment pond operated at a depth of 3 feet holds approximately 1 million gallons.

*a. True \\
b. False 


\item  Algae is primarily responsible for the effective working of an anaerobic pond

a. True \\
*b. False 


\item  Tertiary (polishing) ponds typically receive raw (untreated wastewater) as the feed stream

a. True \\
*b. False \\

\item  In a facultative pond, a drop in the pH will normally be accompanied by a rise in the dissolved oxygen.

*a. True \\
b. False 


\item  One short-term corrective measure for an overloaded facultative pond might be to add:\\


a. copper sulfate . \\

b. sodium sulfide. \\

c. ammonium sulfide . \\

*d. sodium nitrate. \\

e. potassium chloride. \\


\item  Photosynthesis is an essential part of the biological activity associated with:\\


a. Activated sludge \\

b. Trickling filters \\

c. Oxidation ditches \\

d. Aerobic digesters \\

*e. Sewage lagoons \\


\item  During the process of algal photosynthesis:\\


a. Chlorophyll converts sunlight into energy for growth. \\

b. Algae produces oxygen \\

c. Algae converts CO2, NH3, and PO4, into additional algae cells \\

*d. All of the above \\


\item  pH of the facultative pond will be the highest\\


*a. during daytime when the consumption of CO2 is the highest \\

b. during daytime when the consumption of CO2 is the lowest \\

c. during nighttime when the production of CO2 is highest \\

d. during nighttime when the production of CO2 is lowest \\


\item  A lagoon operator collects a sample of effluent at 2:15 pm. on a sunny July day and tests it. for dissolved oxygen. The dissolved oxygen is 22 mg/1 and the pH is 9.2.  The lagoon has a green color. Effluent suspended solids have been running at 75 mg/1.  The operator should \\


a. Do nothing. The conditions described are normal. \\

*b. Apply algaecide to the lagoon to kill the algae. \\

c. Drawdown the lagoon to eliminate excess DO. \\

d. Isolate cell. \\


\item  Algae in a stabilization pond is most likely to: \\


a. consume oxygen during daylight hours. \\

b. decrease effluent TSS during the day. \\

*c. change the pH throughout the day. \\

d. increase oxygen at night. \\

e. None of the above. \\


\item  Algae in a stabilization pond is most likely to: \\


a. consume oxygen during daylight hours. \\

b. decrease effluent TSS during the day. \\

*c. change the pH throughout the day. \\

d. increase oxygen at night. \\

e. None of the above. \\


\item  An operator cannot maintain adequate water levels in one of the ponds. What will cause this to happen? \\


a. The pond is hydraulically overloaded. \\

*b. The pond seal leaks. \\

c. The flow control structure leaks. \\

d. The flow control structure does not split the flow evenly. \\


\item  A properly designed and operated wastewater stabilization pond will remove \rule{1.5cm}{0.3mm} percent BOD. \\


a. 40\%-60\%. \\

b. 60\%-70\%. \\

c. 70\%-80\%. \\

*d. 80\%-90\%. \\


\item  At night algae in a conventional lagoon will: \\


*a. Cease to produce oxygen \\

b. Consume oxygen \\

c. Produce less oxygen \\

d. Increase the pH of the lagoon contents \\


\item  At what time of day is the dissolved oxygen content highest in a lagoon? \\


a. 3 a.m. \\

b. 7 a.m. \\

c. 9 a.m. \\

*d. 3 p.m. \\


\item  Cattails growing in lagoon will \\


*a. Cause short circuiting in affected lagoon. \\

b. Eliminate mosquito larvae. \\

c. Increase the diurnal pH fluctuations. \\

d. Increase toxic blue-green algae concentrations in the effluent. \\


\item  Dike vegetation should be controlled by \\


*a. Mowing periodically. \\

b. Burning in the spring and fall. \\

c. Allowing the cattle to graze on the dikes. \\

d. Any of the above would be acceptable. \\


\item  Due to diurnal differences in operation, a lagoon system is likely to experience the lowest dissolved oxygen readings \\


a. At any time since diurnal differences have no bearing on DO values \\

b. During the time when the sun is out and it is the hottest \\

*c. During night, just before dawn \\

d. When BOD loading is the lowest \\


\item  During the process of algal photosynthesis: \\


a. Chlorophyll converts sunlight into energy for growth. \\

b. Algae produces oxygen \\

c. Algae converts CO2, NH3, and PO4, into additional algae cells \\

*d. All of the above \\


\item  Given the following data, what is the most likely cause of the aerated pond problem? ---
DATA: Turbulence over center part of pond.  Solids floating to surface on edge of pond.  DO in center portion= 1.4 mg/L. \\


a. Hydraulic flow too slow through pond. \\

*b. Inadequate aeration on edges on pond. \\

c. Sludge scraper not operating. \\

d. Toxic materials in pond. \\


\item  Hydraulic loading to a facultative pond equals: \\


a. Volume (MG) divided by Flow (MGD) \\

*b. Depth (inches) divided by Detention Time (days) \\

c. Volume (acre-feet) divided by Flow (acre-inches/day) \\

d. Flow ( gallons/day) divided by Area (acres) \\

e. Area (acres) divided by Flow (acre-inches/day) \\


\item  In a properly operating facultative pond, algae live on carbon dioxide and nutrients during the day, and at night produce carbon dioxide. This has what effect on the pH? \\


*a. pH increases during the day, and decreases at night \\

b. pH decreases during the day, and increases at night \\

c. pH stays the same no matter what time of day \\

d. Carbon dioxide has no effect on pH \\


\item  It is important to completely mix and aerate the following type of pond: \\


*a. aerobic. \\

b. facultative. \\

c. anaerobic. \\

d. All of the above. \\

e. None of the above. \\


\item  One short-term corrective measure for an overloaded facultative pond might be to add: \\


a. copper sulfate . \\

b. sodium sulfide. \\

c. ammonium sulfide . \\

*d. sodium nitrate. \\

e. potassium chloride. \\


\item  pH of the facultative pond will be the highest \\


*a. during daytime when the consumption of CO2 is the highest \\

b. during daytime when the consumption of CO2 is the lowest \\

c. during nighttime when the production of CO2 is highest \\

d. during nighttime when the production of CO2 is lowest \\


\item  Photosynthesis is an essential part of the biological activity associated with: \\


a. Activated sludge \\

b. Trickling filters \\

c. Oxidation ditches \\

d. Aerobic digesters \\

*e. Sewage lagoons \\


\item  Due to diurnal differences in operation, a facultative pond is likely to experience the lowest dissolved oxygen readings \\


a. At any time since diurnal differences have no bearing on DO values \\

b. During the time when the sun is out and it is the hottest \\

*c. During night, just before dawn \\

d. When BOD loading is the lowest \\


\item  Due to diurnal differences in operation, a facultative pond is likely to experience the lowest dissolved oxygen readings \\


a. At any time since diurnal differences have no bearing on DO values \\

b. During the time when the sun is out and it is the hottest \\

*c. During night, just before dawn \\

d. When BOD loading is the lowest \\


\item  The main function of algae in a facultative wastewater treatment pond is to: \\


a. produce carbon dioxide, which is then used by the facultative bacteria. \\

b. use up nutrients such as nitrogen and phosphorus. \\

*c. produce oxygen during daylight hours. \\

d. serve as a food source for essential protozoa \\

e. break down complex organics in the wastewater. \\


\item  Which of the following microorganisms are involved in the stabilization of wastewater in a facultative wastewater treatment pond? \\


a. Aerobic bacteria \\

b. Anaerobic \\

c. Facultative bacteria \\

*d. All of the above \\

e. (a) and (c) only \\


\item  Which of the following terms is usually not associated with wastewater treatment ponds? \\


*a. Filamentous bacteria \\

b. Symbiotic relationship \\

c. Photosynthesis \\

d. Sewage lagoon \\

e. Inches/day \\


\item  Which of the following statements is not true regarding a facultative wastewater treatment pond? \\


a. When starting a facultative pond, 1 foot of relatively clean water should be added to the pond prior to the addition of wastewater. \\

b. A facultative pond generally is operated at a detention time of 50 to 60 days or longer. \\

*c. DO concentrations of 10 to 15 mg/L or greater are frequently found in a facultative pond during the afternoon of a sunny day. \\

d. Organic loading to a pond is expressed as pounds of volatile suspended solids per acre per day. \\

e. A facultative pond has an anaerobic layer and an aerobic layer. \\


\item  The main function of algae in a facultative wastewater treatment pond is to: \\


a. produce carbon dioxide, which is then used by the facultative bacteria. \\

b. use up nutrients such as nitrogen and phosphorus. \\

*c. produce oxygen during daylight hours. \\

d. serve as a food source for essential protozoa \\

e. break down complex organics in the wastewater. \\


\item  Which of the following terms is usually not associated with wastewater treatment ponds? \\


*a. Filamentous bacteria \\

b. Symbiotic relationship \\

c. Photosynthesis \\

d. Sewage lagoon \\

e. Inches/day \\


\item  Which of the following statements is not true regarding a facultative wastewater treatment pond? \\


a. When starting a facultative pond, 1 foot of relatively clean water should be added to the pond prior to the addition of wastewater. \\

b. A facultative pond generally is operated at a detention time of 50 to 60 days or longer. \\

*c. DO concentrations of 10 to 15 mg/L or greater are frequently found in a facultative pond during the afternoon of a sunny day. \\

d. Organic loading to a pond is expressed as pounds of volatile suspended solids per acre per day. \\

e. A facultative pond has an anaerobic layer and an aerobic layer. \\


\item  Which of the following would not be a routine operational or maintenance problem in the operation of pond: \\


a. weed control. \\

b. levee maintenance. \\

c. insect control. \\

*d. temperature control. \\

e. scum control. \\


\item  Algae in a stabilization pond is most likely to: \\


a. consume oxygen during daylight hours. \\

b. decrease effluent TSS during the day. \\

*c. change the pH throughout the day. \\

d. increase oxygen at night. \\

e. None of the above. \\


\item  At night algae in a conventional lagoon will: \\


*a. Cease to produce oxygen \\

b. Consume oxygen \\

c. Produce less oxygen \\

d. Increase the pH of the lagoon contents \\


\item  Which of the following microorganisms are involved in the stabilization of wastewater in a facultative wastewater treatment pond? \\


a. Aerobic bacteria \\

b. Anaerobic \\

c. Facultative bacteria \\

*d. All of the above \\

e. (a) and (c) only \\


\item  Which of the following is not a term used to refer to a conventional wastewater treatment lagoon? \\


a. Oxidation pond \\

b. Stabilization pond \\

c. Facultative lagoon \\

*d. Aerobic lagoon \\


\item  The main function of algae in a facultative wastewater treatment pond is to: \\


a. produce carbon dioxide, which is then used by the facultative bacteria. \\

b. use up nutrients such as nitrogen and phosphorus. \\

*c. produce oxygen during daylight hours. \\

d. serve as a food source for essential protozoa. \\

e. break down complex organics in the wastewater. \\


\item  Sodium nitrate is a chemical that may be used in the operation of a wastewater pond to restore normal conditions after a pond "turn over." Its function is: \\


*a. To provide a source of chemically combined oxygen for the facultative bacteria. \\

b. To kill off the unwanted blue-green algae. \\

c. To preserve the nitrogen balance in the pond. \\

d. To control the growth of weeds. \\

e. To neutralize chlorine residuals in the pond effluent. \\


\item  It is important to completely mix and aerate the following type of pond: \\


*a. aerobic. \\

b. facultative. \\

c. anaerobic. \\

d. All of the above. \\

e. None of the above. \\


\item  The laboratory tests most frequently used to monitor a facultative pond on a daily basis are: \\


a. pH, color, and BOD \\

*b. pH, BOD, and suspended solids \\

c. pH, DO, and temperature \\

d. DO, BOD, and SS \\

e. Microscopic examination, color, and DO \\


\item  The effluent from a conventional lagoon should be withdrawn: \\


a. Off the surface \\

b. Near the bottom \\

*c. Six to eighteen inches below the surface \\

d. Intermittently \\


\item  Algae in a stabilization pond is most likely to: \\


a. consume oxygen during daylight hours. \\

b. decrease effluent TSS during the day. \\

*c. change the pH throughout the day. \\

d. increase oxygen at night. \\

e. None of the above. \\


\item  Which of the following terms is usually not associated with wastewater treatment ponds? \\


*a. Filamentous bacteria \\

b. Symbiotic relationship \\

c. Photosynthesis \\

d. Sewage lagoon \\

e. Inches/day \\


\item  Which of the following statements is not true regarding a facultative wastewater
treatment pond? \\


a. When starting a facultative pond, 1 foot of relatively clean water should be added to the pond prior to the addition of wastewater. \\

b. A facultative pond generally is operated at a detention time of 50 to 60 days or longer. \\

*c. DO concentrations of 10 to 15 mg/L or greater are frequently found in a facultative pond during the afternoon of a sunny day. \\

d. Organic loading to a pond is expressed as pounds of volatile suspended solids per acre per day. \\

e. A facultative pond has an anaerobic layer and an aerobic layer. \\


\item  One short-term corrective measure for an overloaded facultative pond might be to add: \\


a. copper sulfate . \\

b. sodium sulfide. \\

c. ammonium sulfide . \\

*d. sodium nitrate. \\

e. potassium chloride. \\


\item  Photosynthesis is an essential part of the biological activity associated with: \\


a. Activated sludge \\

b. Trickling filters \\

c. Oxidation ditches \\

d. Aerobic digesters \\

*e. Sewage lagoons \\


\item  pH of the facultative pond will be the highest \\


*a. during daytime when the consumption of CO2 is the highest \\

b. during daytime when the consumption of CO2 is the lowest \\

c. during nighttime when the production of CO2 is highest \\

d. during nighttime when the production of CO2 is lowest \\


\item  One short-term corrective measure for an overloaded facultative pond might be to add: \\


a. copper sulfate . \\

b. sodium sulfide. \\

c. ammonium sulfide . \\

*d. sodium nitrate. \\

e. potassium chloride. \\


\item  Photosynthesis is an essential part of the biological activity associated with: \\


a. Activated sludge \\

b. Trickling filters \\

c. Oxidation ditches \\

d. Aerobic digesters \\

*e. Sewage lagoons \\


\item  During the process of algal photosynthesis: \\


a. Chlorophyll converts sunlight into energy for growth. \\

b. Algae produces oxygen \\

c. Algae converts CO2, NH3, and PO4, into additional algae cells \\

*d. All of the above \\


\item  pH of the facultative pond will be the highest \\


*a. during daytime when the consumption of CO2 is the highest \\

b. during daytime when the consumption of CO2 is the lowest \\

c. during nighttime when the production of CO2 is highest \\

d. during nighttime when the production of CO2 is lowest \\


\item  A lagoon operator collects a sample of effluent at 2:15 pm. on a sunny July day and tests it. for dissolved oxygen. The dissolved oxygen is 22 mg/1 and the pH is 9.2. The lagoon has a green color. Effluent suspended solids have been running at 75 mg/1. The operator should 

a. Do nothing. The conditions described are normal. \\
*b. Apply algaecide to the lagoon to kill the algae. \\
c. Drawdown the lagoon to eliminate excess DO. \\
d. Isolate cell. 


\item  Algae in a stabilization pond is most likely to: 

a. consume oxygen during daylight hours. \\
b. decrease effluent TSS during the day. \\
*c. change the pH throughout the day. \\
d. increase oxygen at night. \\
e. None of the above. 


\item  An operator cannot maintain adequate water levels in one of the ponds. What will cause this to happen? 

a. The pond is hydraulically overloaded. \\
*b. The pond seal leaks. \\
c. The flow control structure leaks. \\
d. The flow control structure does not split the flow evenly. 


\item  A properly designed and operated wastewater stabilization pond will remove what percent BOD. 

a. 40\%-60\%. \\
b. 60\%-70\%. \\
c. 70\%-80\%. \\
*d. 80\%-90\%. 


\item  At night algae in a conventional lagoon will: 

*a. Cease to produce oxygen \\
b. Consume oxygen \\
c. Produce less oxygen \\
d. Increase the pH of the lagoon contents 


\item  At what time of day is the dissolved oxygen content highest in a lagoon? 

a. 3 a.m. \\
b. 7 a.m. \\
c. 9 a.m. \\
*d. 3 p.m. 


\item  Cattails growing in lagoon will 

*a. Cause short circuiting in affected lagoon. \\
b. Eliminate mosquito larvae. \\
c. Increase the diurnal pH fluctuations. \\
d. Increase toxic blue-green algae concentrations in the effluent. 


\item  Dike vegetation should be controlled by 

*a. Mowing periodically. \\
b. Burning in the spring and fall. \\
c. Allowing the cattle to graze on the dikes. \\
d. Any of the above would be acceptable. 


\item  Due to diurnal differences in operation, a lagoon system is likely to experience the lowest dissolved oxygen readings 

a. At any time since diurnal differences have no bearing on DO values \\
b. During the time when the sun is out and it is the hottest \\
*c. During night, just before dawn \\
d. When BOD loading is the lowest 


\item  Given the following data, what is the most likely cause of the aerated pond problem? ---\\
DATA: Turbulence over center part of pond. Solids floating to surface on edge of pond. DO in center portion= 1.4 mg/L. 

a. Hydraulic flow too slow through pond. \\
*b. Inadequate aeration on edges on pond. \\
c. Sludge scraper not operating. \\
d. Toxic materials in pond. 


\item  Hydraulic loading to a facultative pond equals: 

a. Volume (MG) divided by Flow (MGD) \\
*b. Depth (inches) divided by Detention Time (days) \\
c. Volume (acre-feet) divided by Flow (acre-inches/day) \\
d. Flow ( gallons/day) divided by Area (acres) \\
e. Area (acres) divided by Flow (acre-inches/day) 


\item  In a properly operating facultative pond, algae live on carbon dioxide and nutrients during the day, and at night produce carbon dioxide. This has what effect on the pH? 

*a. pH increases during the day, and decreases at night \\
b. pH decreases during the day, and increases at night \\
c. pH stays the same no matter what time of day \\
d. Carbon dioxide has no effect on pH 


\item  It is important to completely mix and aerate the following type of pond: 

*a. aerobic. \\
b. facultative. \\
c. anaerobic. \\
d. All of the above. \\
e. None of the above. 


\item  Due to diurnal differences in operation, a facultative pond is likely to experience the lowest dissolved oxygen readings 

a. At any time since diurnal differences have no bearing on DO values \\
b. During the time when the sun is out and it is the hottest \\
*c. During night, just before dawn \\
d. When BOD loading is the lowest 


\item  The main function of algae in a facultative wastewater treatment pond is to: 

a. produce carbon dioxide, which is then used by the facultative bacteria. \\
b. use up nutrients such as nitrogen and phosphorus. \\
*c. produce oxygen during daylight hours. \\
d. serve as a food source for essential protozoa \\
e. break down complex organics in the wastewater. 


\item  Which of the following microorganisms are involved in the stabilization of wastewater in a facultative wastewater treatment pond? 

a. Aerobic bacteria \\
b. Anaerobic \\
c. Facultative bacteria \\
*d. All of the above \\
e. (a) and (c) only 


\item  Which of the following statements is not true regarding a facultative wastewater treatment pond? 

a. When starting a facultative pond, 1 foot of relatively clean water should be added to the pond prior to the addition of wastewater. \\
b. A facultative pond generally is operated at a detention time of 50 to 60 days or longer. \\
*c. DO concentrations of 10 to 15 mg/L or greater are frequently found in a facultative pond during the afternoon of a sunny day. \\
d. Organic loading to a pond is expressed as pounds of volatile suspended solids per acre per day. \\
e. A facultative pond has an anaerobic layer and an aerobic layer. 


\item  Which of the following would not be a routine operational or maintenance problem in the operation of pond: 

a. weed control. \\
b. levee maintenance. \\
c. insect control. \\
*d. temperature control. \\
e. scum control. 


\item  Which of the following is not a term used to refer to a conventional wastewater treatment lagoon? 

a. Oxidation pond \\
b. Stabilization pond \\
c. Facultative lagoon \\
*d. Aerobic lagoon 


\item  Sodium nitrate is a chemical that may be used in the operation of a wastewater pond to restore normal conditions after a pond "turn over." Its function is: 

*a. To provide a source of chemically combined oxygen for the facultative bacteria. \\
b. To kill off the unwanted blue-green algae. \\
c. To preserve the nitrogen balance in the pond. \\
d. To control the growth of weeds. \\
e. To neutralize chlorine residuals in the pond effluent. 


\item  The laboratory tests most frequently used to monitor a facultative pond on a daily basis are: 

a. pH, color, and BOD \\
*b. pH, BOD, and suspended solids \\
c. pH, DO, and temperature \\
d. DO, BOD, and SS \\
e. Microscopic examination, color, and DO 


\item  The effluent from a conventional lagoon should be withdrawn: 

a. Off the surface \\
b. Near the bottom \\
*c. Six to eighteen inches below the surface \\
d. Intermittently 


\item  The facultative zone in a pond: 

a. provides no treatment. \\
b. provides oxygen to the anaerobic zone. \\
c. is a barrier to light penetration \\
d. produces the bacteria for the anaerobic zone. \\
*e. uses nitrate as an oxygen source for digestion. 


\item  Which of the following terms is usually not associated with wastewater treatment ponds? 

*a. Filamentous bacteria \\
b. Symbiotic relationship \\
c. Photosynthesis \\
d. Sewage lagoon \\
e. Inches/day \\


\item  A lagoon operator collects a sample of effluent at 2:15 pm. on a sunny July day and tests it. for dissolved oxygen. The dissolved oxygen is 22 mg/1 and the pH is 9.2. The lagoon has a green color. Effluent suspended solids have been running at 75 mg/1. The operator should 

a. Do nothing. The conditions described are normal. \\
*b. Apply algaecide to the lagoon to kill the algae. \\
c. Drawdown the lagoon to eliminate excess DO. \\
d. Isolate cell. 

\item  Given the following data, what is the most likely cause of the aerated pond problem? ---\\
DATA: Turbulence over center part of pond. Solids floating to surface on edge of pond. DO in center portion= 1.4 mg/L. 

a. Hydraulic flow too slow through pond. \\
*b. Inadequate aeration on edges on pond. \\
c. Sludge scraper not operating. \\
d. Toxic materials in pond. 


\item  Hydraulic loading to a facultative pond equals: 

a. Volume (MG) divided by Flow (MGD) \\
*b. Depth (inches) divided by Detention Time (days) \\
c. Volume (acre-feet) divided by Flow (acre-inches/day) \\
d. Flow ( gallons/day) divided by Area (acres) \\
e. Area (acres) divided by Flow (acre-inches/day) 


\item  Which of the following is not a term used to refer to a conventional wastewater treatment lagoon? 

a. Oxidation pond \\
b. Stabilization pond \\
c. Facultative lagoon \\
*d. Aerobic lagoon 


\item  Sodium nitrate is a chemical that may be used in the operation of a wastewater pond to restore normal conditions after a pond "turn over." Its function is: 

*a. To provide a source of chemically combined oxygen for the facultative bacteria. \\
b. To kill off the unwanted blue-green algae. \\
c. To preserve the nitrogen balance in the pond. \\
d. To control the growth of weeds. \\
e. To neutralize chlorine residuals in the pond effluent. 


\item  The laboratory tests most frequently used to monitor a facultative pond on a daily basis are: 

a. pH, color, and BOD \\
*b. pH, BOD, and suspended solids \\
c. pH, DO, and temperature \\
d. DO, BOD, and SS \\
e. Microscopic examination, color, and DO 


\item  The effluent from a conventional lagoon should be withdrawn: 

a. Off the surface \\
b. Near the bottom \\
*c. Six to eighteen inches below the surface \\
d. Intermittently 


\item  Sodium nitrate is a chemical that may be used in the operation of a wastewater pond to restore normal conditions after a pond "turn over." Its function is: 

*a. To provide a source of chemically combined oxygen for the facultative bacteria. \\
b. To kill off the unwanted blue-green algae. \\
c. To preserve the nitrogen balance in the pond. \\
d. To control the growth of weeds. \\
e. To neutralize chlorine residuals in the pond effluent. 


\item  The effluent from a conventional lagoon should be withdrawn: 

a. Off the surface \\
b. Near the bottom \\
*c. Six to eighteen inches below the surface \\
d. Intermittently 


\item  The facultative zone in a pond: 

a. provides no treatment. \\
b. provides oxygen to the anaerobic zone. \\
c. is a barrier to light penetration. \\
d. produces the bacteria for the anaerobic zone. \\
*e. uses nitrate as an oxygen source for digestion. 

\item  One short-term corrective measure for an overloaded facultative pond might be to add:

a. copper sulfate . \\
b. sodium sulfide. \\
c. ammonium sulfide . \\
*d. sodium nitrate. \\
e. potassium chloride. \\



\item  During the process of algal photosynthesis:

a. Chlorophyll converts sunlight into energy for growth. \\
b. Algae produces oxygen \\
c. Algae converts CO2, NH3, and PO4, into additional algae cells \\
*d. All of the above 


\item  pH of the facultative pond will be the highest

*a. during daytime when the consumption of CO2 is the highest \\
b. during daytime when the consumption of CO2 is the lowest \\
c. during nighttime when the production of CO2 is highest \\
d. during nighttime when the production of CO2 is lowest 




\end{enumerate}