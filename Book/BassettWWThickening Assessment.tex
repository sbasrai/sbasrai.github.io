\chapterimage{QuizCover} % Chapter heading image

\chapter{Solids Thickening Assessment}
% \textbf{Multiple Choice}

\section*{Solids Thickening Assessment}



\begin{enumerate}

\item  In a gravity thickener the depth of the sludge is kept minimal (<six inches) to avoid solids going over the effluent weir \\

a. True \\
*b. False \\

\item  Sludge thickening is primarily conducted to reduce costs associated with biosolids hauling\\


a. True \\
*b. False \\


\item  Gravity thickener is commonly used for sludge dewatering \\

a. True \\
*b. False \\

\item  Sludge thickening is primarily conducted to reduce costs associated with biosolids hauling \\

a. True \\
*b. False \\
\item A DAF thickener has effluent solids of 55 mg/L and float solids of 2.0\%. Solids loading and polymer dosing is in the normal range. This data likely indicates: \\

a. This unit is operating normally \\
b. Too low air to solids ratio \\
c. Float blanket too thick \\
*d. Flight speed too fast \\
e. Flight speed too slow \\

\item  An air flotation thickener will produce a thin float if: \\

*a. Flight speed too high and skimmer wiper not adjusted properly \\
b. Excessive air/solids ratio and polymer dosages too low \\
c. High dissolved oxygen and flight speed too low \\
d. Polymer dosages too high and unit overloaded \\

\item  An increase in the pool depth of a scroll-type centrifuge: \\

a. would not affect the moisture content of the cake. \\
b. would produce a drier cake \\
*c. would produce a wetter cake, but produce a greater solids recovery. \\
d. would not affect either solids recovery, nor cake moisture content. \\
e. would require an increase in the cationic polymer dosage. \\

\item  A sludge thickened from 1\% to 4\% solids will be reduced in volume by how much? \\

a. no more than 4\% of original volume \\
b. approximately 17\% of original volume \\
*c. approximately 25\% of original volume* \\
d. more information is needed \\

\item  Gravity thickeners, compared to DAFs, are best suited to: \\

*a. Thickening primary sludge. \\
b. Thickening waste activated sludge. \\
c. Controlling sulfide odors. \\
d. Removing filamentous bacteria. \\
e. Provide highest concentration sludge. \\

\item  Which of the following is not the main reason for thickening sludge \\

a. Improved digester performance due to a lower volume of sludge \\
b. Cost savings in the construction of new digestion facilities \\
c. Reduction in anaerobic digestion heating requirements since less water has to be heated \\
*d. Reduce costs of biosolids hauling \\

\item  What zone is not involved in a belt filter press? \\

a. Gravity \\
b. Low pressure (wedge) \\
c. High pressure \\
*d. Twilight \\

\item  The float blanket in a DAF unit appears well flocculated and concentrated.  Too low a flight speed would likely result in: \\

a. Using excessive amounts of air. \\
b. Float solids that are too thick. \\
c. Too low an air-to-solids ratio. \\
*d. Poor thickener underflow quality. \\
e. De-flocculation of float solids. \\

\item  The least critical operational control of a dissolved air flotation thickener in producing an adequately thickened sludge is the: \\

a. Flight speed. \\
b. Air to solids ratio. \\
c. Polymer dosage. \\
*d. Recycle ratio. \\
e. Pre-thickened WAS concentration. \\

\item  The operational control of a dissolved air flotation thickener most critical for producing an adequately thickened float solids is the: \\

a. Fight speed. \\
*b. Air to solids ratio. \\
c. Polymer dosage. \\
d. Recycle ratio. \\
e. Concentration of WAS being thickened. \\

\item  To increase solids recovery on a dual belt, belt press, the operator should: \\

a. Increase the differential belt speed. \\
b. Increase belt speed. \\
c. Increase sludge feed rate. \\
*d. Decrease sludge feed rate. \\
e. Increase polymer feed. \\

\item  Too high a flight speed in a DAF will likely result in: \\

a. Using excessive amounts of air. \\
b. Excessive underflow volume. \\
*c. Thin float solids. \\
d. Too high an air to solids ratio. \\
e. Too low an air to solids ratio. \\

\item  Which one of the following statements is TRUE in regard to DAF thickeners? \\

*a. The air to solids ratio in a DAF unit is typically 0.03 to 0.05. \\
b. The speed of the so-called "flights" has very little effect on the concentration of the float solids. \\
c. Adjustments in the air to solids ratio in a DAF unit will affect the float solids but not the unit's effluent suspended solids. \\
d. Anionic polymers are typically used to condition the WAS feed to a DAF unit. \\

\item  Which one of the following process units is usually classified as a sludge thickening device as opposed to a dewatering device: \\

*a. DAF unit. \\
b. Sludge drying bed. \\
c. Vacuum filter press. \\
d. Belt press. \\
e. All of the above are thickeners, not dewatering devices. \\

\item  Which one the following statements is TRUE in regard to gravity thickeners? \\

a. Longer solids detention times are desired during summer operation of these units. \\
b. The sludge-volume-ratio (SVR) or sludge detention time is defined as the volume of the sludge blanket divided by the daily volume of sludge withdrawn from the thickener. \\
c. SVRs should be in range of 5 to 10 hours. \\
*d. A likely cause of a gravity thickener producing a poor quality effluent is too low of a sludge blanket.* \\

\item  A DAF thickener has effluent solids of 55 mg/L and float solids of 2.0\%. Solids loading and polymer dosing is in the normal range. This data likely indicates: \\

a. This unit is operating normally \\
b. Too low air to solids ratio \\
c. Float blanket too thick \\
*d. Flight speed too fast \\
e. Flight speed too slow \\

\item  An increase in the pool depth of a scroll-type centrifuge: \\

a. would not affect the moisture content of the cake. \\
b. would produce a drier cake \\
*c. would produce a wetter cake, but produce a greater solids recovery. \\
d. would not affect either solids recovery, nor cake moisture content. \\
e. would require an increase in the cationic polymer dosage. \\

\item  A sludge thickened from 1\% to 4\% solids will be reduced in volume by how much? \\

a. no more than 4\% of original volume \\
b. approximately 17\% of original volume \\
*c. approximately 25\% of original volume* \\
d. more information is needed \\

\item  Gravity thickeners, compared to DAFTs, are best suited to: \\

*a. Thickening primary sludge \\
b. Thickening waste activated sludge \\
c. Controlling sulfide odors \\
d. Removing filamentous bacteria \\
e. Provide highest concentration sludge \\

\item  Identify the incorrect statement regarding gravity thickeners. \\

a. Gravity thickeners are similar in design to primary clarifiers. \\
*b. The sludge blanket depth and the rate of sludge withdrawal are used to calculate the sludge detention time. \\
c. The purpose of pickets in a sludge thickener is to gently stir settling sludge particles to release gases that may prevent the sludge particles from compacting. \\
d. Mixtures of primary and waste activated sludge are never thickened in a gravity thickener due to the possibilility of de-nitrification. \\
e. Solids loading (pounds per day per square foot) is an important guideline for a gravity thickener. \\

\item  On a routine check of a DAFT unit the operator finds suspended solids of 450 mg/L in the effluent.  The float blanket appears well flocculated and concentrated. The operator should: \\

a. increase the flight speed \\
b. do nothing, the unit is operating normally \\
c. reduce the air to solids ratio \\
*d. increase the air to solids ratio \\
e. check unit operating pressure \\

\item  Gravity-thickened primary sludge will contain solids within which of the following concentration ranges? \\

a. 1,000 - 8,000 mg/l \\
b. 10,000 – 40,000 mg/l \\
c. 40,000 - 80,000 mg/l \\
d. 100,000 - 400,000 mg/l \\
*e. none of the above \\




\end{enumerate}