\chapterimage{QuizCover} % Chapter heading image

\chapter{Trickling Filters Assessment}
% \textbf{Multiple Choice}

\section*{Trickling Filters Assessment}
\begin{enumerate}
\item  Compared to activated sludge, trickling filters are sturdy work units not easily up set by shock loads. \\

*a. True \\
b. False \\


\item  Ponding on the surface of trickling filters is almost always caused by plugging of the underdrains immediately below the point of ponding \\

a. True\\
*b. False\\


\item  Zoogleal mass sheared of from the trickling filter media and carried out as part of the effluent flow is termed as sloughing \\

*a. True \\
b. False \\


\item  For multiple trickling filter operation, the parallel mode is more suitable than the series mode for treating influent wastewater with a high BOD content \\

a. True \\
*b. False \\


\item  Synthetic /pre-manufactured media used in the biofilter has the advantage of providing a greater surface area per volume upon which the zoologeal film may grow while providing ample void space for the free circulation of air. \\

*a. True \\
b. False \\


\item  Of the two forms of secondary treatment: trickling filters and conventional activated sludge, trickling filters are more likely to produce a clearer effluent. \\

a. True \\
*b. False \\


\item  A trickling filter reduces the strength of wastewater applied to it by the filtering and straining action of the stones or support media. \\

a. True \\
*b. False \\


\item  A trickling filter provides higher BOD removal in cold weather than in hot weather. \\

a. True \\
*b. False \\


\item  The biological growth on trickling filter media is composed principally of aerobic forms of bacteria, fungi, algae, protozoa, worms and the larvae of insects. \\

a. True \\
*b. False \\


\item  The spaces between the filter media in a trickling filter need to remain open to permit free flow of air throughout the bed. \\

*a. True \\
b. False \\


\item  The flooding of filter media with wastewater is sometimes effective in killing filter flies. \\

*a. True \\
b. False \\


\item  Sandstone, because of its ability to provide excellent ventilation usually is used for media in trickling filters. \\

a. True \\
*b. False \\


\item  When a trickling filter is referred to as a "roughing filter", it means the operational life of the filter is at an end, and a major overhaul is eminent. \\

a. True \\
*b. False \\


\item  The hydraulic loading applied. to a trickling filter is the total volume of liquid, including recirculation. \\

*a. True \\
b. False \\

\item  The main function of a launder in a secondary clarifier is to prevent scum and other floatables from leaving with the effluent flow \\

a. True\\
*b. False\\

\item  The trickling filter recirculation ratio is calculated as Qi/Qr where Qi is the influent flow and Qr is the recirculated flow. \\

a. True\\
*b. False \\


\item  The advantage of trickling filter over the activated sludge lies in its BOD removal efficiency \\

a. True\\
*b. False\\

\item  Recirculation helps prevent excessive sloughing\\

a. True\\
*b. False\\

\item  Series operation of trickling filter is the preferred mode during very low temperature conditions during winter\\

a. True \\
*b. False \\

\item  A good reason to run a trickling filter in the series mode of operation is: \\

 a. During cold weather to prevent ice formation \\
 b. When the influent BOD loading is low \\
 *c. When the influent BOD loading is high \\
 d. When the weather is warm \\


\item  A good reason to run a trickling filter in the parallel mode of operation is: \\

 a. When the weather is warm \\
 b. When the BOD loading is high \\
 *c. When the BOD loading is low \\
 d. To prevent filter flies \\


\item  A pan test should be done monthly on a trickling filter to: \\

 a. Check oil quality in the distributor bearing \\
 *b. Check sewage distribution on filter \\
 c. To check flow rates through the media \\
 d. All the above \\
 e. None of the above \\


\item  A problem associated with trickling filters is \\

 a. Bulking \\
 b. Protozoans \\
 *c. Ponding \\
 d. Liquefaction \\


\item  A trickling filter or activated sludge process causes nitrification. which is \\

 a. Conversion of nitrogen to nitrate \\
 b. Conversion of nitrogen to ammonia \\
 c. Conversion of nitrate to nitrogen \\
 *d. Conversion of ammonia to nitrate and nitrite nitrogen \\


\item  If a trickling filter has been operating with a hydraulic loading (including some recirculation) of 10 to 12 MGD and organic loading of about 80 pounds of BOD per 1,000 cu. ft./day the treatment efficiency will usually increase if the recirculation is increased. This might be attributed to: \\

 a. The increased recirculation wears down the soluble BOD to finer particles \\
 *b. The increased flow more completely wets and contacts all of the slime surfaces in the filter so the food to effective microorganism ration is less as in an activated sludge process \\
 c. The grazing population of warm and other organisms in the filter is flushed out before they consume the slime bacteria \\
 d. The increased flow more completely fills the under drains so cold updrafts are eliminated \\
 e. The statement is just poppycock put out by power companies to get us to use more electricity to run pumps \\


\item  Most common trickling filter operational control method is: \\

 a. Sloughing control \\
 *b. Recirculation \\
 c. Sludge removal \\
 d. Distributor arm speed \\


\item  The media in trickling filters is placed\\
 a. On a rubber tile floor. \\
 *b. On a system of tile or columnar underdrains. \\
 c. Directly in a concrete slab. \\
 d. Directly in the ground \\


\item  Sloughing from a trickling filter refers to \\

 a. A process whereby wastewater is recirculated over the filter. \\
 b. Bypassing of the filter. \\
 c. The breaking of the filter stone as a result of weathering and the sluicing of small stone particles to the final settling tank. \\
 *d. The periodic discharge of large quantities of slime growth with the filter effluent. \\


\item  The primary organisms responsible for treating wastewater in the trickling filter process are: \\

 a. Anaerobic bacteria \\
 b. Anoxic bacteria \\
 c. Facultative bacteria \\
 *d. Aerobic bacteria \\


\item  The primary organisms responsible for treating wastewater in the trickling filter process are: \\

 a. Anaerobic bacteria \\
 b. Anoxic bacteria \\
 c. Facultative bacteria \\
 *d. Aerobic bacteria \\


\item  Ponding of a trickling filter means \\

 a. Flooding the filter. \\
 b. Hosing the growth from the rocks with a high pressure hose. \\
 *c. Blockage in the media that s prevents water to flow through. \\
 d. Running the trickling filter effluent to a wastewater pond. \\


\item  Which of the following would be a cause for trickling filter ponding: \\

 a. Increased recirculation \\
 b. Filter flies \\
 c. Zoogleal mass \\
 *d. An excessive organic loading without a corresponding high recirculation rate \\


\item  A good reason to run a trickling filter in the series mode of operation is: \\

 a. During cold weather to prevent ice formation \\
 b. When the influent BOD loading is low \\
 *c. When the influent BOD loading is high \\
 d. When the weather is warm \\


\item  A good reason to run a trickling filter in the parallel mode of operation is: \\

 a. When the weather is warm \\
 b. When the BOD loading is high \\
 *c. When the BOD loading is low \\
 d. To prevent filter flies \\


\item  A problem associated with trickling filters is \\

 a. Bulking \\
 b. Protozoans \\
 *c. Ponding \\
 d. Liquefaction \\


\item  A trickling filter or activated sludge process causes nitrification. which is \\

 a. Conversion of nitrogen to nitrate \\
 b. Conversion of nitrogen to ammonia \\
 c. Conversion of nitrate to nitrogen \\
 *d. Conversion of ammonia to nitrate and nitrite nitrogen \\


\item  Which test best measures the efficiency of a trickling filter? \\

 a. Total solids. \\
 b. pH \\
 *c. BOD \\
 d. Temperature \\
 e. Sludge age \\


\item  What are sloughings? \\

 a. Troughs. \\
 b. Slop. \\
 *c. Zoogleal mass washed off trickling filter media. \\
 d. Waste-activated sludge. \\
 e. Grit. \\


\item  Most common trickling filter operational control method is: \\

 a. Sloughing control \\
 *b. Recirculation \\
 c. Sludge removal \\
 d. Distributor arm speed \\


\item  Most common trickling filter operational control method is: \\

 a. Sloughing control \\
 *b. Recirculation \\
 c. Sludge removal \\
 d. Distributor arm speed \\


\item  The rotating biological contactors operate based upon the same principles as: \\

 a. Aerated lagoons \\
 b. Activated sludge systems \\
 *c. Trickling filters \\
 d. Extended aeration \\


\item  A two-stage trickling filtration system means: \\

 a. one filter used with recirculation returning from filter effluent to filter influent at 100% of flow \\
 b. one filter used with recirculation from secondary clarifier effluent to trickling filter influent \\
 *c. two filters used in series, either directly or with a clarifier in between \\
 d. two filters used in parallel \\
 e. one filter used with no primary clarification \\


\item  Regardless of shape, one acre-foot of media in a trickling filter is equal to: \\

 a. 33,000 cu. ft. \\
 *b. 43,560 cu. ft. \\
 c. 55,560 cu. ft. \\
 d. 77,840 cu. ft. \\
 e. none of the above \\


\item  The presence of a "rotten egg" odor in the area of a trickling filter generally would indicate which of the following: \\

 a. the presence of the psychoda fly \\
 b. the clogging of the distributor arm orifices \\
 *c. anaerobic conditions within the filter \\
 d. too much recirculation \\
 e. too high a DO in the: wastewater being applied to the filter \\


\item  Analysis of a wastewater effluent from a standard rate trickling filter shows the following:\\
 Nitrite= 8 mg/L \\
 Ammonia= 2 mg/L\\
 Nitrate= 2 mg/L\\
 What do these results indicate? \\

 *a. Plant is operating properly \\
 b. Final end product, nitrite, is too low \\
 c. Final end product, nitrate, is too low \\
 d. DO was insufficient to allow complete aerobic oxidation of the nitrate to nitrite \\


\item  Given the data below, what is the most likely cause of the trickling filter problem? --\\
 DATA: Normal dry weather plant flow rate\\
 Sweep arms operating at normal rate\\
 Water not passing through as rapidly as normal\\
 Increased odor level\\
 Increased chlorine demand in final effluenTrue\\
 Final effluent turbid. \\

 *a. Filter media plugged with dead microorganisms \\
 b. Low solids load to filter \\
 c. seal in sweep arms worn \\
 d. Spray nozzles partially plugged. \\


\item  The rock/media in most trickling filters is placed \\

 a. Directly in the ground. \\
 b. Directly in a concrete slab. \\
 *c. On a system of tile underdrains \\
 d. On a rubber tile floor \\


\item  The primary organisms responsible for treating wastewater in the trickling filter process are: \\

 a. Anaerobic bacteria \\
 b. Anoxic bacteria \\
 c. Facultative bacteria \\
 *d. Aerobic bacteria \\


\item  Trickling filters usually consist of a bed of stone which performs its function in sewage by: \\

 a. Mechanical filtration of organic solids \\
 b. Mechanical filtration of inorganic solids \\
 c. Provides aeration \\
 *d. Supports a growth of organisms which feed upon the sewage and reduce BOD \\
 e. None of the above \\


\item  What are sloughings? \\

 a. Troughs. \\
 b. Slop. \\
 *c. Zoogleal mass washed off trickling filter media. \\
 d. Waste-activated sludge. \\
 e. Grit. \\


\item  Which of the following is not a benefit of recirculation in the trickling filter process? \\

 a. Dilutes high strength or toxic wastes. \\
 b. Helps prevent septic conditions in trickling filter \\
 *c. Helps prevent excessive sloughing \\
 d. Helps control odors, ponding and filter flies \\


\end{enumerate}