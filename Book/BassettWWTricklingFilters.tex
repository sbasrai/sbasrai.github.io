% \documentclass{article}
% %\usepackage[english]{babel}%
% \usepackage{graphicx}
% \usepackage{tabulary}
% \usepackage{tabularx}
% \usepackage[normalem]{ulem}
% \usepackage{cancel}
% \usepackage{tikz} 
% \usepackage{pdflscape}
% \usepackage{colortbl}
% \usepackage{lastpage}
% \usepackage{multirow}
% \usepackage{enumerate}
% \usepackage[shortlabels]{enumitem}
% \usepackage{color,soul}
% \usepackage{pdflscape}
% \usepackage{hyperref}
% %\usepackage[table]{xcolor}
% \usepackage{rotating}
% \usepackage{amsmath}
% \usepackage{fixltx2e}
% \usepackage{framed}
% \usepackage{mdframed}
% \usepackage[T1]{fontenc}
% \usepackage[utf8]{inputenc}
% \usepackage{textcomp}
% \usepackage{siunitx}
% \usepackage{ifthen}
% \usepackage{fancyhdr}
% \usepackage{gensymb}
% \usepackage{newunicodechar}
% \usepackage[document]{ragged2e}
% \usepackage[margin=1in,top=1.1in,headheight=57pt,headsep=0.1in]
% {geometry}
% \usepackage{ifthen}
% \usepackage{fancyhdr}
% \everymath{\displaystyle}
% \usepackage[document]{ragged2e}
% \usepackage{fancyhdr}
% \everymath{\displaystyle}
% \usepackage{empheq}

% \usepackage[most]{tcolorbox}

% \usepackage{booktabs} % Required for nicer horizontal rules in tables


% \usepackage{enumitem}

% %\usepackage[table,xcdraw]{xcolor}
% \usetikzlibrary{arrows}
% \linespread{2}%controls the spacing between lines. Bigger fractions means crowded lines%
% %\pagestyle{fancy}
% %\usepackage[margin=1 in, top=1in, includefoot]{geometry}
% %\everymath{\displaystyle}
% \linespread{1.3}%controls the spacing between lines. Bigger fractions means crowded lines%
% %\pagestyle{fancy}
% \pagestyle{fancy}
% \setlength{\headheight}{56.2pt}

% \definecolor{myblue}{rgb}{.8, .8, 1}
% \newcommand*\mybluebox[1]{%
% \colorbox{myblue}{\hspace{1em}#1\hspace{1em}}}

% \chead{\ifthenelse{\value{page}=1}{\includegraphics[scale=0.3]{SCC}\\ \textbf \textbf Wastewater Constituents Analysis \& Laboratory Methods}}
% \rhead{\ifthenelse{\value{page}=1}{}{}}
% \lhead{\ifthenelse{\value{page}=1}{}{Wastewater Constituents Analysis \& Laboratory Methods}}
% \rfoot{\ifthenelse{\value{page}=1}{Module 1: WATR 048 - Spring 2019}{Module 1: WATR 048 - Spring 2019}}

% \lfoot{Shabbir Basrai}
% \cfoot{Page \thepage\ of \pageref{LastPage}}
% \renewcommand{\headrulewidth}{2pt}
% \renewcommand{\footrulewidth}{1pt}
% \begin{document}
% %\begin{empheq}[box=\mybluebox]{align}
% %a&=b\\
% %E&=mc^2 + \int_a^a x\, dx
% %\end{empheq}

% \newlist{steps}{enumerate}{1} % Defines "Steps" for enumerate as Step 1, Step 2 etc.
% \setlist[steps, 1]{label = Step \arabic*:} % Defines "Steps" for enumerate as Step 1, Step 2 etc.

% \setlist{nolistsep} % Reduce spacing between bullet points and numbered lists


%_______________________________________________________________________________________________________________________________________%
\chapterimage{TFChapterImage1.jpg} % Chapter heading image

\chapter{Trickling Filters}

\section{Theoretical Background}\index{Theoretical Background}
		
Trickling filter is a fixed film secondary treatment process wherein the organic content of the wastewater is removed using biological growth attached to an inert media such as lava rock or plastic\\			
\begin{itemize}
\item In a trickling filter, the wastewater is sprayed evenly on the surface of the media with a rotary type distributor with orifices
\item The wastewater percolates through the media bed, where it comes in contact with biological slime growth – zoogleal film (zooglea)
\item The aerobic biomass - bacteria, protozoa and other microoorganisms in the zooglea capture and consume the suspended and dissolved organics from the wastewater.
\item The microorganisms metabolize the organics and in the process produce more microbial mass resulting in increasing the thickness of the zoogleal layer.
\item The thickness of the zoogleal layer can only increase to a point until the wastewater flow – hydraulic load, shears the slime layer – “sloughs off” and is carried out as part of the effluent flow as sloughing.
\item The treated wastewater cascades from the bottom of the media into the underdrain system – lower portion of the TF comprised of columns which support the media base.  The underdrain has a sloping floor to direct the cascading water into a center channel .
\item The clarifier allows for the separation (settling) of the  of the solids (sloughed off material).  The settled solids is removed - typically pumped to a digester and the clarified effluent flows out of the clarifier.
\item The source of oxygen to support the aerobic growth is from the oxygen dissolved in the wastewater as it is sprayed over the media and from the air currents due to the downward flow of the wastewater and the temperature difference between ambient and the interior of the trickling filter.  Forced ventilation system may be designed as part of the trickling filter

\item Word trickling “filter” is a misnomer - no filtration is involved
\item Advantage includes process simplicity and lower costs
\item Disadvantage include BOD removal efficiency of only about 80-85%
\item The media may be rock, slag, coal, bricks, redwood blocks, molded plastic, or any other sound durable material.
\item The media depth ranges from about three to eight feet for rock media trickling filters and 15 to 30 feet for synthetic media.
\item The media needs to be uniformly sized and have adequate empty spaces (voids) to ensure maintaining aerobic condition necessary for the survival of biomass.  

\item Pre-fabricated (synthetic) media - similar to the one shown below, has an advantage over the "dumped" type media such as lava rock of providing a greater surface area per volume upon which the zoologeal film may grow while providing ample void space for the free circulation of air.

\item Sometimes, due to inadequate hydraulic loading, portions of the zoogleal layer may become too thick and oxygen cannot penetrate its full depth, causing odor issues.

\end{itemize}


\section{Trickling Filter Recirculation}\index{Trickling Filter Recirculation}

Recirculation - where a portion of the treated wastewater is returned back as the feed to the TF.  The parameter recirculation ratio is calculated to quantify the recirculation flow.  Recirculation ratio is a ratio of the recirculated flow - Q$_R$ to the influent flow Q$_I$. Recirculation ratios typically vary from 0.5 to 4.
\begin{center}
\includegraphics[scale=0.27]{TricklingFilterR1RecircRatioFormula}
\end{center}
Recirculation is beneficial for the following reasons:
\begin{itemize}
\item It improves the removal efficiency by increasing the contact time of the zoogleal layer with the wastewater
\item During low flows it prevents the trickling filter from drying out
\item It dilutes any toxic loadings
\item It promotes oxygen transfer and reduces ponding
\item The increased hydraulic loading promotes uniform sloughing, prevents ponding, improves ventilation through the filter and reduces potential for snail and filter fly breeding
\end{itemize}
\section{Operation of Multiple Trickling Filters}\index{Operation of Multiple Trickling Filters}

Multiple trickling filters can be operated in series or in parallel:\\
\begin{itemize}
\item Series operation in which the flow from one flows into the next.
\begin{center}
\includegraphics[scale=1]{TricklingFilterR1Series}
\end{center}  
\begin{itemize}
\item For high strength loading and for nitrification
\end{itemize}
\item Parallel operation in which the trickling filters that are operated side by side.
\begin{center}
\includegraphics[scale=1]{TricklingFilterR1Parallel}
\end{center}
\begin{itemize}
\item For winter operation - prevents freezing in the TF
\end{itemize}
\end{itemize}

\section{Parameters for monitoring and operating trickling filters}\index{Parameters for monitoring and operating trickling filters}



\begin{enumerate}
\item Hydraulic loading is expressed as gpd/$ft^2$
\item BOD Removal (\%)
\item Organic loading lbs BOD/day/ 1000 cu ft
\item Recirculation ratio
\end{enumerate}


\section{Classifying Trickling Filters}\index{Classifying Trickling Filters}			
Trickling filters are classified according to the hydraulic and organic loading applied to the  filter\\

\subsection{Low-rate filter}\index{Low-rate filter}

\begin{itemize}
\item The standard rate or low rate trickling filters (LRTF) are relatively simple treatment units that normally produce a consistent effluent quality even with varying influent strength
\item They are generally not provided with recirculation of effluent
\item Depending upon the dosing system, wastewater is applied intermittently with rest periods which generally do not exceed five minutes at the designed rate of waste flow. 
\item While there is some unloading or sloughing of solids at all times, the major unloadings usually occur several times a year for comparatively short Periods of time.
\end{itemize}

     Hydraulic loading is 25 - 100 gal/day/sq. ft\\
     BOD Removal (\%) 	50 – 80\%\\
     Organic loading is 5 - 25 lbs BOD/day/1000 cu ft\\

\subsection{High-rate filter}\index{High-rate filter}

\begin{itemize}
\item The most important element of a high-rate trickling filter is the provision where a part of the settled treated effluent is pumped to the PST or to the filter.  This is termed as \textbf{Recirculation}
\item High-rate filters are usually characterized by higher hydraulic and organic loadings than low-rate filters
\item The higher BOD loading is accomplished by applying a larger volume of waste per unit surface area of the filter.  \item As a result of the higher flow velocities a more continuous and uniform sloughing of excess zoogleal growth occurs
\end{itemize}

     Hydraulic loading is 100 - 1000 gal/day/sq. ft\\
     BOD Removal (\%) 	65 - 85\%\\
     Organic loading is 25 - 100 lbs BOD/day/ 1000 cu ft\\
			
\subsection{Roughing filter}\index{Roughing filter}

\begin{itemize}
\item Roughing filters are high rate type filters designed with plastic packing
\item In most cases roughing filters are used to treat wastewater prior to secondary treatment
\item One of the advantages of roughing filter is low energy requirement for BOD removal of high strength wastewaters as compared to activated sludge process because the energy required is only for pumping the influent wastewater and recirculation flows
\end{itemize}



\section{Trickling Filter Operational Issues}\index{Trickling Filter Operational Issues}

\subsection{Ponding}\index{Ponding}


If the voids in the media get plugged, flow can collect on the surface in ponds.
Correction:
spraying the surface with high pressure water stream
stopping a rotary distributor over the ponded area
hand-stir the media or open the voids
dose the filter with chlorine for several hours

\subsection{Odors}\index{Odors}


\begin{itemize}
\item Corrective measures should be taken immediately if foul odors develop
\item presence of foul odors indicates anaerobic conditions are predominant
\item Check the under drain system for obstructions or heavy biological growths
\item increase the recirculation rate to provide more oxygen to the filter bed and increase sloughing 
\item keep slime growths off of sidewalks and inside walls of the filter to reduce the odor
\end{itemize}

\subsection{Trickling Filter Flies - Psychoda}\index{Trickling Filter Flies - Psychoda}

\begin{itemize}
\item tiny, gnat-size filter fly, or Psychoda - primary nuisance insect
\item Correction methods include
\begin{itemize}
\item Increase recirculation rate
\item keep orifice openings clear
\item apply insecticides to filter walls
\item dose filter with chlorine
\item keep weeds and tall grass cut around filter
\end{itemize}
\end{itemize}


\subsection{Cold weather problems}\index{Cold weather problems}

\begin{itemize}
\item ice can form on the media of the filter
\item Correction methods include:
\begin{itemize}
\item decrease recirculation to the filter (influent is usually warmer than recycled flows)
\item construct wind screens
\item operate two-stage filters in parallel rather than in series
\end{itemize}
\end{itemize}



\newpage
\section*{Chapter Assessment}
\begin{tcolorbox}[breakable, enhanced,
colframe=blue!25,
colback=blue!10,
coltitle=blue!20!black,  
title= Chapter Assessment]

\begin{enumerate}
\item  Compared to activated sludge, trickling filters are sturdy work units not easily up set by shock loads. \\

a. True \\
b. False \\


\item  Ponding on the surface of trickling filters is almost always caused by plugging of the underdrains immediately below the point of ponding \\

a. True\\
b. False\\


\item  Zoogleal mass sheared of from the trickling filter media and carried out as part of the effluent flow is termed as sloughing \\

a. True \\
b. False \\


\item  For multiple trickling filter operation, the parallel mode is more suitable than the series mode for treating influent wastewater with a high BOD content \\

a. True \\
b. False \\


\item  Synthetic /pre-manufactured media used in the biofilter has the advantage of providing a greater surface area per volume upon which the zoologeal film may grow while providing ample void space for the free circulation of air. \\

a. True \\
b. False \\

\item  A good reason to run a trickling filter in the parallel mode of operation is: \\

 a. When the weather is warm \\
 b. When the BOD loading is high \\
 c. When the BOD loading is low \\
 d. To prevent filter flies \\


\item  A pan test should be done monthly on a trickling filter to: \\

 a. Check oil quality in the distributor bearing \\
 b. Check sewage distribution on filter \\
 c. To check flow rates through the media \\
 d. All the above \\
 e. None of the above \\


\item  A problem associated with trickling filters is \\

 a. Bulking \\
 b. Protozoans \\
 c. Ponding \\
 d. Liquefaction \\


\item  A trickling filter or activated sludge process causes nitrification. which is \\

 a. Conversion of nitrogen to nitrate \\
 b. Conversion of nitrogen to ammonia \\
 c. Conversion of nitrate to nitrogen \\
 d. Conversion of ammonia to nitrate and nitrite nitrogen \\


\item  Most common trickling filter operational control method is: \\

 a. Sloughing control \\
 b. Recirculation \\
 c. Sludge removal \\
 d. Distributor arm speed \\
 
 \end{enumerate}
 \end{tcolorbox}