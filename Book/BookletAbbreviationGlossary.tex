\documentclass{article}


\usepackage{verbatim}


\usepackage{tocloft}


\usepackage{graphicx} % Required for including pictures
\graphicspath{{Pictures/}} % Specifies the directory where pictures are stored

\usepackage{lipsum} % Inserts dummy text

\usepackage{tikz} % Required for drawing custom shapes

\usepackage[english]{babel} % English language/hyphenation

\usepackage[shortlabels]{enumitem}


% \usepackage{enumitem} % Customize lists
\setlist{nolistsep} % Reduce spacing between bullet points and numbered lists

\usepackage{booktabs} % Required for nicer horizontal rules in tables

\usepackage{xcolor} % Required for specifying colors by name
\definecolor{ocre}{RGB}{243,102,25}
\definecolor{golden}{RGB}{252, 29, 3}

 % Define the orange color used for highlighting throughout the book

%----------------------------------------------------------------------------------------
%	MARGINS
%----------------------------------------------------------------------------------------

\usepackage{geometry} % Required for adjusting page dimensions and margins

\geometry{
	paper=a4paper, % Paper size, change to letterpaper for US letter size
	top=3cm, % Top margin
	bottom=3cm, % Bottom margin
	left=3cm, % Left margin
	right=3cm, % Right margin
	headheight=14pt, % Header height
	footskip=1.4cm, % Space from the bottom margin to the baseline of the footer
	headsep=10pt, % Space from the top margin to the baseline of the header
	%showframe, % Uncomment to show how the type block is set on the page
}




\usepackage[normalem]{ulem}
\usepackage{amsmath}
\usepackage[english]{babel}
\usepackage{graphicx}
\usepackage{tabulary}
\usepackage{tabularx}
\usepackage{cancel}
\usepackage{pagecolor}
\usepackage{afterpage}
\usepackage{soul}
\usepackage{fixltx2e}
\usepackage[utf8]{inputenc}
\usepackage{siunitx} %degrees for Laboratory
\usepackage{pdflscape} %sidescape figure in Laboratory
\usepackage{float}
\usepackage{xcolor}
\usepackage{framed}
\usepackage{mdframed}
\usepackage{soul}
%\textsubscript{this}
\usepackage{lastpage}
\usepackage[utf8]{inputenc}
\usepackage{ifthen}
\usepackage{amsmath}
\usepackage{fancyhdr}
\usepackage[document]{ragged2e}
% \usepackage[margin=1in,top=1.2in,headheight=57pt,headsep=0.1in]{geometry}
\usepackage{fancyhdr}
\usepackage{caption}
\usepackage{subcaption}
%Chapter 2
\usetikzlibrary{calc}
\usetikzlibrary{arrows}
\usepackage{rotating}%for sidewaysfigure
\usepackage[final]{pdfpages}
\usepackage{tcolorbox}
%\usepackage[dvipsnames]{xcolor}
\usepackage{colortbl}
\usepackage{chemfig}
\usepackage{lscape}
\usepackage{wrapfig}
\usepackage[final]{pdfpages}
\usepackage{setspace}
%\usepackage[table]{xcolor}
%----------------------------------------------------------------------------------------
%	FONTS
%----------------------------------------------------------------------------------------

\usepackage{avant} % Use the Avantgarde font for headings
%\usepackage{times} % Use the Times font for headings
\usepackage{mathptmx} % Use the Adobe Times Roman as the default text font together with math symbols from the Sym­bol, Chancery and Com­puter Modern fonts

\usepackage{microtype} % Slightly tweak font spacing for aesthetics
\usepackage[utf8]{inputenc} % Required for including letters with accents
\usepackage[T1]{fontenc} % Use 8-bit encoding that has 256 glyphs

%----------------------------------------------------------------------------------------
%	BIBLIOGRAPHY AND INDEX
%----------------------------------------------------------------------------------------

\usepackage[style=numeric,citestyle=numeric,sorting=nyt,sortcites=true,autopunct=true,babel=hyphen,hyperref=true,abbreviate=false,backref=true,backend=biber]{biblatex}
\addbibresource{bibliography.bib} % BibTeX bibliography file
\defbibheading{bibempty}{}

\usepackage{calc} % For simpler calculation - used for spacing the index letter headings correctly
\usepackage{makeidx} % Required to make an index
\makeindex % Tells LaTeX to create the files required for indexing



%----------------------------------------------------------------------------------------
%	MAIN TABLE OF CONTENTS
%----------------------------------------------------------------------------------------

%----------------------------------------------------------------------------------------
%	MAIN TABLE OF CONTENTS
%----------------------------------------------------------------------------------------

\usepackage{titletoc} % Required for manipulating the table of contents

\contentsmargin{0cm} % Removes the default margin

% Part text styling (this is mostly taken care of in the PART HEADINGS section of this file)
\titlecontents{part}
	[0cm] % Left indentation
	{\addvspace{20pt}\bfseries} % Spacing and font options for parts
	{}
	{}
	{}

% Chapter text styling
\titlecontents{chapter}
	[1.25cm] % Left indentation
	{\addvspace{12pt}\large\sffamily\bfseries} % Spacing and font options for chapters
	{\color{ocre!60}\contentslabel[\Large\thecontentslabel]{1.25cm}\color{ocre}} % Formatting of numbered sections of this type
	{\color{ocre}} % Formatting of numberless sections of this type
	{\color{ocre!60}\normalsize\;\titlerule*[.5pc]{.}\;\thecontentspage} % Formatting of the filler to the right of the heading and the page number

% Section text styling
\titlecontents{section}
	[1.25cm] % Left indentation
	{\addvspace{3pt}\sffamily\bfseries} % Spacing and font options for sections
	{\contentslabel[\thecontentslabel]{1.25cm}} % Formatting of numbered sections of this type
	{} % Formatting of numberless sections of this type
	{\hfill\color{black}\thecontentspage} % Formatting of the filler to the right of the heading and the page number

% Subsection text styling
\titlecontents{subsection}
	[1.25cm] % Left indentation
	{\addvspace{1pt}\sffamily\color{golden}} % Spacing and font options for subsections
	{\contentslabel[\thecontentslabel]{1.25cm}} % Formatting of numbered sections of this type
	{} % Formatting of numberless sections of this type
	{\ \titlerule*[.5pc]{.}\;\thecontentspage} % Formatting of the filler to the right of the heading and the page number
	
% Subsection text styling
\titlecontents{subsubsection}
	[1.25cm] % Left indentation
	{\addvspace{1pt}\sffamily\color{ocre}} % Spacing and font options for subsections
	{\contentslabel[\thecontentslabel]{1.25cm}} % Formatting of numbered sections of this type
	{} % Formatting of numberless sections of this type
	{\ \titlerule*[.5pc]{.}\;\thecontentspage} % Formatting of the filler to the right of the heading and the page number

% Figure text styling
\titlecontents{figure}
	[1.25cm] % Left indentation
	{\addvspace{1pt}\sffamily\small} % Spacing and font options for figures
	{\thecontentslabel\hspace*{1em}} % Formatting of numbered sections of this type
	{} % Formatting of numberless sections of this type
	{\ \titlerule*[.5pc]{.}\;\thecontentspage} % Formatting of the filler to the right of the heading and the page number

% Table text styling
\titlecontents{table}
	[1.25cm] % Left indentation
	{\addvspace{1pt}\sffamily\small} % Spacing and font options for tables
	{\thecontentslabel\hspace*{1em}} % Formatting of numbered sections of this type
	{} % Formatting of numberless sections of this type
	{\ \titlerule*[.5pc]{.}\;\thecontentspage} % Formatting of the filler to the right of the heading and the page number
















































%\usepackage{tikz}
%\usepackage[utf8]{inputenc}
%\usepackage{graphicx}
%\usepackage{fancyhdr}
%\usepackage{microtype}
%\usepackage{geometry}
%\usepackage{xcolor} % Required for specifying colors by name
%\definecolor{ocre}{RGB}{243,102,25} % Define the orange color used for highlighting throughout the book
%
%
%\usepackage{amsfonts, amsmath, amssymb, amsthm}
%\usepackage[T1]{fontenc}
%

\newcommand{\water}{Acronyms \& Glossary}

\title{\water}
\author{Shabbir Basrai}
\date{ }

\usepackage{titletoc}

% Section text styling
\titlecontents{section}
	[1.25cm] % Left indentation
	{\addvspace{3pt}\sffamily\bfseries} % Spacing and font options for sections
	{\contentslabel[\thecontentslabel]{1.25cm}} % Formatting of numbered sections of this type
	{} % Formatting of numberless sections of this type
	{} % Formatting of the filler to the right of the heading and the page number


\renewcommand{\sectionmark}[1]{\markright{\sffamily\normalsize\thesection\hspace{5pt}#1}{}} % Styling for the current section in the header

\fancyhf{} % Clear default headers and footers
\fancyhead[LE,RO]{\sffamily\normalsize\thepage} % Styling for the page number in the header
\fancyhead[LO]{\rightmark} % Print the nearest section name on the left side of odd pages
\fancyhead[RE]{\leftmark} % Print the current chapter name on the right side of even pages
%\fancyfoot[C]{\thepage} % Uncomment to include a footer

\renewcommand{\headrulewidth}{0.5pt} % Thickness of the rule under the header

\fancypagestyle{plain}{% Style for when a plain pagestyle is specified
	\fancyhead{}\renewcommand{\headrulewidth}{0pt}%
}

% Removes the header from odd empty pages at the end of chapters
\makeatletter
\renewcommand{\cleardoublepage}{
\clearpage\ifodd\c@page\else
\hbox{}
\vspace*{\fill}
\thispagestyle{empty}
\newpage
\fi}




\begin{document}
\linespread{1.5} 
%\chapterimage{ChapterImageMath} % Table of contents heading image
%
%\pagestyle{empty} % Disable headers and footers for the following pages
%
%\tableofcontents % Print the table of contents itself
%
%\cleardoublepage % Forces the first chapter to start on an odd page so it's on the right side of the book
%
%\pagestyle{fancy} % Enable headers and footers again

%**********************************************************************
%**********************************************************************
\newcommand{\thechapterimage}{CoverSheetR2}
%**********************************************************************
%**********************************************************************
\begin{titlepage}
\pagestyle{empty} % Disable headers and footers for the following pages
\begin{tikzpicture}[remember picture,overlay]
\node at (current page.north west)
{\begin{tikzpicture}[remember picture,overlay]


\node[anchor=north west,inner sep=0pt] (\Gm@lmargin,-9cm) {\includegraphics[width=\paperwidth]{\thechapterimage}};

%\draw[anchor=west] (\Gm@lmargin,-10cm) node [line width=5pt,rounded corners=15pt,draw=ocre,fill=white,fill opacity=0.5,inner sep=15pt]{\strut\makebox[22cm]{}};

%\node[anchor=north west,inner sep=0pt] at (0,-17) {\includegraphics[width=\paperwidth]{WWTB.png}};
%**********************************************************************
%**********************************************************************
\draw[anchor=east] (\Gm@lmargin+10.9 cm,-21.95cm) node {\sffamily\bfseries\color{white}\LARGE{\water}};
%**********************************************************************
%**********************************************************************
\draw[anchor=west] (\Gm@lmargin-1.3cm,-28cm) node {\sffamily\bfseries\color{black}\small{Revision Date: May 2022}};

\end{tikzpicture}};
\end{tikzpicture}
%{\addvspace{24pt}\large\sffamily\bfseries}
%    \centering
%    \vfill
%    {\bfseries\Large
%       \maketitle
%        \vskip2cm
%        A. Uthor\\
%    }    
%    \vfill
%  
%  \includegraphics[width=\linewidth]{ChapterImageMath}
%    \includegraphics[width=4cm]{OCSDLogo.png} % also works with logo.pdf
%    \vfill
%    \vfill
\end{titlepage}

\renewcommand{\cftsecfont}{\sffamily\bfseries}
\renewcommand{\cftsubsecfont}{\sffamily\bfseries}
%\renewcommand{\contentsname}{\sffamily\color{ocre}\Huge}{TABLE OF CONTENTS}
%\renewcommand{\contentsname}{\sffamily\color{ocre}{Table of Contents}} 
%\tableofcontents


\newpage
%\chapterimage{Preliminary.jpg} % Chapter heading image

%\chapter{PreliminaryTreatment}
% \begin{enumerate}[1.]
% 	\definecolor{shadecolor}{RGB}{200, 200, 240}

% 	%%%%%%%%%%%
% 	% LEVEL 2 %
% 	%%%%%%%%%%%

% 	\begin{snugshade*}
% 		\item \noindent\textsc{Wastewater Constituents}%$$$$$$$$$$$$$$$$$$$$%
% 	\end{snugshade*}
% 	Solids, organic matter, nutrients, pathogens and oil \& grease are the main target constituents of wastewater treatment operations.
% 	\begin{enumerate}[A.]%___________%
% 			\definecolor{shadecolor}{RGB}{225, 235, 235}

				%%%%%%%%%%%
				% LEVEL 3 %
				%%%%%%%%%%%
		% \begin{snugshade*}
		% 	\item \noindent\textsc{Organic Matter}%###############################%
		% \end{snugshade*}

\newpage
\vfill
\begin{center}
\Huge{ACRONYMS}
\end{center}
\vfill

AA - Activated alumina
\vspace{0.3cm}\\
ABR - Anaerobic baffled reactor - improved septic tank with baffles
\vspace{0.3cm}\\
ABR:  Anaerobic baffled reactor:  improved septic tank with baffles
\vspace{0.3cm}\\
AC:  Alternating current
\vspace{0.3cm}\\
ACP - Anaerobic contact process
\vspace{0.3cm}\\
ACP:  Anaerobic contact process
\vspace{0.3cm}\\
AD - Anaerobic digestion
\vspace{0.3cm}\\
AD:  Anaerobic digestion
\vspace{0.3cm}\\
ADI:  Acceptable daily intake
\vspace{0.3cm}\\
ADWF:  Average Dry Weather Flow
\vspace{0.3cm}\\
AF - Anaerobic filters
\vspace{0.3cm}\\
Ag:  Silver
\vspace{0.3cm}\\
AOC:  Assimilable Organic Carbon
\vspace{0.3cm}\\
AOS:  Adult onsite exposure
\vspace{0.3cm}\\
AQMD:  Air Quality Management District
\vspace{0.3cm}\\
As - Arsenic
\vspace{0.3cm}\\
AS:  Activated Sludge
\vspace{0.3cm}\\
As:  Arsenic
\vspace{0.3cm}\\
ATS - Aerobic treatment system
\vspace{0.3cm}\\
ATS:  Aerobic treatment system
\vspace{0.3cm}\\
AWT, AWWT:  Advanced Wastewater Treatment
\vspace{0.3cm}\\
AWWA:  American Water Works Association
\vspace{0.3cm}\\
AWWARF:  American Water Works Association Research Foundation
\vspace{0.3cm}\\
AWWF:  Average Wet Weather Flow
\vspace{0.3cm}\\
BAC:  Biological Activated Carbon (water treatment)
\vspace{0.3cm}\\
BACT:  Best Available Control Technology (Air Quality)
\vspace{0.3cm}\\
BAF:  Biological Aerated Filter (wastewater treatment)
\vspace{0.3cm}\\
BATNEEC:  Best Available Techniques Not Entailing Excessive Costs
\vspace{0.3cm}\\
BHP:  Brake Horse Power
\vspace{0.3cm}\\
BMP:  Best Management Practices
\vspace{0.3cm}\\
BOD :  Biochemical Oxygen Demand
\vspace{0.3cm}\\
BOD or BOD5 - Biological oxygen demand (measured for five days)
\vspace{0.3cm}\\
BOD:  Biochemical Oxygen on Demand:  
\vspace{0.3cm}\\
BOD:  Biological oxygen demand
\vspace{0.3cm}\\
BOD5:  Biochemical Oxygen Demand (over a 5 day period)
\vspace{0.3cm}\\
CAA:  Clean air act (EPA)
\vspace{0.3cm}\\
CBOD:  Carbonaceous biochemical oxygen demand 
\vspace{0.3cm}\\
Cd:  Cadmium
\vspace{0.3cm}\\
CFC:  Chlorofluorocarbons
\vspace{0.3cm}\\
CFM:  Cubic feet per minute
\vspace{0.3cm}\\
CFR:  Code of Federal Regulations
\vspace{0.3cm}\\
CFS:  Cubic feet per second
\vspace{0.3cm}\\
CFU - Colony-forming unit
\vspace{0.3cm}\\
cfu:  Colony Forming Unit (Microbiology)
\vspace{0.3cm}\\
CHP:  Combined Heat and Power
\vspace{0.3cm}\\
CIP:  Capital Improvement Program
\vspace{0.3cm}\\
CJD:  Creutzfeld-Jakob's Disease
\vspace{0.3cm}\\
Cl$_2$:  Chlorine
CN:  Cyanide
\vspace{0.3cm}\\
COD: Chemical oxygen demand
\vspace{0.3cm}\\
COD - Controlled open defecation
\vspace{0.3cm}\\
COD:  Chemical oxygen demand 
\vspace{0.3cm}\\
Cr: Chromium
\vspace{0.3cm}\\
CSI:  Conveyance system improvement
\vspace{0.3cm}\\
CSO: - Combined sewer overflow 
\vspace{0.3cm}\\
LTCP:  Long-term control plan
\vspace{0.3cm}\\
CSO:  Combined sewer overflow
\vspace{0.3cm}\\
Cu: Copper
\vspace{0.3cm}\\
CVOC:  Chlorinated volatile organic compound
\vspace{0.3cm}\\
CW - Constructed wetland
\vspace{0.3cm}\\
DAFT:  Dissolved air flotation thickener
\vspace{0.3cm}\\
DAS:  Data Acquisition System
\vspace{0.3cm}\\
DBP:  Disinfection By-Product
\vspace{0.3cm}\\
DEWATS - Decentralized wastewater treatment system
\vspace{0.3cm}\\
DEWATS:  Decentralized wastewater treatment system
\vspace{0.3cm}\\
DFM:  Decennial flow monitoring
\vspace{0.3cm}\\
DMR:  Discharge Monitoring Report
\vspace{0.3cm}\\
DNA:  Deoxyribonucleic acid
\vspace{0.3cm}\\
DO: Dissolved Oxygen
\vspace{0.3cm}\\
DOC:  Dissolved organic carbon
\vspace{0.3cm}\\
EBCT:  Empty bed contact time
\vspace{0.3cm}\\
EC - Electrical conductivity
\vspace{0.3cm}\\
EDI: Electro deionization
\vspace{0.3cm}\\
Eh:  Redox potential
\vspace{0.3cm}\\
EIA:  Environmental impact assessment
\vspace{0.3cm}\\
EIS:  Environmental impact statement
\vspace{0.3cm}\\
EPA:  United States Environmental Protection Agency
\vspace{0.3cm}\\
EQO:  Environmental quality objective
\vspace{0.3cm}\\
EQS:  Environmental quality standard
\vspace{0.3cm}\\
ESA:  Endangered Species Act
\vspace{0.3cm}\\
ESA:  Environmentally sensitive area
\vspace{0.3cm}\\
ESDV:  Emergency shutdown valves
\vspace{0.3cm}\\
$^{\circ}$F:  Fahrenheit
\vspace{0.3cm}\\
F/M Ratio:  Food to microorganism ratio
\vspace{0.3cm}\\
FAC:  Florida Administrative Code
\vspace{0.3cm}\\
FAO:  Food and Agriculture Organization of the United Nations
\vspace{0.3cm}\\
FAQ:  Frequently asked questions
\vspace{0.3cm}\\
FC:  Fecal coliforms
\vspace{0.3cm}\\
FeCl2: Ferrous chloride
\vspace{0.3cm}\\
FeCl3:  Ferric chloride
\vspace{0.3cm}\\
FECR:  Fecal egg count reduction
\vspace{0.3cm}\\
FOG: Fats, Oils, \& Grease
\vspace{0.3cm}\\
FS:  Fecal (or fecal) sludge
\vspace{0.3cm}\\
FSM:  Fecal (or fecal) sludge management
\vspace{0.3cm}\\
FSSM:  Fecal sludge and septage management
\vspace{0.3cm}\\
FSTP:  Fecal sludge treatment plant
\vspace{0.3cm}\\
FTI:  Fecally transmitted infections
\vspace{0.3cm}\\
GAC:  Granular Activated Carbon
\vspace{0.3cm}\\
GAP:  Good agricultural practices
\vspace{0.3cm}\\
GC:  Gas chromatography
\vspace{0.3cm}\\
GCMS:  Gas chromatograph + mass spectrometer
\vspace{0.3cm}\\
GFD:  Gals per foot of membrane per day 
\vspace{0.3cm}\\
GHG:  Greenhouse gases
\vspace{0.3cm}\\
GIS:  Geographical information system
\vspace{0.3cm}\\
GLV:  Guideline value (water quality standards)
\vspace{0.3cm}\\
GMO:  Genetically modified organism
\vspace{0.3cm}\\
gpcd:  Gallons per capita per day
\vspace{0.3cm}\\
GPD:  Gallons Per Day
\vspace{0.3cm}\\
gpcd:  Gallons per capita per day
\vspace{0.3cm}\\
gped:  Gallons per employee per day
\vspace{0.3cm}\\
GPM:  Gallons per minute
\vspace{0.3cm}\\
GPS - Global positioning system
\vspace{0.3cm}\\
GSA:  Gould Sludge Age
\vspace{0.3cm}\\
GSF - Global sanitation fund, closest page is Water Supply and Sanitation Collaborative Council
\vspace{0.3cm}\\
GSI:  Green stormwater Infrastructure
\vspace{0.3cm}\\
HACCP:  Hazard analysis and critical control points
\vspace{0.3cm}\\
HC:  hydrocarbons
\vspace{0.3cm}\\
HDPE:  High density polyethylene
\vspace{0.3cm}\\
HEDF:  Human excreta derived fertilizer
\vspace{0.3cm}\\
HIA:  Health impact assessment
\vspace{0.3cm}\\
HMI:  Human machine interface  
\vspace{0.3cm}\\
HP:  Horsepower
\vspace{0.3cm}\\
HRWS: Human right to water and sanitation
\vspace{0.3cm}\\
HWF - Handwashing facility
\vspace{0.3cm}\\
I/I:  Infiltration/Inflow
\vspace{0.3cm}\\
I/O:  Input/Output
\vspace{0.3cm}\\
IPC :  Inclined plate clarifier
\vspace{0.3cm}\\
IPC:  Integrated pollution control
\vspace{0.3cm}\\
IPPC:  Integrated pollution prevention and control
\vspace{0.3cm}\\
IPS:  Influent pump station
\vspace{0.3cm}\\
IWA:  International Water Association
\vspace{0.3cm}\\
IWSA:  International Water Supply Association
\vspace{0.3cm}\\
IWW:  Industrial wastewater
\vspace{0.3cm}\\
JTU:  Jackson Turbidity Unit
\vspace{0.3cm}\\
kL (or Kl):  Kilo liters (or 1000 liters, same as 1 cubic meter)
\vspace{0.3cm}\\
KM:  Knowledge management
\vspace{0.3cm}\\
L:  Liter
\vspace{0.3cm}\\
LDPE:  Low density polyethylene
\vspace{0.3cm}\\
LMH:  Liters of permeate per square meter of membrane per hour
\vspace{0.3cm}\\
lpcd:  Liters per capita per day (liters per person per day)
\vspace{0.3cm}\\
M\&E:  Monitoring and evaluation
\vspace{0.3cm}\\
MAC:  Maximum admissible concentration
\vspace{0.3cm}\\
MB:  Mixed Bed
\vspace{0.3cm}\\
MBBR:  Moving bed biofilm reactor
\vspace{0.3cm}\\
MBR:  Membrane bioreactor
\vspace{0.3cm}\\
MCC:  Motor control center
\vspace{0.3cm}\\
MCL:  Maximum contaminant level
\vspace{0.3cm}\\
MCRT:  Mean cell residence time
\vspace{0.3cm}\\
MFC:  Microbial fuel cell
\vspace{0.3cm}\\
MFI:  Microfinance institution
\vspace{0.3cm}\\
mg - Milligram
\vspace{0.3cm}\\
MG/L:  Milligrams Per Liter
\vspace{0.3cm}\\
mg:  Milligram
\vspace{0.3cm}\\
MG:  Million Gallons
\vspace{0.3cm}\\
MGD:  Million Gallons Per Day
\vspace{0.3cm}\\
MHM:  Menstrual hygiene management; closest article is Menstrual hygiene day
\vspace{0.3cm}\\
ML:  Megaliter or 1 million liter or 1000 cubic meters
\vspace{0.3cm}\\
mld:  Megalitres per day
\vspace{0.3cm}\\
MLD:  Million liters per day
\vspace{0.3cm}\\
MLSS:  Mixed liquor suspended solids
\vspace{0.3cm}\\
MLVSS:  Mixed liquor Volatile suspended solids
\vspace{0.3cm}\\
MOA:  Memorandum of Agreement
\vspace{0.3cm}\\
MOR:  Monthly Operating Report
\vspace{0.3cm}\\
MoU:  Memorandum of understanding
\vspace{0.3cm}\\
MSW:  Municipal solid waste
\vspace{0.3cm}\\
MTBE:  Methyl-tert-butyl ether
\vspace{0.3cm}\\
NACWA:  National Association of Clean Water Agencies
\vspace{0.3cm}\\
NAPL:  Non Aqueous phase liquid
\vspace{0.3cm}\\
NBS:  National Bureau of Standards.
\vspace{0.3cm}\\
ND:  Not detected
\vspace{0.3cm}\\



NEC:  National Electrical Code.
\vspace{0.3cm}\\



NEMA:  National Electrical Manufacturers Association Standard
\vspace{0.3cm}\\
NIMBY:  Not In My Back Yard
\vspace{0.3cm}\\

NF:  Nanofiltration
\vspace{0.3cm}\\
NGO:  Non-governmental organization
\vspace{0.3cm}\\

NH$_3$-N:  Ammonia Nitrogen
\vspace{0.3cm}\\
NIMBY:  Not In My Back Yard
\vspace{0.3cm}\\
NIST:  National Institute of Standards and Technology.
\vspace{0.3cm}\\
NO2-N:  Nitrite Nitrogen 
\vspace{0.3cm}\\
NO$_3$-N:  Nitrate Nitrogen
\vspace{0.3cm}\\
NOx:  Nitrogen Oxide
\vspace{0.3cm}\\
NPDES:  National Pollutant Discharge Elimination System
\vspace{0.3cm}\\
NPSH:  Net Positive Suction Head
\vspace{0.3cm}\\
NPT:  National Pipe Thread Standard
\vspace{0.3cm}\\

NRW:  Non-revenue water
\vspace{0.3cm}\\
NSS:  Non-sewered sanitation (similar term to fecal sludge management)
\vspace{0.3cm}\\
NTU:  Nephelometer Turbidity Units
\vspace{0.3cm}\\
O \& G:  Oil and Grease
\vspace{0.3cm}\\
O\&M:  Operation and maintenance
\vspace{0.3cm}\\
OD:  Open defecation
\vspace{0.3cm}\\
ODF:  Open defecation free, i.e. a community without open defecation taking place
\vspace{0.3cm}\\

OPEX:  Operating Expenditure on a recurring annual basis
\vspace{0.3cm}\\
OPEX:  Operational expenditure (or operating expense)
\vspace{0.3cm}\\
ORP:  Oxidation reduction potential
\vspace{0.3cm}\\
OSEC:  On-Site electrolytic chlorination
\vspace{0.3cm}\\
OUR:  Oxygen uptake rate
\vspace{0.3cm}\\
P\&ID: Process and Instrumentation Diagram
\vspace{0.3cm}\\
PAC:  Powdered Activated Carbon
\vspace{0.3cm}\\
PAH:  Polycyclic aromatic hydrocarbons
\vspace{0.3cm}\\
PC :  Personal computer
\vspace{0.3cm}\\
PCV:  Prescribed concentration or value
\vspace{0.3cm}\\
pe:  Population Equivalent
\vspace{0.3cm}\\
PFD:  Process flow diagram
\vspace{0.3cm}\\
pH:  Hydrogen potential
\vspace{0.3cm}\\
pH: Potential Hydrogen
\vspace{0.3cm}\\
PLC :  Programmable Logic Controller
\vspace{0.3cm}\\
POTW:  Publicly Owned Treatment Works
\vspace{0.3cm}\\
ppb: Parts Per Billion
\vspace{0.3cm}\\
PPE :  Personal protective equipment  
\vspace{0.3cm}\\
ppm - Parts per million
\vspace{0.3cm}\\
PPP - Public private partnership
\vspace{0.3cm}\\
ppt:  Parts per trillion
\vspace{0.3cm}\\
PRV:  Pressure Reducing Valve (water distribution)
\vspace{0.3cm}\\
PSI:  Pounds Per Square Inch
\vspace{0.3cm}\\
PTE:  Potentially toxic element
\vspace{0.3cm}\\
PVC:  Polyvinyl chloride
\vspace{0.3cm}\\
QA/QC:  Quality assurance / quality control
\vspace{0.3cm}\\
R\&D:  Research and development
\vspace{0.3cm}\\
RAS:  Return activated sludge
\vspace{0.3cm}\\
RBC: Rotating Biological contactor
\vspace{0.3cm}\\
RBF:  Results-based financing, see also Output based aid or Payment by Results
\vspace{0.3cm}\\
RBTS:  Reed Bed Treatment System (wastewater treatment)
\vspace{0.3cm}\\
RCT:  Randomized controlled trial
\vspace{0.3cm}\\
RNA:  Ribonucleic acid
\vspace{0.3cm}\\
RO:  Reverse Osmosis
\vspace{0.3cm}\\
RRR:  Resource Recovery and Reuse
\vspace{0.3cm}\\
RTC:  Real Time Control
\vspace{0.3cm}\\
RTI:  Reproductive tract infection
\vspace{0.3cm}\\
RTTC:  Reinvent the Toilet Challenge, an R\&D funding scheme by the Bill and Melinda Gates Foundation
\vspace{0.3cm}\\
SA:  Sludge Age
\vspace{0.3cm}\\
SAR:  Sodium adsorption ratio
\vspace{0.3cm}\\
SBR:  Sequencing batch reactor
\vspace{0.3cm}\\
SCADA:  Supervisory control and data acquisition
\vspace{0.3cm}\\
SDG:  Sustainable development goal
\vspace{0.3cm}\\
SDG6:  Sustainable Development Goal Number 6: "Clean Water and Sanitation"
\vspace{0.3cm}\\
SDI:  Sludge Density Index
\vspace{0.3cm}\\
SDWA:  Safe Drinking Water Act (US legislation)
\vspace{0.3cm}\\
SHG:  Self-help group
\vspace{0.3cm}\\
SI:  International system of Units
\vspace{0.3cm}\\
SLA:  Service-level agreement
\vspace{0.3cm}\\
SLB:  Service-level benchmarking
\vspace{0.3cm}\\
SME:  Small and Medium Sized Enterprises
\vspace{0.3cm}\\
SMS:  Short message service
\vspace{0.3cm}\\
SOP:  Standard operating procedure
\vspace{0.3cm}\\
SOUR:  Specific Oxygen Uptake Rate
\vspace{0.3cm}\\
Sox:  Sulfur Oxides
\vspace{0.3cm}\\
SPC:  Supervisory process control
\vspace{0.3cm}\\
SRT:  Solids retention time, see also Activated sludge
\vspace{0.3cm}\\
SRT:  Solids Retention Time
\vspace{0.3cm}\\
SS:  Suspended Solids (wastewater treatment)
\vspace{0.3cm}\\
SSSI:  Site of Special Scientific Interest
\vspace{0.3cm}\\
STP:  Sewage treatment plant
\vspace{0.3cm}\\
STW:  Sewage Treatment Works
\vspace{0.3cm}\\
SVI:  Sludge Volume Index
\vspace{0.3cm}\\
T90:  Time at which 90 percent reduction in pathogens is achieved
\vspace{0.3cm}\\
TAD:  Thermophilic Aerobic Digestion (wastewater sludge treatment)
\vspace{0.3cm}\\
TDS:  Total dissolved solids
\vspace{0.3cm}\\
THM:  Trihalomethane
\vspace{0.3cm}\\
ThOD - Theoretical oxygen demand
\vspace{0.3cm}\\
ThOD:  Theoretical oxygen demand
\vspace{0.3cm}\\
TKN: Total Kjeldahl Nitrogen
\vspace{0.3cm}\\
TMDL:  Total Maximum Daily Load
\vspace{0.3cm}\\
TOC:  Total Organic Carbon
\vspace{0.3cm}\\
TOC: Total Organic Carbon
\vspace{0.3cm}\\
TOMP: Toxic Organic Management Plan
\vspace{0.3cm}\\
TP:  Total Phosphorous 
\vspace{0.3cm}\\
TSS:  Total suspended solids 
\vspace{0.3cm}\\
TSS: Total Suspended Solids
\vspace{0.3cm}\\
TTHMs:  Total Trihalomethanes
\vspace{0.3cm}\\
TTO: Total Toxic Organics
\vspace{0.3cm}\\
TWL:  Top Water level
\vspace{0.3cm}\\
UASB:  Upflow anaerobic sludge blanket reactor
\vspace{0.3cm}\\
UCD:  User-centered design
\vspace{0.3cm}\\
UDDT:  Urine-diverting dry toilet
\vspace{0.3cm}\\
UDFT:  Urine Diverting Flush Toilet
\vspace{0.3cm}\\
UDT:  Urine diversion toilet
\vspace{0.3cm}\\
UF: Ultrafiltration
\vspace{0.3cm}\\
UL:  Underwriters Laboratories, Inc
\vspace{0.3cm}\\
umho/cm:  Micromho per centimeter
\vspace{0.3cm}\\
uS/cm:  Microsiemens per centimeter
\vspace{0.3cm}\\
USEPA:  United States Environmental Protection Agency
\vspace{0.3cm}\\
UV:  Ultraviolet
\vspace{0.3cm}\\
UV: Ultraviolet
\vspace{0.3cm}\\
UVGI:  Ultraviolet germicidal irradiation
\vspace{0.3cm}\\
VFA:  Volatile fatty acid
\vspace{0.3cm}\\
VFD:  Variable frequency drive
\vspace{0.3cm}\\
VIP:  Ventilated improved pit latrine
\vspace{0.3cm}\\
VOC:  Volatile organic carbon
\vspace{0.3cm}\\
WAS:  Waste activated sludge
\vspace{0.3cm}\\
WASH or WaSH:  water, sanitation and hygiene
\vspace{0.3cm}\\
WASH2:  Water, sanitation, hygiene and health
\vspace{0.3cm}\\
WC:  Water closet
\vspace{0.3cm}\\
WC:  Water column
\vspace{0.3cm}\\
WEF:  Water-Energy-Food nexus
\vspace{0.3cm}\\
WERF:  Water Environment Research Foundation
\vspace{0.3cm}\\
WHO:  World Health Organization
\vspace{0.3cm}\\
WPM:  Water point mapping
\vspace{0.3cm}\\
WRF:  Water reclamation facility
\vspace{0.3cm}\\
WRRF:  Water recycle and reclamation facility
\vspace{0.3cm}\\
WSH:  Water, sanitation, hygiene
\vspace{0.3cm}\\
WSSCC:  Water Supply and Sanitation Collaborative Council
\vspace{0.3cm}\\
WSUP:  Water and sanitation for the urban poor
\vspace{0.3cm}\\
WTD:  World Toilet Day
\vspace{0.3cm}\\
WTP:  Water treatment plant
\vspace{0.3cm}\\
WWD:  World Water Day
\vspace{0.3cm}\\
WWTF:  Wastewater treatment facility
\vspace{0.3cm}\\
WWTP:  Wastewater treatment plant
\vspace{0.3cm}\\
WYSIWYG:  What you see is what you get
\vspace{0.3cm}\\
Zn:  Zinc


\newpage
\vfill
\begin{center}
\Huge{GLOSSARY}
\end{center}
\vfill
ACID
A substance that: tends to lose a proton, dissolves in water with the formation of hydrogen ions (H+), contains hydrogen which may be replaced by metals to form salts. Highly corrosive.
\vspace{0.3cm}\\
ACIDITY
The capacity of water or wastewater to neutralize bases. Acidity is expressed in milligrams per liter of equivalent calcium carbonate.
\vspace{0.3cm}\\
ACRE FOOT
A volume of water one (1) foot deep and one (1) acre in area, or 43,560 cubic feet.
\vspace{0.3cm}\\
ACTIVATED CARBON
Form of carbon processed to have small, low-volume pores that increase the surface area available for adsorption or chemical reactions.
\vspace{0.3cm}\\
ACTIVATED SLUDGE PROCESS
An aerobic biological treatment process for removing the BOD content of domestic or industrial wastewater.
\vspace{0.3cm}\\
ACTUATOR
Device used to operate a valve using electric, pneumatic or hydraulic means. Often used for remote control or sequencing of valve operations.
\vspace{0.3cm}\\
ADAPTER SPOOL
An extension which is added to a short face-to-face valve, to conform to standard API 6D face-to-face dimensions.
\vspace{0.3cm}\\
AERATION
The process of adding air to wastewater to provide dissolved oxygen for aerobic bacterial treatment, to freshen wastewater (remove septic condition) and to keep solids in suspension.
\vspace{0.3cm}\\
AERATION TANK
A chamber for injecting air into water.
\vspace{0.3cm}\\
AEROBES
Bacteria that must have molecular (dissolved) oxygen (DO) to survive.
\vspace{0.3cm}\\
AEROBIC
A biological environment in which molecular oxygen is present in the aquatic (water) environment. 
\vspace{0.3cm}\\
AEROBIC BACTERIA
Bacteria that requires free (elementary) oxygen for respiration.
\vspace{0.3cm}\\
AIR AERATION
A diffused air activated sludge plant takes air, compresses it, and then discharges the air below the water surface of the aerator through some type of air diffusion device.
\vspace{0.3cm}\\
AIR END
A term referring to the side (or parts) of the pump that come into contact with shop/compressed air or natural gas. This applies to any air operated pump including Air Operated Diaphragm Pumps, air operated piston pumps and air operated drum pumps.
\vspace{0.3cm}\\
AIR LIFT
A type of pump. This device consists of a vertical riser pipe in the wastewater or sludge to be pumped. Compressed air is injected into a tall piece at the bottom of the pipe.   Fine air bubbles mix with the wastewater or sludge to form a mixture lighter than the surrounding water which causes the mixture to rise in the discharge pipe to the outlet. An air-lift pump works like the center of a stand in a percolator coffee pot.
\vspace{0.3cm}\\
AIR TEST
A method of inspecting a sewer pipe for leaks. Inflatable or similar plugs are placed in the line, and the space between these plugs is pressurized with air. A drop in pressure indicates the line or run being tested has leaks. 
\vspace{0.3cm}\\
ALGAE
A class of microscopic plant life that contain chlorophyll, live floating (suspended) in water or are attached to rocks, walls and other surfaces, and grow and multiply through photosynthesis. Algae produce oxygen during sunlight hours, use oxygen during darkness and affect the pH and DO levels in water.
\vspace{0.3cm}\\
ALGAL BLOOM
Sudden, massive growths of algae that develop in lagoons, lakes and reservoirs.
\vspace{0.3cm}\\
ALIQUOT
Portion of a sample. Often an equally divided portion of a sample.
\vspace{0.3cm}\\
ALKALINITY
The capacity of water to neutralize acids, a property imparted by the water’s content of carbonates, bicarbonates, hydroxides, and occasionally borates, silicates, and phosphates.  Typically measured as ppm (mg/l) of CaCO3.
\vspace{0.3cm}\\
ALL WELDED CONSTRUCTION
Pertains to a valve construction in which the body is completely welded and cannot be disassembled and repaired in the field.
\vspace{0.3cm}\\
ANAEROBIC
A biological environment that is deficient in all forms of oxygen, especially molecular oxygen, nitrates and nitrites. The biological decomposition of organic matter in wastewater in the absence of dissolved oxygen is classed as anaerobic.
\vspace{0.3cm}\\
ANAEROBIC DIGESTION
Process of decomposing wastewater solids (complex organic material) by anaerobic bacteria.
\vspace{0.3cm}\\
ANCHOR PIN
A pin welded onto the body of ball valves. This pin aligns the adapter plate and restrains the plate and gear operator from moving while the valve is being operated.
\vspace{0.3cm}\\
ANGLE VALVE
A variation of the globe valve, in which the end connections are at right angles to each other, rather than being inline.
\vspace{0.3cm}\\
ANIONIC FLOCCULANT
Negatively charged flocculant. Used in water treatment to aid solid / liquid separation
\vspace{0.3cm}\\
ANOXIC
A biological environment that is deficient in molecular oxygen, but may contain chemically bound oxygen, such as nitrates and nitrites.
\vspace{0.3cm}\\
ANTISCALENT
Material used to control scale formation in water systems such as boiler or cooling water systems.
\vspace{0.3cm}\\
AOD
Air Operated Diaphragm (pump). These types of pumps are powered by compressed air or gas, making them ideal for hazardous applications such as petroleum based products and other flammable materials. With certain materials of construction, such as steel or conductive plastics, they are easily converted into fully explosion proof (Ex-Proof) pumps. Additionally, they can pull a suction lift and are submersible when installed properly. AOD’s can also handle slurries with solids concentrations up to 30\% and can be run against a closed suction or “dead head” situation.
\vspace{0.3cm}\\
AODD
Air Operated Double Diaphragm (pump). These types of pumps are powered by compressed air or gas, making them ideal for hazardous applications such as petroleum based products and other flammable materials. With certain materials of construction, such as steel or conductive plastics, they are easily converted into fully explosion proof (Ex-Proof) pumps. Additionally, they can pull a suction lift and are submersible when installed properly. AOD’s can also handle slurries with solids concentrations up to 30\% and can be run against a closed suction or “dead head” situation.
\vspace{0.3cm}\\
AQUIFER
A natural underground layer of porous materials usually capable of yielding a supply of water.
\vspace{0.3cm}\\
ARSENIC
A heavy metal commonly regulated by wastewater discharge permits, but not commonly found in industrial wastewaters. Other heavy metals include: Cadmium (Cd), Chromium (Cr), Copper (Cu), Lead (Pb), Nickel (Ni), and Zinc (Zn).
\vspace{0.3cm}\\
ASPHYXIATION
An extreme condition often resulting in death due to a lack of oxygen and excess carbon dioxide in the blood from any cause. 
\vspace{0.3cm}\\
AVAILABLE CHLORINE
The amount of chlorine available in compound chlorine sources compared with that of elemental (liquid or gaseous) chlorine.
\vspace{0.3cm}\\
AVERAGE MONTHLY DISCHARGE LIMITATION
The highest allowable discharge over a calendar month 
\vspace{0.3cm}\\
AVERAGE WEEKLY DISCHARGE LIMITATION
The highest allowable discharge over a calendar week. 
\vspace{0.3cm}\\
BACK WASH
Part of water filter, ion exchange or softener cycle that lifts up media bed to release and wash away dirt and other foulants.
\vspace{0.3cm}\\
BACKFILL
(1) Material used to full in a trench or excavation. (2) The act of filling a trench or excavation, usually after a pipe or some type of structure has been placed in the trench or excavation. 
\vspace{0.3cm}\\
BACKFILL COMPACTION
(1) Tamping, rolling or otherwise mechanically compressing material used as backfill for a trench of excavation. Backfill is compressed to increase its density so that it will support the weight of machinery or other loads after the material is in place. 
\vspace{0.3cm}\\
BACKFLOW
A reverse flow condition, created by a difference in water pressures, which causes water to flow back into the distribution pipes of a potable water supply from any source or sources other than an intended source. Also see BACKSIPHONAGE.
\vspace{0.3cm}\\
BACKFLUSHING
A procedure used to wash settled waste matter off upstream to prevent odors from developing after a main line stoppage has been cleared. 
\vspace{0.3cm}\\
BACKSEAT
A Shoulder on the stem of a valve which seals against a mating surface inside the bonnet to permit replacement, under pressure, of stem seals or packing.
\vspace{0.3cm}\\
BACKSIPHONAGE
A form of backflow caused by a negative or below atmospheric pressure within a water system. Also see BACKFLOW.
\vspace{0.3cm}\\
BACTERIA
Single-celled, living, microscopic organisms which use organic matter for food and produce waste products.
\vspace{0.3cm}\\
BACTERIAL COUNT
Number of bacteria developed under controlled conditions after 25 hours incubation period. In unpolluted waters count is frequently less than 10 per milliliter.
\vspace{0.3cm}\\
BAFFLE
A flat board or plate, deflector, guide or similar device constructed or placed in flowing water, wastewater, or slurry systems to cause more uniform flow velocities, to absorb energy, and to divert, guide, or agitate liquids (water, chemical solutions, slurry).
\vspace{0.3cm}\\
BALL
The spherical closure element of a ball valve.
\vspace{0.3cm}\\
BALL CHECK
A fitting with a small ball that seals against a seat preventing flow in one direction and allowing flow in the other direction.
\vspace{0.3cm}\\
BALL VALVE
A valve using a spherical closure element (ball) which is rotated thru 90$^{\circ}$ to open and close the valve.
\vspace{0.3cm}\\
BALLING
A method of hydraulically cleaning a sewer or storm drain by using the pressure of a water head to create a high cleansing velocity of water around the ball. In normal operation, the ball is restrained by a cable while water washes past the ball at high velocity. Special sewer cleaning balls have an outside tread that causes them to spin or rotate, resulting in a “scrubbing” action of the flowing water along the pipe wall. 
\vspace{0.3cm}\\
BAR RACK
A screen composed of parallel bars, either vertical or inclined, placed in a sewer or other waterway to catch debris. The screenings may be raked from it. 
\vspace{0.3cm}\\
BARREL
(1) The cylindrical part of a pipe that may have a bell on one end. (2) The cylindrical part of a manhole between the cone at the top and the shelf at the bottom. 
\vspace{0.3cm}\\
BASE
A substance that takes up or accepts protons, dissociates in water to produce hydroxyl (OH-) ions, reacts with metals and is corrosive.
\vspace{0.3cm}\\
BEDDING
The prepared base or bottom of a trench or excavation on which a pipe or other underground structure is supported. 
\vspace{0.3cm}\\
BEDDING COMPACTION
(1) Tamping, rolling or otherwise mechanically compressing material used as bedding for a pipe or other underground structure to a density that will support expected loads. (2) Bedding compaction can be expressed as a percentage of the maximum load capacity of the bedding material. (3) Bedding compaction also can be expressed in load capacity or pounds per square foot. 
\vspace{0.3cm}\\
BEDDING GRADE
(1) In a gravity-flow sewer system, pipe bedding is constructed and compacted to the design grade of the pipe. This is usually expressed in a percentage. A 0.5 percent grade would be a drop of one-half of foot per hundred feet of pipe. (2) Bedding grade for a gravity-flow sewer pipe can also be specified as elevation above mean sea level at specific points. 
\vspace{0.3cm}\\
BELL
(1) In pipe fitting, the enlarged female end of a pipe into which the male end fits. (2) In plumbing, the expanded female end of a wiped joint. 
\vspace{0.3cm}\\
BELLEVILLE SPRING
A spring resembling a dished washer, used in some ball valves to push the seats against the ball.
\vspace{0.3cm}\\
BERM
The earthen dike that surrounds ponds, lagoons and containment areas for hazardous material.
\vspace{0.3cm}\\
BEVEL GEAR OPERATOR
Device facilitating operation of a gate or globe valve by means of a set of bevel gears having the axis of the pinion gear at right angles to that of the larger ring gear. The reduction ratio of this gearset determines the multiplication of torque achieved.
\vspace{0.3cm}\\
BRAKE HORSE POWER
It is the actual amount of horsepower being consumed by the pump as measured on a pony brake or dynamometer.
\vspace{0.3cm}\\
BIOCHEMICAL OXYGEN ON DEMAND
An indirect reading of the organic content present in wastewater. Specifically, it refers to the amount of oxygen consumed to biologically degrade the organic material. It’s very expensive to treat, typically requiring a biological treatment technology like activated sludge.
\vspace{0.3cm}\\
BIOCIDE
Chemical substance designed for killing living organisms in water. Often characterized by type of organism killed: bactericide, fungicide or algaecide.
\vspace{0.3cm}\\
BIOLOGICAL OXIDAITON
The process by which bacteria and other types of micro-organisms consume dissolved oxygen and organic substances in wastewater, using the energy released to convert organic carbon into carbon dioxide and cellular material.
\vspace{0.3cm}\\
BIOLOGICAL OXIDATION
The process by which bacteria and other types of microorganisms consume dissolved oxygen and organic substances in biological wastewater.  The energy released is then used to convert organic carbon into carbon dioxide and cellular material. 
\vspace{0.3cm}\\
BIOMASS
A mass or clump of living organisms feeding on wastes in wastewater, dead organisms and other debris. 
\vspace{0.3cm}\\
BIOSOLIDS
Biosolids are stabilized wastewater solids which may be beneficially reused.
\vspace{0.3cm}\\
BIOSOLIDS CAKE
Solid discharge from the apparatus such as a centrifuge or a belt filter press used for dewatering stabilized wastewater solids. 
\vspace{0.3cm}\\
BIT
(1) Cutting blade used in rodding (pipe cleaning) operations. (2) Cutting teeth on the auger head of a sewer boring tool. 
\vspace{0.3cm}\\
BLANK
A bottle containing only dilution water or distilled water, but the sample being tested is not added. Tests are frequently run on a SAMPLE and a BLANK and the differences are compared.
\vspace{0.3cm}\\
BLOCK AND BLEED
The capability of obtaining a seal across the upstream and downstream seat rings of a valve when the body pressure is bled off to atmosphere through blow down valves or vent plugs. Useful in testing for integrity of seat seals and in accomplishing minor repairs under pressure.
\vspace{0.3cm}\\
BLOCKAGE
(1) Partial or complete interruption of flow as a result of some obstruction in a sewer. (2) When a collection system becomes plugged and the flow backs up, “blockage.” 
\vspace{0.3cm}\\
BOD
Biochemical oxygen demand (BOD) represents the amount of oxygen consumed by bacteria and other microorganisms while they decompose organic matter under aerobic (oxygen is present) conditions at a specified temperature.
\vspace{0.3cm}\\
BLOW DOWN VALVE (BDV)
A small ball valve that is installed on the above ground end of an extended drain line. This valve also serves to vent body cavity pressure in the "block and bleed" mode.
\vspace{0.3cm}\\
BODY
The principal pressure containing part of a valve, in which the closure element and seats are located.
\vspace{0.3cm}\\
BODY RELIEF VALVE
A relief valve (optional) installed on ball valves used in liquid service to provide for the relief of excess body pressure caused by thermal expansion.
\vspace{0.3cm}\\
BOLTED BONNET
A bonnet which is connected to a valve body with bolts or studs and nuts.
\vspace{0.3cm}\\
BOLTED CONSTRUCTION
Describes a valve construction in which the pressure shell elements are bolted together, and thus can be taken apart and repaired in the field.
\vspace{0.3cm}\\
BONNET
The top part of a valve, attached to the body, which contains the packing gland, guides the stem, and adapts to extensions or operators.
\vspace{0.3cm}\\
BOWLUS FLUME
A flow measuring device consisting of a preformed flume.
\vspace{0.3cm}\\
BRANCH MANHOLE
A sewer or drain manhole which has more than one pipe feeding into it. A standard manhole will have one outlet and one inlet. A branch manhole will have one outlet and two or more inlets. 
\vspace{0.3cm}\\
BRANCH SEWER
A sewer that receives wastewater from a relatively small area and discharges into a main sewer servicing more than one branch sewer area. 
\vspace{0.3cm}\\
BUCKET
(1) A special device designed to be pulled along a sewer for the removal of debris from the sewer. The bucket has one end open with the opposite end having a set of jaws. When pulled from the jaw end, the jaws are automatically opened. When pulled from the other end, the jaws close. In operation, the bucket is pulled into the debris from the jaw end and to a point where some of the debris has been forced into the bucket. The bucket is then pulled out of the sewer from the other end, causing the jaws to close and retain the debris. Once removed from the manhole, the bucket is emptied and the process repeated. (2) A conventional pail or bucket used in BUCKETING OUT and also for lowering and raising tools and materials from manholes and excavations. 
\vspace{0.3cm}\\
BUCKET BAIL
The pulling handle on a bucket machine. 
\vspace{0.3cm}\\
BUCKETING OUT
An expression used to describe removal of debris from a manhole with a pail on a rope. In balling or high-velocity cleaning of sewers, debris is washed into the downstream manhole. Removal of this debris by scooping it into pails and hauling debris out is called “bucketing out.” 
\vspace{0.3cm}\\
BUFFER
A substance or solution that resists changes in pH. 
\vspace{0.3cm}\\
BULKING
Clouds of billowing sludge that occur throughout secondary clarifiers and sludge thickeners when the sludge does not settle properly.   In the activated sludge process bulking is usually caused by filamentous bacteria or bound water.
\vspace{0.3cm}\\
BULKING SLUDGE
A phenomenon that occurs in activated sludge plants whereby the sludge occupies excessive volumes and will not concentrate readily. This condition refers to a decrease in the ability of the sludge to settle and consequent loss over the settling tank weir. Bulking in activated sludge aeration tanks is caused mainly by excess suspended solids (SS) content. Sludge bulking in the final settling tank of an activated sludge plant may be caused by improper balance of the BOD load, SS concentration in the mixed liquor, or the amount of air used in aeration.
\vspace{0.3cm}\\
BURIED SERVICE
An application in which valves are installed in lines which are buried below ground level.
\vspace{0.3cm}\\
BUTTERFLY VALVE
A short face-to-face valve which has a movable vane, in the center of the flow stream, which rotates 90 degrees as the butterfly valve opens and closes.
\vspace{0.3cm}\\
BVR
BALL VALVE REGULATOR:  An automatic throttling valve controlling flow or pressure in a pipeline; comprising a package involving al ball valve actuator, positioner, and controlling instrument.
\vspace{0.3cm}\\
BYPASS
A pipe, valve, gate, weir, trench or other device designed to permit all or part of a wastewater flow to be diverted from usual channels or flow. Sometimes refers to a special line which carries the flow around a facility or device that needs maintenance or repair. 
\vspace{0.3cm}\\
BYPASSING
The act of causing all or part of a flow to be diverted from its usual channels. In a wastewater treatment plant, overload flows should be bypassed into a holding pond for future treatment. 224 
\vspace{0.3cm}\\

CADMIUM
A heavy metal commonly regulated by wastewater discharge permits and typically found in the metal finishing industry.  Other heavy metals include: Arsenic (As), Chromium (Cr), Copper (Cu), Lead (Pb), Nickel (Ni), and Zinc (Zn).
\vspace{0.3cm}\\


CALIBRATION
The comparison of a measuring device (an unknown) against an equal or better standard.
\vspace{0.3cm}\\
CALIBRATION OFFSET
An adjustment to eliminate the difference between the indicated value and the actual value.
\vspace{0.3cm}\\











CARBON DIOXIDE
A common gas, CO2, found abundantly in air, is a product of bacterial respiration and used by algae in photosynthesis. The concentration of carbon dioxide in the lagoon water governs the pH of the lagoon.
\vspace{0.3cm}\\


CARCINOGEN
Any substance that tends to produce cancer in an organism.
\vspace{0.3cm}\\

CASCADE CONTROL
Control in which the output of one controller is the setpoint for another.  In this system two (or more) controllers are configured such that the output of the "Master" controller is the setpoint for the "Slave" controller. Cascade control is commonly used to control level, with the slow "Master" loop controlling the level by demanding more or less flow from the faster "Slave" loop which regulates flow into the vessel. This allows the "Slave" loop to quickly counteract any changes in flow due to pressure variations etc., rather than having to wait until they affect the level.
\vspace{0.3cm}\\

CASING
The body of the pump which encloses the impeller. Primarily used in reference to centrifugal pumps.
\vspace{0.3cm}\\
CAST
The form of a particular part of a valve, where the basic shape is formed by molding rather than fabricating.
\vspace{0.3cm}\\
CASTING
A product or the act of producing a product made by pouring molten metal into a mold and allowing it to solidify, thus taking the shape of the mold.
\vspace{0.3cm}\\
CATCH BASIN
A chamber or well used with storm or combined sewers as a means of removing grit which might otherwise enter and be deposited in sewers. 
\vspace{0.3cm}\\
CATIONIC FLOCCULANT
Positively charged high molecular weight polyelectrolyte water soluble organic polymer designed to agglomerate solids in water substrates.
\vspace{0.3cm}\\
CAVITATION
The formation and collapse of a gas pocket or bubble on the blade of an impeller or the gate of a valve. The collapse of this gas pocket or bubble drives water into the impeller or gate with a terrific force that can cause pitting on the impeller or gate surface. Cavitation is accompanied by loud noises that sound like someone is pounding on the impeller or gate with a hammer.
\vspace{0.3cm}\\
CENTRIFUGAL FORCE
A force associated with a rotating body. In the case of a pump, the rotating impeller pushes fluid on the back of the impeller blade, imparting motion. Since the motion is circular there is a centrifugal force associated with it. The force pushes the fluid against a fixed pump casing thereby pressurizing the fluid and forcing it through the outlet.
\vspace{0.3cm}\\
CENTRIFUGAL PUMP
A centrifugal pump is a rotodynamic pump that uses a rotating impeller to increase the velocity of a fluid. Centrifugal pumps are commonly used to move liquids through a piping system. The fluid enters the pump impeller along or near to the rotating axis and is accelerated by the impeller, flowing radially outward into a diffuser or volute chamber, from there it exits into the downstream piping system. .
\vspace{0.3cm}\\
CENTRIFUGE
A mechanical device that uses centrifugal or rotational forces to separate solids from liquids.
\vspace{0.3cm}\\
CHAIN WHEEL OPERATED VALVE
An overhead valve operated by a chain drive wheel instead of a handwheel.
\vspace{0.3cm}\\
CHARACTERIZED GATE OR BALL
A ball or gate, the shape of whose port has been specially altered to provide a specific throttling capability.
\vspace{0.3cm}\\
CHECK VALVE
A one-directional valve which is opened by the fluid flow in one direction and closed automatically when the flow stops or is reversed.
\vspace{0.3cm}\\
CHEMICAL GROUT
Two chemical solutions that form a solid when combined. Solidification time is controlled by the strength of the mixtures used and the temperature. 
\vspace{0.3cm}\\
CHEMICAL OXYGEN DEMAND
An indirect reading of the organic content of wastewater. Specifically, it refers to the amount of oxygen that’s required to chemically degrade the organic material.
\vspace{0.3cm}\\
CHLORINATION
The application of chlorine to water or wastewater, generally for the purpose of disinfection, but frequently for accomplishing other biological or chemical results.
\vspace{0.3cm}\\
CHLORINATOR
A metering device which is used to add chlorine to water.
\vspace{0.3cm}\\
CHLORINE CONTACT UNIT
A baffled basin that provides sufficient time for disinfection to occur.
\vspace{0.3cm}\\
CHLORINE DEMAND
Chlorine demand is the difference between the amount of chorine added to wastewater and the amount of residual chlorine remaining after a given contact time. Chlorine demand may change with dosage, time temperature, pH, and nature and amount of the impurities in the water.
\vspace{0.3cm}\\
CHLORINE REQUIREMENT
The amount of chlorine which is needed for a particular purpose. Some reasons for adding chlorine are reducing the number of coliform bacteria (Most Probable Number), obtaining a particular chlorine residual, or oxidizing some substance in the water. In each case a definite dosage of chlorine will be necessary. This dosage is the chlorine requirement.
\vspace{0.3cm}\\
CHLORINE RESIDUAL
The amount of free chlorine remaining after meeting chlorine demand under given conditions and is necessary to complete disinfection.
\vspace{0.3cm}\\
CHOPPER PUMP
A chopper pump is a centrifugal pump, which is equipped with a cutting system to facilitate chopping/maceration of solids that are present in the pumped liquid. The main advantage of this type of pump is that it prevents clogging of the pump itself and of the adjacent piping, as all the solids and stringy materials are macerated by the chopping system. Chopper pumps exist in various configurations, including submersible and dry-installed design and they are typically equipped with an electric motor to run the impeller and to provide torque for the chopping system. Due to its high solids handling capabilities, the chopper pump is often used for pumping sewage, sludge, manure slurries, and other liquids that contain large or tough solids.
\vspace{0.3cm}\\
CHROMIUM
A heavy metal commonly regulated by wastewater discharge permits and found in metals-related industries and products (including stainless steel). It is typically regulated in two forms: total chromium and hexavalent chromium. Other heavy metals include: Arsenic (As), Cadmium (Cd), Copper (Cu), Lead (Pb), Nickel (Ni), and Zinc (Zn).
\vspace{0.3cm}\\
CIRCUITING
A condition that occurs in tanks or basins when some of the water travels faster than the rest of the flowing water. This is usually undesirable since it may result in shorter contact, reaction, or settling times in comparison with the theoretical (calculated) or presumed detention times.
\vspace{0.3cm}\\

CLAPPER
The hinged closure element of a swing check valve.
\vspace{0.3cm}\\
CLARIFICATION
Any process or processes used to reduce the concentration of suspended matter in a liquid, such as quiescent settling or sedimentation. Lagoons provide clarification across the cells and in quiescent zones in aerated systems, allowing solids to settle into a sludge layer
\vspace{0.3cm}\\
CLARIFIER
Settling Tank, Sedimentation Basin. A large, circular or rectangular tank or basin in which wastewater is held for a period of time during which the heavier solids settle to the bottom and the lighter material will float to the water surface.
\vspace{0.3cm}\\
CLEAN WATER ACT
Federal legislation passed in 1972 creating the Environmental Protection Agency, requiring a nationwide system for controlling pollutant discharges and providing for construction and regulation of publicly owned treatment works.
\vspace{0.3cm}\\
CLEANING
Sewer line cleaning, commonly done by high-velocity cleaners, that is done prior to the TV inspection of a pipeline to remove grease, slime, and grit to allow for a clearer and more accurate identification of defects and problems. 235 
\vspace{0.3cm}\\
CLEANOUT
An opening (usually covered or capped) in a wastewater collection system or conveyance piping used for inserting tools, rods or snakes while cleaning a pipeline or clearing a stoppage. 
CLOSED LOOP CONTROL
A control system in which all adjustments necessary to maintain the system occur automatically through a feedback signal from the sensor.
\vspace{0.3cm}\\

CLOSURE ELEMENT
The moving part of a valve, positioned in the flowstream which controls flow thru the valve. Ball. Gate, Plug, Clapper, Disc, etc., are specific names for closure elements.
\vspace{0.3cm}\\
COAGULANT
Positively charged electrolytes (chemicals) most commonly associated with coagulation. Coagulants are generally categorized as inorganic (alum, aluminum chloride, polyaluminum chloride, ferric chloride) or organic (epiamines, polyamines or DADMACs – diallyl dimethyl ammonium chloride).
\vspace{0.3cm}\\
COAGULANTS
Chemicals that cause very fine particles to clump (floc) together into larger particles. This makes it easier to separate the solids from the water by settling, skimming, draining or filtering.
\vspace{0.3cm}\\
COAGULATION
The agglomeration of colloidal or suspended matter brought about by the addition of some chemical to the liquid, by contact, or by other means.
\vspace{0.3cm}\\
COLIFORM
A type of bacteria. The presence of coliform-group bacteria is an indication of possible pathogenic bacterial contamination. The human intestinal tract is one of the main habitats of coliform bacteria. They may also be found in the intestinal tracts of warm-blooded animals, and in plants, soil, air, and the aquatic environment. Fecal coliforms are those coliforms found in the feces of various warm-blooded animals; whereas the term “coliform” also includes various other environmental sources.
\vspace{0.3cm}\\
COLIFORM INDEX
Escherichia Coli is an organism normally found in the intestinal tract of man and animals but rare elsewhere.  Indicators of this organism family most reliable as index of pollution, purification efficiency and potability of water.
\vspace{0.3cm}\\
COLIFORM ORGANISMS
A group of bacteria recognized as indicators of fecal pollution (see also escherichia coliform).
\vspace{0.3cm}\\
COLLECTION SYSTEM
A network of pipes, manholes, cleanouts, traps, siphons, lift stations and other structures used to collect all wastewater and wastewater-carried wastes of an area and transport them to a treatment plant or disposal system. The collection system includes land, wastewater lines and appurtenances, pumping stations and general property. 
\vspace{0.3cm}\\
COLORIMETRIC MEASUREMENT
A means of measuring unknown chemical concentrations in water by MEASURING A SAMPLE’S COLOR INTENSITY. The specific color of the sample, developed by addition of chemical reagents, is measured with a photoelectric colorimeter or is compared with “color standards” using, or corresponding with, known concentrations of the chemical.
\vspace{0.3cm}\\
COMBINED SEWER
A sewer designed to carry both sanitary wastewater and storm or surface-water runoff.
\vspace{0.3cm}\\
COMMINUTOR
A device used to reduce the size of the solid chunks in wastewater by shredding (comminuting). The shredding action is like many scissors cutting or chopping to shreds all the large influent solids material in the wastewater.
\vspace{0.3cm}\\
COMPOSITE
A composite sample is a collection of individual samples obtained at regular intervals, usually every one or two hours during a 24-hour time span. Each individual sample is combined with the others in proportion to the rate of flow when the sample was collected. The resulting mixture (composite sample) forms a representative sample and is analyzed to determine the average conditions during the sample period.
\vspace{0.3cm}\\
COMPOSITE SAMPLE
To have significant meaning, samples for laboratory tests on wastewater should be representative of the wastewater. The best method of sampling is proportional composite sampling over several hours during the day. Composite samples are collected because the flow and characteristics of the wastewater are continually changing. A composite sample will give a representative analysis of the wastewater conditions.
\vspace{0.3cm}\\
COMPUTED TOTAL CONTRIBUTION
The total anticipated load on a wastewater treatment plant or the total anticipated flow in any collection system area based on the combined computed contributions of all connections to the system. 
\vspace{0.3cm}\\
CONCENTRATE
The high TDS discharge from a reverse osmosis filtration process.
\vspace{0.3cm}\\
CONCRETE CRADLE
A device made of concrete that is designed to support sewer pipe. 225 
\vspace{0.3cm}\\
CONDENSATE
steam that has lost heat and condensed into water.
\vspace{0.3cm}\\
CONDUCTIVITY
Transmittance of an electric current through water. Usually measured in microsiemens per centimeter (uS/cm) or micromho per centimeter (umho/cm).
\vspace{0.3cm}\\
CONDUIT
An expression characterizing valves when in the open position, wherein the bore presents a smooth uninterrupted interior surface across seat rings and thru the valve port, thus affording minimum pressure drop. There are no cavities or large gaps in the bore between seat rings and body closures or between seat rings and ball/gate. Consequently, there are no areas that can accumulate debris to impede pipeline cleaning equipment or restrict the valve's motion.
\vspace{0.3cm}\\


CONFINED SPACE
A confined space is a space with limited entry and egress and not suitable for continuous occupancy.
\vspace{0.3cm}\\
CONNECTION
A connection between a drinking water system and an unapproved system.
\vspace{0.3cm}\\
CONTAMINATION
The introduction into water of microorganisms, chemicals, toxic substances, wastes, or wastewater in a concentration that makes the water unfit for its next intended use.
\vspace{0.3cm}\\

CONTROL MODE
The output form or type of control action used by a controller to control temperature process, i.e. on/off, time proportioning, PI, PID, or manual.
\vspace{0.3cm}\\

CONTROL VALVE
A valve that controls a process variable, such as pressure, flow or temperature by modulating its opening in response to a signal from a controller.
\vspace{0.3cm}\\
CONTROL VOLUME
Limits imposed for the theoretical study of a system. The limits are usually set to intersect the system at locations where conditions are known.
\vspace{0.3cm}\\

CONTROLLER
A device that measures a controlled variable, compares it with a predetermined setting and signals the actuator to read just the opening of the valve in order to re-establish the original control setting.
\vspace{0.3cm}\\
COPPER
A heavy metal commonly regulated by wastewater discharge permits. It is found in the metal finishing and electrical industries. Other heavy metals include: Arsenic (As), Cadmium (Cd), Chromium (Cr), Lead (Pb), Nickel (Ni), and Zinc (Zn).
\vspace{0.3cm}\\
CORROSION
The gradual decomposition or destruction of a material due to chemical action, often due to an electrochemical reaction. Corrosion starts at the surface of a material and moves inward, such as the chemical action upon manholes and sewer pipe materials. 
\vspace{0.3cm}\\
CORROSION INHIBITOR
chemical additive designed to control / minimize metal corrosion in water system.
\vspace{0.3cm}\\
COULISSE
Of or using runners or slides as a guiding mechanism; as in a "Coulisse" style gate valve.
\vspace{0.3cm}\\
COUPLING
(1) A threaded sleeve used to connect two pipes. (2) A device used to connect two adjacent parts, such as pipe coupling, hose coupling or drive coupling. 
\vspace{0.3cm}\\
COUPON
A steel specimen inserted into wastewater to measure the corrosiveness of the wastewater. The rate of corrosion is measured as the loss of weight of the coupon or change in its physical characteristics. Measure the weight loss (in milligrams) per surface area (in square decimeters) exposed to the wastewater per day. 
\vspace{0.3cm}\\
CREST
The bottom edge of a weir plate.
\vspace{0.3cm}\\
CROSS CONNECTION
A connection between a drinking (potable) water system and an unapproved water supply. For example, if you have a pump moving non-potable water and hook into the drinking water system to supply water for the pump seal, a cross connection or mixing between the two water systems can occur. This mixing may lead to contamination of the drinking water.
\vspace{0.3cm}\\
CRUSTACEANS
A class of microscopic water animals that consume large quantities of bacteria and algae.
\vspace{0.3cm}\\
CRYOGENIC VALVE
A valve capable of functioning at cryogenic temperatures.
\vspace{0.3cm}\\
CYANIDE
A toxic element often found in wastewater from metal finishing industries. It’s commonly regulated by wastewater permits.
\vspace{0.3cm}\\
CYCLE
A single complete operation or process returning to the starting point. A valve, stroked from full open to full close and back to full open, has undergone one cycle.
\vspace{0.3cm}\\
CYCLES OF CONCENTRATION
Ratio of boiler or cooling water to make up (feed water). Typically measured by monitoring total dissolved solids, conductivity, silica or chlorides.
\vspace{0.3cm}\\
CYLINDER OPERATOR
A power-piston valve operator using either hydraulic or pneumatic pressure. A sealed piston converts applied pressure into a linear piston rod (stem) motion.
\vspace{0.3cm}\\
DAPHNIA
A crustacean commonly found in wastewater lagoons.
\vspace{0.3cm}\\
DATA ACQUISITION SYSTEM (DAS)
Converts physical conditions into digital form, for further storage and analysis. Typically, signals from sensors (sometimes processed by sensor conditioners) are sampled, converted to digital, and stored by a computer, or by a standalone device.
\vspace{0.3cm}\\

DATUM PLANE
A reference plane.   A conveniently accessible known surface from which all vertical measurements are taken or referred to.
\vspace{0.3cm}\\
DEADEND MANHOLE
A manhole located at the upstream end of a sewer and having no inlet pipe. 
\vspace{0.3cm}\\
DEBRIS
Any material in wastewater found floating, suspended, settled, or moving along the bottom of a sewer. This material may cause stoppages by getting hung up on roots or settling out in a sewer. Debris includes grit, paper, rubber, silt, and all materials except liquid. 
\vspace{0.3cm}\\
DECHLORINATION
The removal of chlorine from the effluent of a treatment plant.
\vspace{0.3cm}\\
DENITRIFICATION
A biological process by which nitrate is converted to nitrogen gas.
\vspace{0.3cm}\\
DETENTION TIME
The time required to fill a tank at a given flow or the theoretical time required for a given flow of wastewater to pass through a tank.  Usually expressed in days of time or in hours.
\vspace{0.3cm}\\
DETRITUS
The heavy, coarse mixture of grit and organic material carried by wastewater. (also called grit).
\vspace{0.3cm}\\
DEWATER
To drain or remove water from an enclosure. A structure may be dewatered so that it can be inspected or repaired. Dewater also means draining or removing water from sludge to increase the solids concentration. 226 
\vspace{0.3cm}\\
DEWATERING
Removing free water from a sludge or slurry to form a high solids cake. belt filter presses, centrifuges, rotary fan presses and vacuum presses are dewatering devices.
\vspace{0.3cm}\\
DEWATERING PUMP
Any pump capable of removing water from an unwanted area. They are usually small, portable pumps that run on single phase power, compressed air or a small engine, but can be large permanently installed units as well.
\vspace{0.3cm}\\
DIAPHRAGM
A round, thin flexible sealing device secured and sealed around its outer edge - and sometimes around a central hole in the diaphragm - with its unsupported area free to move by flexing.
\vspace{0.3cm}\\
DIFFERENTIAL CONTROL
A control algorithm where the set point represents a desired difference between two processes. The control then manipulates the second process and holds it at the set value relative to the first.
\vspace{0.3cm}\\

DIFFUSED AIR
Method of aeration.
\vspace{0.3cm}\\
DIFFUSER
A device used to break the air stream from the blower system into fine bubbles in an aeration tank or reactor.
\vspace{0.3cm}\\
DIGESTER
A tank in which sludge is placed to allow decomposition by microorganisms. Digestion may occur under anaerobic (more common) or aerobic conditions.
\vspace{0.3cm}\\
DIGESTION
The biochemical decomposition of organic matter that results in the formation of mineral and simpler organic compounds.
\vspace{0.3cm}\\
DIP
A point in the sewer pipe where a drain grade defect results in a puddle of standing water when there is no flow. 
\vspace{0.3cm}\\
DIP TUBE
Extending the blow down valve on large gate valves requires a tube which is located inside of the valve. The tube is called the "dip tube" and extends through the bonnet to the bottom of the body cavity.
\vspace{0.3cm}\\
DISC
The closure element of a globe angle or small regulator valve. The disc (sometimes referred to as "valve," "poppet" or "plug") moves to and from the seat in a direction perpendicular to the seat face. Depends on stem force for tight shutoff.
\vspace{0.3cm}\\
DISCHARGE STATIC HEAD
The difference in elevation between the liquid level of the discharge tank and the centerline of the pump. This head also includes any additional head that may be present at the discharge tank fluid surface.
\vspace{0.3cm}\\

DISCRETE I/O
Senses or sends either "on or off" signals to the field. For example a discrete input would sense the position of a switch. A discrete output would turn on a pump or light.
\vspace{0.3cm}\\

DISINFECTION
The process designed to kill most microorganisms in water, including the destruction or inactivation of pathogenic bacteria. Disinfection differs from sterilization which destroys all living forms.
\vspace{0.3cm}\\
DISPERSANT
A non-surface active compound or an active substance added to a suspension, usually a mix, to increase the separation of particles and to prevent subsiding or clumping.
\vspace{0.3cm}\\
DISSOLVED AIR FLOTATION
Method of removing oil and suspended solids.
\vspace{0.3cm}\\
DISSOLVED AIR FLOTATION
A physical/chemical wastewater treatment technology that can be cost-effectively used by industry to remove FOG (fats, oils and grease), suspended solids, and some metals.
\vspace{0.3cm}\\
DISSOLVED OXYGEN
Molecular (atmospheric) oxygen dissolved in water or wastewater, usually abbreviated as DO.
\vspace{0.3cm}\\
DISSOLVED SOLIDS
The salts and other residues left after evaporation of water that has been passed through a laboratory filter. Dissolved solids cannot be filtered out. Some colloidal solids may not be in true solution, but if they pass through the standard membrane filter, they are considered dissolved solids. (See suspended solids)
\vspace{0.3cm}\\
DIURNAL
Having a daily cycle; usually a 24-hour period from 12:00am to 12:00am-next day.
\vspace{0.3cm}\\
DO
Dissolved Oxygen:  An indication of how much oxygen is present in water. If a facility discharges directly to a stream or river, it will usually have a permit limit related to dissolved oxygen.
\vspace{0.3cm}\\
DRAGLINE
A machine that drags a bucket down the intended line of a trench to dig or excavate the trench. Also used to dig holes and move soil or aggregate. 
\vspace{0.3cm}\\
DRAIN PLUG
A fitting at the bottom of a valve, the removal of which permits draining and flushing the body cavity.
\vspace{0.3cm}\\
DREDGE PUMP
A dredge pump is a submersible, centrifugal pump capable of handling high solids concentrations and is typically used for clearing out and/or deepening harbors and waterways. The material being moved (i.e.) sand, dirt, soil, etc.) is carried away along with the water it is suspended in.
\vspace{0.3cm}\\
DRIFT
A change in reading or value that occurs over long periods. Changes in ambient temperature, component aging, contamination, humidity and line voltage may contribute to drift.
\vspace{0.3cm}\\
DRIVE PINS
The two pins which fit into the bottom of a ball valve stem and engage corresponding holes in the ball. As the operator turns the stem, the drive pins turn the ball.
\vspace{0.3cm}\\
DROOP
A drop in set (outlet) pressure of a regulator or control valve due to the travel of its valve or poppet, as the required flow increases from low to maximum. A slight change in the control spring length due to the valve travel, will result in spring force variations, translating into a change of set (outlet) pressure.
\vspace{0.3cm}\\
DROP MANHOLE
A main line or house service line lateral entering a manhole at a higher elevation than the main flow line or channel. If the higher elevation flow is routed to the main manhole channel outside of the manhole, it is called an “outside drop.” If the flow is routed down through the manhole barrel, the pipe down to the manhole channel is called an “inside drop.” 
\vspace{0.3cm}\\
DRY WELL
A dry room or compartment in a lift station, near or below the water level, where the pumps are located. 
\vspace{0.3cm}\\
DUCKWEED
A water plant with single small leaf that floats and accumulates on the surface of lagoons.
\vspace{0.3cm}\\
EASEMENT
Legal right to use the property of others for a specific purpose. For example, a utility company may have a five-foot easement along the property line of a home. This gives the utility the legal right to install and maintain a sewer line within the easement. 
\vspace{0.3cm}\\
EFFICIENCY
A ratio of total power output to the total power input, expressed as a percent.
\vspace{0.3cm}\\
EFFLUENT
Wastewater or other liquid—raw (untreated), partially, or completely treated—flowing from a reservoir, basin, treatment process, or treatment plant. 
\vspace{0.3cm}\\
ELASTOMER
A natural or synthetic elastic material, often used for o-ring seals. Typical materials are viton, buna-n, EPDM (ethylene propylene dimonomer), etc.
\vspace{0.3cm}\\
ELECTRO DEIONIZATION
Water treatment technology that utilizes an electricity, ion exchange membranes and resin to deionize water and separate dissolved ions (impurities) from water.
\vspace{0.3cm}\\
ELEVATION
The height to which something is elevated, such as the height above sea level. 
\vspace{0.3cm}\\
ELUTRIATION
The washing of digested sludge with fresh water, plant effluent or other wastewater. The goal is to remove fine particles and/or the alkalinity in the sludge. This process reduces the demand for conditioning chemicals and improves settling or filtering characteristics of the sludge.
\vspace{0.3cm}\\
EMERGENCY SEAT SEAL
To obtain tight shut off in an emergency situation, a sealant can be injected into a specially designed groove in the seat rings. Available for most ball valves and gate valves.
\vspace{0.3cm}\\
EMBEDDED SYSTEM
A combination of computer hardware and software, and perhaps additional mechanical or other parts, designed to perform a dedicated function. In some cases, embedded systems are part of a larger system or product, as is the case of an anti-lock braking system in a car. Contrast with general-purpose computer.
\vspace{0.3cm}\\

ENTHALPY
A thermodynamic property of a fluid. The enthalpy of a fluid consist of the energy associated with the fluid at a microscopic level (related to the temperature of the fluid) plus the energy present in the form of pressure at the inlet and outlet of a system.
\vspace{0.3cm}\\
EODD
EODD stands for Electrically Operated Double Diaphragm (pumps). These are diaphragm pumps driven either directly or indirectly with an electric motor. Offering many of the same advantages as AOD/AODD pumps with less noise and no compressed air/gas requirements. They cannot run against a closed discharge, which air operated models can, except Graco’s e-Series models.
\vspace{0.3cm}\\
EPA
United States Environmental Protection Agency. 
\vspace{0.3cm}\\
EQUALIZING BASIN
A holding basin in which variations in flow and composition of a liquid are averaged. Such basins are used to provide a flow of reasonably uniform volume and composition to a treatment unit.  Also called a balancing reservoir.
\vspace{0.3cm}\\
ESCHERICHIA COLIFORM
A species of bacteria found in large numbers in the intestinal tract of warm-blooded animals.
\vspace{0.3cm}\\
ESDV (EMERGENCY SHUT DOWN VALVES)
A valve or a system of valves which, when activated, initiate a shut-down of the plant, process, or platform they are tied to.
\vspace{0.3cm}\\
ESTUARY
Bodies of water that are located at the lower end of a river and are subject to tidal fluctuations.
\vspace{0.3cm}\\
ETHERNET
The standard for local communications networks developed jointly by Digital Equipment Corp., Xerox, and Intel. Ethernet baseband coaxial cable transmits data at speeds up to 10 megabits per second. Ethernet is used as the underlying transport vehicle by several upper-level protocols, including TCP/IP.
\vspace{0.3cm}\\

EUTROPHICATION
The increase of nutrient levels of a lake or other body of water; this usually causes in increase in the growth of aquatic animal and plant life.
\vspace{0.3cm}\\
EXFILTRATION
Liquid wastes and liquid-carried wastes which unintentionally leak out of a sewer pipe system and into the environment. 
\vspace{0.3cm}\\
EXPANDING GATE VALVE
A gate valve that is comprised of a separate gate and segment that as the valve operates the gate and segment move without touching the seats, permitting the valve to be opened and closed without wear. In the closed position the gate and segment are forced against the seat. Continued downward movement of the gate causes the gate and segment to expand against the seats. When the valve reaches its full open position, the gate and segment seal off against the seats while the flow is isolated from the valve body.
\vspace{0.3cm}\\
EXPLOSION-PROOF ENCLOSURE
An enclosure designed to withstand an explosion of gases inside, to isolate sparks inside from explosive or flammable substance outside, and to maintain an external temperature that will not ignite surrounding flammable gases or liquids.
\vspace{0.3cm}\\

EXTENDED AERATION
A modification of the activated sludge process which provides for aerobic sludge digestion within the aeration system.
\vspace{0.3cm}\\
EXTENSIONS
The equipment applied to buried valves to provide above grade accessibility to operating gear, blowdown and seat lubrication systems.
\vspace{0.3cm}\\
FACE
The overall dimension from the inlet face of a valve to the outlet face of the valve (one end to the other). This dimension is governed by ASME B16.10 and API-6D to ensure that such valves are mutually inter changeable, regardless of the manufacturer.
\vspace{0.3cm}\\
FACULTATIVE
A combination of both aerobic and anaerobic conditions. Facultative cells have both aerobic and anaerobic zones. Facultative bacteria are able to exist in both aerobic and anaerobic conditions. A facultative pond is commonly used to treat wastewater flows in small communities, It has an upper aerobic zone, a lower anaerobic zone, and algae provide most of the oxygen for the bacteria.
\vspace{0.3cm}\\
FACULTATIVE ORGANISMS
Organisms that can survive and function in the presence or absence of free, elemental oxygen. Basically, organisms that can switch from aerobic or anaerobic depending on its environment. 
\vspace{0.3cm}\\
FAIL SAFE VALVE
A valve designed to fail in a preferred position (open or closed) in order to avoid an undesirable consequence in a piping system.
\vspace{0.3cm}\\
FAIR LEAD PULLEY
A pulley that is placed in a manhole to guide TV camera electric cables and the pull cable into the sewer when inspecting pipelines. 
\vspace{0.3cm}\\
FERMENTATION
A process of decomposition of organic solid materials by bacteria and other biological actions.
\vspace{0.3cm}\\
FERRIC CHLORIDE
Metal salt (FeCl3) commonly used as an coagulant in water clarification and as etching agent in chemical-etching.
\vspace{0.3cm}\\
FIELD SERVICEABLE
A statement indicating that normal repair of the valve or replacement of operating parts can be accomplished in the field without return to the manufacturer.
\vspace{0.3cm}\\
FILAMENTOUS
The property of growing in long strings, or filaments. Algae and bacteria have filamentous forms. Algae filaments can clog up equipment and be a nuisance in receiving waters. Bacterial filaments area common cause of bulking in activated sludge.
\vspace{0.3cm}\\
FILAMENTOUS ORGANISMS
Organisms that grow in a thread or filamentous form. Common types are Thiothrix and Actinomycetes. A common cause of sludge bulking in the activated sludge process.
\vspace{0.3cm}\\
FILTER PRESS
Solids dewatering device that uses pressure differential applied to sludge within a series of plates with filter clothes. The plates (with clothes in them) are arranged in a plate pack with the sludge filling chambers created by recesses within each plate. Filter presses are often called Plate and Frame or Recessed Chamber Filter Presses.
\vspace{0.3cm}\\
FIRE GATE
A gate or ball valve which is positioned in a pipeline at the entrance to a compressor station. This valve is closed in case of fire in the compressor station. Closing the valve prevents the gas in the pipeline from feeding the fire.
\vspace{0.3cm}\\
FIRE SAFE
A statement associated with a valve design which is capable of passing certain specified leakage and operational tests after exposure to fire. Must be referenced to a particular specification.
\vspace{0.3cm}\\
FLEXIBLE TUBE VALVE
A special valve using a flexible sleeve or tube which acts as the closure element. Pressure applied to the jacket space surrounding the outside of the tube, controls the opening and closing of the valve.
\vspace{0.3cm}\\
FLOAT LINE
A length of rope or heavy twine attached to a float, plastic jug or parachute to be carried by the flow in a sewer from one manhole to the next. This is called “stringing the line” and is used for pulling through winch cables, such as for a bucket machine work or closed-circuit television work. 
\vspace{0.3cm}\\
FLOAT VALVE
A valve which automatically opens or closes as the level of a liquid changes. The valve is operated mechanically by a float which rests on the top of the liquid.
\vspace{0.3cm}\\
FLOATING BALL
A ball valve having a non-trunnion mounted ball. The ball is free to float between the seat rings, and thus causes higher torques.
\vspace{0.3cm}\\
FLOC
The agglomeration of smaller particles in a gelatinous mass - typically bacteria and particulate impurities that have come together and formed a cluster that can be more easily removed from the liquid than the individual small particles.
\vspace{0.3cm}\\
FLOCCULANT
high molecular weight polyelectrolyte water soluble organic polymer designed to agglomerate solids in water substrates. Characteristics of flocculants in water treatment are determined by their molecular weight, charge type (anionic, nonionic or cationic) and charge density.
\vspace{0.3cm}\\
FLOCCULATION
The agglomeration of settleable solids through a bridging mechanism to produce larger particle that is more easily separated from water.
\vspace{0.3cm}\\
FLOODED SUCTION
In a flooded suction system, the liquid flows to the pump inlet from an elevated source by means of gravity. This is generally recommended for centrifugal pumps.
\vspace{0.3cm}\\
FLOTATION
(1) The stress or forces on a pipeline or manhole structure below a water table which tend to lift or float the pipeline or manhole structure. (2) The process of raising suspended matter to the surface of the liquid in a tank where it forms a scum layer that can be removed by skimming. The suspended matter is raised by aeration, the evolution of gas, the use of chemicals, electrolysis, heat or bacterial decomposition. 
\vspace{0.3cm}\\
FLOW
The continuous movement of a liquid from one place to another. It is measured in the unit of volume per time - gallons per minute/hour (gpm, gph), liters per minute/hour (L/min, L/hour), milliliters per minute (mL/min), cubic meters per hour (m3/h). 
\vspace{0.3cm}\\
FLOW ISOLATION
A procedure used to measure inflow and infiltration (I/I). A section of sewer is blocked off or isolated and the flow from the section is measured. 
\vspace{0.3cm}\\
FLUSHER BRANCH
A line built specifically to allow the introduction of large quantities of water to the collection system so the lines can be “flushed out” with water. Also installed to provide access for equipment to clear stoppages in a sewer. 
\vspace{0.3cm}\\
FLUSHING
The removal of deposits of material which have lodged in sewers because of inadequate velocity of flows. Water is discharged into the sewers at such rates that the larger flow and higher velocities are sufficient to remove the material. 
\vspace{0.3cm}\\
FLUX
the permeate rate per unit area of membrane surface. Typical units are gals per foot of membrane per day (gfd) or liters of permeate per square meter of membrane per hour (lmh).
\vspace{0.3cm}\\
FORCE MAIN
A pipe that carries wastewater under pressure from the discharge side of a pump to a point of gravity flow downstream.
\vspace{0.3cm}\\
FREE AVAILABLE RESIDUAL CHLORINE
That portion of the total available chlorine residual composed of dissolved chlorine gas (Cl$_2$), hypochlorous acid (HOCl), and/or hypochlorite ion (OCl$^-$) remaining in water after chlorination.
\vspace{0.3cm}\\
FREE OXYGEN
Oxygen can be dissolved in water as the soluble gas O$_2$ when it is called free oxygen and measured as dissolved oxygen.
\vspace{0.3cm}\\
FREEBOARD
The vertical distance from the normal water surface to the top of the confining wall.
\vspace{0.3cm}\\
FRICTION
The force produced as a reaction to movement. All fluids produce friction when they are in motion. The higher the fluid viscosity, the higher the friction force for the same flow rate. Friction is produced internally as one layer of fluid moves with respect to another and also at the fluid/surface interface.
\vspace{0.3cm}\\
FRICTION HEAD
The pressure expressed in pounds per square inch or feet of liquid needed to overcome the resistance to the flow in pipes and fittings.
\vspace{0.3cm}\\
FRICTION HEAD DIFFERENCE
The difference in head required to move a mass of fluid from one position to another at a certain flow rate.
\vspace{0.3cm}\\
FRICTION LOSS
The head lost by water flowing in a stream or conduit as the result of the disturbances set up by the contact between the moving water and its containing conduit and by intermolecular friction. 
\vspace{0.3cm}\\
FULL OPENING
Describes a valve whose bore (port) is nominally equal to the bore of the connecting pipe.
\vspace{0.3cm}\\

GATE
The closure element of a gate valve (sometimes called wedge or disc).
\vspace{0.3cm}\\
GATE VALVE
A straight-thru pattern valve whose closure element is a wedge or parallel-sided slab, situated between two fixed seating surfaces, with means to move it in or out of the flow stream in a direction perpendicular to the pipeline axis.
\vspace{0.3cm}\\
GLAND FOLLOWER OR GLAND FLANGE
The component used to hold down or retain the gland in the stuffing box.
\vspace{0.3cm}\\
GLAND OR GLAND BUSHING
That part of a valve which retains or compresses the stem packing in a stuffing box (where used) or retains a stem O-ring, lip seal, or stem O-ring bushing. Sometimes manually adjustable.
\vspace{0.3cm}\\
GLAND PLATE
The plate in a valve which retains the gland, gland bushing or stem seals and sometimes guides the stem.
\vspace{0.3cm}\\
GLOBE VALVE
A valve whose closure element is a flat disc or conical plug sealing on a seat which is usually parallel to the flow axis. Can be used for throttling services.
\vspace{0.3cm}\\
GO
GEAR OPERATED:  The actuation of a valve thru a - ear set which multiplies the torque applied to the valve stem.
\vspace{0.3cm}\\
GRAB SAMPLE
A single sample of water collected at a particular time and place which represents the composition of the water only at that time and place.
\vspace{0.3cm}\\
GRADE
(1) The elevation of the invert (or bottom) of a pipeline, canal, culvert, sewer, or similar conduit. (2) The inclination of slope of a pipeline, conduit, stream channel, or natural ground surface; usually expressed in terms of the ratio or percentage of number of units of vertical rise or fall per unit of horizontal distance. A 0.5 percent grade would be a drop of one-half foot per hundred feet of pipe. 
\vspace{0.3cm}\\
Granular Activated Carbon (GAC)
A material used to absorb organics from wastewater. This charcoal-like material can be used in filtration systems to remove solvent contamination.
\vspace{0.3cm}\\
GRAVITY FLOW
Water or wastewater flowing from a higher elevation to a lower elevation due to the force of gravity. The water does not flow due to energy provided by a pump. Wherever possible, wastewater collection systems are designed to use the force of gravity to convey waste liquids and solids. 
\vspace{0.3cm}\\
GREASE
In wastewater, a group of substances, including fats, waxes, free fatty acids, calcium and magnesium soaps, mineral oils, and certain other non-fatty materials.
\vspace{0.3cm}\\
GREASE BUILDUP
Any point in a collection system where coagulated and solidified greases accumulate and build up. Many varieties of grease have high adhesive characteristics and collect other solids, forming restrictions and stoppages in collection systems. 
\vspace{0.3cm}\\
GREASE FITTING
A fitting through which lubricant or sealant is injected.
\vspace{0.3cm}\\
GREASE TRAP
A receptacle designed to collect and retain grease and fatty substances usually found in kitchens or from similar wastes. It is installed in the drainage system between the kitchen or other point of production of the waste and the building wastewater collection line. Commonly used to control grease from restaurants. 
\vspace{0.3cm}\\
GREEN ALGAE
The common forms of algae essential for treatment in an facultative stabilization pond/lagoon.
\vspace{0.3cm}\\
GRINDER PUMP
A grinder pump is a waste management device. Waste from water-using household appliances (toilets, bathtubs, washing machines, etc.) flows through the home’s pipes into the grinder pump’s holding tank. Once the waste inside the tank reaches a certain level, the pump will turn on, grind the waste into fine slurry, and pump it to the central sewer system.
\vspace{0.3cm}\\
GRIT
The heavy inorganic solids material present in wastewater such as sand, coffee grounds, eggshells, gravel and cinders. Grit tends to settle out at flow velocities below 2 ft. /sec, 
\vspace{0.3cm}\\
GRIT CATCHER
A chamber usually placed at the upper end of a depressed collection line or at other points on combined or storm water collection lines where wear from grit is possible. The chamber is sized and shaped to reduce the velocity of flow through it and thus permit the settling out of grit. 
\vspace{0.3cm}\\
GRIT REMOVAL
Grit removal is accomplished by providing an enlarged channel or chamber which causes the flow velocity to be reduced and allows the heavier grit to settle to the bottom of the channel where it can be removed.
\vspace{0.3cm}\\
GRIT TRAP
A permanent structure built into a manhole (or other convenient location in a collection system) for the accumulation and easy removal of grit. 
\vspace{0.3cm}\\
HAND WHEEL
A wheel-shaped valve operating device intended to be grasped with one or both hands which allows turning the valve stem or operator shaft to which it is attached.
\vspace{0.3cm}\\
HARD FACING
A surface preparation in which an alloy is deposited on a metal surface usually by weld overlay to increase resistance to abrasion and or corrosion.
\vspace{0.3cm}\\
HARD WATER
Water having a high concentration of calcium and magnesium ions.
\vspace{0.3cm}\\
HARDNESS
It is typically the concentration of calcium and magnesium salts in water. However, it may include other metal salts such as Al, Mn, Sr and Zn. Normally measured as CaCO3 equivalents.
\vspace{0.3cm}\\
HEAD
Refers to the pressure produced by a vertical column of fluid.  It is a measure of pressure, expressed in feet of head for pumps. Water is used as the default where 10 meters (33.9 ft.) of water equals one atmosphere (14.7 psi. or 1 bar).
\vspace{0.3cm}\\
HEADWORKS
The facilities where wastewater enters a wastewater treatment plant. Preliminary treatment occurs at the Headworks.
\vspace{0.3cm}\\
HETEROTROPH
a type of bacteria that uses organic matter for energy and can use free oxygen, nitrates or sulfates as and oxygen source for respiration.
\vspace{0.3cm}\\
HUBS
The end connection tubes on a gate valve.
\vspace{0.3cm}\\
HUMAN MACHINE INTERFACE (HMI)
Software and hardware that allows human operators to monitor the state of a process under control, modify control settings to change the control objective, and manually override automatic control operations in the event of an emergency. The HMI also allows a control engineer or operator to configure set points or control algorithms and parameters in the controller. The HMI also displays process status information, historical information, reports, and other information to operators, administrators, managers, business partners, and other authorized users. Operators and engineers use HMIs to monitor and configure set points, control algorithms, send commands, and adjust and establish parameters in the controller. The HMI also displays process status information and historical information.
\vspace{0.3cm}\\
HYDRAULIC LOADING
The flow of water per acre of surface area.
\vspace{0.3cm}\\
HYDROGEN SULFIDE
A very odorous and poisonous gas. Commonly known as rotten egg gas. It is a combined form of hydrogen and sulfur with the formula H$_2$S.
\vspace{0.3cm}\\
HYDROLOGIC CYCLE
The process of evaporation of water into the air and its return to earth by precipitation. (Also called the WATER CYCLE)
\vspace{0.3cm}\\
HYPOCHLORINATORS
Chlorine pumps, chemical feed pumps or devices used to dispense chlorine solutions made from hypochlorites into the water being treated.
\vspace{0.3cm}\\
IMPELLER
The rotating element of a centrifugal pump which imparts movement and pressure to a fluid.
\vspace{0.3cm}\\
INCLINED PLATE CLARIFIER
a solids / liquid separation device (clarifier) that is filled with parallel (sometimes called Lamella) plates that are inclined at an angle between 45 and 55 degrees. The plates reduce (compared to a gravity clarifier) the foot-print required to properly settle solids.
\vspace{0.3cm}\\
INCREMENTAL SEAT TEST
The leakage testing of valve seats in an assembled valve by increasing the applied pressure in prescribed pressure steps.
\vspace{0.3cm}\\
INFILTRATION
The seepage of groundwater into a sewer system, including service connections. Seepage frequently occurs through defective or cracked pipes, pipe joints, connections, or manhole walls. 
\vspace{0.3cm}\\
INFILTRATION HEAD
The distance from a point of infiltration leaking into a collection system to the water table elevation. This is the pressure of the water being forced through the leak in the collection system. 
\vspace{0.3cm}\\
INFLATABLE PIPE STOPPER
An inflatable ball or bag used to form a plug to stop flows in a sewer pipe. 
\vspace{0.3cm}\\
INFLOW
Water discharged into a sewer system and service connections from sources other than regular connections. This includes flow from yard drains, foundation drains and around manhole covers. Inflow differs from infiltration in that it is a direct discharge into the sewer rather than a leak in the sewer itself.
\vspace{0.3cm}\\
INFLUENT
Wastewater or other liquid—raw (untreated) or partially treated—flowing into a reservoir, basin, treatment process, or treatment plant. 
\vspace{0.3cm}\\
INLET
(1) A surface connection to a drain pipe. (2) A chamber for collecting storm water with no well below the outlet pipe for collecting grit. Often connected to a CATCH BASIN or a “basin manhole” with a grit chamber. 
\vspace{0.3cm}\\
INLET PORT
That end of a valve which is connected to the upstream pressure zone of a fluid system.
\vspace{0.3cm}\\
INNER SEAT RING
The inner part of a two-piece valve seat assembly.
\vspace{0.3cm}\\
INORGANIC
Material such as sand, salt, iron, calcium salts and other mineral materials and other than of plant or animal origin or of carbon compounds (ORGANIC).
\vspace{0.3cm}\\
INSERTION PULLER
A device used to pull long segments of flexible pipe material into a sewer line when sliplining to rehabilitate a deteriorated sewer. 
\vspace{0.3cm}\\
INSITUFORM
A method of installing a new pipe within an old pipe without excavation. The process involves the use of a polyester-fiber felt tube, lined on one side with polyurethane and fully impregnated with a liquid thermal setting resin. 
\vspace{0.3cm}\\
INSPECTION TELEVISION EQUIPMENT
Television equipment that is superior to standard commercial quality, providing 600 to 650 lines of resolution, and designed for industrial inspection applications. 
\vspace{0.3cm}\\
INTERNAL PRESSURE RELIEF
A self relieving feature in non-independent seating valves that automatically relieves excessive internal body pressure caused by sudden changes in line pressures. By means of the piston effect principal the excessive body pressure will move the seat away from its seating surface and relieve it to the lower pressure side.
\vspace{0.3cm}\\
INVERT
The lowest point of the channel inside a pipe or manhole. 
\vspace{0.3cm}\\
INVERTED SIPHON
A pressure pipeline used to carry wastewater flowing in a gravity collection system under a depression such as a valley or roadway or under a structure such as a building.
\vspace{0.3cm}\\
ION EXCHANGE
Process that removes dissolved ions from solution of a certain charge by absorption onto a resin that releases (exchanges) an ion of the same charge.
\vspace{0.3cm}\\
ISRS
Inside screw, rising stem - common term for any valve design in which the stem threads are exposed to the fluid below the packing and the stem rises up through the packing when the valve is opened.
\vspace{0.3cm}\\
KEY MANHOLE
In collection system evaluation, a key manhole is one from which reliable or specific data can be obtained. 
\vspace{0.3cm}\\
KEY STOP
A method of restricting the travel of a ball valve from fully open to fully closed. The stem key bears against the ends of an arc machined in the adaptor plate.
\vspace{0.3cm}\\
KILOWATT (kW)
Unit of electrical power equal to 1000 watts or 3412 BTU per hour when the power factor equals 1.0.
\vspace{0.3cm}\\
KILOWATT HOUR (kWh)
Unit of electrical energy, or work, expended by one kilowatt in one hour. Also expressed as 1000 watt hours.
\vspace{0.3cm}\\
KITE
A device for hydraulically cleaning sewer lines. Resembling an airport wind sock and constructed of canvas-type material, the kite increases the velocity of a flow at its outlet to wash debris ahead of it. 
\vspace{0.3cm}\\
LAG
The delay between the output of a signal and the response of the instrument to which the signal is sent.
\vspace{0.3cm}\\

LAMINAR
A distinct flow regime that occurs at low Reynolds number (Re < 2000). It is characterized by particles in successive layers moving past one another in a well behaved manner (little to no turbulence).
\vspace{0.3cm}\\
LAMPING
Using reflected sunlight or a powerful light beam to inspect a sewer between two adjacent manholes. The light is directed down the pipe from one manhole. If it can be seen from the next manhole, it indicates that the line is open and straight. 
\vspace{0.3cm}\\
LATERAL
(See LATERAL SEWER) 
\vspace{0.3cm}\\
LATERAL CLEANOUT
A capped opening in a building lateral, usually located on the property line, through which the pipelines can be cleaned. 
\vspace{0.3cm}\\
LATERAL SEWER
A sewer that discharges into a branch or other sewer and has no other common sewer tributary to it. Sometimes called a “street sewer” because it collects wastewater from individual homes. 
\vspace{0.3cm}\\
LEAD
A heavy metal commonly regulated by wastewater discharge permits. Other heavy metals include: Arsenic (As), Cadmium (Cd), Chromium (Cr), Copper (Cu), Nickel (Ni), and Zinc (Zn).
\vspace{0.3cm}\\
LEVER
A handle type operating device for quarter-turn valves.
\vspace{0.3cm}\\
LIFT STATION
A wastewater pumping station that lifts the wastewater to a higher elevation when continuing the sewer at reasonable slopes would involve excessive depths of trench. Also, an installation of pumps that raise wastewater from areas too low to drain into available sewers. These stations may be equipped with air-operated ejectors or centrifugal pumps. Sometimes called a PUMP STATION, but this term is usually reserved for a similar type of facility that is discharging into a long FORCE MAIN, while a lift station has a discharge line or force main only up to the downstream gravity sewer. 
\vspace{0.3cm}\\
LIFTING LUGS
Lugs provided on large ball, gate, and check valves, for lifting and positioning valves. Also called lifting eyes.
\vspace{0.3cm}\\

LIMIT OR LIMIT CONTROLLER
A highly reliable, discrete safety device (redundant to the primary controller) that monitors and limits the temperature of the process, or a point in the process. When temperature exceeds or falls below the limit set point, the limit controller interrupts power through the load circuit. A limit controller can protect equipment and people when it is correctly installed with its own power supply, power lines, switch and sensor.
\vspace{0.3cm}\\

LIMIT SWITCH
An electrical device providing a signal to a remote observation station indicating when the valve is in the fully open or fully closed position. Usually a component of a valve operator.
\vspace{0.3cm}\\
LIQUID END
A term referring to the side (or parts) of the pump that come into contact with the process fluid. This applies to any air operated pump including Air Operated Diaphragm Pumps, air operated piston pumps and air operated drum pumps.
\vspace{0.3cm}\\
MAGNETIC DRIVE
Also referred to as a Mag-drive. This is a method of connecting the motive force to the pump which uses a series of magnets coupled together, with a containment chamber separating them. Magnetic Drives keep the fluid sealed from atmosphere and other environmental factors and eliminate the need for seals and seal maintenance. Special considerations must be taken into account when specifying a Mag-Drive pump or mixer. Ask Springer Pumps for more information.
\vspace{0.3cm}\\
MAIN LINE
Branch or lateral sewers that collect wastewater from building sewers and service lines. 
\vspace{0.3cm}\\
MAIN SEWER
A sewer line that receives wastewater from many tributary branches and sewer lines and serves as an outlet for a large territory or is used to feed an intercepting sewer. 
\vspace{0.3cm}\\
MAN MACHINE INTERFACE (MMI)
Refers to the software that the process operator "sees" the process with. An example MMI screen may show you a tank with levels and temperatures displayed with bar graphs and values. Valves and pumps are often shown and the operator can "click" on a device to turn it on, off or make a setpoint change.
\vspace{0.3cm}\\

MANDREL
(1) A special tool used to push bearings in or to pull sleeves out. (2) A gage used to measure for excessive deflection in a flexible conduit. 
\vspace{0.3cm}\\
MANHOLE
An opening in a sewer provided for the purpose of permitting operators or equipment to enter or leave a sewer. 
\vspace{0.3cm}\\
MANHOLE ELEVATION
The height (elevation) of the invert or lowest point in the bottom of a manhole above mean sea level. 
\vspace{0.3cm}\\
MANHOLE FLOW
(1) The depth or amount of wastewater flow in a manhole as observed at any selected time. (2) The total or the average flow through a manhole in gallons on any selected time interval.
\vspace{0.3cm}\\
MANHOLE INFLOW
Surface waters flowing into a manhole, usually through the vent holes in the manhole lid. 
\vspace{0.3cm}\\
MANHOLE INVERT
The lowest point in a trough or flow channel in the bottom of a manhole. 
\vspace{0.3cm}\\
MANHOLE LID
The heavy cast-iron or forged-steel cover of a manhole. The lid may or may not have vent holes. 
\vspace{0.3cm}\\
MANHOLE LID DUST PAN
A sheet metal or cast-iron pan located under a manhole lid. This pan serves to catch and hold pebbles and other debris falling through vent holes, preventing them from getting into the pipe system. 
\vspace{0.3cm}\\
MANHOLE VENTS
One or a series of one-inch diameter holes through a manhole lid for purposes of venting dangerous gases found in sewers. 
\vspace{0.3cm}\\
MASKING AGENTS
Substances used to cover up or disguise unpleasant odors. Liquid masking agents are dripped into the wastewater, sprayed into the air, or evaporated (using heat) with the unpleasant fumes or odors and then discharged into the air by blowers to make an undesirable odor less noticeable.
\vspace{0.3cm}\\
MBR
Membrane Bio Reactor:  A wastewater treatment technology that combines biological treatment with physical treatment involving membrane filtration. It’s used primarily to treat BOD, COD, and suspended solids to very low levels where effluent may be able to be reused or recycled.
\vspace{0.3cm}\\
MECHANICAL AERATION
The use of machinery including surface mixers to mix air and water inorder to dissolve oxygen into the water.
\vspace{0.3cm}\\
MECHANICAL CLEANING
Clearing pipe by using equipment that scrapes, cuts, pulls or pushes the material out of the pipe. Mechanical cleaning devices or machines include bucket machines, power rodders and hand rods. 
\vspace{0.3cm}\\
MECHANICAL PLUG
A pipe plug used in sewer systems that is mechanically expanded to create a seal. 
\vspace{0.3cm}\\
MECHANICAL SEAL
In a valve, a shut off that is accomplished by a mechanical means rather than with fluid or line pressure. The wedging action of a gate against the seats or the seat springs pushing the seat against the ball or gate are examples of mechanical seals in a valve.
\vspace{0.3cm}\\
MEMBRANE
Material layer that is a selective barrier between two phases and remains impermeable to specific particles, molecules or substances.
\vspace{0.3cm}\\
MEMBRANE BIO REACTOR
Aerobic biological wastewater treatment process that utilizes membrane filtration (rather than clarification) for solids / liquid separation. The membrane filters (ultra filters) can be either submerged or external to the biological mixed liquor tanks.
\vspace{0.3cm}\\
METAL SEAL
The seal produced by metal-to-metal contact between the sealing face of the seat ring and the closure element, without benefit of a synthetic seal.
\vspace{0.3cm}\\
METERING PUMP
Pumps used for precise introductions of chemicals into a tank, existing fluid stream or some other liquid handling equipment. Types of pumps for these include Diaphragm Pumps (AOD or EOD), Peristaltic Pumps, Hose Pumps, Gear Pumps, Bellows Pumps, Piston Pumps and other less commonly used pump types.
\vspace{0.3cm}\\
METHANE
A combustible gas produced during anaerobic fermentation of organic matter, such as by anaerobic digestion of wastewater solids.
\vspace{0.3cm}\\
MGO
MANUAL GEAR OPERATOR:  A gear operator that is operated manually (with a handwheel).
\vspace{0.3cm}\\
MICRO FILTRATION
Type of membrane filtration that separates suspended solids and solutes of high molecular from a liquid and low molecular weight solutes. This separation process is used in industry processes, water treatment and research for purifying and concentrating macromolecular solutions.
\vspace{0.3cm}\\
MICROORGANISMS
Microscopic living organisms. Microorganisms, specifically bacteria are utilized as part of the biological treatment of wastewater.
\vspace{0.3cm}\\
MIL
One thousandth of an inch, or 0.001 inches in decimal form.
\vspace{0.3cm}\\

MIXED BED ION EXCHANGE
Anion exchange process that uses a mixture of cation and anion resin combined in a single ion exchange column. With proper pretreatment, product water purified from a single pass through a mixed bed ion exchange column is the purest that can be made.
\vspace{0.3cm}\\
MIXED LIQUOR
The combination of primary effluent and active biological solids (return sludge) in the activated sludge process that is fed into the aeration tank.
\vspace{0.3cm}\\
MIXED MEDIA GRAVITY FILTER
A filter using more than one filtering media (such as coal and sand.)
\vspace{0.3cm}\\
MONITORING WELL
Wells used to collect groundwater samples for analysis to determine the amount, type, and spread of contaminants in groundwater.
\vspace{0.3cm}\\
MOVING BED BIO REACTOR
Aerobic biological wastewater treatment process that utilizes the fixed film (media) process. High surface area media is suspended in biological mixed liquor and bacteria grow on the media (attached growth) surface. A clarifier is typically used downstream for solid / liquid separation.
\vspace{0.3cm}\\

NATIONAL ELECTRICAL CODE (NEC)
A set of specifications devised for the safe application and use of electric power and devices in the United States.
\vspace{0.3cm}\\
NATIONAL ELECTRICAL MANUFACTURERS ASSOCIATION (NEMA)
A United States association that establishes specifications and ratings for electrical components and apparatuses. Conformance by manufacturers is voluntary.
\vspace{0.3cm}\\
NATIONAL INSTITUTE OF STANDARDS AND TECHNOLOGY (NIST)
A United States government agency responsible for establishing scientific and technical standards. Formerly the National Bureau of Standards.
\vspace{0.3cm}\\
NEEDLE VALVE
A type of small valve, used for flow metering, having a tapered needle-point plug or closure element and a seat having a small orifice.
\vspace{0.3cm}\\
NEGATIVE PRESSURE
Pressure that is less than the pressure in the external environment.
\vspace{0.3cm}\\
NEPHELOMETRIC TURBIDITY UNIT
The standard unit to measure turbidity or how cloudy water is. It’s used as a visual indicator for how well a wastewater treatment system is working.
\vspace{0.3cm}\\
NEWTONIAN FLUID
A fluid where the relation between shear stress and shear rate is linear, related to viscosity.
\vspace{0.3cm}\\
NICKEL
A heavy metal commonly regulated by wastewater discharge permits. It can be found in metals-related industries and products, including stainless steel. Other heavy metals include: Arsenic (As), Cadmium (Cd), Chromium (Cr), Copper (Cu), Lead (Pb), and Zinc (Zn).
\vspace{0.3cm}\\
NITRIFICATION
An aerobic process in which bacteria change ammonia and organic nitrogen into nitrite and nitrate forms of nitrogen.
\vspace{0.3cm}\\
NITRIFYING BACTERIA
Bacteria that change the ammonia and organic nitrogen in wastewater into oxidized nitrogen (usually nitrate).
\vspace{0.3cm}\\
NITROGEN
Nitrogen is present in wastewater in many forms: total Kjeldahl nitrogen, ammonia nitrogen, organic nitrogen.
\vspace{0.3cm}\\
NITROGEN CYCLE
The cycle of life, death, and decay involving organic nitrogenous matter is known as the nitrogen cycle. In the nitrogen cycle ammonia is produced from proteins.
\vspace{0.3cm}\\
NONIONIC
Neutral charged high molecular weight polyelectrolyte water soluble organic polymer designed to agglomerate solids in water substrates.
\vspace{0.3cm}\\
NONPOINT DISCHARGE
A source of wastewater that comes from a relatively large area and would have to be controlled by a management or conservation practice. Storm waters and most agricultural waters are nonpoint sources.
\vspace{0.3cm}\\
NONPOTABLE
Water that is considered unsafe and/or unpalatable for drinking.
\vspace{0.3cm}\\
NORMALLY CLOSED SOLENOID VALVE
An electrically operated valve whose inlet orifice is closed when the solenoid coil is not energized. Energize to open.
\vspace{0.3cm}\\
NPDES PERMIT
National Pollutant Discharge Elimination System permit is the regulatory agency document issued by either a federal or state agency which is designed to control all discharges of pollutants from all point sources and storm water runoff into U.S. waterways. A treatment plant that discharges to a surface water will have a NPDES permit.
\vspace{0.3cm}\\
NRS
NON-RISING STEM:  A gate valve having its stem threaded into the gate. As the stem turns the gate moves but the stem does not rise. Stem threads are exposed to the line fluid.
\vspace{0.3cm}\\
NUTRIENTS
Substances required by living plants and organisms. Forms of nitrogen and phosphorous are nutrients that can cause problems in receiving waters.
\vspace{0.3cm}\\
OBSTRUCTION
Any solid object in or protruding into a wastewater flow in a collection line that prevents a smooth or even passage of the wastewater. 
\vspace{0.3cm}\\

OCCUPATIONAL SAFETY AND HEALTH ACT. OR THE OCCUPATIONAL SAFETY AND HEALTH ADMINISTRATION
The act/United States governmental agency that establishes and enforces safety standards in the workplace.
\vspace{0.3cm}\\

OFF
A phrase used in describing the sealing ability of a valve. During air pressure testing of a new valve in the closed position, leakage past the seats is collected and bubbled thru water. To qualify as "bubble tight," no bubbles should be observed in a prescribed time span.
\vspace{0.3cm}\\
OFF HEAD
The Total Head corresponding to zero flow on the pump performance curve.
\vspace{0.3cm}\\
OFF VALVE
A valve designed only for on/off service. Not a throttling valve. Sometimes referred to as a "block valve."
\vspace{0.3cm}\\
OFFSET
(1) A combination of elbows or bends which brings one section of a line of pipe out of line with, but into a line parallel with, another section. (2) A pipe fitting in the approximate form of a reverse curve, made to accomplish the same purpose. (3) A pipe joint that has lost its bedding support and one of the pipe sections has dropped or slipped, thus creating a condition where the pipes no longer line up properly. 
\vspace{0.3cm}\\
ON/OFF CONTROL
A method of temperature control in which the controller acts as a switch, turning the final control element either "on" or "off," depending on the value of the setpoint.
\vspace{0.3cm}\\

OPERATING POINT
The point on the system curve corresponding to the flow and head required to meet the process requirements.
\vspace{0.3cm}\\
OPERATOR
A device which converts manual, hydraulic, pneumatic or electrical energy into mechanical motion to open and close a valve.
\vspace{0.3cm}\\
ORGANIC
Material which comes from mainly animal or plant sources and contains carbon. 
\vspace{0.3cm}\\
ORGANIC ACIDS
Weak acids formed from organic compounds, such as acetic acid and citric acid. These acids form first in anaerobic digesters and then are converted to methane. The organic acids in wastewater lagoons are much more complex and generally weaker.
\vspace{0.3cm}\\
ORGANIC MATERIAL
Material that can be broken down by bacteria (fats, meats, plant life).
\vspace{0.3cm}\\
ORGANIC MATTER
The waste from homes or industry of plant or animal origin. Volatile fraction of solids.
\vspace{0.3cm}\\
ORGANISMS
Microscopic plants and animals such as bacteria, molds, protozoa, algae, and small metazoa.
\vspace{0.3cm}\\
ORTHOPHOSPHATE
A simple, most common form of phosphorous in wastewater.
\vspace{0.3cm}\\
OSANDY
OUTSIDE SCREW AND YOKE:  A valve in which the fluid does not come in contact with the stem threads. The stem sealing elements is between the valve body and the stem threads.
\vspace{0.3cm}\\
OUT AIR SEAT TEST
A pressure test that can be performed only on independent seating trunnion mounted ball valves. By closing the valve and pressurizing the body cavity, all of the seals in an independent seating ball valve can then be pressure tested.
\vspace{0.3cm}\\
OUTFALL
(1) The point, location or structure where wastewater or drainage discharges from a sewer, drain, or other conduit. (2) The conduit leading to the final disposal point or area. 
\vspace{0.3cm}\\
OUTFALL SEWER
A sewer that receives wastewater from a collection system or from a wastewater treatment plant and carries it to a point of ultimate or final discharge in the environment. 
\vspace{0.3cm}\\
OUTLET
Downstream opening or discharge end of a pipe, culvert, or canal. 
\vspace{0.3cm}\\
OVERFLOW MANHOLE
A manhole which fills and allows raw wastewater to flow out onto the street or ground. 
\vspace{0.3cm}\\
OVERFLOW RELIEF LINE
Where a system has overload conditions during peak flows, an outlet may be installed above the invert and leading to a less loaded manhole or part of the system. This is usually called an “overflow relief line.” 
\vspace{0.3cm}\\
OXIC
A biological environment which is aerobic.
\vspace{0.3cm}\\
OXIDATION
The conversion of organic material to a more stable form using bacteria, chemicals, or oxygen.
\vspace{0.3cm}\\
OXIDATION
Oxidation is the addition of oxygen, removal of hydrogen, or the removal of electrons from an element or compound. In wastewater treatment, organic matter is oxidized to more stable substances.
\vspace{0.3cm}\\
OXIDATION POND
A term often used interchangeably with lagoon. Oxidation ponds are used after other treatment processes.
\vspace{0.3cm}\\
OXIDATION PONDS OR LAGOONS
Holding ponds designed to allow the decomposition of organic wastes by aerobic or anaerobic means.
\vspace{0.3cm}\\
OXYGEN REDUCTION POTENTIAL
A measure that indicates the capacity of wastewater to gain or reduce electrons during a chemical reaction. It is used as a control parameter for treating hexavalent chromium wastewater in the metal finishing industry.
\vspace{0.3cm}\\
PACKAGE TREATMENT PLANT
A small wastewater treatment plant often fabricated at the manufacturer’s factory, hauled to the site, and installed as one facility. The package may be either a small primary or a secondary wastewater treatment plant.
\vspace{0.3cm}\\
PACKING
The deformable sealing material inserted into a valve stem stuffing box, which, when compressed by a gland, provides a tight seal about the stem.
\vspace{0.3cm}\\
PARACHUTE
A device used to catch wastewater flow to pull a float line between manholes. 
\vspace{0.3cm}\\
PARSHALL FLUME
A flow measuring device consisting of a preformed flume with restrictive area called the throat. The head of water at a stilling well just upstream from the narrow part of the throat is measured and a chart is used to obtain flow rate.
\vspace{0.3cm}\\
PARTS PER BILLION
A unit of concentration for pollutants in the wastewater. It’s the equivalent of one microgram in 1 liter (ug/l).
\vspace{0.3cm}\\
PARTS PER MILLION
A unit of concentration for pollutants in the wastewater. It’s the equivalent of 1 milligram in 1 liter (mg/l).
\vspace{0.3cm}\\
PATHOGENIC ORGANISMS
Organisms including bacteria, viruses or cysts, which can cause disease (typhoid, cholera, dysentery) in a host such as a human.  Also called Pathogens.
\vspace{0.3cm}\\
PEAKING FACTOR
Ratio of a maximum flow to the average flow, such as maximum hourly flow or maximum daily flow to the average daily flow. 
\vspace{0.3cm}\\
PERCOLATION
The movement or flow of waster through soil or rocks.
\vspace{0.3cm}\\
PERFORMANCE CURVE
A curve of flow vs. Total Head for a specific pump model and impeller diameter.
\vspace{0.3cm}\\
pH
A convenient scale ranging from 0-14 used for expressing differences in the acidity or alkalinity of solutions. Liquid is considered neutral at a pH of 7; values lower than 7 indicate increasing acidity while values higher values greater than 7 indicate increasing alkalinity.
\vspace{0.3cm}\\
PHOTOGRAPHIC INSPECTIONS
A method of obtaining photographs of a pipeline by pulling a time-lapse motion picture camera through the line. 
\vspace{0.3cm}\\
PHOTOSYNTHESIS
A complex process in all green plants that contain chlorophyll. The process uses sunlight as energy to convert carbon dioxide into plant growth. As a by-product oxygen is released.
\vspace{0.3cm}\\
PID (PROPORTIONAL, INTEGRAL, DERIVATIVE) CONTROL
Controllers are designed to eliminate the need for continuous operator attention. A three-mode control action in which the controller has proportioning, integral (reset), and derivative (rate) action. Proportional action dampens the system response; integral corrects the offset; and derivative prevents over- and undershoot.  The output of PID controllers will change in response to a change in process variable or set-point.
\vspace{0.3cm}\\


PIG
Refers to a poly pig which is a bullet-shaped device made of hard rubber or similar material. 
\vspace{0.3cm}\\
PIPE CAPACITY
In a gravity-flow sewer system, pipe capacity is the total amount in gallons a pipe is able to pass in a specific time period. 
\vspace{0.3cm}\\
PIPE CLEANING
Removing grease, grit, roots and other debris from a pipe run by means of one of the hydraulic cleaning methods. 
\vspace{0.3cm}\\
PIPE DIAMETER
The nominal or commercially designated inside diameter of a pipe, unless otherwise stated. 
\vspace{0.3cm}\\
PIPE DISPLACEMENT
The cubic inches of soil or water displaced by one foot or one section of pipe. 
\vspace{0.3cm}\\
PIPE FRICTION LOSS
The positive head loss from the friction resistance between the pipe walls and the moving liquid.
\vspace{0.3cm}\\
PIPE GRADE
The angle of a sewer or a single section of a sewer as installed. Usually expressed in a percentage figure to indicate the drop in feet or tenths of a foot per hundred feet. For example, 0.5 percent grade means a drop of one-half foot per 100 feet of length. 
\vspace{0.3cm}\\
PIPE JOINT
A place where two sections of pipe are coupled or joined together. 
\vspace{0.3cm}\\
PIPE JOINT SEAL
(1) The tightness or lack of leakage at a pipe joint. (2) The method of sealing a pipe coupling. 
\vspace{0.3cm}\\
PIPE LINER
A plastic liner pulled or pushed into a pipe to eliminate excessive infiltration or exfiltration. Other solutions to the problem of infiltration/exfiltration 
\vspace{0.3cm}\\
PIPE PLUG
(1) A temporary plug placed in a sewer pipe to stop a flow while repair work is being accomplished or other functions are performed. (2) In construction of a new sewer system, service saddles are sometimes installed before a building or a building lateral is in existence. Under such circumstances, a plug will be placed in the off-lead of the saddle of a “Y.” 
\vspace{0.3cm}\\
PIPE RODDING
A method of opening a plugged or blocked pipe by pushing a steel rod or snake, or pulling same, through the pipe with a tool attached to the end of the rod or snake. Rotating the rod or snake with a tool attached increases effectiveness.
\vspace{0.3cm}\\
PIPE ROUGHNESS
A measurement of the average height of peaks producing roughness on the internal surface of pipes. Roughness is measured in many locations, and is usually defined in micro-inches RMS (root mean square).
\vspace{0.3cm}\\
PIPE RUN
(1) The length of sewer pipe reaching from one manhole to the next. (2) Any length of pipe, generally assumed to be in a straight line. 
\vspace{0.3cm}\\
PIPE SECTION
A single length of pipe between two joints or couplers. 
\vspace{0.3cm}\\
PISTON EFFECT
The sealing principle involved in utilizing line pressure to effect a seal across the floating seats of some valves.
\vspace{0.3cm}\\
PISTON PUMP
The piston pump is a positive displacement type of pump. As the piston is pulled back it draws in the fluid, and then as it’s pushed forward it pushes the liquid out. A piston pump can have up to four pistons depending on the application. They should only be used for clear liquids as any solids and/or abrasives in the fluid can damage the pump. Piston pumps are for low flow, high head applications. Frequently used for high-accuracy metering applications.
\vspace{0.3cm}\\
PLAN
A drawing showing the TOP view of sewers, manholes and streets. Also means approved contract drawings, town standards, working drawings, detail sheets or exact reproductions thereof, which show the location, character, dimensions and details of the work to be done. 
\vspace{0.3cm}\\


PLUG
The rotating closure element of a plug valve. Also a threaded fitting used to close off and seal an opening into a pressure containing chamber, e.g., pipe plug.
\vspace{0.3cm}\\
PLUG VALVE
A quarter turn valve whose closure element is usually a tapered plug having a rectangular port.
\vspace{0.3cm}\\
PNEUMATIC EJECTOR
A device for raising wastewater, sludge or other liquid by compressed air. The liquid is alternately admitted through an inward-swinging check valve into the bottom of an airtight pot. When the pot is filled compressed air is applied to the top of the liquid. The compressed air forces the inlet valve closed and forces the liquid in the pot through an outward-swinging check valve, thus emptying the pot. 
\vspace{0.3cm}\\
POLISHING POND
A final lagoon cell after other treatment which completes the treatment, or "polishes" the effluent.
\vspace{0.3cm}\\
POLY PAK STEM SEAL
An O-ring energized lip-seal which replaces O-ring stem seals in certain gate valves. Also used for stem seals in some ball valves.
\vspace{0.3cm}\\
POLYELECTROLYTES
Synthetic chemicals used as a coagulant aid.
\vspace{0.3cm}\\
PRIMARY TREATMENT
Physical (by gravity) separation of solids, grease, and scum from wastewater. With the aid of flocculating agents, primary treatment can eliminate 50 to 65\% of the suspended solids. Solids removed by primary treatment may comprise as much as 30 to 40\% of the original BOD of the water.
\vspace{0.3cm}\\
POLYMER
Polymers are used with other chemical coagulants to aid in binding small suspended particles to larger chemical flocs for their removal from water.
\vspace{0.3cm}\\
POLYPHOSPHATE
A large compound formed of several orthophosphate molecules connected by phosphate-storing microorganisms.
\vspace{0.3cm}\\
POLYVINYL CHLORIDE
The most common material used for wastewater piping. It is a type of plastic.
\vspace{0.3cm}\\
PONDING
A condition occurring on trickling filters when the hollow spaces (voids) become plugged to the extent that water passage through the filter is inadequate. Ponding may be the result of excessive slime growths, trash, or media breakdown.
\vspace{0.3cm}\\
POPULATION EQUIVALENT
An average BOD contribution by each person to a domestic sewage. The accepted population equivalent is 0.17 pounds of BOD per person per day.
\vspace{0.3cm}\\
PORCUPINE
A sewer cleaning tool the same diameter as the pipe being cleaned. The tool is a steel cylinder having solid ends with eyes cast in them to which a cable can be attached and pulled by a winch. Many short pieces of cable or bristles protrude from the cylinder to form a round brush. 
\vspace{0.3cm}\\
POTABLE WATER
Water that is considered satisfactory for human consumption.
\vspace{0.3cm}\\
POTENTIAL ENERGY
A thermodynamic property. The energy associated with the mass and height of a body above a reference plane.
\vspace{0.3cm}\\
POTENTIAL HYDROGEN
A measurement of how acidic or basic wastewater is on a scale of 0 to 14.
\vspace{0.3cm}\\
POUNDS PER SQUARE INCH
A measurement of pressure. It’s often used when discussing physical wastewater treatment technologies involving filtration, but is also used with pumps. Filtration system PSI can indicate when it’s time to backwash or change a filter.
\vspace{0.3cm}\\
POWDER PUMP
Powder pumps are normally of the Air Operated Diaphragm type and really can pump just powder and powder like materials such as flour and other fine grained, low bulk, dry-density powders in a dust free operation.
\vspace{0.3cm}\\
POWER OPERATOR
Powered valve operators are of the following general types.. Electric motor, pneumatic or hydraulic motor, pneumatic or hydraulic cylinder. Operators can either be adapted directly to the valve stem or side mounted on existing gear or scotch-yoke operators.
\vspace{0.3cm}\\
POWER RODDER
A sewer cleaning machine fitted with auger rods which are inserted in a sewer line to dislodge and cut roots and debris. 
\vspace{0.3cm}\\
PPM
Abbreviation for Parts Per Million. See MILLIGRAMS PER LITER (mg/L)
\vspace{0.3cm}\\
PRECIPITATION
(1) The total measurable supply of water received directly from clouds as rain, snow, hail, or sleet; usually expressed as depth in a day, month, or year, and designated as daily, monthly, or annual precipitation. (2) Term to describe compound used in water treatment to aid precipitation of substances normally soluble under the conditions employed. Common co-precipitants used in water are iron, aluminum, calcium and magnesium.
\vspace{0.3cm}\\
PRESSURE
The application of external or internal forces to a body producing tension or compression within the body. This tension divided by a surface is called pressure.
\vspace{0.3cm}\\
PRESSURE DROP
Referring to the loss of pressure between two points in a pipeline system. Generally, this occurs because of pipe friction loss or differences in elevation between the two points.
\vspace{0.3cm}\\
PREVENTIVE MAINTENANCE
Crews assigned the task of cleaning sewers (for example, balling or high-velocity cleaning crews) to prevent stoppages and odor complaints. Preventive maintenance is performing the most effective cleaning procedure, in the area where it is most needed, at the proper time in order to prevent failures and emergency situations. 
\vspace{0.3cm}\\
PRIMARY CELL
The first cell in a series, generally receiving raw wastewater.
\vspace{0.3cm}\\
PRIMARY CONTAMINANTS
The contaminants identified by the EPA as harmful to human health. In order to protect public health, the primary contaminants must not exceed certain specified levels known as Maximum Contaminant Levels (MCL). 
\vspace{0.3cm}\\
PRIMARY TREATMENT
A wastewater treatment process that takes the place in a rectangular or circular tank and allows those substances in wastewater that readily settle or float to be separated from the water being treated.
Primary treatment can eliminate 50\% t 65\% of the suspended solids. Solids removed by the primary treatment may comprise as much as 30\% to 40\% of the original BOD of the water.
\vspace{0.3cm}\\
PRIMING PUMP
Self-priming pumps are centrifugal pumps with an abnormally large and specially shaped volute. The purpose of the large volute is to allow the pump to pull or “lift” liquid up to the impeller. Initially the pump volute (casing), must be filled with liquid manually to “pre-prime” the pump. As the pump starts it pumps out the liquid that was manually put into it while also drawing up the air in the suction pipe along with pulling up the liquid to be pumped. As the lifted liquid enters the volute the final volume of air is pumped out of the discharge and through an air release valve. Once the liquid hits the air valve, it closes and the pump now operates as a standard centrifugal pump.
\vspace{0.3cm}\\
PROCESS AND INSTRUMENTATION DIAGRAM
An engineering drawing for a wastewater treatment system. It’s a schematic flow diagram that shows the relationship between different instrumentation and equipment.
\vspace{0.3cm}\\
PROCESS CONTROL
An engineering discipline that deals with architectures, mechanisms and algorithms for maintaining the output of a specific process within a desired range.
\vspace{0.3cm}\\
PROCESS FLOW DIAGRAM
A diagram commonly used in engineering to indicate the general flow of plant processes and equipment. The PFD displays the relationship between major equipment of a plant facility and does not show minor details.
\vspace{0.3cm}\\
PROFILE
A drawing showing the SIDE view of sewers and manholes. 
\vspace{0.3cm}\\
PROGRAMMABLE CONTROLLER
A solid-state control system that has a user-programmable memory for storage of instructions to implement specific functions such as I/O control, logic, timing, counting, report generation, communication, arithmetic, and data file manipulation. A controller consists of a central processor, input/output interface, and memory. A controller is designed as an industrial control system.
\vspace{0.3cm}\\

PROGRAMMABLE LOGIC CONTROLLER (PLC)
These computers replace relay logic and usually have PID controllers built into them. PLCs are very fast at processing discrete signals (like a switch condition).
\vspace{0.3cm}\\
PROGRESSIVE CAVITY PUMP
A pump that uses a stator and rotor in a screw shape. The rotor turning inside the stator causes cavities to move in the direction of flow. Progressive Cavity Pumps cannot run dry for any amount of time.
\vspace{0.3cm}\\
PROOF PPRESSURE
A hydrostatic test pressure, usually 1.5 times the rated working pressure, applied to an assembled valve to verify the structural integrity of the pressure containing parts. Synonymous with hydrostatic shell test. (Table 5.1, API-6D).
\vspace{0.3cm}\\
PROPELLER PUMP
Propeller pumps are similar to other centrifugal impeller pumps, but the fluid being pumped is not sent in a circular path. Rather, it proceeds more or less in a straight direction up to the discharge. The motor sits above the discharge shaft. The propeller can be placed below the surface of the liquid, where it will always be primed. Propeller pumps are generally low-speed but low heads. They can be quite large, measuring over a dozen feet in diameter and moving over 50,000 gallons per minute. Some have adjustable-pitch blades.
\vspace{0.3cm}\\
PROTECTIVE SLEEVES
A circular "pipe like" sleeve inserted in place of the ball and seats of a top entry ball valve. This protective sleeve remains in place inside the valve during valve installation and ultimate pigging of a pipeline to clear debris from the line before placing the pipeline into service. Once the pipeline has been purged of all debris, the protective sleeve is removed entirely from the ball valve cavity and operating trim (i.e. ball and seats) is then installed for normal service conditions.
\vspace{0.3cm}\\
PUBLIC SEWER
Sewer provided by or subject to the jurisdiction of the City. It includes sewers within or outside the City boundaries that ultimately discharge into the City sanitary sewer or combined sewer system, even though those sewers may not have been constructed with City funds.
\vspace{0.3cm}\\
PUBLICLY OWNED TREATMENT WORK
A term to describe a city or municipal sewage treatment facility. Since most industries discharge wastewater to these facilities, they’re typically regulated by these POTWs.
\vspace{0.3cm}\\
PUMP
A mechanical device for causing flow, for raising or lifting water or other fluid, or for applying pressure to fluids. 
\vspace{0.3cm}\\
PUMP CONTROL VALVE
A ball valve that is not meant for on-off service, but whose specific function is to control flow and prevent cavitation in pumps on liquid pipelines.
\vspace{0.3cm}\\
PUMP IMPELLER
The moving element in a centrifugal pump that drives the fluid.
\vspace{0.3cm}\\
PUMP PIT
A dry well, chamber or room below ground level in which a pump is located. 
\vspace{0.3cm}\\
PUMP STATION
Installation of pumps to lift wastewater to a higher elevation in places where flat land would require excessively deep sewer trenches. Also used to raise wastewater from areas too low to drain into available collection lines. These stations may be equipped with 
\vspace{0.3cm}\\
RACHET DRIVE
A shaft or valve that is operated by means of a ratchet mechanism. The ratchet delivers an intermittent stepped rotation through a gear in one direction only.
\vspace{0.3cm}\\
RAP
Erosion control by placement of large rocks along an embankment.
\vspace{0.3cm}\\
RAW WASTEWATER
Plant influent or wastewater BEFORE any treatment.
\vspace{0.3cm}\\
REACTOR
A tank where a wastewater stream is mixed with bacterial sludge and biochemical reactions occur.
\vspace{0.3cm}\\
RECEIVING WATER
A stream, river, lake, ocean or other surface or groundwater into which treated or untreated wastewater is discharged.
\vspace{0.3cm}\\
RECEIVING WATERS
Rivers, lakes, or other water sources that receive treated or untreated wastewaters.
\vspace{0.3cm}\\
RECIRCULATION
The return of part of the effluent from a treatment process to the incoming flow.
\vspace{0.3cm}\\
RELATIVE HUMIDITY (RH)
The moisture content of air, in relation to the maximum it can contain at that same pressure and temperature.
\vspace{0.3cm}\\

REDUCED PORT
A valve port opening that is smaller than the line size or the valve end connection size.
\vspace{0.3cm}\\
REGULAR PORT VALVE
A term usually applied to plug valves. The "regular" port of such a valve is customarily about 40\% of the line pipe area.
\vspace{0.3cm}\\
REGULATOR
A throttling valve which exercises automatic control over some variable (usually pressure). Not an on-off valve.
\vspace{0.3cm}\\
RELIEF VALVE
A quick acting, spring loaded valve that opens (relieves) when the pressure exceeds the spring setting. Often installed on the body cavity of ball and gate valves to relieve thermal overpressure in liquid services.
\vspace{0.3cm}\\
REMOTE CONTROL
The operation of a valve or other flow control device from a point at a distance from the device being controlled. Can be accomplished by electrical, pneumatic or hydraulic means.
\vspace{0.3cm}\\
RESILIENT SEAT
A valve seat containing a soft seal, such as an o-ring, to assure tight shut-off.
\vspace{0.3cm}\\
RETENTION
(1) That part of the precipitation falling on a drainage area which does not escape as surface stream flow during a given period. It is the difference between total precipitation and total runoff during the period, and represents evaporation, transpiration, subsurface leakage, infiltration, and when short periods are considered, temporary surface or underground storage on the area. (2) The delay or holding of the flow of water and water carried wastes in a pipe system. This can be due to a restriction in the pipe, a stoppage or a dip. Also, the time water is held or stored in a basin or wet well. 
\vspace{0.3cm}\\
RETENTION TIME
The time water, sludge or solids are retained or held in a clarifier or sedimentation tank.
\vspace{0.3cm}\\
RETURN ACTIVATED SLUDGE
Activated return sludge is normally returned continuously to the aeration tank. Recycling of activated sludge back to the aeration tank provides bacteria for incoming wastewater. Its should be brown in color with no obnoxious odor and is often also returned in small portions to the primary settling tanks to aid sedimentation. Settled activated sludge is generally thinner than raw sludge. Some activated sludge will be wasted to prevent excessive solids build up.
\vspace{0.3cm}\\
RETURN SLUDGE
Settled activated sludge returned to mix with incoming raw or primary settled wastewater. When the return sludge rate in the activated sludge process is too low, there will be insufficient organisms to meet the waste load entering the aerator.
\vspace{0.3cm}\\
REVERSE OSMOSIS
A physical treatment technology based around the use of a semipermeable membrane for removing dissolved solids, molecules and larger particles from water. Applied pressure is used to overcome osmotic pressure and produce high purity water. Reverse osmosis is used to produce ultra pure water for a variety of applications.
\vspace{0.3cm}\\
RIM PULL
The force required at the edge of the handwheel to generate the required torque at the center of the handwheel.
\vspace{0.3cm}\\
RISING SLUDGE
Rising sludge occurs in the secondary clarifiers of activated sludge plants when the sludge settles to the bottom of the clarifier, is compacted, and then starts to rise to the surface, usually as a result of denitrification.
\vspace{0.3cm}\\
RISING STEM
A valve stem which rises as the valve is opened.  A valve stem with threads arranged so that as the stem turns, the threads engage a stationary threaded area and lift the stem along with the closure element attached to it.
\vspace{0.3cm}\\
RISING STEM BALL VALVE
A single seated ball valve that is designed to seal by using the valves stem to mechanically wedge the valves ball into a stationary seat effecting a bubble tight seal. The valves stem operates through a guide sleeve assembly that guides the stem through a quarter turn of rotation as the stem is raised or lowered by a handwheel (or actuator). The mechanical action of the stem moves the ball away from the seat prior to the 90$^{\circ}$ rotation of the ball. This design provides lower operating torques and longer seat life while assuring bubble tight shut off.
\vspace{0.3cm}\\
ROD GUIDE
A bent pipe inserted in a manhole to guide hand and power rods into collection lines so the rods can dislodge obstructions. 
\vspace{0.3cm}\\
RODDING MACHINE
A machine designed to feed a rod into a pipe while rotating the rod.
\vspace{0.3cm}\\
RODDING TOOLS
Special tools attached to the end of a rod or snake to accomplish various results in pipe rodding. 
\vspace{0.3cm}\\
ROOF LEADER
A downspout or pipe installed to drain a roof gutter to a storm drain or other means of disposal. 
\vspace{0.3cm}\\
ROOT MOP
When roots from plant life enter a sewer system, the roots frequently branch to form a growth that resembles a string mop. 
\vspace{0.3cm}\\
ROOT SEWER
Any part of a root system of a plant or tree that enters a collection system. 
\vspace{0.3cm}\\
ROTARY GEAR PUMP
A Gear pump uses the meshing of gears to pump fluid by displacement. They are one of the most common types of pumps for hydraulic fluid power applications. Gear pumps are also widely used in chemical installations to pump fluid with a certain viscosity. There are two main variations; external gear pumps which use two external spur gears and internal gear pumps which use an external and an internal spur gear. Gear pumps are fixed displacement, meaning they pump a constant amount of fluid for each revolution. Some gear pumps are designed to function as either a motor or pump.
\vspace{0.3cm}\\
ROTATING BIOLOGICAL CONTACTOR
A biological treatment technology most often used in city treatment systems to reduce BOD.
\vspace{0.3cm}\\
ROTIFER
A form of microscopic animal that feeds on algae and bacteria. The free swimming protozoa are common in lagoons. Rotifers require aerobic conditions.
\vspace{0.3cm}\\
SADDLE
A fitting mounted on a pipe for attaching a new connection. This device makes a tight seal against the main pipe by use of a clamp, adhesive, or gasket and prevents the service pipe from protruding into the main. 
\vspace{0.3cm}\\
SADDLE CONNECTION
A building service connection made to a sewer main 
\vspace{0.3cm}\\
SAFETY VALVE
A quick opening, pop action valve used for fast relief of excessive pressure.
\vspace{0.3cm}\\
SAMPLE
A collection of individual samples obtained at regular intervals during a 24-hour period. Each individual sample is combined with the others in proportion to the rate of flow when the sample was collected. The resulting mixture, or composite, forms a representative sample and is analyzed to determine the average conditions during the sampling period.
\vspace{0.3cm}\\
SAND TRAP
A device which can be placed in the outlet of a manhole to cause a settling pond to develop in the manhole invert, thus trapping sand, rocks and similar debris heavier than water. Also may be installed in outlets from car wash areas. 
\vspace{0.3cm}\\
SANITARY COLLECTION SYSTEM
The pipe system for collecting and carrying liquid and liquid-carried wastes from domestic sources to a wastewater treatment plant. 
\vspace{0.3cm}\\
SANITARY PUMP
Sanitary pumps describe the materials used for construction of how a pump is built and if they meet specific criteria set forth by certifying agencies. Typical describing words are “FDA Compliant”, “Food Grade” and “CIP (Clean in Place)” and EHEDG. Sanitary pumps are normally built from stainless steel, PTFE, EPDM and other “clean” materials.
\vspace{0.3cm}\\
SANITARY SEWER
A pipe or conduit (sewer) intended to carry wastewater or waterborne wastes from homes, businesses, and industries to the POTW. Storm water runoff or unpolluted water should be collected and transported in a separate system of pipes or conduits (storm sewers) to natural water courses. 
\vspace{0.3cm}\\
SATURATION
Oxygen saturation is the concentration of free dissolved oxygen in water that is in equilibrium with atmospheric oxygen. It is measured in milligrams per liter (mg/l). It varies with both temperature and atmospheric pressure.
\vspace{0.3cm}\\

SCADA (SUPERVISORY CONTROL AND DATA ACQUISITION)
The level of applications that monitor and control devices such as programmable controllers. These systems are usually PC or workstation based.
\vspace{0.3cm}\\

SCOOTER
A sewer cleaning tool whose cleansing action depends on the development of high water velocity around the outside edge of a circular shield. The metal shield is rimmed with a rubber coating and is attached to a framework on wheels (like a child’s scooter). The angle of the shield is controlled by a chain-spring system which regulates the head of water behind the scooter and thus the cleansing velocity of the water flowing around the shield. 
\vspace{0.3cm}\\
SCREEN
A device used to retain or remove suspended or floating objects in wastewater. The screen has openings that are generally uniform in size. It retains or removes objects larger than the openings. A screen may consist of bars, rods, wires, gratings, wire mesh, or perforated plates.
\vspace{0.3cm}\\
SCREW CENTRIFUGAL PUMP
A pump which uses an open channel impeller with a screw shape. These pumps are ideal for sludges, large/stringy solids laden fluids, shear sensitive fluids and delicate or highly abrasive materials. Offering true non-clog performance and a steep head curve make these pumps ideal for wastewater and other sludge applications. They also handle solids more gently than other pumps. They are specifically used for transporting live fish without harm and delicate foodstuffs without bruising.
\vspace{0.3cm}\\
SCUM
(1) A layer or film of foreign matter (such as grease, oil) that has risen to the surface of water or wastewater. (2) A residue deposited on the ledge of a sewer, channel, or wet well at the water surface. (3) A mass of solid matter that floats on the surface. 
\vspace{0.3cm}\\
SEAT
That part of a valve against which the closure element (gate, ball) effects a tight shut-off. In many ball valves and gate valves, it is a floating member containing a soft seating element (usually an o-ring).
\vspace{0.3cm}\\
SECONDARY
The second in a series of cells.
\vspace{0.3cm}\\
SECONDARY CONTAMINANTS
Contaminants in drinking water that are not harmful to human health but are unpleasant. Secondary contaminants include substances that cause unpleasant tastes and odors or color the water. A Recommended Maximum Level (RCM) has been set for each of the secondary contaminants in order to make sure the water is pleasant to drink.
\vspace{0.3cm}\\
SECONDARY TREATMENT
Processing by various types of systems that employ aeration and biological oxidation stages to decompose dissolved and colloidal organic contaminants (inorganic plant nutrients may also be partially removed). . Usually follows primary sedimentation treatment and uses biological processes to convert wastes to solids that settle in secondary clarifiers. Also occurs in lagoon systems.
\vspace{0.3cm}\\

SEDIMENT
Solid material settled from suspension in a liquid. 
\vspace{0.3cm}\\
SEDIMENTATION
The process of settling and depositing of suspended matter carried by wastewater. Sedimentation usually occurs by gravity when the velocity of the wastewater is reduced below the point at which it can transport the suspended material. 
\vspace{0.3cm}\\
SEDIMENTATION TANKS
Provide a period of quiescence during which suspended waste material settles to the bottom of the tank and is scraped into a hopper and pumped out for disposal. During this period, floatable solids (fats, oils) rise to the surface of the tank and are skimmed off into scum pipes for disposal.
\vspace{0.3cm}\\
SELECT BACKFILL
Material used in backfilling of an excavation, selected for desirable compaction or other characteristics. 
\vspace{0.3cm}\\
SELECT BEDDING
Material used to provide a bedding or foundation for pipes or other underground structures. This material is of specified quality for desirable bedding or other characteristics and is often imported from a different location. 
\vspace{0.3cm}\\
SELF RELIEVING
The process whereby excessive internal body pressure, in some valves, is automatically relieved either into the upstream or downstream line by forcing the seats away from the closure element.
\vspace{0.3cm}\\
SEPTIC
A condition that exists when there is no dissolved oxygen (see anaerobic). Anaerobic bacteria and other microorganisms continue to use parts of the waste for food, but produce foul odors and black colored water. The waste in the common septic tank is typical of this condition.
\vspace{0.3cm}\\
SEQUENCING BATCH REACTORS
A biological treatment technology based on the activated sludge process. It is sometimes used in small municipalities and at food processing facilities who discharge directly to streams or rivers.
\vspace{0.3cm}\\
SERVICE ROOT
A root entering the sewer system in a service line and growing down the pipe and into the sewer main. 
\vspace{0.3cm}\\
SETPOINT
The desired value of a controlled variable.
\vspace{0.3cm}\\

SEWAGE
Largely the water supply of a community after it has been fouled by various uses. From the standpoint of course, it may be a combination of the liquid or water-carried wastes from residences, business buildings, and institutions, together with those from industrial establishments, and with such ground water, surface water, and storm water as may be present.
\vspace{0.3cm}\\
SEWER
A pipe or conduit that carries wastewater or drainage water. 
\vspace{0.3cm}\\
SEWER BALL
A spirally grooved, inflatable, semi-hard rubber ball designed for hydraulic cleaning of sewer pipes. 
\vspace{0.3cm}\\
SEWER CLEANOUT
A capped opening in a sewer main that allows access to the pipes for rodding and cleaning. Usually such cleanouts are located at terminal pipe ends or beyond terminal manholes. 
\vspace{0.3cm}\\
SEWER GAS
(1) Gas in collection lines (sewers) that results from the decomposition of organic matter in the wastewater. When testing for gases found in sewers, test for lack of oxygen and also for explosive and toxic gases. (2) Any gas present in the wastewater collection system, even though it is from such sources as gas mains, gasoline, and cleaning fluid. 
\vspace{0.3cm}\\
SEWER JACK
A device placed in manholes which supports a yoke or pulley that keeps wires or cables from rubbing against the inlet or outlet of a sewer. 
\vspace{0.3cm}\\
SEWER MAIN
A sewer pipe to which building laterals are connected. 
\vspace{0.3cm}\\
SEWER USE DISCHARGE PERMIT
Permit required or issued jointly by the Authority and a Municipality for the discharge of industrial waste.
\vspace{0.3cm}\\
SEWERAGE
System of piping with appurtenances for collecting, moving and treating wastewater from source to discharge. 
\vspace{0.3cm}\\
SEWERAGE SYSTEM
Any device, equipment or works used in the transportation, pumping, storage, treatment, recycling, and reclamation of Wastewater and Industrial Wastes. 
\vspace{0.3cm}\\
SEWERS
A system of pipes used for collecting domestic and industrial waste, as well as storm water run-off. Lateral sewers connect homes and industries to trunk sewers, which channel waste into interceptor sewers for delivery to sewage treatment plants. Sanitary sewers carry only domestic and industrial wastewater. Storm sewers carry only storm water run-off. Combined sewers carry both.
\vspace{0.3cm}\\
SHORING
Material such as boards, planks or plates, and jacks used to hold back soil around trenches and to protect workers in a trench from cave-ins. 
\vspace{0.3cm}\\
SHORT GATE
A gate valve whose seat rings contact the gate only in the closed position. Such valves are not through conduit, as the gate is completely withdrawn from the flow area in the open position.
\vspace{0.3cm}\\
SHORT PATTERN VALVE
A valve whose face-to-face dimension is less than the API-6D standard.
\vspace{0.3cm}\\
SILT DENSITY INDEX
Measurement of silt, colloids, bacteria and other foulants of Reverse Osmosis (RO) membranes. SDI is used to help determine the suitability of water or other liquids for the RO process.
\vspace{0.3cm}\\
SILTING
Silting takes place when the pressure of infiltrating waters is great enough to carry silt, sand and other small particles from the soil into the sewer system. Where lower velocities are present in the sewer pipes, settling of these materials results in silting of the sewer system. 
\vspace{0.3cm}\\
SILVER
A metal element regulated by wastewater discharge permits and common in metal finishing wastewater.
\vspace{0.3cm}\\
SINGLE-POLE, SINGLE THROW (SPST)
Refers to the function of an electrical switch often used in the control system of electric valve operators.
\vspace{0.3cm}\\
SIPHON
Is a system of piping or tubing where the exit point is lower than the entry point.
\vspace{0.3cm}\\
SLAB GATE
A gate having flat, finely finished, parallel faces - as opposed to a wedge gate. Such a closure element slides across the seats and does not depend on stem force to achieve tight shut off.
\vspace{0.3cm}\\
SLAM RETARDER
A device designed to prevent the clapper of a check valve from slamming as it closes upon flow reversal. Hydraulic damping cylinders, rotary vanes, and torsional springs are all used for this purpose.
\vspace{0.3cm}\\
SLEEVE
A pipe fitting for joining two pipes of the same nominal diameter in a straight line. 
\vspace{0.3cm}\\
SLIPLINING
A sewer rehabilitation technique accomplished by inserting flexible polyethylene pipe into an existing deteriorated sewer. 
\vspace{0.3cm}\\
SLOPE
The slope or inclination of a sewer trench excavation is the ratio of the vertical distance to the horizontal distance or “rise over run.” The inclination of a trench bottom or a trench sidewall, expressed as a ratio of vertical distance to the horizontal distance. For example, a 3:1 slope shall rise or fall 3’ vertical feet in a distance of 1’ horizontal foot. 
\vspace{0.3cm}\\
SLUDGE
The settleable solids separated from wastewater during treatment.
\vspace{0.3cm}\\
SLUDGE AGE
In the activated sludge process, a measure of the length of time a particle of suspended solids has been undergoing aeration, expressed in day. It is usually computed by dividing the weight of the suspended solids in the aeration tank by the weight of excess activated sludge discharged from the system per day.
\vspace{0.3cm}\\
SLUDGE DIGESTION
The process of changing organic matter in sludge into a gas or liquid or a more stable solid form. These changes take place as microorganisms feed on sludge in anaerobic (more common) or aerobic digesters.
\vspace{0.3cm}\\
SLUDGE INDEX
Properly called sludge volume index (SVI). It is the volume in millimeters occupied by 1 g of activated sludge after settling of the aerated liquid for 30 minutes.
\vspace{0.3cm}\\
SLUDGE JUDGE
A clear tubular device used to measure sludge depth in clarifiers or other tanks.
\vspace{0.3cm}\\
SLUDGE REAERATION
The continuous aeration of sludge after initial aeration for the purpose of improving or maintaining its condition.
\vspace{0.3cm}\\
SLURRY
Slurry is defined as a suspension of solids in a liquid. Typically, the liquid is water and the solids can be anything from soft materials such as sewage and food processing waste (potato skins, fish parts) to abrasive solids like sand, fly ash and coal. Keep in mind a typical centrifugal pump really can’t handle more than 3\% solids by weight.
\vspace{0.3cm}\\
SMOKE TEST
A method of blowing smoke into a closed-off section of a sewer system to locate sources of surface inflow. 
\vspace{0.3cm}\\
SNAKE
A stiff but flexible cable that is inserted into sewers to clear stoppages. 
\vspace{0.3cm}\\
SOAP CAKE OR SOAP BUILDUP
A combination of detergents and greases that accumulate in sewer systems, build up over a period of time, and may cause severe flow restrictions. 
\vspace{0.3cm}\\
SOIL POLLUTION
The leakage (exfiltration) of raw wastewater into the soil or ground area around a sewer pipe. 
\vspace{0.3cm}\\
SOLENOID VALVE
A small electrically operated valve used in the control piping of powered by hydraulic or pneumatic cylinder operators.
\vspace{0.3cm}\\
SOLIDS HANDLING PUMP
Many types of pumps can be used for solids handling. The size and concentration of solids in the fluid will determine the best type of pump for the application. For sewage applications such as lift stations where the solids concentration does not exceed 3\% but the solids size can reach 3" or 4", a centrifugal pump is usually the best choice. For solids concentration above 3\%, Air Operated Diaphragm, Progressive Cavity or even specialty centrifugal pumps, like Hydrostal Pumps, can be used.
\vspace{0.3cm}\\
SOLUBLE BOD
Soluble BOD is the BOD of water that has been filtered in the standard suspended solids test.
\vspace{0.3cm}\\
SOLUTION
A liquid mixture of dissolved substances. In a solution it is impossible to see all the separated parts.
\vspace{0.3cm}\\
SOUNDING ROD
A T-shaped tool or shaft that is pushed or driven down through the soil to locate underground pipes and utility conduits. 
\vspace{0.3cm}\\
SPECIFIC GRAVITY
The ratio of the density of a fluid to that of water at standard conditions
\vspace{0.3cm}\\
SPECIFIC SPEED
A formula that describes the shape of a pump impeller. The higher the specific speed the less N.P.S.H. required.
\vspace{0.3cm}\\
SPIGOT JOINT
A form of joint used on pipes which have an enlarged diameter or bell at one end, and a spigot at the other which fits into and is laid in the bell. The joint is then made tight by lead, cement, rubber O-ring, or other jointing compounds or materials. 
\vspace{0.3cm}\\
SPLIT CASE PUMP
The split case is almost synonymous with multi-stage and can be either horizontal or vertical. Used where high pressures are needed such as boiler feed.
\vspace{0.3cm}\\
SPLITTER BOX
A division box that splits the incoming flow into two or more streams. A device for splitting and directing discharge from the head box to two separate points of application.
\vspace{0.3cm}\\
SPOIL
Excavated material such as soil from the trench of a sewer. 
\vspace{0.3cm}\\

SPUR GEAR
The simplest of gears. In a gear set, the input spur gear and output spur gear are aligned on parallel shafts. An idler gear may be used to the direction of rotation on the two shafts is in the same direction.
\vspace{0.3cm}\\
SQUARE OPERATING NUT
A nut, usually 2" x 2", which is attached to a valve stem or the pinion shaft of a gear operator allowing use of wrenches to quickly operate the valve.
\vspace{0.3cm}\\
STABILIZATION
The conversion of biodegradable materials into more stable solids. Stabilization is the primary function of wastewater lagoons and treatment plants. Lagoons are often called stabilization ponds.
\vspace{0.3cm}\\
STAGE PUMP
The multi-stage pump is used for clean, clear liquids requiring significant discharge pressure. A multi-stage pump is nothing more than a standard centrifugal pump with the discharge of the initial volute discharging directly into the suction of the next volute. The numerous “volutes” are all internal to the pump and many times the volutes are hard to spot individually. The number of stages is dependent on the desired Total Discharge Head required by the application. These types of pump can be either horizontal or vertical configuration. Commonly used for boiler feed.
\vspace{0.3cm}\\
STATION
A point of reference or location in a pipeline is sometimes called a “station.” As an example, a building service is located 51 feet downstream from a manhole could be reported to be at “station 51.” 
\vspace{0.3cm}\\
STEM
A rod or shaft used to transmit motion from an operator to the closure element of a valve.
\vspace{0.3cm}\\
STEM NUT
A one or two-piece nut which engages the stem threads of a valve and transmits torque from an operator to the valve stem.
\vspace{0.3cm}\\
STILLING WELL
A well or chamber which is connected to the main flow channel by a small inlet. Waves and surges in the main flow stream will not appear in the well due to the small diameter inlet. The liquid surface in the well will be quiet, but will follow all of the steady fluctuations of the open channel. The liquid level in the well is measured to determine the flow in the main channel. 
\vspace{0.3cm}\\
STOP COLLAR
The collar on a ball valve which restricts the ball to 90$^{\circ}$ of rotation from the fully open to the fully closed position.
\vspace{0.3cm}\\
STOPPAGE
(1) Partial or complete interruption of flow as a result of some obstruction in a sewer. (2) When a sewer system becomes plugged and the flow backs up, it is said to have a “stoppage.” 
\vspace{0.3cm}\\
STORM COLLECTION SYSTEM
A system of gutters, catch basins, yard drains, culverts and pipes for the purpose of conducting storm waters from an area, but intended to exclude domestic and industrial wastes. 
\vspace{0.3cm}\\
STORM SEWER
A separate pipe, conduit or open channel (sewer) that carries runoff from storms, surface drainage, and street wash, but does not include domestic and industrial wastes. Storm sewers are often the recipients of hazardous or toxic substances due to the illegal dumping of hazardous wastes or spills created by accidents involving vehicles and trains transporting these substances. 
\vspace{0.3cm}\\
STRAIN
The ratio between the absolute displacements of a reference point within a body to a characteristic length of the body.
\vspace{0.3cm}\\
STRATIFICATION
The formation of indistinct layers of slightly variable density of waters. Often caused by warming of the surface with an absence of mixing.
\vspace{0.3cm}\\
STRESS
In this case refers to tangential stress or the force between the layers of fluid divided by the surface area between the layers.
\vspace{0.3cm}\\
STRETCH
Length of sewer from manhole to manhole.
\vspace{0.3cm}\\
STUFFING BOX
The annular chamber provided around a valve stem in a sealing system into which deformable packing is introduced.
\vspace{0.3cm}\\
SUBMERSIBLE
A submersible mixer is a mechanical device that is used to mix sludge tanks and other liquid volumes. Submersible mixers are often used in sewage treatment plants to keep solids in suspension in the various process tanks and/or sludge holding tanks. The submersible mixer is operated by an electric motor, which is coupled to the mixer’s propeller, either direct-coupled or via a planetary gear-reducer. The propeller rotates and creates liquid flow in the tank, which in turn keeps the solids in suspension. The submersible mixer is typically installed on a guide rail system, which enables the mixer to be retrieved for periodic inspection and preventive maintenance.
\vspace{0.3cm}\\
SUBMERSIBLE PUMP
Just like it sounds, these guys operate within the fluid they are pumping. Submersible pumps can be either centrifugal or AOD type pumps. The centrifugal versions are common used in sewage lift stations, while the AODs are used in chemical transfers.
\vspace{0.3cm}\\
SUCKER RODS
Rigid, coupled sewer rods of metal or wood used for clearing stoppages. Usually available in 3-ft, 39-in, 4-ft, 5-ft and 6-ft lengths. 
\vspace{0.3cm}\\
SUCTION HEAD
1) The POSITIVE pressure (in feet or pounds per square inch (psi)) on the suction side of a pump. The pressure can be measured from the centerline of the pump UP TO the elevation of the hydraulic grade line on the suction side of the pump. 2)  Condition that occurs when the liquid source is above the centerline of the pump.
\vspace{0.3cm}\\
SUCTION LIFT
1) The NEGATIVE pressure (in feet or inches of mercury vacuum) on the suction side of the pump. The pressure can be measured from the centerline of the pump DOWN TO (lift) the elevation of the hydraulic grade line on the suction side of the pump. 2) Condition that occurs when the liquid source is below the centerline of the pump.
\vspace{0.3cm}\\
SUCTION STATIC HEAD
The difference in elevation between the liquid level of the source of supply and the centerline of the pump. This head also includes any additional head that may be present at the suction tank fluid surface.
\vspace{0.3cm}\\
SUCTION STATIC LIFT
The same definition as the Suction Static head. This term is only used when the pump centerline is above the suction tank fluid surface.
\vspace{0.3cm}\\
SUPERNATANT
Liquid removed from settling sludge. Supernatant commonly refers to the liquid between the sludge on the bottom and the scum on the surface of an anaerobic digester. The liquid is usually returned to the influent wet well or to the primary clarifier.
\vspace{0.3cm}\\
SUPERVISORY PROCESS CONTROL (SPC)
The use of a single processing entity to generate setpoints for a control system. Often the generation is based on data from many sources, for example the control system itself, laboratory results, and production schedules. Note that the entity implementing SPC can have a backup in case of failure.
\vspace{0.3cm}\\

SURCHARGE
Sewers are surcharged when the supply of water to be carried is greater than the capacity of the pipes to carry the flow. The surface of the wastewater in manholes rises above the top of the sewer pipe, and the sewer is under pressure or a head, rather than at atmospheric pressure. 
\vspace{0.3cm}\\
SURCHARGED MANHOLE
A manhole in which the rate of the water entering is greater than the capacity of the outlet under gravity flow conditions. When the water in the manhole rises above the top of the outlet pipe, the manhole is said to be “surcharged.” 
\vspace{0.3cm}\\
SURFACE LOADING
Lagoon loading is rated organically in pounds of BOD per acre of surface area per day. Northern climates require lower loading rates than warmer areas, because cold weather slows down the stabilization processes of microorganisms. Treatment plants clarifiers are rated hydraulically in flow (gpd) per surface area (sq ft).
\vspace{0.3cm}\\
SURFACTANT
Compounds that lower the surface tension (or interfacial tension) between two liquids or between a liquid and a solid. Surfactants may act as detergents, wetting agents, emulsifiers, foaming agents, and dispersants.
\vspace{0.3cm}\\
SUSPENDED SOLID
Solids that either float on the surface or are suspended in water, wastewater, or other liquids, and which are largely removable by laboratory filtering.
\vspace{0.3cm}\\
SUSPENDED SOLIDS
(1) Solids that either float on the surface or are suspended in water, wastewater, or other liquids, and which are largely removable by laboratory filtering. (2) The quantity of material removed from wastewater in a laboratory test, as prescribed in STANDARD METHODS FOR THE EXAMINATION OF WATER AND WASTEWATER, and referred to as Total Suspended Solids 
\vspace{0.3cm}\\
SWAB
A circular sewer cleaning tool almost the same diameter as the pipe being cleaned. As a final cleaning procedure after a sewer line has been cleaned with a porcupine, a swab is pulled through the sewer and the flushing action of water flowing around the tool cleans the line. 
\vspace{0.3cm}\\
SWING CHECK VALVE
A check valve in which the closure element is a hinged clapper which swings or rotates about a supporting shaft.
\vspace{0.3cm}\\
SYSTEM
Systems, as far as pumps are concerned, include all the piping with or without a pump, starting at the inlet point (often the fluid surface of the suction tank) and ending at the outlet point (often the fluid surface of the discharge tank).
\vspace{0.3cm}\\
SYSTEM CURVE
Is a plot of flow vs. Total Head that satisfies the system requirements.
\vspace{0.3cm}\\
SYSTEM EQUATION
The equation for Total Head vs. flow for a specific system.
\vspace{0.3cm}\\
SYSTEM REQUIREMENTS
Friction and system inlet and outlet conditions (i.e. velocity, elevation and pressure).
\vspace{0.3cm}\\
TAG LINE
A line, rope or cable that follows equipment through a sewer so that equipment can be pulled back out if it encounters an obstruction or becomes stuck. Equipment is pulled forward with a pull line. 
\vspace{0.3cm}\\
TAP
A small hole in a sewer where a wastewater service line from a building is connected (tapped) into a lateral or branch sewer. 
\vspace{0.3cm}\\
TELEVISION INSPECTION
An inspection of the inside of a sewer pipe made by pulling a closed-circuit television camera through the pipe. 
\vspace{0.3cm}\\
TERTIARY WASTE TREATMENT
It is the treatment following secondary treatment, where the secondary treated effluent is subjected to additional aeration and/or other chemical treatment for oxidation of the residual BOD, destroy bacteria remaining from the secondary treating stage, and/or to remove nitrogen and phosphorous. 
\vspace{0.3cm}\\
THROTTLING
The intentional restriction of flow by partially closing or opening a valve. A wide range of throttling is accomplished automatically in regulators and control valves.
\vspace{0.3cm}\\
THRUST
The net force applied to a part in a particular direction - e.g., on the end of a valve stem.
\vspace{0.3cm}\\
TOP ENTRY
The design of a particular valve or regulator where the unit can be serviced or repaired by leaving its body in the line, and its internals can be accessed by removing a top portion of the unit.
\vspace{0.3cm}\\
TORQUE
The turning effort required to operate a valve. Usually expressed in "pound-feet" and referred to the stem nut, handwheel or operator pinion shaft.
\vspace{0.3cm}\\
TORQUE SWITCH
An electrical device on a motor operator which cuts off power to the operator when allowable torque is exceeded, thus preventing damage to the valve and/or the operator.
\vspace{0.3cm}\\
TORSIONAL SPRING
A coiled spring which exerts a force by twisting about its axis rather than by compression or elongation. The spring in a check valve slam retarder which is restrained at one end and fastened to the clappershaft on the other end. As the clapper opens, the spring resists the motion creating a closing force. During a rapid decrease in flow rate, the clapper is urged toward the closed position and is virtually closed just prior to the instant of actual flow reversal - thus slamming is avoided.
\vspace{0.3cm}\\
TOTAL DISSOLVED SOLIDS
Total dissolved solids are inorganic molecules of metals, minerals or salts present in water at such a small size that you can’t see them. Because of their very small size, they can be difficult to remove with any technology other than fine membrane filtration technologies such as Reverse Osmosis (RO).
\vspace{0.3cm}\\
TOTAL DYNAMIC HEAD
The amount of head produced by the pump. Calculated by summing the static head, friction head, pressure head, and velocity head.
\vspace{0.3cm}\\
TOTAL KJELDAHL NITROGEN
A pollutant found in domestic sewage that is typically a surcharge parameter for industries.
\vspace{0.3cm}\\
TOTAL ORGANIC CARBON
A direct measurement of how much organic matter is in wastewater.
\vspace{0.3cm}\\
TOTAL SOLIDS
The total amount of solids in solution and suspension.
\vspace{0.3cm}\\
TOTAL STATIC HEAD
The difference between the discharge and suction static head including the difference between the surface pressure of the discharge and suction tanks.
\vspace{0.3cm}\\
TOTAL SUSPENDED SOLIDS
Visible solids present in wastewater that can be filtered out through traditional physical treatment technologies. In the metal finishing industry, for example, FOG (fats, oils and grease) and dirt particles might make up part of the total suspended solids.
\vspace{0.3cm}\\
TOTAL TOXIC ORGANICS
A wastewater parameter that refers to the entire amount of toxic organic compounds present.   EPA has developed a specific list of chemicals that are defined as toxic organic compounds.
\vspace{0.3cm}\\
TOXIC
A substance which is poisonous to a living organism.
\vspace{0.3cm}\\
TOXIC ORGANIC MANAGEMENT PLAN
A spill plan federally required for specific industries, including metal finishing. It outlines what specific toxic organic compounds are used and how they are disposed in a manner that prevents discharge to the sewer system.
\vspace{0.3cm}\\
TOXICITY
The relative degree of being poisonous or toxic. A condition which may exist in wastes and will inhibit or destroy the growth or function of certain organisms.
\vspace{0.3cm}\\
TRANSPIRATION
See Evapotranspiration.
\vspace{0.3cm}\\
TRICKLING FILTER
An aerobic biological wastewater treatment process used as secondary treatment of sewage. Effluent from the primary clarifier is distributed over a bed of rocks. As the liquid trickles over the rocks, a biological growth on the rocks breaks down the organic matter in the sewage. The effluent is then taken to a clarifier to remove biological matter coming from the filter.
\vspace{0.3cm}\\
TRIM
Commonly refers to the valve's working parts and to their materials. Usually includes seat ring sealing surfaces, closure element sealing surfaces, stems, and back seats. Trim numbers which specifythe materials are defined in API 600 and API 602.
\vspace{0.3cm}\\
TRUNNION
That part of a ball valve which holds the ball on a fixed vertical axis and about which the ball turns. The torque requirement of a trunnion mounted ball valve is significantly less than that for a floating ball design.
\vspace{0.3cm}\\
TSS
Abbreviation for TOTAL SUSPENDED SOLIDS, a test measuring the amount of filterable solids in wastewater.
\vspace{0.3cm}\\
TURBIDITY
The cloudy appearance of water caused by the presence of suspended and colloidal matter. A turbidity measurement is used to indicate the clarity of water.
\vspace{0.3cm}\\
TURBULENT
A type of flow regime characterized by the rapid movement of fluid particles in many directions as well as the general direction of the overall fluid flow.
\vspace{0.3cm}\\
TURNS TO OPERATE
The number of complete revolutions of a handwheel or the pinion shaft of a gear operator required to stroke a valve from fully open to fully closed or vice versa.
\vspace{0.3cm}\\
ULTRAFILTRATION
A type of membrane filtration that’s similar to reverse osmosis, that separates suspended solids and solutes of high molecular from a liquid and low molecular weight solutes, but not as restrictive. It will not remove the smallest dissolved solids from water (for example salt) unless they can be chemically treated first.
\vspace{0.3cm}\\
ULTRAVIOLET
In some industries, ultraviolet light is used to sterilize water treated wastewater prior to reuse or recycling. UV light keeps algae and other bacteria from growing in the recycled wastewater.
\vspace{0.3cm}\\
UNION BONNET
A type of valve construction in which the bonnet is held on by a union nut with threads on the body.
\vspace{0.3cm}\\
UV
Ultraviolet light. UV is useful as a method of disinfection. It leaves no residual and is often used where no chlorine residual (or a very low residual) is allowed to be discharged.
\vspace{0.3cm}\\
VALVE
A device used to control the flow of fluid contained in a pipe line.
\vspace{0.3cm}\\
VALVE DATA SHEET (VDS)
A data sheet defining the minimum level of a valve design, including the materials, testing, inspection, and certification requirements.
\vspace{0.3cm}\\
VAPOR PRESSURE
The pressure at which a liquid boils at a specified temperature.
\vspace{0.3cm}\\
VARIABLE FREQUENCY DRIVE
A variable-frequency drive (VFD) is a system for controlling the rotational speed of an alternating current (AC) electric motor by controlling the frequency of the electrical power supplied to the motor. A variable frequency drive is a specific type of adjustable-speed drive. Variable-frequency drives are also known as adjustable-frequency drives (AFD), variable-speed drives (VSD), AC drives, micro drives or inverter drives. Since the voltage is varied along with frequency, these are sometimes also called VVVF (variable voltage variable frequency) drives. Variable-frequency drives are widely used. For example, in water booster stations, pump speed is controlled by the VFD based on system demand.
\vspace{0.3cm}\\
VARIABLE ORIFICE
A small variable profile valve put in a flow line and used with a pilot to restrict the flow into the pilot and make the pilot more or less sensitive to changing conditions.
\vspace{0.3cm}\\

VELOCITY HEAD DIFFERENCE
The difference in velocity head between the outlet and inlet of the system.
\vspace{0.3cm}\\
VENT PLUG/VENT PLUG ASSEMBLY/SAFETY VENT PLUG
A special pipe plug having a small allen-wrench operated vent valve. These special plugs are located at the bottom of most ball valves. With the line valve closed (and under pressure) the body cavity pressure can be vented through this small valve to check tightness of seat seals or to make minor repairs. Having vented the body pressure, the vent plug may be removed to blow out debris and foreign material or to flush the body cavity. On some gate valves, the vent plug is installed on the bonnet for the sole purpose of venting the body. Such valves have separate drain valves.
\vspace{0.3cm}\\
VENTURI VALVE
A reduced bore valve. A valve having a bore smaller in diameter than the inlet or outlet. For example, an 8"x 6" x 8" ball valve has 8" inlet and outlet connections while the ball and seats are 6". The flow through a venture valve will be reduced because of the smaller port. Venturi valves can often be economically substituted for plug valves.
\vspace{0.3cm}\\
VERTICAL
This style of mixer uses an extended shaft between the motor and mixing blade such that the motor is above and out of the liquid. These also incorporate a gear reducer to slow the speed of the mixing blades to achieve the desired mixing/tank turnover rate. Vertical mixers are often used in reactor vessels to ensure thorough chemical reactions or to mix different ingredients together in food/beverage applications.
\vspace{0.3cm}\\
VERTICAL TURBINE PUMP
A vertical turbine pump is a centrifugal type pump, often with multiple stages, where the motor is set at ground level and connect via shaft to the pump below. Used as well pumps for irrigation, they can also pump from rivers, lakes and other bodies of water.
\vspace{0.3cm}\\
VISCOSITY
A property, which measures a fluid’s resistance to movement. The resistance is caused by friction between the fluid and the boundary wall and internally by the fluid layers moving at different velocities.
\vspace{0.3cm}\\
VOLATILE ORGANIC COMPOUNDS
In wastewater, VOCs typically show up as cleaning solvents. These chemicals can kill the microbes in POTWs (publicly owned treatment work) if they are discharged in large quantities, so they are carefully regulated. 
\vspace{0.3cm}\\
VOLATILE SOLIDS
Those solids in water, wastewater, or other liquids that are lost on ignition of the dry solids at 550oC.
\vspace{0.3cm}\\
VPI (VISIBLE POSITION INDICATOR)
A position indicating rod supplied with gate valves. It extends from the top of the valve stem and serves to indicate the relative position of the gate.
\vspace{0.3cm}\\
WALL THICKNESS
The thickness of the wall of the pressure vessel or valve. For steel valves, minimum thickness requirements are defined in ASME B16.34, API 600, and API 602.
\vspace{0.3cm}\\
WASTE ACTIVATED SLUDGE
That portion of sludge from the secondary clarifier in the activated sludge process that is wasted to avoid a buildup of solids in the system.
\vspace{0.3cm}\\
WASTE TREATMENT SLUDGE
A series of tanks, screens, filters, and other processes by which most pollutants are removed from water.
\vspace{0.3cm}\\
WASTEWATER
The used water and solids from a community that flow to a treatment plant. Storm water, surface water, and groundwater infiltration also may be included in the wastewater that enters a wastewater treatment plant. The term “sewage” usually refers to household wastes, but this word is being replaced by the term “wastewater.”
\vspace{0.3cm}\\
WASTEWATER TREATMENT PLANT
A city, municipal or industrial facility that’s treating wastewater.
\vspace{0.3cm}\\
WATER HAMMER
The physical effect, often accompanied by loud banging, produced by pressure waves generated within the piping by rapid change of velocity in a liquid system.
\vspace{0.3cm}\\
WATER POLLUTION
A general term signifying the introduction into water of micro-organisms, chemicals, wastes, or sewage which renders the water unfit for its intended use.
\vspace{0.3cm}\\
WATER SYSTEM
Means a public water system that is not a community water system and that regularly serves at least 25 of the same persons over six months per year, including schools, day care centers, factories, restaurants and hospitals.
\vspace{0.3cm}\\
WATER TREATMENT FACILITY
A city or municipal water treatment facility that’s treating water you drink or use in an industrial process.
\vspace{0.3cm}\\
WEDGE GATE
A gate whose seating surfaces are inclined to the direction of closing thrust so that mechanical force on the stem produces tight contact with the inclined seat rings.
\vspace{0.3cm}\\
WEIR
(1) A wall or plate placed in an open channel and used to measure the flow of water. The depth of the flow over the weir can be used to calculate the flow rate, or a chart or conversion table may be used. (2) A wall or obstruction used to control flow (from settling tanks and clarifiers) to assure uniform flow rate and avoid short-circuiting.
\vspace{0.3cm}\\
WELL PUMP
Well pumps are a centrifugal, submersible type of pump used for bringing underground water up to the surface for domestic use. They can consist of one or several “stages” depending on the well depth and desired discharge pressure. Electrically powered the motor is typically on the bottom of the pump with the suction in the middle and the water is pumped upwards through the impeller(s) and upwards toward the surface.
\vspace{0.3cm}\\
WET OXIDATION
A method of treating or conditioning sludge before the water is removed. Compressed air is blown into the sludge; the air and sludge mixture is fed into a pressure vessel where the organic material is stabilized.
\vspace{0.3cm}\\
WET WELL
A compartment or tank in which wastewater is collected. The suction pipe of a pump may be connected to the wet well or a submersible pump may be located in the wet well.
\vspace{0.3cm}\\
WETTED PARTS
A term used for any part that comes into contact with the process fluid. These parts must be checked for chemical compatibility with the process fluid.
\vspace{0.3cm}\\
WORK
The energy required to drive the fluid through the system.  A measure of a liquid’s resistance to flow. Essentially it’s a how thick the liquid is. The viscosity determines the type of pump used, the speed it can run at, and with gear pumps, the internal clearances required.
\vspace{0.3cm}\\
WRENCH OPERATED
The operation of a valve by means of a handle or lever. Used on smaller size and lower pressure class valves.
\vspace{0.3cm}\\
YOKE
That part of a gate valve which serves as a spacer between the bonnet and the operator or actuator.
\vspace{0.3cm}\\
ZINC
A heavy metal commonly regulated by wastewater permits. It is widely present in all industries and can be difficult to treat to low levels through typical physical/chemical treatment technologies.  It is very important to determine the sources of zinc in process wastewater in order to adequately control discharge levels.  Other heavy metals include: Arsenic (As), Cadmium (Cd), Chromium(Cr), Copper (Cu), Lead (Pb), and Nickel (Ni).
\vspace{0.3cm}\\
ZOOGLEAL MASS
Jelly like masses of bacteria found in both the trickling filter and activated sludge processes. See also Biomass.
\vspace{0.3cm}\\





\end{document}



\end{document}
