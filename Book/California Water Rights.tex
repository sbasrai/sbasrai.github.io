\documentclass{article}
%\usepackage[english]{babel}%
\usepackage{graphicx}
\usepackage{tabulary}
\usepackage{tabularx}
\usepackage[table,xcdraw]{xcolor}
\usepackage{pdflscape}
\usepackage{lastpage}
\usepackage{multirow}
\usepackage{cancel}
\usepackage{amsmath}
\usepackage[table]{xcolor}
\usepackage{fixltx2e}
\usepackage[T1]{fontenc}
\usepackage[utf8]{inputenc}
\usepackage{ifthen}
\usepackage{fancyhdr}
\usepackage[document]{ragged2e}
\usepackage[margin=1in,top=1.2in,headheight=57pt,headsep=0.1in]
{geometry}
\usepackage{ifthen}
\usepackage{fancyhdr}
\everymath{\displaystyle}
\usepackage[document]{ragged2e}
\usepackage{fancyhdr}
\everymath{\displaystyle}
\linespread{2}%controls the spacing between lines. Bigger fractions means crowded lines%
%\pagestyle{fancy}
%\usepackage[margin=1 in, top=1in, includefoot]{geometry}
%\everymath{\displaystyle}
\linespread{1.3}%controls the spacing between lines. Bigger fractions means crowded lines%
%\pagestyle{fancy}
\pagestyle{fancy}
\setlength{\headheight}{56.2pt}


\chead{\ifthenelse{\value{page}=1}{\includegraphics[scale=0.3]{BassettCTCLogo}\\ \textbf \textbf California Water Rights}}
\rhead{\ifthenelse{\value{page}=1}{Shabbir Basrai}{Shabbir Basrai}}
\lhead{\ifthenelse{\value{page}=1}{}{\textbf California Water Rights}}


\cfoot{}
\lfoot{Page \thepage\ of \pageref{LastPage}}
\rfoot{Module 8}
\renewcommand{\headrulewidth}{2pt}
\renewcommand{\footrulewidth}{1pt}
\begin{document}
\begin{enumerate}
\item Surface Water Rights 
\begin{itemize}
\item California has a unique system of surface water rights that combines a traditional riparian system with the appropriative system found elsewhere in the West.\\
\item For purposes of California law, surface water includes underflow of streams, underground streams, and any other subsurface flow that is identified with a defined bed, bank or channel. 
\item Therefore, wells extracting water near a surface water supply may, in fact, be pumping "surface water" for purposes of a water rights analysis. \\

\item On many other streams in California, the surface water rights are a tangle of various categories of rights that are virtually impossible to distinguish from one another.\\

\item Often, historical practice is far more relevant in determining how water is actually allocated than are the underlying water rights. \\

Nevertheless, that historical practice is founded on basic water rights law, which re cognizes four basic types of surface water rights. 
\begin{enumerate}
\item Riparian Rights\\
\begin{itemize}
\item The riparian right is a natural appurtenance to land abutting a watercourse. 
\item However, the fact that a parcel of land presently abuts the watercourse does not mean that the entire parcel possesses riparian water rights. 
\item California adheres to the "source of title" rule. Under this rule, riparian land is the smallest parcel abutting the stream which has continuously been held under single ownership in the chain of title. In other words, if a 20 acre parcel originally a butting a river is split into a 15 acre portion separated from the river, and a 5 acre parcel is still touching the river, the 15 acre parcel will forever have lost its riparian character. Even if the 15 acre parcel is later purchased by the owner of the 5 acre parcel, the 15 acre parcel will not be restored to its former riparian character. 
\item Riparian rights can be explicitly severed from otherwise riparian land. Thus, the verification of riparian rights requires a careful examination of the chain of title back to the original patent, together with a detailed examination of each deed in the chain to determine if riparian rights were reserved to an otherwise severed parcel, or conveyed from an otherwise riparian parcel. 
\item The riparian right is a right to the natural flow of a watercourse. Therefore, there can be no riparian right to store water. 
\item Generally, "storage" means the impoundment of water for more than 30 days; 
\item Riparian water which is "stored" for less than 30 days is usually deemed to have me rely been "regulated" within the permissible scope of the underlying riparian right. \item Riparian rights are generally senior to pre-1914 and post-1914 appropriative water rights (see below), and are not lost by non-use. 
\item However, recent California court decisions suggest that unexercised riparian rights can be subordinated to longstanding downstream appropriative rights in order to avoid unfair disruption of water allocation schemes upon which water users have come to rely. \item As a result, an unexercised riparian right may be junior to other rights; in a case wher e a stream is fully appropriated, a junior right may be tantamount to no right at all, and the holder of an unexercised riparian right might find himself or herself with little or no recourse as against his or her neighbors. 
\item In addition, the right of a riparian to object to conflicting uses can be lost by prescription (see below). Riparian right holders generally do not have priorities with respect to other riparians. Instead, each has a "correlative right" to the use of a reasonable share of the total riparian water available in the watercourse, to the extent the riparian can place that water to beneficial use on the riparian's land. As a result, quantification of the riparian right is almost impossible unless there ha s been a stream-wide adjudication.
\item In 1928, the California Constitution was amended making the exercise of all water rights (both surface and groundwater) subject to a paramount limitation of reasonable and beneficial use (see below). 
\begin{itemize}
\item This amendment did not affect priorities as among different users and classes of users, but simply put a cap on the right of any user to that amount of water which can be applied to reasonable, beneficial use. 
\end{itemize}
\end{itemize}
\item Pre-1914 Appropriative Rights
\begin{itemize}
\item Appropriative water can generally be defined as water that is diverted for use on non-riparian land. 
\item Prior to 1914, there was no comprehensive permit sys tem available to establish appropriative water rights in California, and the establishment of such a right required simply posting and recording a notice of intended diversion and the construction and use of actual diversion facilities.
\item The measure of the right was the nature and scope of the use of the water diverted. 
\item Pre-1914 appropriative rights are relatively common. 
\item However, they are also fairly difficult to establish, and require evidence of original use prior to 1914 and continued use thereafter. Recorded notices of diversion can sometimes be obtained through county recorder's offices; 
\item Some pre-1914 diverters also file notices or reports of appropriation with the State Water Resources Control Board (the "SWRCB"). 
\item The appropriative right is lost by non-use for the prescriptive period, and therefore the continuity of use is as important as the origin of the right. 
\item Even if the existence of the right is established, the priority of the right is often difficult to determine unless all rights along the watercourse have been adjudicated. 
\item Nevertheless, in the realm of appropriative rights, California adheres t o the "first in time, first in right" rule, and a true pre-1914 right will have priority over a post-191 4 right. 
\end{itemize}
\item Post-1914 Appropriative Rights. 
\begin{itemize} 
\item In 1914, a comprehensive permit system was established in California and all new appropriative uses (both for diversion and storage) subsequent to that year require application to what is now the SWRCB.
\item A "post-1914" appropriative water right wi ll be granted by the SWRCB only after a public process in which the applicant is required to demonstrate the availability of unappropriated water and the ability to place that water to beneficial use. 
\item The SWRCB can verify the issuance and priority of any post-1914 water right. 
\item However, since even post-1914 rights may be lost by non-use, the continuing vitality of those rights still requires confirmation that the rights have been continually exercised without lengthy interruption (except, of course, for lack of water).
\end{itemize} 
\item Prescriptive Rights. 
\begin{itemize}
\item This final category of surface water rights is obtained by open, notorious, continuous and adverse use for the prescriptive period (in California, five years). 
\item Since the use must be adverse, a use which harms one water user may not harm another (for example an up stream water user). 
\item The prescriptive right is therefore less of a "water right" than it is the right to prevent another from objecting to one's own water use. 
\item One cannot prescript upstream. Since the adverse use must be continuous for the prescriptive period, one year of surplus water can cut off the prescriptive period and will require the would-be prescriptor to begin the prescriptive period again. Furthermore, in one case, the courts have held that since prescription does not run against the State, the SWRCB is not bound to recognize a prescriptive right and that the State may (i) require a prescriptor to apply for an appropriative permit and to comply with all conditions imposed thereon by the SWRCB, and (ii) enjoin the prescriptive use of water by a prescriptor who refuses to do so.
\item As a result, a prescriptive right is also difficult to establish, unless it has been adjudicated; a SWRCB adjudication or court proceeding is necessary to confirm the existence and scope of a prescriptive right.
\end{itemize} 
\end{enumerate}
\item Groundwater Rights\\
\begin{itemize}
\item At present, California groundwater law is found almost entirely in reported court decisions. 
\item Unlike the law governing rights to surface water and true underground streams (which is largely statutory), there is no comprehensive, statewide regulatory scheme governing the extraction or use of groundwater. 
\item Therefore, a great many aspects of groundwater law remain unclear or subject to interpretation. 
\item The recent drought resulted in unprecedented groundwater pumping due to surface water shortages. It is therefore predictable that a great many groundwater cases have been (or will be) commenced, potentially resulting in a number of significant appellate decisions in the next few years. 
\item It is also quite possible that legislative changes in groundwater law will occur in the foreseeable future. California is one of the few states in the West without a comprehensive statutory framework for groundwater regulation, and there have been a number of recent efforts in the Legis lature to enact sweeping groundwater legislation. Although those efforts have been unsuccessful , the recent enactment of AB 3030 (permitting local agencies to develop and implement groundwater management plans) indicates the continued interest in regulating groundwater through legislation. There has also been a recent effort by California counties to regulate groundwater by virtue of their general municipal police powers. While counties have generally not attempted to regulate groundwater extraction, except with respect to well drilling standards and health and safety concerns, demands of groundwater during the recent drought inspired counties to become more p roactive in the groundwater arena. A California court has recently held that groundwater regulati on is within a county's police powers and is not otherwise preempted by general State law. As a result of this case, many counties are considering adopting sweeping groundwater ordinances. In particular, counties are concerned with potential mining of groundwater resources for use outsid e the county. The extent to which counties can regulate groundwater is still an open question. Prior to 1903, California courts generally applied the English common law rule that a landowner owns beneath the surface of his or her property to "the depths of the earth and up to the heavens." This rule was known as the "absolute ownership" rule because it resulted in a landowner having the right to use as much groundwater as s/he could physically extract from beneath his or her property. There was no limitation on this right. However, in a landmark case decided in 1903, the California Supreme Court determined t hat the absolute ownership rule had no place in the arid climate of California. In the wake of the rejection of the rule, the courts established three categories of groundwater rights with respect to native percolating groundwaters (i.e., those not resulting from importation and/or artificial recharge and which are not surface water for purposes of regulation). 
\end{itemize}
\begin{enumerate}
\item Overlying Rights
\begin{itemize} 
\item The courts have consistently upheld the right of a landowner whose land was overlying a groundwater basin to extract and use that groundwater on the overlying land, but have restricted that right to an amount which is reasonable in light of the competing demands of other overlying users. 
\item Each such landowner is called an "overlying user"; the right that each such user has is an "overlying right." 
\item Since an overlying user's right is limited in relation to other overlying users, this right is sometimes called a correlative right. \item The quantification of each overlying user's correlative right depends entirely on the facts and circumstances as they exist in th e basin.
\item However, the overlying user's correlative right is generally to a reasonable share of the groundwater in the common groundwater basin for use on such landowner's land that overlies the basin. 
\item As among overlying users, it is generally irrelevant who first developed the groundwater. 
\item Each overlying user has a right in the common supply, and the exercise of that right entitles each to make a reasonable use of the water for the benefit and enjoyment of his or her overlying land. 
\item The correlative right belongs to all overlying landowners in common, and each may use only a reasonable share when the water is insufficient to meet the needs of all. 
\item The overlying right may be used for any reasonable, beneficial use. However, water devoted to public uses (for example, water acquired by municipalities and public utilities for distribution to the public) is not an overlying use. 
\item Consequently, at least in theory, the rights of a party extracting groundwater for a public use are no greater, as against other parties, than would be the case if the water was taken out of land that party did not own. 
\item However, as a practical matter, overlyers can find it difficult to stop truly public uses of groundwater, even if those uses are based on junior rights (see below). 
\end{itemize}
\item Appropriative Rights. 
\begin{itemize}
\item Any party who does not own land overlying the basin, who owns overlying land but uses the water on nonoverlying land, or who sells the water to the public gene rally is an "appropriator" and not an overlying user. 
\item The courts generally acknowledge the right of an appropriator to take the available surplus from a groundwater basin and apply it to beneficial use inside or outside the basin. 
\item For this purpose, "surplus" means available water (that is, water the use of which will not create an overdraft condition) not needed to provide for the needs of all overlying users. (Overdraft is discussed more fully below.) 
\item There is no restriction as to where the water may be used, and no requirement that the appropriator be a landowner. The water may generally be used for private or public uses without restriction, subject to the requirement that t he use of the water must be reasonable and beneficial.
\item Among appropriators, the priority of each appropriator's right is determined by the relative timing of the commencement of use, i.e., first in time is first in right. 
\end{itemize}
\item Prescriptive Rights.\\
\begin{itemize}
\item There is some question in California as to whether prescriptive rights to groundwater can be asserted. 
\item At least one case suggests that the doctrine of prescription (or at least the doctrine of "mutual prescription" pursuant to which all users of a basin prescript as against each other) no longer has a place in California. 
\item However, the better view seems to be that prescription can occur relative to groundwater, just as it can with respect to surface water. 
\item Prescriptive rights do not begin to accrue until a condition of overdraft begins. 
\begin{itemize}
\item Therefore, it is first necessary to determine when a condition of surplus ends and overdraft begins. 
\item The definition of overdraft was articulated by the California Supreme Court in 1975. 
\item The re, the court held that overdraft begins when extractions exceed the safe yield of a basin plus any temporary surplus. 
\item Safe yield is defined as the maximum quantity of water which can be withdrawn annually from a groundwater supply under a given set of conditions without causing a gradual lowering of the groundwater levels resulting, in turn, in the eventual depletion of the supply. 
\item "Temporary surplus" is the amount of water which can be pumped from a basin to provide storage space for surface water which would be wasted during wet years if it could not be stored in the basin.
\end{itemize}
 
\item Once a groundwater basin reaches a condition of overdraft, no new appropriative uses may be lawfully made. 
\item If overlying users (who, as discussed below, have priority over appropriative users) begin to consume a greater share of the safe yield, the existing appropriators must cease pumping in reverse order of their priority as against other appropriators. Typically, however, appropriators continue extraction activities unless and until demand is made and/or suit is brought. 
\item If an appropriator continues pumping from an overdrafted basin for the prescriptive period (which, as in other contexts, is five years) after the other users from the basin have notice of the overdraft condition (through decline of groundwater levels or otherwise), then that appropriator may obtain a prescriptive right good as against any other private (i.e., overlying) user.2 If the groundwater basin comes out of an overdraft condition, i.e., there is a surplus, during the five year period, the "continuous adverse use" requirement is not satisfied. 
\item In that situation, the five year period begins anew once overdraft conditions return. 
\end{itemize}
\item Prescription generally may not occur as against public entities and public utilities. As against other prescriptive users, the first in time probably is first in right. It has been held, however, that if multiple prescriptors continue their prescriptive uses for an extended period of time, the concept of "mutual prescription" may apply. Under the mutual prescription doctrine, all such prescriptive users would bear proportionate reductions caused by water shortages, rather than on the basis of temporal priority. However, as noted above, questions exist about the continued viability of the mutual prescription doctrine. As with prescriptive surface water rights, an adjudication or court proceeding is necessary to confirm the existence and scope of prescriptive rights. 2 Some Southern California counties are subject to the additional requirement that notice of extraction in excess of 25 acre-feet per year be filed. If the required notice is not filed in any one year, the prescriptive period starts over. 
\item Overlying User v. Appropriator.
\begin{itemize} 
\item As long as surplus water is available from the basin, both overlying users and appropriators may pump without restriction, provided the water is applied t o reasonable and beneficial uses.
\item Therefore, if the groundwater basin can supply the needs of all overlying users and appropriators without creating a condition of overdraft, all may continue to extract water. 
\item If there is a condition of overdraft, the overlying user will generally prevail in a dispute over priority of rights as against an appropriator (even if the appropriator is a public entity) . This is because the appropriative right is only in the surplus; if there is no surplus, there is no possibility of an appropriative right (although a prescriptive right may develop or exist). Therefore, it is unlikely an appropriator could prevail as against individual overlying users in a dispute over the right to pump native groundwater. Notwithstanding the priority of overlying users as against appropriators, it does not necessarily follow that overlying users may prevent extractions by an appropriator depending upon the timing of an action against the appropriator and the appropriator's use of the water. Where the appropriated water has been put to public use, an injunction prohibiting further appropriation may not necessarily be issued. One court has stated that "where the interests of the public are involved and the court can arrive in terms of money at the loss . . . an absolute injunction should not be granted, but an injunction conditional merely upon the failure of the defendant to make good the damage which results from its work. 
\item Such an action, if successful, should be regarded in its nature as the reverse of an action in condemnation." 
\item Also, an absolute injunction will not be granted where other forms of relief are available and would be adequate.
\end{itemize}
\item Overlying User v. Prescriptive User. 
\begin{itemize}
\item Prescriptive use establishes a prescriptive right good against the overlying users as to whom the prescription has been effected. 
\item The priority between such users depends on the amount used by the overlying users during the prescriptive period. 
\item If the overlying users continue to pump at the same or increasing levels during the prescriptive period, then neither the prescriptive user nor the overlying user has priority over the other. Rather, the prescriptive user will obtain in effect a parity, according to the following formula announced by the California Supreme Court: \\
\textit{The effect of the prescriptive right would be to give to the party acquiring it and take away from the private defendant against whom it was acquired either 
\begin{enumerate}
\item Enough water to make the ratio of the prescriptive right to the remaining rights of the private defendant as favorable to the former in time of subsequent shortage as it was throughout the prescriptive period... or 
\item The amount of the prescriptive taking, whichever is less...
\end{enumerate}}
\item If an overlying user's use declines during the prescriptive period, the overlying user will lose his or her right (as against a prescriptive user) to the extent of that reduction. 
\item Ironically, those who are not exercising their overlying use rights at all may fare quite well in the face of prescriptive uses; based on comments by some courts, it appears prescriptive rights do not impair an overlyer's right to groundwater for new overlying uses for which the need had not yet come into existence during the prescriptive period. 
\item When prescriptive rights have vested and an overlying user continues to pump during the prescriptive period, the overlyer's right to continue pumping will usually be protected. 
\item In that case, a court would more likely order a proportionate reduction in pumping by both parties. 
\end{itemize}
\item Appropriator v. Prescriptive User. 
\begin{itemize}
\item Technically, this condition does not often exist, since one cannot be an appropriator unless there is surplus, and one cannot acquire a prescriptive right unless there is overdraft.
\item Nevertheless, a prescriptive user is simply an appropriator whose use has continued for a sufficient period of time in the face of an overdraft condition. 
\item If both become prescriptive users, and one is a public entity, the public entity will likely prevail because it can prescript against the other user, while the private user cannot prescript against the public entity. 
\item However, even though a public entity cannot lose its rights by prescription, it is subject to limitations in prescription by the exercise of self help by an overlying user. Groundwater Resulting From Imported Water. 
\item The preceding discussion relates to native groundwater, i.e., percolating groundwater which occurs naturally and is not imported. 
\item Imported water is water derived from outside the watershed which is purposefully recharged into the groundwater basin, essentially creating an "account" for the recharger. 
\item Imported water does not include the return flow from extracted native groundwater since that water does not add to the overall groundwater supply but instead decreases the amount of extraction from the basin. 
\item Assuming no prescriptive rights have attached to imported water used to recharge a basin, the imported water belongs solely to the importer, who may extract it (even if the basin is in overdraft) and use or export it without liability to other basin users. 
\end{itemize}
\item Common Groundwater Practices. 
\begin{itemize}
\item While the legal principles summarized above are those that govern groundwater throughout the State it is important to understand that those principles are often ignored--or at least discounted--in practice. 
\item Groundwater is frequently pumped by one landowner and sold or given to another, and groundwater has often been exported from one over drafted basin to another (especially during the recent drought). 
\item Probably more than any other body of natural resource law, groundwater law is often honored more in the breach than in the compliance. 
\item Historical practices therefore frequently overrun technicalities, and courts often attempt to honor past practices by finding (sometimes tortured) ways to make the law "fit" the circumstances. 
\item Thus, the failure to use groundwater in accordance with the principles summarized above does not necessarily mean that a water user is violating the law or is without rights to the ground water in question. 
\end{itemize}

\item Adjudicated Water Rights Many "water rights" in California are not quantified, but are simply claimed and/or exercised without objection by other parties. However, when competing demands for a common water supply--whether surface water, groundwater or both--become too great, formal adjudications are sometimes commenced by one or more of the competing claimants. 
\item Both the SWRCB and the courts can conduct adjudications under appropriate circumstances, which typically result in an enforceable order allocating the water (and the water rights) in the adjudicated stream system, groundwater basin or combined water source. 
\item Adjudications typically take years (or even decades) to complete because of the often complex legal and factual issues involved. 
\item Frequently, the result of an adjudication is an equitable apportionment of water that does not "track" with a technical application of water law principles. For example, in a recently completed adjudication in the Mojave Basin, the court noted that strict adherence to priority of rights and correlative rights among water users of equal status created uncertainty and potential economic consequences. Therefore, the court applied a "physical solution" requiring all users of the common water source to share equitably both in the water and in the reduction in use necessary to reduce extractions to safe yield. As is commonly the case in judicial adjudications, the court also retained continuing jurisdiction over the implementation of the adjudication order, making the co urt an ongoing "player" in the administration of the basin. Such physical solutions may produce the most appropriate allocation of the water resource, but they also create a number of issues. 
\item The adjudication order effectively supersedes water rights law, and any interested party must become familiar with the order's impacts on existing and future involvement with impacted water users. 
\item Depending on the adjudication order, a watermaster may be in place with jurisdiction over the affected water, and special procedures may be imposed on parties dealing with the water and water rights involved. 
\item Even more vexing is the relatively common situation in which the adjudication order effectively severs the water rights from the l and, making them freely transferable separate from the land on which those rights originally arose. 
\item Adjudicated water rights therefore can fall into a category distinct from more traditional water rights. 
\item Beneficial Use and the Public Trust Doctrine Regardless of the nature of the water right in question, two very important principles will always apply. 
\begin{enumerate}
\item Under the California Constitution, water must be put to reasonable and beneficial use. No water right grants any party the right to waste or make unreasonable use of water, and any water right can be curtailed or revoked if it is determined that the holder of that right has engaged in a wasteful or unreasonable use of water. 
\item no water user in the State "owns" any water. Instead, a water right grants the holder thereof only the right to use water (called a "usufructuary right").
\end{enumerate} 
\item The owner of "legal title" to all water is the State in its capacity as a trustee for the benefit of the public. The so-called "public trust doctrine" requires the State, as a trustee, to manage its public trust resources (including water) so as to derive the maximum benefit for its citizenry. 
\item The benefits to be considered and balanced include economic, recreational, aesthetic and environmental; if at any time the trustee determines that a use of water other than the then current use would better serve the public trust, the State has the power and the obligation to reallocate that water in accordance with the public's interest. 
\item Even if the water at issue has been put to beneficial use (and relied upon) for decades, it can be taken from one user in favor of another need or use. 
\item The public trust doctrine therefore means that no water rights in California are truly "vested" in the traditional sense of property rights

\item Water Contracts, Districts and Mutual Water Companies 
\begin{itemize}
\item At least in theory, all water used in California is developed and diverted based on one or more of the basic rights described above. \item However, it is common for the water rights relied upon by a water user to be held by another party, as in the case of water users receiving water from a district or mutual water company. 
\item In fact, most water users in California probably do not hold th e water rights underlying much of their water supply. 
\item Nevertheless, those water users have a rig ht to receive water separate and distinct from the water rights which support the diversion of the water in question. 
\item Some water suppliers hold the rights to the water they deliver, while many others must ac quire water from the ultimate water rights holder and themselves own nothing more than a contract right. 
\item For example, many older districts were formed in order to acquire water rights, and the districts themselves therefore hold the water rights which produce the water they distribute. 
\item Conversely, the United States is the record holder of the water rights used to operate the Central Valley Project; districts receiving CVP water supplies simply contract with the United States and distribute their contract supplies to their water users
\item In many (but not all) districts which provide agricultural water supplies, the right of a landowner to receive a share of the district's water supply is a matter of statute which accrues automatically by virtue of land ownership.
\item No additional documentation is required. In other situation s, a formal contractual relationship between the district and the water user is established, and the contract (rather than a statute) establishes the scope of the water user's right to receive a portion of the district's water supply. 
\item Districts currently have broad discretion relative to the use and transferability by water users of water they distribute; 
\item However, there are ongoing legislative efforts to grant water users more freedom to transfer district water allocated to them without the consent of the district, effectively transforming district water allocations into the personal property of each water user. (Most CVP water users believe themselves to actually be the beneficial owners of the water rights underlying CVP ope rations, and that the United States is merely a trustee for those rights holding bare legal title. That important distinction is beyond the scope of these materials.)
\item In the case of mutual water companies, the right to receive water from the company follows stock ownership. 
\item Mutual water company stock can be either appurtenant to the land in the company's service are or completely separate therefrom. 
\item Generally, the stock of mutual water companies formed within the past 25 years is appurtenant to the lands served and passes with conveyances of that land (although separate assignments of stock should still be prepared). 
\item For many older mutual water companies, the stock (and thus the right to receive water) is completely separate from the land served, and separate stock assignments are required to transfer the right to receive water evidenced by shares. 
\item As with districts, mutual water companies currently can control transfers of water allocated to shareholders, but could have that authority significantly curtailed by legislation granting water users rights to transfer water allocations over the objection of water sup pliers.
\end{itemize}
\end{enumerate}
\end{itemize}
\end{enumerate}
\end{document}