
\chapterimage{WaterChapterImage1}
\chapter{Drinking Water Systems}

\item Each Public Water System is required to have domestic water supply permits and is responsible for providing affordable, safe drinking water to their customers 24 hours a day, 7 days a week, 365 day a year.

\item All public water systems are subject to the same health based standards and laws whether they are a big city, a small community, or a rural restaurant. However, there are some minor adjustments that are made to monitoring frequencies based on population and water system type.

\item Requirements of a Public Water Agency in California include:

\begin{itemize}

\item Permitting engineering and technical reports, including pump tests, at least two water supply well sources for communities, a 50-foot radius source protection zone around all new wells, a minimum of a 50-foot annular seal on new wells , a well flow meter and initial monitoring \\
\item  Construction, including elevated storage or backup electricity for pumps to maintain 40 pounds per square inch (psi) minimum pressure at all times, proper construction of distribution systems, adequate storage capacity and fire fighting capacity \\
\item  As-built maps.\\
\item  Annual water-treatment chemicals and equipment for distribution monitoring of any added chemical treatment (dependent on the type of needed treatment) \\
\item  Ongoing raw water chemical monitoring sampling and analysis. \\
\item  Ongoing raw water bacteriological monitoring sampling and analysis. \\
\item  Ongoing treated water bacteriological monitoring sampling and analysis.\\
\item  Maintenance of bacteriological plans and emergency notification plans for water quality emergencies . \\
\item  Ongoing lead and copper monitoring including sampling and analysis and maintenance of a lead and copper plan. \\
\item  Ongoing disinfection byproducts monitoring and maintenance of an associated plan.\\
\item  Maintaining a customer water quality complaint program.\\
\item  Main flushing, valve and meter maintenance, and maintaining system maps.\\
\item  Cross connection program and annual back-flow device testing.\\
\item  Licensed water treatment operator and distribution staff.\\
\item  Written procedures for system maintenance, for example pipeline break procedures, etc.\\
\item  Source capacity planning studies and permit amendments for any additional growth.\\
\item  Annual Consumer Confidence Report preparation and distribution. Requirements continue on next page.\\
\item  Annual Electronic Report submittal to State Water Resource Control Board-Division of Drinking Water \\
\item  Records of the estimated life of all pumps, treatment, storage, and distribution system and an annual capital improvement plan to fund infrastructure replacement.\\
\item  Metering and billing staff.\\
\item  Emergency reserves for drought, regulatory changes, public notice of bacteriological or chemical failures, etc. (CHSC §116540) \\
\item  Maintaining of business licenses, annual drinking water permit fees and payment of any State enforcement fees for actions resulting from water system non-compliance .\\
\item  Appropriate working area for staff, chemicals, and records \\
\item  Insurance and liability for staff, with duties including climbing tanks, handling hazardous chemicals, etc. \\
\item  Management staff that is knowledgeable about drinking water. Staff coordinate the above and maintain financial controls.\\
\item  If the source is surface water, there may be additional requirements: o A water treatment plant meeting all Surface Water Treatment Rule requirements  Continuous operator supervision of the water treatment plant when in service o Chemical monitoring equipment, at minimum turbidity and chlorine  o Operations Plan and Alarms. Monthly monitoring reports to the Division of Drinking Water o Additional raw water sampling requirements.



\end{itemize}

\item Different types of water systems have different treatment requirements. Water systems are classified on this basis. Regulatory requirements vary from one class to another, and operator certifications are specific to certain classifications of systems.

\item Types of Public Water Systems
\begin{enumerate}
\item Community Water Systems:\\
These include city, county, regulated utilities, regional water systems and even small water companies and districts where people live.  The Community Water Systems can be either:
\begin{enumerate}
\item Small Water Systems - Water systems that serve 3,300 persons or fewer.
\item Large Water Systems - Water systems that serve more than 3,300 people.  For certain specific regulations, a system must serve more than 10,000 people or 50,000 to be considered a “Large Water System.”  Large water systems have to meet more stringent monitoring requirements under certain regulations.
\end{enumerate}

\item Noncommunity Water Systems:\\
The Noncommunity water system include:
\begin{itemize}
\item Nontransient water systems are places like schools and businesses that provide their own water. The same people have a regular opportunity to consume the water, but they do not reside there.

\item Transient water systems include entities like rural gas stations, restaurants and State and National parks that provide their own potable water source.  Most people that consume the water neither reside nor regularly spend time there.
\end{itemize}
\end{enumerate}

\end{itemize}
