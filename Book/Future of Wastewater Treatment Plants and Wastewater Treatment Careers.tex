\chapterimage{OCSDHeadworksScrubbersBW.jpg}
\chapter{Wastewater Treatment - Future and Careers}
\begin{itemize}

\item Wastewater treatment plants are the biggest energy consumers in a community. Biological aeration accounts for
60\% to 80\% of a wastewater treatment plant’s total energy consumption.

\item What is a Distributed System?
There are many ways in which to define a distributed system. Typically, distributed systems are in different geographical locations, but are linked to a central system either physically, or by management. The most likely is a “distributed management” scenario, wherein distributed management provides the opportunity for overall single-entity management of disparate or remote systems. “Distributed wastewater management is an approach to wastewater collection, treatment, and disposition (discharge, reuse, dispersal) that uses appropriately scaled systems—which can vary from onsite to cluster to centralized—across a service area, watershed, or other political or natural boundary.” [D’Amato, et al., p. 3]
An array of decentralized wastewater technologies are considered and implemented in small to mid-sized municipalities, as well as large municipalities, and in new land development projects. These technologies can supplement service areas for municipalities that have an existing centralized wastewater system. This application of multiple systems under a single management entity is called distributed wastewater management. [WEF, Kreissl, et al., p.2]
On the other hand, a decentralized system can be located in a different geographical location, but is not linked physically, or is not managed under the umbrella of a centralized system.
Effectively planned, implemented, and managed distributed or decentralized water systems are critical elements of sustainable infrastructure in the United States. These systems can be in rural or urban settings and range from small systems found on homeowner properties to small-system water resource recovery facilities (average daily flow of less than 1 MGD and serving a population of less than 10,000). The systems can be either discharging (surface or subsurface) or reuse systems. As noted in Charting New Waters, using the term ‘distributed’ in an urban environment places the focus on what the systems are instead of what they are not as the term ‘decentralized’ does. [The Johnson Foundation at Wingspread, p.3]
\end{itemize}

decentralized wastewater treatment systems
decentralized/distributed systems of wastewater
management are emerging as a new paradigm complimentary to conventional centralised
approaches. Decentralized/distributed systems not only offer a cost effective way of treating
sewage close the source but also offer the possibility of reuse of alternative water supplies for
non-potable use (eg. toilet flushing, gardening, laundry). Although there is increasing numbers of
decentralised systems being installed, not all regulators have kept pace with the rate of change,
with decentralized/distributed systems being installed ahead of institutional support by regulatory
bodies. In this paper we report on a project aimed to fill the gap in informing and supporting
potential innovators in developing and operationalising these systems

Sewage infrastructure is typically highly centralized, with significant ‘sunk’ investments so that
marginal extensions can be seen to make sense from a short term financial perspective.

The process of shifting from the use of predominantly centralized sewerage infrastructure to
alternative scales of service provision is challenging. Least of all because mature socio-technical
systems such as sanitation are part of a broad and complex system in which incumbent
technologies, associated infrastructure, organisations, institutions and social habits of practice
have evolved into an intertwined and interdependent system (Unruh 2000) favouring existing
technologies. Therefore decentralized systems, with radically different environmental and social
sustainability potentials do not ‘fit’ within the environment of the existing sanitation regime.
Therefore any significant shift toward the adoption of decentralized sewage infrastructure will
therefore need to consider not only the introduction of new technology but also the development
of supportive institutional arrangements and the development of different socio-cultural habits of
use.

distributed infrastructure refers to situations where the scale of the physical
infrastructure is small, serving from tens of households to a few thousand households

Distributed systems are in different geographical locations, but are linked to a central system either physically, or by management.
Decentralized systems can be located in a different geographical location, but are not linked physically, or are not managed under the umbrella of a centralized system.


Value of
Distributed or D ecentralized
Systems
Economic – Economic challenges to constructing a conventional, centralized facility in a rural setting include difficulties caused by terrain, climate, lack of personnel, and an inability to achieve the economies of scale needed to support a centralized facility. Urban communities must address the infrastructure, maintenance and energy costs of extending the collection systems network to connect outer city developments to a centralized facility. Distributed or decentralized systems can be a lower-cost alternative due to smaller infrastructure and reduced energy, operations and maintenance costs. Distributed or decentralized systems can be ‘modular’ in nature and allow communities to increase treatment capacity as the community grows thereby avoiding larger up-front financing costs. Decentralized collection systems generally come at a lower cost.
Environmental
Distributed or decentralized systems can mitigate aquifer depletion. Effluent is either discharged or reused in the watershed in which it originated. They also help to maintain a community’s desired land use patterns. Finally, they can be less energy intensive. The smaller size of distributed or decentralized systems inherently results in less consumption of energy. However, lower energy consumption is also a result of smaller distances over which wastewater is conveyed, reducing pumping needs of the facility.
The relatively compact size of a small distributed or decentralized system allows communities more say in where the system is located and thus lends to better integration with and less disruption to the landscape.
Like centralized systems, distributed or decentralized systems can provide environmental benefits, such as nutrient and pathogen removal and water reuse opportunities through the implementation of ONWS and DPR technologies. Distributed or decentralized systems and small water resource recovery facilities can be designed to remove phosphorus and nitrogen before the effluent is returned to the environment. Nutrient removal technologies can be added to lagoon systems and even conventional septic systems can achieve significant nutrient removal through drainfield design. Distributed or decentralized systems can provide water for direct potable reuse and non-potable water in both rural and urban settings for purposes such as flushing, cooling and heating, landscaping, and subsurface irrigation drip. The New York City Solaire building is an early example of distributed system within a heavily urbanized area effectively rendering the complex as a small community for the purpose of non-potable water reuse.
Technical/Logistical
Due to the smaller size of decentralized systems, installation and implementation can be less intensive. The smaller footprint results in easier layout and siting of the system; alternative sewers can be placed in ground at shallower depths; and these systems can be used in challenging terrain often being routed around obstacles or following the contour of the land.
Sustainability/Resilience – The economic, environmental and technical advantages of small distributed or decentralized systems increase a community’s sustainability and resilience through the use of alternate water sources, less need of potable water for non-potable uses, a reduced Copyright © 2019 Water Environment Federation, Water Science \& Engineering Center.
All Rights Reserved. WSEC-2019-FS-012 Page 3 of 4
strain on wastewater systems, energy conservation, and replenishment of the local aquifer.
One recent phenomenon is new housing development that incorporates urban-like housing densities with significant open or common areas. These “conservation style” developments are generally best served by alternative collection systems (ACSs), because distributed or decentralized systems can be built in nearby open spaces for treated wastewater dispersal and/or reuse. This frees such developments from being dependent on costly extensions of existing sewer systems. [Kreissl, p. 27]

Decentralized
wastewater treatment systems have the potential to provide a critical element of infrastructure,
wastewater treatment, in a cost-effective and beneficial manner and also serve as a tool to
encourage and enable green development and more efficient land use. Experience with these
systems in several southeastern states has demonstrated their benefits and have provided models
of sound management to assure successful operation at an affordable cost.



Wastewater Careers

The wastewater industry offers opportunities for well paid and stable employment, which comes with the added perk of  gratification being an environmental steward and protecting the health and safety of the public.

will result in growth of treatment capacity, an expansion of
the collection/distribution systems, and a demand for the professionals required to address these
issues