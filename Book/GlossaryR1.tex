\documentclass{article}
%\usepackage[english]{babel}%
\usepackage{graphicx}
\usepackage{tabulary}
\usepackage{tabularx}
\usepackage[normalem]{ulem}
\usepackage{cancel}
\usepackage{tikz} 
\usepackage{pdflscape}
\usepackage{soul}
\usepackage{colortbl}
\usepackage{lastpage}
\usepackage{multirow}
\usepackage{enumerate}
\usepackage{color,soul}
\usepackage{pdflscape}
\usepackage{hyperref}
%\usepackage[table]{xcolor}
\usepackage{rotating}
\usepackage{amsmath}
\usepackage{fixltx2e}
\usepackage{framed}
\usepackage{mdframed}
\usepackage[T1]{fontenc}
\usepackage[utf8]{inputenc}
\usepackage{textcomp}
\usepackage{siunitx}
\usepackage{ifthen}
\usepackage{fancyhdr}
\usepackage{gensymb}
 \usepackage{newunicodechar}
\usepackage[document]{ragged2e}
\usepackage[margin=1in,top=1.2in,headheight=57pt,headsep=0.1in]
{geometry}
\usepackage{ifthen}
\usepackage{fancyhdr}
\usepackage[margin=1in,top=1.2in,headheight=57pt,headsep=0.1in]
{geometry}
\usepackage{ifthen}
\usepackage{fancyhdr}
\everymath{\displaystyle}
\usepackage[document]{ragged2e}
\usepackage{fancyhdr}
\everymath{\displaystyle}
%\usepackage[table,xcdraw]{xcolor}
\usetikzlibrary{arrows}
\linespread{2}%controls the spacing between lines. Bigger fractions means crowded lines%
%\pagestyle{fancy}
%\usepackage[margin=1 in, top=1in, includefoot]{geometry}
%\everymath{\displaystyle}
\linespread{1.3}%controls the spacing between lines. Bigger fractions means crowded lines%
%\pagestyle{fancy}
\pagestyle{fancy}
\setlength{\headheight}{56.2pt}


\chead{\ifthenelse{\value{page}=1}{\textbf \textbf Collections}}
\rhead{\ifthenelse{\value{page}=1}{Shabbir Basrai}{Shabbir Basrai}}
\lhead{\ifthenelse{\value{page}=1}{WATR 048 - Spring 2019}{\textbf Collections}}
\rfoot{\ifthenelse{\value{page}=1}{}{WATR 048 - Spring 2019}}

\cfoot{}
\lfoot{Page \thepage\ of \pageref{LastPage}}
\renewcommand{\headrulewidth}{2pt}
\renewcommand{\footrulewidth}{1pt}
\begin{document}
AA - Activated alumina
\vspace{0.3cm}\\
ABR - Anaerobic baffled reactor - improved septic tank with baffles
\vspace{0.3cm}\\
ABR:  Anaerobic baffled reactor:  improved septic tank with baffles
\vspace{0.3cm}\\
AC:  Alternating current
\vspace{0.3cm}\\
ACP - Anaerobic contact process
\vspace{0.3cm}\\
ACP:  Anaerobic contact process
\vspace{0.3cm}\\
AD - Anaerobic digestion
\vspace{0.3cm}\\
AD:  Anaerobic digestion
\vspace{0.3cm}\\
ADI:  Acceptable daily intake
\vspace{0.3cm}\\
ADWF:  Average Dry Weather Flow
\vspace{0.3cm}\\
AF - Anaerobic filters
\vspace{0.3cm}\\
Ag:  Silver
\vspace{0.3cm}\\
AOC:  Assimilable Organic Carbon
\vspace{0.3cm}\\
AOS:  Adult onsite exposure
\vspace{0.3cm}\\
AQMD:  Air Quality Management District
\vspace{0.3cm}\\
As - Arsenic
\vspace{0.3cm}\\
AS:  Activated Sludge
\vspace{0.3cm}\\
As:  Arsenic
\vspace{0.3cm}\\
ATS - Aerobic treatment system
\vspace{0.3cm}\\
ATS:  Aerobic treatment system
\vspace{0.3cm}\\
AWT, AWWT:  Advanced Wastewater Treatment
\vspace{0.3cm}\\
AWWA:  American Water Works Association
\vspace{0.3cm}\\
AWWARF:  American Water Works Association Research Foundation (www.awwarf.com)
\vspace{0.3cm}\\
AWWF:  Average Wet Weather Flow
\vspace{0.3cm}\\
BAC:  Biological Activated Carbon (water treatment)
\vspace{0.3cm}\\
BACT:  Best Available Control Technology (Air Quality)
\vspace{0.3cm}\\
BAF:  Biological Aerated Filter (wastewater treatment)
\vspace{0.3cm}\\
BATNEEC:  Best Available Techniques Not Entailing Excessive Costs
\vspace{0.3cm}\\
BHP:  Brake Horse Power
\vspace{0.3cm}\\
BMP:  Best Management Practices
\vspace{0.3cm}\\
BOD :  Biochemical Oxygen Demand
\vspace{0.3cm}\\
BOD or BOD5 - Biological oxygen demand (measured for five days)
\vspace{0.3cm}\\
BOD:  Biochemical Oxygen on Demand:  
\vspace{0.3cm}\\
BOD:  Biological oxygen demand
\vspace{0.3cm}\\
BOD5:  Biochemical Oxygen Demand (over a 5 day period)
\vspace{0.3cm}\\
CAA:  Clean air act (EPA)
\vspace{0.3cm}\\
CBOD:  Carbonaceous Biochemical Oxygen Demand Chlorine
\vspace{0.3cm}\\
Cd:  Cadmium
\vspace{0.3cm}\\
CFC:  Chlorofluorocarbons
\vspace{0.3cm}\\
CFM:  Cubic Feet Per Minute
\vspace{0.3cm}\\
CFR:  Code of Federal Regulations
\vspace{0.3cm}\\
CFS:  Cubic Feet Per Second
\vspace{0.3cm}\\
CFU - Colony-forming unit
\vspace{0.3cm}\\
cfu:  Colony Forming Unit (Microbiology)
\vspace{0.3cm}\\
CHP:  Combined Heat and Power
\vspace{0.3cm}\\
CIP:  Capital Improvement Program
\vspace{0.3cm}\\
CJD:  Creutzfeld-Jakob's Disease
\vspace{0.3cm}\\
CN:  Cyanide
\vspace{0.3cm}\\
COD - Chemical oxygen demand
\vspace{0.3cm}\\
COD - Controlled open defecation
\vspace{0.3cm}\\
COD:  Chemical Oxygen Demand 
\vspace{0.3cm}\\
Cr: Chromium
\vspace{0.3cm}\\
CSI Conveyance System Improvement
\vspace{0.3cm}\\
CSO - Combined sewer overflow 
\vspace{0.3cm}\\
CSO LTCP:  CSO Long-Term Control Plan
\vspace{0.3cm}\\
CSO:  Combined sewer overflow
\vspace{0.3cm}\\
Cu: Copper
\vspace{0.3cm}\\
CVOC:  Chlorinated volatile organic compound
\vspace{0.3cm}\\
CW - Constructed wetland
\vspace{0.3cm}\\
DAF:  Dissolved Air Flotation
\vspace{0.3cm}\\
DBP:  Disinfection By-Product
\vspace{0.3cm}\\
DEWATS - Decentralized wastewater treatment system
\vspace{0.3cm}\\
DEWATS:  Decentralized wastewater treatment system
\vspace{0.3cm}\\
DFM:  Decennial flow monitoring
\vspace{0.3cm}\\
DMR:  Discharge Monitoring Report
\vspace{0.3cm}\\
DNA:  Deoxyribonucleic acid
\vspace{0.3cm}\\
DO: Dissolved Oxygen
\vspace{0.3cm}\\
DOC:  Dissolved organic carbon
\vspace{0.3cm}\\
EBCT:  Empty Bed Contact Time
\vspace{0.3cm}\\
EC - Electrical conductivity
\vspace{0.3cm}\\
EDI: Electro Deionization
\vspace{0.3cm}\\
Eh:  Redox potential
\vspace{0.3cm}\\
EIA:  Environmental impact assessment
\vspace{0.3cm}\\
EIS:  Environmental Impact Statement
\vspace{0.3cm}\\
EPA:  United States Environmental Protection Agency
\vspace{0.3cm}\\
EQO:  Environmental Quality Objective
\vspace{0.3cm}\\
EQS:  Environmental Quality Standard
\vspace{0.3cm}\\
ESA:  Endangered Species Act
\vspace{0.3cm}\\
ESA:  Environmentally Sensitive Area
\vspace{0.3cm}\\
F/M Ratio:  Food to Microorganism Ratio
\vspace{0.3cm}\\
FAC:  Florida Administrative Code
\vspace{0.3cm}\\
FAO:  Food and Agriculture Organization of the United Nations
\vspace{0.3cm}\\
FAQ:  Frequently asked questions
\vspace{0.3cm}\\
FC:  Fecal coliforms
\vspace{0.3cm}\\
FeCl2: Ferrous chloride
\vspace{0.3cm}\\
FeCl3:  Ferric chloride
\vspace{0.3cm}\\
FECR:  Fecal egg count reduction; closest article is on anthelmintic
\vspace{0.3cm}\\
FOG: Fats, Oils, \& Grease
\vspace{0.3cm}\\
FS:  Fecal (or fecal) sludge; closest article is septage
\vspace{0.3cm}\\
FSM:  Fecal (or fecal) sludge management
\vspace{0.3cm}\\
FSSM:  Fecal sludge and septage management
\vspace{0.3cm}\\
FSTP:  Fecal sludge treatment plant
\vspace{0.3cm}\\
FTI:  Fecally transmitted infections; closest article is fecal-oral transmission
\vspace{0.3cm}\\
GAC:  Granular Activated Carbon
\vspace{0.3cm}\\
GAP:  Good agricultural practices
\vspace{0.3cm}\\
GC:  Gas chromatography
\vspace{0.3cm}\\
GCMS:  Gas chromatograph + mass spectrometer
\vspace{0.3cm}\\
GFD:  Gals per foot of membrane per day 
\vspace{0.3cm}\\
GHG:  Greenhouse gases
\vspace{0.3cm}\\
GIS:  Geographical Information System
\vspace{0.3cm}\\
GLV:  Guideline Value (water quality standards)
\vspace{0.3cm}\\
GMO:  Genetically Modified Organism
\vspace{0.3cm}\\
gpcd:  Gallons per capita per day
\vspace{0.3cm}\\
GPD:  Gallons Per Day
\vspace{0.3cm}\\
gped:  Gallons per employee per day
\vspace{0.3cm}\\
GPM:  Gallons Per Minute
\vspace{0.3cm}\\
GPS - Global positioning system
\vspace{0.3cm}\\
GSA:  Gould Sludge Age
\vspace{0.3cm}\\
GSF - Global sanitation fund, closest page is Water Supply and Sanitation Collaborative Council
\vspace{0.3cm}\\
GSI:  Green Stormwater Infrastructure
\vspace{0.3cm}\\
HACCP:  Hazard Analysis and Critical Control Point
\vspace{0.3cm}\\
HACCP:  Hazard analysis and critical control points
\vspace{0.3cm}\\
HC:  hydrocarbons
\vspace{0.3cm}\\
HDPE:  High density polyethylene
\vspace{0.3cm}\\
HEDF:  Human excreta derived fertilizer (see reuse of excreta)
\vspace{0.3cm}\\
HIA:  Health impact assessment
\vspace{0.3cm}\\
HMI:  Human Machine Interface  
\vspace{0.3cm}\\
HP:  Horse Power
\vspace{0.3cm}\\
HRWS - Human right to water and sanitation
\vspace{0.3cm}\\
HWF - Handwashing facility
\vspace{0.3cm}\\
I/I:  Infiltration/Inflow
\vspace{0.3cm}\\
IPC :  Inclined Plate Clarifier
\vspace{0.3cm}\\
IPC:  Integrated Pollution Control
\vspace{0.3cm}\\
IPPC:  Integrated Pollution Prevention and Control
\vspace{0.3cm}\\
IPS:  Influent pump station
\vspace{0.3cm}\\
IWA:  International Water Association
\vspace{0.3cm}\\
IWSA:  International Water Supply Association
\vspace{0.3cm}\\
IWW:  Industrial Wastewater
\vspace{0.3cm}\\
JTU:  Jackson Turbidity Unit
\vspace{0.3cm}\\
kL (or Kl):  Kilo liters (or 1000 liters, same as 1 cubic meter)
\vspace{0.3cm}\\
KM:  Knowledge management
\vspace{0.3cm}\\
L:  Liter
\vspace{0.3cm}\\
LDPE:  Low density polyethylene
\vspace{0.3cm}\\
LMH:  Liters of permeate per square meter of membrane per hour
\vspace{0.3cm}\\
lpcd:  Liters per capita per day (liters per person per day), e.g. for daily wastewater flowrate
\vspace{0.3cm}\\
M\&E:  Monitoring and evaluation
\vspace{0.3cm}\\
MAC:  Maximum Admissible Concentration
\vspace{0.3cm}\\
MB:  Mixed Bed
\vspace{0.3cm}\\
MBBR:  Moving Bed Biofilm Reactor
\vspace{0.3cm}\\
MBR:  Membrane Bioreactor
\vspace{0.3cm}\\
MCC:  Motor Control Center
\vspace{0.3cm}\\
MCL:  Maximum Contaminant Level
\vspace{0.3cm}\\
MCRT:  Mean Cell Residence Time
\vspace{0.3cm}\\
MFC:  Microbial fuel cell
\vspace{0.3cm}\\
MFI:  Microfinance institution
\vspace{0.3cm}\\
mg - Milligram
\vspace{0.3cm}\\
MG/L:  Milligrams Per Liter
\vspace{0.3cm}\\
mg:  Milligram
\vspace{0.3cm}\\
MG:  Million Gallons
\vspace{0.3cm}\\
MGD:  Million Gallons Per Day
\vspace{0.3cm}\\
MHM:  Menstrual hygiene management; closest article is Menstrual hygiene day
\vspace{0.3cm}\\
ML:  Megaliter or 1 million liter or 1000 cubic meters
\vspace{0.3cm}\\
mld:  Megalitres per day
\vspace{0.3cm}\\
MLD:  Million liters per day
\vspace{0.3cm}\\
MLSS:  Mixed Liquor Suspended Solids
\vspace{0.3cm}\\
MLVSS:  Mixed Liquor Volatile Suspended Solids
\vspace{0.3cm}\\
MOA:  Memorandum of Agreement
\vspace{0.3cm}\\
MOR:  Monthly Operating Report
\vspace{0.3cm}\\
MoU:  Memorandum of understanding
\vspace{0.3cm}\\
MSW:  Municipal solid waste
\vspace{0.3cm}\\
MTBE:  Methyl-tert-butyl ether
\vspace{0.3cm}\\
NACWA:  National Association of Clean Water Agencies
\vspace{0.3cm}\\
NAPL:  Non Aqueous phase liquid
\vspace{0.3cm}\\
ND:  Not detected
\vspace{0.3cm}\\
NF:  Nanofiltration
\vspace{0.3cm}\\
NGO:  Non-governmental organization
\vspace{0.3cm}\\
NH3-N:  Ammonia Nitrogen
\vspace{0.3cm}\\
NIMBY:  Not In My Back Yard
\vspace{0.3cm}\\
NO2-N:  Nitrite Nitrogen 
\vspace{0.3cm}\\
NO3-N:  Nitrate Nitrogen
\vspace{0.3cm}\\
NOx:  Nitrogen Oxide
\vspace{0.3cm}\\
NPDES:  National Pollutant Discharge Elimination System
\vspace{0.3cm}\\
NPSH:  Net Positive Suction Head
\vspace{0.3cm}\\
NRW:  Non-revenue water
\vspace{0.3cm}\\
NSS:  Non-sewered sanitation (similar term to fecal sludge management)
\vspace{0.3cm}\\
NTU:  Nephelometer Turbidity Units
\vspace{0.3cm}\\
O \& G:  Oil and Grease
\vspace{0.3cm}\\
O\&M:  Operation and maintenance
\vspace{0.3cm}\\
OD:  Open defecation
\vspace{0.3cm}\\
ODF:  Open defecation free, i.e. a community without open defecation taking place
\vspace{0.3cm}\\
oF:  Fahrenheit
\vspace{0.3cm}\\
OPEX:  Operating Expenditure on a recurring annual basis
\vspace{0.3cm}\\
OPEX:  Operational expenditure (or operating expense)
\vspace{0.3cm}\\
ORP - Oxidation reduction potential
\vspace{0.3cm}\\
OSEC:  On-Site Electrolytic Chlorination
\vspace{0.3cm}\\
OUR:  Oxygen Uptake Rate
\vspace{0.3cm}\\
P\&ID: Process and Instrumentation Diagram
\vspace{0.3cm}\\
PAC:  Powdered Activated Carbon
\vspace{0.3cm}\\
PAH:  Polycyclic Aromatic Hydrocarbons
\vspace{0.3cm}\\
PC :  Personal computer
\vspace{0.3cm}\\
PCV:  Prescribed Concentration or Value
\vspace{0.3cm}\\
pe:  Population Equivalent
\vspace{0.3cm}\\
PFD:  Process flow diagram
\vspace{0.3cm}\\
pH:  Hydrogen potential
\vspace{0.3cm}\\
pH: Potential Hydrogen
\vspace{0.3cm}\\
PLC :  Programmable Logic Controller
\vspace{0.3cm}\\
POTW:  Publicly Owned Treatment Works
\vspace{0.3cm}\\
ppb: Parts Per Billion
\vspace{0.3cm}\\
PPE :  Personal Protective Equipment  
\vspace{0.3cm}\\
ppm - Parts per million
\vspace{0.3cm}\\
PPP - Public private partnership
\vspace{0.3cm}\\
ppt:  Parts per trillion
\vspace{0.3cm}\\
PRV:  Pressure Reducing Valve (water distribution)
\vspace{0.3cm}\\
PSI:  Pounds Per Square Inch
\vspace{0.3cm}\\
PTE:  Potentially Toxic Element
\vspace{0.3cm}\\
PVC:  Polyvinyl chloride
\vspace{0.3cm}\\
QA/QC:  Quality assurance / quality control
\vspace{0.3cm}\\
R\&D:  Research and development
\vspace{0.3cm}\\
RAS:  Return Activated Sludge
\vspace{0.3cm}\\
RBC: Rotating Biological Contactor
\vspace{0.3cm}\\
RBF:  Results-based financing, see also Output based aid or Payment by Results
\vspace{0.3cm}\\
RBTS:  Reed Bed Treatment System (wastewater treatment)
\vspace{0.3cm}\\
RCT:  Randomized controlled trial
\vspace{0.3cm}\\
RNA:  Ribonucleic acid
\vspace{0.3cm}\\
RO:  Reverse Osmosis
\vspace{0.3cm}\\
RRR:  Resource Recovery and Reuse
\vspace{0.3cm}\\
RTC:  Real Time Control
\vspace{0.3cm}\\
RTI:  Reproductive tract infection
\vspace{0.3cm}\\
RTTC:  Reinvent the Toilet Challenge, an R\&D funding scheme by the Bill and Melinda Gates Foundation
\vspace{0.3cm}\\
SA:  Sludge Age
\vspace{0.3cm}\\
SAR:  Sodium adsorption ratio
\vspace{0.3cm}\\
SBA (or SBM):  Swachh Bharat Abhiyan, Clean India Mission or Swachh Bharat Mission
\vspace{0.3cm}\\
SBR:  Sequencing Batch Reactor
\vspace{0.3cm}\\
SBR: Sequencing Batch Reactors
\vspace{0.3cm}\\
SCADA:  Supervisory Control and Data Acquisition
\vspace{0.3cm}\\
SDG:  Sustainable Development Goal
\vspace{0.3cm}\\
SDG6:  Sustainable Development Goal Number 6: "Clean Water and Sanitation"
\vspace{0.3cm}\\
SDI:  Sludge Density Index
\vspace{0.3cm}\\
SDWA:  Safe Drinking Water Act (US legislation)
\vspace{0.3cm}\\
SHG:  Self-help group
\vspace{0.3cm}\\
SI:  International system of Units
\vspace{0.3cm}\\
SLA:  Service-level agreement
\vspace{0.3cm}\\
SLB:  Service-level benchmarking
\vspace{0.3cm}\\
SME:  Small and Medium Sized Enterprises
\vspace{0.3cm}\\
SMS:  Short message service
\vspace{0.3cm}\\
SOP:  Standard operating procedure
\vspace{0.3cm}\\
SOUR:  Specific Oxygen Uptake Rate
\vspace{0.3cm}\\
Sox:  Sulfur Oxides
\vspace{0.3cm}\\
SRT - Solids retention time, see also Activated sludge
\vspace{0.3cm}\\
SRT:  Solids Retention Time
\vspace{0.3cm}\\
SS:  Suspended Solids (wastewater treatment)
\vspace{0.3cm}\\
SSSI:  Site of Special Scientific Interest
\vspace{0.3cm}\\
STP:  Sewage treatment plant
\vspace{0.3cm}\\
STW:  Sewage Treatment Works
\vspace{0.3cm}\\
SVI:  Sludge Volume Index
\vspace{0.3cm}\\
T90 - Time at which 90 percent reduction in pathogens is achieved
\vspace{0.3cm}\\
TAD:  Thermophilic Aerobic Digestion (wastewater sludge treatment)
\vspace{0.3cm}\\
TDS:  Total dissolved solids
\vspace{0.3cm}\\
THM:  Trihalomethane
\vspace{0.3cm}\\
ThOD - Theoretical oxygen demand
\vspace{0.3cm}\\
ThOD:  Theoretical oxygen demand
\vspace{0.3cm}\\
TKN: Total Kjeldahl Nitrogen
\vspace{0.3cm}\\
TMDL:  Total Maximum Daily Load
\vspace{0.3cm}\\
TOC:  Total Organic Carbon
\vspace{0.3cm}\\
TOC: Total Organic Carbon
\vspace{0.3cm}\\
TOMP: Toxic Organic Management Plan
\vspace{0.3cm}\\
TP:  Total Phosphorous 
\vspace{0.3cm}\\
TSS:  Total suspended solids 
\vspace{0.3cm}\\
TSS: Total Suspended Solids
\vspace{0.3cm}\\
TTHMs:  Total Trihalomethanes
\vspace{0.3cm}\\
TTO: Total Toxic Organics
\vspace{0.3cm}\\
TWL:  Top Water level
\vspace{0.3cm}\\
UASB:  Upflow anaerobic sludge blanket reactor
\vspace{0.3cm}\\
UCD:  User-centered design
\vspace{0.3cm}\\
UDDT:  Urine-diverting dry toilet
\vspace{0.3cm}\\
UDFT:  Urine Diverting Flush Toilet
\vspace{0.3cm}\\
UDT:  Urine diversion toilet
\vspace{0.3cm}\\
UF: Ultrafiltration
\vspace{0.3cm}\\
umho/cm:  Micromho per centimeter
\vspace{0.3cm}\\
uS/cm:  Microsiemens per centimeter
\vspace{0.3cm}\\
USEPA:  United States Environmental Protection Agency
\vspace{0.3cm}\\
UV:  Ultraviolet Radiation
\vspace{0.3cm}\\
UV: Ultraviolet
\vspace{0.3cm}\\
UVGI:  Ultraviolet germicidal irradiation
\vspace{0.3cm}\\
VFA:  Volatile Fatty Acid
\vspace{0.3cm}\\
VFD:  Variable Frequency Drive
\vspace{0.3cm}\\
VIP:  Ventilated improved pit latrine
\vspace{0.3cm}\\
VOC:  Volatile Organic Carbon
\vspace{0.3cm}\\
WAS:  Waste Activated Sludge
\vspace{0.3cm}\\
WASH or WaSH:  Water, sanitation and hygiene
\vspace{0.3cm}\\
WASH2:  Water, sanitation, hygiene and health
\vspace{0.3cm}\\
WC - Water closet
\vspace{0.3cm}\\
WC:  Water column
\vspace{0.3cm}\\
WEF - Water-Energy-Food nexus
\vspace{0.3cm}\\
WERF:  Water Environment Research Foundation
\vspace{0.3cm}\\
WHO - World Health Organization
\vspace{0.3cm}\\
WPM - Water point mapping
\vspace{0.3cm}\\
WRF:  Water Reclamation Facility
\vspace{0.3cm}\\
WRRF:  Water Recycle and Reclamation Facility
\vspace{0.3cm}\\
WSH - Water, sanitation, hygiene
\vspace{0.3cm}\\
WSSCC - Water Supply and Sanitation Collaborative Council
\vspace{0.3cm}\\
WSUP:  Water and sanitation for the urban poor
\vspace{0.3cm}\\
WTD - World Toilet Day
\vspace{0.3cm}\\
WTP:  Water Treatment Plant
\vspace{0.3cm}\\
WWD:  World Water Day
\vspace{0.3cm}\\
WWTF:  Wastewater Treatment Facility
\vspace{0.3cm}\\
WWTP:  Wastewater Treatment Plant
\vspace{0.3cm}\\
WYSIWYG - What You See is What You Get
\vspace{0.3cm}\\
Zn:  Zinc




\end{document}
