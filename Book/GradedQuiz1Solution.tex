\documentclass{article}
%\usepackage[english]{babel}%
\usepackage{graphicx}
\usepackage{tabulary}
\usepackage{tabularx}
\usepackage[table,xcdraw]{xcolor}
\usepackage{pdflscape}
\usepackage{lastpage}
\usepackage{multirow}
\usepackage{xcolor}
\usepackage{cancel}
\usepackage{amsmath}
\usepackage[table]{xcolor}
\usepackage{fixltx2e}
\usepackage[T1]{fontenc}
\usepackage[utf8]{inputenc}
\usepackage{ifthen}
\usepackage{fancyhdr}
\usepackage[document]{ragged2e}
\usepackage[margin=1in,top=1.2in,headheight=57pt,headsep=0.1in]
{geometry}
\usepackage{ifthen}
\usepackage{fancyhdr}
\everymath{\displaystyle}
\usepackage[document]{ragged2e}
\usepackage{fancyhdr}
\usepackage{mathabx}
\usepackage[shortlabels]{enumitem}
\everymath{\displaystyle}
\linespread{2}%controls the spacing between lines. Bigger fractions means crowded lines%
%\pagestyle{fancy}
%\usepackage[margin=1 in, top=1in, includefoot]{geometry}
%\everymath{\displaystyle}
\linespread{1.3}%controls the spacing between lines. Bigger fractions means crowded lines%
%\pagestyle{fancy}
\pagestyle{fancy}
\setlength{\headheight}{56.2pt}
\usepackage{soul}

\chead{\ifthenelse{\value{page}=1}{\includegraphics[scale=0.3]{BassettCTCLogo}\\ \textbf \textbf Graded Quiz \#1}}
\rhead{\ifthenelse{\value{page}=1}{Shabbir Basrai}{Shabbir Basrai}}
\lhead{\ifthenelse{\value{page}=1}{Water Treatment - October 2022}{\textbf Graded Quiz \#1}}
\rfoot{\ifthenelse{\value{page}=1}{}{Graded Quiz \#1}}

\cfoot{}
\lfoot{Page \thepage\ of \pageref{LastPage}}
\renewcommand{\headrulewidth}{2pt}
\renewcommand{\footrulewidth}{1pt}
\begin{document}

\section{Math Problems}
\begin{enumerate}
\item A sedimentation basin is 60 feet in diameter. What is the surface area of the tank?\\

\textbf{Solution:}
\vspace{0.2cm}
Surface Area=$\dfrac{\pi}{4}*\mathrm{D}^2=0.785*60^2 \mathrm{ft}^2 =\boxed{2,826 \mathrm{ft}^2}$
\vspace{0.2cm}


\item A rectangular cross section irrigation channel is 3.25 feet wide and is conveying a water flow of 3.5 MGD. The water flow is 8 inches deep. Calculate the velocity of this flow in ft/s.\\
Solution:\\
\vspace{0.3cm}
\includegraphics[scale=0.5]{ChannelFlow3}\\
$Q=V*A \implies V=\dfrac{Q}{A}$\\
$\implies \mathrm{V}\dfrac{\mathrm{ft}}{\mathrm{s}}=\dfrac{3.5\dfrac{\cancel{\mathrm{MG}}}{\cancel{\mathrm{day}}}*\dfrac{1000000\cancel{gal}}{\cancel{\mathrm{MG}}}*\dfrac{ft^{\cancel{3}}}{7.48\cancel{\mathrm{gal}}}*\dfrac{\cancel{\mathrm{day}}}{(1440*60)\mathrm{s}}}{(3.25*0.75)\cancel{\mathrm{ft}^2}}=\boxed{2.2\dfrac{\mathrm{ft}}{\mathrm{s}}}$
\vspace{0.5cm}

\item A circular tank has a diameter of 40 feet and is 10 feet deep. How many gallons will it hold?

\textbf{Solution}\\
Volume=Surface Area*Height$\implies (\dfrac{\pi}{4}*D^2=0.785*40^2 \mathrm{\enspace ft}^2 * 10 \mathrm{\enspace ft})*7.48 \mathrm{\enspace gallons} =\boxed{93,949 \mathrm{gallons}}$


\item A 50,000 gallon tank receives 250,000 gpd flow. What is the detention time in hours?\\

\textbf{Solution}\\

\vspace{0.2cm}
DT=$\dfrac{50,000 \enspace \mathrm{gallons}}{250,000 \enspace \dfrac{\mathrm{gallons}}{\mathrm{day}}*\dfrac{\mathrm{day}}{24 \enspace \mathrm{hrs}}}=\boxed{4.8 \enspace \mathrm{hours}}$\\
\vspace{0.2cm}

\item A tank is 44' in diameter and 22' high and is dosed with 50 ppm of chlorine. How many pound of 70\% HTH is needed?\\
$\mathrm{lbs}=\mathrm{Volume}{\mathrm{(MG)}}* \mathrm{Concentration}\dfrac{\mathrm{mg}}{\mathrm{l}}*8.34 \hspace{0.2cm} \implies $\\
\vspace{0.2cm}

$\Bigg(\Big( (0.785*44^2*22)\cancel{\mathrm{ft}^3}*\dfrac{7.48\cancel{\mathrm{gallon}}}{\cancel{\mathrm{ft}^3}}*\dfrac{\mathrm{MG}}{1,000,000\cancel{\mathrm{gallon}}}\Big)*50*8.34 \Bigg)*\dfrac{1 \enspace \mathrm{lb \enspace of \enspace 70\% \enspace HTH}}{0.7 \enspace \mathrm{lb \enspace HTH}}  =\boxed{149 \mathrm{lbs \enspace HTH}}$\\
\vspace{0.2cm}
$ \implies \mathrm{Concentration}\dfrac{\mathrm{mg}}{\mathrm{l}}=\dfrac{ \mathrm{lbs}}{\mathrm{Volume}\mathrm{(MG)}*8.34}=\dfrac{40 \enspace \mathrm{lbs}}{80 \enspace\mathrm{gallons}*\dfrac{\mathrm{MG}}{1,000,000 \enspace \cancel{\mathrm{gallons}}}*8.34}$
\vspace{0.2cm}


\item A flow of 2,200 gpm  is pumped against a total head of 14.0 feet. · The pump is 80\% efficient and the motor' is 85\% efficient. Calculate the brake Hp.\\


\vspace{0.4cm}\includegraphics[scale=0.08]{PumpProblem}\\
\vspace{0.3cm}
$\mathrm{pump \enspace efficiency}=\dfrac{\mathrm{water Hp}}{\mathrm{brake \enspace Hp}}$\\
$\implies \mathrm{brake \enspace Hp}=\dfrac{\mathrm{water Hp}}{\mathrm{pump \enspace efficiency}}=\dfrac{2,200\mathrm{GPM}*14\mathrm{ft}*\dfrac{\mathrm{Hp}}{3,960 \mathrm{GPM-ft}}}{0.8}=\boxed{10Hp}$
\end{enumerate}
\end{document}