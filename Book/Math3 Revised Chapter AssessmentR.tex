\chapterimage{QuizCover} % Chapter heading image

\chapter*{Chapter Assessment}
\section*{Practice Problems - Fractions}
\begin{enumerate}
\item Convert 22$\dfrac{1}{4}$ into a fraction
\item Express 10ft, 6in into fraction
\item Express 10ft, 6in into decimal
\end{enumerate}

\vspace{1cm}
\section*{Practice Problems - Decimals and Powers of Ten}\index{Math practice!Decimals and powers of ten}
\begin{enumerate}
\item Write the equivalent of 10,000,000 as a power of ten
\item Find the product of $3.4564*10^2$
\item Find the product of $534.567*10^{-2}$
\vspace{0.2cm}
\item Find the value of $\dfrac{165.93}{10^{-2}}$
\vspace{0.2cm}
\item Find the value of $0.023*10^4$
\end{enumerate}
\vspace{1cm}


\section*{Practice Problems - Rounding and Significant Digits}\index{Math practice!Rounding and significant digits}
Round the following to the nearest hundredths (the second place after the decimal).\\
A. $2.4568$\\
B. $27.2534$\\
C. $128.2111$\\
D. $364.8762$\\
E. $354.777777$\\
F. $34.666666$\\
G. $67.33333$\\
\vspace{0.5cm}
Round the following to the nearest tenths (the first place after the decimal).\\
A. $2.4568$\\
B. $27.2534$\\
C. $128.2111$\\
D. $364.8762$\\
E. $354.777777$\\
F. $34.666666$\\
G. $67.33333$


\vspace{0.5cm}

Round the following answers off to the most significant digit.\\

\begin{tabular}{|l|l|l|}
\hline
 & Problem & Accurate Answer \\
\hline
A. & $25.1+26.43$ &  \\
\hline
B. & $128.456-121.4$ &  \\
\hline
C. & $85-7.924328$ &  \\
\hline
D. & $8.564+5$ &  \\
\hline
\end{tabular}

\begin{tabular}{|l|l|l|}
\hline
 & Problem & Accurate Answer \\
\hline
A. & $26.34 \times 124.34567$ &  \\
\hline
B. & $23.58 \times 34.251$ &  \\
\hline
C. & $12,453 / 13.9$ &  \\
\hline
D. & $12,457.92 \times 3$&  \\
\hline
\end{tabular}

\section*{Practice Problems - Totalizing and Averages}\index{Math practice!Totalizing and averages}
\begin{enumerate}

\item Find the average of the following set of numbers:\\
$
\begin{aligned}
&0.2 \\
&0.2 \\
&0.1 \\
&0.3 \\
&0.2 \\
&0.4 \\
&0.6 \\
&0.1 \\
&0.3
\end{aligned}
$
\vspace{0.2cm}
\item The chemical used for each day during a week is given below. Based on these data, what was the average lb/day chemical used during the week?\\

\begin{tabular}{|l|l|}
\hline
Monday & 92 lb/day\\
\hline
Tuesday & 93 lb/day \\
\hline
Wednesday & 98 lb/day\\
\hline
Thursday & 93 lb/day \\
\hline
Friday & 89 lb/day\\
\hline
Saturday & 93 lb/day \\
\hline
Sunday & 97 lb/day\\
\hline
\end{tabular}
\vspace{0.2cm}
\item The average chemical use at a plant is 77 lb/day. If the chemical inventory is 2800 lbs, how many days supply is this?
\vspace{0.2cm}
\item A well pumped for 45 days. The beginning gallon meter reading was 7,456,400 and 45 days later the same meter was 15,154,400. What was the average flow in gallons per day?


\end{enumerate}
\section*{Practice Problems - Percentage}\index{Math practice!Percentage}
\begin{enumerate}
\item $25 \%$ of the chlorine in a 30-gallon vat has been used. How many gallons are remaining in the vat?\\

\item The annual public works budget is $\$ 147,450$. If $75 \%$ of the budget should be spent by the end of September, how many dollars are to be spent? How many dollars will be remaining?

\item A 75 pound container of calcium hypochlorite has a purity of $67 \%$. What is the actual weight of the calcium hypochlorite in the container? \\

\item $3 / 4$ is the same as what percentage?

\item A $2 \%$ chlorine solution is what concentration in $\mathrm{mg} / \mathrm{L}$ ?

\item A water plant produces 84,000 gallons per day. 7,560 gallons are used to backwash the filter. What percentage of water is used to backwash?

\item The average day winter demand of a community is 14,500 gallons. If the summer demand is estimated to be $72 \%$ greater than the winter, what is the estimated summer demand? Demand - When related to use, the amount of water used in a period of time. The term is in reference to the "demand" put onto the system to meet the need of customers.

\item The master meter for a system shows a monthly total of 700,000 gallons. Of the total water, 600,000 gallons were used for billing. Another 30,000 gallons were used for flushing. On top of that, 15,000 gallons were used in a fire episode and an estimated 20,000 gallons were lost to a main break that was repaired that same day. What is the total unaccounted for water loss percentage for the month?

\item Your water system takes 75 coliform tests per month. This month there were 6 positive samples. What is the percentage of samples which tested positive?
\end{enumerate}

\section*{Practice Problems - Ratio and Proportion}\index{Math practice!Ratio and proportion}
\begin{enumerate}
\item It takes 6 gallons of chlorine solution to obtain a proper residual when the flow is 45,000 gpd. How many gallons will it take when the flow is 62,000 gpd?

\item A motor is rated at 41 amps average draw per leg at $30 \mathrm{Hp}$. What is the actual $\mathrm{Hp}$ when the draw is 36 amps? C. 

\item If it takes 2 operators $4.5$ days to clean an aeration basin, how long will it take three operators to do the same job?

\item It takes 3 hours to clean 400 feet of collection system using a sewer ball. How long will it take to clean 250 feet?

\item It takes 14 cups of $\mathrm{HTH}$ to make a $12 \%$ solution, and each cup holds 300 grams. How many cups will it take to make a $5 \%$ solution?

\item A bike travelling at 5 miles/hr completes a journey in 40 minutes. How long would the same journey take if the speed was increased to 8 miles/hr?

\item Water is flowing at a velocity of $1.3 \mathrm{ft} / \mathrm{sec}$ in a 4.0 -in. diameter pipe. If the pipe changes from the 4.0-inch to a 3.0-in. pipe, what will the velocity be in the 3.0-in. pipe?\\

\end{enumerate}
\vspace{1cm}

\section*{Practice Problems - Area and Volume}\index{Math practice!Area and volume}
\begin{enumerate}

\item A 60-foot diameter tank contains 422,000 gallons of water. Calculate the height of water in the storage tank.

\item What is the volume of water in ft$^3$, of a sedimentation basin that is 22 feet long, and 15 feet wide, and filled to 10 feet?

\item What is the volume in ft$^3$ of an elevated clear well that is 17.5 feet in diameter, and filled to 14 feet?

\item What is the area of the top of a storage tank that is 75 feet in diameter?\\

\item  What is the area of a wall $175 \mathrm{ft}$. in length and $20 \mathrm{ft}$. wide?\\

\item  You are tasked with filling an area with rock near some of your equipment. 1 Bag of rock covers 250 square feet. The area that needs rock cover is 400 feet in length and 30 feet wide. How many bags do you need to purchase?\\

\item How many gallons of paint will be required to paint the walls of a 40 ft long x 65 ft wide x 20 ft high tank if the paint coverage is 150 sq. ft per gallon.  Note:  We are painting walls only.  Disregard the floor and roof areas.\\

\item A new 24" diameter pipe is to be installed with a pipe depth, to top of pipe, of 48" and a length of 12,000 feet The trench will be backfilled with sand. The trench walls will be excavated one foot wider than the pipe on each side and six inches below the pipe.  How much excavated material must be hauled away?

\item A chemical feed pump with a 6-inch bore and a 6-inch stroke pumps 60 cycles per minute. Find the pumping rate in gpm.
  

\end{enumerate}
\vspace{1cm}
\section*{Practice Problems - Flow and Velocity}\index{Math practice!Flow and velocity}
\begin{enumerate}

\item Flow in an 8-inch pipe is 500 gpm. What is the average velocity in ft/sec? (Assume pipe is flowing full)

\item A pipeline is 18” in diameter and flowing at a velocity of 125 ft. per minute. What is the flow in gallons per minute?

\item The velocity in a pipeline is 2 ft./sec. and the flow is 3,000 gpm. What is the diameter of the pipe in inches?



\item Find the flow in a 4-inch pipe when the velocity is $1.5$ feet per second.

  \item A 42-inch diameter pipe transfers 35 cubic feet of water per second. Find the velocity in $\mathrm{ft} / \mathrm{sec}$. 
  
  \item A plastic float is dropped into a channel and is found to travel 10 feet in $4.2$ seconds. The channel is $2.4$ feet wide and $1.8$ feet deep. Calculate the flow rate of water in cfs.
  
  \item A channel is 3.25 feet wide and is conveying a a flow of 3.5 MGD. The depth of the water is 8 inches. Calculate the velocity of this flow.\\

\end{enumerate}



\section*{Practice Problems - Unit Conversions}\index{Math practice!Unit conversions}
Convert the following:\\
\begin{enumerate}
\item Convert 1000 $ft^3$ to cu. yards\\

\item Convert 10 gallons/min to $ft^3$/hr\\

\item Convert 100,000 $ft^3$ to acre-ft.\\

\item Find the flow in gpm when the total flow for the day is 65,000 gpd.

\item Find the flow in gpm when the flow is $1.3 \mathrm{cfs}$.

\item Find the flow in gpm when the flow is $0.25 \mathrm{cfs}$.

\item The flow rate through a filter is 4.25 MGD. What is this flow rate expressed as gpm?\\

\item After calibrating a chemical feed pump, you've determined that the maximum feed rate is $178 \mathrm{~mL} /$ minute. If this pump ran continuously, how many gallons will it pump in a full day?

\item A plant produces 2,000 cubic foot of water per hour. How many gallons of water is produced in an 8-hour shift?

\item Change 70 °F to °C

\item Change 4 °C to °F
\end{enumerate}


\section*{Practice Problems - Concentration}\index{Math practice!Concentration}
\begin{enumerate}
\item What is the concentration in mg/l of  4.5\% solution of that substance.\\
\item How many lbs of salt needs to be dissolved in water to make 1 liter of 5\% salt solution?\\
\item An operator mixes 40 lb of lime in a 100-gal tank containing 80 gal of water. What is the percent of lime in the slurry?

\end{enumerate}

\vspace{1cm}
\section*{Practice Problems - Density and Specific Gravity}\index{Math practice!Density and specific gravity}
\begin{enumerate}

\item What is the specific gravity of a 1 ft$^3$ concrete block which weighs 145 lbs?

\item What is the specific gravity of a chlorine solution if 1 (one) gallon weighs 10.2lbs?

\item How much does each gallon of zinc orthophosphate weigh (pounds) if it has a specific gravity of 1.46?

\item How much does a 55 gallon drum of 25\% caustic soda weigh (pounds) if the specific gravity is 1.28?

\end{enumerate}



\section*{Practice Problems - Pounds Formula}\index{Math practice!Pounds formula}
\begin{enumerate}

\item A water treatment plant operates at the rate of 75 gallons per minute. They dose soda ash at 14 mg/L. How many pounds of soda ash will they use in a day?
\item What is the influent plant loading of phosphorus in lbs/day if the plant flow is 4.5 MGD and the influent phosphorous concentration is 1.5 mg/l?\\
\item A water treatment plant uses 8 pounds of chlorine daily and the dose is 17 mg/l. How
many gallons are they producing?
\end{enumerate}

\section*{Practice Problems - Chemical Dosing}\index{Math practice!Chemical dosing}
\begin{enumerate}

\item In order to disinfect a sedimentation basin measuring $20 \mathrm{ft}$ in width, 60 feet in length, and is 10 feet deep to obtain $50 \mathrm{ppm}$ would require how many lbs. of $65 \%$ available $\mathrm{HTH}$ ?\\

\item Determine the chlorinator setting (lb/day) required to treat a flow of $4 \mathrm{MGD}$ with a chlorine dose of $5 \mathrm{mg} / \mathrm{L}$.

\item A pipeline that is 12 inches in diameter and $1400 \mathrm{ft}$ long is to be treated with a chlorine dose of $48 \mathrm{mg} / \mathrm{L}$. How many lb of chlorine will this require?

\item What should the chlorinator setting be (lb/day) to treat a flow of $2.35 \mathrm{MGD}$ if the chlorine demand is $3.2 \mathrm{mg} / \mathrm{L}$ and a chlorine residual of $0.9 \mathrm{mg} / \mathrm{L}$ is desired?

\item A water treatment plant operates at the rate of 75 gallons per minute. They dose soda ash at 14 mg/L. How many pounds of soda ash will they use in a day?

\item A water treatment plant is producing 1.5 million gallons per day of potable water, and uses 38 pounds of soda ash for pH adjustment. What is the dose of soda ash at that plant?

\item A water treatment plant produces 150,000 gallons of water every day. It uses an
average of 2 pounds of permanganate for iron and manganese removal. What is the dose of the
permanganate? 


\end{enumerate}


\section*{Practice Problems - Blending and Dilution}\index{Math practice!Blending and dilution}

\begin{enumerate}
\item Ferric chloride is being added as a coagulant to the raw water entering a plant. Sampling
shows that the concentration of ferric in the raw water is 25 ppm. A quick check of the chemical
metering pump shows that it is operating at a flow rate of 4.3 gpm. If the flow through the water
plant is 800 gpm, what is the concentration of raw chemical in the dosing tank?

\item A water plant is fed by two different wells. The first well produces water at a rate of 600
gpm and contains arsenic at 0.5 mg/L. The second well produces water at a rate of 350 gpm and
contains arsenic at 12.5 mg/L. What is the arsenic concentration of the blended water?
\end{enumerate}

\section*{Practice Problems - Pumping Rates}\index{Math practice!Pumping rates}
\begin{enumerate}
\item How long will it take (hrs) to fill a 2 ac-ft pond if the pumping rate is 400 GPM?

\item A pump is set to pump 5 minutes each hour. It pumps at the rate of 35 gpm. How many gallons of water are pumped each day?\\

\item A pump operates 5 minutes each 15 minute interval.  If the pump capacity is 60 gpm, how many gallons are pumped daily?\\

\item Given the tank is 10ft wide, 12 ft long and 18 ft deep tank including 2 ft of freeboard when filled to capacity. How much time (minutes) will be required to pump down this tank to a depth of 2 ft when the tank is at maximum capacity using a 600 GPM pump\\

\end{enumerate}


\section*{Practice Problems - Pressure, Force and Head Relationship}\index{Math practice!Pressure, force and head relationships}
\begin{enumerate}
  \item A 42-inch main line has a shut off valve. The same line has a 10-inch bypass line with another shut-off valve. Find the amount of force on each valve if the water pressure in the line is 80 psi. Express your answer in tons.\\

  \item A water tank is 15 feet deep and 30 feet in diameter. What is the force exerted on a 6-inch valve at the bottom of the tank?\\

\item A water tank is filled to depth of 22 feet. What is the psi at the bottom of the tank?\\

\item The static pressure in a water main is 85 psi. What elevation of water is needed to provide that kind of pressure?\\

\item The pressure at the top of the hill is 62 psi. The pressure at the bottom of the hill, 60 feet below, is 100 psi. The water is flowing uphill at 120 gpm. What is the friction loss, in feet, in the pipe?\\
\end{enumerate}
\vspace{1cm}

\section*{Practice Problems - Pumping Power Requirements}\index{Math practice!Pumping power requirements}
\begin{enumerate}
  \item If a pump is operating at 2,200 gpm and 60 feet of head, what is the water horsepower? If the pump efficiency is 71\%, what is the brake horsepower?

\item The water horsepower of a pump is $10 \mathrm{Hp}$ and the brake horsepower output of the motor is $15.4 \mathrm{Hp}$. What is the efficiency of the pump?

\item The water horsepower of a pump is $25 \mathrm{Hp}$ and the brake horsepower output of the motor is $48 \mathrm{Hp}$. What is the efficiency of the pump?

\item The efficiency of a well pump is determined to be $75 \%$. The efficiency of the motor is estimated at $94 \%$. What is the efficiency of the well?

\item If a motor is $85 \%$ efficient and the output of the motor is determined to be 10 $\mathrm{BHp}$, what is the electrical horsepower requirement of the motor?

\item The water horsepower of a well with a submersible pump has been calculated at 8.2 WHp. The Output of the electric motor is measured as $10.3 \mathrm{BHp}$. What is the efficiency of the pump?

  \item Water is being pumped from a reservoir to a storage tank on a hill. The elevation difference between water levels is 1200 feet. Find the pump size required to fill the tank at a rate of 120 gpm. Express your answer in horsepower.

  \item A $25 \mathrm{hp}$ pump is used to dewater a lake. If the pump runs for 8 hours a day for 7 days a week, how much will it cost to run the pump for one week? Assume energy costs $\$ 0.07$ per kilowatt hour.

  \item A pump station is used to lift water 50 feet above the pump station to a storage tank. The pump rate is $500 \mathrm{gpm}$. If the pump has an efficiency of $85 \%$ and the motor has an efficiency of $90 \%$, find each of the following: Water Horsepower, Brake Horsepower, Motor Horsepower, and Wire-to-Water Efficiency.


  \item Find the brake horsepower for a pump given the following information: Total Dynamic Head $=75$ feet, Pump Rate $=150$ gpm, Pump Efficiency $=90 \%$, Motor Efficiency $=85 \%$

  \item Water is being pumped from a reservoir to a storage tank on a hill. The elevation difference between water levels is 1200 feet. Find the pump size required to fill the tank at a rate of 120 gpm. Express your answer in horsepower.

\end{enumerate}






\section*{Practice Problems - Well Hydraulics}\index{Math practice!Well hydraulics}
\begin{enumerate}

\item A well yields 2,840 gallons in exactly 20 minutes. What is the well yield in gpm?\\

\item Before pumping, the water level in a well is 15 ft. down. During pumping, the water level is 45 ft. down. The drawdown is:\\


\item A well produces 365 gpm with a drawdown of 22.5 ft.  What is	the specific yield in gallons per minute per foot?\\


\item A well is located in an aquifer with a water table elevation 20 feet below the ground surface. After operating for three hours, the water level in the well stabilizes at 50 feet below the ground surface. The pumping water level is:\\


\item Calculate drawdown, in feet, using the following data:\\
The water level in a well is 20 feet below the ground surface when the pump is not in operation, and the water level is 35 feet below the ground surface when the pump is in operation.\\


\item Calculate the well yield in gpm, given a drawdown of 14.1 ft and a specific yield of 31
gpm/ft.\\

\end{enumerate}


\section*{Practice Problems - Sedimentation}\index{Math practice!Sedimentation}


\begin{enumerate}

\item Calculate the detention time for a sedimentation tank that is 48 feet wide, 210 feet long and 9 feet deep with a flow of 5 MGD.\\

\item  At a 2.5 MGD wastewater treatment plant the primary clarifier has a detention time of 2 hours. How many gallons does this clarifier hold?\\

\item A circular clarifier has a diameter of 80 ft. If the flow to the clarifier is 1800 gpm, what is the surface overflow rate in gpm/ft

\item A sedimentation basin 70 ft by 25 ft receives a flow of 1000 gpm. What is the surface overflow rate in gpm/ft$2$?


\item A circular clarifier receives a flow of 3.55 MGD. If the diameter of the weir is 90 ft, what is the weir loading rate in gpm/ft?
\end{enumerate}


\section*{Practice Problems - Filtration}\index{Math practice!Filtration}

\begin{enumerate}
	\item At an average flow of 4,000 gpm, how long of a filter run in hours would be required to produce 25 MG of filtered water?
	
	\item A filter is $40 \mathrm{ft}$ long by $20 \mathrm{ft}$ wide. During a test of flow rate, the influent valve to the filter is closed for 6 minutes. The water level drop during this period is 16 inches. What is the filtration rate for the filter in $\mathrm{gpm} / \mathrm{ft}^{2}$ ?\\

\item A water treatment plant treats 6.0 MGD with four filters. Each filter use 60,000 gallons per wash. What is the percent backwash at the plant?\\

\item A treatment plant filter washes at a rate of 10,000 GPM. The filter measures 18ft. wide by 24ft. long. What is the rate of rise expressed in inches per minute?\\


\item If a filter measures 20 feet by 30 feet by 7 foot deep and the backwash flow is $3.5 \mathrm{cuft} / \mathrm{sec}$, what is the backwash rate?\\

\end{enumerate}
