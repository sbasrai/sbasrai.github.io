\chapterimage{MathCover.png}
\chapter{Water Math}

\section*{Practice Problems - Fractions}
\begin{enumerate}

\item Convert 22$\dfrac{1}{4}$ into a fraction\\
Solution:\\
$=\dfrac{22*4 + 1}{4}=\boxed{\dfrac{89}{4}}$

\item Express 10ft 6in as a fraction\\
Solution:\\
$6" = 6 in * \dfrac{ft}{12 \enspace in} = \dfrac{6}{12}=\dfrac{1}{2}ft \enspace or \enspace 0.5ft\implies 10ft 6in = 10\dfrac{1}{2}ft=\dfrac{10*2+1}{2}=\boxed{\dfrac{21}{2}}$\\
$Alternatively: 10 ft \enspace 6" \enspace is \enspace 10 ft \enspace + \enspace 0.5 ft= 10.5 ft \implies 10.5=\dfrac{105}{10} = \dfrac{\cancel{5}*21}{\cancel{5}*2}=\boxed{\dfrac{21}{2}}$
\item Express 10ft, 6in as decimal\\
Solution:\\
$10 ft \enspace 6" \enspace is \enspace 10 + 0.5 = \boxed{10.5 ft}$
\item Add $\dfrac{3}{4}+\dfrac{1}{7}$\\
For the two fraction $\dfrac{3}{4} \enspace \dfrac{1}{7} $
Lowest common denominator would be 4*7=28\\
\vspace{0.2cm}
Write each of the fractions as a fraction with a denominator of 28 by multiplying with a fraction which equals to 1\\
\vspace{0.2cm}
$\dfrac{3}{4}*\dfrac{7}{7}+\dfrac{1}{7}*\dfrac{4}{4}=\dfrac{21}{28}+\dfrac{4}{28}$\\
\vspace{0.2cm}
As both fractions have a common denominator, the numerators can now be added\\
\vspace{0.2cm}
$\dfrac{21}{28}+\dfrac{4}{28}=\dfrac{21+4}{28}=\boxed{\dfrac{25}{28}}$\\
\item Multiply $\dfrac{4}{9}*\dfrac{3}{16}$\\
Solution:\\
$\dfrac{4}{9}*\dfrac{3}{16} \implies \dfrac{\cancelto{1}{4}}{\cancelto{3}{9}}*\dfrac{\cancelto{1}{3}}{\cancelto{4}{16}}=\boxed{\dfrac{1}{12}}$\\
\end{enumerate}
\section*{Practice Problems - Decimals and Powers of Ten}
\begin{enumerate}
\item Write the equivalent of 10,000,000 as a power of ten\\
Solution:\\
$\boxed{10^7}$
\item Find the product of $3.4564*10^2$\\
Solution:\\
$\boxed{345.64}$
\vspace{0.2cm}
\item Find the product of $534.567*10^{-2}$\\
Solution:\\
$\boxed{5.34567}$
\vspace{0.2cm}
\item Find the value of $\dfrac{165.93}{10^{-2}}$\\
Solution:\\
$\boxed{1.6593}$
\vspace{0.2cm}
\item Find the value of $0.023*10^4$\\
Solution:\\
$\boxed{230}$
\end{enumerate}
\newpage
\section*{Practice Problems - Totalizing and Averages}
\begin{enumerate}
\item Find the average of the following set of numbers:\\
$
\begin{aligned}
&0.2 \\
&0.2 \\
&0.1 \\
&0.3 \\
&0.2 \\
&0.4 \\
&0.6 \\
&0.1 \\
&0.3
\end{aligned}$\\
\vspace{0.2cm}
Solution:\\
\vspace{0.2cm}
$\dfrac{0.2+0.2+0.1+0.3.+0.2+0.4+0.6+0.1+0.3}{9}=\boxed{2.67}$
\vspace{0.2cm}
\item The chemical used for each day during a week is given below. Based on these data, what was the average lb/day chemical used during the week?\\

\begin{tabular}{|l|l|}
\hline
Monday & 92 lb/day\\
\hline
Tuesday & 93 lb/day \\
\hline
Wednesday & 98 lb/day\\
\hline
Thursday & 93 lb/day \\
\hline
Friday & 89 lb/day\\
\hline
Saturday & 93 lb/day \\
\hline
Sunday & 97 lb/day\\
\hline
\end{tabular}\\
\vspace{0.3cm}
Solution:\\
\vspace{0.2cm}
$\dfrac{92+93+98+93+89+93+97}{7}=\boxed{93.6}$
\vspace{0.2cm}
\item The average chemical use at a plant is 77 lb/day. If the chemical inventory is 2800 lbs, how many days supply is this?\\
Solution:\\
$2800 \enspace \cancel{lbs} * \dfrac{day}{77 \enspace \cancel{lb}}=\boxed{36 \enspace days}$
\end{enumerate}

\section*{Practice Problems - Percentage}


\begin{enumerate}
\item $25 \%$ of the chlorine in a 30-gallon vat has been used. How many gallons are remaining in the vat?\\
Solution:\\
Amount of chlorine remaining in the vat is 100\%-25\%=75\%\\

Gallons of chlorine remaining in the vat: $30*0.75=\boxed{22.5 \enspace gallons}$


\item The annual public works budget is $\$ 147,450$. If $75 \%$ of the budget should be spent by the end of September, how many dollars are to be spent? How many dollars will be remaining?\\
\vspace{0.2cm}
Solution:\\
Amount to be spent = \$147,450*0.75 = $\boxed{\$110,812.50}$\\
\vspace{0.2cm}
Amount remaining = \$ 147,450 - 110,812.50 = $\boxed{\$36,367.50}$

\item A 75 pound container of calcium hypochlorite has a purity of $67 \%$. What is the actual weight of the calcium hypochlorite in the container? \\
\vspace{0.2cm}
Solution:\\
Note: Calcium Hypochlorite can be written as Ca(OCl)$_2$\\
$75 \enspace lbs \enspace Ca(OCl)_2 \enspace - \enspace product \enspace in \enspace container*\dfrac{0.67 \enspace lbs \enspace Ca(OCl)_2 }{lb \enspace Ca(OCl)_2  \enspace - \enspace product \enspace in \enspace container} = \boxed{50.25 \enspace lbs \enspace Ca(OCl)_2}$\\
\vspace{0.2cm}


\item $3 / 4$ is the same as what percentage?\\
\vspace{0.2cm}
$\dfrac{3}{4}=0.75 \enspace which \enspace is \enspace \dfrac{75}{100} = \boxed{75\%}$\\
\vspace{0.2cm}
\end{enumerate}

\section*{Practice Problems - Ratio and Proportion}

\begin{enumerate}
\item It takes 6 gallons of chlorine solution to obtain a proper residual when the flow is 45,000 gpd. How many gallons will it take when the flow is 62,000 gpd?\\
\vspace{0.2cm}
Solution:\\
\vspace{0.2cm}
Required gallons of chlorine is directly proportional to the flow being treated.\\
\vspace{0.2cm}
Thus, $\dfrac{6 \enspace gallons}{45,000 \enspace gpd }=\dfrac{X \enspace gallons}{62,000 \enspace gpd}$
\vspace{0.2cm}
Solving for X:\\
\vspace{0.2cm}
$\implies \enspace X=\dfrac{6*62,000}{45,000}=\boxed{8.3 \enspace lbs \enspace bleach}$
\vspace{0.2cm}

\item A motor is rated at 41 amps average draw per leg at $30 \mathrm{Hp}$. What is the actual $\mathrm{Hp}$ when the draw is 36 amps? C. 
\vspace{0.2cm}
Solution:\\
\vspace{0.2cm}
Ampere draw and horsepower (Hp) are directly proportional - when Hp goes up, the ampere draw goes up\\
\vspace{0.2cm}
Thus, $\dfrac{30 \enspace Hp}{41 \enspace Amperes }=\dfrac{X \enspace Hp}{36 \enspace amperes}$
\vspace{0.2cm}
Solving for X:\\
\vspace{0.2cm}
$\implies \enspace X=\dfrac{6*62,000}{45,000}=\boxed{8.3 \enspace lbs \enspace bleach}$
\vspace{0.2cm}
\item If it takes 2 operators $4.5$ days to clean an aeration basin, how long will it take three operators to do the same job?
\vspace{0.2cm}
Solution:\\
\vspace{0.2cm}
Number of operators and the time required to accomplish a certain task are inversely proportional - when more operators are involved, the task will take less time.\\
\vspace{0.2cm}
$(2 \enspace \mathrm{Operators} * 4.5 \enspace \mathrm{days})=(3 \enspace \mathrm{Operators} * X \enspace \mathrm{days})$
\vspace{0.2cm}
Solving for X:\\
\vspace{0.2cm}
$\implies \enspace X=\dfrac{2*4.5}{3}=\boxed{3 \enspace days}$
\vspace{0.2cm}
\end{enumerate}
\section*{Practice Problems - Area and Volume}
\begin{enumerate}
\item A 60-foot diameter tank contains 422,000 gallons of water. Calculate the height of water in the storage tank.

Volume = Area * Height $\implies Height (ft) =\dfrac{Volume - \cancelto{ft}{ft^3}}{Area \cancel{ft^2}}$\\
\vspace{0.2cm}
\

$ Volume \enspace (ft^3) = \dfrac{\pi}{4}*D^2 * fill height = 0.785*17.5^2 \enspace ft^2 * 14 ft=\boxed{240\enspace ft^3}$


\item What is the volume of water in ft$^3$, of a sedimentation basin that is 22 feet long, and 15 feet wide, and filled to 10 feet?\\

Volume = Length * Width * Height = 22 ft * 15 ft * 10 ft = $\boxed{3300 \enspace ft^3}$\\
\vspace{0.2cm}
$ Volume \enspace (ft^3) = \dfrac{\pi}{4}*D^2 * fill \enspace height = 0.785*17.5^2 \enspace ft^2 * 14 ft=\boxed{240\enspace ft^3}$

\item What is the volume in ft$^3$ of an elevated clear well that is 17.5 feet in diameter, and filled to 14 feet?

Volume = Area * Height\\
\vspace{0.2cm}
$ Volume \enspace (ft^3) = \dfrac{\pi}{4}*D^2 * fill height = 0.785*17.5^2 \enspace ft^2 * 14 ft=\boxed{240\enspace ft^3}$

\item What is the area of the top of a storage tank that is 75 feet in diameter?\\

$Area \enspace (ft^2)= \dfrac{\pi}{4}*D^2= 0.785*75^2 \enspace ft^2=0.785 = \boxed{4416\enspace ft^2}$\\
\vspace{0.2cm}

\item  What is the area of a wall $175 \mathrm{ft}$. in length and $20 \mathrm{ft}$. wide?\\
\vspace{0.2cm}
Solution:\\
\vspace{0.2cm}
Area = $175 * 20 \enspace = \enspace \boxed{3,500 ft^2}$
\vspace{0.2cm}
\item  You are tasked with filling an area with rock near some of your equipment. 1 Bag of rock covers 250 square feet. The area that needs rock cover is 400 feet in length and 30 feet wide. How many bags do you need to purchase?\\

\vspace{0.2cm}
Solution:\\
\vspace{0.2cm}
Area to be covered = 400' * 30' = 12,000 $ft^2$
\vspace{0.2cm}
$\implies 12,000 \enspace \cancel{ft^2} \enspace * \dfrac{Bag}{250 \enspace \cancel{ft^2}}=\boxed{48 \enspace bags}$


\end{enumerate}

\section*{Practice Problems - Flow and Velocity}
\begin{enumerate}

\item Flow in an 8-inch pipe is 500 gpm. What is the average velocity in ft/sec? (Assume pipe is flowing full)\\
Solution:\\
\vspace{0.2cm}

$Flow \enspace(\mathrm{Q})= Velocity \enspace(\mathrm{V})  \times Area \enspace(\mathrm{A}) \implies Q=V*A \implies V=\dfrac{Q}{A}$\\
We need to convert Q which is given in gpm to ft${^3}$/sec and calculate the area of the pipe in ft${^2}$ so velocity can be valculated in ft/sec.\\
\vspace{0.2cm}
$ V \dfrac{ft}{sec} = \dfrac{Q \enspace \dfrac{\cancelto{ft}{ft^3}}{sec}}{A \enspace\cancel{ft}}$\\
\vspace{0.2cm}
Step 1 - Converting Q - 500 gpm to ft${^3}/min$:\\
\vspace{0.2cm}
$\dfrac{500 \enspace \cancel{gallons}}{\bcancel{min}}*\dfrac{ft^3}{7.48 \enspace \cancel{gallon}}*\dfrac{\bcancel{min}}{60 \enspace sec}=1.1\dfrac{ft^3}{sec}$\\
\vspace{0.2cm}
Step 2 - Calculating area in ft${^2}$:\\
\vspace{0.2cm}
$Area \enspace ft^2= \dfrac{\pi}{4}*D^2= 0.785*\Big(\dfrac{8}{12}\Big)^2 \enspace ft^2=0.785*\dfrac{64}{144}=0.349 \enspace ft^2$\\
\vspace{0.2cm}
$\implies V \dfrac{ft}{sec} = \dfrac{ 1.1 ft^3/sec}{0.349 \enspace ft^2} = \boxed{3.2 ft/sec}$\\
\vspace{0.3cm} 


\item A pipeline is 18” in diameter and flowing at a velocity of 125 ft. per minute. What is the flow in gallons per minute?\\
\vspace{0.2cm}
Solution:\\
$Flow \enspace(\mathrm{Q}) \enspace = Velocity \enspace(\mathrm{V})  \times Area \enspace(\mathrm{A})$\\
\vspace{0.2cm}
As the velocity is given in ft/min, and the area can be calculated in ft$^2$, flow can be calulated in ft$^3$/min and then converted to gal/min.\\
\vspace{0.2cm}

\vspace{0.2cm}
Step 1:  Calculating area in ft${^2}$:\\
\vspace{0.2cm}
$Area \enspace (ft^2)= \dfrac{\pi}{4}*D^2= 0.785*\Big(\dfrac{18}{12}\Big)^2 \enspace ft^2=0.785*\dfrac{324}{144}=0.349 \enspace ft^2$\\
\vspace{0.2cm}

Step 2: Calculate flow in ft$^3$/min:\\

$ Q \enspace ft^3/min = 125 \dfrac{ft}{min}*1.77 \enspace ft^2 = 221.25 \dfrac{ft^3}{min}$\\

\vspace{0.2cm}

Step 3: Convert Q to gallons per minute

\vspace{0.2cm}

$Q=221.25 \dfrac{\cancel{ft^3}}{min}*7.48\dfrac{gal}{\cancel{ft^3}}=\boxed{1655 \dfrac{gal}{min}}$


\item The velocity in a pipeline is 2 ft./sec. and the flow is 3,000 gpm. What is the diameter of the pipe in inches?

Solution:\\
\vspace{0.2cm}

$Flow \enspace(\mathrm{Q})= Velocity \enspace(\mathrm{V})  \times Area \enspace(\mathrm{A}) \implies Q=V*A \implies A=\dfrac{Q}{V}$\\
We need to convert Q which is given in gpm to ft${^3}$/sec and calculate the area of the pipe in ft${^2}$ given the velocity.\\
From the calculated area of the pipe, the pipe diameter can be calculated.\\
\vspace{0.2cm}
$ A \dfrac{ft}{sec} = \dfrac{Q \enspace \dfrac{\cancelto{ft^2}{ft^3}}{\cancel{sec}}}{V \enspace \dfrac{\cancelto{}{ft}}{\cancel{sec}}}$\\
\vspace{0.2cm}
Step 1 - Converting Q - 3000 gpm to ft${^3}$/sec:\\
\vspace{0.2cm}
$\dfrac{3000 \enspace \cancel{gallons}}{\bcancel{min}}*\dfrac{ft^3}{7.48 \enspace \cancel{gallon}}*\dfrac{\bcancel{min}}{60 \enspace sec}=6.68\dfrac{ft^3}{sec}$\\
\vspace{0.2cm}
Step 2 - Calculating area in ft${^2}$:\\
\vspace{0.2cm}
$\implies A \enspace ft^2 = \dfrac{ 6.68 ft^3/sec}{2 \dfrac{ft}{sec}} = 3.34 ft^2$\\
\vspace{0.3cm} 
$Area \enspace (A)= \dfrac{\pi}{4}*D^2 = 0.785*D^2 \implies D^2=\dfrac{A}{0.785} \implies D=\Big(\dfrac{A }{0.785}\Big)^{\dfrac{1}{2}}$\\
$\implies D=\Big(\dfrac{3.34}{0.785}\Big)^{\dfrac{1}{2}}=\boxed{2 \enspace ft}$\\
\vspace{0.2cm}

\item Find the flow in a 4-inch pipe when the velocity is $1.5$ feet per second.

Solution:\\
$Flow \enspace(\mathrm{Q}) \enspace = Velocity \enspace(\mathrm{V})  \times Area \enspace(\mathrm{A})$\\
\vspace{0.2cm}
The velocity is given in ft/sec and after calculating the area in ft$^2$, flow can be calulated in ft$^3$/min.\\
\vspace{0.2cm}

\vspace{0.2cm}
Step 1:  Calculating area in ft${^2}$:\\
\vspace{0.2cm}
$Area \enspace (ft^2)= \dfrac{\pi}{4}*D^2= 0.785*\Big(\dfrac{4}{12}\Big)^2 \enspace ft^2=0.785*\dfrac{324}{144}=0.087 \enspace ft^2$\\
\vspace{0.2cm}

Step 2: Calculate flow in ft$^3$/min:\\

$ Q \enspace ft^3/min = 1.5 \dfrac{ft}{sec}*0.087 \enspace ft^2 = 0.13 \dfrac{ft^3}{sec}$\\

\vspace{0.2cm}

Q can be converted to a more commonly used gallons per minute unit

\vspace{0.2cm}

$Q=0.13 \dfrac{\cancel{ft^3}}{sec}*7.48\dfrac{gal}{\cancel{ft^3}}*60\dfrac{sec}{\cancel{min}}=\boxed{59 \dfrac{gal}{min}}$
  \item A 42-inch diameter pipe transfers 35 cubic feet of water per second. Find the velocity in $\mathrm{ft} / \mathrm{sec}$. 

  Solution:\\
\vspace{0.2cm}

$Flow \enspace(\mathrm{Q})= Velocity \enspace(\mathrm{V})  \times Area \enspace(\mathrm{A}) \implies Q=V*A \implies V=\dfrac{Q}{A}$\\
Q is already given in ft${^3}$/sec.  We need to first calculate the area of the pipe in ft${^2}$ so velocity can be valculated in ft/sec.\\
\vspace{0.2cm}
$ V \dfrac{ft}{sec} = \dfrac{Q \enspace \dfrac{\cancelto{ft}{ft^3}}{sec}}{A \enspace\cancel{ft}}$\\
\vspace{0.2cm}
Step 1 - Calculating area in ft${^2}$:\\
\vspace{0.2cm}
$Area \enspace ft^2= \dfrac{\pi}{4}*D^2= 0.785*\Big(\dfrac{42}{12}\Big)^2 \enspace ft^2=0.785*\dfrac{1764}{144}=9.616 \enspace ft^2$\\
\vspace{0.2cm}
$\implies V \dfrac{ft}{sec} = \dfrac{ 35 ft^3/sec}{9.616 \enspace ft^2} = \boxed{3.6 ft/sec}$\\
\vspace{0.3cm} 

  
  \item A plastic float is dropped into a channel and is found to travel 10 feet in $4.2$ seconds. The channel is $2.4$ feet wide and the water is flowing $1.8$ feet deep. Calculate the flow rate of water in cfs.\\
  \vspace{0.2cm}
  Solution:\\
  $Flow \enspace(\mathrm{Q})= Velocity \enspace(\mathrm{V})  \times Area \enspace(\mathrm{A})$\\

The speed of the float travelling is the velocity of the water $\implies Velocity = \dfrac{10 \enspace ft}{4.2 \enspace sec}$

Thus flow = $\dfrac{10 \enspace ft}{4.2 \enspace sec} * (2.4*1.8) ft^2 = \boxed{4.32 \dfrac{ft^3}{sec}} $\\

\vspace{0.2cm}
\end{enumerate}
\section*{Practice Problems - Unit Conversions}

\item Convert 1000 $ft^3$ to cu. yards\\
Solution:\\
\vspace{0.2cm}




\vspace{0.2cm}
\item Convert 10 gallons/min to $ft^3$/hr\\
Solution:\\
\vspace{0.2cm}
$\dfrac{ft^3}{hr}=10\dfrac{\cancel{gal}}{\cancel{min}}*\dfrac{ft^3}{7.48\cancel{gal}}*\dfrac{60\cancel{min}}{hr}=\boxed{80.2 \dfrac{ft^3}{hr}}$

\vspace{0.2cm}
\item Find the flow in gpm when the flow is $0.25 \mathrm{cfs}$.\\
Solution:\\
\vspace{0.2cm}
$\dfrac{gal}{min}=0.25\dfrac{\cancel{ft^3}}{\cancel{sec}}*\dfrac{7.48gal}{\cancel{ft^3}}*\dfrac{60\cancel{sec}}{min}=\boxed{112.2 \dfrac{gal}{min}}$


\vspace{0.2cm}
\item The flow rate through a filter is 4.25 MGD. What is this flow rate expressed as gpm?\\
Solution:\\
\vspace{0.2cm}
$\dfrac{gal}{min}=4.25\dfrac{\cancel{MG}}{\cancel{day}}*\dfrac{1,000,000gal}{\cancel{MG}}*\dfrac{\cancel{day}}{1,440min}=\boxed{2,951 \dfrac{gal}{min}}$

\vspace{0.2cm}
\item After calibrating a chemical feed pump, you've determined that the maximum feed rate is $178 \mathrm{~mL} /$ minute. If this pump ran continuously, how many gallons will it pump in a full day?\\
Solution:\\
\vspace{0.2cm}
$\dfrac{gal}{day}=178\dfrac{\cancel{mL}}{\cancel{min}}*\dfrac{L}{1000\cancel{mL}}*\dfrac{1,440\cancel{min}}{day}=\boxed{119,680 \dfrac{gal}{day}}$


\vspace{0.2cm}
\item A plant produces 2,000 cubic foot of water per hour. How many gallons of water is produced in an 8-hour shift?\\
Solution:\\
\vspace{0.2cm}

$\dfrac{gal}{8-hr \enspace shift}=2,000\dfrac{\cancel{ft^3}}{\cancel{hr}}*\dfrac{7.48gal}{\cancel{ft^3}}*\dfrac{8\cancel{hr}}{8-hr \enspace shift}=\boxed{253.6 \dfrac{gal}{day}}$

\vspace{0.2cm}
\item Change 70 °F to °C\\
Solution:\\
\vspace{0.2cm}
$\degree{C} = \dfrac{\degree{F}-32}{1.8} = \dfrac{70-32}{1.8}=\boxed{21.1\degree{C}}$
\vspace{0.2cm}
\item Change 4 °C to °F\\
Solution:\\
\vspace{0.2cm}
$\degree{F}=(\degree{C} \times 1.8)+32 = (4*1.8)+32=\boxed{39.2\degree{F}}$


\vspace{0.2cm}

\section*{Practice Problems - Concentration}
\begin{enumerate}
\item What is the concentration in mg/l of  4.5\% solution of that substance.\\
\vspace{0.2cm}
Solution:\\
\vspace{0.2cm}
$\boxed{45,000mg/l}$

\end{enumerate}

\section*{Practice Problems - Density and Specific Gravity}
\begin{enumerate}


\item How much does each gallon of zinc orthophosphate weigh (pounds) if it has a specific gravity of 1.46?\\
\vspace{0.2cm}
Solution:\\
\vspace{0.2cm}
$8.34\dfrac{lb}{gal}*1.46=\boxed{12.18\dfrac{lb}{gal}}$
\vspace{0.2cm}
\item How much does a 55 gallon drum of 25\% caustic soda weigh (pounds) if the specific gravity is 1.28?\\
\vspace{0.2cm}
Solution:\\
\vspace{0.2cm}
$8.34\dfrac{lb}{\cancel{gal}}*1.28*55\cancel{gal}=\boxed{12.18\dfrac{lb}{gal}}$
\vspace{0.2cm}
\end{enumerate}

\section*{Practice Problems - Detention Time}


\section*{Practice Problems - Pounds Formula}



\section*{Practice Problems - Pressure-Force Relationship}

 \item Find the force on a 12-inch valve if the water pressure within the line is 60 psi. Express your answer in tons.

$\textrm{Force}= \textrm{Pressure} \times \textrm{Area}$\\
\vspace{0.3cm}
$\implies 60 \enspace \dfrac{\mathrm{lbs}}{\mathrm{in^2}}*0.785 *(12 \mathrm{in})^2*\dfrac{1 \mathrm{ton}}{2000 \mathrm{lbs}} =\boxed{3.39 \enspace\mathrm{tons}}$
\vspace{0.3cm}

\item A water tank is 15 feet deep and 30 feet in diameter. What is the force exerted on a 6-inch valve at the bottom of the tank?\\
\vspace{0.5cm}
$\textrm{Force}= \textrm{Pressure} \times \textrm{Area}$\\
\vspace{0.5cm}
$\implies 15 \enspace\mathrm{ft}* \dfrac{0.433 \enspace \mathrm{psi}}{\mathrm{ft}}*0.785 *(6 \mathrm{in})^2 =\boxed{183 \enspace\mathrm{lbs}}$\\



\section*{Practice Problems - Well Hydraulics}

\begin{enumerate}


\item A well yields 2,840 gallons in exactly 20 minutes. What is the well yield in gpm?\\
Solution:\\
\vspace{0.2cm}
$\dfrac{2,840gal}{20min}=\boxed{142\dfrac{gal}{min}}$


\vspace{0.2cm}
\item Before pumping, the water level in a well is 15 ft. down. During pumping, the water level is 45 ft. down. The drawdown is:\\
Solution:\\
\vspace{0.2cm}
$45-15=\boxed{30 ft}$


\vspace{0.2cm}
\item A well is located in an aquifer with a water table elevation 20 feet below the ground surface. After operating for three hours, the water level in the well stabilizes at 50 feet below the ground surface. Calculate the pumping water level.\\
Solution:\\
\vspace{0.2cm}
$\boxed{50ft}$


\vspace{0.2cm}
\item Calculate drawdown, in feet, using the following data:\\
The water level in a well is 20 feet below the ground surface when the pump is not in operation, and the water level is 35 feet below the ground surface when the pump is in operation.\\
Solution:\\
\vspace{0.2cm}

  $\text {Drawdown} = 35 \mathrm{ft}-20 \mathrm{ft}=\boxed{15ft}$\\


\vspace{0.2cm}
\item Calculate the well yield in gpm, given a drawdown of 14.1 ft and a specific yield of 31 gpm/ft.\\
Solution:\\
\vspace{0.2cm}
  $Specific \enspace Yield \enspace(gpm/ft) =\dfrac{ Yield \enspace(gpm)}{ Drawdown \enspace(ft)}$\\
  \vspace{0.2cm}

  $\implies Yield \enspace(gpm) = Specific \enspace Yield \enspace(gpm/ft)*Drawdown \enspace(ft)=31*14.1=\boxed{437.1 \enspace gpm}$

\end{enumerate}

\section*{Practice Problems - Pumping Rates Calculations}



\section*{Practice Problems - Pressure-Head Relationship}
\begin{enumerate}
\item A water tank is filled to depth of 22 feet. What is the psi at the bottom of the tank?\\
 \vspace{0.2cm}
Solution:\\ 
 \vspace{0.2cm}
$
22 \enspace \cancel{ft}*\dfrac{0.433psi}{\cancel{ft \enspace head}}=\boxed{9.5 \text { psi }}
$
  \vspace{0.2cm}
\item The static pressure in a water main is 85 psi. What elevation of water is needed to provide that kind of pressure?\\
 \vspace{0.2cm}
Solution:\\ 
 \vspace{0.2cm}
$
85 \enspace \cancel{psi}*\dfrac{ft \enspace head}{0.433\cancel{psi}}=\boxed{196.3 \text { feet }}
$
 
 \vspace{0.2cm}

\item The pressure at the top of the hill is 62 psi. The pressure at the bottom of the hill, 60 feet below, is 100 psi. The water is flowing uphill at 120 gpm. What is the friction loss, in feet, in the pipe?\\
\vspace{0.2cm}
\begin{tikzpicture}[scale=2]
\draw[ultra thick,-](0.8,0.8) -- (1,0.8)node [at start, below,  black]{\small{}} node [anchor=north west, black]{} node [at start, left, black] (n){};;
\draw[ultra thick,-](-1,0) -- (-0.2,0)node [at start, below,  black]{\small{}} node [anchor=north west, black]{} node [at start, left, black] (n){};;
\draw[ultra thick,-](-0.2,0) -- (0.8,0.8)node [at start, below,  black]{\small{}} node [anchor=north west, black]{} node [at start, left, black] (n){};;
\draw [<->] (1,0) -- (1,0.78) node [midway, midway] {\hspace{1.5cm}60'};
 \node at (-0.5,0.11) {\includegraphics[width=0.5cm]{PressureGuageIcon.png}};
  \node at (1,0.91) {\includegraphics[width=0.5cm]{PressureGuageIcon.png}};

\draw (0,0) .. controls (0.98,1.05) and (1.02,1.05) .. (2,0);
\draw (1.1,1.05) node[anchor=north west] {$Pressure=62psi$};
\draw (-0.65,0.5) node[anchor=north east] {$Flow=120 gpm$};
\draw (-0.65,0.3) node[anchor=north east] {$Pressure=100psi$};
\end{tikzpicture}\\
\vspace{0.2cm}
Total headloss = Headloss due to elevation gain + Headloss due to friction\\
\vspace{0.2cm}
$\implies$ Headloss due to friction = Total headloss - Headloss due to elevation gain\\
\vspace{0.2cm}
Total headloss = $(100 - 62)\enspace \cancel{psi}* \dfrac{ft \enspace head}{0.433\cancel{psi}}=87.8 ft $\\
Headloss due to elevation gain = $60 \enspace ft $\\
$\implies$ Headloss due to friction = $87.8-60=\boxed{27.8 \enspace ft}$\\
\vspace{0.2cm}
\end{enumerate}

\section*{Practice Problems - Pumping Power Requirements}
\begin{enumerate}
\item The efficiency of a well pump is determined to be $75 \%$. The efficiency of the motor is estimated at $94 \%$. What is the efficiency of the well?\\
 \vspace{0.2cm}
Solution:\\ 
 \vspace{0.2cm}
$Well \enspace efficiency=\eta_m * \eta_p \implies 0.94 \times 0.75=0.705 \times 100=\boxed{71 \%}$
 \vspace{0.2cm}

  \item Water is being pumped from a reservoir to a storage tank on a hill. The elevation difference between water levels is 1200 feet. Find the pump size (in Hp) required to fill the tank at a rate of 120 gpm.\\
  \vspace{0.2cm}
\begin{tikzpicture}[scale=1]
\draw (0,0) .. controls (1.98,3.5) and (2.02,3.5) .. (4,0);
\node[cylinder, 
    draw = violet, 
    text = purple,
    cylinder uses custom fill, 
    cylinder body fill = blue!10, 
    cylinder end fill = magenta!40,
    aspect = 0.1, 
    shape border rotate = 90] (c) at (2,3.0) {Storage};
\node[cylinder, 
    draw = violet, 
    text = purple,
    cylinder uses custom fill, 
    %cylinder body fill = magenta!10, 
    %cylinder end fill = magenta!40,
    minimum size = 0.3cm, aspect = 0.1,
    shape border rotate = 90] (c) at (2,3.5) {\hspace{0.25cm}{Water}\hspace{0.25cm}};

  \node at (-3,0.1) {\includegraphics[width=3cm]{PumpIcon.png}};
   \node at (-3,-0.8) {\includegraphics[width=2cm]{WaterReservoirIcon.png}};
   
\draw [ultra thick, -] (-3,-.9) -- (-3,0) node [midway, below] {};
\draw [ultra thick, -] (-3.28,0.63) -- (-3.28,3.1) node [midway, below] {};
\draw [ultra thick, ->] (-3.28,3.1) -- (1.2,3.1) node [midway, below] {};
\draw [ultra thick, ->] (-3.28,3.1) -- (1.2,3.1) node [midway, below] {};
\draw[dashed] (-1.8,-1) -- (2.5,-1);
\draw [<->] (2,-1) -- (2,3.25) node [midway, below] {\hspace{5cm}Elevation difference = 1200 ft};
\draw (0.5,3.7) node[anchor=north east] {$Flow = 120 gpm$};
\end{tikzpicture}\\
\vspace{0.2cm}
Solution:\\
\vspace{0.2cm}
water Hp = flow * head\\
$120GPM*1,200ft*\dfrac{Hp}{3,960 GPM-ft}=\boxed{Water \enspace Hp = 36.4Hp}$\\
\item If a pump is operating at 2,200 gpm and 60 feet of head, what is the water horsepower? If the pump efficiency is 71\%, what is the brake horsepower?\\
\vspace{0.2cm}
Solution:\\
\vspace{0.2cm}
water Hp = flow * head\\
$2,200GPM*60ft*\dfrac{Hp}{3,960 GPM-ft}=\boxed{Water \enspace Hp = 33.3Hp}$\\
\vspace{0.4cm}
pump Hp = brake Hp * pump efficiency\\
$brake \enspace Hp = \dfrac{33.3}{0.71}=\boxed{Brake \enspace Hp=47Hp}$
 \vspace{0.2cm}


\item The water horsepower of a pump is $10 \mathrm{Hp}$ and the brake horsepower output of the motor is $15.4 \mathrm{Hp}$. What is the efficiency of the pump?\\
\vspace{0.2cm}
Solution:\\ 
 \vspace{0.2cm}
 \vspace{0.4cm}\includegraphics[scale=0.08]{PumpProblem}\\
 \vspace{0.2cm}
 \includegraphics[scale=0.32]{PumpingProblemPump}
 $\eta_p=\dfrac{10 \mathrm{BHp}}{15.4 \mathrm{EHp}} \times 100=\boxed{65 \%}$\\
 \vspace{0.2cm}
 \item The water horsepower of a pump is $25 \mathrm{Hp}$ and the brake horsepower output of the motor is $48 \mathrm{Hp}$. What is the efficiency of the pump?\\
 Solution:\\
  \vspace{0.2cm}
 \vspace{0.32cm}\includegraphics[scale=0.08]{PumpProblem}\\
 \vspace{0.2cm}
 \includegraphics[scale=0.4]{PumpingProblemPump}
 \vspace{0.2cm}
$\eta_p=\dfrac{25 \mathrm{\enspace Water \enspace Hp}}{48 \mathrm{\enspace brake \enspace Hp}} \times 100=\boxed{52 \%}$
  \vspace{0.4cm}
\end{enumerate}

\section*{Practice Problems - Bleach and Chemical Dosing}
\begin{enumerate}

\item A water treatment plant operates at the rate of 75 gallons per minute. They dose soda ash at 14 mg/L. How many pounds of soda ash will they use in a day?
Solution:\\
\vspace{0.2cm}
\begin{figure}[h]
\begin{tikzpicture}
    \newcommand{\R}{1.5}

\path[help lines,step=.2] (0,0) grid (16,3);
\path[help lines,line width=.6pt,step=1] (0,0) grid (16,3);
%\foreach \x in {0,1,2,3,4,5,6,7,8,9,10,11,12,13,14,15,16}
%\node[anchor=north] at (\x,0) {\x};
%\foreach \y in {0,1,2,3,4,5,6}
%\node[anchor=east] at (0,\y) {\y};
%-------------CIRCLE-----------------------------------
\draw[black,fill=gray!10] (8,3) circle (\R);
\draw[black, very thick, rotate=0](6.5,3) -- (9.5,3);
\draw (8,3.6) node[text width=3cm,align=center]
  {\scriptsize{lbs/day}};
\draw (7.1,2.5) node[text width=3cm,align=center]
  {\tiny{14 mg/l}};
\draw (8.9,2.5) node[text width=3cm,align=center]
  {\tiny{75 GPM}};
  \draw (8,2)node[text width=3cm,align=center]
  {\tiny{8.34}};
\draw[black, very thick, rotate=0](7.2,1.7) -- (8,3);
\draw[black, very thick, rotate=0](8.8,1.7) -- (8,3);
\end{tikzpicture}
\end{figure}
$\dfrac{\mathrm{lbs}}{\mathrm{day}}=\mathrm{Flow}\dfrac{{\mathrm{MG}}}{\mathrm{day}}* \mathrm{Concentration}\dfrac{\mathrm{mg}}{\mathrm{l}}*8.34$
\\
\vspace{0.2cm}
$\dfrac{\mathrm{lbs}}{\mathrm{day}}=75 \dfrac{\cancel{\mathrm{gallons}}}{\cancel{\mathrm{min}}}* 1440\dfrac{\cancel{\mathrm{min}}}{\mathrm{day}}*\dfrac{\mathrm{MG}}{1,000,000 \enspace \cancel{\mathrm{gallons}}}*14\dfrac{\mathrm{mg}}{\mathrm{l}}*8.34 = \boxed{12.6\dfrac{lbs}{day}}$
\vspace{0.2cm}

\item A water treatment plant is producing 1.5 million gallons per day of potable water, and uses 38 pounds of soda ash for pH adjustment. What is the dose of soda ash at that plant?\\
Solution:\\
 \begin{figure}[h!]
\begin{tikzpicture}
    \newcommand{\R}{1.5}

\path[help lines,step=.2] (0,0) grid (16,3);
\path[help lines,line width=.6pt,step=1] (0,0) grid (16,3);
%\foreach \x in {0,1,2,3,4,5,6,7,8,9,10,11,12,13,14,15,16}
%\node[anchor=north] at (\x,0) {\x};
%\foreach \y in {0,1,2,3,4,5,6}
%\node[anchor=east] at (0,\y) {\y};
%-------------CIRCLE-----------------------------------
\draw[black,fill=gray!10] (8,3) circle (\R);
\draw[black, very thick, rotate=0](6.5,3) -- (9.5,3);
\draw (8,3.6) node[text width=3cm,align=center]
  {\scriptsize{38 lbs/day}};
\draw (7.1,2.5) node[text width=3cm,align=center]
  {\tiny{? mg/l}};
\draw (8.9,2.5) node[text width=3cm,align=center]
  {\tiny{1.5 MGD}};
  \draw (8,2)node[text width=3cm,align=center]
  {\tiny{8.34}};
\draw[black, very thick, rotate=0](7.2,1.7) -- (8,3);
\draw[black, very thick, rotate=0](8.8,1.7) -- (8,3);
\end{tikzpicture}
\end{figure}
$\dfrac{\mathrm{lbs}}{\mathrm{day}}=\mathrm{Flow}\dfrac{{\mathrm{MG}}}{\mathrm{day}}* \mathrm{Concentration}\dfrac{\mathrm{mg}}{\mathrm{l}}*8.34 \hspace{0.2cm} \implies \mathrm{Concentration}\dfrac{\mathrm{mg}}{\mathrm{l}}=\dfrac{ \dfrac{\mathrm{lbs}}{\mathrm{day}}}{\mathrm{Flow}\dfrac{{\mathrm{MG}}}{\mathrm{day}}*8.34}$
\vspace{0.2cm}
$\mathrm{Concentration}\dfrac{\mathrm{mg}}{\mathrm{l}}=\dfrac{ 38\dfrac{\mathrm{lbs}}{\mathrm{day}}}{1.5\dfrac{{\mathrm{MG}}}{\mathrm{day}}*8.34}=\boxed{3\dfrac{\mathrm{mg}}{\mathrm{l}}}$
\\
\vspace{0.2cm}


\item A water treatment plant produces 150,000 gallons of water every day. It uses an average of 2 pounds of permanganate for iron and manganese removal. What is the dose of the permanganate? \\
 Solution:\\
 \vspace{0.2cm}
 \begin{figure}[h!]
\begin{tikzpicture}
    \newcommand{\R}{1.5}

\path[help lines,step=.2] (0,0) grid (16,3);
\path[help lines,line width=.6pt,step=1] (0,0) grid (16,3);
%\foreach \x in {0,1,2,3,4,5,6,7,8,9,10,11,12,13,14,15,16}
%\node[anchor=north] at (\x,0) {\x};
%\foreach \y in {0,1,2,3,4,5,6}
%\node[anchor=east] at (0,\y) {\y};
%-------------CIRCLE-----------------------------------
\draw[black,fill=gray!10] (8,3) circle (\R);
\draw[black, very thick, rotate=0](6.5,3) -- (9.5,3);
\draw (8,3.6) node[text width=3cm,align=center]
  {\scriptsize{38 lbs/day}};
\draw (7.1,2.5) node[text width=3cm,align=center]
  {\tiny{? mg/l}};
\draw (8.9,2.5) node[text width=3cm,align=center]
  {\tiny{1.5 MGD}};
  \draw (8,2)node[text width=3cm,align=center]
  {\tiny{8.34}};
\draw[black, very thick, rotate=0](7.2,1.7) -- (8,3);
\draw[black, very thick, rotate=0](8.8,1.7) -- (8,3);
\end{tikzpicture}
\end{figure}
$\dfrac{\mathrm{lbs}}{\mathrm{day}}=\mathrm{Flow}\dfrac{{\mathrm{MG}}}{\mathrm{day}}* \mathrm{Concentration}\dfrac{\mathrm{mg}}{\mathrm{l}}*8.34 \hspace{0.2cm} \implies \mathrm{Concentration}\dfrac{\mathrm{mg}}{\mathrm{l}}=\dfrac{ \dfrac{\mathrm{lbs}}{\mathrm{day}}}{\mathrm{Flow}\dfrac{{\mathrm{MG}}}{\mathrm{day}}*8.34}$
\vspace{0.2cm}
$\mathrm{Concentration}\dfrac{\mathrm{mg}}{\mathrm{l}}=
\dfrac{ 2\dfrac{\mathrm{lbs}}{\mathrm{day}}}
{\Bigg(150,000 \dfrac{\cancel{\mathrm{Gallons}}}
{\mathrm{day}}*
\dfrac{\mathrm{MG}}
{1,000,000 \cancel{\enspace \mathrm{Gallons}}}*8.34\Bigg)}
=\boxed{3\dfrac{\mathrm{mg}}{\mathrm{l}}}$
\\
\vspace{0.2cm}

\item A water treatment plant uses 8 pounds of chlorine daily and the dose is 17 mg/l. How many gallons are they producing?\\
 Solution:\\
 \begin{figure}[h!]
\begin{tikzpicture}
    \newcommand{\R}{1.5}

\path[help lines,step=.2] (0,0) grid (16,3);
\path[help lines,line width=.6pt,step=1] (0,0) grid (16,3);
%\foreach \x in {0,1,2,3,4,5,6,7,8,9,10,11,12,13,14,15,16}
%\node[anchor=north] at (\x,0) {\x};
%\foreach \y in {0,1,2,3,4,5,6}
%\node[anchor=east] at (0,\y) {\y};
%-------------CIRCLE-----------------------------------
\draw[black,fill=gray!10] (8,3) circle (\R);
\draw[black, very thick, rotate=0](6.5,3) -- (9.5,3);
\draw (8,3.6) node[text width=3cm,align=center]
  {\scriptsize{8 lbs/day}};
\draw (7.1,2.5) node[text width=3cm,align=center]
  {\tiny{17 mg/l}};
\draw (8.9,2.5) node[text width=3cm,align=center]
  {\tiny{? MGD}};
  \draw (8,2)node[text width=3cm,align=center]
  {\tiny{8.34}};
\draw[black, very thick, rotate=0](7.2,1.7) -- (8,3);
\draw[black, very thick, rotate=0](8.8,1.7) -- (8,3);
\end{tikzpicture}
\end{figure}
$\dfrac{\mathrm{lbs}}{\mathrm{day}}=\mathrm{Flow}\dfrac{{\mathrm{MG}}}{\mathrm{day}}* \mathrm{Concentration}\dfrac{\mathrm{mg}}{\mathrm{l}}*8.34 \hspace{0.2cm}$\\
\vspace{0.2cm}
$\implies \mathrm{Flow}\dfrac{{\mathrm{MG}}}{day}=\dfrac{ \dfrac{\mathrm{lbs}}{\mathrm{day}}}{\mathrm{Concentration}\dfrac{\mathrm{mg}}{\mathrm{l}}*8.34}=\dfrac{8 \dfrac{\mathrm{lbs}}{\mathrm{day}}}{17\dfrac{\mathrm{mg}}{\mathrm{l}}*8.34}=0.056425\dfrac{{\mathrm{MG}}}{day}$\\
\vspace{0.2cm}
$0.056425\dfrac{{\mathrm{MG}}}{day}*\dfrac{1,000,000 \enspace \mathrm{Gallons}}{\mathrm{MG}}=\boxed{56,425 \enspace \mathrm{Gallons}}$
\vspace{0.2cm}

\vspace{0.2cm}
\item Ferric chloride is being added as a coagulant to the raw water entering a plant. Sampling shows that the concentration of ferric in the raw water is 25 ppm. A quick check of the chemical metering pump shows that it is operating at a flow rate of 4.3 gpm. If the flow through the water plant is 800 gpm, what is the concentration of raw chemical in the dosing tank?\\
\vspace{0.2cm}
Solution:\\
\vspace{0.3cm}
\begin{tikzpicture}

\draw [-] (-3.2,4.2) -- (-0.4,4.2);
\draw [->] (-0.2,4) -- (-0.2,1.9);
\draw [->] (-3.2,1.9) -- (4,1.9);
\draw [shift={(-0.4,4)}] plot[domain=0:1.57,variable=\t]({1*0.2*cos(\t r)+0*0.2*sin(\t r)},{0*0.2*cos(\t r)+1*0.2*sin(\t r)});
\draw (-3.1,4.1) node[anchor=north west] {V$_{\tiny{FeCl_3}}$=$4.3 gpm$};
\draw (-3.1,3.6) node[anchor=north west] {C$_{\tiny{FeCl_3}}$ = ?};
\draw (-4.2,4.5) node[anchor=north west] {FeCl$_3$};
\draw (-4.2,2.2) node[anchor=north west] {Water};
\draw (-2.1,1.8) node[anchor=north west] {$800 gpm$};
\draw (0.7,1.8) node[anchor=north west] {C$_2$=25ppm FeCl$_3$};
\draw (0.7,1.3) node[anchor=north west] {V$_2$=4.3+800=804.3 gpm};
\end{tikzpicture}\\
\vspace{0.2cm}
C$_1$ * V$_1$ = C$_2$ * V$_2$ \\
\vspace{0.2cm}
C$_{\tiny{FeCl_3}}$ * V$_{\tiny{FeCl_3}}$  =  C$_2$ * (V$_{\tiny{FeCl_3}}$+V$_{\tiny{Water}}$)\\
\vspace{0.2cm}
C$_{\tiny{FeCl_3}}$ * 4.3 =  25 * (804.3)\\
\vspace{0.2cm}
C$_{\tiny{FeCl_3}}=\dfrac{25 * (804.3)}{4.3}=\boxed{4,676 \enspace \mathrm{ppm} \enspace \mathrm{or} \enspace 0.47\%}$\\
\vspace{0.3cm}
\item A water plant is fed by two different wells. The first well produces water at a rate of 600 gpm and contains arsenic at 0.5 mg/L. The second well produces water at a rate of 350 gpm and contains arsenic at 12.5 mg/L. What is the arsenic concentration of the blended water?\\
\vspace{0.2cm}
Solution:\\
\vspace{0.2cm}
C$_1$ * V$_1$ + C$_2$ * V$_2$ + =  C$_3$ * V$_3$=C$_3$*(V$_1$ + V$_2$)\\
\vspace{0.2cm}
C$_{Well \enspace 1}$ * V$_{Well \enspace 1}$ + C$_{Well \enspace 2}$ * V$_{Well \enspace 2}$ =  C$_{Blend}$ * V$_{Blend}$=C$_{Blend}$*(V$_{Well \enspace1}$ + V$_{Well \enspace 2}$)\\
\vspace{0.3cm}
$\implies C_{Blend}=\dfrac{C_{Well \enspace 1} * V_{Well \enspace 1} + C_{Well \enspace 2} * V_{Well \enspace 2}}{V_{Well \enspace 1} + V_{Well \enspace 2}}=\dfrac{0.5*600+12.5*350}{600+350}=\boxed{4.9 \enspace \textrm{mg/l}}$
\end{enumerate}

\section*{Practice Problems - Blending and Dilution}


\section*{Practice Problems - Sedimentation}

\begin{enumerate}

\item A circular clarifier has a diameter of 80 ft. If the flow to the clarifier is 1800 gpm, what is the surface overflow rate in gpm/ft
\vspace{0.2cm}
\textbf{Solution}\\

$\mathrm{Surface \enspace overflow \enspace rate}=\dfrac{\mathrm{Flow, \enspace gpm}}{\mathrm{Clarifier \enspace surface \enspace area, \enspace ft}^2}=\dfrac{1,800 \enspace \mathrm{gpm}}{(0.785*80^2 )\mathrm{ft}^2}=\boxed{0.36 \enspace \mathrm{gpm/ft}^2}$

\vspace{0.2cm}
\item A sedimentation basin 70 ft by 25 ft receives a flow of 1000 gpm. What is the surface overflow rate in gpm/ft$2$?
\vspace{0.2cm}
\textbf{Solution}\\
$\mathrm{Surface \enspace overflow \enspace rate}=\dfrac{\mathrm{Flow, \enspace gpm}}{\mathrm{Clarifier \enspace surface \enspace area, \enspace ft}^2}=\dfrac{1,000 \enspace \mathrm{gpm}}{(70 \mathrm{ft} \enspace * 25 \mathrm{ft})\mathrm{ft}^2}=\boxed{0.6 \enspace \mathrm{gpm/ft}^2}$

\vspace{0.2cm}

\item A circular clarifier receives a flow of 3.55 MGD. If the diameter of the weir is 90 ft, what is the weir loading rate in gpm/ft?
\end{enumerate}

\vspace{0.2cm}
\textbf{Solution}\\

$\mathrm{Weir \enspace overflow \enspace rate}=\dfrac{\mathrm{Flow, \enspace gpm}}{\mathrm{Weir} \enspace \mathrm{length} \enspace ft}$\\
\vspace{0.3cm}
$\implies \dfrac{ \dfrac{3.55 \enspace \mathrm{MG}}{\mathrm{day}}*\dfrac{1,000,000 \enspace \mathrm{gal}}{\mathrm{MG}}*\dfrac{\mathrm{day}}{1440 \enspace \mathrm{min}}}{ (3.14*90) \enspace \mathrm{ft}}=\boxed{2,465 \enspace  \mathrm{gpm/ft}}$\\
\vspace{0.3cm}
\textit{Note: The concentration and volume (or flow) units need to be the same.  Thus, the gpm flow rate of Line 1 was converted to math the MGD flow rate unit of Line 2.}


\end{enumerate}

\section*{Practice Problems - Filtration}


\begin{enumerate}
	\item At an average flow of 4,000 gpm, how long of a filter run in hours would be required to produce 25 MG of filtered water?
	\vspace{0.2cm}
	\textbf{Solution}\\
	
	$Flow \enspace rate \enspace (gpm)=\dfrac{Total \enspace flow \enspace (gal)}{Filter \enspace run \enspace time \enspace (min)}$

\vspace{0.3cm}

$\implies Filter \enspace run \enspace time \enspace (min)=\dfrac{Total \enspace flow \enspace (gal)}{Flow \enspace rate \enspace (gpm)}$\\

\vspace{0.3cm}

$\implies Filter \enspace run \enspace time \enspace (hr)=25 \enspace MG*\dfrac{1,000,000 \enspace \cancel{gal}}{MG}*\dfrac{\cancel{min}}{4,000 \enspace \cancel{gal}}*60 \enspace \dfrac{hr}{\cancel{min}}=\boxed{104 \enspace hrs}$

	
	\item A filter is $40 \mathrm{ft}$ long by $20 \mathrm{ft}$ wide. During a test of flow rate, the influent valve to the filter is closed for 6 minutes. The water level drop during this period is 16 inches. What is the filtration rate for the filter in $\mathrm{gpm} / \mathrm{ft}^{2}$ ?\\
	\vspace{0.2cm}
	\textbf{Solution}\\
	
	$\text{Filtration rate, } \mathrm{gpm} / \mathrm{ft}^{2} = 
\dfrac{(
40 \mathrm{ ft}*20 \mathrm{ ft} * 16 \mathrm{\cancel{in}}*
\dfrac{ft}{12 \enspace \cancel{in}}
)
\cancel{ft^3}*7.48 \enspace 
\dfrac
{gal}
{\cancel{ft^3}}}
{40 \enspace ft * 20 \enspace feet}= \boxed{1.7\enspace gpm/ft^2}$\\
	
	
	\vspace{0.2cm}
  \item A water plant has three filters. Each filter is 12 feet wide by 12 feet long. Find the hydraulic loading rate in gpm/sf when all three filters are on-line and the raw water enters the plant at $9.5$ mgd.
	\vspace{0.2cm}
	\textbf{Solution}\\
	
	
	
	
	\vspace{0.2cm}
  \item A sand filter will be backwashed at a rate of $8 \mathrm{gpm} / \mathrm{sf}$. If the filter is 10 feet wide by 15 feet long, what will the filter backwash rise rate be in inches per minute?
  	\vspace{0.2cm}
	\textbf{Solution}\\
	
	
	
	
	\vspace{0.2cm}

  \item A series of filters must be backwashed. Each filter is 20 feet square. If the goal is to achieve a filter backwash rise rate of 30 inches per minute, what should the backwash rate be in gpm/sf?
  	\vspace{0.2cm}
	\textbf{Solution}\\
	
	
	
	
	\vspace{0.2cm}

  \item A water plant has 3 filters. The plant is currently treating $5 \mathrm{mgd}$. If each filter is 12 feet wide by 20 feet long, what is the minimum number of filters that should be placed into service to keep the hydraulic loading rate below 20 gpm/sf?
  	\vspace{0.2cm}
	\textbf{Solution}\\
	
	
	
	
	\vspace{0.2cm}

  \item Find the yield for a filter in lbs/hr/sf given the following information: Filter operates for 12 hours of each day and captures $95 \%$ of the influent solids. The solids load to the filter is 200 pounds per day. The filter is 40 feet square.
  	\vspace{0.2cm}
	\textbf{Solution}\\
	
	
	
	
	\vspace{0.2cm}

  \item Coagulated raw water contains $120 \mathrm{mg} / \mathrm{L}$ of total suspended solids. The water plant produces $2.0 \mathrm{mgd}$ and has two sand filters that are 20 feet wide by 20 feet long. If the filters operate 22 hours of each day and capture $99 \%$ of the coagulated solids, what is the filter yield in lbs/hr/sf? What is the filter yield total in pounds per day?
  	\vspace{0.2cm}
	\textbf{Solution}\\
	
	
	
	
	\vspace{0.2cm}

  \item A series of filters discharge into a combined effluent trough. The trough is 5 feet wide by 80 feet long. A weir runs the full length of the trough. If the water plant capacity is 2 mgd, what is the weir overflow rate in gpd/sf?
  	\vspace{0.2cm}
	\textbf{Solution}\\
	
	
	
	
	\vspace{0.2cm}

\end{enumerate}

\textbf{Solution}
\begin{enumerate}

\item 
\item The volume of the water dropped after the inlet valve was closed would be the filter flow rate.  Since the dimensions to calculate are in feet and inches, the volume needs to be converted from ft$^3$ to gallons\\
\vspace{0.2cm}


\end{enumerate}
\end{enumerate}
\begin{table}[H]
\begin{tabular}{| m{1cm} | m{15cm} |}
\hline
\multicolumn{2}{|l|}{\textbf{Expected   Range of Knowledge for Water Math}}                                                                          \\ \hline
\multicolumn{2}{|l|}{\textit{Water   Distribution System Operator License Exams}}                                                                                      \\ \hline
D1 & Ability to   calculate flow rates for a storage facility                     \\ \hline
D1 & Ability to calculate   the volume of a storage facility                      \\ \hline
D1 & Knowledge of unit   conversions                                              \\ \hline
D1 & Ability to calculate   flow rates                                            \\ \hline
D1 & Ability to calculate   pipe volumes                                          \\ \hline
D1 & Ability to calculate   the area of a pipe cross-section                      \\ \hline
D1 & Ability to calculate   the surface area of a valve face                      \\ \hline
D1 & Ability to calculate   the volume of a cylinder, rectangle, and square       \\ \hline
D1 & Ability to calculate   the volume of a pipe                                  \\ \hline
D1 & Ability to calculate   the volume of a well, storage reservoir, pipe, trench \\ \hline
D1 & Ability to calculate   the well draw down                                    \\ \hline
D1 & Ability to calculate   total force on a valve                                \\ \hline
D1 & Ability to convert   pressure to feet of head                                \\ \hline
D1 & Ability to convert   units of volume, area, and time                         \\ \hline
D1 & Ability to convert   units of volume, area, pressure, and time               \\ \hline
D1 & Ability to convert   units of volume, pressure and area                      \\ \hline
D1 & Ability to convert   water units                                             \\ \hline
D1 & Ability to calculate   a disinfectant dosage                                 \\ \hline
D1 & Ability to calculate   a dosage                                              \\ \hline
D1 & Ability to calculate   the volume of a storage reservoir                     \\ \hline
D1 & Ability to calculate   the volume of a well                                  \\ \hline
D1 & Ability to calculate   the volume of a trench                                \\ \hline

\end{tabular}
\end{table}

\newpage



\begin{table}[H]
\begin{tabular}{| m{1cm} |m{15cm} |}
\hline
\multicolumn{2}{|l|}{\textbf{Expected   Range of Knowledge for Water Math}}                                                                      \\ \hline
\multicolumn{2}{|l|}{\textit{Water   Distribution System Operator License Exams (Continued)}}                                                                  \\ \hline
D2 & Ability to calculate   pipe capacity                                         \\ \hline
D2 & Ability to calculate   the velocity of water                                 \\ \hline
D2 & Ability to calculate   thrust block size                                     \\ \hline
D2 & Ability to calculate   water velocity                                        \\ \hline
D2 & Ability to convert a   pressure reading to depth of water                    \\ \hline
D2 & Ability to convert a   scale to actual distance                              \\ \hline
D3 & Ability to calculate   brake-horsepower                                      \\ \hline
D3 & Ability to calculate   pump efficiency                                       \\ \hline
D3 & Ability to calculate   specific yield of a well                              \\ \hline
D3 & Ability to calculate   the cost of water production                          \\ \hline
D4 & Ability to calculate a water loss rate                                       \\ \hline
D4 & Ability to calculate the cost of pumping   water                             \\ \hline
D4 & Ability to calculate the hydraulic gradient                                  \\ \hline
D4 & Ability to calculate water production costs                                  \\ \hline
\multicolumn{2}{|l|}{\textit{Water   Treatment Operator License Exams}}                                                                  \\ \hline
T1 & Ability to calculate   flow rates and water velocity                         \\ \hline
T1 & Ability to calculate   the volume of water in a storage facility             \\ \hline
T1 & Ability to calculate   well head pressure                                    \\ \hline
T1 & Ability to convert   common water units (e.g. gallons per minute to MGD)     \\ \hline
T1 & Ability to convert   head pressure to water elevation                        \\ \hline
T1 & Ability to convert   units of length, volume, flow and pressure              \\ \hline
T1 & Ability to determine   water level in a storage tank, reservoir, or well     \\ \hline
T1 & Ability to calculate   a chemical dosage                                     \\ \hline
T1 & Ability to calculate   a chemical solution concentration                     \\ \hline
T1 & Ability to calculate   chlorine demand and chlorine residual                 \\ \hline
T1 & Ability to convert   common water units, (gallons per minute to MGD, etc...) \\ \hline
T1 & Ability to determine   water level in a storage tank, reservoir or well      \\ \hline
T1 & Ability to calculate   well drawdown                                                                                              \\ \hline
T1 & Ability to calculate   detention time                                                                                             \\ \hline
T1 & Ability to calculate   well specific capacity                                                                                     \\ \hline
T2 & Ability to calculate   daily filter production                               \\ \hline
T2 & Ability to calculate   filter backwash rate                                  \\ \hline
T2 & Ability to calculate   a CT value                                            \\ \hline
T3 & Ability to perform   blending calculations                                   \\ \hline
T3 & Ability to calculate   a dilution factor                                     \\ \hline
T3 & Ability to mix   chemicals and prepare reagents                              \\ \hline
T3 & Ability to perform   dilutions                                               \\ \hline
T3 & Ability to calculate   a coagulant dose from a jar test                      \\ \hline
T3 & Ability to calculate   a filter-aid dosage                                   \\ \hline
T3 & Ability to calculate   a filtration rate                                     \\ \hline
T3 & Ability to calculate   filter loading rate                                   \\ \hline
T3 & Ability to calculate   percent or log removal of contaminants from water     \\ \hline
T3 & Ability to calculate   the cost of water treatment operations                \\ \hline
\end{tabular}
\end{table}
\newpage










\section{Fractions}\index{Fractions}
\begin{itemize}
\item A fraction is defined as part of whole.  If in a class there are 20 male students and 30 male students, the fraction of male students is $\dfrac{20}{50} or \dfrac{2}{5}$.
\item It is composed of three items: two numbers and a line.
\item The number on the top is called the numerator, the number on the bottom is called the denominator, and the line in between them means to divide. 
$$
\text { Divide } \longrightarrow \dfrac{3}{4} \quad \begin{aligned}
&\text { Numerator } \\
&\text { Denominator }
\end{aligned}
$$
\item A proper fraction is a fraction that has no whole number part and its numerator is smaller than its denominator. An improper fraction is a fraction that has a larger numerator than denominator and it represents a number greater than one.\\
Proper Fraction Examples: $\dfrac{1}{2}$, $\dfrac{5}{8}$, $\dfrac{11}{12}$\\
\vspace{0.2cm}
Improper Fraction Examples: $\dfrac{12}{2}$, $\dfrac{5}{2}$
\item Any whole number can be expressed as a fraction by placing a "1" in the denominator. For example:

2 is the same as $\dfrac{2}{1}$ and 45 is the same as $\dfrac{45}{1}$

\item Only fractions with the same denominator can be added/subtracted, and only the numerators are added/subtracted. For example:
$$
\dfrac{1}{8}+\dfrac{3}{8}=\dfrac{4}{8}  \enspace  \text {and},  \enspace \dfrac{7}{8}-\dfrac{3}{8}=\dfrac{4}{8}
$$

\item A fraction combined with a whole number is called a mixed number. For example:
$$
4 \dfrac{1}{8}, \enspace 16 \dfrac{2}{3}, \enspace  8 \dfrac{3}{4}, \enspace  45  \dfrac{1}{2} \text { and, } 12\dfrac{17}{32}
$$
These numbers are read, four and one eighth, sixteen and two thirds, eight and three fourths, forty-five and one half, and twelve and seventeen thirty seconds.\\
Mized numbers 

\item A fraction can be changed by multiplying the numerator and denominator by the same number. This does not change the value of the fraction, only how it looks. For instance:
$$
\dfrac{1}{2} \text { is the same as } \dfrac{1}{2} \times \dfrac{2}{2} \text { which is } \dfrac{2}{4}
$$

\item Steps to convert $\dfrac{17}{4}$ to a mixed number:
\begin{enumerate}[Step 1.]
\item How many times can 4 fit into 17? 4 because 4×4=16.  Thus, 4 becomes the whole number part
\item How much is left over in the numerator? 1 because $17-16=1$.  Thus, 1 becomes the numerator of the fractional part
\item $\dfrac{17}{4} = 4\dfrac{1}{4}$
\end{enumerate}
\vspace{0.2cm}
\item To turn a mixed number into an improper fraction, multiply the whole number part by the denominator and add the numerator. This becomes the new numerator over the original denominator.

Example: Converting 1.5 feet to fraction\\
$1.5ft=1\dfrac{1}{2}$\\
\vspace{0.2cm}
$1\dfrac{1}{2}=\dfrac{1*2+1}{2}=\dfrac{2+1}{2}=\dfrac{3}{2}$
\vspace{0.2cm}
\item A mixed value - say a circumference is given in feet and fraction of feet (say $7 \enspace 3/4$), needs to be converted to a fraction for calculation purposes.
\end{itemize}

%\begin{tcolorbox}[
%colframe=blue!25,
%colback=blue!10,
%coltitle=blue!20!black,  
%title= Practice Problems]
%\begin{enumerate}
%\item Convert 22$\dfrac{1}{4}$ into a fraction
%\item Express 10ft, 6in as a fraction
%\item Express 10ft, 6in as decimal
%\item Add $\dfrac{3}{4}+\dfrac{1}{7}$
%\item Multiply $\dfrac{4}{9}*\dfrac{3}{16}$
%\end{enumerate}
%\end{tcolorbox}


\section{Decimals \& Powers of Ten}\index{Decimals \& Powers of Ten}
\begin{itemize}
\item A decimal is composed of two sets of numbers: the numbers to the left of the decimal are whole numbers, and numbers to the right of the decimal are parts of whole numbers, a fraction of a number.\\

\item The term used to express the fraction component is dependent on the number of characters to the right of the decimal.

\begin{itemize}
  \item The first character after the decimal point is tenths: $0.1$ - tenths

  \item The second character is hundredths: $0.01$ - hundredths

  \item The third character is thousandths: $0.001$ - thousandths
\end{itemize}

\item Powers of 10 notation enables us to work with these very large and small quantities efficiently.
\item In water math, the most common application of this concept is related to parts per million (ppm) or parts per billion (ppb).
\item 1 million - 1,000,000 can be represented as 10$^6$.  Likewise, 1 billion - 1,000, 000,000 can be represented as 10$^9$
\item The sequence of powers of ten can also be extended to negative powers.
\item 1 part per million (1/1,000,000) can be written as 10$^-6$
\end{itemize}



\begin{table}[ht]
\begin{tabular}{|l|l|l|l|l|}
\hline
\multicolumn{1}{|c|}{\textbf{Name}} & \multicolumn{1}{c|}{\textbf{Power}} & \multicolumn{1}{c|}{\textbf{Number}} & \multicolumn{1}{c|}{\textbf{SI symbol}} & \multicolumn{1}{c|}{\textbf{SI prefix}} \\ \hline
one                                 & $10^0$& 1                                    &                                         &                                         \\ \hline
ten                                 & $10^1$                                   & 10                                   & da (D)                                  & deca                                    \\ \hline
hundred                             & $10^2$                                   & 100                                  & h (H)                                   & hecto                                   \\ \hline
thousand                            & $10^3$                                  & 1,000                                & k (K)                                   & kilo                                    \\ \hline
million                             & $10^6$                                   & 1,000,000                            & M                                       & mega                                    \\ \hline
billion                             & $10^9$                                  & 1,000,000,000                        & G                                       & giga                                    \\ \hline
tenth                               & $10^{-1}$                                 & 0.1                                  & d                                       & deci                                    \\ \hline
hundredth                           & $10^{-2}$                                  & 0.01                                 & c                                       & centi                                   \\ \hline
thousandth                          & $10^{-3} $                                 & 0.001                                & m                                       & milli                                   \\ \hline
millionth                           &$10^{-6} $                               & 0.000 001                            & $\mu$                                      & micro                                   \\ \hline
billionth                           & $10^{-9} $                               & 0.000 000 001                        & n                                       & nano                                    \\ \hline
\end{tabular}
\end{table}

%\begin{tcolorbox}[
%colframe=blue!25,
%colback=blue!10,
%coltitle=blue!20!black,  
%title= Practice Problems]
%\begin{enumerate}
%\item Write the equivalent of 10,000,000 as a power of ten
%\item Find the product of $3.4564*10^2$
%\item Find the product of $534.567*10^{-2}$
%\vspace{0.2cm}
%\item Find the value of $\dfrac{165.93}{10^{-2}}$
%\vspace{0.2cm}
%\item Find the value of $0.023*10^4$
%\end{enumerate}
%\end{tcolorbox}

\section{Rounding and Significant Digits}\index{Rounding and Significant Digits}

\begin{itemize}
\item Significant digits (also called Significant Figures) are digits which give us useful information about the accuracy of a measurement and are related to rounding.
\item This concept is used to determine the direction to round a number (answer). The basic idea is that no answer can be more accurate than the least accurate piece of data used to calculate the answer.\\
\item Significant digits is the count of the numerals in a measured quantity (counting from the left) whose values are considered as known exactly, plus one more whose value could be one more or one less.\\
\item Rules for determining the number of significant digits:
\begin{enumerate}
\item All nonzero digits are significant:\\
1.234 g has 4 significant figures, and 1.2 g has 2 significant figures.
\item Zeroes between nonzero digits are significant:
1002 kg has 4 significant figures, 3.07 mL has 3 significant figures.
\item Zeroes to the left of the first nonzero digits are not significant; such zeroes merely indicate the position of the decimal point:
\SI{0.001}{\celsius} has only 1 significant figure, 0.012 g has 2 significant figures.
\item Zeroes to the right of a decimal point in a number are significant:
0.023 mL has 2 significant figures, 0.200 g has 3 significant figures.
\item When a number ends in zeroes that are not to the right of a decimal point, the zeroes are not necessarily significant:
190 miles may be 2 or 3 significant figures, 50,600 calories may be 3, 4, or 5 significant figures. The potential ambiguity in the last rule can be avoided by the use of standard exponential, or ”scientific,” notation. For example, depending on whether 3, 4, or 5 significant figures is correct, we could write 50,600 calories as: $5.06*10^4$ calories (3 significant figures) $5.060*10^4$ calories (4 significant figures), or
$5.0600*10^4)$ calories (5 significant figures).
\end{enumerate}
\item Examples of significant figures:
% Please add the following required packages to your document preamble:
% \usepackage[normalem]{ulem}
% \useunder{\uline}{\ul}{}
% Please add the following required packages to your document preamble:
% \usepackage[normalem]{ulem}
% \useunder{\uline}{\ul}{}
\begin{table}[h]
\begin{tabular}{|p{16cm}|}
\hline
\scriptsize{1000 has   one significant digit: only the 1 is interesting (only it tells us anything   specific); we don't know anything for sure about the hundreds, tens, or units   places; the zeroes may just be placeholders; they may have rounded something   off to get this value.                                    } \\ \hline
\scriptsize{1000.0 has five significant   digits: the ".0" tells us something interesting about the presumed   accuracy of the measurement being made; namely, that the measurement is   accurate to the tenths place, but that there happen to be zero tenths.                                                               } \\ \hline
\scriptsize{0.00035 has two significant   digits: only the 3 and 5 tell us something; the other zeroes are   placeholders, only providing information about relative size.                                                                                                                                                    } \\ \hline
\scriptsize{0.000350 has three significant   digits: the last zero tells us that the measurement was made accurate to that   last digit, which just happened to have a value of zero.                                                                                                                                         } \\ \hline
\scriptsize{1006 has four significant   digits: the 1 and 6 are interesting, and we have to count the zeroes, because   they're between the two interesting numbers.                                                                                                                                                          } \\ \hline
\scriptsize{560 has two significant   digits: the last zero is just a placeholder.                                                                                                                                                                                                                                            } \\ \hline
\scriptsize{560. : notice that   "point" after the zero! This has three significant digits, because   the decimal point tells us that the measurement was made to the nearest unit,   so the zero is not just a placeholder.                                                                                                  } \\ \hline
\scriptsize{560.0 has four significant   digits: the zero in the tenths place means that the measurement was made   accurate to the tenths place, and that there just happen to be zero tenths;   the 5 and 6 give useful information, and the other zero is between   significant digits, and must therefore also be counted.} \\ \hline
\end{tabular}
\end{table}
\item Addition and Subtraction\\
\begin{itemize}
\item When you are adding or subtracting a bunch of numbers and need to be concerned with significant figures, first add (or subtract) the numbers given in their entire format, and then round the final answer. When rounding the final answer after adding or subtracting, the answer must be written with the same significant figures as the least accurate decimal place given.\\
\textbf{Example:} 13.214 + 234.6 + 7.0350 + 6.38\\
\begin{itemize}
\item 13.214 + 234.6 + 7.0350 + 6.38 = 261.2290\\
\item 234.6 is only accurate to the tenths place making it the least accurate number. My answer must be rounded to the same place as the least accurate number:\\
\item 261.2290 rounds to 261.2 (one decimal place)\\
\end{itemize}
\end{itemize}
\item Multiplication and Division\\
\begin{itemize}
\item When multiplying or dividing multiple numbers you would do these calculations as normal. When the answer must be written in the appropriate significant figure your answer must round to the same number of significant figures as the least number of significant figures.\\
\textbf{Example 1:}  Simplify, and round, to the appropriate number of significant digits \\
\begin{center}
16.235 x 0.217 x 5\\
\end{center}
\begin{enumerate}[Step 1.]
\item First off, 5 has only one significant figure, thus the final answer needs to be rounded to one significant digit\\
\item 16.235 x 0.217 x 5 = 17.614975\\
\item To round 17.614975 to one digit. I'll start with the 1 in the tens place. Immediately to its right is a 7, which is greater than 5, so 1 is rounded up to 2, and then replacing the 7 with a zero, and dropping the decimal point and everything after it.
\item 17.614975 rounds to 20\\
\end{enumerate}
\textbf{Example 2:}  Simplify, and round, to the appropriate number of significant figures\\
\begin{center}
0.00435 x 4.6
\end{center}
\begin{enumerate}[Step 1.]
\item 4.6 has only 2 significant figures, so the final answer should be rounded to two significant figures.
\item 0.00435 x 4.6 = 0.02001\\
\item 0.02001 would round to 0.020, which has 2 significant figures (0.020). The answer cannot be 0.02, because that value would have only one significant figure.\\
\end{enumerate}
\end{itemize}
\end{itemize}

\begin{itemize}
\item \emph{A number is rounded off by dropping one or more numbers from the right and adding zeroes, if necessary, to maintain the decimal point.} 
\item \emph{If the last figure dropped is 5 or more, increase the last retained figure by 1. If the last digit dropped is less than 5, do not increase the last retained figure.}
\end{itemize}


\section{Averages}\index{Averages}
\begin{itemize}
\item Also known as \emph{arithmetic mean}, this value is arrived at by adding the quantities in a series and dividing the total by the number in the series.
\end{itemize}
Example 1: Find the average of the following series of numbers: 12,8,6,21,4,5 , 9 , and 12.\\
Adding the numbers together we get 77.\\
There are 8 numbers in this set.\\
Divide 77 by 8.\\

$\dfrac{77}{8}=9.6$ is the average of the set\\

Example 2:  Find the average of the set of daily turbidity data - 0.3,0.4,0.3,0.1,and 0.8\\
The total is 1.9.\\
There are 5 numbers in the set.\\
Therefore:
$$
\dfrac{1.9}{5}=0.38, \text { rounding off }=0.4
$$






%\section*{Practice Problems - Averages}
%
%\begin{tcolorbox}[
%colframe=blue!25,
%colback=blue!10,
%coltitle=blue!20!black,  
%title= Practice Problems]
%\begin{enumerate}
%\item Find the average of the following set of numbers:\\
%$
%\begin{aligned}
%&0.2 \\
%&0.2 \\
%&0.1 \\
%&0.3 \\
%&0.2 \\
%&0.4 \\
%&0.6 \\
%&0.1 \\
%&0.3
%\end{aligned}
%$
%
%\item The chemical used for each day during a week is given below. Based on these data, what was the average lb/day chemical used during the week?\\
%
%\begin{tabular}{|l|l|}
%\hline
%Monday & 92 lb/day\\
%\hline
%Tuesday & 93 lb/day \\
%\hline
%Wednesday & 98 lb/day\\
%\hline
%Thursday & 93 lb/day \\
%\hline
%Friday & 89 lb/day\\
%\hline
%Saturday & 93 lb/day \\
%\hline
%Sunday & 97 lb/day\\
%\hline
%\end{tabular}
%
%\item The average chemical use at a plant is 77 lb/day. If the chemical inventory is 2800 lbs, how many days supply is this?
%\end{enumerate}
%\end{tcolorbox}




\section{Working with Percent}\index{Working with Percent}
\begin{itemize}
\item Percent expresses portions of the whole.  
\item \texthl{The whole is considered as 1 or 100\% and a part of the whole can be expressed as a percent.}
\textbf{Example:} If a tank is $1 / 2$ full, we say that it contains $50 \%$ of the original solution.
\item Percentage is written as a whole number with a \% sign after it. 
\item In a calculation, percent is expressed as a decimal. 
\item \texthl{The decimal form of a percent value is obtained by dividing the percent by 100.}\\
 \textbf{Example:} $11 \%$ is expressed as the decimal $0.11$, since $11 \%$ is equal to $11 / 100$. This decimal is obtained by dividing 11 by 100.
\item \texthl{To determine what percentage a part is of the whole, divide the part by the whole.}\\
\textbf{Example:} There are 80 water meters to read, Jim has finished 24 of them. What percentage of the meters have been read?\\
$$24 \div 80=0.30$$\\
The $0.30$ is converted to percent by multiplying the answer by 100.\\
$$0.30 \times 100=30 \%$$\\
Thus $30 \%$ of the 80 meters have been read.\\

\item \texthl{To find the percentage of a number, multiply the number by the decimal equivalent of the percentage given in the problem.}\\
\textbf{Example:}\\
What is $28 \%$ of $286 ?$\\

\begin{enumerate}[Step 1.]
\item Change the $28 \%$ to a decimal equivalent:  $$28 \% \div 100=0.28$$
\item Multiply $286 \times 0.28=80$\\
Thus $28 \%$ of 286 is 80.
\end{enumerate}

\item \texthl{To increase a value by a percent, add the decimal equivalent of the percent to " 1 " and multiply it times the number.}

\textbf{Example:} A filter bed will expand $25 \%$ during backwash. If the filter bed is 36 inches deep, how deep will it be during backwash?\\

\begin{enumerate}[Step 1.]
\item Change the percent to a decimal.
$$
25 \% \div 100=0.25
$$
\item Add the whole number 1 to this value.
$$
1+0.25=1.25
$$
\item Multiply times the value.
$$
36 \text { in } \times 1.25=45 \text { inches }
$$
\end{enumerate}
\end{itemize}



%
%\begin{tcolorbox}[
%colframe=blue!25,
%colback=blue!10,
%coltitle=blue!20!black,  
%title= Practice Problems]
%\begin{enumerate}
%\item $25 \%$ of the chlorine in a 30-gallon vat has been used. How many gallons are remaining in the vat?
%
%\item The annual public works budget is $\$ 147,450$. If $75 \%$ of the budget should be spent by the end of September, how many dollars are to be spent? How many dollars will be remaining?
%
%\item A 75 pound container of calcium hypochlorite has a purity of $67 \%$. What is the total weight of the calcium hypochlorite? 
%
%\item $3 / 4$ is the same as what percentage?
%
%\end{enumerate}
%\end{tcolorbox}









\section{Area \& Volume}\index{Area \& Volume}
% \section{Area \& Volume}\index{Area \& Volume}

% \begin{snugshade*}
% 	\item \noindent\textsc{Area \& Volume}
% \end{snugshade*}

\begin{center}
\includegraphics[scale=0.5]{Area&VolumeFormula}
\end{center}
\textbf{Example 1:} The floor of a rectangular building is 20 feet long by 12 feet wide and the inside walls are 10 feet high. Find the total surface area of the inside walls of this building\\
Solution:\\
% \begin{center}
\begin{tikzpicture}
	%%% Edit the following coordinate to change the shape of your
	%%% cuboid
      
	%% Vanishing points for perspective handling
	\coordinate (P1) at (-7cm,1.5cm); % left vanishing point (To pick)
	\coordinate (P2) at (8cm,1.5cm); % right vanishing point (To pick)

	%% (A1) and (A2) defines the 2 central points of the cuboid
	\coordinate (A1) at (0em,0cm); % central top point (To pick)
	\coordinate (A2) at (0em,-2cm); % central bottom point (To pick)

	%% (A3) to (A8) are computed given a unique parameter (or 2) .8
	% You can vary .8 from 0 to 1 to change perspective on left side
	\coordinate (A3) at ($(P1)!.8!(A2)$); % To pick for perspective 
	\coordinate (A4) at ($(P1)!.8!(A1)$);

	% You can vary .8 from 0 to 1 to change perspective on right side
	\coordinate (A7) at ($(P2)!.7!(A2)$);
	\coordinate (A8) at ($(P2)!.7!(A1)$);

	%% Automatically compute the last 2 points with intersections
	\coordinate (A5) at
	  (intersection cs: first line={(A8) -- (P1)},
			    second line={(A4) -- (P2)});
	\coordinate (A6) at
	  (intersection cs: first line={(A7) -- (P1)}, 
			    second line={(A3) -- (P2)});

	%%% Depending of what you want to display, you can comment/edit
	%%% the following lines

	%% Possibly draw back faces

	\fill[gray!40] (A2) -- (A3) -- (A6) -- (A7) -- cycle; % face 6
	\node at (barycentric cs:A2=1,A3=1,A6=1,A7=1) {\tiny Floor=W*L};
	
	\fill[gray!50] (A3) -- (A4) -- (A5) -- (A6) -- cycle; % face 3
	\node at (barycentric cs:A3=1,A4=1,A5=1,A6=1) {\tiny Wall - W*H};
	
	\fill[gray!10, opacity=0.2] (A5) -- (A6) -- (A7) -- (A8) -- cycle; % face 4
	\node at (barycentric cs:A5=1,A6=1,A7=1,A8=1) {\tiny Wall - L*H};
	
	\fill[gray!10,opacity=0.5] (A1) -- (A2) -- (A3) -- (A4) -- cycle; % f2
	\node at (barycentric cs:A1=1,A2=1,A3=1,A4=1) {\tiny Wall - L*H};
	
	\fill[gray!40,opacity=0.2] (A1) -- (A4) -- (A5) -- (A8) -- cycle; % f5
	\node at (barycentric cs:A1=1,A4=1,A5=1,A8=1) {\tiny Ceiling=W*L};	
	
	\draw[thick,dashed] (A5) -- (A6);
	\draw[thick,dashed] (A3) -- (A6);
	\draw[thick,dashed] (A7) -- (A6);

	%% Possibly draw front faces

	%\fill[orange] (A1) -- (A8) -- (A7) -- (A2) -- cycle; % face 1
	\node at (barycentric cs:A1=1,A8=1,A7=1,A2=1) {\tiny Wall - W*H};
	


	%% Possibly draw front lines
	\draw[thick] (A1) -- (A2);

	\draw[<->] (-1.8,0.38) -- (-1.8,-1.3)node [midway, above=-1.8mm] {\hspace{-1.3cm}\tiny Height=10'};
	\draw[<->] (-1.6,-1.4) -- (-.3,-2.1)node [midway, above=-2.6mm] {\hspace{-1.3cm}\tiny Length=20'};
	\draw[<->] (2.6,-1.13) -- (0.2,-2.2)node [midway, below=.6mm] {\hspace{1.2cm}\tiny Width=12'};
	\draw[thick] (A3) -- (A4);
	\draw[thick] (A7) -- (A8);
	\draw[thick] (A1) -- (A4);
	\draw[thick] (A1) -- (A8);
	\draw[thick] (A2) -- (A3);
	\draw[thick] (A2) -- (A7);
	\draw[thick] (A4) -- (A5);
	\draw[thick] (A8) -- (A5);
	
	% Possibly draw points
	% (it can help you understand the cuboid structure)
%	\foreach \i in {1,2,...,8}
%	{
%	  \draw[fill=black] (A\i) circle (0.15em)
%	    node[above right] {\tiny \i};
%	}
	% \draw[fill=black] (P1) circle (0.1em) node[below] {\tiny p1};
	% \draw[fill=black] (P2) circle (0.1em) node[below] {\tiny p2};
\end{tikzpicture}\\
% \end{center}
2 Walls W*H + 2 Walls L*H= $2*12*10ft^2 + 2*20*10ft^2$\\
$=240+400=\boxed{640ft^2}$\\

2 Walls W*H + 2 Walls L*H + Floor + Ceiling= $2*12*10ft^2 + 2*20*10ft^2 + 2*12*20ft^2$\\
$=240+400+480=\boxed{1,120ft^2}$\\

\textbf{Example 2:} How many gallons of paint will be required to paint the inside walls of a 40 ft long x 65 ft wide x 20 ft high tank if the paint coverage is 150 sq. ft per gallon.  Note:  We are painting walls only.  Disregard the floor and roof areas.\\
Solution:\\
\vspace{0.3cm}
% \begin{center}
\begin{tikzpicture}
	%%% Edit the following coordinate to change the shape of your
	%%% cuboid
      
	%% Vanishing points for perspective handling
	\coordinate (P1) at (-7cm,1.5cm); % left vanishing point (To pick)
	\coordinate (P2) at (8cm,1.5cm); % right vanishing point (To pick)

	%% (A1) and (A2) defines the 2 central points of the cuboid
	\coordinate (A1) at (0em,0cm); % central top point (To pick)
	\coordinate (A2) at (0em,-2cm); % central bottom point (To pick)

	%% (A3) to (A8) are computed given a unique parameter (or 2) .8
	% You can vary .8 from 0 to 1 to change perspective on left side
	\coordinate (A3) at ($(P1)!.8!(A2)$); % To pick for perspective 
	\coordinate (A4) at ($(P1)!.8!(A1)$);

	% You can vary .8 from 0 to 1 to change perspective on right side
	\coordinate (A7) at ($(P2)!.7!(A2)$);
	\coordinate (A8) at ($(P2)!.7!(A1)$);

	%% Automatically compute the last 2 points with intersections
	\coordinate (A5) at
	  (intersection cs: first line={(A8) -- (P1)},
			    second line={(A4) -- (P2)});
	\coordinate (A6) at
	  (intersection cs: first line={(A7) -- (P1)}, 
			    second line={(A3) -- (P2)});

	%%% Depending of what you want to display, you can comment/edit
	%%% the following lines

	%% Possibly draw back faces

	\fill[gray!40] (A2) -- (A3) -- (A6) -- (A7) -- cycle; % face 6
	\node at (barycentric cs:A2=1,A3=1,A6=1,A7=1) {};
	
	\fill[gray!50] (A3) -- (A4) -- (A5) -- (A6) -- cycle; % face 3
	\node at (barycentric cs:A3=1,A4=1,A5=1,A6=1) {\tiny Wall - W*H};
	
	\fill[gray!10, opacity=0.2] (A5) -- (A6) -- (A7) -- (A8) -- cycle; % face 4
	\node at (barycentric cs:A5=1,A6=1,A7=1,A8=1) {\tiny Wall - L*H};
	
	\fill[gray!10,opacity=0.5] (A1) -- (A2) -- (A3) -- (A4) -- cycle; % f2
	\node at (barycentric cs:A1=1,A2=1,A3=1,A4=1) {\tiny Wall - L*H};
	
	\fill[gray!40,opacity=0.2] (A1) -- (A4) -- (A5) -- (A8) -- cycle; % f5
	\node at (barycentric cs:A1=1,A4=1,A5=1,A8=1) {};	
	
	\draw[thick,dashed] (A5) -- (A6);
	\draw[thick,dashed] (A3) -- (A6);
	\draw[thick,dashed] (A7) -- (A6);

	%% Possibly draw front faces

	%\fill[orange] (A1) -- (A8) -- (A7) -- (A2) -- cycle; % face 1
	\node at (barycentric cs:A1=1,A8=1,A7=1,A2=1) {\tiny Wall - W*H};
	


	%% Possibly draw front lines
	\draw[thick] (A1) -- (A2);

	\draw[<->] (-1.8,0.38) -- (-1.8,-1.3)node [midway, above=-1.8mm] {\hspace{-1.3cm}\tiny Height=20'};
	\draw[<->] (-1.6,-1.4) -- (-.3,-2.1)node [midway, above=-2.6mm] {\hspace{-1.3cm}\tiny Length=40'};
	\draw[<->] (2.6,-1.13) -- (0.2,-2.2)node [midway, below=.6mm] {\hspace{1.2cm}\tiny Width=65'};
	\draw[thick] (A3) -- (A4);
	\draw[thick] (A7) -- (A8);
	\draw[thick] (A1) -- (A4);
	\draw[thick] (A1) -- (A8);
	\draw[thick] (A2) -- (A3);
	\draw[thick] (A2) -- (A7);
	\draw[thick] (A4) -- (A5);
	\draw[thick] (A8) -- (A5);
	
	% Possibly draw points
	% (it can help you understand the cuboid structure)
%	\foreach \i in {1,2,...,8}
%	{
%	  \draw[fill=black] (A\i) circle (0.15em)
%	    node[above right] {\tiny \i};
%	}
	% \draw[fill=black] (P1) circle (0.1em) node[below] {\tiny p1};
	% \draw[fill=black] (P2) circle (0.1em) node[below] {\tiny p2};
\end{tikzpicture}\\
% \end{center}
\vspace{0.3cm}
2 Walls W*H + 2 Walls L*H = $2*65*20ft^2 + 2*40*20ft^2= 2,600+1,600=4,200ft^2$\\
$\implies @150\dfrac{ft^2}{gal} \enspace paint \enspace coverage \enspace \rightarrow \enspace \dfrac{4,200\cancel{ft^2}}{150\dfrac{\cancel{ft^2}}{gal}}=\boxed{28 \enspace gallons}$
\vspace{0.3cm}
\textbf{Example 3:}  What is the circumference of a 100 ft diameter circular sedimentation tank?\\
\vspace{0.3cm}
Solution:\\
\vspace{0.3cm}
$Circumference=\pi*D=3.14*100ft=\boxed{314ft}$
\vspace{0.3cm}

\textbf{Example 4:} If the surface area of a clarifier is 5,025$ft^2$, what is its diameter?\\
\vspace{0.3cm}
Solution:\\
\vspace{0.3cm}
$Surface \enspace area=\dfrac{\pi}{4}*D^2 \enspace \implies 5025(ft^2)=0.785*D^2 (ft^2)$\\
$\implies D^2=\dfrac{5025}{0.785} \implies D=\sqrt{6401.3}=\boxed{80ft}$
\vspace{0.3cm}

\textbf{Example 5:} How many gallons of water would 600 feet of 6-inch diameter pipe hold, approximately?\\
\vspace{0.3cm}
Solution:\\

\vspace{0.3cm}
% \begin{center}
\begin{tikzpicture}
\draw (0,0) ellipse (0.1cm and 0.3cm);
\draw (10,0) ellipse (0.1cm and 0.3cm);
\draw [-] (0,-0.29) -- (10,-0.29);
\draw [-] (0,0.29) -- (10,0.29);
\draw [<->] (10,-0.28) -- (10,0.28) node [midway, below=-3mm] {\hspace{2.6cm}Diameter=6"};
\draw [<->] (0,-.68) -- (10,-.68)node [midway, below] {\hspace{0.9cm}Length=600'};
\end{tikzpicture}
% \end{center}
\vspace{0.3cm}
$Volume=\dfrac{\pi}{4}D^2*L=0.785*\Big(\dfrac{6}{12}\Big)^2*600\cancel{ft^3}*7.48\dfrac{gallons}{\cancel{ft^3}}=\boxed{881 \enspace gallons}$

%\begin{tcolorbox}[
%colframe=blue!25,
%colback=blue!10,
%coltitle=blue!20!black,  
%title= Practice Problems]
%\begin{enumerate}
%
%\item A 60-foot diameter tank contains 422,000 gallons of water. Calculate the height of water in the storage tank.
%
%\item What is the volume of water in ft$^3$, of a sedimentation basin that is 22 feet long, and 15 feet wide, and filled to 10 feet?
%
%\item What is the volume in ft$^3$ of an elevated clear well that is 17.5 feet in diameter, and filled to 14 feet?
%
%\item What is the area of the top of a storage tank that is 75 feet in diameter?\\
%
%\item  What is the area of a wall $175 \mathrm{ft}$. in length and $20 \mathrm{ft}$. wide?\\
%
%\item  You are tasked with filling an area with rock near some of your equipment. 1 Bag of rock covers 250 square feet. The area that needs rock cover is 400 feet in length and 30 feet wide. How many bags do you need to purchase?\\
%\end{enumerate}
%\end{tcolorbox}

\section{Flow and Velocity}\index{Flow and Velocity}
\begin{itemize}
\item Flow Rate - Q (volume/time) = velocity (distance or length traveled /time) * surface area
\item Velocity is the speed at which the water is flowing.  It is measured in units of length/time – ft./sec.
\item Velocity of water flowing through can be calculated by dividing the flow rate by area of the flow stream.\\
\vspace{0.5cm}
$$Velocity \enspace \dfrac{length}{time}= \dfrac{flow \enspace rate(\dfrac{volume \enspace or \enspace cubic \enspace length}{time})}{surface \enspace area \enspace in \enspace the \enspace direction \enspace of \enspace flow-square \enspace length}$$
\vspace{0.5cm}
\textbf{For a flow in a channel:}\\
\vspace{0.5cm}
\includegraphics[scale=0.5]{ChannelFlow3}\\

\textbf{For a flow in a pipe:}\\
\vspace{0.5cm}
\includegraphics[scale=0.65]{VelocityinPipe}\\
\vspace{0.5cm}
\end{itemize}
\subsection*{Example Problems}
\textbf{Example:} If a chemical is added in a pipe where water is flowing at a velocity of 3.1 feet per second, how many minutes would it take for the chemical to reach a point 7 miles away?  \\

Note - we want the answer in minutes\\

$$\textrm{Min } = \dfrac{1}{3.1}\dfrac{sec}{ft}*\dfrac{5280ft}{mile}*7 miles*\dfrac{min}{60 sec} = \boxed{199 min}$$
\\

\textbf{Example:} Find the flow in cfs in a 6 -inch line, if the velocity is 2 feet per second.

\begin{enumerate}
\item Determine the cross-sectional area of the line in square feet. Start by converting the diameter of the pipe to inches.

The diameter is 6 inches: therefore, the radius is 3 inches. 3 inches is $3 / 12$ of a foot or $0.25$ feet.

\item Now find the area in square feet.
$$
\begin{aligned}
&A=\pi \times r^{2} \\
&A=\pi \times\left(0.25 \mathrm{ft}^{2}\right. \\
&A=\pi \times 0.0625 \mathrm{ft}^{2} \\
&A=0.196 \mathrm{ft}^{2}
\end{aligned}
$$
Or
$$
\begin{aligned}
&A=0.785 \times D^{2} \\
&A=0.785 \times 0.5^{2} \\
&A=0.785 \times .05 \times .05 \\
&A=0.196 \mathrm{ft}^{2}
\end{aligned}
$$

\item Now find the flow.

$\mathrm{Q}=\mathrm{V} \times \mathrm{A}$

$\mathrm{Q}=2 \mathrm{ft} / \mathrm{sec} \times 0.196 \mathrm{ft}^{2}$

$\mathrm{Q}=0.3927 \mathrm{cfs}$ or $0.4 \mathrm{cfs}$

\end{enumerate}


\textbf{Example:} A rectangular channel 3 ft. wide contains water 2 ft. deep flowing at a velocity of 1.5 fps.
What is the flow rate in cfs?

\includegraphics[scale=0.5]{ChannelFlow3}\\
$Q=V*A \implies Q = 1.5 \dfrac{ft}{sec}*(3*2)ft^2=\boxed{9\dfrac{ft^3}{sec}}$

%\vspace{1cm}
%
%\begin{tcolorbox}[
%colframe=blue!25,
%colback=blue!10,
%coltitle=blue!20!black,  
%title= Practice Problems]
%\begin{enumerate}
%
%\item Flow in an 8-inch pipe is 500 gpm. What is the average velocity in ft/sec? (Assume pipe is flowing full)
%
%\item A pipeline is 18” in diameter and flowing at a velocity of 125 ft. per minute. What is the flow in gallons per minute?
%
%\item The velocity in a pipeline is 2 ft./sec. and the flow is 3,000 gpm. What is the diameter of the pipe in inches?
%
%
%
%\item Find the flow in a 4-inch pipe when the velocity is $1.5$ feet per second.
%
%  \item A 42-inch diameter pipe transfers 35 cubic feet of water per second. Find the velocity in $\mathrm{ft} / \mathrm{sec}$. 
%  
%  \item A plastic float is dropped into a channel and is found to travel 10 feet in $4.2$ seconds. The channel is $2.4$ feet wide and $1.8$ feet deep. Calculate the flow rate of water in cfs.
%  \end{enumerate}
%  \end{tcolorbox}

\section{Contaminant Removal Efficiency}\index{Contaminant Removal Efficiency}
\begin{itemize}
\item Contaminant removal efficiency can be expressed as the percentage of the inlet concentration removed and can be established based upon the amount of a particular contaminant entering and leaving a treatment process.

\item $Percent \enspace Removal \enspace (\%) = \dfrac{Concentration \enspace  In-Concentration\enspace  Out}{Concentration \enspace In}*100$\\

\item If 10 units of a contaminant are entering a process and 8 units of pollutant are leaving (process removes 2 units), then the process removal rate for that pollutant is (10-8)/10*100=20\%.  In this example the process is 20\% efficient in removing that particular contaminant.

\item Besides percent removal, removal efficiency can also be expressed in terms of Log removal.
\item Background of log:\\ 
Log of a number $x$ to the \textbf{base} $B$ is the exponent to which $B$ must be \textbf{raised} to produce $x$.\\
\vspace{0.3cm}
\begin{center}
$\log_B x=A \implies B^A=x$\\
For Example: $\log_{10} 1000=3 \implies 10^3=1000$ \\
\end{center}
\vspace{0.3cm}
Log Rules\\
\begin{enumerate}
\item $\log_{b} 1=0$
\item $\log_{b} ac=\log_{b} a + \log_{b} c$
\item $\log_{b} \dfrac{a}{c}=\log_{a} a - \log_{b} c$
\item $\log_{b} a^r=r\log_{b} a$
\item $\log_{b} \dfrac{1}{c}=-\log_{b} c$
\end{enumerate}
\vspace{0.3cm}
\item Log removal is:  $\log_{10} {(Concentration \enspace In)} - \log_{10} {(Concentration \enspace Out)}$
\item \textbf{CASE 1: } Say the initial (before treatment) and final (after treatment) cryptosporidium concentrations are 100 oocysts and 10 oocysts per L respectively. \\
\vspace{0.3cm}
Thus log removal is $\log_{10} 100 - \log_{10} 10 = \log_{10}\dfrac{100}{10}=\log_{10} 10 = \boxed{1} \enspace as \enspace 10^1=10 $\\
\vspace{0.3cm}
\item The removal on a percentage basis: $\mathrm{Percent \enspace removal} = \dfrac{\mathrm{initial}-\mathrm{final}}{\mathrm{intial}}*100=\dfrac{100-10}{100}*100=90\%$\\
\item \textbf{CASE 2: } Say the initial (before treatment) and final (after treatment) cryptosporidium concentrations are 100 oocysts and 1 oocysts per L respectively. \\
\vspace{0.3cm}
Thus log removal is $\log_{10} 100 - \log_{10} 1 = \log_{10}\dfrac{100}{1}=\log_{10} 100 = \boxed{2} \enspace as \enspace 10^2=100 $\\
\vspace{0.3cm}
\item The removal on a percentage basis: $\mathrm{Percent \enspace removal} = \dfrac{\mathrm{initial}-\mathrm{final}}{\mathrm{intial}}*100=\dfrac{100-1}{100}*100=99\%$\\
\vspace{0.5cm}
\item \textbf{CASE 3: } If the initial (before treatment) and final (after treatment) cryptosporidium concentrations are 1000 oocysts and 1 oocysts per L respectively. (unreal values....) \\
\vspace{0.3cm}
Thus log removal is $\log_{10} 1000 - \log_{10} 10 = \log_{10}\dfrac{1000}{1}=\log_{10} 1000 = \boxed{3} \enspace as \enspace 10^3=1000 $\\
\vspace{0.3cm}
The removal on a percentage basis: $\mathrm{Percent \enspace removal} = \dfrac{\mathrm{initial}-\mathrm{final}}{\mathrm{intial}}*100=\dfrac{1000-1}{1000}*100=99.9\%$
\vspace{0.3cm}
\item Thus:\\
1 log removal =90\% removal efficiency\\
2 log removal =99\% removal efficiency\\
3 log removal =99.9\% removal efficiency\\
4 log removal =99.99\% removal efficnecy\\
\end{itemize}


\section{Ratio and Proportion}\index{Ratio and Proportion}
\textbf{Ratio:}\\
\begin{itemize}
\item Ratio is used for comparing the size of two or more quantities.
\item Say if there are 10 red cubes and 5 pink marbles in a bag, the ratio $\dfrac{5}{10}$ is the ratio of pink marbles and red cubes.  It can also be represented by 5:10.
\item 5 lbs of chemical in 10 gallons solution is a ratio.  So is 30 miles per gallon.
\item Unlike fractions, ratio does not compare things that have the same units.
\end{itemize}
\textbf{Proportion:}\\
\begin{itemize}
\item Two quantities are said to be in proportion if one changes, the other changes in a specific way.
\item Two quantities are said to be directly proportional, if the \textbf{increase} in one will \textbf{increase} the other value proportionally.  
\begin{itemize}
\item Thus, if two quantities x and y are directly proportional, its ratio $\dfrac{x}{y}$ will be a fixed value. Thus for x$_1$ and y$_1$ different values of x and y respectively will be related by the equation $\dfrac{x}{y}=\dfrac{x_1}{y_1}$.  

\item This relationship is useful for calculating unknown values in water treatment calculations as in the following example: \\
\vspace{0.2cm}
Knowing 200 lbs of bleach is needed to disinfect 5 MG of water at a treatment plant, calculate the lbs of bleach required to disinfect 3.2 MGD flow.\\

\vspace{0.2cm}

The ratio $\dfrac{200 \enspace pounds \enspace bleach}{5 MG}$ or 40 lbs bleach per MG is a constant.  
Using this known proportion the lbs of bleach is needed to disinfect 3.2 MG at this plant can be calculated as follows by setting up the equation as:\\
\vspace{0.2cm}
$\dfrac{40 \enspace pounds \enspace bleach}{MG}=\dfrac{X}{3.2 \enspace MG}$ where X is the unknown lbs of bleach that is required to disinfect the 3.2 MG flow.\\
\vspace{0.2cm}
X can be calculated by cross multiplying the above equation: $X=\dfrac{3.2*40}{1}=128 \enspace lbs \enspace bleach$
\end{itemize}
\item Two quantities are said to be inversely proportional if the \textbf{increase} in one will \textbf{decrease} the other value proportionally.  
\begin{itemize}
\item Thus, if two quantities x and y are inversely proportional, its product $x * \text{y}$ will be a fixed value and different values of x and y respectively will be related by the equation $x *y = x_1 * y_1$.
\item Examples of inversely proportional relationship include:\\
\vspace{0.2cm}
\begin{itemize}
\item Labor hours required to perform a certain task or time required to pump down a wetwell depending on the size of the pump.  An increase in assignment of labor hours will reduce the time required to perform the task 
\item Using a larger pump will reduce the time to pump down the wetwell.  
\item In the Pounds formula:\\
\vspace{0.2cm}
$$lbs \enspace \textbf{or} \enspace \dfrac{lbs}{day}=Concentration\Big(\dfrac{mg}{l}\Big)*8.34*volume(MG) \enspace \textbf{or} \enspace Flow (MGD)$$\\
 
\vspace{0.2cm}

for the same lbs or lbs/day, concentration varies inversely with volume or flow.  Thus, for a certain pounds added, the concentration will go down if the flow increases and vice versa.
\item In the flow equation, Q=V*A, for the same flow (Q), velocity (V) and surface area (A) are inversely related.  If Q is remaining the same, an increase in surface area will reduce the velocity and vice versa.\\

Additionally, for a flow through a pipe as the surface area of the pipe is proportional to the square of the diameter, the velocity in the pipe is inversely proportional to the square of the diameter.\\

\vspace{0.2cm}

For a constant Q:  $V *A = V_1 *A_1$ or $V *D^2 = V_1 *D_1^2$

\end{itemize}
\vspace{0.2cm}


\vspace{0.2cm}

\item Application of inversely proportional relationship in water related calculation can be demonstration with the following example:

If it takes 20 minutes to pump a wet well down with one pump pumping at 125 gpm, then how long will it take if a 200 gpm pump is used?

As this is an inversely proportional relationship ( a larger pump will reduce the time required):\\
\vspace{0.2cm}
$(20 \mathrm{minutes} * 125 \mathrm{gpm})=(X \mathrm{minutes} * 200 \mathrm{gpm})$ \\

where X is the unknown time to pump down the wetwell with the 200 gpm pump.\\
\vspace{0.2cm}
Solving for X: $X=\dfrac{20*125}{200}=12.5 \enspace \mathrm{minutes}$
\end{itemize}
\end{itemize}


%\begin{tcolorbox}[breakable, enhanced,
%colframe=blue!25,
%colback=blue!10,
%coltitle=blue!20!black,  
%title= Practice Problems]
%
%\begin{enumerate}
%\item It takes 6 gallons of chlorine solution to obtain a proper residual when the flow is 45,000 gpd. How many gallons will it take when the flow is 62,000 gpd?
%
%\item A motor is rated at 41 amps average draw per leg at $30 \mathrm{Hp}$. What is the actual $\mathrm{Hp}$ when the draw is 36 amps? C. 
%
%\item If it takes 2 operators $4.5$ days to clean an aeration basin, how long will it take three operators to do the same job?
%
%\item It takes 3 hours to clean 400 feet of collection system using a sewer ball. How long will it take to clean 250 feet?
%
%\item It takes 14 cups of $\mathrm{HTH}$ to make a $12 \%$ solution, and each cup holds 300 grams. How many cups will it take to make a $5 \%$ solution?
%
%\item A bike travelling at 5 miles/hr completes a journey in 40 minutes. How long would the same journey take if the speed was increased to 8 miles/hr?
%\end{enumerate}
%\end{tcolorbox}








\newpage

\section{Unit Conversions}\index{Unit Conversions}
\begin{itemize}
\item A conversion is a number that is used to multiply or divide into a measure in order to change the units of the original measure.

\begin{table}[h!]

\begin{center}
    \begin{tabular}{ | p{4cm} |p{8cm}|}
    \hline
    
\textbf{Measure} & \textbf{Units}\\
\hline   
Length  & inches, ft, miles\\
\hline 
Area  & ft$^2$, acres \\
\hline 
Volume & ft$^3$, gallons, acres-ft.\\
\hline 
Density & weight per volume, lbs/ft$^3$, lbs/gallon\\
\hline 
Flow & ft$^3$/min, MGD, acres-ft/day\\
\hline 

	

    \end{tabular}
 \caption{Common units in water calculations}	
    \end{center}

    \end{table}

\item In most instances, the conversion factor cannot be derived. It must be known. Therefore, tables such as the one below are used to find the common conversions.\\
\begin{tabular}{|l|l|}
\hline
Some Common Conversions & Weight \\
\hline
Linear Measurements & $1 \mathrm{ft}^{3}$ of water $=62.4 \mathrm{lbs}$ \\
\hline
1 inch $=2.54 \mathrm{~cm}$ & $1 \mathrm{gal}=8.34 \mathrm{lbs}$ \\
$1 \mathrm{foot}=30.5 \mathrm{~cm}$ & $1 \mathrm{lb}=453.6 \mathrm{grams}$ \\
$1 \mathrm{~meter}=100 \mathrm{~cm}=3.281 \mathrm{feet}=39.4$ inches 1 & $1 \mathrm{~kg}=1000 \mathrm{~g}=2.2 \mathrm{lbs}$ \\
acre $=43,560 \mathrm{ft}^{2}$ & $1 \%=10,000 \mathrm{mg} / \mathrm{L}$ \\
$1 \mathrm{yard}=3 \mathrm{feet}$ & $1 \mathrm{pound}=16 \mathrm{oz} \mathrm{dry} \mathrm{wt}$ \\
 & $1 \mathrm{ft}^{3}=62.4 \mathrm{lbs}$ \\
\hline
Volume & Pressure \\
\hline
$1 \mathrm{gal}=3.78$ liters & $1 \mathrm{ft}$ of head $=0.433 \mathrm{psi}$ \\
$1 \mathrm{ft}=7.48$ gal & $1 \mathrm{psi}=2.31 \mathrm{ft}$ of head \\
$1 \mathrm{~L}=1000 \mathrm{~mL}$ &  \\
$1 \mathrm{gal}=16 \mathrm{cups}$ &  \\
\hline
Flow &  \\
\hline
$1 \mathrm{cfs}=448 \mathrm{gpm}$ &  \\
$1 \mathrm{gpm}=1440 \mathrm{gpd}$ &  \\
\hline
\end{tabular}
\vspace{0.2cm}
\item Common conversions in water related calculations include the following:

\begin{itemize}
  \item gpm to cfs

  \item Million gallons to acre feet

  \item Cubic feet to acre feet

  \item Cubic feet of water to gallons


  \item gpm to MGD 

  \item psi to feet of head

\end{itemize}

\item Steps for unit conversion:\\
\begin{enumerate}[Step 1:]
\item \texthl{Make sure the original unit is for the same measurement as the converted (desired) unit.}  So if the original unit is for area, say in ft$^2$ the converted unit should be another area unit such as in$^2$ or acre but it cannot be gallons as gallon is a unit of volume.\\
Note:  Calculating the weight of a certain volume of water involves the use of density which is the mass per volume -  value in units including lbs/gallon or lbs/$ft^3$\\

\item Write down the conversion formula as:\\

$Quantity \enspace in \enspace converted \enspace unit = Quantity \enspace (\cancel{Original \enspace Unit}) *   Conversion  \enspace Factor \enspace  \dfrac{Conversion \enspace unit}{\cancel{Original \enspace unit}}$\\
\end{enumerate}

\item Note:  If you wish to convert cubic feet of water to pounds, you have to use its density which is the known mass per unit volume.\\
$\dfrac{8.34 \enspace lbs}{gallon}$ or $\dfrac{62.4 \enspace lbs}{ft^3}$\\
$mass \enspace of \enspace water = \cancel{Volume} *   Density  (\dfrac{mass}{\cancel{Volume}})$\\

\end{itemize}

Example Problems:\\
\begin{enumerate}
\item Convert 1000 $ft^3$ to cu. yards\\

$1000 \cancel{ft^3}*\dfrac{cu.yards}{27\cancel{ft^3}} = 37 cu.yards$

\item Convert 10 gallons/min to $ft^3$/hr\\
Note:  This involves use of two conversion factors - one for converting gallons to cubic feet and another for converting minute to gallons.\\ 
$\dfrac{10 \cancel{gallons}}{\cancel{min}}*  \dfrac{ft^3}{7.48 \cancel{gallons}}  * \dfrac{60 \cancel{min}}{hr}   = \dfrac{80.2ft^3}{hr}$


\item Convert 100,000 $ft^3$ to acre-ft.\\
$100,000 \cancel{ft^3} * \dfrac{acre-ft}{43,560 \cancel{ft^2-ft}} =  2.3 acre-ft$\\

\item Convert 8 $ft^3$ of water to pounds.\\
Here the conversion is from a volume ($ft^3$) to a weight (lbs).  It involves use of a standard correlation of the volume of water to its weight - its density. 

$Weight \enspace of \enspace water \enspace in \enspace lbs=8 \cancel{ft^3} *   62.4  (\dfrac{lbs}{\cancel{ft^3}}) = 499.2 \enspace lbs $\\

\end{enumerate}

\subsection{Temperature Conversion}\index{Temperature Conversion}
\begin{itemize}
\item Two scales are commonly used to measure temperature: degrees Fahrenheit (\degree{F}) and degrees Centigrade or Celsius(\degree{C}). 
\item Fahrenheit is the standard scale used in the U.S. and Celsius is the metric scale. 
\item In the Celsius scale, water freezes at 0\degree{C} and boils at 100\degree{C}. In the Fahrenheit scale, water freezes at32\degree{F} and boils at 212\degree{F}. 
\item The following factors can be used when converting from one temperature scale to another:
$$\degree{C} = \dfrac{\degree{F}-32}{1.8}$$

$$\degree{F}=(\degree{C} \times 1.8)+32$$
\end{itemize}

%\newpage
%
%\begin{tcolorbox}[breakable, enhanced,
%colframe=blue!25,
%colback=blue!10,
%coltitle=blue!20!black,  
%title= Practice Problems]
%
%\begin{enumerate}
%\item Convert 1000 $ft^3$ to cu. yards\\
%
%\item Convert 10 gallons/min to $ft^3$/hr\\
%
%\item Convert 100,000 $ft^3$ to acre-ft.\\
%
%\item Find the flow in gpm when the total flow for the day is 65,000 gpd.
%
%\item Find the flow in gpm when the flow is $1.3 \mathrm{cfs}$.
%
%\item Find the flow in gpm when the flow is $0.25 \mathrm{cfs}$.
%
%\item The flow rate through a filter is 4.25 MGD. What is this flow rate expressed as gpm?\\
%
%\item After calibrating a chemical feed pump, you've determined that the maximum feed rate is $178 \mathrm{~mL} /$ minute. If this pump ran continuously, how many gallons will it pump in a full day?
%
%\item A plant produces 2,000 cubic foot of water per hour. How many gallons of water is produced in an 8-hour shift?
%
%\item Change 70 °F to °C
%\item Change 140 °F to °C
%\item Change 20 °C to °F
%\item Change 85 °C to °F
%\item Change 4 °C to °F
%\end{enumerate}
%\end{tcolorbox}










\section{Well Hydraulics Calculations} \index{Well Hydraulics Calculations}

\begin{center}
\includegraphics[scale=0.5]{Well1} \hspace{1cm} \includegraphics[scale=0.6]{WellDrawdownCalc}
\end{center}

The amount of water a well will produce depends mainly on the type of aquifer, well construction, and the depth of the zone of saturation. The annual recharge rate from percolation, along with the ability of the water bearing formation to transmit water to any given point, will also influence well production. The performance of a well can be determined by taking readings of the hydraulic conditions. An operator must be familiar with these terms and definitions*, in order to accurately troubleshoot problems that may be discovered.\\
\vspace{0.3cm}
\textbf{Static level }is the water level in a well when the pump is not operating.\\
\vspace{0.3cm}
\textbf{Pumping level} is the water level in the well when it is producing.\\
\vspace{0.3cm}
\textbf{Drawdown} is the difference in elevations between the static level and the pumping level. The amount of water produced is approximately proportional to the draw-down. For example, increasing the yield by 10\% will increase the drawdown by 10\%. The draw-down that occurs when a well is running is roughly equal to the head loss encountered in moving the water into the well. Water bearing formations of gravel, limestone and course sand will usually provide more water with less draw-down than formations containing fine sand or clay.\\
\vspace{0.3cm}
\textbf{Specific capacity} is the relationship between the yield of a well and the amount of drawdown in the well. It can be expressed as a ratio of the yield, in terms of gallons per minute, to the drawdown in feet. A well producing 100 gpm with a drawdown of 20 feet would have a specific capacity of 5 gpm per foot of draw-down. In this particular case every time the yield is increased by 5 gpm the drawdown will increase by one foot. This relationship will exist until the yield exceeds the aquifer’s ability to deliver water to any single point, When this limit is reached, the draw-down increases dramatically with little or no increase in the yield.\\
\vspace{0.3cm}
\textbf{Cone of depression} is directly related to the drawdown in the well. As the pump draws down the water level, a portion of the aquifer surrounding the well is drained of water. A cone shaped depression is formed in the water table around the well. The shape of the cone will vary depending on the type of formation in which the well is located. A fine sand formation will usually create a steep cone of depression, while a shallow cone is usually found in coarse sand and gravel formations.\\
\vspace{0.3cm}
\textbf{Radius of influence} is the farthest distance from the well that the cone of depression affects the water table. This distance can be determined by sinking test holes around the well and monitoring the water levels in them while the well is pumping.\\
\vspace{0.3cm}
\textbf{Recovery time} is the amount of time required for the aquifer to stabilize at its static water level once pumping has stopped. This can also be determined by monitoring the water levels in the test holes used to determine the radius of influence.\\

\textbf{Example Problem \#1:} A well is drilled through an unconfined aquifer. The top of the aquifer is 80 feet below grade. After the well was in service for a year, the water level in the well stabilized at 110 feet below grade. What is the drawdown?\\
$\text {Drawdown} =\text { Static Level}-\text { Pumping Level } =80 \mathrm{ft}-110 \mathrm{ft}=30 \mathrm{feet}$

\textbf{Example Problem \#2:} A well produces 300 gpm. If the drawdown is 30 feet, find the specific yield.\\
$\text { Specific Yield } =\dfrac{\text { Yield }}{\text { Drawdown }} =\dfrac{300 \mathrm{gpm}}{30 \mathrm{ft}} =10 \mathrm{gpm} / \mathrm{ft}$\\
  \vspace{0.3cm}
\textbf{Example Problem \#3:} The specific yield for a well is $10 \mathrm{gpm} / \mathrm{ft}$. If the well produces $550 \mathrm{gpm}$, what is the drawdown?\\
  \vspace{0.3cm}
$\text {Specific Yield }=\dfrac{\text { Yield }}{\text { Drawdown }}\implies 10 \mathrm{gpm} / \mathrm{ft}=\dfrac{550 \mathrm{gpm}}{\text { Drawdown }} \implies \text {Drawdown }= \dfrac{550}{10}=\boxed{55 \mathrm{ft}}$ \\
\textbf{Example Problem \#4:} The pumped water level of a well is 400 feet below the surface. The well produces 350 gpm. If the aquifer level is 250 feet below the surface, what is the specific yield for the well?\\
  $\text {Drawdown} =\text { Static Level}-\text { Pumping Level } =400 \mathrm{ft}-350 \mathrm{ft}=50 \mathrm{feet}$\\
  $\text {Specific Yield }=\dfrac{\text { Yield }}{\text { Drawdown }}=\dfrac{350 \mathrm{gpm}}{\text { 50 }}=7 \mathrm{gpm} / \mathrm{ft}$ \\


%\begin{tcolorbox}[breakable, enhanced,
%colframe=blue!25,
%colback=blue!10,
%coltitle=blue!20!black,  
%title= Practice Problems]
%
%\begin{enumerate}
%
%\item A well yields 2,840 gallons in exactly 20 minutes. What is the well yield in gpm?\\
%
%\item Before pumping, the water level in a well is 15 ft. down. During pumping, the water level is 45 ft. down. The drawdown is:\\
%
%\item A well is located in an aquifer with a water table elevation 20 feet below the ground surface. After operating for three hours, the water level in the well stabilizes at 50 feet below the ground surface. The pumping water level is:\\
%
%\item Calculate drawdown, in feet, using the following data:\\
%The water level in a well is 20 feet below the ground surface when the pump is not in operation, and the water level is 35 feet below the ground surface when the pump is in operation.\\
%
%
%\item Calculate the well yield in gpm, given a drawdown of 14.1 ft and a specific yield of 31
%gpm/ft.\\
%
%
%\item A well is producing 0.00125 MGD. Its static water level was 35 ft and its current pumping water level is 115 ft. What is the specific capacity of this well? \\
%
%
%\item The specific capacity for a well is 10 gpm-ft. If the well produces 550 gallons per minute, what is the drawdown?
%
%\item The distance between the ground surface to the water level in a well when the pump is not operating is 98 ft.  Distance from the ground surface to the water in the well when the pump is operating is 116 feet. Calculate the drawdown in the well under these conditions.
%
%
%\end{enumerate}
%\end{tcolorbox}

\section{Density}\index{Density}
\begin{itemize}
\item Density is defined as the weight of a substance per a unit of its volume. For example, pounds per cubic foot or pounds per gallon.

\item Here are a few key facts about density:
\begin{itemize}

\item Density is measured in units of lb/ft3, lb/gal, or mg/L. Density of water = 62.4 lb/ft3 = 8.34 lb/gal.
\end{itemize}
\end{itemize}

\section{Specific Gravity}\index{Specific Gravity}
\begin{itemize}
\item Specific gravity is the ratio of the density of a substance (liquid or solid) to the density water.
\item It is the ratio of the weight of the substance of a certain volume to the weight of water of the same volume.

\item Any substance with a density greater than that of water will have a specific gravity greater than 1.0. Any substance with a density less than that of water will have a specific gravity less than 1.0. 

\item Specific gravity examples:
\begin{itemize}

\item Specific gravity of water = 1.0 
\item Specific gravity of concrete = 2.5 (depending on ingredients)
\item Specific gravity of alum (liquid @ 60°F) = 1.33 
\item Specific gravity of hydrogen peroxide (35\%) = 1.132
\end{itemize}

\item Specific gravity is used in two ways:
\begin{enumerate}
\item To calculate the total weight of a \% solution (either as a single gallon or a drum volume).\\
Total Weight = Drum Vol X SG X 8.34
\item To calculate the “active ingredient” weight of a single gallon or a drum.\\

Active Ingredient Weight within Drum = Drum Volume X SG X 8.34 X \% solution as a decimal. (i.e., Total Weight X \% solution as a decimal)\\

NOTE: Both ways start with solving for the total weight (Drum Vol X SG X 8.34). When solving for “active ingredient” weight, you have to then multiply by \% solution as a decimal.

\end{enumerate}
\end{itemize}

\textbf{Example:} What is the weight of 5 gallons of a 40\% ferric chloride solution given its specific gravity of 1.43?
$$(8.34 * 1.43) \enspace lbs/gal*5 \enspace gallons = \boxed{59.6 \enspace lbs}$$

The weight of active ferric chloride in the drum will be 59.6*0.4=23.84 lbs (as ferric chloride is 40\% strength)

%\begin{tcolorbox}[
%colframe=blue!25,
%colback=blue!10,
%coltitle=blue!20!black,  
%title= Practice Problems]
%\begin{enumerate}
%\item What is the specific gravity of a 1 ft$^3$ concrete block which weighs 145 lbs?
%
%\item What is the specific gravity of a chlorine solution if 1 (one) gallon weighs 10.2lbs?
%
%\item How much does each gallon of zinc orthophosphate weigh (pounds) if it has a specific gravity of 1.46?
%
%\item How much does a 55 gallon drum of 25\% caustic soda weigh (pounds) if the specific gravity is 1.28?
%
%\end{enumerate}
%
%\end{tcolorbox}


\section{Concentration}\index{Concentration}
\begin{itemize}
\item Concentration is typically expressed as mg/l which is the weight of the constituent (mg) in 1 liter of water.
\item As 1 liter of water weighs 1 million mg, a concentration of 1 mg/l implies 1 mg of constituent per 1 million mg of water or one part per million (ppm).   \texthl{Thus, mg/l and ppm are synonymous.}
\item Sometimes the constituent concentration is expressed in terms of percentage.\\
\vspace{6pt}
\textbf{Example:} 12.5\% chlorine concentration solution.\\
\vspace{0.2cm}
100\% would mean 1,000,000 mg/l or 1,000,000 ppm\\
\vspace{0.2cm}
$\implies$1\% would be $\dfrac{1,000,000}{100}\textrm{mg/l} = \textrm{10,000 mg/l or 10,000 ppm}$\\
\vspace{0.2cm}
$\implies$12.5\% chlorine concentration is 125,000 mg/l or 125,000 ppm.
\vspace{6pt}

$1\% \enspace concentration = 10,000 \enspace ppm \enspace or \enspace\dfrac{mg}{l}$\\
$0.1\% \enspace concentration = 1,000 \enspace ppm \enspace or \enspace \dfrac{mg}{l}$\\
$0.01\% \enspace concentration = 100 \enspace ppm \enspace or \enspace \dfrac{mg}{l}$\\
$10\% \enspace concentration = 100,000 \enspace ppm \enspace or \enspace \dfrac{mg}{l}$\\
$5\% \enspace concentration = 50,000 \enspace ppm \enspace or \enspace \dfrac{mg}{l}$\\
$12.5\% \enspace concentration = 125,000 \enspace ppm \enspace or \enspace \dfrac{mg}{l}$\\
\end{itemize}

\vspace{0.3cm}
Above concepts are used for chemicals such as fluoride and hypochlorites - the strength of the product as used is commonly expressed as a percentage.
\vspace{0.3cm}

\textbf{Example 1:} A chlorine solution was made to have a $4 \%$ concentration. It is often desirable to determine this concentration in $\mathrm{mg} / \mathrm{L}$. This is relatively simple: the $4 \%$ is four percent of a million.

To find the concentration in $\mathrm{mg} / \mathrm{L}$ when it is expressed in percent, do the following:

\begin{enumerate}
  \item Change the percent to a decimal.
\end{enumerate}
$$
4 \% \div 100=0.04
$$

\begin{enumerate}
  \setcounter{enumi}{2}
  \item Multiply times a million.
\end{enumerate}
$$
0.04 \times 1,000,000=40,000 \mathrm{mg} / \mathrm{L}
$$
We get the million because a liter of water weighs $1,000,000 \mathrm{mg} .1 \mathrm{mg}$ in 1 liter is 1 part in a million parts ( $\mathrm{ppm}) .1 \%=10,000 \mathrm{mg} / \mathrm{L}$.


\textbf{Example 2:} How much $65 \%$ calcium hypochlorite is required to obtain 7 pounds of pure chlorine?\\
$65 \%$ implies that in every lb of calcium hypochlorite has $65 \%$ lbs of available chlorine.\\
\vspace{0.2cm}
Therefore, $\dfrac{0.65 \textrm{ lbs available chlorine}}{\textrm{lb of calcium hypochlorite}} $ or conversely $\dfrac{\textrm{lb of calcium hypochlorite}}{0.65 \textrm{ lbs available chlorine}}$\\
\vspace{0.2cm}
$\implies{\textrm{lbs calcium hypchlorite required}}=\dfrac{\textrm{lb of calcium hypochlorite}}{0.65 \cancel{\textrm{ lbs available chlorine}}}*\dfrac{7\cancel{\textrm{ lb of available chlorine}}}{}$\\
\vspace{0.2cm}
$=\boxed{10.8 \textrm{ lbs of calcium hypochlorite with } 65\%\textrm{available chlorine is required}}$

%\begin{tcolorbox}[
%colframe=blue!25,
%colback=blue!10,
%coltitle=blue!20!black,  
%title= Practice Problems]
%\begin{enumerate}
%\item What is the concentration in mg/l of  4.5\% solution of that substance.
%\item How many lbs of salt is needed to make 5 gallons of a 2,500mg/l solution
%\end{enumerate}
%\end{tcolorbox}


\section{Pounds Formula}\index{Pounds Formula}
\begin{itemize}
\item Pounds formula: 
$$lbs \enspace \textbf{or} \enspace \dfrac{lbs}{day}=Concentration\Big(\dfrac{mg}{l}\Big)*8.34*volume(MG) \enspace \textbf{or} \enspace Flow (MGD)$$\\
\item So if the concentration of a particular constituent (in mg/liter) and the volume or flow of wastewater is given, one can calculate the amount of that constituent or using this formula.\\
\texthl{Important notes:}\\
\begin{enumerate}
\item \texthl{The unit of the constituent loading rate will be in lbs per the unit of time the flow is expressed in.  So if the flow is in MG per day the calculated loading rate will be in lbs/day.  Likewise if the flow value used is in MG per minute, the calculated loading rate will be in lbs/min.}
\item \texthl{If volume is used, the calculated value will be the mass of the constituent in that volume.  If flow is used, the calculated value will be the mass of the constituent in that flow.}
\item \texthl{For the Pound Formula to work, the volume or flow needs to be expressed in MG.  Volume or flows in other units - gallons, $ft^3$ etc. needs to be converted to MG.}
\end{enumerate}

\item The formula assumes that all of the material found in water (TSS, BOD, MLSS, Chlorine, etc.) weighs the same as water, that is, $8.34$ pounds per gallon.
\item In the Pounds Formula, there are three variables – lbs, concentration and volume, and one constant - 8.34.  Knowing any of the two variables in the formula, one can calculate the third (unknown) variable by rearranging the equation.\\
\begin{figure}[h]
\begin{tikzpicture}
    \newcommand{\R}{3}

\path[help lines,step=.2] (0,0) grid (16,6);
\path[help lines,line width=.6pt,step=1] (0,0) grid (16,6);
%\foreach \x in {0,1,2,3,4,5,6,7,8,9,10,11,12,13,14,15,16}
%\node[anchor=north] at (\x,0) {\x};
%\foreach \y in {0,1,2,3,4,5,6}
%\node[anchor=east] at (0,\y) {\y};
%-------------CIRCLE-----------------------------------
\draw[black,fill=gray!10] (8,3) circle (\R);
\draw[black, very thick, rotate=0](5,3) -- (11,3);
\draw (8,4.5) node[text width=3cm,align=center]
  {\scriptsize{lbs or lbs/day}};
\draw (6.4,2) node[text width=3cm,align=center]
  {\scriptsize{Concentration\\mg/l}};
\draw (9.7,2) node[text width=3cm,align=center]
  {\scriptsize{Volume(MG)\\Flow(MGD)}};
  \draw (8,1)node[text width=3cm,align=center]
  {\scriptsize{8.34}};
\draw[black, very thick, rotate=0](6.4,0.5) -- (8,3);
\draw[black, very thick, rotate=0](9.6,0.5) -- (8,3);
  \node [circle split,draw,double,fill=red!20] at (4,3)
  {
    % No \nodepart has been used, yet. So, the following is put in the
    % ``text'' node part by default.
    $\div$
    \nodepart{lower} % Ok, end ``text'' part, start ``output'' part
    $=$
  };
  
    \node [circle split,draw,double,fill=red!20] at (5.8,-0.2)
  {
    % No \nodepart has been used, yet. So, the following is put in the
    % ``text'' node part by default.
    \scriptsize{$X$}
    \nodepart{lower} % Ok, end ``text'' part, start ``output'' part
    \tiny{$Multiply$}
  };
  
    \node [circle split,draw,double,fill=red!20] at (10,-0.2)
  {
    % No \nodepart has been used, yet. So, the following is put in the
    % ``text'' node part by default.
    \scriptsize{$X$}
    \nodepart{lower} % Ok, end ``text'' part, start ``output'' part
    \tiny{$Multiply$}
  };
\end{tikzpicture}
\caption{Davidson Pie}
\end{figure}
\vspace{0.2cm}
\item Davidson Pie provides a pictorial reference for calculating any unknown variable.  If for example, if Concentration is unknown, it can be calculated as follows: \\$$Concentration\Big(\dfrac{mg}{l}\Big)=\dfrac{lbs \enspace \textbf{or} \enspace \dfrac{lbs}{day}}{8.34*Volume(MG) \enspace \textbf{or} \enspace Flow (MGD)}$$\\
\vspace{0.2cm}
\item Likewise, if Volume (or Flow) is the unknown variable. it can be calculated as:  \\$$Volume (MG) \enspace or \enspace Flow(MGD)=\dfrac{lbs \enspace \textbf{or} \enspace \dfrac{lbs}{day}}{Concentration\Big(\dfrac{mg}{l}\Big)* \enspace 8.34  }$$
\vspace{0.2cm}
\item Pounds formula is used for:
\begin{itemize}
\item Calculating the quantity in pounds of a particular wastewater constituent entering or leaving a wastewater treatment process
\item Calculating the pounds of chemicals to be added\\
\end{itemize}
\end{itemize}


\textbf{Example 1:} If a 5 MGD flow is to be dosed with 25 mg/l of a certain chemical, calculate the lbs/day that chemical required.\\

Solution\\

Applying lbs formula:\\
$\dfrac{lbs}{day}=5 MGD *250\dfrac{mg}{l}*8.34 = \boxed{1,042\dfrac{lbs}{day}}$
\\
\vspace{6pt}
\textbf{Example 2:} Calculate the lbs of chemical in 7,500 gallons of 4.5\% active solution of that chemical.\\
Solution\\
Applying lbs formula:\\
$lbs chemical = \dfrac{7500}{1,000,000}MG * 4.5*10,000 *8.34 = \boxed{2,815 \enspace lbs \enspace chemical}$\\
\textbf{Note:}\\  
1) 7500 gallons was converted to MG by dividing by 1,000,000\\
$7500 \enspace gallons * \dfrac{1 MG}{1,000,000 \enspace gallon}$\\
2) 4.5\% was converted to mg/l by multiplying by 10,000 as 1\%=10,000mg/l

%\begin{tcolorbox}[
%colframe=blue!25,
%colback=blue!10,
%coltitle=blue!20!black,  
%title= Practice Problems]
%
%\begin{enumerate}
%
%\item A water treatment plant operates at the rate of 75 gallons per minute. They dose soda ash at
%14 mg/L. How many pounds of soda ash will they use in a day?
%
%\item A water treatment plant is producing 1.5 million gallons per day of potable water, and
%uses 38 pounds of soda ash for pH adjustment. What is the dose of soda ash at that plant?
%
%\item A water treatment plant produces 150,000 gallons of water every day. It uses an
%average of 2 pounds of permanganate for iron and manganese removal. What is the dose of the
%permanganate? 
%
%\item A water treatment plant uses 8 pounds of chlorine daily and the dose is 17 mg/l. How
%many gallons are they producing?
%
%\item An operator mixes 40 lb of lime in a 100-gal tank containing 80 gal of water. What is the percent of lime in the slurry?
%
%\end{enumerate}
%\end{tcolorbox}

\section{Chemicals Related Math Problems}\index{Chemicals Related Math Problems}
\subsection{Chemical Dosing}\index{Chemical Dosing}

\begin{itemize}
\item Use lbs formula to calculate the lbs of chemicals required\\
\item Using the calculated lbs chemical required value, calculate the amount of that chemical at the concentration available
\end{itemize}

\textbf{Example 1:} If a 5 MGD flow is to be dosed with 25 mg/l of a certain chemical, calculate the lbs/day that chemical required.\\

Solution\\

Applying lbs formula:\\
$\dfrac{lbs}{day}=5 MGD *250\dfrac{mg}{l}*8.34 = \boxed{1,042\dfrac{lbs}{day}}$
\\
\vspace{6pt}
\textbf{Example 2:} Calculate the lbs of chemical in 7,500 gallons of 4.5\% active solution of that chemical.\\
Solution\\
Applying lbs formula:\\
$lbs chemical = \dfrac{7500}{1,000,000}MG * 4.5*10,000 *8.34 = \boxed{2,815 \enspace lbs \enspace chemical}$\\

\subsection{Chlorine dosing problems}\index{Chlorine dosing problems}
\textbf{Example 4:} Determine the chlorinator setting (lb/day) required to treat a flow of $4 \mathrm{MGD}$ with a chlorine dose of $5 \mathrm{mg} / \mathrm{L}$.

Chlorine feed rate $(\mathrm{lb} /$ day $)=$ Chlorine $(\mathrm{mg} / \mathrm{L}) \times$ Flow $(\mathrm{MGD}) \times 8.34 \mathrm{lb} / \mathrm{gal}$

Chlorine feed rate $(\mathrm{lb} /$ day $)=5 \mathrm{mg} / \mathrm{L} \times 4 \mathrm{MGD} \times 8.34 \mathrm{lb} / \mathrm{gal}$

Chlorine feed rate $(\mathrm{lb} /$ day $)=167 \mathrm{lb} /$ day

\textbf{Example 5:} A pipeline that is 12 inches in diameter and $1400 \mathrm{ft}$ long is to be treated with a chlorine dose of $48 \mathrm{mg} / \mathrm{L}$. How many lb of chlorine will this require?

First determine the gallon volume of the pipeline:

Volume $(\mathrm{gal})=0.785 \times \mathrm{D}^{2} \times$ length $(\mathrm{ft}) \times 7.48 \mathrm{gal} / \mathrm{cu} \mathrm{ft}$

Volume $(\mathrm{gal})=0.785 \times(1 \mathrm{ft})^{2} \times 1400 \mathrm{ft} \times 7.48 \mathrm{gal} / \mathrm{cu} \mathrm{ft}$ Volume $(\mathrm{gal})=8221 \mathrm{gal}$

Next calculate the amount of chlorine required:

Chlorine feed rate $(\mathrm{lb} /$ day $)=$ Chlorine $(\mathrm{mg} / \mathrm{L})$ x Flow $($ MGD) $\times 8.34 \mathrm{lb} / \mathrm{gal}$

Chlorine feed rate $(\mathrm{lb} /$ day $)=48 \mathrm{mg} / \mathrm{L} \times 0.008221 \mathrm{MGD} \times 8.34 \mathrm{lb} / \mathrm{gal}$

Chlorine feed rate $(\mathrm{lb} /$ day $)=3.3 \mathrm{lb}$

\textbf{Example 6:} A water sample is tested and found to have a chlorine demand of $1.7 \mathrm{mg} / \mathrm{L}$. If the desired chlorine residual is $0.9 \mathrm{mg} / \mathrm{L}$, what is the desired chlorine dose (in $\mathrm{mg} / \mathrm{L}$ )?

Chlorine Dose $(\mathrm{mg} / \mathrm{L})=$ Chlorine Demand $+$ Chlorine Residual

Chlorine Dose $(\mathrm{mg} / \mathrm{L})=1.7 \mathrm{mg} / \mathrm{L}+0.9 \mathrm{mg} / \mathrm{L}$

Chlorine $\operatorname{Dose}(\mathrm{mg} / \mathrm{L})=2.6 \mathrm{mg} / \mathrm{L}$

\textbf{Example 7:}\\
The chlorine dosage for water is $2.7 \mathrm{mg} / \mathrm{L}$. If the chlorine residual after a 30-minute contact time is found to be $0.7 \mathrm{mg} / \mathrm{L}$, what is the chlorine demand (in $\mathrm{mg} / \mathrm{L}$ )?

Chlorine Demand $=$ Chlorine Dose $-$ Chlorine Residual

Chlorine Demand $=2.7 \mathrm{mg} / \mathrm{L}-0.7 \mathrm{mg} / \mathrm{L}$

Chlorine Demand $=2.0 \mathrm{mg} / \mathrm{L}$

\textbf{Example 8:} How many gallons per day of bleach solution (SG 1.2)containing 12.5\% available chlorine is required to disinfect a 10 MGD flow of water given the required chlorine dosage of 7 mg/l.\\
\begin{enumerate}
\item Calculate the lbs of chlorine required using the lbs formula:\\
\vspace{0.5cm}
=$10 MGD \enspace * \enspace 7 \dfrac{mg}{l} \enspace * \enspace 8.34\enspace=\enspace 583.8 \enspace lbs \enspace chlorine \enspace per \enspace day$\\
\vspace{0.5cm}
\item Calculate the gallons of bleach which will provide the 583.8 lbs chlorine\\
\vspace{0.5cm}
Applying the lbs formula - note that 8.34 * SG will give the actual lbs/gal of bleach.  If SG is not provided, use only 8.34 lbs per gallon:\\
\vspace{0.5cm}
$583.8 \dfrac{lbs \enspace bleach}{day}\enspace=\enspace x \dfrac{gal}{day} \enspace * \enspace 8.34 * 1.2 \dfrac{lbs \enspace bleach}{gal} \enspace * \enspace 0.0125 \dfrac{lbs \enspace chlorine}{lb \enspace bleach} \enspace $\\
\vspace{0.5cm}
$ \implies x \dfrac{gal}{day}\enspace = \enspace \dfrac{583.8}{8.34*1.2*0.125} \enspace = \boxed{467 \dfrac{gal}{day}}$
\end{enumerate}
\vspace{0.3cm}
\textbf{The above problem can be solved directly using the formula below given in the SWRCB Water Treatment Exam Formula Sheet.}\\
\vspace{0.3cm}
 $\textrm{GPD}=\dfrac{\textrm{(MGD)}*\textrm{(ppm or mg/l)}*8.34 \enspace \textrm{lbs/gal}}{\textrm{\% \enspace purity}*\textrm{Chemical \enspace Wt. (lbs/gal)}}$ 
 \vspace{0.3cm}
 $\textrm{GPD}=\dfrac{10*7*8.34}{0.125*(1.2*8.34)}=\boxed{467 \dfrac{\textrm{gal}}{\textrm{day}}}$ 

%\begin{tcolorbox}[
%colframe=blue!25,
%colback=blue!10,
%coltitle=blue!20!black,  
%title= Practice Problems]
%\begin{enumerate}
%
%  \item Determine the chlorinator setting in pounds per day if a water plant produces $300 \mathrm{gpm}$ and the desired chlorine dose is $2.0 \mathrm{mg} / \mathrm{L}$.
%
%  \item The finished water chlorine demand is $1.2 \mathrm{mg} / \mathrm{L}$ and the target residual is $2.0 \mathrm{mg} / \mathrm{L}$. If the plant flow is $5.6 \mathrm{mgd}$, how many pounds per day of $65 \%$ hypochlorite solution will be required?
%
%  \item Fluoride is added to finished water at a dose of $4 \mathrm{mg} / \mathrm{L}$. Find the feed rate setting for a fluoride saturator in gal/min if the water plant produces $5 \mathrm{mgd}$.
%
%  \item If chlorine costs $\$ 0.21$ per pound, what is the daily cost to chlorinate a $5 \mathrm{mgd}$ flow rate at a dosage of $2.6 \mathrm{mg} / \mathrm{L}$ ?
%
%  \item One gallon of sodium hypochlorite laundry bleach, with $5.25 \%$ available chlorine, contains how many pounds of active chlorine?
%
%
%\end{enumerate}
%\end{tcolorbox}


\subsection{Blending and Dilution Calculations}\index{Blending and Dilution Calculations}
\begin{itemize}
\item Blending and dilution calculations apply to the following scenarios:
\begin{itemize}
\item Blending involves mixing two streams - each with a different concentration of contaminant/chemical, to obtain a certain volume or flow containing the target concentration of contaminant/chemical.  For example: \textit{Finding the correct blend of two source water streams - one with 15 mg/L of iron and other containing  4 mg/L of iron to get a 100 gpm product water containing 8 mg/l of iron.} \textbf{OR}\\
\textit{Calculating the actual combined TDS concentration obtained by mixing two known flows with known TDS concentrations.}
\item Dilution involves makedown of a higher concentration of a chemical to a lower concentration using water as a dilutant.   For example: \textit{How much initial volume of a 4\% polymer solution is needed to make 3500 gallons of polymer at 0.25\% concentration?}\\
\end{itemize}
\item These type of problems are solved using C*V relationship where:
\begin{itemize}
 \item C is the concentration expressed in ppm or mg/l or as \% purity.
 \item V is either the volume or flow.
\item The product - C*V - $\dfrac{\textrm{\textrm{mass}}}{\textrm{volume}/\textrm{flow}}*\textrm{volume/flow} = \textrm{mass}$  
\end{itemize}
\item For blended streams, the sum of the mass from each of the two source streams will equal to the mass in the target stream:

\item Thus, \textbf{for blending calculations}, if:\\

C$_1$ and V$_1$ is the concentration and volume respectively of the one of the sources streams and\\
\vspace{0.2cm}
 C$_2$ and V$_2$ is the concentration and volume respectively of the second source stream, and \\
 \vspace{0.2cm}
C$_3$ and and V$_3$ is the concentration and volume respectively of the target stream\\
\vspace{0.3cm}
The sum of the mass from each of the two source streams will equal to the mass in the target stream:\\
\vspace{0.3cm}
\textbf{C$_1$ * V$_1$ + C$_2$ * V$_2$ =  C$_3$ * V$_3$.}\\
\vspace{0.3cm}
This equation can be manipulated algebraically to calculate anyone of the unknown values in the equation.\\
\vspace{0.2cm}
Also, any of the three volume variables can be expressed as the sum or difference of the other two - , or V$_1$ + V$_2$ = V$_3$ or V$_1$ = V$_3$ - V$_2$ or V$_2$ = V$_T$ - V$_1$\\

\item \textbf{For dilution}, the mass of the target chemical will remain the same, as only water is added to the source (concentrated chemical).
\item Thus, for dilution calculations, if:\\
\vspace{0.2cm}
C$_1$ and V$_1$ is the concentration and volume respectively of the concentrated product used for the dilution, and\\
\vspace{0.2cm}
C$_2$ and V$_2$ is the concentration and volume of the resultant product after dilution with water\\
\vspace{0.2cm}
The mass of the target chemical in the volume of the concentrated product used for dilution will remain the same in the final diluted product:\\
\vspace{0.3cm}
\textbf{C$_1$ * V$_1$ =  C$_2$ * V$_2$.}\\

\end{itemize}

\textbf{Example Problem \#1:} Two wells are used to satisfy demand during the summer months. One well produces water that contains 22 mg/L of Arsenic. The other well produces water that contains 3 mg/L of Arsenic. If the total demand for water is 400 gpm and the target Arsenic concentration in the finished water is 8 mg/L, what is the highest pumping rate possible for the first well?\\
\vspace{0.3cm}
\textbf{Solution:}\\
C$_1$ * V$_1$ + C$_2$ * V$_2$ =  C$_3$ * V$_3$\\
\vspace{0.3cm}
Thus 22 * V$_{22}$ + 3 * V$_3$ =  8 * V$_8$\\
\vspace{0.3cm}

V$_{22}$ + V$_3$ = V$_8$ = 400 gpm\\
\vspace{0.3cm}
As we want to solve for V$_{22}$, we can express V$_3$ as: V$_3$ = 400-V$_{22}$\\
\vspace{0.3cm}
Thus, 22 * V$_{22}$ + 3 * (400-V$_{22}$) =  8 * 400=3,200\\
\vspace{0.3cm}
22V$_{22}$ + 1200-3V$_{22}$ =  3,200\\
\vspace{0.3cm}
V$_{22}$(22-3) =  2,000\\
\vspace{0.3cm}
V$_{22}$ = $ \dfrac{2,000}{19}=\boxed{105.3 \enspace gpm}$\\
\vspace{0.3cm}
Also, V$_3$=400-105.3=294.7\\
\vspace{0.3cm}

NOTE:  If one does not want to utilize algebraic manipulation, one may memorize the following formula:\\
\vspace{0.3cm}
$V_{1/2}=\dfrac{\lvert C_3 - C_{2/1}\rvert*V_3}{C_1-C_2}$\\
\vspace{0.3cm}
Applying the formula above to Example Problem \#2:\\
\vspace{0.3cm}
$V_{22}=\dfrac{\lvert 8 - 3\rvert*400}{22-3}=\boxed{105.3 \enspace gpm}$\\
\vspace{0.3cm}
$V_{3}=\dfrac{\lvert 8 - 22\rvert*400}{22-3}=\boxed{294.7 \enspace gpm}$\\
\vspace{0.3cm}
\textbf{Example Problem \#2:}  How many gallons of a 4\% polymer solution is required to make a 3,500 gallon batch of 0.25\% polymer solution.\\

\textbf{Solution:}\\
\vspace{0.3cm}
Here, we are adding water - which has zero percent of polymer concentration to the 4\% polymer to make a 0.25\% polymer solution.\\
\vspace{0.3cm}
C$_1$ * V$_1$ = C$_2$ * V$_2$\\
\vspace{0.3cm}
C$_{4\%}$ * V$_{4\%}$ =  C$_{0.25\%}$ * V$_{0.25\%}$\\
\vspace{0.3cm}
4 * V$_{4\%}$ =  0.25 * 3,500\\
\vspace{0.3cm}
$\implies V_{4\%} = \dfrac{0.25 \enspace * \enspace 3500}{4}= \boxed{219 \enspace\textrm{gal}} $\\
\vspace{0.3cm}
Take 219 gallons of the 4\% polymer and dilute to 3,500 gallons to give a 0.25\% polymer solution.\\

%\begin{tcolorbox}[
%colframe=blue!25,
%colback=blue!10,
%coltitle=blue!20!black,  
%title= Practice Problems]
%\begin{enumerate}
%\item Ferric chloride is being added as a coagulant to the raw water entering a plant. Sampling
%shows that the concentration of ferric in the raw water is 25 ppm. A quick check of the chemical
%metering pump shows that it is operating at a flow rate of 4.3 gpm. If the flow through the water
%plant is 800 gpm, what is the concentration of raw chemical in the dosing tank?
%
%\item A water plant is fed by two different wells. The first well produces water at a rate of 600
%gpm and contains arsenic at 0.5 mg/L. The second well produces water at a rate of 350 gpm and
%contains arsenic at 12.5 mg/L. What is the arsenic concentration of the blended water?
%\end{enumerate}
%\end{tcolorbox}

\section{Force, Pressure and Head} \index{Force, Pressure and Head}

\textbf{Force:}  In the English system force and weight are often used in the same way. The weight of the cubic foot of water is $62.4$ pounds. The force exerted on the bottom of the one foot cube is $62.4$ pounds. If we have two cubes stacked on top of one another, the force on the bottom will be $124.8$ pounds.

\textbf{Pressure:} Pressure is a force per unit of area, pounds per square inch or pounds per square foot are common expressions of pressure. 

\textbf{Head:}  Pressure is directly related to the height of a column of fluid. This height is called head or feet of head. Pressure and feet of head head are directly related - \emph{for every one foot of head there is a pressure of $0.433$ psi.}

\vspace{0.2cm}
Thus, $\dfrac{0.433 \enspace psi}{ft \enspace (water \enspace column)}$ or conversely $\dfrac{1 \enspace ft \enspace (water \enspace column)}{2.31 \enspace psi}$\\
\texthl{Note:  This pressure/head will include the height the water pumped and also the head associated with friction losses - energy loss because of the water moving through the pipe and fittings.}\\

\begin{figure}[h]
\begin{tikzpicture}
\path[help lines,step=.2] (0,0) grid (16,6);
\path[help lines,line width=.6pt,step=1] (0,0) grid (16,6);
%\foreach \x in {0,1,2,3,4,5,6,7,8,9,10,11,12,13,14,15,16}
%\node[anchor=north] at (\x,0) {\x};
%\foreach \y in {0,1,2,3,4,5,6}
%\node[anchor=east] at (0,\y) {\y};
\pgfmathsetmacro{\cubex}{3}
\pgfmathsetmacro{\cubey}{3}
\pgfmathsetmacro{\cubez}{3}
\draw(13.5,5,3) -- ++(-\cubex,0,0) -- ++(0,-\cubey,0) -- ++(\cubex,0,0) -- cycle;
\draw(13.5,5,3) -- ++(0,0,-\cubez) -- ++(0,-\cubey,0) -- ++(0,0,\cubez) -- cycle;
\draw(13.5,5,3) -- ++(-\cubex,0,0) -- ++(0,0,-\cubez) -- ++(\cubex,0,0) -- cycle;
\draw (8.7,2.5) node[text width=3cm,align=center]
  {\scriptsize{12"}};
\draw (14.1,3.4) node[text width=3cm,align=center]
  {\scriptsize{12"}};
  \draw (10.9,0.5)node[text width=3cm,align=center]
  {\scriptsize{12"}};
    \draw (2.8,4.8)node[text width=3.8cm,align=left]
  {\small{$Pressure=\dfrac{Force}{Area}$}};
      \draw (2.8,3.1)node[text width=3.8cm,align=left]
  {\small{Pressure exerted by}};
        \draw (2.8,2.8)node[text width=3.8cm,align=left]
  {\small{a 1ft column of water}};
        \draw (5.3,2.9)node[text width=3cm,align=center]
  {\small{$=\dfrac{62.4 \enspace lb}{12 in \enspace x \enspace 12 in}$}};
          \draw (7.3,2.9)node[text width=3cm,align=center]
  {\small{$=0.43 \enspace psi$}};
         \draw (3.45,1.2)node[text width=5cm,align=left]
  {\small{As 1$ft^3$ of water weighs 62.4 lbs}};
\end{tikzpicture}
\end{figure}


The pressure at the bottom of a container is affected only by the height of water in the container and not by the shape or the volume of the container. In the drawing below there are four containers all of different shapes and sizes. The pressure at the bottom of each is the same.

\begin{center}
\includegraphics[scale=0.2]{2022_11_03_65aa625ded296bdfd01fg-17}
\end{center}
The pressure exerted at the bottom of a tank is relative only to the head on the tank and not the volume of water in the tank. For example, below are two tanks each containing 5000 gallons. The pressure at the bottom of each is 22 psi. If half of the water were drained from the tanks the pressure at the bottom of the elevated tank would be $17.3$ psi while the pressure at the bottom of the standpipe would be 11 psi.\\

\begin{center}
\includegraphics[scale=0.25]{2022_11_03_65aa625ded296bdfd01fg-18}
\end{center}


\begin{itemize}
\item A reservoir is 40 feet tall. Find the pressure at the bottom of the reservoir.

$40 \mathrm{ft} \times 0.433 \mathrm{psi} / \mathrm{ft}=17.3 \mathrm{psi}$

\vspace{0.4cm}

\item Find the height of water in a tank if the pressure at the bottom of the tank is 12 psi.

$12 \mathrm{psi} \div 0.433 \mathrm{psi} / \mathrm{ft}=27.7 \mathrm{ft}$

\vspace{0.4cm}

\item If a pump discharge pressure gauge read 10 psi, the height of the water corresponding to this pressure would be:
$$10 \enspace psi \times \dfrac{2.31 \enspace ft}{psi}=23.1 \enspace ft$$\\
\vspace{0.4cm}
\end{itemize}

%\begin{tcolorbox}[breakable, enhanced,
%colframe=blue!25,
%colback=blue!10,
%coltitle=blue!20!black,  
%title= Practice Problems]
%
%\begin{enumerate}
%\item Convert 45 psi to feet of head
%
%\item If the pressure at a water main is 50 psi, what would the static pressure (psi) be at a faucet on the top floor of a four story building? (Assuming 10 ft. per story)
%
%\item A water tower has water pressure of 98 psi at its base. What would be. the pressure at a hydrant three blocks away if there is a 65-foot head loss in the pipe?\\
%
%\end{enumerate}
%
%\end{tcolorbox}






\section{Pumping Calculations}\index{Pumping Calculations}
\begin{itemize}
\item Pump is a machine used for moving water (and other fluids) through a piping system and raise the pressure of the water.
\item Pumping is accomplished by transforming the input energy - typically from an electric motor or from other sources such as high-pressure air.
\item The pump calculations in this section are for electrically driven rotodynamic pumps.
\item To move water, a pump will need to overcome resistance due its density, gravitational force and friction.
\item This resistance is dependent on:
\begin{itemize}
\item Height the water needs to be raised.  This height of the fluid in a container is referred to as head. 
\item Quantity of water involved
\end{itemize}
\end{itemize}

\subsection{Glossary of Pump Calculations Terms}\index{Glossary of Pump Calculations Terms}

\textbf{Static Pressure: } Static implies a non-moving condition.  The pressure measured when there is no water moving in a line or the pump is not running is called static $^{32}$ pressure. This is the pressure represented by the gauges on the tanks in the discussion above.

\textbf{Dynamic Pressure: } When water is allowed to run through a pipe and the pressure (called pressure head) measured at various points along the way we find that the pressure decreases the further we are from the sources.
\begin{center}
\includegraphics[scale={0.2}]{2022_11_03_65aa625ded296bdfd01fg-18(1)}
\end{center}
\textbf{Headloss: }  The reason for this reduction in pressure is a phenomenon called headloss. Headloss is the loss of energy (pressure) due to friction. The energy is lost as heat.

If the headloss in a certain pipe is 25 feet, it means the amount of energy required to overcome the friction in the pipe is equivalent to the amount of energy that would be required to lift this amount of water straight in the air 25 feet.

In a pipe, the factors that contribute to headloss include the following:

\begin{itemize}
  \item Roughness of pipe - If the roughness of a pipe were doubled the headloss would double.

  \item Length of pipe - If the length of the pipe were doubled the headloss would double.

  \item Diameter of pipe - If the diameter of a pipe were doubled the headloss would be cut in half

  \item Velocity of water - If the velocity of the water in a pipe were doubled the headloss would be increased by about four times. It should be apparent that velocity, more than any other single factor, affects headloss. To double the velocity we would have to double the flow in the line.
  
  \item Pumping System Components and Fittings - Each type of fitting has a specific headloss depending upon the velocity of water through the fitting. For instance the headloss though a check valve is two and one quarter times greater than through a ninety degree elbow and ten times greater than the headloss through an open gate valve.

\end{itemize}

\textbf{Static Head: }  Static head is the distance between the suction and discharge water levels when the pump is shut off. 

\textbf{Suction Lift: } Suction lift is the distance between the suction water level and the center of the pump impeller. This term is only used when the pump is in a suction lift condition. A pump is said to be in a suction lift condition any time the eye (center) of the impeller is above the water being pumped.

\textbf{Velocity Head: } The amount of energy required to bring a fluid from standstill to its velocity. For a given quantity of flow, the velocity head will vary indirectly with the pipe diameter.

\textbf{Total Dynamic Head (TDH):}  The total energy needed to move water from the center line of a pump (eye of the first impeller of a lineshaft turbine) to some given elevation or to develop some given pressure. This includes the static head, velocity head and the headloss due to friction. 

\textbf{Horsepower: } Horsepower is a measurement of the amount of energy required to do work. Motors are rated in horsepower. The horsepower of an electric motor is called brake horsepower. The horsepower requirements of a pump are dependent on the flow and the total dynamic head.  33,000 foot pounds per minute of work is 1 horsepower.

\textbf{Suction Head: } Suction head is the distance between the suction water level and the center of the pump impeller when the pump is in a suction head condition. A pump is said to be in a suction head condition any time the eye (center) of the impeller is below the water level being pumped.

\textbf{Velocity Head: } Velocity head is the amount of energy required by the pump and motor to overcome inertia and bring the water up to speed. Velocity head is often shown mathematically as $\mathrm{V}^{2} / 2 \mathrm{~g}$. ( $\mathrm{g}$ is the acceleration due to gravity $-32.2 \mathrm{ft} / \mathrm{sec}^{2}$ ).
\begin{center}
\includegraphics[scale=0.25]{2022_11_03_65aa625ded296bdfd01fg-20}
\end{center}
\textbf{Total Dynamic Head: }  Total dynamic head (TDH) is a theoretical distance. It is the static head, velocity head and headloss required to get the water from one point to another.

The horsepower output of an electric motor is directly reflected to the amperage that the motor draws. Any increase in horsepower requirements will give a corresponding increase in amperage.

\textbf{Cavitation: }  Cavitation in pumps is the rapid creation and subsequent collapse of air bubbles occuring as a result of the inlet pressure falling below the design inlet pressure or when the pump is operating at a flow rate higher than the design flow rate. This collapse of the air bubbles typically manifests as a pinging or crackling noise.  Cavitation is undesirable because it can damage the impeller, cause noise and vibration, and decrease pump efficiency.

\begin{figure}[h]
\begin{center}
\includegraphics[scale=0.6]{CalculatingStaticHead}\\
\includegraphics[scale=0.6]{PumpHead}\\
\end{center}
\end{figure}

\newpage
\subsection{Pumping Rate Calculations}\index{Pumping Rate Calculations}
\begin{itemize}
\item \texthl{For calculating volume pumped given the pump flow rate:} Multiply the pump flow rate by the time interval\\
\textbf{Make sure:}
\begin{itemize}
\item The time units - in the given time interval and in the pump flow rate match
\end{itemize}
\item \texthl{For calculating time to pump a certain volume:}
\begin{enumerate}[Step 1.]
\item Calculate the total volume pumped
\item Divide the total volume by the pump flow rate
\end{enumerate}
\textbf{Make sure:}
\begin{itemize}
\item The volume units - in the volume that needs to be pumped and in the pump flow rate match
\item The time unit in the pump flow rate needs to be converted to the time unit that you need the answer in
\end{itemize}
\end{itemize}
% \end{enumerate}

\textbf{Example 1:}  A pump is set to pump 5 minutes each hour. It pumps at the rate of 35 gpm. How many gallons of water are pumped each day?\\
Solution:\\
$\dfrac{35 \enspace gal \enspace sludge}{\cancel{min}}*\dfrac{5 \enspace \cancel{min}}{\cancel{hr}} *\dfrac{24 \enspace \cancel{hr}}{day}=\boxed{\dfrac{4,200 \enspace gallons}{day}}$\\
\vspace{0.5cm}

\textbf{Example 2:}  A pump operates 5 minutes each 15 minute interval.  If the pump capacity is 60 gpm, how many gallons are pumped daily?

$\dfrac{60 \enspace gal \enspace sludge}{\xcancel{min}}*\dfrac{5 \enspace \xcancel{min}}{15 \enspace \cancel{min}}*1440\dfrac{\cancel{min}}{day}=\boxed{\dfrac {28,800 \enspace gal \enspace sludge }{day}}$\\
\vspace{0.5cm}

\textbf{Example 3:}  Given the tank is 10ft wide, 12 ft long and 18 ft deep tank including 2 ft of freeboard when filled to capacity. How much time (minutes) will be required to pump down this tank to a depth of 2 ft when the tank is at maximum capacity using a 600 GPM pump\\
Solution:\\
\vspace{0.5cm}


\begin{tikzpicture}

\pgfmathsetmacro{\cubexx}{4}
\pgfmathsetmacro{\cubeyy}{1.5}
\pgfmathsetmacro{\cubezz}{2}
\pgfmathsetmacro{\cubex}{4}
\pgfmathsetmacro{\cubey}{0.5}
\pgfmathsetmacro{\cubez}{2}
\pgfmathsetmacro{\cubexxx}{4}
\pgfmathsetmacro{\cubeyyy}{4}
\filldraw [fill=cyan!10!white, draw=black] (0,-\cubey,0) -- ++(-\cubexx,0,0) -- ++(0,-\cubeyy,0) -- ++(\cubexx,0,0) -- cycle ;
\filldraw [fill=cyan!0!white, draw=black] (0,-\cubey,0) -- ++(0,0,-\cubezz) -- ++(0,-\cubeyy,0) -- ++(0,0,\cubezz) -- cycle;
\filldraw [fill=cyan!10!white, draw=black] (0,-\cubey,0) -- ++(0,0,-\cubezz) -- ++(0,-\cubeyy,0) -- ++(0,0,\cubezz) -- cycle;
%\filldraw [fill=cyan!10!white, draw=black] (0,-\cubey,0) -- ++(-\cubexx,0,0) -- ++(0,0,-\cubezz) -- ++(\cubexx,0,0) -- cycle;
%%%\draw (0,-0.5,0) -- ++(-\cubex,0,0) -- ++(0,-\cubey,-\cubez) -- ++(\cubex,0,0) -- cycle;
\draw (-\cubex,0,0) -- ++(0,0,-\cubez) -- ++(0,-\cubey,0) -- ++(0,0,\cubez) -- cycle;
\draw (0,-\cubey,0) -- ++(-\cubex,0,0) -- ++(0,0,-\cubez) -- ++(\cubex,0,0) -- cycle;
\filldraw [fill=white, draw=black] (0,0,0) -- ++(-\cubex,0,0) -- ++(0,-\cubey,0) -- ++(\cubex,0,0) -- cycle ;
\filldraw [fill=white, draw=black] (0,0,0) -- ++(0,0,-\cubez) -- ++(0,-\cubey,0) -- ++(0,0,\cubez) -- cycle;
\filldraw [fill=white, draw=black] (0,0,0) -- ++(0,0,-\cubez) -- ++(0,-\cubey,0) -- ++(0,0,\cubez) -- cycle;
\filldraw [fill=white, draw=black] (0,0,0) -- ++(-\cubex,0,0) -- ++(0,0,-\cubez) -- ++(\cubex,0,0) -- cycle;

%\filldraw [fill=RoyalBlue!10!white, draw=black] (0,-1.5,0) -- ++(-\cubex,0,0) -- ++(0,-\cubey,0) -- ++(\cubex,0,0) -- cycle ;

%\filldraw [fill=RoyalBlue!10!white, draw=black] (0,-1.5,0) -- ++(0,0,-\cubez) -- ++(0,-\cubey,0) -- ++(0,0,\cubez) -- cycle;



%%\draw (0,-0.5,0) -- ++(-\cubex,0,0) -- ++(0,0,-\cubez) -- ++(\cubex,0,0) -- cycle;
%%\filldraw [fill=white, draw=black] (-\cubex,0,0) -- ++(0,0,-\cubez) -- ++(0,-\cubey,0) -- ++(0,0,\cubez) -- cycle;
%%\filldraw [fill=white, draw=black] (0,-\cubey,0) -- ++(-\cubex,0,0) -- ++(0,0,-\cubez) -- ++(\cubex,0,0) -- cycle ;

\draw [<->] (-4,-2.3) -- (0,-2.3) node [midway, below] {12' Long};
\draw [<->] (1,-1.3) -- (1,.2) node [midway, midway] {\hspace{4.5cm}16' Water Depth (Initial)};
\draw [<->] (0.4,-1.62) -- (0.4,-1.1) node [midway, midway] {\hspace{-4.8cm} 2' Water Depth (Final)};
\draw [<->] (1,.8) -- (1,.2) node [midway, midway] {\hspace{2.4cm}2' Freeboard};
\draw [<->] (1,-1.3) -- (0,-2.3) node [midway, midway] {\hspace{2.3cm}10' Wide};
\end{tikzpicture}\\
Volume to be pumped=$12 \enspace ft*10 \enspace ft *(16-2)\enspace ft=1,680ft^3$\\
\vspace{0.3cm}
$\implies \dfrac{1,680\cancel{ft^3}*7.48\dfrac{\cancel{gal}}{\cancel{ft^3}}}{600\dfrac{\cancel{gal}}{min}}=\boxed{21min}$

%\begin{tcolorbox}[breakable, enhanced,
%colframe=blue!25,
%colback=blue!10,
%coltitle=blue!20!black,  
%title= Practice Problems]
%\begin{enumerate}
%\item Convert 45 psi to feet of head
%
%\item How long (in minutes) will it take to pump down 25 feet of water in a 110 ft diameter cylindrical tank when using a 1420 gpm pump\\
%
%\item How long will it take (hrs) to fill a 2 ac-ft pond if the pumping rate is 400 GPM?
%
%\item A tank is filling at the rate of 300 gpm for a 20 minute period. How many of water will be contained in the tank at the end of 16 minutes?
%\end{enumerate}
%\end{tcolorbox}






\subsection{Power Requirements for Pumping}\index{Power Requirements for Pumping}
\begin{center}
\includegraphics[scale=0.12]{PumpProblem}\\
\end{center}
Where:\\
\begin{itemize}
\item \textbf{Input Hp} is the input power to the motor which produces the \textbf{Output Hp or Brake Hp} - the mechanical power which runs the pump.  
\item The ratio of Output Hp and Input Hp is the motor efficiency - $\eta_m$.
\item The Output Hp is the input power (Brake Hp) to the pump to pump the water.
\item Water Hp is the rate of energy transferred to the water being pumped and can be calculated by the formula:\\
$$\dfrac{\mathrm{H \enspace - \enspace Head \enspace of \enspace water \enspace (ft) \enspace * \enspace Q \enspace - \enspace Flow \enspace (GPM)}}{3,960 \enspace \mathrm{(Conversion \enspace factor \enspace for \enspace converting \enspace GPM-ft \enspace to \enspace Hp)}}$$
\item The ratio of Output Hp and Water Hp is the pump efficiency - $\eta_p$.
\end{itemize}
\subsection{Example Problems}
\begin{enumerate}


\item 1 MGD is pumped against a 14’ head.  What is the water Hp?  The pump mechanical efficiency is 85\%.  What is the brake horsepower?\\
\vspace{0.4cm}
water Hp = flow * head\\
\vspace{0.4cm}
$\dfrac{1,000,000 \enspace gal}{day}*\dfrac{day}{1440 \enspace min}*14 \enspace ft*\dfrac{Hp}{3,960 \enspace GPM-ft}=\boxed{Water \enspace Hp = 2.46 \enspace Hp}$\\
\vspace{0.4cm}
pump Hp = brake Hp * pump efficiency\\
\vspace{0.4cm}
$Brake \enspace Hp = \dfrac{2.46}{0.85}=\boxed{Brake \enspace Hp=2.89Hp}$\\
\vspace{0.4cm}

\item A flow of 200 gpm  is pumped against a total head of 4.0 feet. The pump is 78\% efficient and the motor' is 90\% efficient. Calculate the input Hp.\\
\vspace{0.4cm}
water Hp = flow * head\\
\vspace{0.2cm}
$200GPM*4ft*\dfrac{Hp}{3,960 GPM-ft}=0.2Hp$\\
\vspace{0.4cm}\includegraphics[scale=0.08]{PumpProblem}\\
water Hp=brake Hp*pump efficiency, and\\
brake Hp=input Hp*motor efficiency\\
Therefore, water Hp=input Hp*motor efficiency*pump efficiency\\
\vspace{0.4cm}
input Hp=$\dfrac{water \enspace Hp}{motor \enspace efficiency*pump \enspace efficiency}=\dfrac{0.2}{0.9*0.78}=\boxed{0.28Hp}$
\vspace{0.2cm}
\end{enumerate}

%\begin{tcolorbox}[breakable, enhanced,
%colframe=blue!25,
%colback=blue!10,
%coltitle=blue!20!black,  
%title= Practice Problems]
%\begin{enumerate}
%
%  \item If a pump is operating at 2,200 gpm and 60 feet of head, what is the water
%horsepower? If the pump efficiency is 71\%, what is the brake horsepower?
%
%\item The water horsepower of a pump is $10 \mathrm{Hp}$ and the brake horsepower output of the motor is $15.4 \mathrm{Hp}$. What is the efficiency of the pump?
%
%\item The water horsepower of a pump is $25 \mathrm{Hp}$ and the brake horsepower output of the motor is $48 \mathrm{Hp}$. What is the efficiency of the pump?
%
%\item The efficiency of a well pump is determined to be $75 \%$. The efficiency of the motor is estimated at $94 \%$. What is the efficiency of the well?
%
%\item If a motor is $85 \%$ efficient and the output of the motor is determined to be 10
%$\mathrm{BHp}$, what is the electrical horsepower requirement of the motor?
%
%\item The water horsepower of a well with a submersible pump has been calculated at 8.2 WHp. The Output of the electric motor is measured as $10.3 \mathrm{BHp}$. What is the efficiency of the pump?
%
%  \item Water is being pumped from a reservoir to a storage tank on a hill. The elevation difference between water levels is 1200 feet. Find the pump size (in Hp) required to fill the tank at a rate of 120 gpm.
%  
%  \end{enumerate}
%  \end{tcolorbox}

