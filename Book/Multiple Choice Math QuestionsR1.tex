%\chapter*{Multiple Choice Math Questions}
\section*{Multiple Choice Math Questions}
\begin{enumerate}
\item How many pounds per day of $100 \%$ chlorine gas are needed to arrive at a dosage of $2 \mathrm{mg} / \mathrm{L}$, when the flow is $8.8 \mathrm{mgd}$ and a zero chlorine demand exists?\\
a. $* 147$\\
b. $1,097.9$\\
c. 1,468\\
d. 211.3\\
e. 417\\
\item How many pounds per day of $100 \%$ chlorine gas are needed to arrive at a residual of $2.3 \mathrm{mg} / \mathrm{L}$, when the flow is $8.25 \mathrm{mgd}$ and a chlorine demand is $0.35 \mathrm{mg} / \mathrm{L}$ ?\\
a. 21.8\\
c. 27.4\\
e. 812.3\\
b. $* 182.3$\\
d. 158.3\\
\item When a filter whose surface loading rate is $1,500 \mathrm{gpd} / \mathrm{sq}$.ft. and its size is 400 -sq. ft. Determine the total flow through the filter in gallons per day.\\
a. 0.6\\
c. 7.85\\
e. $* 600,000$\\
b. 375,000\\
d. 3.75\\
\item Determine the Unit Filter Run Volume of a 15-ft. x 20-ft. filter when it registered 2,000,000 gallons during its run.\\
a. 600\\
c. 989\\
b. 9423\\
d. $* 6667$\\
e. 7200\\
\item A filter has the dimensions of $15-\mathrm{ft}$. $x 20-\mathrm{ft}$ and a backwash rate of $19.5 \mathrm{gpm} / \mathrm{sq} . \mathrm{ft}$. Determine its backwash rise rate in, inches per minute.\\
a. 58.5\\
c. 5.85\\
e. $* 31.2$\\
b. 37\\
d. 81.3\\
\item Using the Quantity formula $(\mathrm{Q}=\mathrm{AV})$, determine the $\mathrm{Q}$ when $\mathrm{A}=15 \mathrm{sq}$. $\mathrm{ft}$. and velocity is $3.3 \mathrm{ft} / \mathrm{sec}$.\\
a. $* 49.5 \mathrm{cfs}$\\
c. $19.88 \mathrm{cfs}$\\
e. $4.9 \mathrm{cfs}$\\
b. $495 \mathrm{cfs}$\\
d. $4.545 \mathrm{cfs}$\\
\item Find the gpm/sq.ft. filtration rate when 6,775,000 gallons were produced in 24-hours through a filter that measures $30-\mathrm{ft} . x 54-\mathrm{ft}$.\\
a. 0.029\\
c. 2904.2\\
e. 4.18\\
b. 0.29\\
d. $* 2.9$\\
\item What is the grain per gallon (gpg) hardness of water that has a total hardness of $228 \mathrm{mg} / \mathrm{L}$ ?\\
a. *14\\
c. 18\\
e. 133.3\\
b. 3898.8\\
d. 39\\
\item A water tank had a pressure gauge reading of 14 psig on its bottom. Determine the water level in the tank.\\
a. 23.3 feet\\
c. $* 32.3$ feet\\
e. 19.8\\
b. 28.6 feet\\
d. 38.3 feet\\
\item A tank had a diameter of 22 -feet and a pressure of $7.7 \mathrm{psi}$ on its bottom. Determine how many pounds of $65 \%$ calcium hypochlorite (dry powder chlorine) are needed to arrive at a dosage of $1 \mathrm{ppm}$.\\
a. 0.42 pounds\\
b. $* 0.65 \mathrm{lbs}$\\
c. $50,302.5$ pounds\\
d. $41.9 \mathrm{lbs}$\\
e. $6.5 \mathrm{lbs}$\\
\item An iron removal plant processes water with an average iron concentration of $2.5 \mathrm{mg} / \mathrm{l}$. If the iron concentration is $0.01 \mathrm{mg} / \mathrm{l}$ after treatment and the total daily pumpage is one million gallons, how many pounds of iron will be removed per day?\\
a. 10.77 pounds\\
a. $* 20.77$ pounds\\
b. 25.77 pounds\\
c. 30.77 pounds\\
d. 35.77 pounds\\
\item A water system bills quarterly at a rate of $25 \notin / 1000$ gallons for the first 10,000 gallons, $30 \notin / 1000$ gallons for the next 10,000 gallons, $35 \phi / 1000$ gallons for all over 20,000 gallons. If a customer uses 35,000 gallons per quarter, what is the water bill?\\
a. $\$ 9.50$\\
b. $* \$ 10.75$\\
c. $\$ 12.25$\\
d. $\$ 12.50$\\
e. $\$ 13.25$\\
\item A ground level storage tank is 25 feet long, 20 feet wide, and 10 feet deep. When the storage tank is completely empty, calculate how many minutes it will take to fill the tank with a pump that has a capacity of 300 gallons per minute.\\
a. 60 minutes\\
b. 100 minutes\\
c. $* 125$ minutes\\
d. 150 minutes\\
e. 200 minutes\\
\item A room measures $12 \mathrm{ft}$ high, $30 \mathrm{ft}$ long, and $17 \mathrm{ft}$ wide. How many cubic feet per minute of air must a blower in an air exchange unit move to completely change the air every 10 minutes?\\
a. 102\\
b. 612\\
c. 1,020\\
d. 6,120\\
\item If a trench is $526 \mathrm{ft}$ long, $4.0 \mathrm{ft}$ wide, and $5.5 \mathrm{ft}$ deep, how many cubic yards of soil were excavated?\\
$$
526 \times 4 \times 5.5=\frac{11,572 \mathrm{f}^{3}}{27}=428
$$
\item If exactly $100 \mathrm{gal}$ of polymer costs $\$ 19.50$, what will $5,500 \mathrm{gal}$ cost, assuming no quantity discount?\\
$$
\frac{19.50}{100} \times 5500=1,072.5
$$
\item What is the velocity of flow in feet per second for an 8.0-in. diameter pipe if it delivers 675 gprax?\\
$$
\begin{gathered}
\frac{150}{6.35} \\
\frac{675}{449}=15 \mathrm{cuft} / \mathrm{s} \\
430 \mathrm{ft} / \mathrm{s}\\
\end{gathered}
$$
\item What should the setting be on a chlorinator in pounds per day if the dosage desired is $2.90 \mathrm{mg} / \mathrm{L}$ and the pumping rate from the well is $975 \mathrm{gpm}$ ?\\
$$
\frac{975}{69)^{1}}=1.4 M G D
$$
$$
1.4 \times 8.34 \times 2.90
$$
\item A treatment plant uses $278 \mathrm{lb} / \mathrm{d}$ of chlorine gas. If the chlorine demand is $0.85 \mathrm{mg} / \mathrm{L}$ and the chlorine residual is $1.50 \mathrm{mg} / \mathrm{L}$, how many million gallons per day are being treated?
$$
\begin{gathered}
\frac{278}{8.34 \times 2 \times 35}= \\
\text { dojarge } ; \\
0.85+1.50= \\
2.35 \mathrm{mg} / \mathrm{h}\\
\end{gathered}
$$
\item A water tank that is $105 \mathrm{ft}$ in diameter needs to be disinfected with a $5.0 \%$ sodium hypochlorite solution. If the tank is to be filled to only a depth of 5.0ft and the concentration required is $20.0 \mathrm{mg} / \mathrm{L}$, how many gallons of sodium hypochlorite are needed? Assume the sodium hypochlorite solution weighs $8.92 \mathrm{lb} / \mathrm{gal}$.\\
$$
\begin{gathered}
\frac{0.323 M G P}{0.323 \times 8.34 \times 20}= \\
9 / 23 / 19\\
\end{gathered}
$$
\item Convert $8.0 \mathrm{cfs}$ to gpm.\\
a. $1.07 \mathrm{gpm}$\\
b. $64.2 \mathrm{gpm}$\\
c. $480 \mathrm{gpm} \quad 8 \times 449$\\
e. 3,436gpm\\
\item Conyert $4,000 \mathrm{gpm}$ to $\mathrm{cfs}$.\\
a. $8.91 \mathrm{cfs}$\\
b. $66.65 \mathrm{cfs}$\\
c. $499 \mathrm{cfs}$\\
d. $535 \mathrm{cfs}$\\
e. $32,076 \mathrm{cfs}$\\
\item Convert 12MGD to gpm.\\
a. $0.00833 \mathrm{gpm}$\\
b. $7,200 \mathrm{gpm} 12 \times 700$\\
d. $17,280 \mathrm{gpm}$\\
e. $199,992 \mathrm{gpm}$\\
\item Convert 5.5 cfs to MGD.\\
a. $0.059 \mathrm{MGD}$\\
b. $0.148 \mathrm{MGD}$\\
c. $0.475 \mathrm{MGD}$\\
(64) more Aaumole\\
e. 7,920 MGD\\
\item Convert 45 Acre-feet into million gallons.\\
a. $6.02 \mathrm{Mgal}$\\
b. $1.96 \mathrm{Mgal}$\\
c. $14.7 \mathrm{Mgal}$\\
d. $45 \mathrm{Mgal}$\\
e. $336.6 \mathrm{Mgal}$\\
\item Convert 6.5 feet per second into miles per hour.\\
a. $4.43 \mathrm{mph}$\\
c. $13.3 \mathrm{mph}$\\
d. $106 \mathrm{mph}$\\
e. $266 \mathrm{mph}$\\
\item Convert 3.4 miles into feet.\\
a. 5,000 feet\\
b. 5,280 feet\\
c. 5,984 feet\\
d. 10,000 feet\\
(e.) 17,952 feet\\
$$
=\frac{6.5 / 5280}{1 / 60 \times 60}
$$
$1 \mathrm{~m} / \mathrm{e}=25 \mathrm{ft}$\\
\item Convert 2,250gpm into MGD.\\
a. $0.054 \mathrm{MGD}$\\
b. 3.24MGD\\
d. 2, 250MGD\\
e. 3,240 MGD\\
\item Convert 9.75MGD into cfs.\\
a. $15.1 \mathrm{cfs}$\\
b. $37.75 \mathrm{cfs}$\\
c. $113 \mathrm{cfs}$\\
d. $363 \mathrm{cfs}$\\
e. $845 \mathrm{cfs}$\\
\item Convert $1,000,000$ cubic feet into Acre-feet.\\
a. $0.04356 \mathrm{AF}$\\
b. $0.325829 \mathrm{AF}$\\
c. $3.07 \mathrm{AF}$\\
d. $22.96 \mathrm{AF}$\\
e. $172 \mathrm{AF}$\\
\item What is the chlorine residual in a treated water if the dosage is $2.1 \mathrm{mgi} 1$ and has a demand of $0.8 \mathrm{mg} / 1$\\
a. $0.8 \mathrm{mg} / 1$\\
b. $1 \mathrm{mg} / \mathrm{l}$\\
c. $2.1 \mathrm{mg} /$ !\\
d. $2.9 \mathrm{mg} / 1$\\
\item What is the maximum amount of ch1orine gas that can be removed from a 150-lb cylinder in $24 \mathrm{hrs}$ ?\\
a. 26 lbs.\\
b. $40 \mathrm{Jbs}$.\\
c. $75 \mathrm{lbs}$.\\
d. there is no maxim.um\\
\item Bow many gallons would be contained in a circular tank that is $100 \mathrm{ft}$. in diameter and $10 \mathrm{ft}$ deep?\\
a. $587,0.00$ gallons\\
b. 657,000 gallons\\
c. 1,340,000 gallons\\
d. 2,349,000 gallons\\
\item In order to rebuild a manhole, it will be necessary to remove the asphalt from a 35-foot diameter circle in a street. The pavement area involved is:\\
a. $\quad 208$ sq.ft\\
b. 241, sq.ft\\
c. $\quad 962$ sq.ft\\
d. 1125 sq.ft\\
\item If the chlorine demand of water is $2.5 \mathrm{mg} / 1$ and you want a residual of $0.5 \mathrm{mg} / \mathrm{l}$, how much chlorine would need to be fed to one million gallons?\\
a. 25lbs.\\
b. 30lbs.\\
c. 34lbs.\\
d. 38lbs.\\
\item If you need to feed chlorine at orate of $2.1 \mathrm{mg} / 1$ and you treat $2,300,000$ gallons. How many pounds of chlorine should you use?\\
a. 4lbs.\\
b. 17lbs.\\
c. 35lbs.\\
d. 40lbs.\\
\item What is the head on a system exerting a static pressure of 62 psi?\\
a. 89 feet\\
b. 107 feet\\
c. 143 feet\\
d. 189 feet\\
\item A head of 200 feet would equal:\\
a. $46.6 \mathrm{psi}$\\
b. $56.6 \mathrm{psi}$\\
c. $66.6 \mathrm{psi}$\\
d. $86.6 \mathrm{psi}$\\
\item If a 3,000,000 gpd flow is to be dosed with $1.2 \mathrm{mg} / \mathrm{l}$, what should the chlorinator feed rate be set at in lbs. of chlorine per day?\\
a. 3.0lbs./ day\\
b. $4.5 \mathrm{1bs} . / \mathrm{day}$\\
c. 10lbs./ day\\
d. 30lbs./ day\\
\item Calculate the area in square feet of: a space $100 \mathrm{ft}$ long and $75 \mathrm{ft}$ svide. $100 \times 75$ Ans. TTur Sq. Fr.\\
\item Calculate the volume of a rectangular tank 20 feet high, $100 \mathrm{ft}$ long, and 75 feet wike. Ans. $\frac{10 \text { wo }}{1} \mathrm{Cu} . \mathrm{Ft}$.\\
\item Calculate the gallons the tank in the preceding problem will hold. Ans. 1,122,000 Gallons\\
\item Calculate the area in square feet of a space $40 \mathrm{ft}$ long and 50 feet vvide. Ans, 2000 Sq. Ft\\
\item Calculate the volume of a rectangular tank $40 \mathrm{ft}$ long, $50 \mathrm{ft}$ wide and 25 feet tall.\\
$$
\text { Ans. 50, } 000 \text { Cu.Ft }
$$\\
\item Calculate the gallons the tank in the preceding problem will contain. Ans. 374,000 Gallons\\
\item Calculate the area of a circle with a $10 \mathrm{ft}$ radius.\\
Ans. $314 \mathrm{Sq} \mathrm{Ft}$\\
\item Calculate the aren of a circle with a $10 \mathrm{ft}$ diameter.\\
Ans. $78-5$\\
\item Calculate the yolmue of a tank with a $50 \mathrm{ft}$ diameter that is 20 feet high.\\
Ans. 39, 260\\
$\mathrm{Cu} . \mathrm{Ft}$\\
\item How many gallons will the tank in the preceding problem hold?\\
\item Calculate the area of a circle with a $100 \mathrm{ft}$ dibmeter.\\
$$
\text { Ans. 293, } 590
$$
\\textbf{Ans. 7, 800SqFt}\\
\item Calculate the volume of a tank with a $100 \mathrm{ft}$ diameter that is 50 feet high. Ans.\\
\item How many gallons ryill the tank in the preceding problen hold? Auls.\\
$2,935,960$\\
$$
18 \times 18 \times 0^{\prime} 0408 \times 1200
$$
\item How many gallons will an $18^{\prime \prime}$ diameter pipeline, 1200 ' long contain?\\
$$
\text { imi }=5280 \text { Ans. } 15,863 \text { Gallons }
$$
\item How many gallons will a 24 " pipeline, 2 miles long contain?\\
$$
24 \times 24 \times 0.0408 \times 248,168
$$\\
\item 500 GPM is how many gallons per hour?\\
10560\\
$$
\frac{500 \mathrm{~g}^{\prime}}{1 \mathrm{~m}}=\frac{500}{1 / 10 h}=\text { Ans } \frac{30 \mathrm{e} 0 \mathrm{r}}{\mathrm{gph}}
$$\\
\item $30,000 \mathrm{gph}$ is how many gallons per day?\\
$$
\frac{30,000 \mathrm{~g}}{h}=\frac{30,000}{1 / 24}=30,020 \times 24 \text { Ans. } \frac{720,010}{\mathrm{gpd}}
$$
\item A flow of $25 \mathrm{gpm}$ is low many gpd?\\
$$
\frac{25 \mathrm{~g}}{m}=\frac{25}{1 / 60} \times \frac{1}{24} \quad(25 \times 1440) \text { Ans. 36, } 0 w \mathrm{gpd}
$$\\
\item A flow of $800,000 \mathrm{gpd}$ is how many gpm? $\frac{800,000}{0.8 M G D \times 700}=560 \times 24 \times 60 \quad 555.55$ $0.80 \mathrm{gpm}$\\
\item A flow of $150 \mathrm{gpnn}$ is how many MGD?\\
$$
700 \text { (wm. }
$$\\
Ans.\\
MGD\\
\item How many gallons will an $8^{\prime \prime}$ pipeline 550 ' long contain?\\
$$
8 \times 8 \times 0.0408 \times 550
$$\\
Ans. $1436 \cdot \frac{16}{\text { gallons }}$\\
\item Water is filling a tank at the rate of $50 \mathrm{gpm}$ for a $10 \mathrm{~min}$. period, How many gallons of water are contained in the tank at the end of the 10 minute time period?\\
\item A well pump is discharging water at the rate of $400 \mathrm{gpm}$ into a tank for 15 minutes. Haw many gallons will be in the tank at the end of this time period?\\
$$
6,000 \ldots \text { Gal. }
$$\\
Dose $=$ Demand T Residual $\frac{300}{20} \times 16$\\
\item A tank is filling at the rate of $300 \mathrm{gpm}$ for a 20 minute period. How many of water will be contained in the tank at the end of 16 minutes?\\
4880 Gal.\\
\item Before pumping, the static water level in a well is 15 feet. During pumping, the water lewel drops to 45 feet. What is the drawdown? $45-15=30$\\
a. 15\\
b. 30\\
c. 45\\
d. 60\\
e. 90\\
\item Over a four year period, the hour meter on a electrical panel at a well site had the following readings at the end of each year: $1^{\text {st }}$ year $-976.3,2^{\text {nd }}$ year $-1325.8,3^{\text {rd }}$ year -2007.1 , and $4^{\text {th }}$ year -2371.4 . How many hours does the meter show the well ran during the $3^{\text {rd }}$ year?\\
a. $349.5 \mathrm{hrs}$\\
b. $3364.3 \mathrm{hrs}$\\
c. $981.3 \mathrm{hrs}$\\
d. $830.2 \mathrm{hn} / \mathrm{s}$\\
e. 900.1\\
\item One gallon of water weighs how many lbs?\\
a. 7.48\\
b. 8.34\\
c. 2.31\\
d. 43318 .\\
\item A water tank is filled to depth of 22 feet. What is the psi at the bottom of the tank?\\
$$
22 / 2.31 \quad 9.52
$$\\
\item The static pressure in a water main is 85 psi. What elevation of water is needed to provide that kind of pressure?\\
$$
85 \times 2.31 \quad 196
$$\\
\item Calculate the pressure at the bottom of a water tank if it is filled to a depth of 33 feet.\\
$$
33 / 2.31
$$
$\mathrm{ft}$\\
\item A psi gauge is located at the bottom of a water tank and reads 24 psi. What is the elevation of the water inside the tank?\\
$$
24 \times 2.31 \quad 55
$$\\
\item A gauge is reading the pressure at the outlet of a fire hydrant. A tank is elevated 200 feet above the hydrant. What is the gaige pressure at the hydirant?\\
$$
20 s / 2.31
$$\\
\item A gauge is attached to a hose bib at a house. The gauge reads $45 p$ ps. How much elevation is needed to supply that pressure?\\
$$
45 \times 2.31 \quad 104
$$\\
\item How many galtons will the above cylinder hold?\\
$$
100 \times 7.48=748
$$\\
\item Calculate the area in square feet of a space $100 \mathrm{ft}$. long and $75 \mathrm{ft}$ widie.\\
Ans. 7500 Sq. Fit\\

\item Calculate the volume of a rectangular tank 20 feet high, $100 \mathrm{ft}$ long, and 75 feet wide. Ans. 150,00 Cu.F.\\
\item Calculate the gallons the tauk in the preceding problem will hold.\\
\item Calculate the area in square feet of a space $40 \mathrm{ft}$ long and 50 feet wide. Ans, $\frac{2,000}{1}$ Sq. Ft\\
\item Calculate the volume of a rectangular tank $40 \mathrm{ft}$ long, $50 \mathrm{ft}$ wide and 25 feet tall. Ans. 50000 Cu.Ft\\
\item Calculate the gallons the tank in the preceding problem will contain.\\
$$
50,000 \times 7.48 \text { Ans. 374,000 Gallons }
$$\\
\item Caiculate the area of a circle with a $10 \mathrm{ft}$ radius.\\
$$
D=20 
$$\\
Ans. $314 \mathrm{SqFt}$\\
\item Calculate the area of a circle with a $10 \mathrm{ft}$ diameter. Ans. 78.5 SqFt\\
\item Calculate the volume of a tank with a $50 \mathrm{ft}$ diameter that is 20 feet high.\\
$$
\text { Ans. } 250 \text { Cu. Ft }
$$\\
\item How many gallons will the tank in the preceding problem hold?\\
\item Calculate the area of a circle with a $100 \mathrm{ft}$ ciameter. Ans. $\frac{293590}{7850}$ Gallons\\
\item Caiculate the volume of a tank zwith a $100 \mathrm{ft}$ diameter that is 50 feet higl,\\
$$
\frac{300 \text { gallons }}{1 \text { mute }} \times 6 \mathrm{~mm}
$$\\
\item A tank is filling at the rate of $300 \mathrm{gpm}$ for a 20 minute period. How many of water will be contained in the tank at the end of 16 minutes?\\
\item Before pumping, the static water level in a well is 15 feet. During pumping, the water level drops to 45 feet. What is the drawdown?\\
a. 15\\
b. 30\\
c. 45\\
d. 60\\
e. 90\\
\item Over a four year period, the hour meter on a electrical panel at a well site had the following.readings at the end of each year: $1^{\text {sl }}$ year $-976.3,2^{\text {td }}$ year $-1325.8,3^{\text {rd }}$ year -20071 , and $4^{\text {th }}$ year -2371 . 4 . How many hours does the meter show the well ran during the $3^{\text {rd }}$ year?\\
a. $349.5 \mathrm{hrs}$\\
b. 3364.3 his\\
c. $\oint 81.3 \mathrm{hrs}$\\
d. $830.2 \mathrm{hrs}$\\
e. 900.1\\
\item Approximately how many gallons of water can fit into a reservoir that is 35 feet tall and has a 100 foot diameter?\\
a. 2,000,000\\
b. $2.055 \mathrm{MG}$\\
c. $4 \mathrm{MG}$\\
d. 275,000\\
\item Determine the detention time in hours for the following water treatment system:\\
Distribution pipe from water plant to storage tank is $549 \mathrm{ft}$ in length and $14 \mathrm{in}$. in diameter - Storage tank averages 2,310,000 gal of water at any given time.  Flow through system is $6.72 \mathrm{mgd}$\\
a. $7.2 \mathrm{hr}$\\
b. $7.4 \mathrm{hr}$\\
c. $8.0 \mathrm{hr}$\\
d. $8.3 \mathrm{hr}$\\

\item If chlorine is being fed at a rate of $260 \mathrm{lb} / \mathrm{day}$ for a flow rate of $23 \mathrm{cfs}$, what should be the adjustment on the chlorinator when the flow rate is decreased to $16 \mathrm{cfs}$, if all other water parameters remain the same?\\
a. $160 \mathrm{lb} / \mathrm{day}$\\
b. $180 \mathrm{lb} / \mathrm{day}$\\
c. $310 \mathrm{lb} /$ day\\
d. $370 \mathrm{lb} /$ day\\
\item How many gallons of a sodium hypochlorite solution that contains $12.1 \%$ available chlorine are needed to disinfect a 1.5 -ft diameter pipeline that is $283 \mathrm{ft}$ long, if the dosage required is $50.0 \mathrm{mg} / \mathrm{L}$ ? Assume the sodium hypochlorite is $9.92 \mathrm{lb} / \mathrm{gal}$.\\
a. 0.87 gal sodium hypochlorite\\
b. 1.0 gal sodium hypochlorite\\
c. 1.3 gal sodium hypochlorite\\
d. 1.5 gal sodium hypochlorite\\
\item A storage tank has a 60.0-ft radius and averages $25.5 \mathrm{ft}$ in water depth. Calculate the average detention time in hours for this storage tank, if flow through the tank averages 2.91 mgd during the month in question.\\
a. $17.5 \mathrm{hr}$\\
b. $17.8 \mathrm{hr}$\\
c. $18.6 \mathrm{hr}$\\
d. $19.8 \mathrm{hr}$ 95. A 24.0-in. pipeline, $427 \mathrm{ft}$ long, was disinfected with calcium hypochlorite tablets with $65.0 \%$ available chlorine. Determine the chlorine dosage in $\mathrm{mg} / \mathrm{L}$, if $7.0 \mathrm{1b}$ of calcium hypochlorite was used.\\
a. $25 \mathrm{mg} / \mathrm{L}$ chlorine\\
b. $39 \mathrm{mg} / \mathrm{L}$ chlorine\\
c. $43 \mathrm{mg} / \mathrm{L}$ chlorine\\
d. $54 \mathrm{mg} / \mathrm{L}$ chlorine\\
\item A well yields 2,840 gallons in exactly 20 minutes. What is the well yield in gpm?\\
a. $140 \mathrm{gpm}$\\
b. $142 \mathrm{gpm}$\\
c. $145 \mathrm{gpm}$\\
d. $150 \mathrm{gpm}$\\
\item What is the area of a circular tank pad in $\mathrm{ft} 2$, if it has a diameter of $102 \mathrm{ft}$ ?\\
a. $6,160 \mathrm{ft} 2$\\
b. $6,167 \mathrm{ft} 2$\\
c. $8,170 \mathrm{ft} 2$\\
d. $8,200 \mathrm{ft} 2$\\
\item What is the pressure at 1.85 feet from the bottom of a water storage tank if the water level is 28.7 feet?\\
a. $11.6 \mathrm{psi}$\\
b. $12.4 \mathrm{psi}$\\
c. $62.0 \mathrm{psi}$\\
d. 66.3 psi\\
\item How many gallons are in a pipe that is 18.0 inches in diameter and 1,165 feet long?\\
a. 2,060 gal\\
b. $10,300 \mathrm{gal}$\\
c. 15,400 gal\\
d. 17,200 gal\\
\item Convert 37.4 degrees Fahrenheit to degrees Celsius.\\
a. $3.0 \mathrm{C}$\\
b. $5.3 \mathrm{C}$\\
c. $7.9 \mathrm{C}$\\
d. $9.7 \mathrm{C}$\\
\item If 288 is $70.3 \%$, how much is $100 \%$ ?\\
a. 410\\
b. 202\\
c. 218\\
d. 438\\
\item If the pressure head on a fire hydrant is $134 \mathrm{ft}$, what is the pressure in psi?\\
a. $50 \mathrm{psi}$\\
b. $52 \mathrm{psi}$\\
c. 54 psi\\
d. $58 \mathrm{psi}$\\
\item A meter indicates the water flow from a fire hydrant is $5.5 \mathrm{ft} 3 / \mathrm{min}$. How many gallons will flow from the hydrant in 20 minutes?\\
a. 820 gal\\
b. $850 \mathrm{gal}$\\
c. $880 \mathrm{gal}$\\
d. 920 gal\\
\item Records for a pump show that on June 1 at exactly 9:00 a.m. the number of pumped gallons was 71,576,344 and on July 1 at exactly 9:00 a.m. it was 72,487,008 gallons. Determine the average gallons pumped per day (gal/day) for this month to the nearest gallon.\\
a. $18,605 \mathrm{gal} / \mathrm{day}$\\
b. $25,875 \mathrm{gal} / \mathrm{day}$\\
c. $30,355 \mathrm{gal} / \mathrm{day}$\\
d. $34,325 \mathrm{gal} / \mathrm{day}$\\
\item How much paint will it take for a single coat of the top and sidewalls of the storage tank that is 100 -feet in diameter and 30 -feet tall, if one gallon of paint covers 200 square feet?\\
a. 86 gallons\\
b. 96 gallons\\
c. 106 gallons\\
d. 116 gallons\\
e. 126 gallons\\
\item Under like conditions, how much more water would an 8-inch pipe carry than a 4-inch pipe?\\
a. 2 times\\
b. 3 times\\
c. 4 times\\
d. not enough information given\\
\item If a lake is 574 feet deep, what is the pressure in pounds per square inch at the bottom of the lake?\\
a. $248 \mathrm{psi}$\\
b. $1326 \mathrm{psi}$\\
c. $69 \mathrm{psi}$\\
d. $62.4 \mathrm{psi}$\\
\item A pressure gauge reading is 80 psi. How many feet of head is this?\\
a. 173 feet\\
b. 185 feet\\
c. 200 feet\\
d. 212 Feet 
\item The pump is 150 feet below the reservoir level. What is the pressure reading on the gauge in psi?\\
a. $52 \mathrm{psi}$\\
b. 60 psi\\
c. 65 psi\\
d. $75 \mathrm{psi}$\\
\item A tank is $20^{\prime}$ ' $60^{\prime}$ ' by $15^{\prime}$ deep. What is the volume in gallons?\\
a. 115, 000 gallons\\
b. 128,000 gallons\\
c. 135,000 gallons\\
d. 154,000 gallons\\
\item A tank is 60 ' in diameter and 22' high. How many gallons will it hold?\\
a. 465,000 gallons\\
b. 528,000 gallons\\
c. 640,000 gallons\\
d. 710,000 gallons\\
\item A dosage of $2.4 \mathrm{mg} / \mathrm{l}$ of chlorine gas is added to $3.8 \mathrm{mgd}$. How many pounds per day of chlorine are needed?\\
a. $68 \mathrm{lbs} /$ day\\
b. $76 \mathrm{lbs} / \mathrm{day}$\\
c. $82 \mathrm{lbs} /$ day\\
d. $88 \mathrm{lbs} /$ day\\
\item How many gallons are in a 6" pipe 950 feet long?\\
a. 1108 gallons\\
b. 1253 gallons\\
c. 1308 gallons\\
d. 1395 gallons\\
\item A 12 " pipe is carrying water at a velocity of $5.8 \mathrm{fps}$. What is the flow?\\
a. $4.55 \mathrm{cfs}$\\
b. $5.36 \mathrm{cfs}$\\
c. $5.67 \mathrm{cfs}$\\
d. $6.04 \mathrm{cfs}$\\
\item The pressure at the top of the hill is 62 psi. The pressure at the bottom of the hill, 60 feet below, is $100 \mathrm{psi}$. The water is flowing uphill at $120 \mathrm{gpm}$. What is the friction loss, in feet, in the pipe?\\
a. 24.6 feet\\
b. 27.8 feet\\
c. 31.2 feet\\
d. 33.8 feet

\item A tank is 82 ' in diameter and 31 feet high. The flow is $1600 \mathrm{gpm}$. What is the detention time in hours?\\
a. 12.75 hours\\
b. 14.80 hours\\
c. 16.00 hours\\
d. 18.25 hours\\
\item A tank is $120^{\prime}$ x $50^{\prime}$ ' $14^{\prime}$ 'deep. The flow is $2.8 \mathrm{mgd}$. What is the detention time in hours?\\
a. 3.8 hours\\
b. 4.4 hours\\
c. 5.3 hours\\
d. 6.2 hours\\
\item A 16" pipe is 1250 feet long. How much $65 \%$ HTH is needed to dose it with $50 \mathrm{mg} / \mathrm{l}$ of chlorine?\\
a. $6.50 \mathrm{lbs}$\\
b. $7.25 \mathrm{lbs}$\\
c. $7.96 \mathrm{lbs}$\\
d. $8.34 \mathrm{lbs}$\\
\item A solution of hydrofluosilisic acid is $22 \%$ fluoride. If $750 \mathrm{ppb}$ are added to 5,600,000 gallons/day, how many $\mathrm{ml} / \mathrm{min}$ should the pump be feeding?\\
a. $26 \mathrm{ml} / \mathrm{min}$\\
b. $35 \mathrm{ml} / \mathrm{min}$\\
c. $42 \mathrm{ml} / \mathrm{min}$\\
d. $50 \mathrm{ml} / \mathrm{min}$\\
\item A bleach system feeds $12 \%$ bleach. The dosage is $1.4 \mathrm{mg} / \mathrm{l}$ for $8.2 \mathrm{mgd}$. How many $\mathrm{ml} / \mathrm{min}$ should the pump feed?\\
a. $200 \mathrm{ml} / \mathrm{min}$\\
b. $250 \mathrm{ml} / \mathrm{min}$\\
c. $300 \mathrm{ml} / \mathrm{min}$\\
d. $350 \mathrm{ml} / \mathrm{min}$\\
\item Pump Data:\\
Feet - Positive Suction Head\\
158 Feet - Discharge Head\\
26 Feet - Friction Loss\\
1200 gpm - Flow\\
Motor Efficiency - 86\%\\
Pump Efficiency - 78\%\\
What is the motor horsepower?\\
a. $60 \mathrm{MHp}$\\
b. $65 \mathrm{MHp}$\\
c. $70 \mathrm{MHp}$\\
d. $75 \mathrm{MHp}$\\
\item Pump Data:\\
20 Feet - Positive Suction Head\\
185 Feet - Discharge Head\\
18 Feet - Friction Loss\\
$300 \mathrm{gpm}$ - Flow\\
Motor Efficiency - 90\%\\
Pump Efficiency - 80\%\\
$\mathrm{Kw}$-Hour Cost $=0.11 / \mathrm{Kw}-\mathrm{Hr}$\\
Average Run Time - 6 Hours/day\\
What is the cost to run the pump for 30 days?\\
a. $\$ 245.08$\\
b. $\$ 284.34$\\
c. $\$ 410.50$\\
d. $\$ 463.82$\\
\item Determine the drawdown from a well measuring a static water level of 120 feet and a pumping water level of 205 feet?\\
a. $105 \mathrm{ft}$\\
b. 320 feet\\
c. 85 feet\\
d. 310 feet\\
\item Before pumping, the static water level in a well is 15 feet. During pumping, the water level drops to 45 feet. What is the drawdown?\\
a. $15 \mathrm{ft}$\\
b. $30 \mathrm{ft}$\\
c. $45 \mathrm{ft}$\\
d. $60 \mathrm{ft}$\\
e. $90 \mathrm{ft}$\\
\item What is the chlorine demand of a tank that is dosed at $3.5 \mathrm{ppm}$ and has a residual of 1.25 ppm.\\
a. 2.25\\
b. 4.75\\
c. 1.25\\
d. 3.5\\
e. not enough information\\
\item What $1 \mathrm{~s}$ the area of a trench that is $22.4 \mathrm{ft}$ Jong and 3.3 feet wide?\\
a. 26 sq.ft\\
b. 74 sq. ft.\\
c. 143 sq. $\mathrm{ft}$\\
d. 187 sq. ft. 

\item What is the pounds per square inch pressure at the bottom ofa tank if the water level is 38.29 feet?\\
a. $7.3 \mathrm{psi}$\\
b. $16.6 \mathrm{psi}$\\
c. $53.9 \mathrm{psi}$\\
d. $88.4 \mathrm{psi}$\\
\item What is the pressure head on a system exerting a static pressm.e of 62 psi?\\
a. $27 \mathrm{ft}$\\
b. $89 \mathrm{ft}$\\
c. $143 \mathrm{ft}$\\
d. $175 \mathrm{ft}$\\
\item How many gallons are in a pipe that is 18 " in diameterand 216 feet long?\\
a.1908 gallons\\
b. 2246 gallons\\
c. 2430 gallons\\
d. 2861 gallons\\
\item How many pounds of chlorine are required to u.eat $8.65 \mathrm{mgd}$ if the dosage is $2.75 \mathrm{ppm}$ ?\\
a. $11 \mathrm{lb} /$ day\\
b. $24 \mathrm{lb} / \mathrm{day}$\\
c. $72 \mathrm{lb} /$ day\\
d. $198 \mathrm{lb} /$ day\\
\item What should the setting be on a chlorinator in pounds per day if the dosage desired is 2.9 $\mathrm{mg} / \mathrm{l}$ and the pumping rate from the well is $975 \mathrm{gpm}$ ?\\
a. $29 \mathrm{lb} /$ day\\
b. $34 \mathrm{lb} / \mathrm{day}$\\
c. $41 \mathrm{lb} /$ day\\
d. $336 \mathrm{lb} / \mathrm{day}$\\
\item What is the chlorine residual in a system that bas a chlorine dosage of $2.75 \mathrm{mg} / 1 \mathrm{and} \mathrm{a}$ chlorine demand of $1.93 \mathrm{mg} ! 1$ ?\\
a. $0.82 \mathrm{mg} / \mathrm{l}$\\
b. $1.75 \mathrm{mg} / \mathrm{l}$\\
c. $4.67 \mathrm{mg} / \mathrm{l}$\\
d. $5.31 \mathrm{mg} / \mathrm{l}$\\
\item How many pounds per day of chlorine are needed to treat $38.75 \mathrm{mgd}$ if the residual is 2.0 $\mathrm{mg} / 1$ and the demand is $1.5 \mathrm{mg} / \mathrm{l}$ ?\\
a. $42 \mathrm{lb} /$ day\\
b. $136 \mathrm{lb} / \mathrm{day}$\\
c. $323 \mathrm{lb} / \mathrm{day}$\\
d. $1131 \mathrm{lb} / \mathrm{day}$ 

\item A pump discharges 680 gpm, How many gallons will it discharge in 8 hours?\\
a. 5440 gallons.\\
b. 130560 gallons\\
c. 1 ac-ft\\
d. 408000 gallons.\\
\item How many gallons are contained in 2167 cu.ft?\\
a. 260 gallons\\
b. 295 gallons\\
c. 16253 gallons\\
d. 18070 gallons\\
\item What is the typical strength of calcium hypochlorite, i.e., available chlorine range?\\
a. 5 to $10 \%$\\
b. 45 to $50 \%$\\
c. 65 to $70 \%$\\
d. 80 to $85 \%$\\
\item A four log removal is\\
a. $90.00 \%$\\
b. $99.00 \%$\\
c. $99.90 \%$\\
d. $99.99 \%$\\
\item A circular clearwell is 150 feet in diameter and 40 feet tall. The Clearwell has an overflow at 35 feet. What is the maximum amount of water the clearwell can hold in Million gallons rounded to the nearest hundredth?\\
a. $0.92 \mathrm{MG}$\\
b. $4.62 \mathrm{MG}$\\
c. $18.50 \mathrm{MG}$\\
d. 7.50 MG\\
\item A sedimentation basin is 400 feet length, 50 feet in width, and 15 feet deep. What is the volume expressed in cubic feet?\\
a. $100,000 \mathrm{ft}^{3}$\\
b. $200,000 \mathrm{ft}^{3}$\\
c. $300,000 \mathrm{ft}^{3}$\\
d. $400,000 \mathrm{ft}^{3}$\\
\item A clearwell holds $314,000 \mathrm{ft}^{3}$ of water. It is $100 \mathrm{ft}$ in diameter. What is the height of the clearwell?\\
a. $25 \mathrm{ft}$\\
b. $30 \mathrm{ft}$\\
c. $35 \mathrm{ft}$\\
d. $40 \mathrm{ft}$\\
\item A treatment plant operator must fill a clearwell with $10,000 \mathrm{ft} 3$ of water in 90 minutes. What is the rate of flow expressed in GPM?\\
a. 111 GPM\\
b. 831 GPM\\
c. 181 GPM\\
d. 900 GPM\\
\item A water tank has a capacity of 6MG. It is currently half full. It will take 6 hours to fill. What is the flow rate of the pump?\\
a. 3,333 GPM\\
b. 6,333 GPM\\
c. 8,333 GPM\\
d. 16,666 GPM\\
\item A clearwell with the capacity of $2.5 \mathrm{MG}$ is being filled after a maintenance period. The flow rate is 2,500 GPM. The operator begins filling at 7 AM. At what time will the clearwell be full?\\
a. 10:00 PM\\
b. 10:40 PM\\
c. 11:00 PM\\
d. 11:40 PM\\
\item There are four filters at a water treatment plant. The filters measure 20 feet wide by 30 feet in length. What is the filtration rate if the plant processes 8.0 MGD?\\
a. $1.51 \mathrm{GPM} / \mathrm{sq} . \mathrm{ft}$.\\
b. $2.31 \mathrm{GPM} / \mathrm{sq} . \mathrm{ft}$.\\
c. $2.61 \mathrm{GPM} / \mathrm{sq} . \mathrm{ft}$.\\
d. $2.91 \mathrm{GPM} / \mathrm{sq} . \mathrm{ft}$.\\
\item A water treatment plant treats 6.0 MGD with four filters. The filters use 60,000 gallons per wash. What is the percent backwash at the plant?\\
a. $10 \%$\\
b. $8 \%$\\
c. $6 \%$\\
d. $4 \%$\\
\item A treatment plant filter washes at a rate of $10,000 \mathrm{GPM}$. The filter measures $18 \mathrm{ft}$. wide by $24 \mathrm{ft}$. long. What is the rate of rise expressed in inches per minute?\\
a. $17 \mathrm{inch} / \mathrm{min}$\\
b. $27 \mathrm{inch} / \mathrm{min}$\\
c. $37 \mathrm{inch} / \mathrm{min}$\\
d. $47 \mathrm{inch} / \mathrm{min}$\\



\end{enumerate}

\newpage
\section{TREATMENT I AND II MATH}
\begin{enumerate}
  \item Which is the hardness in mg/l of a treatment plant's well water if the hardness is 18.44 grains per gallon (gpg)?\\
a. 1.1 mg/l\\
*b. 315.7 mg/l\\
c. 415.7 mg/l\\
d. 535.2 mg/l\\
  \item Find the specific yield in gpm/ft if a well produces 105 gpm and the drawdown for the well is 16.3 ft.\\
a. 6.00 gpm / ft\\
*b. 6.44 gpm / ft\\
c. 7.20 gpm / ft\\
d. 7.28 gpm / ft\\
  \item If the static level in the well was $138.6 ft and the drawdown was $21.1 ft, which must have been the pumping water level in the well?\\
a. $\quad 117.5 ft$\\
b. $\quad 129.0 ft$\\
c. $\quad 150.0 ft$\\
*d. $\quad 159.7 ft$\\
  \item Which is the pressure in $1 \mathrm{~b} / \mathrm{ft}^{2}, 189 ft below a lake's surface if the lake is $386 ft in depth?\\
a. $11,789 \mathrm{lb} / \mathrm{ft}^{2}$\\
b. $\quad 11,790 \mathrm{lb} / \mathrm{ft}^{2}$\\
c. $11,793 \mathrm{lb} / \mathrm{ft}^{2}$\\
*d. $11,800 \mathrm{lb} / \mathrm{ft}^{2}$\\
  \item A circular clarifier has a weir length of 162 ft. Which is the weir overflow rate in gpd/ ft, if the flow is 2,330,000 gallons per day (gpd)?\\
a. $14,000 gpd / ft$\\
b. $14,380 gpd / ft$\\
c. $14,383 gpd / ft$\\
*d. $14,400 gpd / ft$\\
  \item Convert $35.1 \mathrm{cfs}$ to gpm.\\
a. $14,200 \mathrm{gpm}$\\
*b. $15,800 \mathrm{gpm}$\\
c. $\quad 17,600 \mathrm{gpm}$\\
d. $18,300 \mathrm{gpm}$ 
  \item Convert 7.7 million gallons a day (mgd) into cubic feet per second (cfs).\\
a. $11 \mathrm{cfs}$\\
*b. $12 \mathrm{cfs}$\\
c. $15 \mathrm{cfs}$\\
d. $19 \mathrm{cfs}$\\
  \item How many million gallons (mil gal) are there in 318 acre-ft?\\
*a. 104 mil gal\\
b. $107 \mathrm{mil} \mathrm{gal}$\\
c. 110 mil gal\\
d. 116 mil gal\\
  \item Convert 68 degrees Fahrenheit to degrees Celsius.\\
*a. $20^{\circ} \mathrm{C}$\\
b. $37^{\circ} \mathrm{C}$\\
c. $45^{\circ} \mathrm{C}$\\
d. $65^{\circ} \mathrm{C}$\\
  \item Calculate $81.5 \%$ of 316 .\\
a. 219\\
b. 232\\
*c. 258\\
d. 267\\
  \item If 8.25 pounds of soda ash are mixed into 45 gallons of water, which is the percent of soda ash in the slurry?\\
a. $2.0 \%$ soda ash slurry\\
b. $2.1 \%$ soda ash slurry\\
*c. $2.2 \%$ soda ash slurry\\
d. $2.3 \%$ soda ash slurry\\
  \item Calculate the area of a circular reservoir in $\mathrm{ft}^{2}$, with a diameter of 411 ft.\\
a. $108,000 \mathrm{ft}^{2}$\\
b. $112,000 \mathrm{ft}^{2}$\\
c. $125,000 \mathrm{ft}^{2}$\\
*d. $\quad 133,000 \mathrm{ft}^{2}$\\
  \item Determine the circumference of a clarifier, if the radius is 95 ft.\\
a. 300 ft\\
b. 400 ft\\
c. 500 ft\\
*d. 600 ft\\
  \item Determine the volume in gallons for a pipe completely full of water given the following data:\\
\begin{itemize}
  \item Diameter $=1.5$ feet\\
  \item Length $=1.09$ miles\\
\end{itemize}
a. $65,000 \mathrm{gal}$\\
b. $68,000 \mathrm{gal}$\\
c. $74,000 \mathrm{gal}$\\
*d. $76,000 \mathrm{gal}$ \\
\item Which is the concentration of alum in mg/l, if $5.0 \mathrm{~mL}$ of a 0.30 grams/liter alum solution is added to $1,000 \mathrm{~mL}$ of deionized water?\\
a. 1.2 mg/l alum\\
*b. 1.5 mg/l alum\\
c. 1.8 mg/l alum\\
d. 1.9 mg/l alum\\
\item Which is the phenolphthalein alkalinity as mg/L CaCO3 of a water sample given the following parameters?\\
\begin{itemize}
  \item Sample size $=100 \mathrm{~mL}$\\
  \item Normality of the sulfuric acid $=0.02 \mathrm{~N}$\\
  \item Titrant used to pH of $8.3=1.8 \mathrm{~mL}$ (designated by convention as A)\\
\end{itemize}
a. 1.8 mg/l as CaCO3\\
b. 3.6 mg/l as CaCO3\\
*c. 18 mg/l as CaCO3\\
d. mg/l as CaCO3\\
  \item How many gallons are there in 28.65 acre-ft?\\
a. $9,354,282 \mathrm{gal}$\\
b. $9,322,137 \mathrm{gal}$\\
*c. $9,335,000 \mathrm{gal}$\\
d. $9,763,599$ gal\\
  \item If $7.3 \mathrm{lb}$ of polymer (assume 100\%) are mixed into 35 gal of water, determine the percentage of polymer in the slurry.\\
a. $2.1 \%$ slurry\\
*b. $2.4 \%$ slurry\\
c. $2.5 \%$ slurry\\
d. $2.8 \%$ slurry\\
  \item Which is the percentage of removal across a settling basin, if the influent is 17.1 NTU and the effluent is 1.13 NTU ?\\
a. $90.5 \%$ NTU removed\\
b. $92.5 \%$ NTU removed\\
c. $93.0 \%$ NTU removed\\
*d. $93.4 \%$ NTU removed\\
  \item Find the detention time in hours for a clarifier that has an inner diameter of 112.2 ft and a water depth of 10.33 ft if the flow rate is $7.26 \mathrm{mgd}$.\\
a. $2.10 \mathrm{hr}$\\
b. $2.14 \mathrm{hr}$\\
*c. $2.52 \mathrm{hr}$\\
d. $2.96 \mathrm{hr}$ \\
\item Calculate the lime dosage in mg/l that is required given the following parameters:\\
\begin{itemize}
  \item Jar test determines the alum dosage 8.5 mg/l\\
  \item Raw alkalinity 9.0 mg/l\\
  \item Residual alkalinity needed for precipitation 14 mg/l\\
\end{itemize}
Know: 1 mg/l of alum reacts with 0.45 mg/l alkalinity\\
1 mg/l of alum reacts with 0.35 mg/l lime\\
*a. 6.9 mg/l lime\\
b. 11.3 mg/l lime\\
c. 11.34 mg/l lime\\
d. 20.9 mg/l lime\\
  \item A dosage of 0.35 mg/l of copper sulfate pentahydrate is desired to control algae in an 8,850 acre-ft capacity reservoir. If the available copper is 25\%, how many pounds of copper sulfate pentahydrate are required?\\
Know: 1 ac-ft=43,560 ft$^{3}$\\
a. 4,495 lb copper sulfate\\
*b. 4,500 lb copper sulfate\\
c. 4,510 lb copper sulfate\\
d. 4,511 lb copper sulfate\\
  \item Determine the specific gravity (SG) for a solution that weighs $11.87 \mathrm{lb} / \mathrm{gal}$.\\
a. $1.38 \mathrm{SG}$\\
b. $1.40 \mathrm{SG}$\\
*c. $1.42 \mathrm{SG}$\\
d. $2.58 \mathrm{SG}$\\
  \item A filter is 24 ft by 28 ft. Calculate the filtration rate in gpm, if it receives a flow of $3,250 \mathrm{gpm}$.\\
a. $\quad 4.4 \mathrm{gpm} / \mathrm{ft}^{2}$\\
*b. $\quad 4.8 \mathrm{gpm} / \mathrm{ft}^{2}$\\
c. $\quad 5.0 \mathrm{gpm} / \mathrm{ft}^{2}$\\
d. $\quad 5.1 \mathrm{gpm} / \mathrm{ft}^{2}$\\
  \item Determine the backwash rate in $\mathrm{gpm} / \mathrm{ft}^{2}$ given the following:\\
\begin{itemize}
  \item Backwash flow of $13 \mathrm{cfs}\left(\mathrm{ft}^{3} / \mathrm{sec}\right)$\\
  \item Filter is 25 ft by 18.2 ft\\
\end{itemize}
a. 12.8 gpm/ft2\\
b. 12.9 gpm/ft2\\
*c. 13 gpm/ft2\\
d. 14 gpm/ft2\\
  \item Calculate the backwash pumping rate if a filter requires a backwash rate of $18 \mathrm{gpm} / \mathrm{ft}^{2}$ and the filter is 20.0 ft by 24.0 ft.\\
a. $8,600 \mathrm{gpm}$\\
*b. $8,640 \mathrm{gpm}$\\
c. $8,700 \mathrm{gpm}$\\
d. $8,780 \mathrm{gpm}$ \\
\item Which is the raw water alkalinity in mg/l as CaCO3 of a water sample with a beginning pH of 7.18, given the following parameters?\\
\begin{itemize}
  \item Sample size $=100 \mathrm{~mL}$\\
  \item Normality of the sulfuric acid $=0.02 \mathrm{~N}$\\
  \item Titrant used to pH of $4.6=11.4 \mathrm{~mL}$\\
\end{itemize}
a. 100 mg/l as CaCO3\\
b. 110 mg/l as CaCO3\\
*c. 114 mg/l as CaCO3\\
d. 120 mg/l as CaCO3\\
  \item A chemical metering pump is pumping 26.3 gallons per day (gal/day). How many $\mathrm{mL} / \mathrm{min}$ is this?\\
a. $\quad 47.6 \mathrm{~mL} / \mathrm{min}$\\
b. $\quad 60.8 \mathrm{~mL} / \mathrm{min}$\\
*c. $\quad 69.1 \mathrm{~mL} / \mathrm{min}$\\
d. $\quad 75.4 \mathrm{~mL} / \mathrm{min}$\\
  \item Find the number of $\mathrm{gal} / \mathrm{ft}^{3}$ of a solution, if it weighs $112.7 \mathrm{lb} / \mathrm{ft}^{3}$.\\
*a. $13.5 \mathrm{gal} / \mathrm{ft}^{3}$\\
b. $\quad 16.2 \mathrm{gal} / \mathrm{ft}^{3}$\\
c. $18.0 \mathrm{gal} / \mathrm{ft}^{3}$\\
d. $19.9 \mathrm{gal} / \mathrm{ft}^{3}$\\
  \item Find the drawdown of a well that has a specific yield of 28.4 , if the well yields 325 gpm.\\
a. 9.8 ft\\
*b. 11.4 ft\\
c. 12.9 ft\\
d. 14.1 ft\\
  \item A water treatment plant has an emergency shutdown. How many water supply hours are left in a 119.8-ft diameter tank given the following data?\\

\begin{itemize}
  \item Tank's water level =27.6 ft\\
  \item Water cannot go below 16.0 ft at any time to comply with fire control\\
  \item Water usage averages $483 \mathrm{gpm}$\\
*a. $\quad 33.7 \mathrm{hr}$\\
b. $\quad 35.0 \mathrm{hr}$\\
c. $35.4 \mathrm{hr}$\\
d. $36.2 \mathrm{hr}$\\
\end{itemize}

  \item A lime tank is conical at the bottom and cylindrical at the top. If the diameter of the cylinder is 14 ft with a depth of 24 ft and the cone depth is 12.5 ft, calculate the volume of the tank in cubic feet.\\
a. $\quad 3,700 \mathrm{ft}^{3}$\\
b. $\quad 4,000 \mathrm{ft}^{3}$\\
c. $4,200 \mathrm{ft}^{3}$\\
*d. $\quad 4,300 \mathrm{ft}^{3}$ 
  \item Flow through a channel 5.8 ft wide is $20.3 \mathrm{cfs}$. If the velocity is $1.4 \mathrm{ft} / \mathrm{sec}$, how deep is the water in the channel?\\
a. 2.3 ft\\
*b. 2.5 ft\\
c. 2.6 ft\\
d. 2.7 ft\\
  \item A lake is 107 ft deep. Which is the psi on the bottom?\\
a. $44.8 \mathrm{psi}$\\
b. $\quad 45.2 \mathrm{psi}$\\
c. $\quad 45.6 \mathrm{psi}$\\
*d. $46.3 \mathrm{psi}$\\
  \item How many gallons per day flow through a 2.18-mil gal capacity sedimentation basin, if the detention time is $2.79 \mathrm{hr}$ ?\\
a. $17,100,000 \mathrm{gal} /$ day\\
*b. $18,800,000 \mathrm{gal} /$ day\\
c. $19,700,000 \mathrm{gal} /$ day\\
d. $20,300,000 \mathrm{gal} / \mathrm{day}$\\
  \item An ion exchange softener is treating a flow rate of $245 \mathrm{gpm}$. Which is the operating time in hours, if the softener unit treats 434,000 gal before it requires regeneration?\\
a. $\quad 27.7 \mathrm{hr}$\\
b. $28.6 \mathrm{hr}$\\
*c. $29.5 \mathrm{hr}$\\
d. $30.1 \mathrm{hr}$\\
  \item Zinc orthophosphate (ZOP) is used at a treatment plant for corrosion control. The plant is treating 12.1 MGD with a dosage of 0.15 mg/l. Determine the feeder setting for ZOP in mL/min, if the specific gravity of the ZOP is 1.63 .\\
*a. 2.9 mL/min of ZOP\\
b. 3.2 mL/min of ZOP\\
c. 3.5 mL/min of ZOP\\
d. 3.9 mL/min of ZOP\\
  \item Determine the feed rate for alum in mL/min with the following conditions:\\
\begin{itemize}
  \item Plant flow= 30.9 MGD\\
  \item Alum dosage rate =10.4 mg/l\\
  \item Alum percentage 48.4 \%\\
  \item Alum specific gravity = 1.31\\
  \end{itemize}
a. $1,250 \mathrm{~mL} / \mathrm{min}$\\
b. $1,270 \mathrm{~mL} / \mathrm{min}$\\
*c. $1,330 \mathrm{~mL} / \mathrm{min}$\\
d. $1,410 \mathrm{~mL} / \mathrm{min}$ \\
\item A 1.50 -ft diameter pipe that is 1.62 miles long was disinfected with chlorine. If $47.2 \mathrm{lb}$ of chlorine were used, which was the dosage in mg/l ?\\
a. 25.0 mg/l\\
b. 30.0 mg/l\\
*c. 50.0 mg/l\\
d. 60.0 mg/l\\
  \item Calculate the specific gravity (SG) for an unknown liquid with a density of 87.6 lb/ft3.\\
*a. $1.40 \mathrm{SG}$\\
b. $1.43 \mathrm{SG}$\\
c. $1.51 \mathrm{SG}$\\
d. $1.62 \mathrm{SG}$\\
\end{enumerate}

\newpage
\section{TIII AND TIV MATH}

\begin{enumerate}
  \item The alum dosage for a plant with a flow of $26.5 \mathrm{cfs}$ is $655 \mathrm{~mL} / \mathrm{min}$. If the raw water flow rate is adjusted to $18.5 \mathrm{cfs}$, which should be the theoretical alum dosage in $\mathrm{mL} / \mathrm{min}$, if all water parameters remain the same?\\
a. $\quad 410 \mathrm{~mL} / \mathrm{min}$\\
b. $418 \mathrm{~mL} / \mathrm{min}$\\
c. $436 \mathrm{~mL} / \mathrm{min}$\\
*d. $457 \mathrm{~mL} / \mathrm{min}$\\
  \item Which is the percentage recovery for a reverse osmosis unit with a 4-2-1 arrangement given the following data?\\

\begin{itemize}
  \item Product flow is 570 gpm\\
  \item Feed flow is $1.03 \mathrm{mgd}$\\
*a. $79.7 \%$\\
b. $80.0 \%$\\
c. $80.2 \%$\\
d. $80.5 \%$\\
\end{itemize}

  \item How many $\mathrm{lb}$ /day of sodium fluorosilicate $\left(\mathrm{Na}_{2} \mathrm{SiF}_{6}\right)$ are required given the following parameters?\\

\begin{itemize}
  \item Flow rate is $1,750 \mathrm{gpm}$\\
  \item Fluoride desired is 1.20 mg/l\\
  \item Fluoride in raw water is 0.15 mg/l\\
  \item Sodium fluorosilicate is $98.1 \%$ pure\\
  \item Fluoride $(\mathrm{F})$ ion percent is $60.6 \%$\\
  \end{itemize}
a. $34 \mathrm{lb} /$ day, $\mathrm{F}$\\
*b. $37 \mathrm{lb} /$ day, $\mathrm{F}$\\
c. $\quad 42 \mathrm{lb} /$ day, $\mathrm{F}$\\
d. $48 \mathrm{lb} /$ day, $\mathrm{F}$\\


  \item An ion exchange softener is treating a flow rate of $125 \mathrm{gpm}$. Which is the operating time in hours if the softener unit treats 297,000 gallons before it requires regeneration?\\
a. $37 \mathrm{hrs}$\\
*b. $\quad 39.6 \mathrm{hrs}$\\
c. $40 \mathrm{hrs}$\\
d. $41.1 \mathrm{hrs}$ 
\item A water treatment plant has 8 filters with an average flow rate of $4.89 \mathrm{gpm} / \mathrm{ft}^{2}$. If the plant flow is $32.7 \mathrm{cfs}$, what is the filtration area of each filter?\\
*a. $\quad 375 \mathrm{ft}^{2} /$ filter\\
b. $\quad 398 \mathrm{ft}^{2} /$ filter\\
c. $\quad 400 \mathrm{ft}^{2} /$ filter\\
d. $\quad 410 \mathrm{ft}^{2} /$ filter\\
  \item Calculate the amount of iron removed in pounds per year from a water plant that treats an average of $20.2 \mathrm{mgd}$ if the average iron concentration is 0.52 mg/l and the removal efficiency is 84 \%.\\
a. $26,859 \mathrm{lb} / \mathrm{yr}$ of Fe removed\\
*b. $27,000 \mathrm{lb} / \mathrm{yr}$ of Fe removed\\
c. $31,975 \mathrm{lb} / \mathrm{yr}$ of Fe removed\\
d. $32,000 \mathrm{lb} / \mathrm{yr}$ of Fe removed\\
  \item Determine the percent mineral rejection from a reverse osmosis plant if the feedwater contains $1,230 mg/l TDS and the product water contains $135 mg/l TDS.\\
a. $88 \%$\\
*b. $89 \%$\\
c. $90 \%$\\
d. $91 \%$\\
  \item Calculate the log removal for a water treatment plant if the samples show a raw water coliform count of 295/100 $\mathrm{mL}$ (through extrapolation) and the finished water shows 2.0/100 mL.\\
a. $1.8 \log$ removal\\
b. 2 log removal\\
c. $2.1 \log$ removal\\
*d. $2.2 \log$ removal\\
  \item A 0.25 Normal solution of $\mathrm{H}_{3} \mathrm{PO}_{4}$ (phosphoric acid) is to be prepared. If 4.5 liters of solution is desired, how many grams of $\mathrm{H}_{3} \mathrm{PO}_{4}$ are required? The gram formula for $\mathrm{H}_{3} \mathrm{PO}_{4}$ is 98.00 . Give answer to nearest 100 th of a gram.\\
a. 8.17 grams\\
b. 24.50 grams\\
*c. $\quad 36.75$ grams\\
d. 42.31 grams\\
  \item A $2.00 \%$ stock polymer solution (20,000 ppm or 20,000 mg/L) is desired for performing a jar test. If the polymer has a specific gravity of 1.27 and is 84.5 \% polymer, how many milliliters are required to make exactly $1,000 \mathrm{~mL}$ stock solution?\\
a. $\quad 12.3 \mathrm{~mL}$ polymer\\
b. $\quad 15.3 \mathrm{~mL}$ polymer\\
*c. $\quad 18.6 \mathrm{~mL}$ polymer\\
d. $\quad 21.8 \mathrm{~mL}$ polymer \\
\item Calculate the theoretical detention time in hours for the following water treatment plant:\\
\begin{itemize}
  \item Flow rate of $12.2 \mathrm{mgd}$\\
  \item Four flocculation basins measuring: 45.0 ft by 10.0 ft by 11.0 ft in average depth each\\
  \item Sedimentation basin measuring: 285 ft by 65.0 ft by 11.4 ft in average depth\\
  \item Eight filters measuring: 35.0 ft by 28.0 ft by 12.3 ft in depth each\\
  \item Clear well averages 2.05 million gallons (mil gal)\\
\end{itemize}
a. $\quad 7.61 \mathrm{hr}$\\
b. $8.60 \mathrm{hr}$\\
c. $8.78 \mathrm{hr}$\\
*d. $8.85 \mathrm{hr}$\\
  \item Calculate the $\mathrm{CT}$ and inactivation ratio for a water treatment plant that has the following parameters; and does this treatment facility meet the CT?\\
\begin{itemize}
  \item Daily Parameters:\\
  \item Detention time $=83 \mathrm{~min}$\\
  \item $\mathrm{pH}=7.8$\\
  \item Lowest Temperature $=12^{\circ} \mathrm{C}$\\
  \item Lowest chlorine residual 0.60 mg/l\\
  \item A 1.0 log removal is required for this system\\
\end{itemize}
a. 0.9 inactivation ratio\\
*b. 1.2 inactivation ratio\\
c. $\quad 1.3$ inactivation ratio\\
d. 1.35 inactivation ratio\\
  \item Calculate the feed rate for fluorosilicic acid in $\mathrm{mL} / \mathrm{min}$ given the following data:\\
\begin{itemize}
  \item Flow rate is $11.8 \mathrm{mgd}$\\
  \item Fluoride desired is 1.20 mg/l\\
  \item Fluoride in raw water is 0.20 mg/l\\
  \item Treated with $20.5 \%$ solution of $\mathrm{H}_{2} \mathrm{SiF}_{6}$\\
  \item Fluoride ion percent is $79.0 \%$\\
  \item $\mathrm{H}_{2} \mathrm{SiF}_{6}$ weighs $9.8 \mathrm{lb} / \mathrm{gal}$\\
\end{itemize}
a. $30 \mathrm{~mL} / \mathrm{min}, \mathrm{H}_{3} \mathrm{SiF}_{6}$\\
b. $\quad 32 \mathrm{~mL} / \mathrm{min}, \mathrm{H}_{3} \mathrm{SiF}_{6}$\\
*c. $\quad 33 \mathrm{~mL} / \mathrm{min}, \mathrm{H}_{3} \mathrm{SiF}_{6}$\\
d. $35 \mathrm{~mL} / \mathrm{min}, \mathrm{H}_{3} \mathrm{SiF}_{6}$\\
\begin{center}
\begin{tabular}{|l|c|}
\hline\\
\multicolumn{2}{|c|}{MOLECULAR WEIGHTS OF CHEMICAL COMPOUNDS} \\\\
\hline\\
\multicolumn{1}{|c|}{COMPOUND} & MOLECULAR WEIGHT \\\\
\hline\\
Alkalinity, as $\mathrm{CaCO}_{3}$ & 100.1 \\\\
\hline\\
Carbon Dioxide, $\mathrm{CO}_{2}$ & 44.0 \\\\
\hline\\
Hardness, as $\mathrm{CaCO}_{3}$ & 100.1 \\\\
\hline\\
Hydrated Lime, $\mathrm{Ca}(\mathrm{OH})_{2}$ & 74.1 \\\\
\hline\\
Magnesium, $\mathrm{Mg}^{2+}$ & 24.3 \\\\
\hline\\
Magnesium $\mathrm{Hydroxide}_{\mathrm{Mg}(\mathrm{OH})_{2}}$ & 58.3 \\\\
\hline\\
Quicklime, $\mathrm{CaO}$ & 56.1 \\\\
\hline\\
Soda Ash, $\mathrm{Na}_{2} \mathrm{CO}_{3}$ & 106.0 \\\\
\hline\\
\end{tabular}
\end{center}
Use this table to solve problem 14.\\
  \item Determine the hydrated lime dose required in mg/l for water with the following characteristics:\\
\begin{center}
\begin{tabular}{|l|c|c|}
\hline\\
 & Source Water & Softened Water \\\\
\hline\\
Total Alkalinity, mg/L & 165 mg/l as CaCO3 & 39 mg/l \\\\
\hline\\
Total Hardness, mg/l & 248 mg/l as CaCO3 & 76 mg/l \\\\
\hline\\
CO2, mg/l & 13 mg/l & 0 mg/l \\\\
\hline\\
{Mg}$^{2+}$ & 21 mg/l & 7.8 mg/l \\\\
\hline\\
pH & 7.0 & 7.8 \\\\
\hline\\
Lime Purity & $92 \%$ &  \\\\
\hline\\
\end{tabular}
\end{center}
Use an excess lime dosage of $15 \%$ (115\% or 1.15 in decimal form)\\
*a. $190 \mathrm{mg} / \mathrm{L}, \mathrm{Ca}(\mathrm{OH})_{2}$\\
b. $194 \mathrm{mg} / \mathrm{L}, \mathrm{Ca}(\mathrm{OH})_{2}$\\
c. $\quad 195 \mathrm{mg} / \mathrm{L}, \mathrm{Ca}(\mathrm{OH})_{2}$\\
d. $200 \mathrm{mg} / \mathrm{L}, \mathrm{Ca}(\mathrm{OH})_{2}$\\
  \item What is the flow through a membrane unit in $\mathrm{gpd} / \mathrm{ft}^{2}$, if the water flux of the unit is $4.75 \times 10^{-4} \mathrm{gm} / \mathrm{cm}^{2} / \mathrm{s}$ ?\\
a. $10 \mathrm{gpd} / \mathrm{ft}^{2}$\\
*b. $\quad 10.1 \mathrm{gpd} / \mathrm{ft}^{2}$\\
c. $100 \mathrm{gpd} / \mathrm{ft}^{2}$\\
d. $101 \mathrm{gpd} / \mathrm{ft}^{2}$ \\
\item A conventional water treatment plant had to discontinue pre-chlorination, that is, no addition of chlorine to the flocculation basins and the sedimentation basin due to elevated trihalomethane levels. Consequently, the chlorine dose was increased before the filters and the clear well and a lithium chloride tracer study was performed. The plant requires a 1.5 log removal for Giardia cysts. Given the following parameters on the first day of this process change and referring to the CT values table in Appendix B, determine if this plant is in CT compliance:\\
\begin{center}
\begin{tabular}{|l|c|c|}
\hline\\
Unit Process or Piping & $\mathbf{T}^{\mathbf{1 0}}$ Value, Min & Lowest Chlorine Residual \\\\
\hline\\
Filtration & 12 & 0.45 mg/l \\\\
\hline\\
Piping (filter to clear well) & 4.5 & 0.40 mg/l \\\\
\hline\\
Clearwell & 51 & 1.20 mg/l \\\\
\hline\\
\end{tabular}
\end{center}
\begin{center}
\begin{tabular}{|l|c|c|c|}
\hline\\
Unit Process or Piping & Temperature & pH & CT Value, Tables \\\\
\hline\\
Filtration & 11 & 6.9 & 47.5 \\\\
\hline\\
Piping (filter to clear well) & 11 & 6.9 & 47.1 \\\\
\hline\\
Clear well & 12 & 7.6 & 45.5 \\\\
\hline\\
\end{tabular}
\end{center}
a. 0.9 inactivation ratio, plant is out of compliance\\
*b. 1.1 inactivation ratio, plant is in compliance\\
c. 1.13 inactivation ratio, plant is in compliance\\
d. 1.27 inactivation ratio, plant is in compliance\\
  \item A 5-min drawdown test result showed that $106 \mathrm{~mL}$ of a polymer aid was being used to help treat the raw water. The specific gravity (SG) of the polymer aid is 1.26. If the plant is treating $3,225 \mathrm{gpm}$, what is the polymer dosage in mg/l ? Give answer to three significant figures.\\
a. $\quad 1.74 mg/l polymer aid$\\
*b. $\quad 2.19 mg/l polymer aid$\\
c. $\quad 10.94 mg/l polymer aid$\\
d. $\quad 15.12 mg/l polymer aid$\\
  \item A lime tank is conical at the bottom and cylindrical at the top. If the diameter of the cylinder is 15 ft with a depth of 28 ft and the cone depth is 12 ft, what is the volume of the tank in cubic feet? Give answer to three significant figures.\\
a. $1,040 \mathrm{ft}^{3}$\\
b. $\quad 5,510 \mathrm{ft}^{3}$\\
*c. $\quad 5,650 \mathrm{ft}^{3}$\\
d. $\quad 7,060 \mathrm{ft}^{3}$\\
  \item A watershed, 158 square miles, receives an average of 22.6 inches of rain each year. The amount of rain collected for treatment is $6.75 \%$. How many million gallons (mil gal) of water are available per year for the small community that resides there?\\
a. 4,060 mil gal/year\\
*b. $4,190 \mathrm{mil} \mathrm{gal} / \mathrm{year}$\\
c. $25,400 \mathrm{mil} \mathrm{gal} /$ year\\
d. 50,300 mil gal/year 

  \item A softener unit has $118 \mathrm{ft}^{3}$ of resin with a capacity of 25.5 kilograins $/ \mathrm{ft}^{3}$. How many gallons of water will the unit treat, if the water contains 14.2 gpg?\\
a. 195,000 gal\\
b. 210,000 gal\\
*c. 211,900 gal\\
d. 212,500 gal\\
  \item A solution of lime needs to be prepared for a jar test. How many grams of quicklime, $\mathrm{CaO}$, would you mix with $1 \mathrm{~L}$ of water to make a $1.0 \%$ (Wt-volume) solution?\\
a. $0.1 \mathrm{~g}$ of $\mathrm{CaO}$\\
b. $1 \mathrm{~g}$ of $\mathrm{CaO}$\\
*c. $10 \mathrm{~g}$ of $\mathrm{CaO}$\\
d. $100 \mathrm{~g}$ of $\mathrm{CaO}$\\
  \item Calculate the percent removal across a settling basin and filter complex, if the raw water influent is $5.45 \mathrm{ntu}$ and the effluent (post filters) is $0.018 \mathrm{ntu}$. Give answer to three significant figures.\\
a. $98.4 \%$\\
b. $99.0 \%$\\
c. $99.3 \%$\\
*d. $99.7 \%$\\
  \item Determine the psi at the bottom of an alum storage tank if the level of the alum in the tank is 8.95 ft and the density of the alum is $11.32 \mathrm{lb} / \mathrm{gal}$.\\
a. $\quad 5.13 \mathrm{psi}$\\
*b. $\quad 5.26 \mathrm{psi}$\\
c. $\quad 5.37 \mathrm{psi}$\\
d. $\quad 5.41 \mathrm{psi}$\\
  \item Find the detention time in minutes for a clarifier that has a diameter of 152 ft and a water depth of 14.8 ft, if the flow rate is $4.25 \mathrm{mgd}$.\\
a. $650 \mathrm{~min}$\\
*b. $680 \mathrm{~min}$\\
c. $700 \mathrm{~min}$\\
d. $710 \mathrm{~min}$\\
  \item A 3-minute drawdown test used $191 \mathrm{~mL}$ of polymer for treating the raw water. The specific gravity of the polymer is 1.34 . Which is the polymer dosage in mg/l, if the plant is treating $3,280 \mathrm{gpm}$ ?\\
a. $\quad 6.63 mg/l$\\
b. $\quad 6.72 mg/l$\\
*c. $\quad 6.87 mg/l$\\
d. $\quad 6.99 mg/l$\\
  \item A polymer solution has a specific gravity of 1.35 and a concentration of $80 \%$. How many microliters are required to do a jar test, if he test uses 2-liter jars and the dosage needed is 6 mg/l ? Give result to nearest tenth of a microliter.t\\
*a. $\quad 11.1$ microliters\\
b. 11.3 microliters\\
c. 11.4 microliters\\
d. 11.7 microliters \\
\item Which should be the chemical feeder setting in lb/min if $12.5 \mathrm{mgd}$ is treated with 7.25 mg/l of soda ash?\\
a. $\quad 0.486 \mathrm{lb} / \mathrm{min}$\\
*b. $\quad 0.525 \mathrm{lb} / \mathrm{min}$\\
c. $\quad 0.548 \mathrm{lb} / \mathrm{min}$\\
d. $\quad 0.561 \mathrm{lb} / \mathrm{min}$\\
  \item How many pounds per day of $65 \%$ calcium hypochlorite are required for maintaining a 2.5 mg/l dosage for a $2,575 \mathrm{gpm}$ treatment plant?\\
a. $100 \mathrm{lb} /$ day\\
b. $110 \mathrm{lb} /$ day\\
*c. $120 \mathrm{lb} /$ day\\
d. $130 \mathrm{lb} /$ day\\
  \item Determine the chemical feeder setting in $\mathrm{mL} / \mathrm{min}$ for a polymer solution, if the desired dosage is 3.95 mg/l and the treatment plant is treating $10.0 \mathrm{mgd}$. The specific gravity of the polymer is 1.33 .\\
a. $\quad 70.1 \mathrm{~mL} / \mathrm{min}$\\
b. $\quad 73.9 \mathrm{~mL} / \mathrm{min}$\\
c. $\quad 76.2 \mathrm{~mL} / \mathrm{min}$\\
*d. $\quad 78.1 \mathrm{~mL} / \mathrm{min}$\\
  \item Four filters have a surface area of 450 ft each, measured to the nearest foot. Calculate the filtration rate in gpm, if the flow received is $21.5 \mathrm{ft}^{3} / \mathrm{s}$. Give answer to three significant figures.\\
a. $\quad 5.11 \mathrm{gpm} / \mathrm{ft}^{2}$\\
b. $\quad 5.21 \mathrm{gpm} / \mathrm{ft}^{2}$\\
*c. $\quad 5.36 \mathrm{gpm} / \mathrm{ft}^{2}$\\
d. $\quad 5.58 \mathrm{gpm} / \mathrm{ft}^{2}$\\
  \item Determine the hardness of a particular body of raw water in mg/l, if the hardness of a water sample is 19.4 grains per gallon (gpg).\\
*a. 332 mg/l\\
b. 339 mg/l\\
c. 342 mg/l\\
d. 350 mg/l\\
  \item The exchange capacity of a softener is 8,850,000 grains. The softener treats water with an average hardness of 347 mg/l. Which is the capacity of the softener in gallons?\\
a. 429,000 gal\\
*b. 437,000 gal\\
c. 444,000 gal\\
d. 449,000 gal\\
  \item Find the backwash rate in gpm per $\mathrm{ft}^{2}$, if a filter has an area of $620 \mathrm{ft}^{2}$ with a backwash rate of $13.5 \mathrm{cfs}$.\\
a. $\quad 9.1 \mathrm{gpm} / \mathrm{ft}^{2}$\\
b. $\quad 9.4 \mathrm{gpm} / \mathrm{ft}^{2}$\\
c. $\quad 9.6 \mathrm{gpm} / \mathrm{ft}^{2}$\\
*d. $\quad 9.8 \mathrm{gpm} / \mathrm{ft}^{2}$ 
  \item The level in a storage tank drops 4.25 ft in exactly 12 hours. If the tank has a diameter of 50.0 ft and the plant is producing $2.95 \mathrm{mgd}$, which is the average discharge rate of the treated water discharge pumps in gallons per minute?\\
a. 2,090 gpm\\
b. 2,100 gpm\\
c. 2,120 gpm\\
*d. 2,140 gpm\\
  \item Ten filters have a surface area of 480 ft each. Calculate the filtration rate in gpm, if the total flow through the filters is 16.5 cubic feet per second.\\
a. $\quad 1.2 \mathrm{gpm} / \mathrm{ft}^{2}$\\
b. $\quad 1.3 \mathrm{gpm} / \mathrm{ft}^{2}$\\
*c. $\quad 1.5 \mathrm{gpm} / \mathrm{ft}^{2}$\\
d. $\quad 1.9 \mathrm{gpm} / \mathrm{ft}^{2}$\\
  \item Determine the amount of iron removed per year, if the iron concentration is 0.21 mg/l, the plant treats an average of $14.1 \mathrm{mgd}$, and the removal efficiency is $95.7 \%$ (0.957).\\
a. $8,000 \mathrm{lb} / \mathrm{yr}$\\
b. $8,200 \mathrm{lb} / \mathrm{yr}$\\
*c. $8,600 \mathrm{lb} / \mathrm{yr}$\\
d. $9,000 \mathrm{lb} / \mathrm{yr}$\\
\end{enumerate}


\newpage
\section{DI AND DII MATH}
\begin{enumerate}
  \item A well yields 2,840 gallons in exactly 20 minutes. What is the well yield in gpm?\\
a. $140 \mathrm{gpm}$\\
*b. $142 \mathrm{gpm}$\\
c. $145 \mathrm{gpm}$\\
d. $150 \mathrm{gpm}$\\
  \item Convert 37.4 degrees Fahrenheit to degrees Celsius.\\
*a. $3.0^{\circ} \mathrm{C}$\\
b. $\quad 5.3^{\circ} \mathrm{C}$\\
c. $\quad 7.9^{\circ} \mathrm{C}$\\
d. $\quad 9.7^{\circ} \mathrm{C}$\\
  \item What is the area of a circular tank pad in $\mathrm{ft}^{2}$, if it has a diameter of 102 ft ?\\
a. $\quad 6,160 \mathrm{ft}^{2}$\\
b. $\quad 6,167 \mathrm{ft}^{2}$\\
*c. $8,170 \mathrm{ft}^{2}$\\
d. $8,200 \mathrm{ft}^{2}$\\
  \item What is the pressure 1.85 feet from the bottom of a water storage tank if the water level is 28.7 feet?\\
*a. $\quad 11.6 \mathrm{psi}$\\
b. $\quad 12.4 \mathrm{psi}$\\
c. $\quad 62.0 \mathrm{psi}$\\
d. $\quad 66.3 \mathrm{psi}$\\
  \item Calculate the well yield in gpm, given a drawdown of 14.1 ft and a specific yield of 31 gpm / ft.\\
a. $\quad 2.2 \mathrm{gpm}$\\
b. $\quad 7.3 \mathrm{gpm}$\\
c. $\quad 45.1 \mathrm{gpm}$\\
*d. $440 \mathrm{gpm}$\\
  \item How many gallons are in a pipe that is 18.0 in. in diameter and 1,165 ft long?\\
a. 2,060 gal\\
b. 10,300 gal\\
*c. 15,400 gal\\
d. 17,200 gal\\
\item A water tank with a capacity of 5.75 million gallons (mil gal) is being filled at a rate of $2,105 \mathrm{gpm}$. How many hours will it take to fill the tank?\\
a. $\quad 31.6 \mathrm{hr}$\\
b. $37.8 \mathrm{hr}$\\
c. $42.9 \mathrm{hr}$\\
*d. $45.5 \mathrm{hr}$\\
  \item Determine the detention time in hours for the following water treatment system:\\
\begin{itemize}
  \item Distribution pipe from water plant to storage tank is 549 ft in length and $14 \mathrm{in}$. in diameter\\
  \item Storage tank averages 2,310,000 gal of water at any given time\\
  \item Flow through system is $6.72 \mathrm{mgd}$\\
\end{itemize}
a. 7.2 hr\\
b. 7.4 hr\\
c. 8.0 hr\\
*d. 8.3 hr\\
  \item Convert 28.7 cubic feet per second (cfs) to gallons per minute (gpm).\\
a. $12,477 \mathrm{gpm}$\\
b. $12,700 \mathrm{gpm}$\\
c. $12,880 \mathrm{gpm}$\\
*d. $12,900 \mathrm{gpm}$\\
  \item Convert $16,912,000$ liters to acre-feet.\\
*a. 13.7 acre-ft\\
b. 41.5 acre-ft\\
c. 51.9 acre-ft\\
d. 767 acre-ft\\
  \item Convert -22.6Deg. C to degrees Fahrenheit.\\
a. -4.6Deg. F\\
*b. -8.7Deg. F\\
c. -11.8Deg. F\\
d. -12.8Deg. F\\
  \item A sodium hypochlorite solution contains $11.3 \%$ hypochlorite. Calculate the mg/l hypochlorite in the solution.\\
a. 11.3 mg/l sodium hypochlorite\\
b. 1,130 mg/l sodium hypochlorite\\
c. 11,300 mg/l sodium hypochlorite\\
*d. 113,000 mg/l sodium hypochlorite\\
  \item Records for a pump show that on June 1st at exactly 9:00 a.m. the number of pumped gallons was 71,576,344 and on July 1st at exactly 9:00 a.m. it was 72,487,008 gallons. Determine the average gallons pumped per day (gal/day) for this month to the nearest gallon.\\
a. $18,605 \mathrm{gal} / \mathrm{day}$\\
b. $25,875 \mathrm{gal} / \mathrm{day}$\\
*c. $30,355 \mathrm{gal} / \mathrm{day}$\\
d. $34,325 \mathrm{gal} / \mathrm{day}$ \\
\item If chlorine is being fed at a rate of $260 \mathrm{lb} /$ day for a flow rate of $23 \mathrm{cfs}$, which should be the adjustment on the chlorinator when the flow rate is decreased to $16 \mathrm{cfs}$, if all other water parameters remain the same?\\
a. $160 \mathrm{lb} /$ day\\
*b. $180 \mathrm{lb} /$ day\\
c. $310 \mathrm{lb} /$ day\\
d. $370 \mathrm{lb} /$ day\\
  \item Calculate the diameter of a clarifier with a circumference of 215 ft.\\
a. 34.8 ft\\
b. 56.7 ft\\
*c. 68.5 ft\\
d. 76.2 ft\\
  \item Determine the depth of water in a reservoir, if the psi is 31.9 .\\
a. 13.8 ft deep\\
b. 24.9 ft deep\\
c. 45.6 ft deep\\
*d. 73.7 ft deep\\
  \item Calculate the area of a tank, if the tank's radius is 39.8 ft.\\
*a. $4,970 \mathrm{ft}^{2}$\\
b. $\quad 5,670 \mathrm{ft}^{2}$\\
c. $\quad 7,820 \mathrm{ft}^{2}$\\
d. $9,940 \mathrm{ft}^{2}$\\
  \item Determine the specific gravity (SG) of an unknown liquid, if the density of the liquid is $70.9 \mathrm{lb} / \mathrm{ft}^{3}$.\\
a. 1.05 \\
*b. 1.14 \\
c. 1.18 \\
d. 1.21 \\
  \item A water treatment plant is feeding an average of $295 \mathrm{lb} /$ day of chlorine. If the dosage is 2.25 mg/l, which is the number of millions of gallons per day (mgd) being treated?\\
*a. $\quad 15.7 \mathrm{mgd}$\\
b. $\quad 35.1 \mathrm{mgd}$\\
c. $\quad 58.3 \mathrm{mgd}$\\
d. $\quad 79.6 \mathrm{mgd}$\\
  \item How many gallons of a sodium hypochlorite solution that contains $12.1 \%$ available chlorine are needed to disinfect a 1.5 -ft diameter pipeline that is $283 ft long, if the dosage required is $50.0 mg/l ? Assume the sodium hypochlorite is $9.92 \mathrm{lb} / \mathrm{gal}$.\\
a. 0.87 gal sodium hypochlorite\\
b. 1.0 gal sodium hypochlorite\\
*c. 1.3 gal sodium hypochlorite\\
d. 1.5 gal sodium hypochlorite \\
\item A water treatment plant is treating 16.4 million gallons per day (mgd). If the chlorine feed rate is $415 \mathrm{lb} / \mathrm{day}$, which is the chlorine dosage in mg/l ?\\
*a. 3.03 mg/l\\
b. 3.38 mg/l\\
c. 3.43 mg/l\\
d. 3.67 mg/l\\
  \item A 1.65-million gallon (mil gal) storage tank needs to be disinfected with a sodium hypochlorite solution that has $11.8 \%$ available chlorine. The tank is to be filled at $10 \%$ capacity, and the initial chlorine dosage required is 50.0 mg/l. How many gallons of sodium hypochlorite will be needed, if it weighs $9.84 \mathrm{lb} / \mathrm{gal}$ ?\\
a. 50 gal sodium hypochlorite\\
b. 53 gal sodium hypochlorite\\
*c. 59 gal sodium hypochlorite\\
d. 63 gal sodium hypochlorite\\
  \item How many pounds of a calcium hypochlorite that contains $64.3 \%$ available chlorine are needed to disinfect a water main that is $24 \mathrm{in}$. in diameter, if the pipeline is 781 ft long and the dosage required is 50.0 mg/l ?\\
a. $\quad 5.95 \mathrm{lb}$ calcium hypochlorite\\
b. $\quad 8.25 \mathrm{lb}$ calcium hypochlorite\\
*c. $\quad 11.9 \mathrm{lb}$ calcium hypochlorite\\
d. $\quad 13.8 \mathrm{lb}$ calcium hypochlorite\\
  \item A well is pumping water at a rate of $428 \mathrm{gpm}$. Which should be the setting on a chlorinator in pounds per day, if the dosage desired is 1.20 mg/l and the chlorine demand is 3.85 mg/l ?\\
a. $\quad 19.8 \mathrm{lb} /$ day of chlorine\\
b. $20.6 \mathrm{lb} /$ day of chlorine\\
c. $\quad 23.7 \mathrm{lb} /$ day of chlorine\\
*d. $26.0 \mathrm{lb} /$ day of chlorine\\
  \item Convert 48.1 million gallons a day (mgd) to cubic feet per second (cfs).\\
a. $\quad 68.7 \mathrm{cfs}$\\
*b. $74.4 \mathrm{cfs}$\\
c. $\quad 79.1 \mathrm{cfs}$\\
d. $82.0 \mathrm{cfs}$\\
  \item Convert 184 gpm to liters per second (L/s).\\
a. $\quad 10.3 \mathrm{~L} / \mathrm{s}$\\
b. $\quad 11.1 \mathrm{~L} / \mathrm{s}$\\
*c. $\quad 11.6 \mathrm{~L} / \mathrm{s}$\\
d. $12.3 \mathrm{~L} / \mathrm{s}$ 

\item Which is the average turbidity in ntu at the end of a sedimentation basin given the following data?\\
\begin{center}
\begin{tabular}{|c|c|c|c|c|c|c|}
\hline\\
1 & 2 & 3 & 4 & 5 & 6 & 7 \\\\
\hline\\
$1.08 \mathrm{ntu}$ & $0.98 \mathrm{ntu}$ & $0.94 \mathrm{ntu}$ & $0.88 \mathrm{ntu}$ & $0.96 \mathrm{ntu}$ & $1.03 \mathrm{ntu}$ & $1.25 \mathrm{ntu}$ \\\\
\hline\\
\end{tabular}
\end{center}
a. $\quad 1.00 \mathrm{ntu}$\\
b. $\quad 1.01 \mathrm{ntu}$\\
*c. $\quad 1.02 \mathrm{ntu}$\\
d. $1.03 \mathrm{ntu}$\\
  \item If 288 is $70.3 \%$, how much is $100 \%$ ?\\
*a. 410\\
b. 412\\
c. 415\\
d. 418\\
  \item The iron $(\mathrm{Fe})$ content of a water source averages 0.81 mg/l iron. Which is the percent removal, if the treated water averages 0.01 mg/l iron?\\
a. $96 \%$ Fe removal efficiency\\
b. $97 \%$ Fe removal efficiency\\
c. $98 \%$ Fe removal efficiency\\
*d. $99 \%$ Fe removal efficiency\\
  \item Which will be the percent of soda ash in the resulting slurry, if 28.2 pounds of soda ash are mixed with exactly 100.0 gallons of water?\\
*a. $\quad 3.27 \%$ soda ash slurry\\
b. $3.38 \%$ soda ash slurry\\
c. $3.45 \%$ soda ash slurry\\
d. $3.54 \%$ soda ash slurry\\
  \item Which is the exposed exterior surface area of a ground-level storage tank that is 24.0 ft high and has a diameter of 80.1 ft ? Assume top is flat.\\
a. $10,800 \mathrm{ft}^{2}$\\
b. $10,900 \mathrm{ft}^{2}$\\
c. $11,000 \mathrm{ft}^{2}$\\
*d. $11,100 \mathrm{ft}^{2}$\\
  \item A pipe is 1.43 miles long and has an inner diameter of 18.0 inches. How many gallons are in the pipeline if it is full?\\
a. 66,500 gal\\
b. 79,200 gal\\
*c. 99,800 gal\\
d. 104,000 gal\\
  \item A storage tank has a 60.0-ft radius and averages 25.5 ft in water depth. Calculate the average detention time in hours for this storage tank, if flow through the tank averages 2.91 MGD during a particular month in question.\\
a. $17.5 \mathrm{hr}$\\
*b. $17.8 \mathrm{hr}$\\
c. $18.6 \mathrm{hr}$\\
d. $19.8 \mathrm{hr}$ \\
\item If the pressure head on a fire hydrant is 134 ft, which is the pressure in psi?\\
a. $50 \mathrm{psi}$\\
b. $52 \mathrm{psi}$\\
c. $54 \mathrm{psi}$\\
*d. $58 \mathrm{psi}$\\
  \item A meter indicates the water flow from a fire hydrant is $5.5 \mathrm{ft}^{3} / \mathrm{min}$. How many gallons will flow from the hydrant in 20 minutes?\\
*a. $820 \mathrm{gal}$\\
b. $850 \mathrm{gal}$\\
c. $880 \mathrm{gal}$\\
d. $920 \mathrm{gal}$\\
  \item A polymer weighs $8.25 \mathrm{lb}$ and occupies 3.150 liters. Which is the density of the polymer in $\mathrm{g} / \mathrm{cm}^{3}$ ?\\
a. $\quad 1.18 \mathrm{~g} / \mathrm{cm}^{3}$\\
*b. $\quad 1.19 \mathrm{~g} / \mathrm{cm}^{3}$\\
c. $\quad 1.20 \mathrm{~g} / \mathrm{cm}^{3}$\\
d. $\quad 1.21 \mathrm{~g} / \mathrm{cm}^{3}$\\
  \item Determine the percent accuracy for a meter being tested, if it reads 245.7 cubic feet and the volumetric tank used to measure the water that flowed through the meter indicates the actual volume as 1,863 gallons.\\
a. $97.8 \%$ meter efficiency\\
*b. $98.6 \%$ meter efficiency\\
c. $\quad 99.0 \%$ meter efficiency\\
d. $99.3 \%$ meter efficiency\\
  \item Which is the chlorine dosage at a water treatment plant, if the chlorinator is set on $320 \mathrm{lb} /$ day and the plant is treating $11.6 \mathrm{mgd}$ ?\\
a. 2.8 mg/l\\
b. 3.0 mg/l\\
*c. 3.3 mg/l\\
d. 3.7 mg/l\\
  \item A 1.75-mil gal storage tank needs to be disinfected with a sodium hypochlorite solution that contains $12.0 \%$ available chlorine and weighs $8.97 \mathrm{lb} / \mathrm{gal}$. If the chlorine dosage is to be 50.0 mg/l, how many gallons of sodium hypochlorite are required?\\
*a. 678 gal\\
b. 729 gal\\
c. 750 gal\\
d. 791 gal\\
  \item A 24.0-in. pipeline, $427 ft long, was disinfected with calcium hypochlorite tablets with $65.0 \%$ available chlorine. Determine the chlorine dosage in $mg/l, if $7.0 \mathrm{lb}$ of calcium hypochlorite was used. Assume that the hypochlorite is so diluted that it weighs 8.34 lb/gal.\\
a. 25 mg/l chlorine\\
b. 39 mg/l chlorine\\
c. 43 mg/l chlorine\\
*d. 54 mg/l chlorine \\
\item Water from a well is treated with a sodium hypochlorite solution that contains $10.3 \%$ available chlorine and weighs $8.95 \mathrm{lb} / \mathrm{gal}$. The well is pumping water at $260 \mathrm{gpm}$. Calculate the chlorine dosage, if the chlorinator is pumping at a rate of 95 liters/day.\\
a. 5.6 mg/l sodium hypochlorite\\
b. 6.3 mg/l sodium hypochlorite\\
*c. 7.4 mg/l sodium hypochlorite\\
d. 8.8 mg/l sodium hypochlorite\\
  \item Which should be the setting on a chlorinator in pounds per day, if the dosage desired is 1.75 mg/l, the chlorine demand averages 2.45 mg/l, and the pumping rate from the well is 208 gpm?\\
*a. $\quad 10.5 \mathrm{lb} /$ day chlorine\\
b. $\quad 11.2 \mathrm{lb} /$ day chlorine\\
c. $\quad 12.0 \mathrm{lb} /$ day chlorine\\
d. $\quad 13.1 \mathrm{lb} /$ day chlorine\\
  \item What is the maximum pumping rate (in gpm) of a pump that is producing 15 water horsepower against a head of 65 ft ?\\
a. $115 \mathrm{gpm}$\\
*b. $910 \mathrm{gpm}$\\
c. $17,000 \mathrm{gpm}$\\
d. $63,000 \mathrm{gpm}$\\
  \item A water plant serves 23,210 people. If it treats a yearly average of $2.98 \mathrm{mgd}$, what are the gallons per capita per day (gpcd)? Note: A capita = 1 person.\\
a. 115 gpcd\\
b. 120 gpcd\\
c. 122 gpcd\\
*d. 128 gpcd\\
\end{enumerate}


\section{DIII AND DIV MATH}

\begin{enumerate}
  \item What is the velocity of flow in feet per second for a 6.0-in. diameter pipe, if it delivers $122 \mathrm{gpm}$ ? Assume pipe is full.\\
a. $\quad 1.3 \mathrm{ft} / \mathrm{sec}$\\
b. $\quad 1.35 \mathrm{ft} / \mathrm{sec}$\\
c. $\quad 1.38 \mathrm{ft} / \mathrm{sec}$\\
*d. $\quad 1.4 \mathrm{ft} / \mathrm{sec}$\\
  \item A small cylinder on a hydraulic jack is $10 \mathrm{in}$. in diameter. A force of $130 \mathrm{lb}$ is applied to the small cylinder. If the diameter of the large cylinder is 2.5 ft, what is the total lifting force?\\
a. $1,170 \mathrm{lb}$\\
*b. $1,200 \mathrm{lb}$\\
c. $1,250 \mathrm{lb}$\\
d. $1,300 \mathrm{lb}$\\
  \item A 2.0-ft diameter pipe that is 2.45 miles long was disinfected with chlorine. If $126.9 \mathrm{lb}$ of chlorine were used, what was the initial dosage in mg/l ?\\
a. 25 mg/l\\
b. 40 mg/l\\
*c. 50 mg/l\\
d. 60 mg/l\\
  \item What is the motor horsepower ( $\mathrm{mhp})$, if 200 horsepower $(\mathrm{hp})$ is required to run a pump with a motor efficiency (Effic.) of $88 \%$ and a pump efficiency of $74 \%$ ? Note: The $200 \mathrm{hp}$ in this problem is called the water horsepower (whp). The whp is the actual energy (horsepower) available to pump water. Give results to two significant figures.\\
a. $130 \mathrm{mhp}$\\
b. $180 \mathrm{mhp}$\\
c. $200 \mathrm{mhp}$\\
*d. $310 \mathrm{mhp}$\\
  \item What is the bowl horsepower (bhp) for a vertical turbine pump given the following parameters?\\
\begin{itemize}
  \item Pumping rate =385 gpm\\
  \item Bowl head = 215 feet\\
  \item Bowl efficiency =81 \%\\
  \end{itemize}
a. $17 \mathrm{bhp}$\\
b. $20 \mathrm{bhp}$\\
*c. $26 \mathrm{bhp}$\\
d. $33 \mathrm{bhp}$\\
\item Water is flowing at a velocity of $1.3 \mathrm{ft} / \mathrm{sec}$ in a 4.0 -in. diameter pipe. If the pipe changes from the 4.0-inch to a 3.0-in. pipe, what will the velocity be in the 3.0-in. pipe?\\
a. $\quad 0.73 \mathrm{ft} / \mathrm{sec}$\\
b. $\quad 1.28 \mathrm{ft} / \mathrm{sec}$\\
*c. $\quad 2.3 \mathrm{ft} / \mathrm{sec}$\\
d. $\quad 2.6 \mathrm{ft} / \mathrm{sec}$\\

  \item The level in a storage tank drops 2.3 ft in exactly $18 \mathrm{hr}$. If the tank has a diameter of 120 ft and the plant is producing $4.75 \mathrm{mgd}$, what is the average discharge rate of the three treated water discharge pumps in gpm?\\
a. $3,479 \mathrm{gpm}$\\
*b. $3,500 \mathrm{gpm}$\\
c. $4,578 \mathrm{gpm}$\\
d. $4,600 \mathrm{gpm}$\\
  \item How many gallons of a $12.5 \%$ sodium hypochlorite solution $(9.34 \mathrm{lb} / \mathrm{gal})$ are required to make exactly 1,000 gal of a 50 mg/l solution?\\
a. 0.3 gal\\
*b. 0.4 gal\\
c. 0.43 gal\\
d. 0.63 gal\\
  \item What percent hypochlorite solution would result, if 350 gal of an $11 \%$ solution were mixed with 225 gal of a $5.8 \%$ solution? Assume both solutions have the same density.\\
a. $8.9 \%$ final solution\\
*b. $9.0 \%$ final solution\\
c. $9.1 \%$ final solution\\
d. $9.12 \%$ final solution\\
  \item What water horsepower (whp) is required for a pump that delivers 650 gpm to a total head of 195 feet?\\
a. 25 whp\\
b. 30 whp\\
*c. 32 whp\\
d. 40 whp\\
  \item Determine the percentage strength of a solution mixture, if $875 \mathrm{lb}$ of a $49.5 \%$ strength solution is mixed with $293 \mathrm{lb}$ of a $17.2 \%$ strength solution.\\
*a. $41.4 \%$\\
b. $42.4 \%$\\
c. $43.0 \%$\\
d. $43.1 \%$ \\
\item How many fluid ounces (oz) of sodium hypochlorite (10.5\% available chlorine and $9.10 \mathrm{lb} / \mathrm{gal}$ ) are required to disinfect a well with the following parameters?\\
\begin{itemize}
  \item Depth of well is 287 ft\\
  \item 12-in. diameter well casing extends down to 100.0 ft\\
  \item The remainder is a 10.0-in. diameter casing\\
  \item The residual desired dose is 50.0 mg/l\\
  \item The depth to water is 168.4 ft\\
  \item The chlorine demand is 4.7 mg/l\\
  \end{itemize}
a. $21 \mathrm{oz}$\\
b. $25 \mathrm{oz}$\\
c. $27 \mathrm{oz}$\\
*d. $30 \mathrm{oz}$\\
  \item Determine the cost to the nearest cent to operate a $300 \mathrm{Hp}$ motor for one month (assume 30 days), if it runs an average of $4.2 \mathrm{hr} /$ day, is $82 \%$ efficient, and the electrical costs are $\$ 0.041$ per $\mathrm{kW}$.\\
a. $\$ 948.04$\\
b. $\$ 970.92$\\
*c. $\$ 1,156.15$\\
d. $\$ 1,184.05$\\
  \item A storage tank has a level capacity of 24.50 ft. Currently the water level is 16.55 ft in the tank. Calculate the SCADA reading on the board in $\mathrm{mA}$ for a $4 \mathrm{~mA}$ to $20 \mathrm{~mA}$ signal.\\
a. $\quad 13.5 \mathrm{~mA}$\\
b. $\quad 13.51 \mathrm{~mA}$\\
c. $\quad 14.8 \mathrm{~mA}$\\
*d. $\quad 14.81 \mathrm{~mA}$\\
  \item A pipe that is 3,270 ft long has a diameter of $14.0 \mathrm{in}$. for two-thirds of its length and 10.0 in. for the remaining one-third. How many gallons will it take to completely fill this pipe?\\
a. 17,598 gal\\
*b. 21,900 gal\\
c. 52,670 gal\\
d. 87,500 gal\\
  \item How many pounds of lime must be added to exactly 200 gal of water to produce a lime slurry of $15 \%$ ?\\
a. $220 \mathrm{lb}$\\
*b. $290 \mathrm{lb}$\\
c. $340 \mathrm{lb}$\\
d. $420 \mathrm{lb}$ \\
\item Determine the volume in gallons of a trapezoid-shaped canal that has the following dimensions:\\
\begin{itemize}
  \item $\quad$ Length =6,091 ft\\
  \item Height =4.10 ft\\
  \item Bottom width $=5.85 \mathrm{ft}\left(\mathrm{b}_{1}\right)$\\
  \item $\quad$ Top width $=10.6 \mathrm{ft}\left(\mathrm{b}_{2}\right)$\\
  \end{itemize}
a. 1,270,000 gal\\
b. 1,330,000 gal\\
c. 1,480,000 gal\\
*d. 1,540,000 gal\\
  \item Calculate the detention time to the nearest $100 \mathrm{hr}$ for the following system:\\
\begin{itemize}
  \item The clear well is 308 ft long, 118 ft wide, and has an average water depth of 12.85 ft\\
  \item Distribution pipe from clear well to storage tank is 1.34 miles long and has a diameter of 2.00 ft\\
  \item The storage tank has a diameter of 99.8 ft and averages a height of 26.48 ft of water\\
  \item The water production for the year averaged $30.02 \mathrm{mgd}$\\
  \end{itemize}
a. 4.02 hr\\
*b. 4.16 hr\\
c. 4.22 hr\\
d. 4.29 hr\\
  \item A well has a depth of 276.5 ft. If the depth to water is 153.8 ft, which is the pressure in psi 5.0 ft above the bottom? Disregard the additional atmospheric pressure in the well.\\
a. 42 psi\\
b. 46 psi\\
c. 48 psi\\
*d. 51 psi\\
  \item How many gallons per minute should a flowmeter register, if a 10.0-in. diameter main is to be flushed at $5.10 \mathrm{ft} / \mathrm{sec}$ ?\\
a. 1,050 gpm\\
b. $1,100 \mathrm{gpm}$\\
*c. $1,250 \mathrm{gpm}$\\
d. $1,350 \mathrm{gpm}$\\
  \item Water is flowing at a velocity of $2.0 \mathrm{ft} / \mathrm{sec}$ in an 8.0-in. diameter pipe. If the pipe changes from the 8.0-in. to a 10.0 -in. pipe, the velocity in the 10.0 -in. pipe will be\\
*a. $\quad 1.3 \mathrm{ft} / \mathrm{sec}$\\
b. $\quad 1.5 \mathrm{ft} / \mathrm{sec}$\\
c. $\quad 1.7 \mathrm{ft} / \mathrm{sec}$\\
d. $\quad 1.8 \mathrm{ft} / \mathrm{sec}$ \\
\item An 18-in. diameter distribution pipe delivers 988,000 gallons in $24 \mathrm{hr}$. Which is the average flow during the $24 \mathrm{hr}$ in $\mathrm{ft} / \mathrm{sec}$ ?\\
a. $\quad 0.60 \mathrm{ft} / \mathrm{sec}$\\
b. $\quad 0.73 \mathrm{ft} / \mathrm{sec}$\\
*c. $\quad 0.87 \mathrm{ft} / \mathrm{sec}$\\
d. $\quad 0.94 \mathrm{ft} / \mathrm{sec}$\\
  \item A $64.5 \%$ calcium hypochlorite solution was used to treat 10.6 mil gal. The tank containing the hypochlorite solution is 6.0 ft in diameter. If the tank dropped $8.03 \mathrm{in}$. during the time the 10.6 mil gal were treated, which must have been the chlorine dosage in mg/l ?\\
a. 5.78 mg/l\\
b. 6.33 mg/l\\
c. 7.25 mg/l\\
*d. 8.61 mg/l\\
  \item A well that is 227 ft deep and $12 \mathrm{in}$. in diameter requires disinfection. Depth to water from the casing top is 143 ft. If the desired dose is 50.0 mg/l, how many gallons of sodium hypochlorite (12.5\% available chlorine) are required? Note: The specific gravity of the sodium hypochlorite is 1.15 or $9.59 \mathrm{lb} / \mathrm{gal}$.\\
*a. 0.17 gal\\
b. 0.19 gal\\
c. 0.21 gal\\
d. 0.25 gal \\
  \item A pipe that is 2.50 ft in diameter and 1,058 ft long is to be disinfected with $64.5 \%$ calcium hypochlorite tablets. If the desired dose is 25.0 mg/l, how many pounds of calcium hypochlorite are required?\\
a. 10.1 lb\\
b. 11.7 lb\\
*c. 12.6 lb\\
d. 13.2 lb\\
  \item A tank 84.0 ft in diameter and 24.25 ft high at the overflow requires disinfection. How much 12.5 \% sodium hypochlorite that is $9.59 \mathrm{lb} / \mathrm{gal}$ will be required for a dosage of 50.0 mg/l ?\\
a. 310 gal\\
*b. 350 gal\\
c. 380 gal\\
d. 410 gal\\
  \item How many calcium hypochlorite tablets, each weighing $0.45 \mathrm{lb}$, are needed to disinfect a water main, given the following information:\\

\begin{itemize}
  \item Length of pipe =513 ft\\
  \item $\quad$ Pipe diameter =2.50 ft\\
  \item Calcium hypochlorite =64.0 \% available chlorine\\
  \item Dosage required =25.0 mg/l\\
  \end{itemize}
a. 10 tablets\\
b. 12 tablets\\
*c. 14 tablets\\
d. 16 tablets \\
\item A well that is 210 ft in depth and $14.0 \mathrm{in}$. in diameter requires disinfection. The depth to water from top of casing is 91 ft. If the desired dose is 50.0 mg/l, which is the number of pounds and ounces of sodium hypochlorite (12.5\% available chlorine) required? Assume the sodium hypochlorite solution is $9.59 \mathrm{lb} / \mathrm{gal}$.\\
*a. $43 \mathrm{oz}$ of $\mathrm{NaOCl}$\\
b. $45 \mathrm{oz}$ of $\mathrm{NaOCl}$\\
c. $49 \mathrm{oz}$ of $\mathrm{NaOCl}$\\
d. $54 \mathrm{oz}$ of $\mathrm{NaOCl}$\\
  \item Soda ash slurry is being added to water being released from a clear well to the distribution system to raise the pH. If the amount of soda ash being added averages 124.5 grams per minute for that day and the water leaving the distribution system averages 3,075 gpm for that day, which must have been the soda ash dosage in mg/l ?\\
a. 8.18 mg/l\\
b. 9.70 mg/l\\
*c. 10.69 mg/l\\
d. 12.47 mg/l\\
  \item The level in a clearwell tank drops 7.08 ft in exactly $12.0 \mathrm{hr}$. If the tank has a diameter of 149.8 ft and the plant is producing $4.75 \mathrm{mgd}$, calculate the average discharge rate for each pump of the four same capacity treated water discharge pumps in gallons per minute.\\
*a. $1,150 \mathrm{gpm}$\\
b. $1200 \mathrm{gpm}$\\
c. $1,250 \mathrm{gpm}$\\
d. $1,680 \mathrm{gpm}$\\
  \item Determine the horsepower (hp) required for a clear well water pump that needs to pump water to a storage tank given the following parameters:\\
\begin{itemize}
  \item Elevation of clear well water pump 170.84 ft\\
  \item Elevation of water storage tank 478.16 ft\\
  \item Length of pipeline from clear well water pump to storage tank = 2,107 ft\\
  \item Pump above clear well (suction lift) 2.5 ft\\
  \item Friction loss in pipeline =1.57 ft per 1,000 ft\\
  \item Assume velocity head =2.38 ft\\
  \item Required flow per day (maximum) 4,000 GPM\\
  \item Pump efficiency =85 \%\\
  \item Motor efficiency =89 \%\\
\end{itemize}
a. $\quad 375 \mathrm{hp}$\\
b. $\quad 400 \mathrm{hp}$\\
*c. $420 \mathrm{hp}$\\
d. $450 \mathrm{hp}$ \\
\item Which is the net positive suction head available (NPSHA) given the following data? Will the pump cavitate, if the net positive suction head required (NPSHR) is 18.4 ft ? Note: There are $1.11 \mathrm{ft} / \mathrm{in}$. of $\mathrm{Hg}$.\\
\begin{itemize}
  \item Atmospheric pressure =29.8 in Hg\\
  \item Static suction lift =15.1 ft\\
  \item Friction headloss = 0.61 ft\\
  \item Vapor pressure at $12^{\circ} \mathrm{F}(\mathrm{VP})=0.50 ft$\\
\end{itemize}
a. 14 ft, therefore NPSHA < NPSHR so cavitation should occur\\
*b. 17 ft, therefore NPSHA < NPSHR so cavitation should occur\\
c. 20 ft, therefore NPSHA > NPSHR so cavitation should not occur\\
d. 22 ft, therefore NPSHA > NPSHR so cavitation should not occur\\
  \item Determine the approximate $\mathrm{C}$ factor for a pipe that is 1.0 ft in diameter and has a flow of $1,225 \mathrm{gpm}$ given the following data:\\
\begin{itemize}
  \item Upstream pressure gauge 120 ft\\
  \item Downstream pressure gauge 105 ft\\
  \item Distance between gauges 2,274 ft\\
  \end{itemize}
*a. 95\\
b. 100\\
c. 110\\
d. 120\\
  \item A storage tank has a capacity of 34 ft. Currently there are 22.89 ft of water in the tank. Which would the SCADA reading be on the board in milliamps (mA) for a 4-mA to 20-mA signal?\\
a. 13.9 mA\\
b. 14.1 mA\\
c. 14.3 mA\\
*d. 14.8 mA\\

  \item How many pounds of lime must be added to exactly 200 gal of water to produce a lime slurry of $15 \%$ ?\\
a. $220 \mathrm{lb}$\\
*b. $290 \mathrm{lb}$\\
c. $340 \mathrm{lb}$\\
d. $420 \mathrm{lb}$ \\

  \item Your water treatment plant uses $39.6 \mathrm{lbs}$. of cationic polymer to treat a flow of 2.71 MGD. What is the polymer dosage?\\
a) $0.07 \mathrm{ppm}$\\
*b) $1.75 \mathrm{ppm}$\\
c) $14.61 \mathrm{ppm}$\\
d) $3.23 \mathrm{ppm}$\\

  \item The sedimentation basin at a water plant measure 60 feet long by 40 feet wide by 8 feet deep. The flow through this plant is 4.1 cuft/sec. What is the detention time?\\
a) 1 hour 18 minutes\\
b) 144 minutes\\
*c) 449 minutes\\
d) 2 hours 24 minutes\\

  \item How many gallons are there in 28.65 acre-ft?\\
a. $9,354,282 \mathrm{gal}$\\
b. $9,322,137 \mathrm{gal}$\\
*c. $9,335,000 \mathrm{gal}$\\
d. $9,763,599$ gal\\

  \item If $7.3 \mathrm{lb}$ of polymer (assume 100\%) are mixed into 35 gal of water, determine the percentage of polymer in the slurry.\\
a. $2.1 \%$ slurry\\
*b. $2.4 \%$ slurry\\
c. $2.5 \%$ slurry\\
d. $2.8 \%$ slurry\\

  \item Find the detention time in hours for a clarifier that has an inner diameter of 112.2 ft and a water depth of 10.33 ft if the flow rate is $7.26 \mathrm{mgd}$.\\
a. $2.10 \mathrm{hr}$\\
b. $2.14 \mathrm{hr}$\\
*c. $2.52 \mathrm{hr}$\\
d. $2.96 \mathrm{hr}$ \\
\end{enumerate}
\end{document}