\documentclass{article}
%\usepackage[english]{babel}%
\usepackage{graphicx}
\usepackage{tabulary}
\usepackage{tabularx}
\usepackage[table,xcdraw]{xcolor}
\usepackage{pdflscape}
%\usepackage{gensymb}
\usepackage{lastpage}
\usepackage{multirow}
\usepackage{xcolor}
\usepackage{cancel}
\usepackage{amsmath}
\usepackage[table]{xcolor}
\usepackage{fixltx2e}
\usepackage[T1]{fontenc}
\usepackage[utf8]{inputenc}
\usepackage{ifthen}
\usepackage{fancyhdr}
\usepackage[utf8]{inputenc}
\usepackage{tikz}
\usepackage[document]{ragged2e}
\usepackage[margin=1 in,top=1.2in,headheight=57pt,headsep=0.1in]
{geometry}
\usepackage{ifthen}
\usepackage{fancyhdr}
\everymath{\displaystyle}
\usepackage[document]{ragged2e}
\usepackage{fancyhdr}
\usepackage{mathabx}
\usepackage{textcomp,mathcomp}
\usepackage[shortlabels]{enumitem}
\everymath{\displaystyle}
\linespread{2}%controls the spacing between lines. Bigger fractions means crowded lines%
\linespread{1.3}%controls the spacing between lines. Bigger fractions means crowded lines%
\pagestyle{fancy}
\setlength{\headheight}{56.2pt}
\usepackage{soul}
\usepackage{siunitx}

%\usepackage{textcomp}
\usetikzlibrary{shapes.multipart, shapes.geometric, arrows}
\usetikzlibrary{calc, decorations.markings}
\usetikzlibrary{arrows.meta}
\usetikzlibrary{shapes,snakes}
\usetikzlibrary{quotes,angles, positioning}
%\chead{\ifthenelse{\value{page}=1}{\includegraphics[scale=0.3]{BassettCTCLogo}}}
%\rhead{\ifthenelse{\value{page}=1}{Final Exam}{}}
%\lhead{\ifthenelse{\value{page}=1}{Water Treatment - Oct-Dec 2022}{\textbf Final Exam}}
%\rfoot{\ifthenelse{\value{page}=1}{}{}}
%
%\cfoot{}
%\lfoot{Page \thepage\ of \pageref{LastPage}}
%\renewcommand{\headrulewidth}{2pt}
%\renewcommand{\footrulewidth}{1pt}
\graphicspath{ {./images/} }

\chead{\ifthenelse{\value{page}=1}{\textbf \\ \textbf November 23, 2023 Online Math Session}{\textbf \\ \textbf November 23, 2023 Online Math Session}}
\rhead{\ifthenelse{\value{page}=1}{}{}}
\lhead{\ifthenelse{\value{page}=1}{}{}}
\rfoot{\ifthenelse{\value{page}=1}{Secondary Treatment - Activated Sludge and Trickling Filter}{Secondary Treatment - Activated Sludge and Trickling Filter}}

\cfoot{}
\lfoot{Page \thepage\ of \pageref{LastPage}}
\renewcommand{\headrulewidth}{2pt}
\renewcommand{\footrulewidth}{1pt}
\begin{document}

\begin{enumerate}


\item At an activated sludge wastewater treatment plant receiving 3.25 MGD, the final effluent suspended solids concentration averages 21.2 mg/L. What would the calculated MCRT value be when the aeration basin carries 2,050 mg/L MLSS and wastes 0.0550 MGD. The waste activated sludge has a concentration of 7,980 mg/L. The aeration tank has a volume of 1.00 MG and the secondary clarifier has an operational volume of 0.250 MG.\\
\vspace{1cm}
$MCRT (days) =  \dfrac{MLSS \enspace in \enspace aeration \enspace tank \enspace (lbs)+MLSS \enspace in \enspace clarifier \enspace (lbs)}{SS \enspace effluent \enspace (lbs/day)+SS \enspace WAS \enspace (lbs/day)}$\\
\vspace{0.3cm} 
$MLSS \enspace in \enspace aeration \enspace tank \enspace (lbs)=1*2050*8.34=17,097lbs$\\
\vspace{0.3cm} 
$MLSS \enspace in \enspace clarifier \enspace (lbs)=0.25*2050*8.34=4,274.3lbs$\\
\vspace{0.3cm} 
$SS \enspace effluent \enspace (lbs/day)=3.25MGD *21.2mg/l*8.34=574.6 lbs/day$\\
\vspace{0.3cm} 
$SS \enspace WAS \enspace (lbs/day)=0.055MGD *7,980mg/l*8.34=3,660.4lbs/day$\\
\vspace{0.3cm} 
Plugging in the values calculated above: $MCRT (days) =  \dfrac{17,097.6+4,274.3}{574.6+3,660.4}=4.8=\boxed{5 \enspace days}$\\
\vspace{0.2cm}

\item Given that an activated sludge plant with an influent flow of 1.2 MGD is operated at an MCRT of 6 days and the parameters below, calculate the WAS flow rate  in gallons per day.\\
\vspace{1cm}
\begin{tabular}{ | m {7 cm} | m {7 cm}| } 
 \hline
Two aeration tanks – 0.5 MG each & Two final clarifiers – 0.25 MG each \\ 
 \hline
 Final effluent $= 20mg/l$ & WAS – 7500 ppm\\ 
 \hline
 MLSS –$3600mg/l$ & MLSS volatile solids content = 80\%  \\
 \hline
\end{tabular}\\
\vspace{1cm}
MCRT$=\dfrac{lbs MLSS (system)}{\dfrac{lbs}{day}Effluent SS + \dfrac{lbs}{day}WAS_{SS}}  $

\noindent lbs MLSS (system)$=(2*0.5 + 2*0.25)MG * 3600\dfrac{mg}{L} * 8.34 = 45036lbs$

\noindent $\dfrac{lbs}{day} Effluent SS= 1.2 MG * 20\dfrac{mg}{L} * 8.34 = 200.2lbs$

\noindent MCRT: $6 days=\dfrac{45036}{200.2 \dfrac{lbs}{day}+ \dfrac{lbs}{day}WAS_{SS}}  $

\noindent $\dfrac{lbs}{day}WAS_{SS} = \dfrac{45036}{6} - 200.2 = 7306 \dfrac{lbs}{day}$

\noindent $7306 \dfrac{lbs}{day} = WAS Flow (MGD) * 7500 * 8.34$\\
$ \implies WAS Flow (MGD)=\dfrac{7306}{7500*8.34}=0.116 MGD = \boxed {116,000 \dfrac{gal}{day}}  $
\newpage

\item The 1.5 MGD influent to a treatment plant has an average BOD of 280 mg/L.  32\% of this BOD is removed in the primary sedimentation tank. The primary effluent flows into an aeration tank containing 7000 lbs of MLSS with 79 \% volatile matter.  Calculate this plant's F to M ratio.\\

\vspace{1cm}

Influent BOD concentration to the AS basin: $280\dfrac{mg}{l}*(1-0.32)= 190.4 \dfrac{mg}{l}$\\

$\dfrac{F}{M}= \dfrac{1.5 MGD * 190.4\dfrac{mg}{l}*8.34}{7000*0.79}= \boxed{0.43\dfrac{F}{M}}$\\
\vspace{0.3cm}


\item The desired F/M ratio is .35 lbs BOD/day/lb MLVSS. If 2,100 lbs of BOD enter the aerator daily, how many lbs of MLVSS should be maintained in the aeration tank?\\
\vspace{1cm}
$F:M=\dfrac{(lbs/day) \enspace primary \enspace effluent  \enspace BOD \enspace entering \enspace the  \enspace aeration \enspace tank}{(lbs) \enspace MLVSS \enspace in \enspace the  \enspace aeration \enspace tank}$\\
\vspace{0.3cm}
$\implies 0.35=\dfrac{2100}{x}\implies x = \boxed{6000lbs \enspace MLVSS}$\\

\item In an aeration tank, the MLSS is 2650 mg/l and recorded 30-minute settling test indicates 221 ml/L.  What is the sludge volume index?\\
\vspace{1cm}
SVI (ml/g)= $\dfrac{Settled \enspace sludge \enspace volume \enspace in \enspace ml/l \enspace after \enspace 30 \enspace min}{MLSS \enspace mg/l}*1000 \dfrac{mg}{g}$\\
\vspace{0.5cm}
SVI=$\dfrac{221ml/l}{2650mg/l}*1000\dfrac{mg}{g}=\boxed{83ml/g}$


\item The total influent flow (including recirculation) to a trickling filter is 1.89 MGD. If the trickling filter is 80 ft in diameter, what is the hydraulic loading in gpd/sq ft on the trickling filter?\\
\vspace{1cm}
 $Hydraulic \enspace loading \enspace \frac{gpd}{ft^2}=\frac{(1.89*10^6)gpd}{(0.785*80^2)ft^2} =\boxed{376\frac{gpd}{ft^2}}$
\item  The suspended solids concentration entering a trickling filter is 236 mg/l. If the suspended solids concentration of the trickling filter effluent is 33 mg/l, what is the suspended solids removal efficiency of the trickling filter?\\
\vspace{1cm}
$\% Removal=\frac{236 mg/l-33 mg/l}{236 mg/l}*100=\boxed{86\%}$
\newpage

\item  A trickling filter, 70 ft in diameter with a media depth of 6 ft, receives a flow of 0.78 MGD. If the BOD concentration of the primary effluent is 167 mg/L, what is the organic loading on the trickling filter in lbs BOD/day/1000 cu ft?\\
\vspace{1cm}
$Organic \enspace loading:\dfrac{lbs \enspace BOD}{day-1000ft^3}=\dfrac{lbs \enspace BOD \enspace feed \enspace to \enspace TF \enspace per \enspace day}{volume \enspace in \enspace 1000ft^3}$\\
\vspace{0.3cm}
$=\dfrac{\dfrac{(0.78*167*8.34)lbs \enspace BOD}{day}}{(0.785*70^2*6)ft^3*\dfrac{1000ft^3}{1000ft^3}}=\boxed{\dfrac{47 lbs \enspace BOD}{day-1000 ft^3}}$


\item A trickling filter has a total flow of 32 MGD.  If the recirculation ratio is 0.8, what is the primary effluent flow to the TF?\\
\vspace{1cm}
$Total \enspace Flow (Q_T) = Influent \enspace Flow (Q_I)*(Recirculation \enspace Ratio(R_R) +1)$\\
$\implies 32 MGD=Q_I*(0.8+1)\implies Q_I=\dfrac{32}{1.8}=\boxed{17.8 MGD}$

\item The desired trickling filter recirculation ratio is 1.4.  If the primary effluent flow is 4.4 MGD what is the trickling filter effluent flow that needs to be recirculated.\\
Solution:\\
$R_R=\dfrac{Q_R}{Q_I}\implies 1.4=\dfrac{Q_R}{4.4}\implies Q_R =1.4*4.4=\boxed{6.2 MGD}$\\





\end{enumerate}

\end{document}