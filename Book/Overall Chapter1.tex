\chapterimage{Water1.png} % Chapter heading image

\chapter{Introduction to Wastewater Treatment}

% \section{Paragraphs of Text}\index{Paragraphs of Text}


% \everymath{\displaystyle}
% \linespread{2}%controls the spacing between lines. Bigger fractions means crowded lines%
% %\pagestyle{fancy}
% %\usepackage[margin=1 in, top=1in, includefoot]{geometry}
% %\everymath{\displaystyle}
% \linespread{2}%controls the spacing between lines. Bigger fractions means crowded lines%
% %\pagestyle{fancy}
% \pagestyle{fancy}
% \setlength{\headheight}{56.2pt}
% \colorlet{Mycolor1}{green!10!orange!90!}

% \chead{\ifthenelse{\value{page}=1}{\includegraphics[scale=0.3]{SCC}\\ \textbf \textbf Introduction to Wastewater Treatment}}
% \rhead{\ifthenelse{\value{page}=1}{}{}}
% \lhead{\ifthenelse{\value{page}=1}{}{\textbf Introduction to Wastewater Treatment}}
% \rfoot{\ifthenelse{\value{page}=1}{Module 1: WATR 048 - Spring 2019}{Module 1: WATR 048 - Spring 2019}}

% \cfoot{Page \thepage\ of \pageref{LastPage}}
% \lfoot{Shabbir Basrai}
% \renewcommand{\headrulewidth}{2pt}
% \renewcommand{\footrulewidth}{1pt}

% \newcommand{\stkout}[1]{\ifmmode\text{\sout{\ensuremath{#1}}}\else\sout{#1}\fi}
% %Defining colour with different models.
% \definecolor{mypink1}{rgb}{0.858, 0.188, 0.478}
% \definecolor{mypink2}{RGB}{219, 48, 122}
% \definecolor{mypink3}{cmyk}{0, 0.7808, 0.4429, 0.1412}
% \definecolor{mygray}{gray}{0.6}
% \colorlet{LightRubineRed}{RubineRed!70!}
% \colorlet{Mycolor1}{green!10!orange!90!}
% \definecolor{Mycolor2}{HTML}{00F9DE}

% %New command used in the table with all available colour names
% \newcommand{\thiscolor}[1]{\texttt{#1} \hfill \fcolorbox{black}{#1}{\hspace{2mm}}}

% %This changes the row separation in the table
% \renewcommand{\arraystretch}{1.5}




%\noindent\textsc{Area \& Volume Math Problems}
%\definecolor{shadecolor}{RGB}{200,200,240}
% \item \noindent\textsc{Why Treat Wastewater}
\section{Why Treat Wastewater}\index{Why Treat Wastewater}
\begin{itemize}
\item Wastewater is used water from home and industries.\\
\item Wastewater must be treated prior to returning it back into the environment - typically into the receiving waters which include lakes, rivers and ocean.


\item Wastewater treatment removes:
\begin{itemize}
\item organic matter
\item inorganic  pollutants including plant nutrients - nitrogen and phosphorous\\
\item pathogenic (disease causing) organisms\\
\end{itemize}

\item Wastewater treatment protects:
\begin{itemize}
\item The environment
\item Human health
\end{itemize}

\item In the receiving waters, inadequately treated wastewater discharge depletes dissolved oxygen levels - \hl{Eutrophication}, potentially destructing its normal aquatic life including fish.  Wastewater discharge promotes eutrophication due to:

\begin{itemize}
\item Nutrients such as nitrogen and phosphorous present in wastewater effluent promotes growth of plant and algal matter.  Dissolved oxygen is consumed as a part of the normal decay of this plant and algal matter.  
\item The consumption of organic material present in wastewater discharge by aerobic bacteria also results in oxygen depletion in the receiving waters.  


\end{itemize}
\end{itemize}

\section{Wastewater Treatment Regulations}\index{Wastewater Treatment Regulations}
% \begin{snugshade*}
% \item \noindent\textsc{Wastewater Treatment Regulations}
% \end{snugshade*}

\begin{itemize}
\item The \hl{National Pollutant Discharge Elimination System (NPDES) permit program} was created in 1972 by the Clean Water Act (CWA).
\item Applies to sources that discharge pollutants to waters of the United States.
\item Requires all facilities discharging “pollutants” into any body of water in the USA to obtain and comply with a \hl{NPDES permit}.
\item NPDES permit \hl{establishes} \textul{discharge limits}, \textul{monitoring} and \textul{reporting} \hl{requirements}\\
\item The NPDES permitting and enforcement responsibilities have been delegated by the EPA to the State of California for implementation through the \hl{State Water Resources Control Board(SWRCB)} and the \textul{nine} \hl{Regional Water Quality Control Boards (Regional Water Boards)}.
\item In California, NPDES permits are also referred to as waste discharge requirements (WDRs) that regulate discharges to waters of the United States.
\end{itemize}


\section{Wastewater Process Overview}\index{Wastewater Process Overview}
Wastewater treatment involves the following elements:

\subsection{Generation}\index{Generation}

Wastewater originates from domestic, industrial, commercial or agricultural activities. The characteristics of wastewater vary depending on the source. Types of wastewater include: 
\begin{itemize}
\item \hl{Domestic Sewage:}  wastewater derived principally from dwellings, business buildings, institutions, and \\
\item \hl{Industrial Sewage:}  liquid waste from industrial processes\\
\end{itemize}
Typical per person generation of wastewater in the USA is about 70-100 gallons per day

\subsection{Collections}\index{Collections}

\begin{itemize}
\item Wastewater is collected from its point of origin - home, businesses, industries etc. and conveyed via sewer lines to a centralized wastewater treatment facility.  
\item When the rainwater drainage is made part of the sewer system, the system is termed as \hl{Combined System}.  
\item The system where the sewage is conveyed separately from the stormwater flows is termed as \hl{Separated System}.  
\item In the Separated System, the Sanitary Sewers convey the wastewater and the Stormwater Sewer conveys the storm water flows.  
\item For the Combined System, rainstorms pose the threat of overwhelming the sewers and the treatment plant
\end{itemize}  
\begin{center}
\includegraphics[scale=0.45]{SeperatedSystem1} \hspace{1 cm} \includegraphics[scale=0.45]{CombinedSystem1}
\end{center}
			\hspace{2.6cm} Separated System \hspace{3.2cm} \parbox{\textwidth}{Combined System}\\

\subsection{Treatment}\index{Treatment}

\begin{itemize}
\item Wastewater treatment can involve physical, chemical or biological processes or combinations of these processes depending on the required outflow standards. 
\item Wastewater treatment typically involves a series of steps with increasing level of treatment:
\begin{itemize}
\item \hl{Preliminary}:  The preliminary process removes large/coarse solids which include rocks, tree branches, grit and other debris present in wastewater.
\item \hl{Primary}:  The primary process is also a physical process where the separable wastewater solids - solids that float and solids that can settle, are removed.  
\item \hl{Secondary}:  Secondary treatment is a biological treatment process where microorganisms consume the organic matter present in the wastewater. 
\item \hl{Tertiary or Advanced Treatment}:  The tertiary/advanced treatment processes improve the quality of treated water beyond the secondary treatment level.  This process may include nutrient removal and disinfection.
\end{itemize}

A generalized layout/process sequencing in a wastewater treatment plant is shown below:
\begin{center}
\includegraphics[scale=0.6]{TreatmentFlow}
\end{center}
Individual wastewater treatment processes involve different process options or sequences which are illustrated in the graphic below:
\begin{center}
\includegraphics[scale=0.42]{Treatment}
\end{center}
\end{itemize}
As the treatment process becomes more advanced along with the increasing awareness of the resources involved in the treatment, there is a move underway to transform Wastewater Treatment Plants (WWTF) to Renewable Resource Recovery Facilities (RRRF) or Water Resource Recovery Facilities (WRRF) - one which produces clean water, recovers energy and generates nutrients.

\subsection{Disposal or Reuse}\index{Disposal or Reuse}

\begin{itemize}
\item Wastewater treatment processes can be designed to \hl{dispose} the treated water where the water is reintroduced to the environment or for \hl{reuse} where the treated water is \hl{reclaimed} or \hl{recycled} - for various purposes including irrigation, industrial use or for potable use.
\item Water disposal methods include:\\
\begin{itemize}
\item \hl{Surface water discharge}
\item \hl{Subsurface discharge}
\end{itemize}
\item Water reuse methods include:\\
\begin{itemize}
\item Potable water reuse
\begin{itemize}
\item \hl{Indirect potable reuse:}  Here the treated water is blended with groundwater or surface water and then reclaimed and treated further 
for drinking (potable) water use
\item \hl{Direct potable reuse:}  Here the treated wastewater is subjected to advanced treatment and introduced directly into a municipal water supply system
\end{itemize}
\item Water reclamation for irrigation or industrial use\\
\item Land application for beneficial use\\
\end{itemize}
\item Solids generated from the wastewater treatment process may be removed and disposed to a landfill or subject to further treatment which may allow for energy recovery - from the organic solids and for beneficial reuse due to its plant nutrient content.\\
\end{itemize}
%\newpage
%\section{Chapter Assessment Quiz}
%
%
%
%1. Eutrophication is the degradation of plant and organic matter in a body of water\\
%
%a. True \\
%b. False \\
%\vspace{0.3cm}
%
%2. Direct potable reuse is when advanced treated wastewater is directly introduced into the water supply system\\
%
%a. True \\
%b. False \\
%\vspace{0.3cm}
%
%9. NPDES stands for\\
%
%a. National Pollutant Discharge Elimination System \\
%b. National Permit for Discharge of Effluents \\
%c. Natural Process for Discharge of Effluent Systems \\
%
%
%\textbf{Answers:}\\
%\vspace{0.3cm}
%1.	b.  \\
%
%\vspace{0.3cm}
%2.	a.  \\
%
%\vspace{0.3cm}
%9.	a.  \\





% \end{enumerate}

% \pagebreak
% \begin{center}
% \phantom{A}
% \vspace{10cm}

% BLANK PAGE
% \end{center}
% \pagebreak