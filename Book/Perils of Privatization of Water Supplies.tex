\documentclass{article}
%\usepackage[english]{babel}%
\usepackage{graphicx}
\usepackage{tabulary}
\usepackage{tabularx}
\usepackage[table,xcdraw]{xcolor}
\usepackage{pdflscape}
\usepackage{lastpage}
\usepackage{multirow}
\usepackage{cancel}
\usepackage{amsmath}
\usepackage[table]{xcolor}
\usepackage{fixltx2e}
\usepackage[T1]{fontenc}
\usepackage[utf8]{inputenc}
\usepackage{ifthen}
\usepackage{fancyhdr}
\usepackage[document]{ragged2e}
\usepackage[margin=1in,top=1.2in,headheight=57pt,headsep=0.1in]
{geometry}
\usepackage{ifthen}
\usepackage{fancyhdr}
\everymath{\displaystyle}
\usepackage[document]{ragged2e}
\usepackage{fancyhdr}
\usepackage[table,xcdraw]{xcolor}
% If you use beamer only pass "xcolor=table" option, i.e. \documentclass[xcolor=table]{beamer}
\usepackage[normalem]{ulem}
\useunder{\uline}{\ul}{}
\everymath{\displaystyle}
\linespread{2}%controls the spacing between lines. Bigger fractions means crowded lines%
%\pagestyle{fancy}
%\usepackage[margin=1 in, top=1in, includefoot]{geometry}
%\everymath{\displaystyle}
\linespread{1.3}%controls the spacing between lines. Bigger fractions means crowded lines%
%\pagestyle{fancy}
\pagestyle{fancy}
\setlength{\headheight}{56.2pt}


\chead{\ifthenelse{\value{page}=1}{\includegraphics[scale=0.3]{BassettCTCLogo}\\ \textbf \textbf Water Treatment I}}
\rhead{\ifthenelse{\value{page}=1}{Shabbir Basrai}{Shabbir Basrai}}
\lhead{\ifthenelse{\value{page}=1}{}{\textbf Water Treatment I}}


\cfoot{}
\lfoot{Page \thepage\ of \pageref{LastPage}}
\rfoot{Module 8}
\renewcommand{\headrulewidth}{2pt}
\renewcommand{\footrulewidth}{1pt}
\begin{document}

Developing world cities with private water-management companies have been plagued by lapses in service, soaring costs, corruption and worse. In Manila, where the water system is controlled by Suez, San Francisco-based Bechtel and the prominent Ayala family, water is only reliably available for a few hours a day, and rate increases have been so severe that the poorest families must choose each month between paying for water and two days’ worth of food. In 2001 the government of Ghana agreed to privatise local water systems as a condition for an IMF loan. To attract investors, the government doubled water rates, setting off protests in a country where the average annual income is less than \$400 a year and the water bill (for those fortunate enough to have running water) can run upwards of \$110. In Cochabamba, the third-largest city in Bolivia, water rates shot up by 35 per cent after a consortium led by Bechtel took over the city’s water system in 1999; some residents found themselves paying 20 per cent of their income on water. An initial round of peaceful street protests led to riots in which six people were killed. Eventually, the Bolivian government voided Bechtel’s contract and told the company’s officials it could not guarantee their safety if they stayed in town. Privatisation has also spawned protests (and, in some cases, even dominated elections) in Paraguay, where police turned water cannons on anti-privatisation protesters, Panama, Brazil, Peru, Colombia, India, Pakistan, Hungary and South Africa.
Louma (2004)\\

It is common practice to treat wastewater to the point where it is cleaner than the local waterways into which it is ultimately released. Eventually, it arrives at the ocean, with absolutely no downstream use—this is referred to as one-and-done usage. Why waste such a valuable resource? Why not reuse it? But we do already reuse it to some extent through de facto water recycling, as shown in Figure 11.1.\\

INTRODUCTION
We live on a planet with a surface that is three-fourths covered with water, so we recognize the irony inherent in the fact that many areas of the world face critical shortages of drinking water. Most of Earth’s water is seawater, of course—far too saline for human consumption. Of the little “fresh” water that remains, most is trapped in polar ice caps, where harnessing it for use is difficult. Much of the accessible natural supply of potable water is stressed by a growing world population, which increases the basic demand for this natural resource while reducing
the supply further through contamination. Major population centers in developing nations (those without established waste treatment or water treatment infrastructure) often suffer from epidemics of waterborne disease. In these areas, raw sewage can directly contaminate the rivers and streams used for drinking, washing, and cooking. In other cases, unchecked industrialization leads to water contamination through improperly disposed of chemical and nuclear wastes. The drinking water purveyor must ensure that the drinking water supplied is safe for human consumption. In fact, the primary reason for the development and installation of a public water system is the protection of public health. Basically, a properly operated water system serves as a line of defense between disease and the public. Properly operated water treatment and supply systems are defined as those that:\\
\begin{itemize}
\item Remove or inactivate pathogenic microorganisms including bacteria, viruses, and protozoa.
\item Reduce or remove chemicals that can be detrimental to health.
\item Provide quality water, thus discouraging the customer from seeking better tasting or better looking water that may be contaminated.
\end{itemize}
This last point is critical, but one often overlooked in the operation and management of public water systems. When the water produced by a system is objectionable because of odor, taste, or appearance, customers will seek another source for their drinking water. Ironically, these alternative sources, although they look, taste, and smell fine (“better than city water”), could contain microorganisms or chemicals that are harmful. This chapter discusses the drinking water practitioner’s most important function: ensuring that water delivered to the public is properly treated and arrives as the clean, wholesome, safe product that it must be. Moreover, it also covers the innovative approach taken by the Hampton Roads Sanitation District (HRSD) to replacing one-and-done usage with one-and-redone usage.\\



\textbf{A PARADIGM SHIFT IN PROGRESS}\\
Water shortage is the lack of adequate accessible water resources to meet water needs within a locality. More than 1.2 billion people lack access to clean drinking water (United Nations, 2017). For localities where access to drinking water is readily available, an issue that is not necessarily recognized at this time is the one-and-done scenario discussed earlier—that is, safe drinking water quality water is drawn from a tap and used for a variety of purposes and that is that. After being used, this water is poured down drains or flushed down toilets—out of sight and out of mind. But, a significant paradigm shift is beginning to occur. The idea of toilet-to-tap reuse is not palatable to many people, but we need water. We cannot live without water. Fortunately, we can clean used water and reuse it, a task that Mother Nature often can do for us naturally. We have no other choice. Regions where water is readily accessible today may not be able to brag about that in the future. Population growth, overuse, misuse, abuse, and other events and actions affect water use and have detrimental impacts on water quality. We need to change the one-and-done scenario to a one-and-redone scenario by using technology to purify used water. The use of advanced treatment and purification of used water (wastewater) to drinking water quality is a paradigm change in progress.\\


\textbf{ADVANCED TREATMENT OF WASTEWATER TO DRINKING WATER QUALITY}\\
Advanced technologies and processes used for wastewater treatment and purification provided at indirect potable reuse (IPR) plants varies (see Figure 11.3) but are typically focused on providing multiple barriers for the removal of pathogens and organics. Nitrogen and TDS removal is provided at some locations where necessary. Table 11.2 shows most of the indirect potable reuse projects that have been implemented in the United States. The table has been sorted according to the type of potable reuse application (i.e., direct aquifer injection, aquifer recharge with surface spreading, and surface water augmentation). The first five projects shown in this table are direct injection
projects that match the proposed HRSD concept. Water extracted from direct injection and surface spreading projects that recharge groundwater is not typically treated again prior to distribution into the potable water system; however, water from surface water augmentation projects is typically treated again at water treatment plants because of water treatment requirements stipulated by the USEPA’s Surface Water Treatment Rule (SWTR). For example, Fairfax County’s Griffith Water Treatment Plant provides coagulation, sedimentation, ozone oxidation, biological activated carbon filtration, and chlorine disinfection for water extracted from the Occoquan Reservoir that is augmented by the Upper Occoquan Service Authority’s indirect potable reuse plant.
As shown in Table 11.2, the treatment provided for indirect potable reuse projects is typically a combination of multiple barriers for the removal of pathogens and organics. Multiple barriers for pathogens are typically provided through a combination of coagulation, flocculation, sedimentation, lime clarification, filtration (granular or membrane), and disinfection (chlorine, ultraviolet, or ozone). Multiple barriers for organics removal are typically provided through a combination of advanced
treatment processes (e.g., reverse osmosis, granular activated carbon, ozone in combination biological activated carbon), although conventional treatment processes (e.g., coagulation, softening) also provide removal at some locations. All potable reuse plants listed in Table 11.2 include a robust organics removal process of granular activated carbon (GAC), granular media filtration (GMF), biological activated carbon (BAC), reverse osmosis (RO), microfiltration (MF), ultraviolet advanced oxidation process (UVAOP), membrane bioreactor (MBR), or soil aquifer treatment (SAT), which are effective barriers to bulk and trace organics and represent the backbone of the potable treatment process. SAT is the controlled application of wastewater to earthen basins in permeable soils at a rate typically measured in terms of meters of liquid per week. The purpose of a soil aquifer treatment system is to provide a receiver aquifer capable of accepting liquid intended to recharge shallow groundwater, and system design and operating criteria are developed to achieve that goal. However, there are several alternatives with respect to the utilization or final fate of the treated water (USEPA, 2006):\\
\begin{itemize}
\item Groundwater recharge
\item Recovery of treated water for subsequent reuse or discharge
\item Recharge of adjacent surface streams
\item Seasonal storage of treated water beneath the site with seasonal recovery for agriculture
\end{itemize}
The SAT process typically includes application of the reclaimed water using spreading basins and subsequent percolation through the vadose zone. SAT provides significant removal of both pathogens and organics through biological activity and natural filtration. However, because some aquifers are confined, it is not possible to utilize the SAT for treatment through the vadose zone to recharge them. On the other hand, movement of reclaimed water through the aquifer after direct injection will provide significant treatment benefits, including excellent removal of pathogens. Advanced water treatment plants based on reverse osmosis and granular activated carbon are often utilized at locations where SAT treatment through the vadose zone is not feasible, because it is possible for these processes to be implemented at most locations.\\

\begin{table}[]
\begin{tabular}{lllll}
Project                                                            & Type of Potable Reuse Application                              & Year & Capacity (mgd) & Advanced Treatment Processes                                       \\
Hueco Bolton Recharge Project; El Paso, TX                         & Groundwater recharge via direct injection and spreading basins & 1985 & 10             & Lime + GMF + ozone + BAC + C                                       \\
West Basin Water Recycling Plant; Carson, CA                       & Groundwater recharge via direct injection                      & 1993 & 12.5           & MF + RO + UVAOP                                                    \\
Scottsdale Water Campus; Scottsdale, AZ                            & Groundwater recharge via direct injection                      & 1999 & 20             & MF + RO + Cl                                                      \\
Los Alamitos Seawater Intrusion Barrier; Long Beach, CA            & Groundwater recharge via direct injection                      & 2006 & 3              & MF + RO + UV disinfection                                          \\
Groundwater Replenishment                                          & Groundwater recharge via direct injection and spreading basins & 2008 & 70             & MF + RO + UVAOP + SAT (spreading basins for a portion of the flow) \\
Montebello Forebay, Groundwater Recharge District, Los Angeles, CA & Groundwater recharge via spreading basins                      & 1962 & 44             & GMF + C                                                            \\
Chino Basin Groundwater Recharge Project; Chico, CA                & Groundwater recharge via spreading basins                      & 2007 & 18             & GMF + O$_2$ + SAT (spreading basins)                                 
\\
Hueco Bolton Recharge Project; El Paso, TX                         & Groundwater recharge via direct injection and spreading basins & 1985 & 10             & Lime + GMF + ozone + BAC + C                                       \\
West Basin Water Recycling Plant; Carson, CA                       & Groundwater recharge via direct injection                      & 1993 & 12.5           & MF + RO + UVAOP                                                    \\
Scottsdale Water Campus; Scottsdale, AZ                            & Groundwater recharge via direct injection                      & 1999 & 20             & MF + RO + Cl$_2$ \\
Los Alamitos Seawater Intrusion Barrier; Long Beach, CA            & Groundwater recharge via direct injection                      & 2006 & 3              & MF + RO + UV disinfection                                          \\
Groundwater Replenishment                                          & Groundwater recharge via direct injection and spreading basins & 2008 & 70             & MF + RO + UVAOP + SAT (spreading basins for a portion of the flow) \\
Montebello Forebay, Groundwater Recharge District, Los Angeles, CA & Groundwater recharge via spreading basins                      & 1962 & 44             & GMF + C                                                            \\
Chino Basin Groundwater Recharge Project; Chico, CA                & Groundwater recharge via spreading basins                      & 2007 & 18             & GMF + O$_2$ + SAT (spreading basins)
\end{tabular}
\end{table}

\end{document}