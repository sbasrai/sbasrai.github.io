Vertical turbine pumps that are used in wells may be oil-lubricated or water-lubricated. Operators should use extreme care not to start any water-lubricated pump before making sure that the:\\
a.	Valve on discharge side is open.\\
b. Bearings are dry.\\
c.	Valve on suction side is closed.\\
d.	Bearings are wet.\\
The head against which a pump must operate:\\
a.	Is the sum of the static head and the head due to friction loss.\\
b.	Must always be above the  shut-off  head.\\
c.	Is the static head.\\
d.	Is the friction head.\\
What term describes the condition that exists when the source of the water supply is below the  centerline of the  pump?\\
a.	Pressure  head\\
b.	Velocity head\\
c.	Suction lift\\
d.	Total discharge head\\
What is the most common use today for a positive-displacement pump?\\
a.	Raw water intake pump\\
b.	System booster pump\\
c.	Chemical feed pump\\
d.	Filter feed pump\\
A pumping condition where the eye of the impeller is above the water is called?\\
a.	Dry Well\\
b.	Suction Head\\
c.	Wet Well\\
d.	Suction Lift\\
The force used in an End-suction pump is called\\
a.	Pressure\\
b.	Centrifugal\\
c.	Velocity\\
d.	Kinetic\\
\rule{9mm}{0.5pt} is the loss of energy as a result of friction.\\
a.	Velocity loss\\
b.	Headloss\\
c.	Elevation Loss\\
d.	Pump Loss\\
As the water travels around the volute towards the discharge line the total energy\\
shifts from\\
a.	High Velocity Head to low PSI\\
b.	Low Velocity Head to high PSI\\
c.	Low Velocity Head to low PSI \\
d.	High Velocity Head to high PSI\\
The part that in an End Suction pump that is used to collect the liquid discharged from the impeller is called?\\
a.	Shaft\\
b.	Packing\\
c.	Suction Head\\
d.	Volute\\
Head is the energy that a body has by virtue of its position or state.\\
a.	Velocity\\
b.	Potential\\
c.	Kinetic\\
d.	Pressure\\
An impeller that has no shrouds and used to pump fluid with large objects is called?\\
a.	Semi-open\\
b.	Open\\
c.	Closed\\
d.	Very-closed\\
A pump station design where the eye of the impeller is submerged in water is called?\\
a.	Dry Well\\
b.	Suction Head\\
c.	Wet Well\\
d.	Suction Lift\\
The discharge valve on a 	pump can be closed for short periods of time or during start up.\\
a.	Piston\\
b.	Progressive Cavity\\
c.	Diaphragm\\
d.	dynamic\\
Velocity of a pump is measured in:\\
a.	Inches per second\\
b.	PSI\\
c.	Feet per second\\
d.	Yards per second\\
An impeller that has shrouds on both sides and is used to pump fluid with little or no objects is called?\\
a.	Semi-open\\
b.	Open\\
c.	Closed\\
d.	Very - closed\\
To change the discharge of displacement you have to change the:\\
a.	Speed\\
b.	Discharge valve\\
c.	Suction valve\\
d.	Rotation\\
Which pump component prevents leakage from the pump discharge to the suction?\\
a.  Lantern ring\\
b.  Volute\\
c.  Wear ring\\
d.  Shaft sleeve\\
Mechanical seals are being installed in pumps because\\
a.	packing requires an undesirable leakage that seals eliminate.\\
b.	seals prevent cross connections with potable water.\\
c.	seals will take more shaft misalignment than packing.\\
d.	there is a shortage of good packing available on the market.\\
A major cause of pump and motor shaft coupling wear is:\\
a.	discharge pressure too high.\\
b.	low suction pressure.\\
c.	misalignment between pumps and motor flanges.\\
d.	worn-out seal.\\
The discharge rate of a piston-type pump:\\
a. Is constant as the main drive rpm changes\\
b. Is constant at a constant speed\\
c. Varies inversely with the head\\
d. Varies with the total dynamic head\\
  The flow of electrical current is measured in\\
a. Amperes\\
b. Ohms\\
c. Volts\\
d. Watts\\
An operator hears a pinging sound coming from the pump. What is the probable cause?\\
a.	Descaling\\
b. Cavitation\\
c. Corrosion \\
d. Hardness\\
During a routine inspection on a centrifugal pump, the operator notices that the bearings are excessively hot. This is most likely caused by:\\
a. Over lubrication\\
b. The speed being too slow\\
c. A worn impeller\\
d. A worn packing\\
The-leakage of seal-wateraround-the-packing on a centrifugal pump is required because it acts as a(n)\\
a. Adhesive\\
b. Coolant\\
c. Corrosion inhibitor\\
d. Scale inhibitor\\
What can happen to a pump if the back pressure on the pump is allowed to drop too low and the pump is operated for a prolonged period of time?\\
a. Efficiency would drop off and the pump would heat up\\
b. No water would flow\\
c. Pump lubricants would disperse more efficiently\\
d. Water hammer would occur upstream in the distribution line\\
At a pumping station equipped with centrifugal pumps, what can cause the discharge pressure to suddenly increase and the discharge quantity to suddenly decrease?\\
a. A discharge valve was closed\\
b. A suction valve was closed\\
c. The pump amperage was decreased\\
d. The voltage was suddenly increased\\
The difference between water levels upstream and downstream of a pump when it is not in operation is known as the\\
a. Suction lift\\
b. Total dynamic head\\
c. Discharge head\\
d. Friction loss\\
e. Total static head\\
Static suction head plus friction suction head plus static discharge head plus friction discharge head is a pump's\\
a. Pump curve\\
b. Operating pressure\\
c. Efficiency\\
d. Total dynamic head\\
e. Velocity head\\
Pumps are primed to\\
a. Replace air inside the pump with water\\
b. Seat the valves\\
c. Wet the packing\\
d. Provide water for flow testing\\
e. Overcome positive suction head\\
Backspin is occurring after well shutdown; this indicates\\
a. A high water table\\
b. A low water table\\
c. A confined aquifer\\
d. A faulty check valve\\
e. A leak in the sanitary seal\\
A water seal on a pump serves many purposes, including\\
a. Acts as a coolant to keep the pump bearing from overheating\\
b. Keeps gritty material from entering the packing box\\
c. Keeps the pumps primed\\
d. Is a reserve water supply\\
e. Prevents cavitation\\
Enclosed, open, and semi-closed are terms used for the designation and selection of:\\
a. Impellers\\
b. Lantern rings\\
c. Sleeves\\
d. Stuffing boxes\\
e. None of the above \\
A device that converts electrical energy into mechanical or kinetic energy is called a\\
a. Motor\\
b. Generator\\
c. Transformer\\
d. Battery\\
e. Pump\\
If a pump sounds like it is pumping rocks, the most likely cause is\\
a. Cavitation\\
b. Corrosion\\
c. Over-tightening of the packing gland\\
d. Misalignment with the motor\\
e. Irregular wear of the mechanical seal\\
The flow of electrical current is measured in\\
a. Amperes\\
b. Volts\\
c. Watts\\
d. Ohms\\
e. Farads\\
  The rotating element in a centrifugal pump is commonly called the\\
a. Fan\\
b. Impeller\\
c. Rotor\\
d. Volute\\
e. Stator\\
The purpose of the packing in a centrifugal pump is\\
a. Comparable to a bearing and is impregnated with lubricant\\
b. To prevent vibration of the shaft\\
c. To provide support for the impeller\\
d. To surround the bearings and lubricate them\\
e. None of the above\\
  Which of the following is a positive displacement pump?\\
a. Air lift pump\\
b. Centrifugal pump\\
c. Reciprocating pump\\
d. Turbine pump\\
e. All of the above\\
The practical maximum suction lift for a centrifugal pump in good condition is \\
a. $\quad 0$ feet\\
b. $\quad 2.31$ feet\\
c. $\quad 14.7$ feet\\
d. 20 feet to 25 -feet\\
e. $\quad 32$-feet to 34-feet\\
 The linkage between a centrifugal pump and its motor is commonly called the\\
a. Coupling\\
b. Impeller\\
c. Bearings\\
d. Volute\\
e. Stator\\
The electrical equivalent to friction in water lines is\\
a. Voltage\\
b: Resistance\\
c. Amperage\\
d. Capacitance\\
e. Inductance\\
The main water-containing body of a centrifugal pump is commonly called the\\
a. Shaft\\
b. Impeller\\
c. Bearings\\
d. Volute\\
e. Stator\\
A type of pump that produces high flow rates with low discharge heads is a\\
a. Radial flow\\
b. Axial flow\\
c. Vertical turbine\\
d. Piston\\
e. Mixed flow\\
Alternating current is produced by\\
a. A single battery\\
b. Two (or more) batteries in series\\
c. Two (or more) batteries in parallel\\
d. A solenoid\\
e. A generator\\
What do electrical transformers do?\\
a. Step-up or step-down current\\
b. Step-up or step-down voltage\\
c. Increase power output\\
d. Decrease power output\\
e. Reduce resistance\\
An "Open" electrical circuit is one in which\\
a. Resistance is low\\
b. Power production is high\\
c. Capacitance is low\\
d. Conductivity is high\\
e. Amperage is zero\\
Adding more stages (bowls) to a deep well turbine pump assembly will\\
a. Increase the pump discharge capacity\\
b. Decrease the pump discharge capacity\\
c. Increase the pump discharge pressure\\
d. Decrease the pump discharge pressure\\
e. None of the above\\
When installing packing in a centrifugal pump, the packing should be\\
a. Water tight\\
b. Pre-heated\\
c. Staggered $90^{\circ}$\\
d. Soaked overnight in potable water\\
e. Re-used\\
  Standard electrical line frequency in the United States is\\
a. $50 \mathrm{~Hz}$\\
b. $\quad 60 \mathrm{~Hz}$\\
c. $\quad 110 \mathrm{~Hz}$\\
d. $\quad 120 \mathrm{~Hz}$\\
e. $240 / 480 \mathrm{~Hz}$\\
In contrast to conventional packing, mechanical seals\\
a. Require no adjustment\\
b. Do not leak\\
c. Are generally more expensive\\
d. Are more difficult to remove/replace\\
e. All of the above\\
The level of water in a reservoir is 200 feet above the main line that carries water into and out of the reservoir. A standpipe in the main line a block away at the same elevation as the reservoir shows a water elevation of 185 feet. Which of the following statements is true?\\
a. There is no flow into or out of the reservoir\\
b. Water is flowing into the reservoir\\
c. Water is flowing out of the reservoir\\
d. There is a pump station adjacent to the pressure gauge\\
e. Nothing can be deduced from the information in this question.\\
Pump motors draw more power starting than during normal operating conditions because:\\
a.	check valves have to be pushed open\\
b.	energy is required to get the water moving\\
c.	the motor and pump have to start turning\\
d.	all of the above\\
Which of the following does not affect the friction loss in a given length of pipe?\\
a.	hardness of the water\\
b.	number of fittings\\
c.	roughness of the interior of the pipe\\
d.	velocity of the flow\\
The component of a centrifugal pump sometimes installed on the end of the suction pipe in order to hold priming is the:\\
a. Casing\\
b. Footvalve\\
c. Impeller\\
d. Lantern ring\\
At a pumping station equipped with centrifugal pumps, what can cause the discharge pressure to suddenly increase and the discharge quantity to suddenly decrease?\\
a. A discharge valve was closed\\
b. A suction valve was closed\\
c. The pump amperage was decreased\\
d. The voltage was suddenly increased\\
The inlet to the pump is called:\\
a. Suction\\
b. Volute\\
c. Impeller\\
d. Effluent\\
The rotating element in a centrifugal pump is commonly called a(n):\\
a. Fan\\
b. Impeller\\
c. Rotor\\
d. Volute\\
Pumps are primed to:\\
a) be sure the pump operates freely\\
b) replace air with water inside the pump\\
c) seat the valves ·\\
d) wet the packing\\
e) none of the above\\
The joints in the rings of pump packing should be:\\
a) placed in line\\
b) placed next to the motor\\
c) placed next to pump\\
d) staggered\\
e) none of the above\\
A vertical turbine pump is an example of a :\\
a) centrifugal pump\\
b) parshall flume\\
c) positive displacement pump\\
d) reciprocating pump\\
e) all of the above\\
 Which type of pump is most commonly used for high capacity wells?\\
a) air lift\\
b) centrifugal\\
c) positive displacement\\
d) plunger\\
e) none of the above\\
What can happen to a pump if the back pressure on the pump is allowed to drop too low and the pump is operated for a prolonged period of time?\\
a. Efficiency would drop off and the pump would heat up\\
b. No water would flow\\
c. Pump lubricants would disperse more efficiently\\
d. Water hammer would occur upstream in the distribution line\\
Check valves are used to prevent\\
a. Excessive pump pressure\\
b. Priming\\
c. Water from flowing in two directions\\
d. Water hammer\\
Positive displacement pumps should be operated when\\
a. Suction and discharge line valves are closed\\
b.  Suction and discharge line valves are open\\
c. Suction line valves are closed and discharge line valves are open\\
d. Suction tine valves are open and discharge line valves are closed\\
When comparing friction loss in various types of pumps, a larger Hazen-Williams ' $C$ ' value indicates the pipe\\
a. is more durable,\\
b. is rougher outside.\\
d. is able to withstand a higher pressure.\\
c. is smoother inside.\\
Proper alignment between two shafts can be checked using a:\\
a. caliper\\
b. micrometer\\
c. straight edge\\
d. feeler gauge\\
The maximum practical suction lift of a properly engineered centrifugal pump is about:\\
a. $5-10 \mathrm{ft}$\\
b. $10-15 \mathrm{ft}$\\
c. $15-25 \mathrm{ft}$\\
d. $25-34 \mathrm{ft}$ \\
Which type of pump is most commonly used for high capacity wells?\\
a. air lift\\
b. centrifugal\\
c. positive displacement\\
d. plunger\\
e. none of the above\\
A vertical turbine is an example of a:\\
a. centrifugal pump\\
b. parshall flume\\
c. positive displacement pump\\
d. reciprocating pump\\
e. all of the above\\
The joints in the rings of pump packing should be:\\
a. placed in line\\
b. placed next to the motor\\
c. placed next to pump\\
d. staggered\\
e. none of the above\\
Ppmps are primed to:\\
a. be sure the pump operates freely\\
b. replace air with water inside the pump\\
c. seat the valyes\\
d. wet the packing\\
e. none of the above\\
When comparing friction loss in various types of pumps, a larger Hazen-Williams 'C' value indicates the pipe\\
a. is more durable.\\
b. is rougher outside.\\
c. is smoother inside.\\
d. is able to withstand a higher pressure.\\
 If the packing on an operating centrifugal pump has a slight leakage, the following action should be taken: \\
a. shut down immediately \\
b. tighten packing gland \\
c. lubricate pump packing gland \\
d. decrease pump speed and head \\
*e. nothing \\
 If bearings on a centrifugal pump are running hot, checking for over lubrication or under lubrication would be. listed as a general preventive maintenance service. If the lubrication is satisfactory, the next preventive maintenance check would be: \\
a. replace bearings \\
b. operate only when needed \\
c. clean the pump \\
d. recheck TDH \\
*e. inspect alignment of pump and motor \\
 If a wastewater pump is to be shut down for a long period of time, the proper procedure is to open and lock out the motor disconnect switch and shut the valves on both sides of the pump. \\
a. True \\
*b. False \\
 Centrifugal pump parts include \\
a. Diaphragm \\
*b. Impeller \\
c. Piston \\
d. Rotor \\
 Where does wear most frequently occur on a plunger pump? \\
*a. Cylinder \\
b. Rotor \\
c. Stators \\
d. Volute\\
 As rotors or stators accumulate wear on progressive cavity pumps, the capacity of the pump is decreased. What is the easiest way to tell if the pump elements are worn? \\
a. Tap into the line between the pump and the discharge valve and determine the pump capacity by timing how long it takes to fill a 20-liter pail \\
*b. Measure the pressure on the discharge side of the pump with valves open and the pump pumping \\
c. Disassemble the pump, measure the parts and compare it to the original specifications \\
d. Close the discharge valve and measure the resultant pressure \\
 A centrifugal pump vibrates and is noisy From the choices below, select the most probable cause \\
a. Impeller too small \\
b. Foot valve too small \\
c. Dirt or grit in sealing liquid \\
*d. Air in the pump \\
 Given the following data, what is the most likely cause of the pump problem?\\
DATA: Pump is running\\
Reduced discharge from lift station\\
Impeller is clear\\
Level sensors are operating properly \\
a. Improper packing \\
b. Misaligned belt drives \\
*c. Pump air bound \\
d. None of the above \\
 Excessive leakage around seals on the shafts and plungers of a plunger pump may indicate what? \\
a. Attempting to pump against too great a head \\
*b. Excessive wear of the shaft and plunger \\
c. The eccentric needs replacement \\
d. The pump needs new ball checks \\
 In operating a small pumping station, which is provided with two identical pumps, it is best to adjust the controls so that \\
a. One pump does most of the work and the second pump is held in reserve being operated intermittently to keep it in good running condition \\
*b. The pumps alternate in operation \\
c. The pumps both turn on together \\
d. None of the above \\
 A positive displacement sludge pump should never be placed into operation \\
a. Without being primed. \\
*b. With the discharge valve closed. \\
c. With the discharge valve opened. \\
d. None of the above \\
 Prior to repairing a pump's electrical circuit, which of the following actions should you take? \\
*a. Disconnect the circuit breaker, place a red tag stating "do not activate," and lock out \\
b. Notify your supervisor \\
c. Tell all of the operators not to activate the circuit \\
d. Turn pump off \\
 Pump maintenance includes \\
a. Checking operating temperature of bearings \\
b. Checking packing gland. \\
c. Operating two or more pumps of the same size alternately to equalize wear \\
*d. All of the above \\
 When carrying out a routine inspection on a centrifugal pump, it is noted by the operator that the bearings are excessively hot This could be caused by \\
*a. Over lubrication \\
b. Speed too slow \\
c. Worn impeller \\
d. Worn packing \\
 In a centrifugal pump, internal leakage is prevented by \\
a. Impellers \\
b. Sleeves \\
c. Volutes \\
*d. Wear rings \\
 A horizontal centrifugal pump has "rope" packing When the pump is operating, water slowly drips from the packing gland. This indicates that the \\
a. Packing bolts or nuts on the packing gland should be tightened. \\
b. Packing bolts or nuts on the packing gland should be loosened. \\
*c. Packing bolts or nuts on the packing gland are properly adjusted. \\
d. Packing should be replaced. \\
 Wear rings are installed in a pump to \\
a. Hold the shaft in position \\
b. Keep the impeller in place \\
*c. Concentrate wear on an economically replaceable part \\
d. Wear out rings instead of sleeves \\
 A water seal on a pump serves a dual purpose It acts as a lubricant and it also \\
a. Acts as a coolant to keep the pump bearing from overheating \\
*b. Keeps gritty material from entering the packing box \\
c. Keeps the pump primed. \\
d. Is a reserve water supply \\
 The elevation of any pump above the source of supply should not exceed  {\underline{\hspace{1cm}}} feet\\
a. 2.2 \\
*b. 22 \\
c. 200 \\
d. 224\\
 What is the vertical distance between the elevation of the free water surface at the suction and that of the free water surface at the discharge of a pump called?\\
a.	Discharge head.\\
b.	Dynamic head.\\
c.	Velocity head.\\
*d.	Static head.\\
In electrical circuits, a device used to reduce the voltage is a(n)\\
a. Ammeter\\
b. Transducer\\
c. Transformer\\
d. Voltmeter\\
What can happen to a pump if the back pressure on the pump is allowed to drop too low and the pump is operated for a prolonged period of time?\\
a. Efficiency would drop off and the pump would heat up\\
b. No water would flow\\
c. Pump lubricants would disperse more efficiently\\
d. Water hammer would occur upstream in the distribution line\\




