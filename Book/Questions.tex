\documentclass{article}

%\usepackage[english]{babel}%

\usepackage{graphicx}

\usepackage{tabulary}

\usepackage{tabularx}

\usepackage[table,xcdraw]{xcolor}

\usepackage{pdflscape}

\usepackage{float}

\usepackage{lastpage}

\usepackage{multirow}

\usepackage{xcolor}

\usepackage{cancel}

\usepackage{amsmath}

\usepackage[table]{xcolor}

\usepackage{fixltx2e}

\usepackage[T1]{fontenc}

\usepackage[utf8]{inputenc}

\usepackage{ifthen}

\usepackage{fancyhdr}

\usepackage[document]{ragged2e}

\usepackage[margin=1in,top=1.2in,headheight=57pt,headsep=0.1in]
{geometry}

\usepackage{ifthen}

\usepackage{fancyhdr}

\everymath{\displaystyle}

\usepackage[document]{ragged2e}

\usepackage{fancyhdr}

\usepackage{mathabx}

\usepackage{gensymb}

\usepackage[shortlabels]{enumitem}
\usepackage{tikz}
\usepackage{mwe}
\usetikzlibrary{calc}
\usetikzlibrary{shapes.multipart, shapes.geometric, arrows}
\usetikzlibrary{calc, decorations.markings}
\usetikzlibrary{arrows.meta}
\usetikzlibrary{shapes,snakes}
\usetikzlibrary{quotes,angles, positioning}

\everymath{\displaystyle}

\linespread{2}%controls the spacing between lines. Bigger fractions means crowded lines%

%\pagestyle{fancy}

%\usepackage[margin=1 in, top=1in, includefoot]{geometry}

%\everymath{\displaystyle}

\linespread{1.3}%controls the spacing between lines. Bigger fractions means crowded lines%

%\pagestyle{fancy}

\pagestyle{fancy}

\setlength{\headheight}{56.2pt}

\usepackage{soul}

 

\chead{\ifthenelse{\value{page}=1}{\includegraphics[scale=0.3]{BassettCTCLogo}\\ \phantom{text}}}

\rhead{\ifthenelse{\value{page}=1}{Multiple Choice Questions from Chapter Assessments}{}}

\lhead{\ifthenelse{\value{page}=1}{Water Distribution - January 2023}{Multiple Choice Questions from Chapter Assessments}}

\rfoot{\ifthenelse{\value{page}=1}{}{}}

 

\cfoot{}

\lfoot{Page \thepage\ of \pageref{LastPage}}

\renewcommand{\headrulewidth}{2pt}

\renewcommand{\footrulewidth}{1pt}

\begin{document}

 


%%%%%%%%%%%%%%%%%%%%%%%%%%%%%%%%%%%%%%%%%%%%%%%%%%%%%%%%%%%%%%%
\section{Water Properties \& Sources}\index{Water Properties \& Sources}
%%%%%%%%%%%%%%%%%%%%%%%%%%%%%%%%%%%%%%%%%%%%%%%%%%%%%%%%%%%%%%%
\begin{enumerate}
\item Which one of the following best defines the term aquifer?\\
\begin{enumerate}
\item A low lying area where water pools
\item Water-bearing stratum of rock, sand, or gravel
\item Impervious stratum near the ground surface
\item Treated water leaving the water system
\end{enumerate}

\item The height to which water will rise in wells located in an artesian aquifer is called the
\begin{enumerate}
\item Pumping water level
\item Water table
\item Piezometric surface
\item Drawdown
\item Radius of influence
\end{enumerate}

\item What percentage of all the earth's water is readily available as a potential drinking water supply in the form of lakes, rivers, and near-surface groundwater?
\begin{enumerate}
\item 97%
\item 50%
\item 2%
\item 1%
\item 0.34%
\end{enumerate}

\item To  prevent the entry of surface contamination into a well is the purpose of
\begin{enumerate}
\item The well casing
\item The water table
\item The louvers or slots
\item Well development
\item The  annular grout seal	
\end{enumerate}

\item An aquifer that is located underneath an aquiclude is called
\begin{enumerate}
\item An unconfined aquifer
\item A confined aquifer
\item A water table
\item Unreachable groundwater
\item An Artesian spring
\end{enumerate}

\item The process by which water changes from the gas to the liquid phase is termed
\begin{enumerate}
\item Condensation	·
\item Evaporation
\item Percolation
\item Precipitation
\item Runoff
\end{enumerate}

\item The free surface of the water in an unconfined aquifer is known as the
\begin{enumerate}
\item Pumping water level
\item Artesian spring
\item Water table
\item Drawdown
\item Percolation
\end{enumerate}

\item The transfer of liquid water from plants and animals on the surface of the earth into water vapor in the atmosphere is called
\begin{enumerate}
\item Transpiration
\item Evaporation
\item Condensation
\item Runoff
\item Percolation
\end{enumerate}

\item The elevation of water in the casing of an operating well is called the
\begin{enumerate}
\item Piezometric surface
\item Water table
\item Pumping water level
\item Drawdown
\item Radius of influence
\end{enumerate}

\item An aquifer under pressure is often termed
\begin{enumerate}
\item Unconfined
\item Pacific
\item Artesian
\item Alluvial
\item Elevated
\end{enumerate}

\item An aquifer is usually composed of
\begin{enumerate}
\item Sand and gravel 
\item Clays and silts
\item Bedrock
\item Large voids in the soil, resembling underground lakes
\item None of the above
\end{enumerate}

\item Which of the following best defines the term specific capacity?
\begin{enumerate}
\item Amount of water a given volume of saturated rock or sediment will yield to gravity
\item Amount of water a given volume of saturated rock or sediment will yield to pumping
\item Rate at which water would flow in an aquifer if the aquifer were an open conduit
\item Amount of water a well will produce for each foot of drawdown
\end{enumerate}

\item The most common type of well used for public water supply systems is a
\begin{enumerate}
\item Jetted well
\item Driven well
\item Drilled well
\item Bored well
\end{enumerate}

\item Which of the following best defines the term static water level?
\begin{enumerate}
\item Water level in a well after a pump has operated for a period of time
\item Water level in a well when the well is not in operation
\item Water level in a well measured from the ground surface to the drawdown water level
\item Waterlevel in a well measured from the natural water level to the drawdown water level
\end{enumerate}

\item The residual drawdown of a well is defined as
\begin{enumerate}
\item Water level in a well after a pump has operated over a period of time
\item Measured distance from the ground to the pumping level
\item Water level below the normal level that persists after a well pump has been off for a period of time
\item Measured distance between the water level and the top of the screen
\end{enumerate}

\item A well is located in an aquifer with a water table elevation 20 feet below the ground surface. After operating for three hours, the water level in the well stabilizes at 50 feet below the ground surface. The pumping water level is:
\begin{enumerate}
\item 20 feet
\item 30 feet
\item 50 feet
\item 70 feet
\item 100 feet
\end{enumerate}

\item What percentage of all the earth's water is readily available as a potential drinking water supply in the form of lakes, rivers, and near-surface groundwater?
\begin{enumerate}
\item 97\%
\item 50\%
\item 2\%
\item 1\%
\item 0.34\%
\end{enumerate}

\item To  prevent the entry of surface contamination into a well is the purpose of
\begin{enumerate}
\item The well casing
\item The water table
\item The louvers or slots
\item Well development
\item The  annular grout seal	
\end{enumerate}

\item The process by which water changes from the gas to the liquid phase is termed
\begin{enumerate}
\item Condensation	·
\item Evaporation
\item Percolation
\item Precipitation
\item Runoff
\end{enumerate}

\item The free surface of the water in an unconfined aquifer is known as the
\begin{enumerate}
\item Pumping water level
\item Artesian spring
\item Water table
\item Drawdown
\item Percolation
\end{enumerate}

\item The transfer of liquid water from plants and animals on the surface of the earth into water vapor in the atmosphere is called
\begin{enumerate}
\item Transpiration
\item Evaporation
\item Condensation
\item Runoff
\item Percolation
\end{enumerate}

\item The term for the combined processes which transfer liquid water on the earth's surface into water in the gas phase in the atmosphere is
\begin{enumerate}
\item Percolation
\item Evapotranspiration
\item Sublimation
\item Overdraft
\item Precipitation
\end{enumerate}

\item A primary advantage of using surface water as a water source includes:
\begin{enumerate}
\item Usually higher in turbidity
\item Generally softer than groundwater
\item Easily contaminated with microorganisms
\item Can be variable in quality
\end{enumerate}

\item Which source of water has the greatest natural protection from bacterial contamination?
\begin{enumerate}
\item Shallow well
\item Deep well
\item Surface water
\item Spring
\end{enumerate}

\item A water-bearing formation in the soil is referred to as
\begin{enumerate}
\item An aquitard or aquiclude
\item An aquifer
\item An aqueduct
\item The drawdown
\item The static water level
\end{enumerate}

\item An operating well will drain the water from a volume of soil around the well during pumping. This volume is referred to as the
\begin{enumerate}
\item Pumping water level
\item Radius of influence
\item Drawdown
\item Cone of depression
\item Recharge zone
\end{enumerate}

\item One acre is 43,560 square feet. If this acre is covered with one foot of water, it contains
\begin{enumerate}
\item 1 acre-foot
\item 43,560 cubic feet
\item 325,829 gallons
\item All of the above
\item None of the above
\end{enumerate}

\item The safe yield of an aquifer is
\begin{enumerate}
\item Determined by the Department of Health Services
\item Variable, depending on rainfall
\item The average amount of water that can be withdrawn each year without causing a long-term drop in the water table
\item The difference between the static water level and the pumping water level
\item All of the above
\end{enumerate}

\item The movement of water from the surface of the earth into the soil is called
\begin{enumerate}
\item Condensation
\item Evaporation
\item Evapotranspiration
\item Runoff
\item None of the above
\end{enumerate}

\item The freezing point of water is
\begin{enumerate}
\item $0\degree{F}$
\item $32\degree{C}$
\item $32\degree{F}$
\item $0\degree{C}$
\item $100\degree{F}$
\end{enumerate}

\item The movement of water from the atmosphere to the surface of the earth is called
\begin{enumerate}
\item Condensation
\item Evaporation
\item Evapotranspiration
\item Runoff
\item Precipitation
\end{enumerate}

\item The movement of water on the surface of the earth is called
\begin{enumerate}
\item Percolation
\item Evaporation
\item Evapotranspiration
\item Runoff
\item Infiltration
\end{enumerate}

\item A formation in the soil that resists water movement (such as a clay layer) is called
\begin{enumerate}
\item An aquitard or aquiclude
\item An aquifer
\item An aqueduct
\item The drawdown
\end{enumerate}

\item Another term for the percolation that transports water from the surface into an aquifer is
\begin{enumerate}
\item Artesian springs
\item Recharge
\item Extraction
\item Overdraft
\item Runoff
\end{enumerate}

\item Water that is safe to drink is called \rule{2cm}{0.3pt} water.
\begin{enumerate}
\item Potable
\item Palatable
\item Good
\item Clear
\end{enumerate}

\item Groundwaters generally have consistent water quality that include
\begin{enumerate}
\item having a higher total dissolved solids content than surface water*
\item having a lower mineral content than surface waters
\item having lower pH values than surface waters
\item having a higher amount of bacteria than surface waters
\end{enumerate}

\item What is the middle layer of a stratified lake called?\\
\begin{enumerate}
\item Thermocline\\
\item Benthic Zone\\
\item Epilimnion\\
\item Hypolimnion
\end{enumerate}

\item  What is the conversion of liquid water to gaseous water known as?\\
\begin{enumerate}
\item Advection\\
\item Condensation\\
\item Precipitation\\
\item Evaporation
\end{enumerate}

\item  Water weighs\\
\begin{enumerate}
\item $7.48 \mathrm{lbs} / \mathrm{gal}$\\
\item $8.34 \mathrm{lbs} / \mathrm{gal}$\\
\item $62.4 \mathrm{lbs} / \mathrm{ft}^{3}$\\
\item Both B. and C.
\end{enumerate}

\item  What is the static level of an unconfined aquifer also known as?\\
\begin{enumerate}
\item Drawdown\\
\item Water Table\\
\item Pumping Water Level\\
\item Aquitard
\end{enumerate}

\item A water bearing geologic formation that accumulates water due to its porousness\\
\begin{enumerate}
\item Aquifer\\
\item Lake\\
\item Aquiclude\\
\item Well
\end{enumerate}

\item  What kind of stream flows continuously throughout the year?\\
\begin{enumerate}
\item Ephemeral\\
\item Perennial\\
\item Intermittent\\
\item Stratified
\end{enumerate}

\item  The surface to atmosphere movement of water is known as\\
\begin{enumerate}
\item Precipitation\\
\item Percolation\\
\item Stratification\\
\item Evapotranspiration
\end{enumerate}

\item  An aquifer that is underneath a layer of low permeability is known as\\
\begin{enumerate}
\item Confined aquifer\\
\item Water Table aquifer\\
\item Unconfined aquifer\\
\item Unreachable groundwater
\end{enumerate}

\item  What is the middle layer of a stratified lake known as?\\
\begin{enumerate}
\item Hypolimnion\\
\item Benthic Zone\\
\item Thermocline\\
\item Epilimnion
\end{enumerate}

\item  The amount of water that can be pulled from a aquifer without depleting\\
\begin{enumerate}
\item Drawdown\\
\item Safe yield\\
\item Overdraft\\
\item Subsidence
\end{enumerate}

  \item The primary origin of coliforms in water supplies is\\
a. Natural algae growth\\
b. Industrial solvents\\
c. Fecal contamination by warm-blooded animals\\
d. Acid raid\\

\item A primary source of volatile organic chemical (VOC) contamination of water supplies is\\
a. Agricultural pesticides\\

b.Industrial solvents\\

c. Acid rain\\

d. Agricultural fertilizers\\

\item The term "surface runoff" refers to\\

a. Rainwater that soaks into the ground\\

b. Rain that returns to the atmosphere from the earth's surface\\

c. Surface water that overflows the banks of rivers\\

d. Water that flows into rivers after a rainfall\\

\item The most important factor to consider in locating a well site from the health point of view is\\
a. Anticipated yield\\
b. Availability of electric power\\
c. Distance from other wells\\
d.  Vulnerability\\

  \item A disease that can be transferred by water is\\
a. Gonorrhea\\
b. Malaria\\
c. Mumps\\
d. Typhoid\\

  \item Final determination of vulnerability is made by\\
a. Private contractor/consultants\\
b. The primacy agency\\
c. The water supplier\\
d. All of the above\\

  \item To prevent the entry of surface contamination into a well is the purpose of\\

a. The well casing\\

b. The water table\\

c. The louvers or slots\\

d. Well development\\

e. The annular grout seal\\

\item Potable water may be defined as\\

a. Water high in organic content\\

b. Any water that occasionally may be polluted from another source\\

c. Any water that, according to recognized standards, is safe for consumption\\

d. Water that indicates a septic condition\\

e. Water that has been transported from outside the service area\\

\item An operating well will drain the water from a volume of soil around the well during pumping. This volume is referred to as the\\
a.	Pumping water level\\
b.	Radius of influence\\
c.	Drawdown\\
d. Cone of depression\\
e.	Recharge zone\\

\item A well screen must be installed in\\
a.	deep wells\\
b.	consolidated materials\\
c.	shallow wells\\
d.	unconsolidated materials

\item A well is acidified in order to\\
a	disinfect\\
b.	increase yield\\
c.	remove objectionable gases\\
d.	 remove disinfection by-products\\

\item The amount of water that a well will produce for each foot of drawdown is called:\\
a.	specific head\\
b.	static yield\\
c.	yield/feet\\
d.	specific capacity\\

\item Surging a well to loosen scale deposits on the screen refers to:\\
a.	turning the pumps on and off as fast as possible to cause a water hammer\\
b.	pumping the water in and out of a well\\
c.	sending shock waves through the aquifer to cause a surge of water\\
d.	using a water jet to surge around the well casing.\\


\item A well is acidized in order to\\
a. Disinfect the water\\
b. Increase yield\\
c. Remove objectionable gasses\\
d. Remove disinfection by-products

\item To prevent the entry of surface contamination into a well is the purpose of\\
a, The well casing\\
b. The water table\\
c. The louvers or slots\\
d. Well development\\
e. The annular grout seal\\

\item The variation in water demand during the course of a day is termed\\
a. Seasonal variation\\
b. Fire flow requirements\\
c. Emergency storage variation\\
d. The straight line equalization method\\
e. Diurnal variation\\

\item The maximum momentary load placed on a water supply system is known as\\
a. Average daily flow\\
b. Average daily demand\\
c. Rated capacity\\
d. A System float\\
e. Peak demand

\item The term aquifer refers to:\\
a. A special type of aqueduct.\\
b. A natural source of water.\\
c. A potable water.\\
d. Water bearing strata.\\


\item The use of a well supply as a source normally results in:\\
a. Water that is high in nitrates\\
b. Water of consistent quality\\
c. Water very high in mineral content\\
d. Water that is considered "soft".\\


\item Maximum Safe Yield of a water source is defined as:\\
a) Where the state health department has approved the source of use.\\
b) The quantity of water that can be taken from a source of supply over a period of years without depleting the source permanently - beyond it's ability to replenish in wet years.\\
c) Water that is free of bacteria.\\
d) Quantity of water that may be treated in the plant.\\

\item Movement of water through the ground is called:\\
a) Hydraulic subsidence\\
b) Runoff\\
c. Percolation\\
d. Infiltration\\



\end{enumerate}
\newpage
%%%%%%%%%%%%%%%%%%%%%%%%%%%%%%%%%%%%%%%%%%%%%%%%%%%%%%%%%%%%%%%
\textbf{Laboratory}
%%%%%%%%%%%%%%%%%%%%%%%%%%%%%%%%%%%%%%%%%%%%%%%%%%%%%%%%%%%%%%%
\begin{enumerate}[1.]

\item Which of the following is an indicator organism?
\begin{enumerate}
\item Giardia
\item Cryptosporidium
\item Hepatitis
\item E. Coli
\end{enumerate}

\item 	What is the primary origin of coliform bacteria in water supplies?
\begin{enumerate}
\item 	Natural algae growth
\item 	Industrial solvents
\item 	Animal or human feces
\item 	Acid rain
\end{enumerate}

\item 	What ls the term for water samples collected at regular intervals and combined in equal volume with each other?
\begin{enumerate}
\item 	Time grab samples
\item 	Timo flow samples
\item Proportional time composite samples
\end{enumerate}

\item 	What is the basis for the number of samples that must be collected for utilities monitoring for lead and copper that are in compliance or have installed corrosion control'?
\begin{enumerate}
\item 	Size of distribution system
\item 	Population
\item 	Amount of water produced
\item 	Number of raw water sources
\end{enumerate}

\item 	Where should bacteriological samples be collected in the distribution system?
\begin{enumerate}
\item 	Uniformly distributed throughout the system based on area
\item 	At locations that are representative of conditions within the system
\item 	Always from extreme locations in the system but occasionally at other locations
\item 	Uniformly throughout the system based on population density
\end{enumerate}
 
\item 	The	quantity of oxygen. that can remain dissolved in water is related to
\begin{enumerate}
\item 	Temperature
\item 	pH
\item 	Turbidity
\item 	Alkalinity
\end{enumerate}

\item 	In coliform analysis using the presence-absence test, a sample should be incubated for	
\begin{enumerate}
\item 	24 hours at 25°C
\item 	36 hours at 35°C
\item 	24 and 36 hours at 25°0
\item 	24 and 48 hours at 35°C
\end{enumerate}

\item A major source of error when obtaining water quality information is improper:
\begin{enumerate}
\item Sampling
\item Preservation
\item Tests of samples
\item Reporting of data
\end{enumerate}

\item What is commonly used as an indicator of potential contamination in drinking water samples?
\begin{enumerate}
\item Viruses
\item Coliform bacteria
\item Intestinal parasites
\item Pathogenic organisms
\end{enumerate}

\item The type of organisms that can cause disease are said to be \rule{2cm}{0.3pt}
microorganisms.
\begin{enumerate}
\item Bad
\item Pathogenic
\item Undesirable
\item Sick
\end{enumerate}

\item Four types of aesthetic contaminants in water include the following:
\begin{enumerate}
\item Odor, turbidity, color, hydrogen sulfide gas
\item Pathogens, microorganisms, arsenic, disinfection by-products
\item Odor, color, turbidity, hardness
\item Color, pathogens, metals, organics
\end{enumerate}

\item What is the purpose of adding fluoride to drinking water?
\begin{enumerate}
\item Increase tooth decay
\item Reduce tooth decay
\item Make teeth white
\item Government conspiracy
\end{enumerate}

\item The test used to determine the effectiveness of disinfection is called the:
\begin{enumerate}
\item Coliform bacteria test
\item Color test
\item Turbidity test
\item Particle test
\end{enumerate}

\item Turbidity is measured as:
\begin{enumerate}
\item mg/L
\item mL
\item gpm
\item NTU
\end{enumerate}

\item Giardia and cryptosporidium are a type of:
\begin{enumerate}
\item Mineral
\item Organism
\item Color
\item Bird
\end{enumerate}

\item Chronic contaminants are those that can cause sickness after:
\begin{enumerate}
\item Prolonged exposure
\item Low levels or low exposure
\end{enumerate}

\item A positive total coliform test indicates that:
\begin{enumerate}
\item Disease-causing organisms may be present in the water supply
\item The water is safe to consume
\item The water supply has high iron levels
\item There is nothing to be concerned about
\end{enumerate}

\item What is the purpose of the bacteriological site sampling plan?
\begin{enumerate}
\item To have a map showing where BacT samples are drawn
\item In case of a positive Bac T sample, the operator will know where to take the
four repeat samples
\item The state will know where you are taking your repeat samples
\item All of the above
\end{enumerate}
\item To ensure that the water supplied by a public water system meets state requirements, the water system operator must regularly collect samples and:
\begin{enumerate}
\item Have water analyzed at an approved water testing laboratory
\item Determine a sampling schedule based on state requirements
\item Send all analyses results to the state
\item All of the above
\end{enumerate}
\item Samples taken for routine bacteriological testing should be preserved by:
\begin{enumerate}
\item Freezing
\item Boiling
\item DPD preservative
\item Refrigeration
\end{enumerate}

\item How many coliform samples are required per month for a water system serving a population between 25 and 100?
\begin{enumerate}
\item 1
\item 2
\item 3
\item 4
\end{enumerate}

\item Before taking a bacteriological (BacT) water sample from a faucet, you should:
\begin{enumerate}
\item Wash hands thoroughly
\item Remove the faucet aerator
\item Flush water until you’re sure water is from the main, not the service line
\item All of the above
\end{enumerate}

\item Monthly BacT samples should be taken from:
\begin{enumerate}
\item The well pump house
\item The distribution system
\item The treatment plant
\item An outside hose spigot
\end{enumerate}

\item If your BacT sample test is positive, how long do you have to collect four repeat samples and deliver them to the lab?
\begin{enumerate}
\item 12 hours
\item 24 hours
\item 48 hours
\item 72 hours
\end{enumerate}

\item \rule{2cm}{0.3pt}is a measure of the capacity of water to neutralize acids.
\begin{enumerate}
\item Concentration
\item Alkalinity
\item pH
\item Conductivity
\end{enumerate}

\item The DPD method is used to determine the \rule{2cm}{0.3pt} of a water sample.
\begin{enumerate}
\item Dissolved oxygen content
\item Conductivity
\item pH
\item Free chlorine residual
\end{enumerate}

\item What color does N,N-diethyl-p-phenylenediamine (DPD) turn in the presence of
chlorine?
\begin{enumerate}
\item Brown
\item Green
\item Blue
\item Pink
\end{enumerate}

\item  The presence-absence (P-A) test used for microbiological testing is also commonly referred to as\\
\begin{enumerate}
\item Multiple Tube Fermentation\\
\item Membrane Filtration\\
\item Confirmed Test\\
\item Colilert
\end{enumerate}

\item  When testing for coliform bacteria with the multiple tube fermentation (MFT) method what is the best indicator for a positive test?\\
\begin{enumerate}
\item Color change\\
\item Gas bubble formation\\
\item Formation of a cyst\\
\item Formation of turbidity
\end{enumerate}

\item  Coliform bacteria share many characteristics with pathogenic organisms. Which of the following is not true?\\
\begin{enumerate}
\item They survive longer in water\\
\item They grow in the intestines\\
\item There are less coliform than pathogenic organisms\\
\item They are still present in water without fecal contamination
\end{enumerate}

\item  What is the second step in the multiple tube fermentation test?\\
\begin{enumerate}
\item Presumptive test\\
\item Negative test\\
\item Completed\\
\item Confirmed
\end{enumerate}

\item What is the removal and deactivation requirement for Giardia?\\
\begin{enumerate}
\item $2 \mathrm{log}$\\
\item $3 \mathrm{log}$\\
\item $4 \mathrm{log}$\\
\item There is no requirement
\end{enumerate}

\item  The multiple barrier approach to water treatment includes removal through which method?\\
\begin{enumerate}
\item Filtration\\
\item Coagulation\\
\item Disinfection\\
\item a and c
\end{enumerate}

\item  A pH reading of 7 is considered\\
\begin{enumerate}
\item Slightly acidic\\
\item Acidic\\
\item Basic\\
\item Neutral
\end{enumerate}

\item EDTA titration is used to determine the \rule{2cm}{0.3pt} of a water sample.
\begin{enumerate}
\item Hardness
\item Conductivity
\item Alkalinity
\item Free chlorine residual
\end{enumerate}

\item  A higher than normal turbidity reading could signify\\
\begin{enumerate}
\item A change in water quality\\
\item Nothing. Keep operating as normal\\
\item Microbiological contamination\\
\item Both $A$ \& $C$
\end{enumerate}

\item  What is the ingredient used during the second multiple tube fermentation test?\\
\begin{enumerate}
\item Colilert\\
\item MMO/MUG\\
\item Brilliant Green Bile\\
\item Chlorine
\end{enumerate}

\item When collecting a distribution system sample for bacteriological testing, the person collecting the sample should allow the water to run before filling the sample bottle.
\begin{enumerate}
\item A minimum of five minutes.
\item 1 hr.
\item 30 min
\item only a few seconds
\end{enumerate}

\item Black stains on plumbing fixtures might be attributed to
\begin{enumerate}
\item calcium.
\item copper.
\item magnesium.
\item manganese.
\end{enumerate}

\item The multiple tube fermentation test consists of three distinct tests. These tests, in the order performed, are the:
\begin{enumerate}
\item preliminary, confirmed, and completed tests.
\item preliminary, presumptive and confirmed tests.
\item presumptive, confirmed, and completed tests.
\item prespumtive, preliminary, and completed tests.
\end{enumerate}

\item What should the sample volume be when testing for total coliform bacteria?
\begin{enumerate}
\item l00mL
\item 250mL
\item 500mL
\item 1,000mL
\end{enumerate}

\item $\mathrm{pH}$ is a measure of :\\
a. conductivity\\
b. water's ability to neutralize acid\\
c. hydrogen ion activity\\
d. dissolved solids\\
\item  Sodium Thiosulfate is used to\\
a. Buffer chlorine solutions\\
b. Neutralize chlorine residuals\\
c. Detect chlorine leaks\\
d. Sterilize sample bottles\\
  \item The presence of total coliforms in drinking water indicates\\
a. The presence of pathogens.\\
b. The absence of an adequate chlorine residual\\
c. The existence of an urgent public health problem\\
d. The potential presence of pathogens\\
\item A primary health risk associated with microorganisms in drinking water is\\
a. Cancer\\
b. Acute gastrointestinal diseases\\
c. Birth defects\\
d. Nervous system disorders\\
  \item After 5 years use, a portion of cast iron pipe shows a white scale about $1 / 2$ inch thick lining the inside. This means\\
a. Red water will soon become a problem\\
b. The water has been corrosive\\
c. The water is chemically unstable and is depositing\\
d. Water should flow easier since the lining is smooth\\
  \item Hardness in water is caused by\\
a. Dissolved minerals\\
b. High $\mathrm{pH}$.\\
c. Low turbidity\\
d. Alkalinity\\
  \item The meniscus on calibrated glassware is read at the\\
a. Bottom of curvature for mercury but the top for water\\
b. Extreme point of contact between the liquid and glass, i.e., where gas, liquid, and air all meet at one point\\
c. Mid-height of the curvature so that beginning and ending readings will results in zero error\\
d. Top of curvature for mercury but at the bottom for most other liquids including water\\
  \item An unknown substance is found on the bottom of the water within a drinking water reservoir. Which of the following statements is true of this substance?\\
a. It has a specific gravity less than $1.0$\\
b. It has a specific gravity equal to $1.0$\\
c. It has a specific gravity greater than $1.0$\\
d. It has no specific gravity\\
e. None of the above\\
  \item The term "Chain of Custody" refers to\\
a. A large accessory to a come-along\\
b. An attachment to a pipe-cutter\\
c. Employee labor laws\\
d. Procedures and documentation required for water quality sampling\\
e. Procedures and documentation required for chemical application\\
  \item Water samples to be analyzed for taste and odor must be\\
a. Analyzed in the field\\
b. Collected in glass sample containers\\
c. Dechlorinated with sodium thiosulfate\\
d. Preserved with dilute hydrochloric acid\\
e. None of the above\\
  \item Bacteriological samples for a distribution system must be collected in accordance with\\
a. The Surface Water Treatment Rule\\
b. OSHA requirements\\
c. An approved sample siting plan\\
d. FLSA requirements\\
e. ANSI/NSF Standard 61\\
  \item Trihalomethanes are classified as\\
a. Metals\\
b. Inorganic constituents\\
c. Secondary drinking water standards\\
d. Radiological contaminants\\
e. Volatile organic compounds\\
 \item The multiple tube fermentation analysis consists of\\
a. Positive, negative, and neutral tests\\
b. Presumptive, confirmed, and completed tests\\
c. Preliminary, presumptive, and confirmed tests\\
d. Preliminary, confirmed, and completed tests\\
e. Presence or absence testing\\
  \item Which of the following is NOT a characteristic of coliform organisms?\\
a. Intestinal origin\\
b. Will produce carbon dioxide from lactose\\
c. Heartier in a water environment than pathogenic organisms\\
d. Far less numerous than pathogenic organisms\\
e. Able to survive with or without oxygen\\
  \item A bacteriological test that measures only the presence or absence of coliforms is\\
a. ColiLert (MMO/MUG)\\
b. Multiple tube fermentation\\
c. Most probable number (MPN)\\
d. Membrane filtration\\
e. Presumptive test\\
  \item After collection, if stored at $4^{\circ} \mathrm{C}$, bacteriological samples must be processed within\\
a. 1 hour\\
b. 6 hours\\
c. 24 hours\\
d. 48 hours\\
e. 72 hours\\
  \item Sample bottles which are furnished by a certified laboratory for collection of bacteriological samples\\
a. Should be rinsed with the water to be sampled before use\\
b. Should be placed in boiling water for at least 10 minutes before use\\
c. Should be rinsed with a chlorine solution before use\\
d. Should be rinsed with distilled water before use\\
e. Are ready to use\\

\item The standard indicator of potential fecal contamination of a water supply is\\
a. Cryptosporidium\\
b. $\mathrm{pH}$\\
c. Alkalinity\\
d. Hardness\\
e. Coliform presence/absence\\

\item Where should bacteriological samples be collected?\\
a. At different locations on each sampling cycle, to make sure the entire system is sampled\\
b. Only from public locations, such as drinking fountains and restrooms\\
c. Only from locations owned by consumers\\
d. Only from specially constructed sampling stations\\
e. From several sampling locations around the entire distribution system, in accordance with a DHS-approved sample siting plan\\
\item Storage of bacteriological samples during transport to a laboratory is best accomplished using\\
a. A clean storage box specifically designed to hold sample containers\\
b. An ice chest packed with ice\\
c. An insulated storage box with "blue ice".\\
d. An insulated storage box with "dry ice"\\
e. No particular sample storage requirements apply, as long as the samples can be delivered to a laboratory prior to the end of the work day\\
  \item Sodium thiosulfate is added in the laboratory to bacteriological sample bottles to:\\
a. Thoroughly disinfect the sample bottle\\
b. -Complete the cleaning and sterilization process\\
c. Neutralize any residual chlorine present in the sample at the time of collection\\
d. Counteract the effects of sunlight on the water sample\\
e. Prevent further growth of bacteria in water samples following collection\\
  \item Radiological contaminant concentrations in drinking water are measured in\\
a. Milligrams per liter\\
b. Micrograms per liter\\
c. Nanograms per liter\\
d. Picograms per liter\\
e. None of the above\\
  \item Which of the following is NOT a characteristic of coliform organisms?\\
a. Intestinal origin\\
b. Will produce carbon dioxide from lactose\\
c. Heartier in a water environment than pathogenic organisms\\
d. Far less numerous than pathogenic organisms\\
e. Able to survive with or without oxygen\\

\item A water supply is found to have a calcium carbonate concentration of 50 mg/L. This water would be considered\\
a.	soft water\\
b.	hard water\\
c.	potable water\\
d.	non-potable water\\

\item Cathodic protection refers to protection against\\
a.	contamination\\
b.	corrosion\\
c.	hardness\\
d.  alkalinity

\item An operator uses \rule{2cm}{0.3pt} to test for residual chlorine\\
a. DPD\\
b. Cresol red\\
c. Methyl orange\\
d. Sulfuric acid\\

\item The meniscus on calibrated glassware is read at the:\\
a. Bottom of curvature for mercury but the top for water\\
b. Extreme point of contact between the liquid and glass, i.e., where gas, liquid, and air all meet at one point\\
c. Mid-height of the curvature so that beginning and ending readings will results in zero error\\
d. Top of curvature for mercury but at the bottom for most other liquids including water

\item The type of corrosion caused by the use of dissimilar metal in a water system is\\
a. Caustic corrosion\\
b. Galvanic corrosion\\
c. Oxygen corrosion\\
d. Tubercular corrosion\\

\item Which of the following can cause tastes and odors in a water supply?\\
a. Dissolved zinc\\
b. Algae\\
c. High pH\\
d. Low pH\\



\end{enumerate}
\newpage
%%%%%%%%%%%%%%%%%%%%%%%%%%%%%%%%%%%%%%%%%%%%%%%%%%%%%%%%%%%%%%%
\textbf{Regulations}
%%%%%%%%%%%%%%%%%%%%%%%%%%%%%%%%%%%%%%%%%%%%%%%%%%%%%%%%%%%%%%%
\begin{enumerate}
\item  What does the acronym $\mathrm{MCL}$ stand for?\\
\begin{enumerate}
\item Minimum contaminant level\\
\item Micron contaminant level\\
\item Maximum contaminant Level\\
\item Milligrams counted last
\end{enumerate}

\item  How long do sanitary surveys have to be retained for records?\\
\begin{enumerate}
\item 3 years\\
\item 5 years\\
\item 7 years\\
\item 10 years
\end{enumerate}

\item The most severe water system violation that requires the fastest public notification\\
\begin{enumerate}
\item Tier I\\
\item Tier II\\
\item Tier III\\
\item Tier IV
\end{enumerate}

\item  The primacy agency may grant a variance or exemption as long as\\
\begin{enumerate}
\item The agency is using the Best Available Technology\\
\item There is no threat to public health\\
\item There is never a scenario for a variance or exemption\\
\item Both A. and B.
\end{enumerate}

\item  A public water system that serves at least 25 people six months out of the year\\
\begin{enumerate}
\item Nontransient noncommunity\\
\item Transient noncommunity\\
\item Community public water system\\
\item None of the above
\end{enumerate}

\item  Regulations based on the aesthetic quality of drinking water\\
\begin{enumerate}
\item Primary Standards\\
\item Secondary Standards\\
\item Microbiological Standards\\
\item Radiological Standards
\end{enumerate}

\item  The lowest reportable limit for a water sample\\
\begin{enumerate}
\item $0.5 \mathrm{mg} / \mathrm{L}$\\
\item Zero\\
\item Public health goal\\
\item Detection Level for reporting
\end{enumerate}

\item  Primary Standards are based on\\
\begin{enumerate}
\item Color and Taste\\
\item Aesthetic quality\\
\item Public Health\\
\item Odor
\end{enumerate}

  \item A disease causing microorganism\\
\begin{enumerate}
\item Pathogen\\
\item Colilert\\
\item Pathological\\
\item Turbidity
\end{enumerate}

\item  According to Surface Water Treatment Rule, what is the combined inactivation and removal for Giardia?\\
\begin{enumerate}
\item 1.0 Logs\\
\item $2.0$ Logs\\
\item 3.0 Logs\\
\item 4.0 Logs
\end{enumerate}

\item  What is the equivalency expressed as a percentage for the SWTR inactivation and removal of viruses?\\
\begin{enumerate}
\item $99.9 \%$\\
\item $99.99 \%$\\
\item $99.0 \%$\\
\item $99.999 \%$
\end{enumerate}

\item  A water agency that takes more than 40 coliform samples must fall under what percentile?\\
\begin{enumerate}
\item $10 \%$\\
\item $7 \%$\\
\item $5 \%$\\
\item No positive samples allowable
\end{enumerate}

%%%%%%%%%%%%%%%%%%%%%%%%%%%%%%%%%%%%%%%%%%%%%%%%%%%%%%%%%%%%%%%%%%%%%%%%%%%%%%%%%%%%%%%%%%%%%%%%%
\item The National Primary Drinking Water Regulations apply to drinking water contaminantR that may have adverse effects on\\
a.	Water color\\
b.	Water taste\\
c.	Water odor\\
d.	Human health\\

\item Which of the following is considered an acute risk to health?\\
a.	Two Tier 2 violations\\
b.	One Tier 2 violation\\
c.	Two Tier 1 violations\\
d.	One Tier 1 violation\\

\item Records on turbidity analyses should be kept for a minimum of\\
a.	5years\\
b.	7 years\\
c.	l0 years\\
d.	25 years\\

\item Records on bacteriological analyses should be kept for a minimum of\\
a.	5 years\\
b.	7 years\\
c.	10 years\\
d.	25 years\\

\item Differecne between primary and secondary standard substances:\\
a.	Primary standards refer to substances that are carcinogenic, secondary standards do not\\
b.	Primary standards refer to substances that are thought to pose a threat to human health, secondary standards do not\\
c.	Primary standards refer to substances that, if not.put in check, will eventually kill humans, secondary standards do not	·\\
d.	Secondary qualities are aesthetic qualities and will only make some people sick, while primary standards refer to substances that will make everyone sick and may possibly cause death\\

\item The SDWA defines a public water system that supplies piped water for human consumption as one that has\\
a.	10 service connection or serves 20 or more people for 60 or more days per year\\
b.	15 service connections or serves 20 or more people for 90 or more days per year\\
c.	10 service connections or serves 25 or more people for 30 or more days per year\\
d-	15 service connections 01' serves 25 or more people for 60 or more days per year\\

\item According to the USEPA regulations, the owner or operator of a public water system that fails to comply with applicable\\
monitoring requirements shall give notice to the public within\\
a.	1 week of the violation in a letter hand-delivered to customers\\
b.	45 days of the violation by posting a notice at the town hall\\
c. 	3 months of the violation in a daily newspaper in the area\\
d.	served by the system 1 year of.the violation by including the notice with the water- bill ·\\

\item What US agency establishes drinking water standards?\\
a.	AWWA\\
b.	USEPA\\
c.	NIOSH\\
d.	NSF\\

\item If a water supply exceeds the MCL, whose responsibility is it to notify the consumer?\\
a.	the testing lab\\
b. 	the supplier\\
c.	the DOH\\
d.	the USEPA\\

\item According to the Lead and Copper Rule. the action for the 90thpercentile lead level is:\\
a.	0. 005 mg/1\\
b.	0. 015 mg/l\\
c.	0. 030 mg/l\\
d.	0.050 mg/l\\


\item The term "maximum contaminant level goal (MCLG)" means the:\\ 
a. Maximum allowable level of a given contaminant in drinking water\\
b. Level of a contaminant .in drinking water below which there are no known or suspected adverse health effects with a margin of safety\\
c. Level of a contaminant in drinking water that will trigger a Tier 1 violation\\
d. Minimum detectable level of a given contaminant\\

\item The maximum contaminant level goal (MCLG) of known or probable carcinogens is:\\
a. Set by the state\\
b. The same number as the maximum contaminant level (MCL)\\
c. Zero\\
d. The minimum detectable level of a given contaminant\\

\item The difference between Tier 1 and Tier 2 violations is:
a. Tier 1 violations·potentially impose·direct and adverse health effects;-Tier 2 violations do not pose a a direct threat to public health.
b. Tier 1 violations require public notification; Tier 2 violations do not require public notification
c. Tier 1 violations are acute; Tier 2 violations are not acute
d. Tier 1 violations have legal consequences; Tier 2 violations do not

\item The Safe Drinking Water Act requires \rule{2cm}{0.3pt} to develop a comprehensive coliform monitoring plan\\
a. Large public water systems (serving >50,000 people)\\
b. Large and medium public water systems (serving >3,300 people)\\
c. Small and medium public water systems (serving >25 and <3,300 people)\\
d. All public water systems\\

\item Final determination of vulnerability is made by:
a. Private contractor/consultants\\
b. The primacy agency\\
c. The water supplier\\
d. All of the above

\item The most important factor to consider in locating a well site from the health point of view is\\
a. Anticipated yield\\
b. Availability of electric power\\
c. Distance from other wells\\
d. Vulnerability\\

\item Trihalomethanes are classified as
a. Metals\\
b. Inorganic constituents\\
c. Secondary drinking water standards\\
d. Radiological contaminants\\
e. Volatile organic compounds\\



\end{enumerate}
\newpage
%%%%%%%%%%%%%%%%%%%%%%%%%%%%%%%%%%%%%%%%%%%%%%%%%%%%%%%%%%%%%%%
\textbf{Treatment}
%%%%%%%%%%%%%%%%%%%%%%%%%%%%%%%%%%%%%%%%%%%%%%%%%%%%%%%%%%%%%%%
\begin{enumerate}
\item What is the recommended loading rate for copper sulfate for algae control at an alkalinity greater than 50 mg/L?
\begin{enumerate}
\item 0.9 lb of copper sulfate per acre of surface area
\item 1.9 lb of copper sulfate per acre of surface area
\item 2-4 lb of copper sulfate per acre of surface area
\item.4 lb of copper sulfate per acre of surface area
\end{enumerate}

If ammonia vapor is passed over a chlorine leak in a cylinder valve, the presence of the leak is indicated by a
\begin{enumerate}
\item Yellow cloud
\item White cloud
\item Gray cloud
\item Brown cloud
\end{enumerate}

\item What is the recommended minimum contact time water mains with the chlorine slug method?
\begin{enumerate}
\item 3 hours
\item 6hours
\item 10 hours
\item. 12 hours·
\end{enumerate}

\item The basic goal for water treatment is to \rule{2cm}{0.3pt}.
\begin{enumerate}
\item Protect public health
\item Make it clear
\item Make it taste good
\item Get stuff out
\end{enumerate}

\item Greensand can be operated in either \rule{2cm}{0.5pt} regeneration or \rule{2cm}{0.5pt} regeneration modes.
\begin{enumerate}
\item Continuous or intermittent
\item Fast or slow
\item Hot or cold
\item Constant or unusual
\end{enumerate}

\item The two most common types of chlorine disinfection by-products include:
\begin{enumerate}
\item TTHM and HAA5
\item TTHA of HMM5
\item Turbidity and color
\item Chloride and fluoride
\end{enumerate}

\item GAC contactors are used to reduce the amount of \rule{2cm}{0.5pt} contaminants in water.
\begin{enumerate}
\item Inorganic
\item Turbidity
\item Particle
\item Organic
\end{enumerate}

\item List the five types of surface water filtration systems.
\begin{enumerate}
\item Bag filtration, cartridge filtration, fine filtration, coarse filtration, media filtration
\item Conventional treatment, direct filtration, slow sand filtration, diatomaceous earth filtration, membrane filtration
\item Turbidity filtration, color filtration, bag filtration, fine filtration, media filtration
\item None of the above
\end{enumerate}

\item Describe two primary methods used to control taste and odor?
\begin{enumerate}
\item Oxidation and adsorption
\item Filtration and sedimentation
\item Mixing and coagulation
\item Sedimentation and clarification
\end{enumerate}

\item The adsorption process is used to remove:
\begin{enumerate}
\item Organics or inorganics
\item Bugs or salts
\item Organisms or dirt
\item Color or particles
\end{enumerate}

\item The solid that adsorbs a contaminant is called the:
\begin{enumerate}
\item Adsorbent
\item Adsorbate
\item Sorbet
\item Rock
\end{enumerate}

\item What is a method of reducing hardness?
\begin{enumerate}
\item Softening
\item Hardening
\item Lightning
\item Flashing
\end{enumerate}


\item Bag and cartridge filters are used to remove which two pathogenic microorganisms?
\begin{enumerate}
\item Viruses and giardia
\item Giardia and cryptosporidium
\item Viruses and bacteria
\item None of the above
\end{enumerate}

\item The process of cleaning a filter by pumping water up through the filter media is called \rule{2cm}{0.3pt} the filter.
\begin{enumerate}
\item Backwashing
\item Rewashing
\item Purging
\item Lifting
\end{enumerate}

\item In a typical water treatment plant, alum would be added into the \rule{2cm}{0.3pt} mixer.
\begin{enumerate}
\item Speed
\item Large
\item Slow
\item Flash
\end{enumerate}

\item When comparing conventional treatment with direct filtration, what process unit is in the conventional treatment plant that is not in the direct filtration plant?
\begin{enumerate}
\item Filter
\item Clarifier
\item Mixer
\item Detention
\end{enumerate}

\item List the basic processes, in the proper order, for a conventional treatment plant.
\begin{enumerate}
\item Coagulation, flocculation, sedimentation, filtration
\item Flocculation, coagulation, sedimentation, filtration
\item Filtration, coagulation, flocculation, sedimentation
\item Coagulation, sedimentation, flocculation, filtration
\end{enumerate}

\item The four most common oxidants include:
\begin{enumerate}
\item Chlorine, potassium permanganate, ozone, chlorine dioxide
\item Chlorides, soap, air, coagulants
\item Air, chemicals, sodium, chloride
\item Flocculants, coagulants, sediments, granules
\end{enumerate}

\item  When operating a filter, one of the operational concerns is the difference between the pressure or head on top of the filter and the pressure or head at the bottom of the filter. This difference is called \rule{2cm}{0.3pt} pressure.
\begin{enumerate}
\item Different
\item Differential
\item High
\item Low
\end{enumerate}

\item  What type of polymer is used to improve the efficiency of the sedimentation
process?
\begin{enumerate}
\item Cationic
\item Nonionic
\item Anionic
\item All of the above
\end{enumerate}

\item A(n) \rule{2cm}{0.3pt} polymer is commonly used as a coagulant.
\begin{enumerate}
\item Anionic
\item Cationic
\item Nonionic
\item Ionic
\end{enumerate}


\item A(n) \rule{2cm}{0.3pt} polymer is used to enhance flocculation.
\begin{enumerate}
\item Anionic
\item Cationic
\item Nonionic
\item Ionic
\end{enumerate}

\item Al2(SO4)3 • 18H20 is the chemical formula for:
\begin{enumerate}
\item Alum
\item Iron
\item Manganese
\item Lead
\end{enumerate}

\item Particles that are less than 1 $\mu$m in size and will not settle easily and are called:
\begin{enumerate}
\item Light particles
\item Colloidal particles
\item Colored particles
\item Flat particles
\end{enumerate}

\item The sedimentation portion of water treatment is also called a(n):
\begin{enumerate}
\item Clarifier
\item Filter
\item Adsorber
\item Water treater
\end{enumerate}

\item Slowly agitating coagulated materials is the process of:
\begin{enumerate}
\item Flocculation
\item Coagulation
\item Sedimentation
\item Filtration
\end{enumerate}

\item The process of decreasing the stability of colloids in water is called:
\begin{enumerate}
\item Flocculation
\item Coagulation
\item Sedimentation
\item Clarification
\end{enumerate}

\item The chemical oxidation process in water treatment is typically used to aid in the
removal of :
\begin{enumerate}
\item Organic contaminants
\item Inorganic contaminants
\item Large contaminants
\item None of the above
\end{enumerate}

\item Flocculation, sedimentation, filtration, and adsorption are \rule{2cm}{0.3pt}
processes.
\begin{enumerate}
\item Physical
\item Chemical
\item Biological
\item Mechanical
\end{enumerate}

\item Oxidation, coagulation, and disinfection are \rule{2cm}{0.3pt} processes.
\begin{enumerate}
\item Physical
\item Chemical
\item Biological
\item Mechanical
\end{enumerate}

\item A precipitate can be formed after which one of the following processes:
\begin{enumerate}
\item Oxidation
\item Flocculation
\item Filtration
\item Adsorption
\end{enumerate}

\item Water that is safe to drink is called \rule{1cm}{0.5pt}  water.
\begin{enumerate}
\item Potable
\item Palatable
\item Good
\item Clear
\end{enumerate}

\item The type of organisms that can cause disease are said to be \rule{1cm}{0.5pt} microorganisms.
\begin{enumerate}
\item Bad
\item Pathogenic
\item Undesirable
\item Sick
\end{enumerate}

\item The basic goal for water treatment is to \rule{1cm}{0.5pt}.
\begin{enumerate}
\item Protect public health
\item Make it clear
\item Make it taste good
\item Get stuff out
\end{enumerate}

\item Four types of aesthetic contaminants in water include the following:
\begin{enumerate}
\item Odor, turbidity, color, hydrogen sulfide gas
\item Pathogens, microorganisms, arsenic, disinfection by-products
\end{enumerate}

\item What does mg/L stand for?
\begin{enumerate}
\item Microorganisms/Liter
\item Milligrams/Loser
\item Milligrams/Liter
\item None of the above
\end{enumerate}

\item Disinfection by-products are a product of:
\begin{enumerate}
\item Filtration
\item Disinfection
\item Sedimentation
\item Adsorption
\end{enumerate}

\item Acute contaminants are those that can cause sickness after:
\begin{enumerate}
\item Prolonged exposure
\item Low levels or low exposure
\end{enumerate}

\item Chronic contaminants are those that can cause sickness after:
\begin{enumerate}
\item Prolonged exposure
\item Low levels or low exposure
\end{enumerate}

\item TTHMs and HAA5s can affect:
\begin{enumerate}
\item Health
\item Aesthetics
\item Color
\item Odor
\end{enumerate}

\item Oxidation, coagulation, and disinfection are \rule{1cm}{0.5pt}  processes.
\begin{enumerate}
\item Physical
\item Chemical
\item Biological
\item Mechanical
\end{enumerate}

\item Flocculation, sedimentation, filtration, and adsorption are \rule{1cm}{0.5pt} processes.
\begin{enumerate}
\item Physical
\item Chemical
\item Biological
\item Mechanical
\end{enumerate}

\item A precipitate can be formed after which one of the following processes:
\begin{enumerate}
\item Oxidation
\item Flocculation
\item Filtration
\item Adsorption
\end{enumerate}

\item Giardia and cryptosporidium are a type of:
\begin{enumerate}
\item Mineral
\item Organism
\item Color
\item Bird
\end{enumerate}

14. The chemical oxidation process in water treatment is typically used to aid in the
removal of :
\begin{enumerate}
\item Organic contaminants
\item Inorganic contaminants
\item Large contaminants
\item None of the above
\end{enumerate}

\item The process of decreasing the stability of colloids in water is called:
\begin{enumerate}
\item Flocculation
\item Coagulation
\item Sedimentation
\item Clarification
\end{enumerate}

\item Slowly agitating coagulated materials is the process of:
\begin{enumerate}
\item Flocculation
\item Coagulation
\item Sedimentation
\item Filtration
\end{enumerate}

\item The sedimentation portion of water treatment is also called a(n):
\begin{enumerate}
\item Clarifier
\item Filter
\item Adsorber
\item Water treater
\end{enumerate}

\item Particles that are less than 1 $\mu\text{m}$ in size and will not settle easily and are called:
\begin{enumerate}
\item Light particles
\item Colloidal particles
\item Colored particles
\item Flat particles
\end{enumerate}

\item One micrometer is also equal to:
\begin{enumerate}
\item 0.1 mm
\item 0.0001 mm
\item 0.001 mm
\item 1 m
\end{enumerate}

\item Particles less than 0.45 $\mu\text{m}$ in size are considered to be:
\begin{enumerate}
\item Dissolved
\item Really little
\item Colored particles
\item Flat particles
\end{enumerate}

\item Turbidity is measured as:
\begin{enumerate}
\item Mg/L
\item mL
\item gpm
\item NTU
\end{enumerate}

\item Al2(SO4)3 • 18H20 is the chemical formula for:
\begin{enumerate}
\item Alum
\item Iron
\item Manganese
\item Lead
\end{enumerate}

\item A(n) \rule{1cm}{0.5pt}  polymer is commonly used as a coagulant.
\begin{enumerate}
\item Anionic
\item Cationic
\item Nonionic
\item Ionic
\end{enumerate}

\item A(n) \rule{1cm}{0.5pt}  polymer is used to enhance flocculation.
\begin{enumerate}
\item Anionic
\item Cationic
\item Nonionic
\item Ionic
\end{enumerate}

\item The concentration of a chemical added to the water is measured in:
\begin{enumerate}
\item mL
\item mg
\item mg/L
\item Liters
\end{enumerate}

\item The quantity of chlorine remaining after primary disinfection is called a
\rule{1cm}{0.5pt}  residual.
\begin{enumerate}
\item Chlorine
\item Permaganate
\item Hot
\item Cold
\end{enumerate}

\item Primary disinfectants are used to \rule{1cm}{0.5pt}  microorganisms.
\begin{enumerate}
\item Hurt
\item Inactivate
\item Burn up
\item Evaporate
\end{enumerate}

\item Secondary disinfectants are used to provide a \rule{1cm}{0.5pt}  in the distribution system.
\begin{enumerate}
\item Color
\item Chemical
\item Smell
\item Residual
\end{enumerate}

\item What type of polymer is used to improve the efficiency of the sedimentation
process?
\begin{enumerate}
\item Cationic
\item Nonionic
\item Anionic
\item All of the above
\end{enumerate}

\item When operating a filter, one of the operational concerns is the difference between the pressure or head on top of the filter and the pressure or head at the bottom of the filter. This difference is called \rule{1cm}{0.5pt}  pressure.
\begin{enumerate}
\item Different
\item Differential
\item High
\item Low
\end{enumerate}

\item List the basic processes, in the proper order, for a conventional treatment plant.
\begin{enumerate}
\item Coagulation, flocculation, sedimentation, filtration
\item Flocculation, coagulation, sedimentation, filtration
\item Filtration, coagulation, flocculation, sedimentation
\item Coagulation, sedimentation, flocculation, filtration
\end{enumerate}

\item The four most common oxidants include:
\begin{enumerate}
\item Chlorine, potassium permanganate, ozone, chlorine dioxide
\item Chlorides, soap, air, coagulants
\item Air, chemicals, sodium, chloride
\item Flocculants, coagulants, sediments, granules
\end{enumerate}

\item When comparing conventional treatment with direct filtration, what process unit is in the conventional treatment plant that is not in the direct filtration plant?
\begin{enumerate}
\item Filter
\item Clarifier
\item Mixer
\item Detention
\end{enumerate}

\item In a typical water treatment plant, alum would be added into the \rule{1cm}{0.5pt}  mixer.
\begin{enumerate}
\item Speed
\item Large
\item Slow
\item Flash
\end{enumerate}

\item The process of cleaning a filter by pumping water up through the filter media is called \rule{1cm}{0.5pt}  the filter.
\begin{enumerate}
\item Backwashing
\item Rewashing
\item Purging
\item Lifting
\end{enumerate}

\item Bag and cartridge filters are used to remove which two pathogenic microorganisms?
\begin{enumerate}
\item Viruses and giardia
\item Giardia and cryptosporidium
\item Viruses and bacteria
\item None of the above
\end{enumerate}

\item List the five types of membrane filtration processes commonly used in water
treatment.
\begin{enumerate}
\item MCF, MF, UF, NF, and RO
\item MNF, MOF, UOF, NOF, and ROO
\item CFM, FM, FU, FN, and OR
\item None of the above
\end{enumerate}

\item What is a method of reducing hardness?
\begin{enumerate}
\item Softening
\item Hardening
\item Lightning
\item Flashing
\end{enumerate}

\item Adsorption of a substance involves its accumulation onto the surface of a:
\begin{enumerate}
\item Solid
\item Rock
\item Pellet
\item Snow ball
\end{enumerate}

\item The solid that adsorbs a contaminant is called the:
\begin{enumerate}
\item Adsorbent
\item Adsorbate
\item Sorbet
\item Rock
\end{enumerate}

\item The adsorption process is used to remove:
\begin{enumerate}
\item Organics or inorganics
\item Bugs or salts
\item Organisms or dirt
\item Color or particles
\end{enumerate}

\item Describe two primary methods used to control taste and odor?
\begin{enumerate}
\item Oxidation and adsorption
\item Filtration and sedimentation
\item Mixing and coagulation
\item Sedimentation and clarification
\end{enumerate}



\item List the five types of surface water filtration systems.
\begin{enumerate}
\item Bag filtration, cartridge filtration, fine filtration, coarse filtration, media filtration
\item Conventional treatment, direct filtration, slow sand filtration, diatomaceous
earth filtration, membrane filtration
\item Turbidity filtration, color filtration, bag filtration, fine filtration, media filtration
\item None of the above
\end{enumerate}

\item GAC contactors are used to reduce the amount of \rule{1cm}{0.5pt}  contaminants in water.
\begin{enumerate}
\item Inorganic
\item Turbidity
\item Particle
\item Organic
\end{enumerate}

\item Greensand can be operated in either \rule{1cm}{0.5pt}  regeneration or \rule{1cm}{0.5pt} regeneration modes.
\begin{enumerate}
\item Continuous or intermittent
\item Fast or slow
\item Hot or cold
\item Constant or unusual
\end{enumerate}

\item  What is the cause of taste and odor problems in raw surface water?\\
\begin{enumerate}
\item Copper sulfate\\
\item Blue-green algae\\
\item Oxygen\\
\item Lake turnover
\end{enumerate}

\item  What chemical reduces blue-green algae growth?\\
\begin{enumerate}
\item Chlorine\\
\item Caustic Soda\\
\item Copper Sulfate\\
\item Alum
\end{enumerate}


\item What is the purpose of adding fluoride to drinking water?
\begin{enumerate}
\item Increase tooth decay
\item Reduce tooth decay
\item Make teeth white
\item Government conspiracy
\end{enumerate}

\item The optimal coagulant dose is determined by a\\
\begin{enumerate}
\item Chlorine Test\\
\item Flocculation test\\
\item Jar Test\\
\item Coagulation test
\end{enumerate}

\item  The most common primary coagulant is\\
\begin{enumerate}
\item Alum\\
\item Cationic polymer\\
\item Fluoride\\
\item Anionic polymer
\end{enumerate}

\item  Bacteria and Viruses belong to a particle size known as\\
\begin{enumerate}
\item Suspended\\
\item Dissolved\\
\item Strained\\
\item Colloidal
\end{enumerate}

\item  The purpose of coagulation is to\\
\begin{enumerate}
\item Increase filter run times\\
\item Increase sludge\\
\item Increase particle size\\
\item Destabilize colloidal particles
\end{enumerate}

\item  The purpose of flocculation\\
\begin{enumerate}
\item Destabilize colloidal particles\\
\item Increase particle size\\
\item Decrease sludge\\
\item Decrease filter run times
\end{enumerate}

\item  Primary coagulant aids used in treatment process are\\
\begin{enumerate}
\item Poly-aluminum chloride\\
\item Aluminum sulfate\\
\item Ferric chloride\\
\item All of the Above
\end{enumerate}

\item  How do water agencies monitor the effectiveness of their filtration process?\\
\begin{enumerate}
\item Alkalinity\\
\item Conductivity\\
\item Turbidity\\
\item $\mathrm{pH}$
\end{enumerate}


\item Flocculation is used to enhance\\
\begin{enumerate}
\item Number of particle collisions to increase floc\\
\item Charge neutralization\\
\item Dispersion of chemicals in water\\
\item Settling speed of floc
\end{enumerate}

\item  If there is a problem with floc formation, what would you consider changing?\\
\begin{enumerate}
\item Adjust coagulant dose\\
\item Stay the course\\
\item Adjust mixing intensity\\
\item Both $A$ \& $C$
\end{enumerate}

\item  Which step in the treatment process is the shortest?\\
\begin{enumerate}
\item Filtration\\
\item Sedimentation\\
\item Flocculation\\
\item Coagulation
\end{enumerate}

\item  To lower the $\mathrm{pH}$ for enhanced coagulation the operator will add\\
\begin{enumerate}
\item Chlorine\\
\item Sulfuric acid\\
\item Lime\\
\item Caustic Soda
\end{enumerate}

\item  The flocculation process lasts how long?\\
\begin{enumerate}
\item Seconds\\
\item 5-10 minutes\\
\item 15-45 minutes\\
\item Over an hour
\end{enumerate}

\item  The function of a flocculation basin is to\\
\begin{enumerate}
\item Settle colloidal particles\\
\item Destabilize colloidal particles\\
\item Mix chemicals\\
\item Allow suspended particles to grow
\end{enumerate}

\item The treatment process that involves coagulation, flocculation, sedimentation, and filtration is known as\\
\begin{enumerate}
\item Direct filtration\\
\item Slow sand Filtration\\
\item Conventional treatment\\
\item Pressure filtration
\end{enumerate}

\item  Sedimentation produces waste known as\\
\begin{enumerate}
\item Backwash water\\
\item Sludge\\
\item Waste water\\
\item Mud
\end{enumerate}

\item  What kind of process is the sedimentation step?\\
\begin{enumerate}
\item Physical\\
\item Chemical\\
\item Biological\\
\item Direct
\end{enumerate}

\item  The weirs at the effluent of a sedimentation basin are also called\\
\begin{enumerate}
\item Effluent weirs\\
\item Baffling\\
\item Launders\\
\item Spokes
\end{enumerate}

\item  Sedimentation is used in water treatment plants to\\
\begin{enumerate}
\item Settle pathogenic material\\
\item Destabilize particles\\
\item Disinfect water\\
\item Reduce loading on Filters
\end{enumerate}

\item  Scouring is a term that describes conditions in a sedimentation tank which\\
\begin{enumerate}
\item Could impact the rest of treatment process\\
\item Higher flow rates in the sludge zone\\
\item Re-suspends settle sludge\\
\item All of the above
\end{enumerate}

The four zones in a Sedimentation basin include\\
\begin{enumerate}
\item Inlet, sedimentation, sludge, outlet\\
\item Inlet, filter, waste, outlet\\
\item Inlet, top, bottom, outlet\\
\item Surface, sedimentation, sludge, outlet
\end{enumerate}

\item The removal and inactivation requirement for Giardia is?\\
\begin{enumerate}
\item $99.9 \%$\\
\item $99.99 \%$\\
\item $99.00 \%$\\
\item $90 \%$
\end{enumerate}

\item Short circuiting in a sedimentation basin could be caused by\\
\begin{enumerate}
\item Surface wind\\
\item Ineffective weir placement, or weirs covered in algae\\
\item Poor baffling in sedimentation inlet zone\\
\item All of the Above
\end{enumerate}

\item How much solids should be removed during sedimentation?\\
\begin{enumerate}
\item $95 \%$ or more\\
\item $80-95 \%$\\
\item $70-80 \%$\\
\item $60-70 \%$
\end{enumerate}

\item The type of basin that includes coagulation and flocculation is\\
\begin{enumerate}
\item Rectangular\\
\item Triangular\\
\item Up-Flow\\
\item None of the above
\end{enumerate}

\item Recarbonation basins are used to stabilize water after
\begin{enumerate}
\item Filtration
\item Disinfection
\item Softening
\item Coagulation
\end{enumerate}

\item Which of the following is an effective way for removing iron water?
\begin{enumerate}
\item 	adding baffles
\item 	adding sodium chloride
\item 	aeration and filtration
\item 	flash mixing
\end{enumerate}

\item How can iron bacteria be controlled in a water distribution system?\\
a.	by aeration\\
b.	filtration\\
c.	chlorination\\
d.	precipitation

\item Which of the following is a hazard when handling hydrofluosilicic acid?\\
a.	fire\\
b.	explosion\\
c.	corrosion\\
d.	inhalation\\

\item Trihalomenthane may be partially removed from water by:\\
a.	fluoridation\\
b.	chlorination\\
c.	oxidation\\
d.	ultraviolet radiation\\

\item Which of the following forms of iron is most soluble in water?\\
a. Ferric (Fe$^{+3}$)\\
b. Ferric hydroxide [Fe(OH$_3$)]\\
c) Ferrous (Fe$^{+2}$)\\
d. Ferrous oxide (FeO)\\

\item Two fundamental treatment requirements for public water systems using surface sources are
a. Coagalat1on and sedimentation
b. Lime softening and disinfection
c. Filtration and aeration 
d. Disinfection and filtration

\item A zeolite softening unit will replace calcium and magnesium ions with \rule{2cm}{0.3pt} ions.\\
a. Fluoride\\
b. Iron\\
c. Sodium\\
d. Sulfur\\

\item One use of polyphosphates is to:\\
a. Control algae\\
b. Improve taste\\
c. Sequester iron and manganese\\
d. Kill bacteria

\item An acceptable means of corrosion control for relatively small systems is\\
a. Activated carbon\\
b. Lime-soda ash softening\\
c. pH control\\
d. zeolite softening

\end{enumerate}
\newpage
%%%%%%%%%%%%%%%%%%%%%%%%%%%%%%%%%%%%%%%%%%%%%%%%%%%%%%%%%%%%%%%
\textbf{Disinfection}
%%%%%%%%%%%%%%%%%%%%%%%%%%%%%%%%%%%%%%%%%%%%%%%%%%%%%%%%%%%%%%%
\begin{enumerate}
\item Disinfection by-products are a product of:
\begin{enumerate}
\item Filtration
\item Disinfection
\item Sedimentation
\item Adsorption
\end{enumerate}

\item Chloramine is most effective as a \rule{2cm}{0.3pt} disinfectant.
\begin{enumerate}
\item Primary
\item Secondary
\item Third
\item First
\end{enumerate}

\item Name the two types of hypochlorites used to disinfect water.
\begin{enumerate}
\item Chloride and monochloride
\item Sodium and calcium
\item Ozone and hydroxide
\item Arsenic and manganese
\end{enumerate}

\item Name two methods commonly used to disinfect drinking water other than chlorination.
\begin{enumerate}
\item Ozone and ultraviolet light
\item Soap and agitation
\item Filtration and adsorption
\item Salt and vinegar
\end{enumerate}

\item In order to determine the effectiveness of disinfection, it is desirable to maintain a disinfectant residual of at least \rule{2cm}{0.3pt} mg/L entering the distribution system.
\begin{enumerate}
\item 0.10
\item 0.5
\item 0.3
\item 0.2
\end{enumerate}

\item Secondary disinfectants are used to provide a \rule{2cm}{0.3pt} in the distribution
system.
\begin{enumerate}
\item Color
\item Chemical
\item Smell
\item Residual
\end{enumerate}

\item Primary disinfectants are used to \rule{2cm}{0.3pt}microorganisms.
\begin{enumerate}
\item Hurt
\item Inactivate
\item Burn up
\item Evaporate
\end{enumerate}

\item The quantity of chlorine remaining after primary disinfection is called a \rule{2cm}{0.3pt} residual.
\begin{enumerate}
\item Chlorine
\item Permaganate
\item Hot
\item Cold
\end{enumerate}

\item The two most common types of chlorine disinfection by-products include:
\begin{enumerate}
\item TTHM and HAA5
\item TTHA of HMM5
\item Turbidity and color
\item Chloride and fluoride
\end{enumerate}

\item In order to determine the effectiveness of disinfection, it is desirable to maintain a disinfectant residual of at least \rule{1cm}{0.5pt}  mg/L entering the distribution system.
\begin{enumerate}
\item 0.10
\item 0.5
\item 0.3
\item 0.2
\end{enumerate}

\item A \rule{1cm}{0.5pt}  residual of chlorine is required throughout the system.
\begin{enumerate}
\item Large
\item High
\item Trace
\item Hot
\end{enumerate}

\item The test used to determine the effectiveness of disinfection is called the:
\begin{enumerate}
\item Coliform bacteria test
\item Color test
\item Turbidity test
\item Particle test
\end{enumerate}


\item Name two methods commonly used to disinfect drinking water other than chlorination.
\begin{enumerate}
\item Ozone and ultraviolet light
\item Soap and agitation
\item Filtration and adsorption
\item Salt and vinegar
\end{enumerate}

\item Name the two types of hypochlorites used to disinfect water.
\begin{enumerate}
\item Chloride and monochloride
\item Sodium and calcium
\item Ozone and hydroxide
\item Arsenic and manganese
\end{enumerate}

\item Free chlorine can only be obtained after \rule{1cm}{0.5pt}  point chlorination has been achieved.
\begin{enumerate}
\item Breakpoint
\item Fastpoint
\item Softpoint
\item Onpoint
\end{enumerate}

\item The meaning of the “C” and the “T” in the term CT stands for:
\begin{enumerate}
\item Concentration and time
\item Color and turbidity
\item Calcium and tortellini
\item Chlorine and turbidity
\end{enumerate}

\item Chloramine is most affective as a \rule{1cm}{0.5pt} disinfectant.
\begin{enumerate}
\item Primary
\item Secondary
\item Third
\item First
\end{enumerate}

\item TTHMs and HAA5s can affect:
\begin{enumerate}
\item Health
\item Aesthetics
\item Color
\item Odor
\end{enumerate}

\item  The multiple barrier treatment approach includes\\
\begin{enumerate}
\item Sterilization and filtration\\
\item Disinfection and filtration\\
\item Disinfection and sterilization\\
\item Infection and filtration
\end{enumerate}

\item The maximum disinfectant residual allowed for chlorine in a water system is\\
\begin{enumerate}
\item $.02 \mathrm{mg} / \mathrm{L}$\\
\item $2.0 \mathrm{mg} / \mathrm{L}$\\
\item $3.0 \mathrm{mg} / \mathrm{L}$\\
\item $4.0 \mathrm{mg} / \mathrm{L}$
\end{enumerate}

\item  What is the disinfectant byproduct caused by ozonation?\\
\begin{enumerate}
\item Trihalomethanes\\
\item Bromate\\
\item Chlorite\\
\item No DBP formation
\end{enumerate}

\item  Haloacitic Acids are also known as\\
\begin{enumerate}
\item TTHM\\
\item $\mathrm{HOCL}$\\
\item Chlorite\\
\item HAA5
\end{enumerate}

\item  What is the MCL for trihalomethanes?\\
\begin{enumerate}
\item $.10 \mathrm{mg} / \mathrm{L}$\\
\item $.06 \mathrm{mg} / \mathrm{L}$\\
\item $.08 \mathrm{mg} / \mathrm{L}$\\
\item $.12 \mathrm{mg} / \mathrm{L}$
\end{enumerate}

\item  What is the MCL for Haloacitic Acids?\\
\begin{enumerate}
\item $100 \mathrm{ppb}$\\
\item $60 \mathrm{ppb}$\\
\item $80 \mathrm{ppb}$\\
\item $120 \mathrm{ppb}$
\end{enumerate}

\item What is the $\mathrm{MCL}$ for bromate?\\
\begin{enumerate}
\item $.010 \mathrm{mg} / \mathrm{L}$\\
\item $.020 \mathrm{mg} / \mathrm{L}$\\
\item $.030 \mathrm{mg} / \mathrm{L}$\\
\item $.040 \mathrm{mg} / \mathrm{L}$
\end{enumerate}

\item What is residual Chlorine?\\
\begin{enumerate}
\item Chlorine used to disinfect\\
\item The amount of chlorine after the demand has been satisfied\\
\item The amount of chlorine added before disinfection\\
\item Film left on DPD kit to measure residual
\end{enumerate}

\item  When Chlorine reacts with natural organic matter in water it can create\\
\begin{enumerate}
\item Disinfectant by-products\\
\item Coliform bacteria\\
\item Chloroform\\
\item Calcium
\end{enumerate}

\item  What are trihalomenthanes classified as\\
\begin{enumerate}
\item Salts\\
\item Inorganic compounds\\
\item Volatile organic compounds\\
\item Radio
\end{enumerate}

\item  What disinfectant is used for emergency purposes and not utilized in the water treatment industry?\\
\begin{enumerate}
\item Chlorine\\
\item Iodine\\
\item Ozone\\
\item Chlorine Dioxide
\end{enumerate}

\item  What is the disinfectant with the least killing power but that has the longest lasting residual?\\
\begin{enumerate}
\item Chlorine\\
\item Ozone\\
\item Chlorine Dioxide\\
\item Chloramines
\end{enumerate}

\item  The active ingredient in household bleach is\\
\begin{enumerate}
\item Calcium hypochlorite\\
\item Calcium hydroxide\\
\item Sodium hypochlorite\\
\item Sodium hydroxide
\end{enumerate}

\item Cryptosporidium is not resistant to this chemical\\
\begin{enumerate}
\item Ozone\\
\item Chlorine Dioxide\\
\item Chlorine\\
\item Both $A$ \& $B$
\end{enumerate}

\item  If a coliform test is positive, how many repeat samples are required at a minimum?\\
\begin{enumerate}
\item None\\
\item 1\\
\item 3\\
\item Depends on the severity of the positive sample
\end{enumerate}

\item  Your water system takes 75 coliform tests per month. This month there were 6 positive samples. What is the percentage of samples which tested positive? Did your system violate regulations?\\
\begin{enumerate}
\item $3 \%$ Yes\\
\item $5 \% \mathrm{No}$\\
\item $8 \%$ Yes\\
\item $10 \%$ No
\end{enumerate}

  \item The form of Chlorine which is $100 \%$ available chlorine is?\\
\begin{enumerate}
\item Sodium Hypochlorite\\
\item Calcium Hypochlorite\\
\item Calcium Hydroxide\\
\item Gaseous Chlorine
\end{enumerate}

\item  What is the minimum amount of chlorine residual required in the distribution system?\\
\begin{enumerate}
\item There is no minimum\\
\item $\mathrm{mg} / \mathrm{L}$\\
\item $0.2 \mathrm{mg} / \mathrm{L}$\\
\item $\mathrm{mg} / \mathrm{L}$
\end{enumerate}

\item  What is the approximate $\mathrm{pH}$ range of sodium hypochlorite?\\
\begin{enumerate}
\item 4-5\\
\item 6-7\\
\item $9-11$\\
\item $12-14$
\end{enumerate}

\item  What is the typical concentration of sodium hypochlorite utilized by water treatment professionals?\\
\begin{enumerate}
\item $5 \%$\\
\item $65 \%$\\
\item $100 \%$\\
\item $12.5 \%$
\end{enumerate}

\item  Chlorine demand refers to\\
\begin{enumerate}
\item Chlorine in the system for a given time\\
\item The difference between chlorine applied and chlorine residual-usually caused by inorganics, organics, bacteria, algae, ammonia, etc.\\
\item Chlorine needed to produce a higher $\mathrm{pH}$\\
\item None of the above
\end{enumerate}

\item  What is the most effective chlorine disinfectant?\\
\begin{enumerate}
\item Dichloramine\\
\item Trichloramine\\
\item Hypochlorite Ion\\
\item Hypochlorous acid
\end{enumerate} 

\item What can form when chlorine reacts with natural organic matter in source water?\\
\begin{enumerate}
\item Disinfectant by-products\\
\item Sulfur\\
\item Algae\\
\item Coliform bacteria
\end{enumerate}

\item  What kind of solution is used to check for a gas chlorine leak?\\
\begin{enumerate}
\item Sodium hydroxide\\
\item Ozone\\
\item Ammonia\\
\item Calcium hypochlorite
\end{enumerate}

\item  Chlorine is\\
\begin{enumerate}
\item Heavier than air\\
\item Lighter than air\\
\item Brown in color\\
\item not harmful to your health
\end{enumerate}

\item  Chlorine demand may vary due to\\
\begin{enumerate}
\item Chlorine demand always stays the same\\
\item Temperature\\
\item $\mathrm{pH}$\\
\item Both B and C
\end{enumerate}

\item  What effect does high turbidity have on disinfection?\\
\begin{enumerate}
\item It can increase chlorine demand\\
\item It has no effect\\
\item It gives the water a milky appearance that will clear out after some time\\
\item You must increase the temperature of the water
\end{enumerate}

  \item What is the target chlorine:ammonia ratio?\\
\begin{enumerate}
\item $2: 1$\\
\item $3: 1$\\
\item $4: 1$\\
\item $5: 1$
\end{enumerate}

\item  What is the MCL for Nitrates?\\
\begin{enumerate}
\item $1 \mathrm{ppm}$\\
\item $10 \mathrm{ppm}$\\
\item $5 \mathrm{ppm}$\\
\item None of the above
\end{enumerate}

\item  What is the molecular weight of Chlorine?\\
\begin{enumerate}
\item 70\\
\item 14\\
\item 65\\
\item 20
\end{enumerate}

\item  What disinfectant has the longest lasting residual?\\
\begin{enumerate}
\item Ozone\\
\item Chlorine\\
\item Chloramine\\
\item Chlorine Dioxide
\end{enumerate}

\item  What are some of the early indicators of Nitrification?\\
\begin{enumerate}
\item Lowering chlorine residual\\
\item Excess ammonia in treated water\\
\item Raise in bacterial heterotrophic plate counts\\
\item All of the above
\end{enumerate}

\item  What are THMs classified as?\\
\begin{enumerate}
\item Turbidity\\
\item Radiological\\
\item Volatile Organic Chemicals\\
\item Salts
\end{enumerate}

\item  What method can operators employ to combat nitrification?\\
\begin{enumerate}
\item Lower residual chlorine target\\
\item Keep reservoir levels static\\
\item Minimize free ammonia in treated water\\
\item Increase water age
\end{enumerate}

\item  How many times stronger is Chlorine compared to monochloramine?\\
\begin{enumerate}
\item 250 times\\
\item 20 times\\
\item 1500 times\\
\item 5 times
\end{enumerate}

\item What chemicals are formed when chlorine is mixed with water?
\begin{enumerate}
\item Hydrogen sulfide and ammonia
\item DPD and carbon dioxide
\item Sodium hypochlorite and calcium hypochlorite
\item Hypochlorous acid and hydrochloric acid
\end{enumerate}

  \item Chlorine residual is measured in the field using the\\
a. Electroconductivity method\\
b. EDTA titrimetric method\\
c. Ortho-tolidine colorimetric method\\
d. DPD colorimetric method\\
e. Differential $\mathrm{pH}$ method\\

\item In nitrification, bacteria consume excess ammonia in the water and produce\\
a. Chloramines\\
b. Free chlorine\\
c. Urine\\
d. Nitrite\\
e. Sodium thiosulfate\\
  \item Which of the following is a form of free chlorine?\\
a. Nitrite\\
b. Hypochlorous acid\\
c. Monochloramine\\
d. Hydrochloric acid\\
e. Trichloramine\\
  \item A distribution system operator measures a total chlorine residual of $1.25 \mathrm{mg} / \mathrm{L}$. How many points on the chlorine breakpoint curve may display this residual?\\
a. Zero\\
b. One\\
c. Two\\
d. Three\\
e. Four\\
  \item What is the chlorine dosage that must be applied when disinfecting a pipeline using the slug method?\\
a. $\quad 300 \mathrm{mg} / \mathrm{L}$\\
b. $\quad 100 \mathrm{mg} / \mathrm{L}$\\
c. $\quad 50 \mathrm{mg} / \mathrm{L}$\\
d. $\quad 25 \mathrm{mg} / \mathrm{L}$\\
e. $\quad 6 \mathrm{mg} / \mathrm{L}$ \\
\item Which of the following is a form of combined chlorine?\\
a. Hypochlorite ion\\
b. Hypochlorous acid\\
c. Monochloramine\\
d. Hydrochloric acid\\
e. Free ammonia\\
 \item A distribution system operator measures a total chlorine residual of $1.25 \mathrm{mg} / \mathrm{L}$, and a free chlorine residual of $1.15 \mathrm{mg} / \mathrm{L}$ : This indicates that\\
a. The system is operating with a chloramine residual\\
b. The chlorine demand is $0.10 \mathrm{mg} / \mathrm{L}$\\
c. The chlorine demand is $2.40 \mathrm{mg} / \mathrm{L}$\\
d. Chloramines are being destroyed by free chlorine\\
e. The system is operating to the right of the breakpoint on the chloramine curve\\
 \item Which of the following is the most desirable form of combined residual chlorine?\\
a. Hypochlorite ion\\
b. Hypochlorous acid\\
c. Monochloramine\\
d. Dichloramine\\
e. Trichloramine\\

  \item Of the following, which is the most effective disinfectant?\\

a. Hypochlorite ion\\
b. Hypochlorous acid\\
c. Monochloramine\\
d. Dichloramine\\
e. Trichloramine\\
  \item A field chlorine residual measurement shows no reading at one minute, but $2.1 \mathrm{mg} / \mathrm{L}$ after three minutes. This indicates that\\
a. The field DPD test kit needs to be returned to the laboratory for maintenance\\
b. There is no chlorine residual\\
c. There is no free chlorine residual, but there are $2.1 \mathrm{mg} / \mathrm{L}$ of chloramines\\
d. There is no combined residual, but the free chlorine residual is $2.1 \mathrm{mg} / \mathrm{L}$\\
e. The analyst should wait an additional three minutes and re-test\\
  \item When disinfecting a storage tank, one method calls for the bottom $6 \%$ of the tank volume to be chlorinated for at least 6 hours with an applied chlorine dosage of\\
a. $\quad 50 \mathrm{mg} / \mathrm{L}$\\
b. $\quad 25 \mathrm{mg} / \mathrm{L}$\\
c. $\quad 6 \mathrm{mg} / \mathrm{L}$\\
d. $\quad 4 \mathrm{mg} / \mathrm{L}$\\
e. $\quad 0.2 \mathrm{mg} / \mathrm{L}$ \\
\item Residual chlorine refers to\\
a. The amount of chlorine in the chlorinated water after several minutes\\
b. The chlorine needed to disinfect the water supply\\
c. The chlorine needed to produce floc in the water\\
d. The sludge in the bottom of the chlorine solution tank\\
e. None of the above\\
 \item While handling sodium hypochlorite, proper safety precautions include
a. Avoiding situations that could splash hypochlorite solution
b. Using a face shield and/or goggles to avoid eye contact
c. Minimizing skin contact with rubber gloves and/or protective clothing
d. All of the above
e. None of the above are necessary\\
  \item The fusible plug that is in all chlorine containers\\
a. Is not necessary\\
b. May be used as a tap for the chlorine source\\
c. Should be removed after the cylinders are empty\\
d. Should never be removed or tampered with\\
e. Should be removed prior to withdrawing chlorine from the container\\
 \item Sodium hypochlorite is a
a. Compound purchased in liquid solution used for disinfection\\
b. Dry neutralizing powder for treating chlorine burns\\
c. Gas delivered in 100-pound, 150-pound, or one-ton containers\\
d. Salt that is formed when hydrochloric acid is neutralized with caustic soda\\
e. None of the above\\
  \item The chlorine demand abruptly jumps in your source water. This may indicate that a. The water source has been contaminated
b. Flow rates in the distribution system have increased\\
c. The hypochlorite solution used for disinfection has deteriorated\\
d. The hypochlorite solution tank is empty\\
e. The hypochlorite ion has a higher concentration than hypochlorous acid\\
  \item The chemical compound typically found in chlorination tablets and granules is\\
a. Sodium hypochlorite\\
b. Sodium hydroxide\\
c. Sodium chloride\\
d. Calcium hypochlorite\\
e. Calcium hydroxide\\ 

\item The maximum rate of withdrawal of gas from a 150-pound chlorine cylinder in 24-hours is\\
a. $\quad 20$ pounds\\
b. $\quad 40$ pounds\\
c. $\quad 100$ pounds\\
d. $\quad 150$ pounds\\
e. None of the above\\
  \item The maximum rate of withdrawal of gas from a one-ton chlorine container in 24-hours is\\
a. $\quad 40$ pounds\\
b. $\quad 100$ pounds\\
c. $\quad 400$ pounds\\
d. One ton\\
e. Variable, depending on chlorine dosage requirements\\
  \item A chlorine leak can be detected by\\
a. An explosimeter\\
b. Checking the leak gauge\\
c. Applying ammonia solution\\
d. A tri-gas detector\\
e. None of the above\\

\item When using the continuous feed method of disinfection, a new water main should be flushed, disinfected at $50 \mathrm{mg} / \mathrm{L}$, and held at above $25 \mathrm{mg} / \mathrm{L}$ for at least\\
a. $\quad 6$ hours\\
b. $\quad 12$ hours\\
c. $\quad 24$ hours\\
d. $\quad 36$ hours\\
e. $\quad 48$ hours\\
  \item If you encounter a liquid chlorine leak in a one-ton container, what action should you take first, to reduce the severity of the leak?\\
a. Apply a caustic solution\\
b. Apply an acidic solution\\
c. Spray the container with water\\
d. Spray the container with an ammonia solution\\
e. Rotate the container to place the leak at the top\\
  \item What should the chlorine dosage be to water that has a chlorine demand of $1.5 \mathrm{mg} / \mathrm{L}$, when a free residual of $1.0 \mathrm{mg} / \mathrm{L}$ is desired?\\
a. $\quad 0.5 \mathrm{mg} / \mathrm{L}$\\
b. $\quad 1.0 \mathrm{mg} / \mathrm{L}$\\
c. $\quad 1.5 \mathrm{mg} / \mathrm{L}$\\
d. $2.5$ pounds per day\\
e. $2.5 \mathrm{mg} / \mathrm{L}$\\
  \item When chlorine reacts with natural organic matter in the water, it is possible to form\\
a. Disinfection by-products \\
b. Arsenic \\
c. MTBE \\
d. Coliforms\\
e. Synthetic organic compounds\\
\item Which of the following best describes the characteristics of chlorine when used for disinfection in drinking water?\\
a.	 Colorless, flammable, heavier than air\\
b. Greenish-yellow, nonflammable, lighter than air\\
c. Greenish-yellow, flammable, lighter than air\\
d.  Greenish-yellow, nonflammable, heavier than air\\
  \item Killing of pathogenic organisms in water treatment is called\\
a. Disinfection\\
b. Oxidätion\\
c. Pasteurization\\
d. Sterilization\\
\item Chlorine reacts with nitrogenous compounds to form\\
a. Ammonia nitrate\\
b. Free chlorine\\
c. Chlorinated hydrocarbons\\
d. Chloramines\\
  \item Sodium Hypochlorite is\\
a. A commercially available chlorine solution\\
b. A commercially available dry chlorine compound\\
c. Chlorine that is available in 100- and 150-pound cylinders\\
d. A reaction product of chlorine and caustic soda\\
\item A hypochlorinator is\\
a. Used to measure residual chlorine\\
b. Used in the treatment of iron and turbidity\\
c. Used to feed a liquid solution into a water supply\\
d. Used to measure an adequate amount of chlorine gas into the supply\\
  \item When calcium hypochlorite is used for disinfecting a water supply, it should be\\
a.	 Dissolved in water, allowed to settle, and the supernatant siphoned off and fed into the water system\\
b. Dissolved in water as a dry chemical then injected into the water system\\
c. Fed as a dry chemical directly into the pipeline\\
d. Fed as a dry powder into the clear well\\
\item The chlorine gas feed rate is usually controlled by adjusting the\\
a. water flow to the injector\\
b. valve on the chlorine cylinder\\
c.pressure in the chlorine cylinder\\
d. rotameter control valve\\
\item If disinfection is incomplete because the chlorine residual is in the hypochlorite ion form, what should you change to improve disinfection?\\
a. Calcium\\
b. Hardness\\
c. pH\\
d. alkalinity\\
\item Breakpoint chlorination is achieved when\\
a. Free ammonia can be tasted in the water\\
b. No chlorine residual is detected\\
c. The strong chlorine tasted at the plant did not persist in the distribution system\\
d. When chlorine dosage is increased, a corresponding increase in residual is detected\\

\item Because chlorine residual is related to the $\mathrm{pH}$ of the water, it may be said that\\
a. A higher $\mathrm{pH}$ requires a higher chiorine residual\\
b. A higher $\mathrm{pH}$ requires a lower chlorine residual\\
c. A lower pH requires a higher chlorine residual\\
d. pH  has no effect on chlorine residual\\


  \item As long as the temperature is steady, the pressure indicator on a chlorine cylinder will until all the chlorine has been gasified\\
a. Remain steady\\
b. Decrease slowly\\
c. Decrease rapidly\\
d. Increase slightly\\

\item When fresh, the typical concentration of sodium hypochlorite solution is\\
a. $\quad 1.25 \%$\\
b. $\quad 6.5 \%$\\
c. $\quad 12.5 \%$\\
d. $\quad 65 \%$\\
e. variable, depending on the manufacturer\\

\item Chlorine in a dry form is called:\\
a.	hypochlorite\\
b.	hypochlorous\\
c.	hydrochlorite\\
d.	hydroxide\\

\item Which of the following procedures is done when preparing to disconnect a chlorine cylinder?\\
a.	close the cylinder valve first to allow time for the chlorine to be drawn off\\
b.	loosen the line to the tank and then shut off the valve to the chlorine cylinder\\
c.	shut off the water supply and allow sufficient time for the chloril1e to be drawn off\\
d.	tum the chlorinator feed rate valve off then turn the valve on the chlorinator cylinder\\


\item A vacuum is formed in the chlorinator by the:\\
a	chlorine cylinder pressure\\
b.	pressure differential through the ejector\\
c.	chlorine feed pump\\
d.	rotameter-\\

\item When calcium hypochlorite is used for disinfecting a water supply, it should be be:\\
a. Dissolved in water, allowed to settle, and the supernatant siphoned off and fed into the water system\\
b. Dissolved in water as a dry chemical then injected into the water system\\
c. Fed as a dry chemical directly into the pipeline\\
d. Fed as a dry powder into the clear well\\

\item Because chlorine residual is related to the pH of the water, it may be said that:
a. A higher pH requires a higher chlorine residual\\
b. A higher pH requires a lower chlorine residual\\
c. A lower pH requires a higher chlorine residual\\
d. A lower pH has no effect on chlorine residual\\

\item If disinfection is incomplete because the chlorine residual is in the hypochlorite ion form, what should one change to improve disinfection?\\
a. Calcium\\
b. Hardness\\
c. pH\\
d. Alkalinity
\end{enumerate}
\newpage

%%%%%%%%%%%%%%%%%%%%%%%%%%%%%%%%%%%%%%%%%%%%%%%%%%%%%%%%%%%%%%%
\textbf{Distribution}
%%%%%%%%%%%%%%%%%%%%%%%%%%%%%%%%%%%%%%%%%%%%%%%%%%%%%%%%%%%%%%%
\begin{enumerate}[1.]
\item The tensile strength of a pipe is its ability to\\ 
a.	Stretch or pull without breakage\\
b.	Resist internal pressure without breakage\\
c.	Resist external pressure without breakage\\
d.	Twist or bend without breakage\\
e.	Resist heating without breakage

\item The lowest point of the inside of a pipe is known as the\\
a	Pervert\\
b.	Soffit\\
c.	Invert\\
d.	Curb stop\\
e.	None of the above

\item A lightweight type of pipe that has a very smooth interior, is essentially corrosion-free, and which is difficult to locate when buried is\\
a.	Polyvinyl chloride\\
b.	Cast iron\\
c.	Ductile iron\\
d.	Concrete cylinder\\
e.	Steel

\item An example of a pipe material that is relatively easy to locate underground is\\
a.	ABS\\
b.	PVC\\
c.	Polyethylene\\
d.	Reinforced concrete cylinder\\
e.	Asbestos-cement

\item \rule{9mm}{.1pt} is a type of valve typically found in a storage tank of a water distribution system it closes to prevent the storage tank from overflowing when a pre-set level is reached\\

a.	Ball valve\\
b.	Altitude valve\\
c.	Gate valve\\
d.	Spring valve\\

\item \rule{9mm}{.1pt} is a valve which opens by lifting a round or rectangular gate/ wedge out of the path of the fluid are designed to fully open or closed service\\

a.	Ball valve\\
b.	Spring valve\\
c.	Altitude valve\\
d.	Gate valve\\

\item A \rule{9mm}{.1pt} is a form of quarter turn valve which uses a hollow perforated and pivoting to control flow through it and is a pivoted 90 degrees by the valve handle.\\
a.	Gate valve\\
b.	Spring valve\\
c.	Ball valve\\
d.	d.  Altitude valve

\item The sudden closure of a check valve will result in\\
a.	water hammer\\
b.	flow reversal\\
c.	cavitation\\
d.	water aeration\\

\item A \rule{9mm}{.1pt} located at the bottom end of suction pipe on a pump this valve opens when the pump operates to allow water to enter the suction pipe but closes when the pump shuts off water from flowing out of the suction pipe\\
a.	Check valve\\
b.	Foot valve\\
c. 	Spring valve\\
d.	Ball valve

\item A valve that automatically shuts off flow into an elevated storage tank when the water level in the tank reaches a preset level is termed a(n)\\
a.	Gate valve\\
b.	Air/ vacuum relief valve\\
c.	Wet-barrel hydrant\\
d.	Altitude valve\\
e.	Angle valve

\item A normally buried valve located on a street water main and leading to a water service is known as\\
a.	Check valve\\
b.	Gate valve \\
c.	Corporation stop\\
d.	Altitude valve\\
e.	Butterfly valve

\item The risk of pipeline damage from water hammer can be reduced by\\ 
a.	Installation of gate valves\\
b.	Air release valves\\
c.	Repair of defective pipes\\
d.	Trimming pump impellers\\
e.	Rapid closing of pump discharge valves

\item The valve type most commonly used for isolation in a water distribution system is:\\
a.	Gate valve\\
b.	Air relief valve\\
c.	Globe valve\\
d.	Ball valve\\
e.	Butterfly valve

\item The proper location for air relief valves is\\
a.	At low points along a pipeline\\
b.	At high points along a pipeline\\
c.	At the bottom of surge tanks\\
d.	At the mid-line of water storage reservoirs\\
e.	At the springline of a pipeline

\item From a sanitary standpoint, the pressure in a distribution system should never be allowed to fall to zero because\\
a.	low pressure allows bacteria to multiply.\\
b.	ground water may enter and back siphonage may occur.\\
c.	the chlorine residual will drop fast.\\
d	the main may collapse.\\

\item When fully open, which of the following will have the highest friction loss?\\
a	Gate valve\\
b.	Butterfly valve\\
c.	Globe valve\\
d.	Ball valve\\
e.	All will have about the same friction loss.\\

\item A nutating disc is found in certain:\\
a.	Centrifugal pumps\\
b.	Positive displacement pumps\\
c.	Main line valves\\
d.	Chemical feeder\\
e.	Water meters\\

\item The drain hole in a fire hydrant is designed to\\
a.	Release air upon closing the valve\\
b.	Relieve vacuum upon opening the valve\\
c.	Allow access for interior inspection\\
d.	Relieve excess water. pressure when closing the valve\\
e.	Remove water from the riser to prevent freezing\\

\item A typical installation site for a compound meter is\\
a.	Any small commercial business\\
b.	A common single location with as many as 12 separate customers\\
c.	A large industrial user\\
d.	Any location that requires the electronic monitoring of peak flows\\
e.	A typical residential water flow meter\\


  \item A main break may cause low pressure in the distributions system, which in turn may result in\\

a. Contamination of the system by backsiphonage\\

b. "ice" formation in the pipes\\

c. Increase in chlorine residual\\

d. Water hammer\\

\item Check valves are used to prevent\\

a. Excessive pump pressure\\

b. Priming\\

c. Water from flowing in two directions\\

d. Water hammer\\

\item The water table is defined as the\\
a. Pumping water level in a well\\
b. Upper surface of the groundwater\\
c. Water level in a reservoir\\
d. Bottom of the aquifer\\

  \item To protect stored water from contamination, a ground storage reservoir should\\


a. Be totally airtight\\

b. Have both the overflow pipe and vent screened\\

c. Have cathodic protection\\

d. Have its interior surface coated with an AWWA-approved paint system\\

\item The peak capacity of water mains is often reduced by\\

a. High pressure\\

b. Looping\\

c. Tuberculation\\

d. Vacuum breakers\\

\item The least amount of head loss in a pipeline would be caused by a fully open\\
a. Angle valve\\
b. Check valve\\
c. Gate valve\\
d. Globe valve\\

  \item The variation in water demand during the course of a day is termed


a. Seasonal variation\\

b. Fire flow requirements\\

c. Emergency storage variation\\

d. The straight line equalization method\\

e. Diurnal variation\\

  \item The maximum momentary load placed on a water supply system is known as\\
a. Average daily flow\\
b. Average daily demand\\
c. Rated capacity\\
d. System float\\
e. Peak demand\\

\item Elevated storage tanks are used primarily to\\
a. Eliminate the need for continuous pumping\\
b. Minimize variations in the system water pressures\\
c. Reduce auxiliary power requirements\\
d. Provide a considerable amount of water for storage\\
e. Protect against backflows

\item A valve that automatically shuts off flow into an elevated storage tank when the water level in the tank reaches a preset level is termed a(n)\\
a. Gate valve\\
b. Air / vacuum relief valve\\
c. Wet-barrel hydrant\\
d. Altitude valve\\
e. Angle valve

  \item Because pipe materials come into contact with drinking water, they must conform with\\
a. Primary drinking water standards\\
b. Secondary drinking water standards\\
c. Surface water treatment rule\\
d. NSF - National Sanitation Foundation\\
d. ANSI/NSF Standard $61^{\prime}$\\
e. All of the above\\

  \item An example of a pipe material that is difficult to locate underground is\\
a. Mortar lined and coated steel\\
b. Reinforced concrete cylinder\\
c. Ductile iron\\
d. Asbestos-cement\\
e. Steel\\

\item Pipe with a " $\mathrm{C}$ " factor of 140 is regarded as having $a(n)$\\
a. Extremely smooth interior\\
b. Extremely rough interior\\
c. Extremely high corrosion resistance\\
d. Extremely low corrosion resistance\\
e. A purple color\\

  \item A lightweight type of pipe that has a very smooth interior, is essentially corrosion-free, and which is difficult to locate when buried is\\
a. Polyvinyl chloride : PVC\\
b. Cast iron\\
c. Ductile iron\\
d. Concrete cylinder\\
e. Steel\\

  \item An example of a pipe material that is relatively easy to locate underground is\\
a. ABS\\
b. PVC\\
c. Reinforced concrete \\
d. Asbestos-cement\\

\item Sleeve-type and "victaulic" couplings are the most common forms of\\

a. Mechanical couplings\\

b. Welded joints\\

c. Asbestos-cement pipe fittings\\

d. PVC pipe fittings\\

e. Flanged joints\\

\item If possible, a water main leak should be repaired under pressure to

a. Prevent contamination of the water line\\

b. Prevent flooding of basements\\

c. Save repair time\\

d. Use fewer materials\\

e. All of the above\\

\item When is the best time to perform a distribution main flushing program?\\
a. During night hours, to minimize traffic and other customer concerns\\
b. During weekday day shift hours, to minimize overtime costs\\
c. During Summer months, due to high system velocities\\
d. During Spring months, prior to high system demands of Summer\\
e. None of the above\\

\item An system for the prevention of corrosion is called\\
a. Water hammer\\
b. Reverse osmosis\\
c. Diurnal variation\\
d. A foot valve\\
e. Cathodic protection\\

\item What category of meters is exemplified by propeller and turbine types?\\
a. Differential pressure\\
b.  Positive displacement \\
c. Mass flow\\
d.  Velocity\\

\item The hydraulic grade line in a pipeline is normally determined by\\
a. Reading pressure gauges\\
b. Checking for backflow\\
c. Opening fire hydrants on each loop of the system\\
d. Using a leak detector\\
e. A venturi meter\\

  \item The slope of the hydraulic grade line is due to\\
a. Well elevations\\
b. Elevations of storage facilities\\
c. Pumping\\
d. Backflows\\
e. Friction loss\\

  \item A normally buried valve located on a street water main and leading to a water service is known as a\\
a. Check valve\\
b. Gate valve\\
c. Corporation stop\\
d. Altitude valve\\
e. Butterfly valve\\

  \item The risk of pipeline damage from water hammer can be reduced by\\
a. Installation of gate valves\\
b. Air release valves\\
c. Repair of defective pipes\\
d. Trimming pump impellers\\
e. Rapid closing of pump discharge valves \\

\item A venturi is a device used to\\
a. Increase water flow\\
b. Decrease water flow\\
c. Regulate water flow\\
d. Stop or start water flow\\
e. Measure water flow\\

  \item The most commonly used meter on small diameter domestic service is the\\
a. Venturi meter\\
b. Propeller meter\\
c. Orifice plate meter\\
d. Compound meter\\
e. Nutating disc meter\\

  \item The valve type most commonly used for isolation in a water distribution system is the\\
a. Gate valve\\

b. Air relief valve\\

c. Globe valve\\

d. Ball valve\\

e. Butterfly valve\\

  \item The proper location for air relief valves is\\
a. At low points along a pipeline\\
b. At high points along a pipeline\\
c. At the bottom of surge tanks\\
d. At the mid-line of water storage reservoirs\\
e. At the springline of a pipeline\\

  \item When fully open, which of the following will have the highest friction loss?\\
a. Gate valve\\
b. Butterfly valve\\
c. Globe valve\\
d. Ball valve\\
e. All will have about the same friction loss.\\

  \item Which of the following is a device used to measure flow?\\
a. Baffle\\
b. Diversion box\\
c. Stop logs\\
d. Weir\\
e. None of the above 

\item A compound meter is a device which\\
a. Is installed to allow automated meter reading\\

b. Can be installed to measure water use by as many as 12 separate customers\\

c. Provides accurate readings over a wide range of flows\\

d. Electronically records peak flows, as a demand meter does for electricity\\

e. Is a typical residential water flow meter\\

\item Magnetic flow meters and ultrasonic flow meters are well suited to measure flow rates of water with a large concentration of suspended solids, because they have\\


a. The best accuracy of any meters\\

b. No parts within the flow stream\\

c. Easily accessed cleanout ports\\

d. Simple recalibration procedures\\

e. All of the above\\

\item A nutating disk is found in certain\\
a. Centrifugal pumps\\

b. Positive displacement pumps\\

c. Main line valves\\

d. Chemical feeders\\

e. Water meters\\

\item The most common valve in a water distribution system is the\\
a. Gate valve\\

b. Air relief valve\\

c. Globe valve\\

d. Ball valve\\

e. Butterfly valve\\

\item The drain hole in a fire hydrant is designed to\\
a. Release air upon closing the valve\\

b. Relieve vacuum upon opening the valve\\

c. Allow access for interior inspection\\

d. Relieve excess water pressure when closing the valve\\

e. Remove water from the riser to prevent freezing\\


\item A typical installation site for a compound meter is\\
a. Any small commercial business\\

b. A common single location with as many as 12 separate customers\\

c. A large industrial user\\

d. Any location that requires the electronic monitoring of peak flows\\

e. A typical residential water flow meter \\

\item An example of a pressure-differential type water meter is a\\
a. Venturi meter\\
b. Propeller meter\\
c. Nutating disk meter\\
d. Magnetic flow meter\\
e. Utrasonic flow meter\\


  \item When closing a hydrant, it should be\\
a. Closed rapidly to minimize water loss\\
b. Closed slowly to reduce surges\\
c. Closed using a standard valve key\\
d. Closed using a standard pipe wrench\\
e. Closed at the street valve and left slightly open at the hydrant valve\\

  \item Dry-barrel fire hydrants have their operating valves\\
a. In the base\\
b. In the head\\
c. Either of the above, depending on the manufacturer\\
d. In the street several feet away from the riser\\
e. None of the above\\

  \item An example of a valve that has a 90 degree travel is a:\\
a. Butterfly valve\\
b. Plug valve\\
c. Ball valve\\
d. All of the above\\
e. None of the above\\

  \item The valve type most commonly found on the discharge of a pump or well, and installed to prevent reverse flows is the\\
a. Gate valve\\
b. Check valve\\
c. Globe valve\\
d. Butterfly valve\\
e. Ball or Plug valve\\

  \item Features that impact the " C " factor for measuring friction in pipelines include\\
a. Pipe length\\
b. Pipe type\\
c. Number of valves\\
d. Type of valves\\
e. All of the above\\

  \item An abnormal flow condition caused by a difference in water pressures is known as:\\
a. Backflow\\

b. Reverse osmosis\\

c. Peak demand\\

d. Fire flow\\

e. Minimum daily requirement\\

\item "Backflow Device" is a term used to describe a device that\\
a. connects three inlet lines with one outlet line\\

b. lets air into valve vaults\\

c. prevents flow of potentially contaminated source into a drinking water supply\\

d. tests for oxygen deficiency in valve vaults\\

e. prevents backflow of water through an out-of-service pump\\

\item A cross-connection means\\
a. Four pipelines tied together\\
b. A T-shaped tool\\
c. A connection between potable water and "unapproved" water supplies\\
d. A backflow caused by negative pressure\\
e. A connection between two or more pressure zones \\

\item Egress is normally required (per OSHA guidelines) for trenches of what minimum depth?\\
a. $\quad 4$-feet\\
b. 5 -feet\\
c. 6 -feet\\
d. $\quad 7$-feet\\
e. 8-feet\\

\item A backflow prevention device that can be used in any cross-connection situation is a\\

a. Pressure vacuum breaker\\

b. Single check valve\\

c. Double check valve\\

d. Reduced pressure zone device\\

e. Atmospheric vacuum breaker\\

  \item A backflow prevention device that is designed for intermittent use in situations where there is no backpressure, such as toilet flush valves and lawn sprinkler systems is a\\
a. Pressure vacuum breaker\\

b. Single check valve\\

c. Double check valve\\

d. Reduced pressure zone device\\

e. Atmospheric vacuum breaker\\


  \item A completely fail-safe means of backflow prevention is\\
a. Atmospheric vacuum breaker\\
b. Pressure vacuum breaker\\
c. Air gap\\
d. Check valve\\
e. Double check valve\\

  \item Two hydraulic conditions can induce backflow. These are backsiphonage and \rule{2cm}{0.3pt}\\
a. Peak flow\\
b. Diurnal flow\\
c. Faulty solenoid valves\\
d. Back pressure\\
e. Fire flow\\

\item When using the continuous feed method of disinfection, a new water main should be flushed, disinfected at $50 \mathrm{mg} / \mathrm{L}$, and held at above $25 \mathrm{mg} / \mathrm{L}$ for at least\\
a. 6 hours\\
b. 12 hours\\
c. 24 hours\\
d. 36  hours\\
e. 48 hours\\

\item To properly disinfect a water main after new construction, you should:\\
a.	 apply 50 mg/l chlorine for 24 hours.\\
b. 	clean the pipe out' with a pig and then disinfect at 10 mg/1 for 24 hours\\
c.	use a 10\% solution of calcium chloride \\
d	 don't use them main for one week\\

\item From a sanitary standpoint. the pressure in a distribution system should never be allowed to fall to zero because:\\
a.	low pressure allows bacteria to multiply\\
b.	ground water may e:oter and back siphonage may occur\\
c.	the chlorine residual will drop faster\\
d.	the main may collapse\\

\item The primary purpose of pressure-reducing valves between water system pressure zones is to\\
a. Minimize surge\\
b. Reduce downstream pressure\\
c. Control flows\\
d. Reduce upstream pressure

\item Because pipe materials come into contact with drinking water, they must conform with\\
a. Primary drinking water standards\\
b. Secondary drinking water standards\\
c. Surface water treatment rule\\
d. ANSI/NSF Standard 61\\
e. All of the above\\

\item An example of a: pipe material that is difficult to locate underground is\\
a. Mortar lined and coated steel\\
b. Reinforced concrete cylinder\\
c. Ductile iron\\
d. Asbestos-cement\\
e. Steel\\

\item A lightweight type of pipe that has a very smooth interior, is essentially corrosion-free, and which is difficult to locate when buried is:\\
a. Polyvinyl chloride\\
b. Cast iron\\
c. Ductile iron\\
d. Concrete cylinder\\
e. Steel

\item Sleeve-type and "victaulic" couplings are the most common forms of\\
a. Mechanical couplings\\
b. Welded joints\\
c. Asbestos-cement pipe fittings\\
d. PVC pipe fittings\\
e. Flanged joints\\

\item The tensile strength of a pipe is its ability to\\
a. Stretch or pull without breakage
b. Resist internal pressure without breakage\\
c. Resist external pressure without breakage\\
d. Twist or bend without breakage\\
e. Resist heating without breakage\\

\item When is the best time to perform a distribution main flushing program?\\
a. During night hours, to minimize traffic and other customer concerns\\
b. During weekday day shift hours, to minimize overtime costs\\
c. During Summer months, due to high system velocities\\
d. During Spring months, prior to high system demands of Summer\\
e. None of the above\\

\item The drain hole in a fire hydrant is designed to\\
a. Release air upon closing the valve\\
b. Relieve vacuum upon opening the valve\\
c. Allow access for interior inspection\\
d. Relieve excess water pressure when closing the valve\\
e. Remove water from the riser to prevent freezing\\

\item A typical installation site for a compound meter is\\
a. Any small commercial business\\
b. A common single location with as many as 12 separate customers\\
c. A large industrial user\\
d. Any location that requires the electronic monitoring of peak flows\\
e. A typical residential water flow meter\\

\item An example of a pressure-differential type water meter is a:\\
a. Venturi meter
b. Propeller meter
c. Nutating disk meter
d. Magnetic flow meter
e. Ultrasonic flow meter

\item When closing a hydrant, it should be\\
a. Closed rapidly to minimize water loss\\
b. Closed slowly to reduce surges\\
c. Closed using a standard valve key\\
d. Closed using a standard pipe wrench\\
e. Closed at the street valve and left slightly open at the hydrant valve\\

\item Dry-barrel fire hydrants have their operating valves\\
a. Inthe base\\
b. In the head\\
c. Either of the above, depending on the manufacturer\\
d. In the street several feet away from the riser\\
e. None of the above\\

\item An example of a valve that has a 90 degree travel is a\\
a. Butterfly valve\\
b. Plug valve\\
c. Ball valve\\
d. All of the above\\
e. None of the above

\item The valve type most commonly found on the discharge of a pump or well, and installed to prevent reverse flows is the\\
a. Gate valve\\
b. Check valve\\
c. Globe valve\\
d. Butterfly valve\\
e. Ball or Plug valve\\


\item Features that impact the "K" factor for measuring friction in pipelines include\\
a. Pipe length\\
b. Pipe type\\
c. Number of valves\\
d. Type of valves\\
e. All of the above\\

\item A potable water supply discharges into an irrigation water storage tank. The 3-inch potable supply line should be terminated\\
a. Above the tank overflow by at least two pipe diameters\\
b. Above the tank outlet by at least two pipe diameters\\
c. Below the tank outlet by at least two pipe diameters\\
d. Level with the tank outlet\\
e. Level with the tank overflow\\

\item A backflow prevention device that is designed for intermittent use in situations where there is no backpressure, such as toilet flush valves and lawn sprinkler systems is a\\
a. Pressure vacuum breaker\\
b. Single check valve\\
c. Double check valve\\
d. Reduced pressure zone device\\
e. Atmospheric vacuum breaker\\

\item A completely fail-safe means of backflow prevention is\\
a. Atmospheric vacunm breaker\\
b. Pressure vacuum breaker\\
C. Air gap\\
d. Check valve\\
e. Double check valve\\

\item Back-siphonage is defined as:\\
a. Back flow that occurs when a vacuum exists.\\
b. Increase in pressure.\\
c. Interconnection between the plumbing systems in the building and water supply.\\
d. Open end of a water supply through which water is discharged in the plumbing fixture.\\

\item A venturi tube increases the velocity and decreases the pressure as water flows through it, This type of tube is used to measure the: ·\\
a. Amount of chlorine in the water.\\
b. Amount of turbidity in the water.\\
c. Rate of aeration.\\
d. Rate of water flowing through it.\\

\item A venturi meter measures flow of a fluid in a pipe based upon the:\\
a. Difference in pressure between a constricted and a fill size portion of the pipe,\\
b. Electronic measurement\\
c. Velocity of the fluid past a given point.\\
d. Weight of the fluid\\

\item Valves are provided in a distribution system to\\
a. Detect any safety hazards.\\
b. Detect weak links in the system.\\
c. Isolate small areas for maintenance and emergency conditions.·\\
d. Reduce costs of maintenance.\\

\item A connection that is made into a main that is under pressure is called a:\\
a. Cross connection\\
b. Dry Tap\\
c. Wet Tap\\
d. Valve Box\\

\item  Because it permits flow in only one direction, which valve would help you determine the direction of the fluid flow?\\
a. Butterfly valve\\
b. Check Valve\\
c. Pressure valve\\
d. Gate valve\\

\item The size of water mains, pumping stations, and storage tanks is primarily
determined by:\\
a. Maximum day demand during a 24 hr. period during the previous year.\\
b. Population served\\
c. Per-capita water use\\
d. Fire protection requirement\\

\item Firefighting may cause low pressure in an area of the distribution system. This low pressure might lead to:\\
a. contamination of the system by back-siphonage\\
b. ice formation in the pipes\\
c. loss of chlorine residual\\
d. None of the above\\
\end{enumerate}




\newpage
%%%%%%%%%%%%%%%%%%%%%%%%%%%%%%%%%%%%%%%%%%%%%%%%%%%%%%%%%%%%%%%
\textbf{Pumping}
%%%%%%%%%%%%%%%%%%%%%%%%%%%%%%%%%%%%%%%%%%%%%%%%%%%%%%%%%%%%%%%
\begin{enumerate}[1.]
\item Vertical turbine pumps that are used in wells may be oil-lubricated or water-lubricated. Operators should use extreme care not to start any water-lubricated pump before making sure that the:\\
a.	Valve on discharge side is open.\\
b. Bearings are dry.\\
c.	Valve on suction side is closed.\\
d.	Bearings are wet.

\item The head against which a pump must operate:\\
a.	Is the sum of the static head and the head due to friction loss.\\
b.	Must always be above the  shut-off  head.\\
c.	Is the static head.\\
d.	Is the friction head.

\item What term describes the condition that exists when the source of the water supply is below the  centerline of the  pump?\\
a.	Pressure  head\\
b.	Velocity head\\
c.	Suction lift\\
d.	Total discharge head

\item What is the most common use today for a positive-displacement pump?\\
a.	Raw water intake pump\\
b.	System booster pump\\
c.	Chemical feed pump\\
d.	Filter feed pump

\item A pumping condition where the eye of the impeller is above the water is called?\\
a.	Dry Well\\
b.	Suction Head\\
c.	Wet Well\\
d.	Suction Lift

\item The force used in an End-suction pump is called\\
a.	Pressure\\
b.	Centrifugal\\
c.	Velocity\\
d.	Kinetic\\

\item \rule{9mm}{0.5pt} is the loss of energy as a result of friction.\\
a.	Velocity loss\\
b.	Headloss\\
c.	Elevation Loss\\
d.	Pump Loss
 
\item As the water travels around the volute towards the discharge line the total energy
shifts from\\
a.	High Velocity Head to low PSI
b.	Low Velocity Head to high PSI
c.	Low Velocity Head to low PSI 
d.	High Velocity Head to high PSI

\item The part that in an End Suction pump that is used to collect the liquid discharged from the impeller is called?\\
a.	Shaft\\
b.	Packing\\
c.	Suction Head\\
d.	Volute\\

\item Head is the energy that a body has by virtue of its position or state.\\
a.	Velocity\\
b.	Potential\\
c.	Kinetic\\
d.	Pressure


\item An impeller that has no shrouds and used to pump fluid with large objects is called?\\
a.	Semi-open\\
b.	Open\\
c.	Closed\\
d.	Very-closed
 
\item A pump station design where the eye of the impeller is submerged in water is called?\\
a.	Dry Well\\
b.	Suction Head\\
c.	Wet Well\\
d.	Suction Lift


\item The discharge valve on a 	pump can be closed for short periods of time or during start up.\\
a.	Piston\\
b.	Progressive Cavity\\
c.	Diaphragm\\
d.	dynamic

\item Velocity of a pump is measured in:\\
a.	Inches per second\\
b.	PSI\\
c.	Feet per second\\
d.	Yards per second\\


\item An impeller that has shrouds on both sides and is used to pump fluid with little or no objects is called?\\
a.	Semi-open\\
b.	Open\\
c.	Closed\\
d.	Very - closed
 
\item To change the discharge of displacement you have to change the:\\
a.	Speed\\
b.	Discharge valve\\
c.	Suction valve\\
d.	Rotation

\item Which pump component prevents leakage from the pump discharge to the suction?\\
a.  Lantern ring\\
b.  Volute\\
c.  Wear ring\\
d.  Shaft sleeve

\item Mechanical seals are being installed in pumps because\\
a.	packing requires an undesirable leakage that seals eliminate.\\
b.	seals prevent cross connections with potable water.\\
c.	seals will take more shaft misalignment than packing.\\
d.	there is a shortage of good packing available on the market.\\

\item A major cause of pump and motor shaft coupling wear is:\\
a.	discharge pressure too high.\\
b.	low suction pressure.\\
c.	misalignment between pumps and motor flanges.\\
d.	worn-out seal.

\item The discharge rate of a piston-type pump:\\

a. Is constant as the main drive rpm changes\\

b. Is constant at a constant speed\\

c. Varies inversely with the head\\

d. Varies with the total dynamic head\\

  \item The flow of electrical current is measured in\\
a. Amperes\\

b. Ohms\\

c. Volts\\

d. Watts\\


\item An operator hears a pinging sound coming from the pump. What is the probable cause?\\
a.	Descaling\\
b. Cavitation\\
c. Corrosion \\
d. Hardness\\

\item During a routine inspection on a centrifugal pump, the operator notices that the bearings are excessively hot. This is most likely caused by:\\
a. Over lubrication\\
b. The speed being too slow\\
c. A worn impeller\\
d. A worn packing\\

\item The-leakage of seal-wateraround-the-packing on a centrifugal pump is required because it acts as a(n)\\

a. Adhesive\\

b. Coolant\\

c. Corrosion inhibitor\\

d. Scale inhibitor\\

\item What can happen to a pump if the back pressure on the pump is allowed to drop too low and the pump is operated for a prolonged period of time?

a. Efficiency would drop off and the pump would heat up

b. No water would flow

c. Pump lubricants would disperse more efficiently

d. Water hammer would occur upstream in the distribution line

\item At a pumping station equipped with centrifugal pumps, what can cause the discharge pressure to suddenly increase and the discharge quantity to suddenly decrease?\\
a. A discharge valve was closed\\
b. A suction valve was closed\\
c. The pump amperage was decreased\\
d. The voltage was suddenly increased\\

\item The component of a centrifugal pump sometimes installed on the end of the suction pipe in order to hold priming is the\\
a. Casing\\
b. Foot valve\\
c. Impeller\\
d. Lantern ring\\

  \item The difference between water levels upstream and downstream of a pump when it is not in operation is known as the\\

a. Suction lift\\

b. Total dynamic head\\

c. Discharge head\\

d. Friction loss\\

e. Total static head\\

\item Static suction head plus friction suction head plus static discharge head plus friction discharge head is a pump's\\

a. Pump curve\\

b. Operating pressure\\

c. Efficiency

d. Total dynamic head\\

e. Velocity head\\

\item Pumps are primed to\\

a. Replace air inside the pump with water\\

b. Seat the valves\\

c. Wet the packing\\

d. Provide water for flow testing\\

e. Overcome positive suction head\\

\item Backspin is occurring after well shutdown; this indicates\\

a. A high water table\\

b. A low water table\\

c. A confined aquifer\\

d. A faulty check valve\\

e. A leak in the sanitary seal\\

  
  
\item A water seal on a pump serves many purposes, including\\

a. Acts as a coolant to keep the pump bearing from overheating\\

b. Keeps gritty material from entering the packing box\\

c. Keeps the pumps primed\\

d. Is a reserve water supply\\

e. Prevents cavitation\\

\item Enclosed, open, and semi-closed are terms used for the designation and selection of:\\

a. Impellers\\

b. Lantern rings\\

c. Sleeves\\

d. Stuffing boxes\\

e. None of the above \\

\item A device that converts electrical energy into mechanical or kinetic energy is called a\\

a. Motor\\

b. Generator\\

c. Transformer\\

d. Battery\\

e. Pump\\

\item If a pump sounds like it is pumping rocks, the most likely cause is\\
a. Cavitation\\

b. Corrosion\\

c. Over-tightening of the packing gland\\

d. Misalignment with the motor\\

e. Irregular wear of the mechanical seal\\

\item The flow of electrical current is measured in\\
a. Amperes\\
b. Volts\\
c. Watts\\
d. Ohms\\
e. Farads\\

  \item The rotating element in a centrifugal pump is commonly called the\\
a. Fan\\

b. Impeller\\

c. Rotor\\

d. Volute\\

e. Stator\\

\item The purpose of the packing in a centrifugal pump is\\

a. Comparable to a bearing and is impregnated with lubricant\\

b. To prevent vibration of the shaft\\

c. To provide support for the impeller\\

d. To surround the bearings and lubricate them\\

e. None of the above\\

  \item Which of the following is a positive displacement pump?\\
a. Air lift pump\\
b. Centrifugal pump\\
c. Reciprocating pump\\
d. Turbine pump\\
e. All of the above

\item The practical maximum suction lift for a centrifugal pump in good condition is \\

a. $\quad 0$ feet\\

b. $\quad 2.31$ feet\\

c. $\quad 14.7$ feet\\

d. 20 feet to 25 -feet\\

e. $\quad 32$-feet to 34-feet

 \item The linkage between a centrifugal pump and its motor is commonly called the\\
a. Coupling\\
b. Impeller\\
c. Bearings\\
d. Volute\\
e. Stator

\item The electrical equivalent to friction in water lines is\\
a. Voltage\\
b: Resistance\\
c. Amperage\\
d. Capacitance\\
e. Inductance

  
\item The main water-containing body of a centrifugal pump is commonly called the\\
a. Shaft\\
b. Impeller\\
c. Bearings\\
d. Volute\\
e. Stator

\item A type of pump that produces high flow rates with low discharge heads is a\\
a. Radial flow\\

b. Axial flow\\

c. Vertical turbine\\

d. Piston\\

e. Mixed flow\\

\item Alternating current is produced by\\
a. A single battery\\
b. Two (or more) batteries in series\\
c. Two (or more) batteries in parallel\\
d. A solenoid\\
e. A generator\\ 

\item What do electrical transformers do?\\
a. Step-up or step-down current\\
b. Step-up or step-down voltage\\
c. Increase power output\\
d. Decrease power output\\
e. Reduce resistance\\

\item An "Open" electrical circuit is one in which\\
a. Resistance is low\\
b. Power production is high\\
c. Capacitance is low\\
d. Conductivity is high\\
e. Amperage is zero

\item Adding more stages (bowls) to a deep well turbine pump assembly will\\

a. Increase the pump discharge capacity\\

b. Decrease the pump discharge capacity\\

c. Increase the pump discharge pressure\\

d. Decrease the pump discharge pressure\\

e. None of the above

\item When installing packing in a centrifugal pump, the packing should be\\
a. Water tight\\
b. Pre-heated\\
c. Staggered $90^{\circ}$\\
d. Soaked overnight in potable water\\
e. Re-used\\

  \item Standard electrical line frequency in the United States is\\
a. $50 \mathrm{~Hz}$\\
b. $\quad 60 \mathrm{~Hz}$\\
c. $\quad 110 \mathrm{~Hz}$\\
d. $\quad 120 \mathrm{~Hz}$\\
e. $240 / 480 \mathrm{~Hz}$\\

\item In contrast to conventional packing, mechanical seals\\
a. Require no adjustment\\
b. Do not leak\\
c. Are generally more expensive\\
d. Are more difficult to remove/replace\\
e. All of the above\\

\item The level of water in a reservoir is 200 feet above the main line that carries water into and out of the reservoir. A standpipe in the main line a block away at the same elevation as the reservoir shows a water elevation of 185 feet. Which of the following statements is true?

a. There is no flow into or out of the reservoir\\

b. Water is flowing into the reservoir\\

c. Water is flowing out of the reservoir\\

d. There is a pump station adjacent to the pressure gauge\\

e. Nothing can be deduced from the information in this question.\\

\item Pump motors draw more power starting than during normal operating conditions because:\\
a.	check valves have to be pushed open\\
b.	energy is required to get the water moving\\
c.	the motor and pump have to start turning\\
d.	all of the above

\item Which of the following does not affect the friction loss in a given length of pipe?\\
a.	hardness of the water\\
b.	number of fittings\\
c.	roughness of the interior of the pipe\\
d.	velocity of the flow\\

\item The component of a centrifugal pump sometimes installed on the end of the suction pipe in order to hold priming is the:\\
a. Casing\\
b. Footvalve\\
c. Impeller\\
d. Lantern ring\\

\item At a pumping station equipped with centrifugal pumps, what can cause the discharge pressure to suddenly increase and the discharge quantity to suddenly decrease?\\
a. A discharge valve was closed\\
b. A suction valve was closed\\
c. The pump amperage was decreased\\
d. The voltage was suddenly increased


\item The inlet to the pump is called:\\
a. Suction\\
b. Volute\\
c. Impeller\\
d. Effluent\\

\item The rotating element in a centrifugal pump is commonly called a(n):\\
a. Fan\\
b. Impeller\\
c. Rotor\\
d. Volute\\

\item Pumps are primed to:\\
a) be sure the pump operates freely\\
b) replace air with water inside the pump\\
c) seat the valves ·\\
d) wet the packing\\
e) none of the above\\

\item The joints in the rings of pump packing should be:\\
a) placed in line
b) placed next to the motor\\
c) placed next to pump\\
d) staggered\\
e) none of the above\\

\item A vertical turbine pump is an example of a :
a) centrifugal pump
b) parshall flume
c) positive displacement pump\\
d) reciprocating pump\\
e) all of the above\\

\item  Which type of pump is most commonly used for high capacity wells?
a) air lift\\
b) centrifugal\\
c) positive displacement\\
d) plunger\\
e) none of the above\\



\end{enumerate}
\newpage
%%%%%%%%%%%%%%%%%%%%%%%%%%%%%%%%%%%%%%%%%%%%%%%%%%%%%%%%%%%%%%%
\textbf{Safety}
%%%%%%%%%%%%%%%%%%%%%%%%%%%%%%%%%%%%%%%%%%%%%%%%%%%%%%%%%%%%%%%
\begin{enumerate}

\item What federal law is designed to protect the safety and health of operators?\\
A.	OSHA\\
B.	FMLA\\
C.	FLSA\\
D.	ADEA

\item What are the two most important safety concerns when entering a confined space?\\
A.	Corrosive chemicals and falls\\
B.	Bad odors and claustrophobia\\
C.	Extreme air temperatures and slippery surfaces\\
D.	Oxygen deficiency and hazardous gases

\item Which document provides a profile of hazardous substances?\\
A.	CERCLA\\
B.	SARA\\
C.	CFR\\
D.	MSDS

\item What is the purpose of a pump guard?\\
A.	Allows operators to turn off pump in emergency situations\\
B.	Notifies operators of excessive temperatures\\
c.	Allows operators to pump against a closed discharge valve\\
D.	Protects operators from rotating parts

\item Atmosphere is considered oxygen deficient when the oxygen level is below\\
A.	21.5\%\\
B.	20\%\\
C.	19.5\%\\
D.  17\%


\item Employee hazards include\\
A. Noxious or toxic gases or vapors\\
B. Oxygen deciency\\
C. Physical injuries\\
D. All of the above\\

\item Before entering a permit-required confined space, you must:\\
A. Check the atmosphere with a calibrated gas detector.\\
B. Make notification that personnel are entering the space.\\
C. Lock out and tag out all equipment.\\
D. All of the above.
	 
\item When making a sulfuric acid dilution, the appropriate method is:\\
A. Add the water to the acid.\\
B. Add the acid to the water.\\
C. Add both at the same time.\\
D. None of the above.

\item When manually lifting any object, be sure to\\
A. Hold it at arm's length.\\
B. Keep your back bent and hold it low.\\
C. Keep it close to your body and use leg strength.\\
D. Keep your knees locked and bend at the waist.


\item What is the proper slope of a ladder?\\
A.	Every 4 feet up the ladder is 1 foot out from the wall.\\
B.	Every 5 feet up the ladder is 1 foot out from the wall. \\
C.	Every 6 feet up the ladder is 1 foot out from the wall.\\
D.	Every 7 feet up the ladder is 1 foot out from the wall.


\item When working on a chemical feed pump, what of the following is not required?\\
A.	Nitrile gloves.\\
B.	Safety glasses.\\
C.	Leather work gloves.\\
D.	Full face shield.

\item When must the atmosphere of a confined space be tested?\\
A.	Only before a worker enters\\
B.	Never, if adequate ventilation exists\\
C.	Continuously\\
D.	Only if welding or painting is being performed

\item Some gases in a confined space can be:\\
A.	Colorless\\
B.	Odorless\\
C.	Deadly\\
D.	All of the above

\item Why should you contact other area companies with underground utilities before starting an underground repair job?\\
a.	 To determine if there have been recent excavations in that location\\
b.	 To ask these companies to mark the location of their utilities in the area of the repair job\\
c.	 To see if they also have excavating to do in the area\\
d.	 To see if they will help route traffic while you are doing the repair job\\
\item The only acceptable breathing device to wear while handling chlorine leaks is the\\
a. Activated carbon canister type\\
b. Potassium tetroxide canister type\\
c. Self-contained breathing apparatus\\
d.	Oxygen supply apparatus\\
\item It is essential to ventilate a vault before entry in order to\\
a. Remove excessive moisture\\
b. Equalize temperature and pressure\\
c. Eliminate foul odors\\
d.	 Remove dangerous gasses\\
\item Permit-required confined space entry requires\\
a. Bright orange jackets, rubber boots, and gloves\\
b. Safety harness and a lifeline\\
c. Tool belts with flashlight attached\\
d. Utility belts with a full complement of tools\\
\item During a confined space entry, how often must the confined space be monitored for hazardous atmospheres?\\
a.  Continuously\\
b.  Every five minutes\\
c. Before entry only\\
d. Before entry and then once per hour during entry\\
  \item Which of the following is the most likely to be a fuel involved in a Class A fire?\\
a. Butane\\
b. Magnesium\\
c. Electrical equipment\\
d. Gasoline\\
e. Paper and/or fabrics\\
  \item In an occupied trench where exits (i.e., ladders) are required, what is the maximum allowed travel distance between an occupant and the nearest exit?\\
i. 25 feet\\
b. 50 feet\\
c. 100 feet\\
d. At the discretion of the safety officer\\
e. None of the above\\
  \item Standard first aid procedures direct that the first step to control bleeding is to\\
a. Apply a tight tourniquet\\
b. Apply pressure directly to the wound\\
c. Let it bleed until natural clotting takes place\\
d. Wash wound and bandage\\
e. None of the above\\
  \item When excavating materials that will not stand in a vertical position, the most suitable form of shoring is\\
a. Air shores\\
b. Hydraulic shores\\
c. Screw jacks\\
d. Solid sheeting\\
e. Cleats\\
  \item A potable water supply discharges into an irrigation water storage tank. The 3-inch potable supply line should be terminated\\
a. Above the tank overflow by at least two pipe diameters\\
b. Above the tank outlet by at least two pipe diameters\\
c. Below the tank outlet by at least two pipe diameters\\
d. Level with the tank outlet\\
e. Level with the tank overflow\\
  \item Which of the following gases is toxic at the lowest concentration?\\
a. Carbon dioxide\\
b. Hydrogen sulfide\\
c. Methane\\
d. Nitrogen\\
e. Oxygen\\
  \item Entry into an atmosphere with high concentrations of chlorine gas requires\\
a. A self-contained breathing apparatus\\
b. An approved and uncontaminated canister mask\\
c. Forced ventilation of the work area\\
d. Atmospheric testing with ammonia solution prior to entry\\
e. Rubber gloves and a full-face shield

\item Shoring is normally required (per OSHA guidelines) for trenches of what minimum depth?\\
a. $\quad 4$-feet\\
b. $\quad 5$-feet\\
c. $\quad 6$-feet\\
d. $\quad 7$-feet\\
e. $\quad 8$-feet\\
  \item First aid for first-degree burns is to\\
a. Bandage tightly\\
b. Cover liberally with salve\\
c. Pack in ice\\
d. Submerge the burned area in cold water\\
e. All of the above\\
  \item What information must be on a warning tag attached to a locked-out switch?\\
a. Directions for removing the tag\\
c. Signature of the person who locked out the switch and who will remove it\\
d. Time to unlock the switch\\
e. None of the above\\
  \item A confined space that contains a material that has the potential for engulfing an entrant is\\
a. A transition zone\\
b. A permit space\\
c. Prohibited by OSHA\\
d. Required to undergo atmospheric testing with ammonia solution prior to entry\\
e. S Required to use a complete "A" suit for personal protective equipment\\
 \item What condition must exist for an area to be considered a confined space?\\
a. Limited or restricted means of entry or exit\\
b. Is large enough for a person to enter and perform work\\
c. Is not designated for continuous occupancy\\
d. All of the above\\
e. None of the above\\
\item Which of the following is the most likely to be a fuel involved in a Class C fire?\\
a. Butane\\
b. Magnesium\\
c. Paper and/or fabrics\\
d. Gasoline\\
e. Electrical equipment\\
  \item Which of the following is the most likely to be a fuel involved in a Class B fire?\\
a. Wood\\
b. Magnesium\\
c. Electrical equipment\\
d. Gasoline\\
e. Paper and/or fabrics\\

\item The angle of repose is the angle of the slope of a\\
a. Sewer\\
b. Graded and/or cut ground elevation\\
c. Trench excavation\\
d. Unsupported loose soil\\
e. Filled and compacted ground elevation\\

\item At least 48 hours prior to conducting excavations in locations where other utilities may be present, whom should you notify?\\
a. WARN\\
b. USA\\
c. AWWA\\
d. DHS\\
e. EPA\\

\item Which of the following compounds emits a "rotten egg" odor?\\
a. Hydrogen sulfide\\
b. Chorine dioxide\\
c. Chloramines\\
d. Hydrochloric acid\\
e. Hypochlorous acid\\

\item Where is the best place to store a self -contained breathing apparatus (SCBA)?\\
a.	inside a cabinet in the chlorinator room\\
b.	in  an unlocked cabinet outside the chlorinator room\\
c.	locked in a cabinet in the office\\
d.	locked in a cabinet just outside the chlorinator room\\

\item Which of the following is a hazard when handling hydrofluosilicic acid?\\
a.	fire\\
b.	explosion\\
c.	corrosion\\
d.	inhalation\\

\item Which of the following chemical substances ii most likely to cause corrosion or deterioration of metal and concrete surfaces\\
a.	carbon dioxide\\
b.	ethanol\\
c.	methane\\
d.	hydrogen sulfide\\


\item An employee ls caught in a room where ch1orine gas is leaking.  He has no SCBA, he should\\
a.	lay down on the floor and quickly crawl out of the room \\
b.	walk out of the room quickly\\
c.	pull shirt over mouth and face and quickly walk out of the room\\
d.	keep mouth closed, head as high as possible, and quickly walk out of the room holding breath.\\



\end{enumerate}


\end{document}