\chapterimage{MathCover.png} % Chapter heading image

\chapter{Math Problems - Part II}

\section{Trickling Filters}\index{Trickling Filters}

\subsection{Example Problems} \index{Example Problems}

\begin{enumerate}


\item The total influent flow (including recirculation) to a trickling filter is 1.89 MGD. If the trickling filter is 80 ft in diameter, what is the hydraulic loading in gpd/sq ft on the trickling filter?\\
Solution:\\
$Hydraulic \enspace loading \enspace \dfrac{gpd}{ft^2}=\dfrac{(1.89*10^6)gpd}{(0.785*80^2)ft^2} =\boxed{376\dfrac{gpd}{ft^2}}$

\item A trickling filter, 70 ft in diameter with a media depth of 6 ft, receives a flow of 0.78 MGD. If the BOD concentration of the primary effluent is 167 mg/L, what is the organic loading on the trickling filter in lbs BOD/day/1000 cu ft?\\
Solution:  $Organic \enspace loading:\dfrac{lbs \enspace BOD}{day-1000ft^3}=\dfrac{lbs \enspace BOD \enspace feed \enspace to \enspace TF \enspace per \enspace day}{volume \enspace in \enspace 1000ft^3}$\\
$=\dfrac{\dfrac{(0.78*167*8.34)lbs \enspace BOD}{day}}{(0.785*70^2*6)ft^3*\dfrac{1000ft^3}{1000ft^3}}=\boxed{\dfrac{47 lbs \enspace BOD}{day-1000 ft^3}}$

\item The suspended solids concentration entering a trickling filter is 236 mg/l. If the suspended solids concentration of the trickling filter effluent is 33 mg/l, what is the suspended solids removal efficiency of the trickling filter?\\
Solution:\\
$\% Removal=\dfrac{236 mg/l-33 mg/l}{236 mg/l}*100=\boxed{86\%}$


\item A trickling filter, 70 ft in diameter with a media depth of 6 ft, receives a flow of 0.78 MGD. If the BOD concentration of the primary effluent is 167 mg/L, what is the organic loading on the trickling filter in lbs BOD/day/1000 cu ft?\\
Solution:\\
$Organic \enspace loading:\dfrac{lbs \enspace BOD}{day-1000ft^3}=\dfrac{lbs \enspace BOD \enspace feed \enspace to \enspace TF \enspace per \enspace day}{volume \enspace in \enspace 1000ft^3}$\\
$=\dfrac{\dfrac{(0.78*167*8.34)lbs \enspace BOD}{day}}{(0.785*70^2*6)ft^3*\dfrac{1000ft^3}{1000ft^3}}=\boxed{\dfrac{47 lbs \enspace BOD}{day-1000 ft^3}}$

\item The influent to the trickling filter is 1.61 MGD. If the recirculated flow is 2.27 MGD, what is the recirculation ratio?\\
Solution:  $R_R=\dfrac{Q_R}{Q_I}=\dfrac{2.27}{1.61}=\boxed{1.4}$\\
\end{enumerate}

\subsection{Practice Problems} \index{Practice Problems}

\begin{enumerate}

\item A trickling filter has a total flow of 32 MGD.  If the recirculation ratio is 0.8, what is the primary effluent flow to the TF?\\


\item A 80-ft diameter trickling filter with a media depth of 7 ft receives a primary influent flow of 2,180,000 MGD. If the BOD concentration of the primary effluent is 139 mg/L, what is the organic loading on the trickling filter in lbs BOD/day/1000 cu ft? (Ans: 72 lbs/day-1000 ft3)\\

\item The desired trickling filter recirculation ratio is 1.4.  If the primary effluent flow is 4.4 MGD what is the trickling filter effluent flow that needs to be recirculated.\\

\item A 90 ft diameter trickling filter treats a 540,000 gpd primary effluent flow. If the recirculated flow to the trickling filter is 120,000 gpd, what is the hydraulic loading on the trickling filter in gpd/ft2.\\

\end{enumerate}


\newpage

\section{Stabilization Ponds}\index{Stabilization Ponds}

\subsection{Example Problems} \index{Example Problems}

\begin{enumerate}

\item What is the surface area in acres of a pond that is 4 feet deep, if it holds 30 million gallons?\\

Solution:\\
$Pond \enspace Volume= Surface \enspace Area*Depth \implies 30MG=Surface \enspace Area*4ft$\\
$ \implies Surface \enspace Area \enspace (acres)=\dfrac{30MG*3.069\dfrac{acre-ft}{MG}}{4ft}=\boxed{23 \enspace acre}$


\item The influent flow to a pond is 10,000 gallons/hour with a suspended solids concentration of 142mg/L in the raw wastewater.  How many lbs of suspended solids are sent to the pond daily?
\\
Solution:\\
$\dfrac{lbs SS}{day}=10000\dfrac{gal}{hr}*\dfrac{24hrs}{day}*\dfrac{MG}{1000000gal}*142\dfrac{mg}{l}*8.34=\boxed{284\dfrac{lbs \enspace SS}{day}}$


\item A pond is 260 ft. long and 80 ft. wide. What is the area of this pond in acres?\\ 
Solution:\\
$(260*80)ft^2*\dfrac{acre}{43,560ft^2}=\boxed{0.48acre}$

\item A pond has a volume of 1,800,000 $ft^3$. If the flow to the pond is 425 gpm, what is the pond detention time in days?
\\
Solution:\\
$DT=\dfrac{volume}{flow}=1,800,000ft^3*\dfrac{1}{425\dfrac{gal}{min}}*\dfrac{day}{1440min}*\dfrac{7.48gal}{ft^3}=\boxed{22days}$

\item Find pond hydraulic loading in inches/day when the depth of the pond is 6 ft. and the detention time is 30 days.\\
Solution:\\

$Hydraulic \enspace Loading \enspace (HL)=\dfrac{flow}{area}$\\
$Detention \enspace time \enspace (DT)=\dfrac{vol}{flow} \implies flow=\dfrac{vol}{DT} $\\
Substituting \enspace for \enspace flow \enspace in \enspace the HL \enspace formula above:\\
$HL=\dfrac{\dfrac{vol}{DT}}{area}\enspace or \enspace \dfrac{vol}{area*DT} \enspace \implies \boxed{HL=\dfrac{pond \enspace depth}{DT}} \enspace as \enspace \dfrac{vol}{area}=pond \enspace depth$\\

$Pond \enspace hydraulic \enspace loading \enspace rate=\dfrac{Pond \enspace depth \enspace (in)}{Pond \enspace detention  \enspace time \enspace \dfrac{Volume}{Flow}}=\dfrac{6*12 \enspace inches}{30 \enspace days}=\boxed{\dfrac{2.4in}{day}}$\\
\vspace{0.5cm}

\item The flow to a pond is 750,000 gpd. If the pond diameter is 100 ft and the BOD in the pond influent is 300 mg/L, what is the organic loading to this pond in lbs BOD/day/acre?
\\
Solution:\\
$Organic \enspace loading=\dfrac{lbs \enspace BOD \enspace per \enspace day}{area \enspace (acres)}=\dfrac{0.75MGD \enspace * \enspace 300mg/l \enspace * \enspace 8.34}{0.785*100^2ft^2}*\dfrac{43,560ft^2}{acre}=\boxed{\dfrac{10,413lbs \enspace BOD}{day-acre}}$
\end{enumerate}

\subsection{Practice Problems} \index{Practice Problems}

\begin{enumerate}

\item A stabilization pond is 1100 ft. long, 600 ft wide, and is operated at a depth of 5 ft. It receives a flow of 500,000 gpd and which has an influent BOD of 185mg/L.  Using this information do the following:
\begin{enumerate}
\item Convert the flow to the pond in units of acre-ft/day.
\item Find the area of this pond in units of acres.
\item Find the volume of pond in units of acre-feet.
\item Calculate the pond detention time in days.
\item Calculate the hydraulic loading to the pond in units of inches per day.
\item Calculate the organic loading to the pond (lbs of BOD/day/acre).
\end{enumerate}

\item  A 2.5 acre stabilization pond is operated at a depth of five (5) ft. What is the pond detention time if the flow to the pond is 18,000 $ft^3$/day? 

\item The flow to a pond is 7.2MGD. If the pond diameter is 350 ft and the BOD in the pond influent is 170mg/L, what is the organic loading to this pond in lbs BOD/day/acre?
\\
\end{enumerate}


\newpage

\section{Activated Sludge}\index{Activated Sludge}

\subsection{Example Problems} \index{Example Problems}

\begin{enumerate}
\item Calculate F/M ratio based on the following data:\\
Secondary influent BOD - 156 mg/l\\
Four (4) aeration basins - 30 ft x 70 ft x 10 ft. deep\\
Influent flow - 0.65 MGD\\
MLSS - 3600 mg/l\\
MLSS average \% volatile - 72\%\\
Solution:\\
\vspace{0.3cm}
$F:M=\dfrac{(lbs/day) \enspace primary \enspace effluent  \enspace BOD \enspace entering \enspace the  \enspace aeration \enspace tank}{(lbs) \enspace MLVSS \enspace in \enspace the  \enspace aeration \enspace tank}$\\
\vspace{0.3cm}
$F:M=\dfrac{156*0.65*8.34}{3600*0.72*4*(30*70*10)ft^3* \dfrac{7.48gal}{ft^3}*\dfrac{MG}{1000000gal}*8.34}=\boxed{0.06}$\\

\item Calculate the MCRT of an activated sludge plant given the following information.\\
Plant flow- 4.25 MGD\\
WAS conc-7980 mg/l\\
Waste flow- 0.055 MGD\\
RAS conc.- 7980 mg/l\\
Aeration tank vol-1MG\\  
Clarifier vol- 0.25 MG\\
Final eff TSS conc. - 21.2 mg/l\\
MLSS conc.- 2050 mg/l\\
\vspace{0.3cm}
Solution:\\
\vspace{0.3cm}
$MCRT (days) =  \dfrac{MLSS \enspace in \enspace aeration \enspace tank \enspace (lbs)+MLSS \enspace in \enspace clarifier \enspace (lbs)}{SS \enspace effluent \enspace (lbs/day)+SS \enspace WAS \enspace (lbs/day)}$\\
\vspace{0.3cm} 
$MLSS \enspace in \enspace aeration \enspace tank \enspace (lbs)=1*2050*8.34=17097lbs$\\
\vspace{0.3cm} 
$MLSS \enspace in \enspace clarifier \enspace (lbs)=0.25*2050*8.34=4274.3lbs$\\
\vspace{0.3cm} 
$SS \enspace effluent \enspace (lbs/day)=4.25MGD *21.2mg/l*8.34=751.4lbs/day$\\
\vspace{0.3cm} 
$SS \enspace WAS \enspace (lbs/day)=0.055MGD *7980mg/l*8.34=3660.4lbs/day$\\
\vspace{0.3cm} 
Plugging in the values calculated above: $MCRT (days) =  \dfrac{17097.6+4274.3}{751.4+3660.4}=4.8=\boxed{5days}$\\
\vspace{0.2cm}


\item A sludge settleability test shows a reading 220 ml after 30 minutes of settling in a one liter graduated cylinder. Lab testing on this mixed liquor shows a MLSS concentration of 1850 mg/L and a MLVSS concentration of 1440 mg/L. Calculate SVI for this mixed liquor sample.\\

Solution:\\
SVI (ml/g)= $\dfrac{Settled \enspace sludge \enspace volume \enspace in \enspace ml/l \enspace after \enspace 30 \enspace min}{MLSS \enspace mg/l}*1000 \dfrac{mg}{g}$\\
\vspace{0.5cm}
SVI=$\dfrac{220ml/l}{1,850mg/l}*1000\dfrac{mg}{g}=\boxed{119ml/g}$


\end{enumerate}

\subsection{Practice Problems} \index{Practice Problems}

\begin{enumerate}

\item What is the food/microorganism ratio given the following conditions:\\
MLSS = 2500 mg/L\\
Secondary Influent BOD$_5$ = 210 mg/L\\
Aeration Tank Volume = 125,000 gallons\\
Primary Effluent BOD$_5$ = 102 mg/L\\
Influent Flow = 235,000 gallons per day\\
Mixed Liquor is 75\% volatile\\

\item Calculate the MCRT given the following.\\
Plant flow - 1.8 MGD\\
MLSS conc -  2800 mg/l\\
WAS flow - 0.04 MGD\\
MLVSS conc. - 2190 mg/l\\
Aerator vol - 0.3 MG\\
Reactor vol. - 0.2 MG\\
RAS conc. - 8150 mg/l\\
Effluent SS conc.-18 mg/l\\

\item In an aeration tank, the MLSS is 2650 mg/l and recorded 30-minute settling test indicates 221 ml/L.  What is the sludge volume index?\\

\item The desired F/M ratio is .35 lbs BOD/day/lb MLVSS. If 2,100 lbs of BOD enter the aerator daily, how many lbs of MLVSS should be maintained in the aeration tank?\\


\end{enumerate}






\newpage

\section{Digestion}\index{Digestion}

\subsection{Example Problems} \index{Example Problems}

\begin{enumerate}

\item Calculate the volatile solids reduction in an anaerobic digester given the following information: Raw sludge feed to digester: 73.7 \% VS and digested sludge: 57.2 \%VS\\

Solution:\\
\vspace{0.5cm}
$Digester \enspace VS \enspace reduction (\%)=\dfrac{VS_{in}-VS_{out}}{VS_{in}-VS_{in}*VS_{out}}*100$\\
\vspace{0.5cm}
$Digester \enspace VS \enspace reduction (\%)=\dfrac{0.737-0.572}{0.737-0.737*0.572}*100=\boxed{ 52.3\%}$\\

\item 10,000 gallons/day of sludge is pumped to an anaerobic digester/day at 4\% solids (70\% VS).  If 50\% of the VS is destroyed, how many lbs of VS is destroyed per day?\\
Solution:\\
$\dfrac{10,000 \enspace Gal}{day}*\dfrac{8.34 \enspace lbs \enspace sludge}{Gal} \dfrac{0.04*0.7 \enspace lbs \enspace VS \enspace feed}{lb \enspace sludge}*\dfrac{0.5 \enspace lbs \enspace VS \enspace destroyed}{lbs \enspace VS \enspace feed}=\boxed{\dfrac{1,168 \enspace lbs \enspace VS \enspace destroyed}{day} } $

\item How many pounds of solids are pumped to a digester each day if the digester receives 10,000 gpd of sludge at 5\% solids concentration?\\

Solution:\\

{
$
	\dfrac{lbs \enspace TS}{day}
	=
	\dfrac{10,000 gal \enspace sludge}{day}
	*
	\dfrac{(8.34*0.05 lbs TS )}{gal \enspace sludge}
	=4,170
	\dfrac{lbs \enspace TS}{day}
$
}\\

\item Calculate the VS loading to the digester in lbs/day if 10,000 gallons of sludge containing 5\% TS with and average VS content of 78\%\\
Solution:\\
Digester VS loading (lbs/day)\\$=\dfrac{10,000 \enspace gallons \enspace sludge}{day}*\dfrac{8.34lbs \enspace sludge}{gal}*\dfrac{0.05*0.78lbs VS}{lb \enspace sludge}=\boxed{3,253lbs \enspace sludge \enspace per \enspace day}$




\end{enumerate}

\subsection{Practice Problems} \index{Practice Problems}

\begin{enumerate}

\item 10,000 gallons/day of sludge is pumped to an anaerobic digester/day at 4\% solids (70\% VS).  If 50\% of the VS is destroyed, how many lbs of VS is destroyed per day?\\

\item Primary sludge containing five percent (5\%) solids is pumped to a digester continuously at a rate of 25 gpm. How many pounds of volatile solids are added to the digester each day if the volatile solids content is 73\% of the total solids?\\

\item Calculate the \% VS reduction in a digester given the volatile solids content of the influent sludge to the digester is 70\% and the volatile solids content of the sludge leaving the digester is 52.5\%\\

\end{enumerate}

\newpage
\section{Chlorine Disinfection}\index{Chlorine Disinfection}

\subsection{Example Problems} \index{Example Problems}

\begin{enumerate}

\item Calculate how many pounds per day of chlorine should be used to maintain a dosage of 12 mg/l at a 5.0 MGD flow.\\
Solution:\\
$lbs/day=conc. (mg/l)*flow(MGD)*8.34$\\
$lbs/day=12*5*8.34=\boxed{500.4lbs/day}$\\
\item What is the chlorine demand if the chlorine dosage is 15 mg/L and the residual is 3 mg/l?
Solution:\\
Chlorine dosage = chlorine demand + chlorine residual\\
$\implies chlorine \enspace demand = chlorine \enspace dosage - chlorine \enspace residual=15-3=\boxed{12mg/l}$
\item If 80 pounds of chlorine are applied each day to a flow of 1.5 MGD, what is the dosage in mg/l?\\
Solution:\\
Applying the pounds formula:\\  $lbs/day=conc. (mg/l)*flow(MGD)*8.34$\\
$\implies conc. (mg/l)=\dfrac{lbs/day}{flow(MGD)*8.34}=\dfrac{80}{1.5*8.34}=\boxed{6.4mg/l}$
\item How many pounds per day of chlorine will be required to disinfect a secondary effluent flow of 1.68 MGD if the chlorine demand is found to be 8.5 mg/l and a residual of 3 mg/l is desired?
Chlorine dosage = chlorine demand + chlorine residual\\
$chlorine \enspace dosage=8.5+3=11.5mg/l$\\
$lbs/day=conc. (mg/l)*flow(MGD)*8.34=1.68*11.5*8.34=\boxed{161.2lbs/day}$\\


\end{enumerate}

\subsection{Practice Problems} \index{Practice Problems}

\begin{enumerate}


\item The chlorine demand is 4.8 mg/l and a chlorine residual is 0.75 mg/l is desired. For a flow of 2.8 MGD, how many pounds per day should the chlorinator be set to deliver.\\

\item Chlorine is being fed at the rate of 75 pounds per day. Plant flow is 1.2 MGD. The chlorine residual is measured and found to be 2.6 mg/L Calculate chlorine demand.\\


\end{enumerate}

