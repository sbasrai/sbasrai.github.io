\chapterimage{Water1.png} % Chapter heading image

\chapter{Wastewater Math - Part II}


\section{Pounds Formula}



Pounds formula is used for:
\begin{itemize}
\item Calculating the quantity in pounds of a particular wastewater constituent entering or leaving a wastewater treatment process
\item Calculating the pounds of chemicals to be added\\
\end{itemize}
So if the concentration of a particular constituent (in mg/liter) and the volume or flow of wastewater is given, one can calculate the amount of that constituent in pounds using the following – Pounds Formula:
$$lbs \enspace \textbf{or} \enspace \dfrac{lbs}{day}=concentration(\dfrac{mg}{l})*8.34*volume(MG) \enspace \textbf{or} \enspace flow(\dfrac{MG}{day}(MGD)$$

\begin{figure}[h!]
\begin{center}
\includegraphics[scale=0.5]{PoundsFormula}
\end{center}
\caption{Pounds formula "nomograph"}
\end{figure}
\vspace{0.3cm}
There are three variables – (lbs, concentration and volume) and one constant (8.34) in the pounds formula.  Knowing any of the two variables in the formula, one can calculate the third (unknown) variable by rearranging the equation.
\subsection{Example Problems}
% \hl{Example Problems}\\
\begin{enumerate}

\item If the influent wastewater flow is 5 MGD and the BOD concentration is 240 mg/l what is the daily BOD loading in lbs/day?\\
Solution:\\
$\dfrac{lbs \enspace BOD}{day}=5MGD*240mg/l*8.34=\boxed{\dfrac{10,000lbs}{day}}$\\

\item Calculate the lbs of solids in the primary sludge if the sludge flow is 7500 gallons and the solids concentration is 4.5\%.\\
Solution\\
Applying lbs formula:\\
$lbs \enspace solids = \dfrac{7500 \enspace MG}{1,000,000} * 4.5*10,000 *8.34 = \boxed{2,815 \enspace lbs \enspace solids}$\\
\textbf{Note:}\\  
1) 7500 gallons was converted to MG by dividing by 1,000,000\\
$7500 \enspace gallons * \dfrac{1 MG}{1,000,000 \enspace gallons}$\\
2) 4.5\% was converted to mg/l by multiplying by 10,000 as 1\%=10,000mg/l

\item An operator dissolves 1,200 lbs of a chemical in 12,000 gallons of water, what is the resultant concentration in mg/l, of the chemical solution?\\
Solution:\\
$Concentration \enspace \dfrac{mg}{l}=\dfrac{lbs}{Volume \enspace MG \enspace * \enspace 8.34}$\\
$Concentration \enspace \dfrac{mg}{l}=\dfrac{1,200}{0.012 \enspace * \enspace 8.34}=\boxed{\dfrac{11,990 \enspace mg}{l} \enspace or \enspace 1.2\% \enspace solution}$\\
\textbf{Note:}\\  
1) 12,000 gallons was converted to MG by dividing by 1,000,000\\
$12,000 \enspace gallons * \dfrac{1 MG}{1,000,000 \enspace gallons}$\\
\end{enumerate}
\section{Process Removal Efficiency Calculations}\index{Process Removal Efficiency Calculations}

\begin{itemize}
\item Process removal rate or removal efficiency is the percentage of the inlet concentration removed.  
\item It is used for quantifying the pollutant removal during wastewater treatment and is established based upon the amount of a particular wastewater constituent entering and leaving a treatment process.

\item $Process \enspace Removal \enspace Rate \enspace (\%) = \dfrac{Pollutant \enspace  In-Pollutant\enspace  Out}{Pollutant \enspace In}*100$\\

\item If 10 units of a pollutant are entering a process and 8 units of pollutant are leaving (process removes 2 units), then the process removal rate for that pollutant is (10-8)/10*100=20\%.  In this example the process is 20\% efficient in removing that particular pollutant.

\item The amount of pollutant can be measured in terms of concentration (mg/l) or in terms of mass loading (lbs).  The pounds formula is used for calculating the mass loadings.  
\end{itemize}
The above example is for calculating the removal efficiency using the inlet and outlet concentrations or mass loading.\\
The methods below can be used for calculating either the inlet or outlet pollutant concentrations, if the removal efficiency and the corresponding inlet or outlet concentrations are given. 


\hl{Case 1:  Calculating outlet conc. (X) given the inlet conc. and removal efficiency (RE\%):}

\tikzstyle{block} = [rectangle, draw, fill=red!40, 
    text width=6em, text centered, rounded corners, minimum height=3em]
\tikzstyle{arrow} = [draw, -latex']
\begin{figure}[!h]
\centering
\begin{tikzpicture}[node distance =1.5cm, auto]
    \draw ++(0,0) node [block] (Process) {Process};
   \node[node distance=1.9in] (dummy_in) [left of=Process] {In};
   \node[node distance=1.9in] (dummy_out) [right of=Process] {Out};
	\node (Removal) [below of=Process, yshift=-0in] {$\tiny{Removal \enspace Efficiency=RE\% \enspace (Given)}$};
    \path [arrow] (dummy_in)-- (Process)  node [above] {\hspace{-5.8cm}$A \enspace mg/l \enspace (Given) $} node [below] {\hspace{-5.8cm}$100 \enspace mg/l$};
    \path [arrow] (Process) -- (dummy_out)  node [above] {\hspace{-4cm}$X \enspace mg/l \enspace (Unknown)$} node [below] {\hspace{-3.9cm}($100-RE\%)\enspace mg/l$};
   \draw[arrow] (Process) -- (Removal);
\end{tikzpicture}
\end{figure}
Using the fact that if the inlet concentration was 100 mg/l, the outlet concentration would be 100 minus the removal efficiency.\\
Setup the equation as:  $\dfrac{Out}{In}: \enspace \dfrac{X \enspace mg/l}{A \enspace mg/l}=\dfrac{100-RE\%}{100}$\\
Calculate X using cross multiplication - if $\dfrac{A}{B}=\dfrac{C}{D} \implies A=B*\dfrac{C}{D}$:\\
$X \enspace mg/l=A \enspace mg/l*\dfrac{100-RE\%}{100}$\\

\hl{Case 2:  Calculating inlet conc. (X) given the outlet conc. and removal efficiency (RE\%):}

\begin{figure}[!h]
\centering
\begin{tikzpicture}[node distance =1.5cm, auto]
    \draw ++(0,0) node [block] (Process) {Process};
   \node[node distance=1.9in] (dummy_in) [left of=Process] {In};
   \node[node distance=1.9in] (dummy_out) [right of=Process] {Out};
	\node (Removal) [below of=Process, yshift=-0in] {$Removal \enspace Efficiency=RE\% \enspace (Given)$};
    \path [arrow] (dummy_in)-- (Process)  node [above] {\hspace{-5.8cm}$X \enspace mg/l \enspace (Unknown)$} node [below] {\hspace{-5.8cm}$100 \enspace mg/l$};
    \path [arrow] (Process) -- (dummy_out)  node [above] {\hspace{-4cm}$A \enspace mg/l \enspace (Given)$} node [below] {\hspace{-3.9cm}($100-RE\%)\enspace mg/l$};
   \draw[arrow] (Process) -- (Removal);
\end{tikzpicture}
\end{figure}
Using the fact that if the inlet concentration was 100 mg/l, the outlet concentration would be 100 minus the removal efficiency.\\
Setup the equation as:  $\dfrac{In}{Out}: \enspace \dfrac{X \enspace mg/l}{A \enspace mg/l}=\dfrac{100}{100-RE\%}$\\
\vspace{0.3cm}
Calculate X using cross multiplication - if $\dfrac{A}{B}=\dfrac{C}{D} \implies A=B*\dfrac{C}{D}$:\\
$X \enspace mg/l=A \enspace mg/l*\dfrac{100}{100-RE\%}$\\

\vspace{0.4cm}
\hl{Example Problems:}\\

\begin{enumerate}

\item What is the \% removal efficiency if the influent concentration is 10 mg/L and the effluent concentration is 2.5 mg/L?\\
$Removal \enspace Rate (\%) = \dfrac{In-Out}{In}*100 \implies \dfrac{10-2.5}{10}*100=\boxed{75\%}$



\item Calculate the outlet concentration if the inlet concentration is 80 mg/l and the process removal efficiency is 60\%\\
Solution:\\

\tikzstyle{block} = [rectangle, draw, fill=red!40, 
    text width=6em, text centered, rounded corners, minimum height=3em]
\tikzstyle{arrow} = [draw, -latex']
\begin{figure}[!h]
\centering
\begin{tikzpicture}[node distance =1.5cm, auto]
    \draw ++(0,0) node [block] (Process) {Process};
   \node[node distance=1.5in] (dummy_in) [left of=Process] {In};
   \node[node distance=1.5in] (dummy_out) [right of=Process] {Out};
	\node (Removal) [below of=Process, yshift=-0in] {$Removal \enspace Efficiency=60\%$};
    \path [arrow] (dummy_in)-- (Process)  node [above] {\hspace{-4.39cm}$80mg/l$} node [below] {\hspace{-4.39cm}$100mg/l$};
    \path [arrow] (Process) -- (dummy_out)  node [above] {\hspace{-3.cm}$Xmg/l$} node [below] {\hspace{-3cm}40mg/l};
   \draw[arrow] (Process) -- (Removal);
\end{tikzpicture}
%\caption[MFCC]{Diagrama en bloques del cálculo de las MFCC para un frame.}
%\label{MFCC}
\end{figure}

$\dfrac{Out}{In} \enspace:\enspace\dfrac{Actual \enspace Outlet (X)}{80}=\dfrac{100-60}{100}$\\
$\implies \dfrac{Actual \enspace Outlet (X)}{80} =0.4$\\
$\implies Actual \enspace  Outlet (X) = 0.4 * 80 = \boxed{32 mg/l}$\\


\item Calculate the inlet concentration if the outlet concentration is 80 mg/l and the process removal efficiency is 60\%\\

\tikzstyle{block} = [rectangle, draw, fill=red!40, 
    text width=6em, text centered, rounded corners, minimum height=3em]
\tikzstyle{arrow} = [draw, -latex']
\begin{figure}[!h]
\centering
\begin{tikzpicture}[node distance =1.5cm, auto]
    \draw ++(0,0) node [block] (Process) {Process};
   \node[node distance=1.5in] (dummy_in) [left of=Process] {In};
   \node[node distance=1.5in] (dummy_out) [right of=Process] {Out};
	\node (Removal) [below of=Process, yshift=-0in] {$Removal \enspace Efficiency=60\%$};
    \path [arrow] (dummy_in)-- (Process)  node [above] {\hspace{-4.39cm}$Xmg/l$} node [below] {\hspace{-4.39cm}$100mg/l$};
    \path [arrow] (Process) -- (dummy_out)  node [above] {\hspace{-3.cm}80mg/l} node [below] {\hspace{-3cm}40mg/l};
   \draw[arrow] (Process) -- (Removal);
\end{tikzpicture}
\end{figure}

$\dfrac{In}{Out} \enspace : \enspace \dfrac{Actual \enspace inlet \enspace  (X)}{80}=\dfrac{100}{100-60}\implies \dfrac{Actual \enspace inlet \enspace  (X)}{80}=2.5$\\    
Rearranging the equation:   $Actual \enspace inlet (X)=2.5*80 = \boxed{200 mg/l}$\\



\end{enumerate}