\chapterimage{Water1.png} % Chapter heading image

\chapter{Wastewater Constituents}

Domestic wastewater is 99.9\% water and the remaining 0.1\% are the pollutants that are the target for removal during the treatment process. The treatment process is designed to reduce the level of the pollutants so that the treated water does not adversely impact the environment and particularly the ecosystem of the receiving waters.

Wastewater is characterized in terms of the concentrations of the pollutants which are present in the wastewater as it is treated in the plant and in the solids produced in order to:

\begin{enumerate}
\item Monitor the quality of water, as the treatment progresses and especially that of the treated effluent to ensure treated water quality standards are met, and 
\item Allow for the the evaluation and control of the treatment processes.
\item Ensuring compliance with NPDES permits and biosolids  
\end{enumerate}

\section{Wastewater Pollutants}



Wastewater pollutants include:

\subsection{Solids}\index{Solids}
		\begin{itemize}
		\item Generally speaking wastewater solids includes feces, food particles, toilet paper, grease, oil, soap, salts, metals, detergents, sand and grit.
			\item The \texthl{solids can be classified as suspended or dissolved} based upon its ability to pass through a standardized filter paper.
			\item When the wastewater is filtered:
			      \begin{itemize}
			      	\item the residual solids remaining on the filter paper after drying in an oven at 103\si{\degree}C is the \hl{suspended solids} portion, and 
			      	\item the solids remaining after drying the filtrate are the \hl{dissolved solids}.
			      \end{itemize}
			\item Suspended solids can be categorized based upon its settling characteristics as:
			      \begin{itemize}
			      	\item \hl{Settleable}
			      	\item \hl{Non-settleable}
			      	      \begin{itemize}
			      	      	\item \hl{Colloidial}-small, charged (typically negative) particles which do not settle easily.  Some of the colloidial particles are small enough to pass through the filter paper used for filtering the suspended solids
			      	      	\item \hl{Floatable}-example oil and grease and small plastics
			      	      \end{itemize}
			      \end{itemize}
			\item Both suspended and dissolved solids can be either \hl{volatile (organic)} or \hl{fixed (inorganic)}.

			      \begin{itemize}
			      	\item The volatile solids are typically of plant or animal origin .
			      	\item The fixed solids include sand, gravel and silt as well as the dissolved salts.
			      \end{itemize}
			     \item \hl{Total Solids is thus a sum of TSS and dissolved solids or volatile and fixed solids or it can be expressed as a sum of the organic and inorganic solids.}
			      \end{itemize}

		
\subsection{Organics}\index{Organics}		
		\begin{itemize}
			\item Organics are substances containing carbon, hydrogen and oxygen, and some of which may be combined with nitrogen, sulfur or phosphorous.
			\item About 50 percent of the solids present in wastewater are organic.  This fraction is generally of animal or vegetable life, dead animal matter, plant tissue or organisms, and also include synthetic organic compounds.
			\item The principal organic compounds present in domestic wastewater are proteins, carbohydrates and fats together with the products of their decomposition.
			\item Organics are subject to decay or decomposition through the activity of bacteria and other living organisms.  \hl{Since the organic fraction can be driven off at high temperatures, they are also called \textbf{volatile solids}}.
			\item \emph{Organics in wastewater is typically quantified in terms of oxygen required to oxidize the carbon based material present} in wastewater using the \hl{Biochemical Oxygen Demand (BOD)} or \hl{Chemical Oxygen Demand (COD)} tests.
		\end{itemize}


	\subsection{Nitrogen}\index{Nitrogen}				
% 			\begin{enumerate}%@@@@@@@@@@@@@@@@@@%
% 				\definecolor{shadecolor}{RGB}{220,220,220}
% 				\begin{snugshade*}
% 					\item \noindent\textsc{Nitrogen}%@@@@@@@@@@@@@@@@@@%
% 				\end{snugshade*}

	\subsubsection{Nutrients}\index{Nutrients}	
Nitrogen and phosphorous content in wastewater is monitored as these plant nutrients, when present in wastewater effluent discharge promote growth of plant and algal matter in the receiving waters causing destruction of the normal aquatic life mainly due to oxygen depletion - eutrophication.  These nutrient from human waste (proteins are nitrogen based), food and certain soaps and detergents. 					 

\subsection{Oil and Grease}\index{Oil and Grease}	
			Fats, oil and grease in wastewater originate from homes, food establishments and industries.
			\begin{itemize}
				\item Presence of excessive oils and grease could potentially impact the secondary treatment process
				\item Oils and grease are removed as floatables in primary treatment and sent with the sludge to the digesters
			\end{itemize}


\subsection{Heavy Metals}\index{Heavy Metals}	
Heavy metals such as - Cd, Pb, Mn, Cu, Zn, Fe and Ni occur naturally in water and some of these are critical for biological growth.  Heavy metals in wastewater either remain as part of the treated wastewater or are removed as part of the wastewater solids (biosolids).  Excessive concentrations of these metals are toxic to the ecosystems of receiving waters and disposal or to the reuse sites of the wastewater solids.  The concentration threshold in the wastewater effluent are established in the facility's NPDES permit and concentrations in the metal content of the wastewater solids are controlled through the facility's compliance with Federal Biosolids Regulations (40 CFR Part 503) requirements.  Compliance with heavy metal discharge requirements are accomplished through the wastewater entities' Pretreatment or Industrial Discharge Control programs.

\subsection{Emerging Pollutants}\index{Pollutants}				
Emerging pollutants are a group of synthetic or naturally-occurring chemical or any microorganism which are not commonly monitored or regulated and are potentially known or suspected to cause adverse ecological and human health effects. These pollutants include a variety of compounds such as antibiotics, drugs, steroids, endocrine disruptors, hormones, industrial additives, chemicals, and also microbeads and microplastics. There is an inextricable link between these pollutants and wastewater.

\section{Wastewater Parameters}\index{Wastewater Parameters}			
		Laboratory and field tests are conducted to measure parameters which are critical for monitoring and controlling treatment.  The following are the key parameters that are measured.	
			
\subsection{pH}\index{pH}	
			\hl{pH is a measure of the hydrogen ion (H$^+$) content or the acidity or basicity of a solution.}  pH impacts the chemical and micribiological elements of wastewater treatment processes and thus pH measurement and control is critical.
			\begin{itemize}
				\item Pure water dissociates into equal concentration of hydrogen ions and hydroxide ions:\\ 
				      $H_2O \rightarrow H^+ + OH^-$.
				\item The H$^+$ are responsible for acidic properties and the OH$^-$ ions for the basic properties.  
				\item pH is the inverse of H$^+$ concentration; pH increases when the concentration of H$^+$ decreases relative to the concentration of OH-. 
				\item pH scale ranges from 0 – 14. When the concentration of both H$^+$ and OH$^-$ are equal, as in pure water, it is considered neutral and its pH is 7.0.  \item If the pH of a sample solution is below 7.0, the sample is termed acidic and is alkaline or basic if its pH is above 7.0. 
				\item Each change of 1 pH unit represents a 10 fold change in concentration.  For example, a sample with a pH of 2.0 is 1000 times more acidic than a sample with a pH of 5.0. 
				\item pH is measured by an electrode that is sensitive only to H$^+$ or using a pH strip which is essentially an adsorbent paper which is pre-impregnated with chemicals which change color under different H$^+$ concentrations.
				\item Most organisms involved in biological wastewater treatment processes do well within a a narrow range of pH near neutral (pH of 7).			
			\end{itemize}
			
			
\subsection{Alkalinity}\index{Alkalinity}	
			\begin{itemize}
				\item \hl{Alkalinity is the ability of a water to neutralize acids.}  
				\item During certain wastewater treatment processes including anaerobic digestion, acids are generated as a result of microbiological activity.  The bacteria and other biological entities which play an active role in wastewater treatment are most effective at a neutral to slightly alkaline pH of 7 to 8.  In order to maintain these optimal pH conditions for biological activity there must be sufficient alkalinity present in the wastewater to neutralize acids generated by the active biomass.
				\item This ability to maintain the proper pH in the wastewater as it undergoes treatment is the reason why alkalinity is so important to the wastewater industry.
				\item The alkalinity is due to the presence of acid neutralizing bases in the water including the hydroxyl (OH$^-$), carbonate (CO$_3$$^-$) and bicarbonate (HCO$_3$$^-$)  ions.  These ions are of mineral origin and are also formed from carbon dioxide which comes from the atmosphere and from the microbial decomposition of organic material.  The resistance to pH change of the water will continue until all the alkalinity contributing ions are neutralized.  
				\item The pH of a water serves as a guide to the types of alkalinity present in the water but is unrelated to the alkalinity content of a water.  Important Note:  Alkalinity is a measure of the ability to neutralize acids whereas a solution is termed alkaline (or basic) if its pH greater than 7. 
				\item Alkalinity is expressed as milligrams per liter of CaCO$_3$
			\end{itemize}
			
	
\subsection{Microbiological Characterization}\index{Microbiological Characterization}	
			
Microbiological testing and monitoring is conducted as part of the wastewater treatment for two main reasons:
	\begin{enumerate}[1.]
		\item Heterotrohic (organisms that consume organic material) microbes are responsible for the biological wastewater treatment processes - secondary treatment process, digestion and nutrient removal; 
			\begin{itemize}
				\item The effectiveness of the biological wastewater treatment processes is primarily due to the presence of a microbial ecosystem with a right balance of populations of different microbial species.
				\item Methods used for monitoring the microbial composition include direct monitoring using a light microscope to see which and how many of the different microbial species are present - typically used for activated sludge process.
				\item The microbial monitoring ensures process stability and helps identify potential process upset conditions caused by changes to the microbial population due to other external factors - toxicty, organic loading, temperature etc.
			\end{itemize}
		\item Pathogens - agents which include bacteria, viruses, protozoas and helmniths that cause disease are present in wastewater effluent.
			\begin{itemize}
				\item As one of the main reasons for treating wastewater is to protect public health, microbiological/pathogen testing of the wastewater effluent and the surface water impacted by the wastewater discharge is conducted to meet the requirements of a wastewater discharge permit, to monitor the pathogen impact of treated wastewater discharge and assess the level of contamination of a public body of water.
				\item The bacteriological tests involves detection and quantification of one or more of the following bacteria:  total coliforms, fecal coliforms, E. Coli, and Enterococcus.  
					 \begin{itemize}
					      \item The main reason why these bacteria such as coliforms and enterococcus are used \hl{as it is not practical to detect and quantify all pathogens associated with wastewater.}  
					      	\item These selected bacteria originate from feces and indicate fecal contamination and thus serve as an indicator organisms for pathogens of wastewater origin.  
					      	\item Also, they are abundant, potentially less harmful, and easy to detect.  E. coli has been shown to be a better predictor of the potential for impacts to human health and therefore many newer wastewater discharge permits require E. Coli testing in lieu of fecal coliform testing requirements.
					   \end{itemize}
			\end{itemize}
	\end{enumerate}
