
\chapterimage{TitleIII.png} % Chapter heading image

\chapter{Constituents and Properties}

			\section{Wastewater Constituents}\index{Wastewater Constituents}			
		\subsection{Organics}\index{Organics}		
		\begin{itemize}
			\item The main reason for treating domestic wastewater is to remove the organic matter.  
			\item Organics are substances containing carbon, hydrogen and oxygen, and some of which may be combined with nitrogen, sulfur or phosphorous.
			\item About 50 percent of the solids present in wastewater are organic.  This fraction is generally of animal or vegetable life, dead animal matter, plant tissue or organisms, and also include synthetic organic compounds.
			\item The principal organic compounds present in domestic wastewater are proteins, carbohydrates and fats together with the products of their decomposition.
			\item Organics are subject to decay or decomposition through the activity of bacteria and other living organisms.  \hl{Since the organic fraction can be driven off at high temperatures, they are also called \textbf{volatile solids}}.\
			\item \emph{Organics in wastewater is typically quantified in terms of oxygen required to oxidize the carbon based material present} in wastewater using the following methods:\\
\subsubsection{Biochemical Oxygen Demand (BOD)}\index{Biochemical Oxygen Demand (BOD)}

			  %     \begin{enumerate}[i.]
			  %     	\definecolor{shadecolor}{RGB}{220,220,220}
					% %%%%%%%%%%%
					% % LEVEL 4 %
					% %%%%%%%%%%%
			  %     	\begin{snugshade*}
			  %     		\item \noindent\textsc{Biochemical Oxygen Demand (BOD)}%@@@@@@@@@@@@@@@@@@%
			  %     	\end{snugshade*}					
			      	\begin{itemize}
			      		\item Oxygen is required for the consumption of organic matter by aerobic bacteria
			      		\item BOD test measures the depletion of oxygen in a wastewater sample over a five day period
			      		\item BOD measures the organic content in terms of oxygen required for the microorganisms to consume the organic material present

			      		\item BOD is typically measured as BOD$_5$ which is the oxygen demand of the wastewater measured after 5 days of the initiation of the test.
			      		\item The test involves incubating a known dilution of wastewater in a 300 ml bottle for 5 days at 20\si{\degree}C.  The dissolved oxygen (DO) content at the start and end of the incubation period is used for calculating the BOD.
			      		\item For the test to be considered valid, the following criteria need to be met: 1) DO consumption during the test must be at least 2 mg/l, 2) DO remaining at the end of the test must be at least 1 mg/l, and 3) DO consumed in blank should be 0.2 mg/l or less
			      		      			
			      		\item BOD is a parameter to measure the strength of wastewater and the measurement of the wastewater treatment plant or treatment process influent and effluent BOD is standard practice to measure its performance.  Typical domestic wastewater BOD is about 200-250 mg/l.
			      		\item The oxygen consumed by the microorganisms during the BOD test is primarily for: 1) Oxidizing the carbonaceous material (cBOD – carbonaceous BOD), and 2) Oxidizing nitrogenous constituents such as ammonia (nBOD – nitrogenous BOD).
			      		\item Thus, BOD (Total) = cBOD + nBOD.  The cBOD and nBOD is measured by adding certain chemical inhibitors which will inhibit the bacteria responsible for consuming the nitrogenous matter, thus measuring only the cBOD as part of the BOD test.
			      		\item Since not all of the organics is metabolized in the 5 days of the regular BOD test, certain wastewater discharge permits require reporting of the ultimate BOD value (BOD$_U$)\\
			      	\end{itemize}

			    \subsubsection{Chemical Oxygen Demand (COD)}\index{Chemical Oxygen Demand (COD)}
			      	% \begin{snugshade*}
			      	% 	\item \noindent\textsc{Chemical Oxygen Demand (COD)}%@@@@@@@@@@@@@@@@@@%
			      	% \end{snugshade*}		  
			      	\begin{itemize}
			      		\item The COD test involves using chemical oxidizers to measure the oxygen demand of the wastewater.
			      		\item As the chemical oxidizers will oxidize other constituents present, including inorganic matter, the COD value of wastewater will be higher than the BOD.  
			      		\item The COD test can be conducted rather quickly than the 5 day BOD test, it is an effective method to quantify the wastewater strength and process efficiencies and allow operators to make timely process adjustments.
			      	\end{itemize}

			    \subsubsection{Total Organic Carbon (TOC)}
			      	% \begin{snugshade*}
			      	% 	\item \noindent\textsc{Total Organic Carbon (TOC):}\\%@@@@@@@@@@@@@@@@@@%
			      	% \end{snugshade*}
			      	The TOC method utilizes laboratory analytical instruments which directly measure the organic carbon content by quantifying the amount of carbon dioxide produced from the complete combustion of the organics present.
			      % \end{enumerate}
		\end{itemize}
		
		
		
			\hl{Note: BOD measures the amount of oxygen required by the microorganisms present to consume the organic material while COD measures the chemical oxidation required to oxidize all chemicals including organics present in wastewater.  BOD value of typical domestic sewage is about 200 - 250 mg/l while the COD value ranges from 300 - 450 mg/l.  Typical BOD:COD ratio ranges from 0.5-0.8.}\\


\subsection{Solids}\index{Solids}
% 		\pagebreak
% 				\begin{snugshade*}
% 			\item \noindent\textsc{Solids}
% 		\end{snugshade*}	
		Like BOD, wastewater solids is another critical parameter for establishing the wastewater strength and determining treatment process efficiencies. 
		\begin{itemize}
			\item The \texthl{solids can be classified as suspended or dissolved} based upon its ability to pass through a standardized filter paper.
			\item When the wastewater is filtered:
			      \begin{itemize}
			      	\item the residual solids remaining on the filter paper after drying in an oven at 103\si{\degree}C is the \hl{suspended solids} portion, and 
			      	\item the solids remaining after drying the filtrate are the \hl{dissolved solids}.
			      \end{itemize}
			\item Suspended solids include larger floating particles and consist of sand, grit, clay, fecal matter, paper, pieces of wood, particles of food and garbage, and similar materials.
			\item Suspended solids can be categorized based upon its settling characteristics as:
			      \begin{itemize}
			      	\item \hl{Settleable}
			      	\item \hl{Non-settleable}
			      	      \begin{itemize}
			      	      	\item \hl{Colloidial}-small, charged (typically negative) particles which do not settle easily.  Some of the colloidial particles are small enough to pass through the filter paper used for filtering the suspended solids
			      	      	\item \hl{Floatable}-example oil and grease and small plastics
			      	      \end{itemize}
			      \end{itemize}
			\item Dissolved solids in wastewater include organics.  However, the major elements of dissolved solids are inorganic ions such as Ca$^{+2}$, Mg$^{+2}$, Cl$^-$, SO$_4$ $^{-2}$ , HCO$_3$ $^-$, Fe$^{+2}$, PO$_4$ $^{-3}$, NO$_3$ $^-$.  These ions are part of the dissolved salts such as sodium chloride (NaCl), calcium bicarbonate (Ca(HCO$_3$)$_2$), magnesium phosphate (Mg$_3$PO$_4$) and others which are normally present in water and wastewater. 
			      \begin{itemize}
			      	\item Conductivity or electrical conductance (EC) measurement is typically conducted as the wastewater enters the plant as \hl{conductivity provides an indirect and simple measure of the amount of dissolved solids present.}  
			      	\item Conductivity or electrical conductance (EC) is a measure the amount of electrical current that can be conducted by a solution.  
			      	\item The conductance of electricity in a solution is due to the presence of dissolved inorganic ions 
			      	\item The higher the concentration of these ions, the higher is the conductivity. 
			      	\item \underline{Conductivity is measured in the units of mhos/cm or Siemens/cm.}  (Note:  mhos is the reverse of ohm which is a measure of resistance).
			      	\item Typical wastewater conductivities range from 50 to 1500 S/cm
			      \end{itemize}
			\item Both suspended and dissolved solids can be either \hl{volatile (organic)} or \hl{fixed (inorganic)}.
			\item \hl{Total Solids is thus a sum of TSS and dissolved solids or volatile and fixed solids.}
			      \begin{itemize}
			      	\item The volatile solids are typically of plant or animal origin .
			      	\item The fixed solids include sand, gravel and silt as well as the dissolved salts.
			      \end{itemize}
			      \begin{minipage}{0.5\textwidth}
			      	\item The volatile or fixed fractions are quantified by incinerating the solids in a muffler furnace at 550\si{\degree} which removes only the volatile solids leaving only the fixed solids behind.
			      	\item In terms of the size of the solids, the distribution is approximately thirty percent suspended and about seventy percent dissolved solids - which includes the colloidal particles which have passed through the filter paper.\\ 
			      	\item As primary treatment process involve settling of solids, establishing the settleable portion of the suspended solids is important.\\  
			      	\item \hl{The settleable solids are quantified using an Imhoff cone and are reported in ml/L}.  Imhoff cone is a 1 liter, clear cone shaped container, with volume graduations (ml) at the bottom.
			      						
			      \end{minipage}	
			      \begin{minipage}{0.5\textwidth}
			      	\begin{center}
			      		\includegraphics[scale=0.7]{ImhoffCone}\\
			      		Imhoff Cone\\
			      		\textit{Note the ml markings at the bottom of the cone}
			      		
			      		
			      	\end{center}
		      \end{minipage}
%			      \end{minipage}
			      	\item One factor which affects settleability is the conveyance time of the sewage to the treatment plant. 			
			      	\item The settleable component of the suspended solids will decrease as the sewage becomes more septic due to longer conveyance times.
			\item Influent and effluent total suspended solids are measured to establish the overall treatment and individual process efficiencies.  
			\item Volatile solids measurements before and after biological processes such as secondary treatment and digestion provide information on the process efficiency.\\
		\end{itemize}


	\subsubsection{Procedures for Solids Analysis}\index{Procedures for Solids Analysis}
% 		\begin{enumerate}%$$$$$$$$$$$$$$$$$$$$%
% 			\definecolor{shadecolor}{RGB}{220, 220, 220}
% 			\begin{snugshade*}
% 				\item \noindent\textsc{Procedures for Solids Analysis}
% 			\end{snugshade*}		
	\textbf{Determining wastewater suspended solids - Total(TSS) and Volatile(VSS) concentrations:}\\	
			% \begin{enumerate}[a.]
			% 	\definecolor{shadecolor}{RGB}{110, 192, 221}
			% 	\begin{snugshade*}
			% 		\item \noindent\textsc{Determining wastewater suspended solids - Total(TSS) and Volatile(VSS) concentrations:}
			% 	\end{snugshade*}
				 
				%\hl{How total suspended solids concentration is established for a wastewater sample:}
				\begin{itemize}
					\item A known volume of wastewater sample is filtered through a pre-weighed filter paper
					\item The suspended solids will be retained by the filter
					\item The water with the dissolved solids will pass through the filter
					\item The filter paper with the filter solids is rinsed with distilled water to remove 
					\item The filter paper with the solids is dried in the oven and then weighed
					\item The difference between the weight of the dried filter paper with the solids and the pre-weighed filter paper, measured in mg, will be the suspended solids in: mg per the original quantity of wastewater sample taken.  This value can be converted to give the suspended solids content in mg/l
					\item A filter paper with the dried solids is incinerated in a muffler furnace
					\item The difference in the weight of the solids, before and after incineration is the fixed solids
					\item The difference between the weight of the solids before incineration and the fixed solids is the volatile solids
				\end{itemize}
				
	\textbf{Determining wastewater and sludge total solids (TS) and volatile solids (VS) concentrations:}\\	
				
				% \begin{snugshade*}
				% 	\item \noindent\textsc{Determining wastewater and sludge total solids (TS) and volatile solids (VS) concentrations:}
				% \end{snugshade*}
				\begin{itemize}
					\item A certain quantity of wastewater (by volume) or sludge (by weight) is taken in a pre-weighed dish and weighed.  Note:  the sample is not filtered.
					\item The dish with the sample is dried in an oven
					\item The difference in the weight of the pre-weighed dish from that of the dish with the dried sample is the total solids
					\item The dried solids are incinerated in a muffler furnace
					\item The difference in the weight of the solids, before and after incineration is the fixed solids
					\item The difference between the fixed solids and the total solids is the volatile solids
					\item Total solids of a sludge sample is reported as a \% of the sludge weight.  A 7\% sludge has 7 lbs of solids for every 100 lbs of sludge.
				\end{itemize}
				
							
				
				
					\hl{For sludge samples, volatile solids is typically reported as the volatile solids fraction in \% of the total solids content of the sludge.  For example, if a 8\% sludge (i.e sludge which has 8\% TS or 80,000mg/l solids), is reported to have 70\% volatile, it means that 70\% of the total solids - 0.7*8\%=5.6\% or 56,000mg/l is the sludge volatile solids content.  \emph{70\% volatile does not meet the sludge has 700,000mg/l volatile solids}}\\
				
% 			\end{enumerate}
	\subsubsection{Summary of Wastewater Solids}\index{Summary of Wastewater Solids}		
% 			\begin{snugshade*}
% 				\item \noindent\textsc{Summary of Wastewater Solids}
% 			\end{snugshade*}
			\begin{itemize}
				\item Solids in wastewater can be categorized as dissolved or suspended
				      \begin{itemize}
				      	\item Suspended solids can be further categorized as settleable or unsettleable
				      \end{itemize}
				\item Solids can also be categorized as organic (aka: volatile) or inorganic (aka: fixed)
				\item Colloidial particles are small sized particles some of which pass through the filter and accounted as part of dissolved solids
				\item TSS - Total Suspended Solids are the solids that are captured on the filter paper upon filtration of the wastewater sample.  
				\item Wastewater samples typically analyzed for TSS include:  plant, primary and secondary processes - influent and effluent.  TSS is reported in mg/l
				\item TS - Total Solids are solids content of sludge.  TS of sludge is established by drying a preweighed quantity of sludge in an oven and is typically reported as \% solids - which is how many parts (by weight) of solids per 100 parts (by weight) of sludge.
				\item Volatile solids are solids that are removed when the solids are incinerated at 550C.  The solids that remain after incineration are fixed or non-volatile or inorganic solids.
			\end{itemize}
	\subsubsection{Wastewater Solids Content}\index{Wastewater Solids Content}			
% 			\begin{snugshade*}
% 				\item \noindent\textsc{Typical influent wastewater contains:}
% 			\end{snugshade*}
			\begin{itemize}
				\item Less than 0.1\% total solids.  Total solids concentration in typical wastewater is about 750mg/l
				\item The total solids are 50\% organic (volatile) and 50\% inorganic (fixed)
				\item Of the total solids, dissolved solids constitute about 70\% of the solids and the remaining 30\% solids are suspended solids
				\item 40\% of the dissolved solids are volatile the remaining 60\% are fixed
				\item 70\% of the suspended solids are volatile and the remaining 30\% are fixed
			\end{itemize}
			% \clearpage\thispagestyle{empty}
			\begin{figure}[!htbp]
			\vspace{2cm}
				\begin{center}
					\includegraphics[scale=0.8]{WastewaterSolids}\\
					\caption{Typical Wastewater Solids Concentrations}
				\end{center}
				\end{figure}
% % 			\end{enumerate}
				
\subsection{Nutrients}\index{Nutrients}	
% 			\begin{snugshade*}
% 				\item \noindent\textsc{Nutrients}
% 			\end{snugshade*}	
			\begin{itemize}
				\item Plant nutrients - nitrogen and phosphorous, present in wastewater effluent discharge, promote growth of plant and algal matter in the receiving waters causing destruction of the normal aquatic life mainly due to oxygen depletion - eutrophication.
				      
				\item Because of the potential impacts of the presence of these nutrients in wastewater effluent on the receiving waters,  limits on the levels of these nutrients is typically stipulated in the treatment plant's wastewater discharge permit.
				      
				\item Typically, conventional secondary treatment processes are designed primarily remove the organics from the wastewater.  Secondary treatment process designed to additionally remove nutrients is deemed as tertiary or advanced treatment is termed as Biological Nutrient Removal (BNR).
			\end{itemize}
	\subsubsection{Nitrogen}\index{Nitrogen}				
% 			\begin{enumerate}%@@@@@@@@@@@@@@@@@@%
% 				\definecolor{shadecolor}{RGB}{220,220,220}
% 				\begin{snugshade*}
% 					\item \noindent\textsc{Nitrogen}%@@@@@@@@@@@@@@@@@@%
% 				\end{snugshade*}

	\textbf{Forms of nitrogen:}\\	
% 				\begin{itemize}
% 					\item Forms of nitrogen:\\
					      \begin{itemize}
					      	\item About 60\% of nitrogen in wastewater is present as ammonia nitrogen (about 60\%).  The ammonium nitrogen is present either in the form of ammonia (NH$_3$ ) or as ammonium (NH$_4^+$ ) ion.   These two forms can rapidly change from one to the other depending on pH and temperature.  Under low pH (acidic) or neutral conditions – pH less than or equal to 7, ammonia exists mostly as ammonium.  Ammonia becomes the dominant form as the pH increases to 8 and beyond.
					      	\item The other dominant form of nitrogen, about 40\% of the total nitrogen is as organic nitrogen
					      	\item Nitrogen measured as Total Kjeldahl Nitrogen (TKN) which is the sum of the organic nitrogen and the ammonia nitrogen concentrations.  Total inorganic nitrogen is the total concentration of ammonia nitrogen, NO3-, and NO2-.   Table provides the concentrations and forms of nitrogen in wastewater.
					      \end{itemize}
					      \setlength{\arrayrulewidth}{0.7mm}
					      \setlength{\tabcolsep}{8 pt}
					      \renewcommand{\arraystretch}{0.8}
					      \begin{center}
					      \begin{figure}[!htbp]
					      	\noindent \begin{tabular}[!htbp]{ |p{6cm}|p{2.0cm}|p{2.5cm}|p{2.cm}|}
					      	\hline
					      	\multicolumn{4}{|c|}{\textbf{Forms of Nitrogen in Wastewater}} \\
					      	\hline
					      	%\thead{A Head} & \thead{A Second \\ Head} & \thead{A Third \\ Head} \\
					      	%\hline%
					      	
					      	\hspace{1.8 cm}Forms of Nitrogen & \hspace{0.25 cm} Formula & \hspace{.4 cm} Found in & \hspace{.4 cm} Typical \newline \hspace{.2 cm}Concentration\\
					      	\hline
					      	\small Ammonia/Ammonium & \small NH$_3$/NH$_4^{\enspace +}$ &  \small Influent wastewater & 30-50 mg/l\\
					      	
					      	Total Kjeldahl Nitrogen \newline  \small (Ammonia/Ammonium + Organic Nitrogen) &  \small TKN &  \small Wastewater \newline  \small effluent  & 30-60 mg/l \\
					      	
					      	\small Total Inorganic Nitrogen \newline  \small (Ammonia/Ammonium + Nitrite + Nitrate) & \small TIN &  \small  Wastewater \newline  \small effluent  & 1-40 mg/l \\
					      	
					      	\small Nitrate  & $NO_3^{\enspace -}$ &  \small Nitrified effluent &  \small 1-35 mg/l \\
					      	
					      	\small Nitrate  &  $NO_2^{\enspace -}$ &  \small Partially nitrified effluent &  \small 0.1-2 mg/l \\
					      	
					      	\hline
					      	\end{tabular}
					      	\caption{Forms of Nitrogen}
					      	\end{figure}
					      \end{center}
					      
		\subsubsection{Phosphorous}\index{Phosphorous}			
		\textbf{Forms of phosphorous:}\\
					      \begin{itemize}
					      	\item The principal forms are organically bound phosphorus, polyphosphates, and orthophosphates.
					      	\item Organically bound phosphorus originates from body and food waste and, upon biological decomposition of these solids, is converted to orthophosphates. 
					      	\item Polyphosphates originate from synthetic detergents and are hydrolyzed to orthophosphates. Thus, the principal form of phosphorus in wastewater is assumed to be orthophosphates, although the other forms may exist. Orthophosphates consist of the negative ions PO$_4$$^{3-}$, HPO$_4$$^{2-}$, and H$_2$PO$_4$ $^-$.  These may form chemical combinations with cations (positively charged ions).
					      \end{itemize}

\subsection{Oil and Grease}\index{Oil and Grease}	
			Fats, oil and grease in wastewater originate from homes, food establishments and industries.
			\begin{itemize}
				\item Oil and grease content of wastewater is established in the laboratory by extracting it with a solvent - \textit{n}-hexane.  The concentration of oil and grease is reported in mg/l and typical oil and grease content of wastewater ranges from 80 - 120 mg/l
				\item Presence of excessive oils and grease could potentially impact the secondary treatment process
				\item Oils and grease are removed as floatables in primary treatment and sent with the sludge to the digesters
			\end{itemize}
		



\section{Wastewater Properties}\index{Wastewater Parameters}			
		Laboratory and field tests are conducted to measure parameters which are critical for monitoring and controlling treatment.  The following are the key parameters that are measured.	
			
\subsection{pH}\index{pH}	
			\hl{pH is a measure of the hydrogen ion (H$^+$) content or the acidity or basicity of a solution.}  pH impacts the chemical and micribiological elements of wastewater treatment processes and thus pH measurement and control is critical.
			\begin{itemize}
				\item Pure water dissociates into equal concentration of hydrogen ions and hydroxide ions:\\ 
				      $H_2O \rightarrow H^+ + OH^-$.
				\item The H$^+$ are responsible for acidic properties and the OH$^-$ ions for the basic properties.  
				\item pH is the inverse of H$^+$ concentration; pH increases when the concentration of H$^+$ decreases relative to the concentration of OH-. 
				\item pH scale ranges from 0 – 14. When the concentration of both H$^+$ and OH$^-$ are equal, as in pure water, it is considered neutral and its pH is 7.0.  \item If the pH of a sample solution is below 7.0, the sample is termed acidic and is alkaline or basic if its pH is above 7.0. 
				\item Each change of 1 pH unit represents a 10 fold change in concentration.  For example, a sample with a pH of 2.0 is 1000 times more acidic than a sample with a pH of 5.0. 
				\item pH is measured by an electrode that is sensitive only to H$^+$ or using a pH strip which is essentially an adsorbent paper which is pre-impregnated with chemicals which change color under different H$^+$ concentrations.
				\item Most organisms involved in biological wastewater treatment processes do well within a a narrow range of pH near neutral (pH of 7).			
			\end{itemize}
			
\subsection{Oxidation Reduction Potential (ORP)}\index{Oxidation Reduction Potential (ORP)}			
			\begin{itemize}
				\item ORP measurements are common in wastewater process control for monitoring conditions and process efficiency
				\item ORP is measured in milivolts (mV) using a probe\\
				\item ORP is a measure of the potential of oxidation/reduction – electron transfer, based chemical reactions to occur 
				\item If the measured ORP value (in mV), is positive it indicates an environment where oxidation will occur and if negative, an environment where reduction reactions will occur
				\item Higher positive value indicates a stronger oxidative environment and likewise, a lower (more negative) ORP value indicates a stronger reductive environment 
				\item For example, during chlorine disinfection, which is an oxidation process, the wastewater will exhibit a positive ORP.  Stronger the oxidative power of chlorine, higher will be the wastewater ORP value
				\item All living matter, including microbes depend upon respiration to generate energy and respiration involves series of chemical oxidation-reduction reactions 
				\item Bacteria grow and thrive only in specific chemical - oxidative-reductive environments which support its inbuilt metabolic pathways
				\item Aerobic bacteria need molecular oxygen as the terminal electron acceptor as part of its cellular respiration proces.  Bacteria adapted to exist in an environment where molecular oxygen is not present (anoxic and anaerobic), rely on electron acceptors such as NO$_3^-$ (denitrification), SO$_4^-$ (sulfide formation) and carbon (methane formers in anaerobic digestion)
				\item Aerobic bacteria responsible for cBOD removal in the secondary treatment process would be inhibited or wiped-out if the wastewater oxidation potential dropped and become reductive.  Likewise, if the wastewater in the sewer pipes which is normally of reductive (negative ORP)  was to become oxidative because of aeration (dissolving oxygen) it would cease the hydrogen slufide activity of the anaerobic bacteria
				\end{itemize}
		
			\setlength{\arrayrulewidth}{0.6mm}
			\setlength{\tabcolsep}{8 pt}
			\renewcommand{\arraystretch}{1.2}
			\begin{center}
			\begin{table}[!htbp]
				\begin{tabular}{ |p{9.5cm}|p{4.0cm}|}
					\hline
					\multicolumn{2}{|c|}{\textbf{Typical Wastewater Process ORPs}} \\
					\hline
					
					\hline
					\small Clorine disinfection & \small +650 to +700 mV  \\
					\small Nitrification & \small +100 to +350 mV   \\
					\small Biological phosphorous removal & \small +20 to +250mV        \\
					\small Activated sludge	cBOD degradation with free molecular oxygen & \small +50 to +250 mV  \\
					\small Denitrification                                              & \small +50 to -50 mV   \\
					\small Influent wastewater                                          & \small - 200 mV  \\ 
					\small Sulfide formation                                & \small -50 to -250 mV  \\
					\small Anaerobic Digestion: Acid formation (Acidogenesis)           & \small -100 to -225 mV \\
					\small Biological phosphorous removal & \small -100 to -250 mV \\
					\small Anaerobic Digestion: Methane production  (Methanogenesis)     & \small -75 to -400 mV  \\
					\hline
				\end{tabular}
	\end{table}			
			\end{center}
			
\subsection{Alkalinity}\index{Alkalinity}	
			\begin{itemize}
				\item \hl{Alkalinity is the ability of a water to neutralize acids.}  
				\item During certain wastewater treatment processes including anaerobic digestion, acids are generated as a result of microbiological activity.  The bacteria and other biological entities which play an active role in wastewater treatment are most effective at a neutral to slightly alkaline pH of 7 to 8.  In order to maintain these optimal pH conditions for biological activity there must be sufficient alkalinity present in the wastewater to neutralize acids generated by the active biomass.
				\item This ability to maintain the proper pH in the wastewater as it undergoes treatment is the reason why alkalinity is so important to the wastewater industry.
				\item The alkalinity is due to the presence of acid neutralizing bases in the water including the hydroxyl (OH$^-$), carbonate (CO$_3$$^-$) and bicarbonate (HCO$_3$$^-$)  ions.  These ions are of mineral origin and are also formed from carbon dioxide which comes from the atmosphere and from the microbial decomposition of organic material.  The resistance to pH change of the water will continue until all the alkalinity contributing ions are neutralized.  
				\item The pH of a water serves as a guide to the types of alkalinity present in the water but is unrelated to the alkalinity content of a water.  Important Note:  Alkalinity is a measure of the ability to neutralize acids whereas a solution is termed alkaline (or basic) if its pH greater than 7. 
				\item Alkalinity is expressed as milligrams per liter of CaCO$_3$
			\end{itemize}
			
\subsection{Dissolved Oxygen}\index{Dissolved Oxygen}	
			\begin{itemize}
				\item Dissolved oxygen (DO) is the concentration of oxygen dissolved in the wastewater sample and is typically measured in the field using a DO probe.  A titration based Winkler Test is used in the laboratory
				\item The \hl{presence of oxygen indicates an aerobic environment} where dissolved, free oxygen is available for aerobic microorganisms to live, BOD removal in the activated sludge process occurs as a result of the activity of aerobic bacteria.  The absence of DO indicates that the environment or condition is either anoxic or anaerobic.  
				\item \hl{In an anoxic environment, free oxygen is not present, but oxygen is available from its combined  forms - nitrate (NO$_3$ $^-$) and sulfate (SO$_4$ $^-$)} for the the consumption of microorganisms.  Example of an anoxic process is denitrification.  In denitrification, the anoxic bacteria in the presence of food (cBOD) consume the combined oxygen in nitrates (NO$_3$ $^-$ ) and convert it to nitrogen gas.
				\item \hl{The complete absence of oxygen including free and combined oxygen is an anaerobic environment.}
				\item Microorganisms are termed as obligate aerobes if they cannot survive without free oxygen.  Facultative aerobes are microorganisms which can survive in both aerobic and anaerobic environments.  
			\end{itemize}
			
\subsection{Microbiological testing and monitoring}\index{Microbiological testing and monitoring}	
			
			Microbes play a critical role in wastewater treatment.  
			\begin{itemize}
				\item Heterotrohic (organisms that consume organic material) microbes are responsible for the biological wastewater treatment processes - secondary treatment process, digestion and nutrient removal; and
				\item Pathogens - agents that cause disease are present in wastewater effluent.
			\end{itemize}
Microbiological testing and monitoring is conducted as part of the wastewater treatment typically for the following:
\begin{enumerate}[1.]
				
				\item Microbiological testing related to monitoring and troubleshooting biological wastewater treatment\\
				
				Microbes involved in biological wastewater treatment processes include:\\
				\begin{itemize}
					\item Fungi - Filamentous fungi occasionally bloom in activated sludge processes due to low pH or nutrient deficiency and cause problems with the settleability.
					\item Protozoa - Protozoas play a important role in the secondary treatment process.  Common protozoas in the activated sludge process include:
					      \begin{itemize}
					      	\item Amoeba
					      	\item Flagellate
					      	\item Cilliate
					      \end{itemize}
					\item Rotifers
					\item Nematodes
					\item Bacteria - Bacteria is the predominant microorganism responsible for the biological wastewater water treatment.  
				\end{itemize}
				\begin{itemize}
					\item The effectiveness of the biological wastewater treatment processes is primarily due to the presence of a microbial ecosystem with a right balance of populations of different microbial species.
					\item Methods used for monitoring the microbial composition include direct monitoring using a light microscope to see which and how many of the different microbial species are present - typically used for activated sludge process.
					\item Indirect method includes monitoring other parameters such as pH and alkalinity which are influenced by microbiological activity.
					\item The microbial monitoring ensures process stability and helps identify potential process upset conditions caused by changes to the microbial population due to other external factors - toxicty, organic loading, temperature etc.
				\end{itemize}

\item Microbiological testing related to monitoring and controlling pathogens in treated wastewater effluent\\

	
				Pathogens in wastewater belong to the following groups:
				\begin{itemize}
					\item Bacteria:  Although, bacteria is present in large numbers in feces, pathogenic or bacteria are present only because of an infection and this pathogenic bacteria can potentially spread the infection to other healthy individuals.  Disease spread by pathogenic bacteria include diarrhea, cholera and typhoid among many others.
					      
					\item Viruses: A large number of viruses may infect humans and are present in feces.  These include enteroviruses (including polioviruses), hepatitis A virus, reoviruses and diarrhea-causing viruses (especially rotavirus).
					      
					\item Protozoa:  Many species of protozoa can infect humans and cause diarrhoea and dysentery. Girardia which casues diarrheal illness is an example of a protozoan pathogen
					      
					\item Helminths:  These are parasitic worms that can infect humans and are transmitted to others through its eggs or larval forms
					      
				\end{itemize}
				
				\begin{itemize}
					\item As one of the main reasons for treating wastewater is to protect public health, microbiological/pathogen testing of the wastewater effluent and the surface water impacted by the wastewater discharge is conducted to meet the requirements of a wastewater discharge permit, to monitor the pathogen impact of treated wastewater discharge and assess the level of contamination of a public body of water.
					\item The bacteriological tests involves detection and quantification of one or more of the following bacteria:  total coliforms, fecal coliforms, E. Coli, and Enterococcus.  
					      \begin{itemize}
					      	\item The main reason why these bacteria such as coliforms and enterococcus are used \hl{as it is not practical to detect and quantify all pathogens associated with wastewater.}  
					      	\item These selected bacteria originate from feces and indicate fecal contamination and thus serve as an indicator organisms for pathogens of wastewater origin.  
					      	\item Also, they are abundant, potentially less harmful, and easy to detect.  E. coli has been shown to be a better predictor of the potential for impacts to human health and therefore many newer wastewater discharge permits require E. Coli testing in lieu of fecal coliform testing requirements.
					      \end{itemize}
					\item The microbiological test sample is always collected as a grab in a clean, sterile borosilicate glass or plastic bottle containing sodium thiosulfate. 
					      \begin{itemize}
					      	\item Sodium thiosulfate is added to remove residual chlorine which will kill coliforms during transit. 
					      	\item If the sample is not preserved or maintained under proper conditions until the test is conducted in the laboratory, the test would provide erroneous results.
					      	\item Samples must be refrigerated if they cannot be analyzed within 1 hour of collection and must be handled with care to prevent contamination and adverse conditions such as prolonged exposure to direct sunlight.
					      	\item The maximum holding time for state or federal permit reporting purposes is 6 hours. 
					      \end{itemize} 
					\item As it is not possible to exactly quantify the number of bacteria present, a statistical based - \hl{Most Probable Number (MPN)} approach is utilized.  The methods for wastewater bacteriological tests include:  multiple-tube fermentation technique, membrane filtration and quanti-tray testing. 
				\end{itemize}
			\end{enumerate}

\subsection{Specific Gravity}\index{Specific Gravity}				
			\begin{itemize}
				\item Specific gravity is a term to express the weight of a solution with respect to that of water
				\item Water weighs 1 kg/L or 8.34 lbs/gallon or 62.4 lbs/ft$^3$
				\item A solution with a specific gravity of 1.2 will weigh 1.2 times the same volume of water.  1 L of that solution will weigh ( 1.2 kg )/L  or  ( 1.2*8.34=10lbs )/gallon.
				\item Typically wastewater and the associated unthickened sludge, for all practical purposes is assumed to have a specific gravity of 1 - implying 8.34 lbs/gallon.
				\item Specific gravity is typically used for calculations related to chemicals used in wastewater treatment.
			\end{itemize}
\chapterimage{SamplingCoverBW.png}
\chapter{Wastewater Sampling}		
\section{Wastewater Sampling}\index{Wastewater Sampling}
		\begin{itemize}
			\item Field or laboratory measurement of a certain parameter is critical in wastewater treatment operations to obtain information about wastewater characteristics in order to either characterize a wastewater stream, or to monitor a treatment process or for permit compliance.  
			\item A sample is a small part of the whole representing the whole.  Thus, a sample needs to be such that it truly represents the entire population – which in a wastewater operations could be either a wastewater stream, wastewater solids or a chemical used.
		\end{itemize}
		
\subsection{Sampling Methods}\index{Sampling Methods}
\subsubsection{Grab Samples}\index{Grab Samples}
				\begin{itemize}
					\item A grab sample is a sample collected at a specific spot at a site over a short period of time.  
					\item Grab sampling allows for instantaneous analysis of parameters such as pH, dissolved oxygen, chlorine residual, temperature and other parameters which change rapidly with time.
					\item A grab sample represents a snapshot of space and time of a process stream.
					\end{itemize}
\subsubsection{Composite Samples}\index{Composite Samples}
				\begin{itemize}
					\item A composite sample is a collection of discrete samples are combined over a certain period or space and therefore represent the average performance of a wastewater treatment plant or a process during the collection period.\\  
					\item Composite sampling can be either based on:
					      
					      1. constant time interval (time proportioned sampling)\\
					      2. constant wastewater volume interval (flow-proportioned sampling), and\\
					      3. treatment process space - includes samples taken at different depths\\
					      
					\item Composite samples are typically collected using automated samplers which can be programmed to collect samples at pre-established time intervals – for time proportional sampling.
					\item Time and space composite samples are collected by adding equal volumes of samples collected from different times or locations.  
					\item Flow proportional composite samples comprise of volume of each subsample based on flow.\\  
				\end{itemize}
				
			\begin{center}
				\includegraphics[scale=0.2]{Autosampler} \hspace{2cm} \includegraphics[scale=0.37]{Grabsampler}\\
			\end{center}
			\hspace{2.3cm} Automated Sampler \hspace{2.0cm} \parbox{\textwidth}{Grab Sampling Using a Long Handle Dipper}\\

\subsubsection{Sampling Precautions and Protocols}\index{Sampling Precautions and Protocols}
			\begin{itemize}
				\item Samples should represent the major portion of the process or the process stream and should be taken from places where the mixing is thorough, avoiding dead spots and areas of heavier or lighter loadings. 
				\item The collected sample is invariably exposed to conditions very different from the original source and is subject to change due to chemical and microbiological activity.  
				\item Thus, in order to ensure integrity of sample, sample preservation techniques specific to the analysis to be performed is needed.  
				      \begin{itemize}
				      	\item The preservation technique should not only allow for stabilizing the parameter to be analyzed, it should also not interfere with the analyses.  
				      	\item The common preservation techniques involve use of proper containers, temperature control, addition of chemical preservatives, and observance of the recommended maximum sample holding time.
				      \end{itemize}
			\end{itemize}

\section{Data Reporting}\index{Data Reporting}	
		\begin{itemize}
			\item Arithmetic mean is typically calculated for reporting data where multiple samples have been collected and analyzed for the same process stream at different times and for reporting average value over a certain time period – daily, monthly etc.\\ \item Arithmetic mean mathematically is calculated by adding all the result values and dividing by the total number of data points.\\
		\end{itemize}
		Mathematically the arithmetic mean is represented as:\\
		$$\bar{x}=\frac{\sum_{i=1}^{n} x^i}{n} = \frac{x_1+x_2+x_3...x_n}{n}$$
		For example:\\
		Arithmetic mean of the following set of data points:  200, 304, 250, 400 is calculated as:\\
		\vspace{10pt}
		Arithmetic Mean = $\frac{200 + 302 + 250 + 400}{4}= 288$\\
		\vspace{10pt}
		For data sets for analysis such as fecal coliform could include values which vary by several orders of magnitudes, using the arithmetic mean to report the average value is not appropriate as the lower or higher values would bias the calculated mean.\\
		\vspace{10pt}
		For example, consider a data set with values:  260, 300, 500, 5,000, 320 and 200.\\
		\vspace{10pt}
		The arithmetic mean = $\frac{260+300+500+5,000+320+200}{6} = 3,444$\\
		Here the 5000 value completely skews the arithmetic mean.
		
		Therefore, for such tests, the geometric mean calculation is used for reporting the average value.\\
		
		
		Mathematically a geometric mean is represented as:\\
		$$\Bigg(\prod_{i=a}^n\Bigg)^{\frac{1}{n}}=\sqrt[n]{a_1*a_2*a_3...a_n}$$
		 
		Calculation method:\\
		1.	Find the product of all the data points (analogous to first calculating the sum of all the data points when calculating the arithmetic mean)\\
		260*300*500*5,000*320*200 = 12,480,000,000,000,000\\
		2.	Raise the product to the inverse of the number of data points\\
		(*Using the power function of a scientific calculator)\\
		Here n (\# data points) = 6 $\implies$ geometric mean = $(12,480,000,000,000,000)^{\frac{1}{6}}   = 482$

\chapterimage{Collections.jpg} % Chapter heading image

\chapter{Collections}






The collection system resembles a tree that branches out from the treatment plant to collect the wastewater from individual sources.

\section{Wastewater Collection Piping}\index{Wastewater Collection Piping}	
	\begin{itemize}
		\item A \hl{lateral} is the piping that connects the public sewer to the building. 
		\item Laterals flow into larger lines called \hl{mains}.
		\item Mains carry the flow into the largest lines in the system, called \hl{trunk lines}. 
		\item A trunk line is the pipe that brings water into the treatment plant.
	\end{itemize}
\section{Sanitary Sewer Systems}\index{Sanitary Sewer Systems}

Sanitary sewer systems collect and convey wastewater from residential, commercial and industrial sources to a centralized wastewater treatment facility for treatment. 

\subsection{Storm-water systems}\index{Storm-water systems}

Storm-water systems are designed solely for the conveyance of storm-waters waters directly to streams, rivers, lakes, or the ocean.
 
\subsection{Combined sewer systems}\index{Combined sewer systems}
\begin{itemize}
\item Combined sewer systems collect and convey sanitary sewage and urban runoff in a common piping system.
\item Combined sewers could potentially cause serious water pollution problems during combined sewer overflow (CSO) events when wet weather flows exceed the sewage treatment plant capacity.
	\end{itemize}
\begin{center}
\includegraphics[scale=0.45]{SeperatedSystem1} \hspace{1 cm} \includegraphics[scale=0.45]{CombinedSystem1}
\end{center}
			\hspace{2.6cm} Separated System \hspace{3.2cm} \parbox{\textwidth}{Combined System}\\

\section{Collections Systems Basics}\index{Collections Systems Basics}
	\begin{itemize}
\item The primary type of a collection system is a \hl{gravity system}. A gravity system is so named because the wastewater flows down gradient in the sewer, driven by forces of gravity. 
\item The collection system includes the gravity sewers, force mains, manholes, pumping equipment, and other facilities that collect and convey the water to a wastewater treatment plant. 
\item Sewers are generally laid at a minimum slope to ensure open channel flow through the pipe at a \hl{minimum velocity of 2.0 feet per second}. The minimum velocity is required to ensure that solids do not settle out in the sewer.  
\item When the sewer lines reach a certain depth, the flow must be lifted back through a lift or pump station.  
\item \hl{Lift stations} are built whenever wastewater must be pumped to a higher altitude, whether it's to lift water up so that it can gravity flow or to pump it over a rise or hill.  
\item The discharge from the pump station may be to another gravity sewer at that location or through a pressurized force main. 
\item Key elements of lift stations include a wastewater receiving well (wet-well), pumps and piping with associated valves.
\item The size of the wet well affects the operating of the station. If a wet well is too small, excessive starting and stopping of the pump motors will occur, resulting in premature failure. If the wet well is too large, solids will tend to settle on the bottom, blocking the pump suction line and leading to the generation of hydrogen sulfide and methane.
\item The dry well is the portion of the dry well/wet well pumping station that houses the necessary equipment required to pump the wastewater. The dry well is so named because it is isolated from the incoming wastewater.
\item Centrifugal pumps are the most common type of pump found in wastewater pumping stations. 
\item In the USA, wastewater generated in a typical home is about 70 gal/day/person
\end{itemize}

\section{Collections Related Operational Issues}\index{Collections Related Operational Issues}
Infiltration and inflow (I/I) is the unwanted flow into the wastewater collections systems.
\subsection{Infiltration}\index{Infiltration}
\begin{itemize} 
\item Groundwater entering sanitary sewers through defective pipe joints and broken pipes is called infiltration. \hl{Remember:  \textbf{ground filters}}
\end{itemize}

\subsection{Inflow}\index{Inflow}
\begin{itemize} 
\item During rainstorms or snow thaws, large volumes of water may flow into the wastewater collections systems through leaky manhole covers or combined storm-water /wastewater connections.  In addition, private residences may have roof, cellar, yard, area, or foundation drains inappropriately connected to sanitary sewers.  These flows are termed as inflows. \hl{Remember: \textbf{rain flow}}
\end{itemize}

\textbf{Implications of I/I:}\\%$$$$$$$$$$$$$$$$$$$$%
\begin{itemize}
\item I/I decrease the efficiency and capacity of wastewater collection systems and treatment systems 
\item I/I can advance the need for capital costs to manage and treat flows
\item I/I contribute to the hydraulic overloading of treatment processes, which can affect public health and the compliance with NPDES permit requirements
\item I/I could cause Sanitary and combined sewer overflows(SSOs and CSOs) when wastewater flow volumes exceed the design capacity of the treatment plant 
\item I/I increase collection system and treatment facility operating costs
\end{itemize}

\subsection{Odors}\index{Odors}
\begin{itemize}
\item Hydrogen sulfide (H$_2$S), its associated compounds and methane are generated due to microbial

 activity in wastewater and the conveyance systems.  The \hl{rotten egg like smell} of hydrogen sulfide causes public nuisance odors, poses a health hazard for collections systems workers and causes corrosion of the system through its conversion to sulfuric acid.  The hydrogen sulfide generation is typically controlled by reducing the potential for septicity in wastewater through proper design of the system -  adequate velocities and adequate air space and through chemical treatment.
 \end{itemize}

\subsection{Fats, Oils and Grease (FOG)}\index{Fats, Oils and Grease (FOG)}
\begin{itemize}
\item FOG from food home, food establishments and industries affect the operation of the collection system
\item FOG has a tendency to accumulate in sewer pipes decreasing its wastewater carrying capacity.
\item Excessive FOG accumulation may cause sewer overflows
 \end{itemize}
 
\chapterimage{QuizIcon.jpg}
\chapter{Week 2 Module Assessment Quiz}

1. A lateral is the largest sewer line which brings the wastewater to the treatment plant\\

a. True \\
b. False \\
\vspace{0.5cm}

2. Total solids are made up of dissolved and suspended solids both of which contain organic and inorganic matter\\

a. True\\ 
b. False \\
\vspace{0.5cm}


3. A 24-hr flow proportioned sample may be collected for a fecal coliform test\\

a. True \\
b. False \\
\vspace{0.5cm}


4. Inflow is storm water entering into the sewer system\\

a. True \\
b. False \\
\vspace{0.5cm}


5. Fats, oils and grease accumulation in sewers can cause sewer overflows\\

a. True \\
b. False \\

\vspace{0.5cm}

6. A sewer system designed to transport only wastewater from homes, industries, institutions and businesses is called:\\

a. Combined sewer \\
b. Sanitary sewer \\
c. Service sewer \\
d. Building sewer \\

\vspace{0.5cm}

7. A device called an Imhoff cone is commonly used to measure settleable solids in:\\

a. \% \\
b. mL/L \\
c. mg/L \\
d. ppm \\
e. SVI units \\

\vspace{0.5cm}

8. An aerobic treatment process is one that requires the presence of:\\

a. Ozone \\
b. organic oxygen \\
c. no oxygen \\
d. combined oxygen \\
e. dissolved oxygen \\

\vspace{0.5cm}

9. Sludge solids in wastewater has an average specific gravity of 1.2; this means they are\\

a. 12\% heavier than water \\
b. 20\% heavier than water \\
c. 2\% lighter than water \\
d. 20\% lighter than water \\

\vspace{0.5cm}

10. Positive ORP indicates the presence of an $\rule{1cm}{0.15mm}$ environment

a. Oxidative \\
b. Reductive \\
c. Anaerobic \\
d. Toxic. \\
\vspace{2cm}
\textbf{Answers:}\\
1.	b. \\ 

\vspace{0.5cm}

2.	a. \\ 

\vspace{0.5cm}

3.	b.  \\
\vspace{0.5cm}


4.	a.  \\
\vspace{0.5cm}


5.	a.  \\
\vspace{0.5cm}


6.	b.  \\
\vspace{0.5cm}


7.	b.  \\

\vspace{0.5cm}

8.	e.  \\








					      
					% \item Nitrogen removal as part of the secondary treatment mainly involves nitrification-denitrification processes which involve conversion of ammonia nitrogen to nitrogen gas.
					      
% 				\end{itemize}
% 				\begin{snugshade*}
% 					\item \noindent\textsc{Phosphorous}
% 				\end{snugshade*}
% 				\begin{itemize}
% 					\item Forms of phosphorous:\\
% 					      \begin{itemize}
% 					      	\item The principal forms are organically bound phosphorus, polyphosphates, and orthophosphates.
% 					      	\item Organically bound phosphorus originates from body and food waste and, upon biological decomposition of these solids, is converted to orthophosphates. 
% 					      	\item Polyphosphates originate from synthetic detergents and are hydrolyzed to orthophosphates. Thus, the principal form of phosphorus in wastewater is assumed to be orthophosphates, although the other forms may exist. Orthophosphates consist of the negative ions PO$_4$$^{3-}$, HPO$_4$$^{2-}$, and H$_2$PO$_4$ $^-$.  These may form chemical combinations with cations (positively charged ions).
% 					      \end{itemize}
% 					\item Phosphorous removal in wastewater is typically accomplished by:
% 					      \begin{itemize}
% 					      	\item Biological uptake which utilizes specific phosphorous accumulating bacteria, and 
% 					      	\item Chemical precipitation
% 					      \end{itemize}
% 				\end{itemize}
% 			\end{enumerate}%@@@@@@@@@@@@@@@@@@%
% 			\pagebreak	
% 			\begin{snugshade*}
% 				\item \noindent\textsc{Oil and Grease}
% 			\end{snugshade*}
% 			Fats, oil and grease in wastewater originate from homes, food establishments and industries.
% 			\begin{itemize}
% 				\item Oil and grease content of wastewater is established in the laboratory by extracting it with a solvent - \textit{n}-hexane.  The concentration of oil and grease is reported in mg/l and typical oil and grease content of wastewater ranges from 80 - 120 mg/l
% 				\item Presence of excessive oils and grease could potentially impact the secondary treatment process
% 				\item Oils and grease are removed as floatables in primary treatment and sent with the sludge to the digesters
% 			\end{itemize}
% 	\end{enumerate}			
				
% 		%______________%
		
% 		\pagebreak
% 		\begin{snugshade*}
% 			\item \noindent\textsc{Wastewater Sampling}\\%$$$$$$$$$$$$$$$$$$$$%
% 		\end{snugshade*}
% 		\begin{itemize}
% 			\item Field or laboratory measurement of a certain parameter is critical in wastewater treatment operations to obtain information about wastewater characteristics in order to either characterize a wastewater stream, or to monitor a treatment process or for permit compliance.  
% 			\item A sample is a small part of the whole representing the whole.  Thus, a sample needs to be such that it truly represents the entire population – which in a wastewater operations could be either a wastewater stream, wastewater solids or a chemical used.
% 		\end{itemize}
		
% 		\begin{enumerate}[A.]
% 			\definecolor{shadecolor}{RGB}{225, 235, 235}
% 			\begin{snugshade*}
% 				\item \noindent\textsc{Sampling Methods:}\\%$$$$$$$$$$$$$$$$$$$$%
% 			\end{snugshade*}
			
% 			\begin{enumerate}[i.
% ]				\definecolor{shadecolor}{RGB}{220, 220, 220}
% 				\begin{snugshade*}
% 					\item \noindent\textsc{Grab Sampling}
% 				\end{snugshade*}
% 				\begin{itemize}
% 					\item A grab sample is a sample collected at a specific spot at a site over a short period of time.  
% 					\item Grab sampling allows for instantaneous analysis of parameters such as pH, dissolved oxygen, chlorine residual, temperature and other parameters which change rapidly with time.
% 					\item A grab sample represents a snapshot of space and time of a process stream. 
% 				\end{itemize}
% 				\begin{snugshade*}
% 					\item \noindent\textsc{Composite Sampling}
% 				\end{snugshade*}
% 				\begin{itemize}
% 					\item A composite sample is a collection of discrete samples are combined over a certain period or space and therefore represent the average performance of a wastewater treatment plant or a process during the collection period.\\  
% 					\item Composite sampling can be either based on:
					      
% 					      1. constant time interval (time proportioned sampling)\\
% 					      2. constant wastewater volume interval (flow-proportioned sampling), and\\
% 					      3. treatment process space - includes samples taken at different depths\\
					      
% 					\item Composite samples are typically collected using automated samplers which can be programmed to collect samples at pre-established time intervals – for time proportional sampling.
% 					\item Time and space composite samples are collected by adding equal volumes of samples collected from different times or locations.  
% 					\item Flow proportional composite samples comprise of volume of each subsample based on flow.\\  
% 				\end{itemize}
				
% 			\end{enumerate}
% 			\begin{center}
% 				\includegraphics[scale=0.2]{Autosampler} \hspace{2cm} \includegraphics[scale=0.37]{Grabsampler}\\
% 			\end{center}
% 			\hspace{2.3cm} Automated Sampler \hspace{2.0cm} \parbox{\textwidth}{Grab Sampling Using a Long Handle Dipper}\\
% 			\begin{snugshade*}
% 				\item \noindent\textsc{Sampling Precautions and Protocols::}\\%$$$$$$$$$$$$$$$$$$$$%
% 			\end{snugshade*}
% 			\begin{itemize}
% 				\item Samples should represent the major portion of the process or the process stream and should be taken from places where the mixing is thorough, avoiding dead spots and areas of heavier or lighter loadings. 
% 				\item The collected sample is invariably exposed to conditions very different from the original source and is subject to change due to chemical and microbiological activity.  
% 				\item Thus, in order to ensure integrity of sample, sample preservation techniques specific to the analysis to be performed is needed.  
% 				      \begin{itemize}
% 				      	\item The preservation technique should not only allow for stabilizing the parameter to be analyzed, it should also not interfere with the analyses.  
% 				      	\item The common preservation techniques involve use of proper containers, temperature control, addition of chemical preservatives, and observance of the recommended maximum sample holding time.
% 				      \end{itemize}
% 			\end{itemize}
% 		\end{enumerate}
		

% 	\begin{snugshade*}
% 		\item \noindent\textsc{Wastewater Parameters}%$$$$$$$$$$$$$$$$$$$$%
% 	\end{snugshade*}
% 		Laboratory and field tests are conducted to measure parameters which are critical for monitoring and controlling treatment.  The following are the key parameters that are measured.	
			
% 		\begin{enumerate}[A.]
% 			\definecolor{shadecolor}{RGB}{225, 235, 235}
% 			\begin{snugshade*}
% 				\item \normalsize{\textbf{pH}}
% 			\end{snugshade*}
% 			\hl{pH is a measure of the hydrogen ion (H$^+$) content or the acidity or basicity of a solution.}  pH impacts the chemical and micribiological elements of wastewater treatment processes and thus pH measurement and control is critical.
% 			\begin{itemize}
% 				\item Pure water dissociates into equal concentration of hydrogen ions and hydroxide ions:\\ 
% 				      H$_2$O → H+ + OH$^-$.
% 				\item The H$^+$ are responsible for acidic properties and the OH$^-$ ions for the basic properties.  
% 				\item pH is the inverse of H$^+$ concentration; pH increases when the concentration of H$^+$ decreases relative to the concentration of OH-. 
% 				\item pH scale ranges from 0 – 14. When the concentration of both H$^+$ and OH$^-$ are equal, as in pure water, it is considered neutral and its pH is 7.0.  \item If the pH of a sample solution is below 7.0, the sample is termed acidic and is alkaline or basic if its pH is above 7.0. 
% 				\item Each change of 1 pH unit represents a 10 fold change in concentration.  For example, a sample with a pH of 2.0 is 1000 times more acidic than a sample with a pH of 5.0. 
% 				\item pH is measured by an electrode that is sensitive only to H$^+$ or using a pH strip which is essentially an adsorbent paper which is pre-impregnated with chemicals which change color under different H$^+$ concentrations.
% 				\item Most organisms involved in biological wastewater treatment processes do well within a a narrow range of pH near neutral (pH of 7).			
% 			\end{itemize}
			
% 			\begin{snugshade*}
% 				\item \noindent\textsc{ Oxidation Reduction Potential (ORP):}
% 			\end{snugshade*}
% 			ORP is a measure of the potential of a wastewater to allow for microbiological oxidation and reduction reactions.\\
% 			An example of wastewater treatment process where microbiological oxidation involved include the activated sludge process.  Whereas, anaerobic digestion is a microbiological reduction based process.
% 			\begin{itemize}
% 				\item ORP value is measured in millivolts (mV), using an electrochemical probe
% 				\item The measured ORP value provides an indication of the potential of oxidation and reduction reactions in that sample.
% 				\item A higher positive ORP is indicative of oxidation potential of the wastewater whereas a negative ORP value indicates potential for a reduction reaction to occur.
% 				\item ORP measurements are used for controlling treatment processes including nitrification, odor control and disinfection.  For example, in the disinfection process utilizing chlorine (bleach), ORP measurements provides the strength of the oxidation potential of chlorine present in the wastewater being disinfected.  This measurement is used for precisely controlling the chlorine dosing.  The proper chlorine control results in optimizing chlorine dosing which reduces potential toxicity issues and dechlorination costs.
% 			\end{itemize}
% 			Typical ORP applications include the following processes and process elements
			
			
% 			\setlength{\arrayrulewidth}{0.3mm}
% 			\setlength{\tabcolsep}{8 pt}
% 			\renewcommand{\arraystretch}{0.8}
% 			\begin{center}
% 				\begin{tabular}{ |p{9.5cm}|p{4.0cm}|}
% 					\hline
% 					\multicolumn{2}{|c|}{\textbf{Typical Wastewater Process ORPs}} \\
% 					\hline
					
% 					\hline
% 					\small Collections	Sulfide formation                                & \small -50 to -250 mV  \\
% 					\small Influent wastewater                                          & \small - 200 mV        \\
% 					\small Activated sludge	cBOD degradation with free molecular oxygen & \small +50 to +250 mV  \\
% 					\small Biological phosphorous removal                               & \small +25 to +250 mV  \\
% 					\small Denitrification                                              & \small +50 to -50 mV   \\
% 					\small Anaerobic Digestion: Acid formation (Acidogenesis)           & \small -100 to -225 mV \\
% 					\small Anaerobic Digestion: Methane production (Methanogenesis)     & \small -75 to -400 mV  \\
% 					\hline
% 				\end{tabular}
				
% 			\end{center}
			
% 			\begin{snugshade*}
% 				\item \noindent\textsc{Alkalinity}
% 			\end{snugshade*}
% 			\begin{itemize}
% 				\item \hl{Alkalinity is the ability of a water to neutralize acids.}  
% 				\item During certain wastewater treatment processes including anaerobic digestion, acids are generated as a result of microbiological activity.  The bacteria and other biological entities which play an active role in wastewater treatment are most effective at a neutral to slightly alkaline pH of 7 to 8.  In order to maintain these optimal pH conditions for biological activity there must be sufficient alkalinity present in the wastewater to neutralize acids generated by the active biomass.
% 				\item This ability to maintain the proper pH in the wastewater as it undergoes treatment is the reason why alkalinity is so important to the wastewater industry.
% 				\item The alkalinity is due to the presence of acid neutralizing bases in the water including the hydroxyl (OH$^-$), carbonate (CO$_3$$^-$) and bicarbonate (HCO$_3$$^-$)  ions.  These ions are of mineral origin and are also formed from carbon dioxide which comes from the atmosphere and from the microbial decomposition of organic material.  The resistance to pH change of the water will continue until all the alkalinity contributing ions are neutralized.  
% 				\item The pH of a water serves as a guide to the types of alkalinity present in the water but is unrelated to the alkalinity content of a water.  Important Note:  Alkalinity is a measure of the ability to neutralize acids whereas a solution is termed alkaline (or basic) if its pH greater than 7. 
% 				\item Alkalinity is expressed as milligrams per liter of CaCO$_3$
% 			\end{itemize}
			
% 			\begin{snugshade*}
% 				\item \noindent\textsc{Dissolved oxygen:}
% 			\end{snugshade*}
% 			\begin{itemize}
% 				\item Dissolved oxygen (DO) is the concentration of oxygen dissolved in the wastewater sample and is typically measured in the field using a DO probe.  A titration based Winkler Test is used in the laboratory
% 				\item The \hl{presence of oxygen indicates an aerobic environment} where dissolved, free oxygen is available for aerobic microorganisms to live, BOD removal in the activated sludge process occurs as a result of the activity of aerobic bacteria.  The absence of DO indicates that the environment or condition is either anoxic or anaerobic.  
% 				\item \hl{In an anoxic environment, free oxygen is not present, but oxygen is available from its combined  forms - nitrate (NO$_3$ $^-$) and sulfate (SO$_4$ $^-$)} for the the consumption of microorganisms.  Example of an anoxic process is denitrification.  In denitrification, the anoxic bacteria in the presence of food (cBOD) consume the combined oxygen in nitrates (NO$_3$ $^-$ ) and convert it to nitrogen gas.
% 				\item \hl{The complete absence of oxygen including free and combined oxygen is an anaerobic environment.}
% 				\item Microorganisms are termed as obligate aerobes if they cannot survive without free oxygen.  Facultative aerobes are microorganisms which can survive in both aerobic and anaerobic environments.  
% 			\end{itemize}
% 			\begin{snugshade*}
% 				\item \noindent\textsc{Microbiological testing and monitoring:}
% 			\end{snugshade*}
% 			Microbiological testing and monitoring is conducted as part of the wastewater treatment for two main reasons:
% 			\begin{enumerate}[1.]
% 				\item Heterotrohic (organisms that consume organic material) microbes are responsible for the biological wastewater treatment processes - secondary treatment process, digestion and nutrient removal; and
% 				\item Pathogens - agents that cause disease are present in wastewater effluent.
% 			\end{enumerate}
% 			\begin{enumerate}
% 				\definecolor{shadecolor}{RGB}{220, 220, 220}
% 				\begin{snugshade*}
% 					\item \noindent\textsc{Microbiological testing related to monitoring and troubleshooting biological wastewater treatment:}\\
% 				\end{snugshade*}
				
% 				Microbes involved in biological wastewater treatment processes include:\\
% 				\begin{itemize}
% 					\item Fungi - Filamentous fungi occasionally bloom in activated sludge processes due to low pH or nutrient deficiency and cause problems with the settleability.
% 					\item Protozoa - Protozoas play a important role in the secondary treatment process.  Common protozoas in the activated sludge process include:
% 					      \begin{itemize}
% 					      	\item Amoeba
% 					      	\item Flagellate
% 					      	\item Cilliate
% 					      \end{itemize}
% 					\item Rotifers
% 					\item Nematodes
% 					\item Bacteria - Bacteria is the predominant microorganism responsible for the biological wastewater water treatment.  
% 				\end{itemize}
% 				\begin{itemize}
% 					\item The effectiveness of the biological wastewater treatment processes is primarily due to the presence of a microbial ecosystem with a right balance of populations of different microbial species.
% 					\item Methods used for monitoring the microbial composition include direct monitoring using a light microscope to see which and how many of the different microbial species are present - typically used for activated sludge process.
% 					\item Indirect method includes monitoring other parameters such as pH and alkalinity which are influenced by microbiological activity.
% 					\item The microbial monitoring ensures process stability and helps identify potential process upset conditions caused by changes to the microbial population due to other external factors - toxicty, organic loading, temperature etc.
% 				\end{itemize}
				

% 				\begin{snugshade*}
% 					\item \noindent\textsc{Microbiological testing related to monitoring and controlling pathogens in treated wastewater effluent :}\\
% 				\end{snugshade*}
				
% 				Pathogens in wastewater belong to the following groups:
% 				\begin{itemize}
% 					\item Bacteria:  Although, bacteria is present in large numbers in feces, pathogenic or bacteria are present only because of an infection and this pathogenic bacteria can potentially spread the infection to other healthy individuals.  Disease spread by pathogenic bacteria include diarrhea, cholera and tyhoid among many others.
					      
% 					\item Viruses: A large number of viruses may infect humans and are present in feces.  These include enteroviruses (including polioviruses), hepatitis A virus, reoviruses and diarrhoea-causing viruses (especially rotavirus).
					      
% 					\item Protozoa:  Many species of protozoa can infect humans and cause diarrhoea and dysentery. Girardia which casues diarrheal illness is an example of a protozoan pathogen
					      
% 					\item Helminths:  These are parasitic worms that can infect humans and are transmitted to others through its eggs or larval forms
					      
% 				\end{itemize}
				
% 				\begin{itemize}
% 					\item As one of the main reasons for treating wastewater is to protect public health, microbiological/pathogen testing of the wastewater effluent and the surface water impacted by the wastewater discharge is conducted to meet the requirements of a wastewater discharge permit, to monitor the pathogen impact of treated wastewater discharge and assess the level of contamination of a public body of water.
% 					\item The bacteriological tests involves detection and quantification of one or more of the following bacteria:  total coliforms, fecal coliforms, E. Coli, and Enterococcus.  
% 					      \begin{itemize}
% 					      	\item The main reason why these bacteria such as coliforms and enterococcus are used \hl{as it is not practical to detect and quantify all pathogens associated with wastewater.}  
% 					      	\item These selected bacteria originate from feces and indicate fecal contamination and thus serve as an indicator organisms for pathogens of wastewater origin.  
% 					      	\item Also, they are abundant, potentially less harmful, and easy to detect.  E. coli has been shown to be a better predictor of the potential for impacts to human health and therefore many newer wastewater discharge permits require E. Coli testing in lieu of fecal coliform testing requirements.
% 					      \end{itemize}
% 					\item The microbiological test sample is always collected as a grab in a clean, sterile borosilicate glass or plastic bottle containing sodium thiosulfate. 
% 					      \begin{itemize}
% 					      	\item Sodium thiosulfate is added to remove residual chlorine which will kill coliforms during transit. 
% 					      	\item If the sample is not preserved or maintained under proper conditions until the test is conducted in the laboratory, the test would provide erroneous results.
% 					      	\item Samples must be refrigerated if they cannot be analyzed within 1 hour of collection and must be handled with care to prevent contamination and adverse conditions such as prolonged exposure to direct sunlight.
% 					      	\item The maximum holding time for state or federal permit reporting purposes is 6 hours. 
% 					      \end{itemize} 
% 					\item As it is not possible to exactly quantify the number of bacteria present, a statistical based - \hl{Most Probable Number (MPN)} approach is utilized.  The methods for wastewater bacteriological tests include:  multiple-tube fermentation technique, membrane filtration and quanti-tray testing. 
% 				\end{itemize}
% 			\end{enumerate}
			
% 			\begin{snugshade*}
% 				\item \noindent\textsc{Specific Gravity}
% 			\end{snugshade*}
% 			\begin{itemize}
% 				\item Specific gravity is a term to express the weight of a solution with respect to that of water
% 				\item Water weighs 1 kg/L or 8.34 lbs/gallon or 62.4 lbs/ft$^3$
% 				\item A solution with a specific gravity of 1.2 will weigh 1.2 times the same volume of water.  1 L of that solution will weigh ( 1.2 kg )/L  or  ( 1.2*8.34=10lbs )/gallon.
% 				\item Typically wastewater and the associated unthickened sludge, for all practical purposes is assumed to have a specific gravity of 1 - implying 8.34 lbs/gallon.
% 				\item Specific gravity is typically used for calculations related to chemicals used in wastewater treatment.
% 			\end{itemize}
			
% 		\end{enumerate}
% 		\pagebreak
% 		\begin{snugshade*}
% 			\item \noindent\textsc{Data Reporting:}\\%$$$$$$$$$$$$$$$$$$$$%
% 		\end{snugshade*}
% 		\begin{itemize}
% 			\item Arithmetic mean is typically calculated for reporting data where multiple samples have been collected and analyzed for the same process stream at different times and for reporting average value over a certain time period – daily, monthly etc.\\ \item Arithmetic mean mathematically is calculated by adding all the result values and dividing by the total number of data points.\\
% 		\end{itemize}
% 		Mathematically the arithmetic mean is represented as:\\
% 		$$\bar{x}=\frac{\sum_{i=1}^{n} x^i}{n} = \frac{x_1+x_2+x_3...x_n}{n}$$
% 		For example:\\
% 		Arithmetic mean of the following set of data points:  200, 304, 250, 400 is calculated as:\\
% 		\vspace{10pt}
% 		Arithmetic Mean = $\frac{200 + 302 + 250 + 400}{4}= 288$\\
% 		\vspace{10pt}
% 		For data sets for analysis such as fecal coliform could include values which vary by several orders of magnitudes, using the arithmetic mean to report the average value is not appropriate as the lower or higher values would bias the calculated mean.\\
% 		\vspace{10pt}
% 		For example, consider a data set with values:  260, 300, 500, 5,000, 320 and 200.\\
% 		\vspace{10pt}
% 		The arithmetic mean = $\frac{260+300+500+5,000+320+200}{6} = 3,444$\\
% 		Here the 5000 value completely skews the arithmetic mean.
		
% 		Therefore, for such tests, the geometric mean calculation is used for reporting the average value.\\
		
		
% 		Mathematically a geometric mean is represented as:\\
% 		$$\Bigg(\prod_{i=a}^n\Bigg)^{\frac{1}{n}}=\sqrt[n]{a_1*a_2*a_3...a_n}$$
		 
% 		Calculation method:\\
% 		1.	Find the product of all the data points (analogous to first calculating the sum of all the data points when calculating the arithmetic mean)\\
% 		260*300*500*5,000*320*200 = 12,480,000,000,000,000\\
% 		2.	Raise the product to the inverse of the number of data points\\
% 		(*Using the power function of a scientific calculator)\\
% 		Here n (\# data points) = 6 $\implies$ geometric mean = $(12,480,000,000,000,000)^{\frac{1}{6}}   = 482$
		
		
		
	
	
	
% 	%
% 	%\begin{enumerate}
% 	%	\definecolor{shadecolor}{RGB}{179, 203, 203}
% 	%\begin{snugshade*}
% 	%\item \noindent\textsc{Solids Analysis}%$$$$$$$$$$$$$$$$$$$$%
% 	%\end{snugshade*}
% 	%
% 	%\vspace{0.4cm}
% 	%\end{enumerate}
% 	%\begin{snugshade*}
% 	%\item \noindent\textsc{BOD}%$$$$$$$$$$$$$$$$$$$$%
% 	%\end{snugshade*}
	
	
% \end{enumerate}
% \pagebreak
% \begin{center}
% \phantom{A}
% \vspace{10cm}

% BLANK PAGE
% \end{center}
% \pagebreak
%	\begin{enumerate}[A.]%___________%
%			\definecolor{shadecolor}{RGB}{225, 235, 235}
%
%				%%%%%%%%%%%
%				% LEVEL 3 %
%				%%%%%%%%%%%
%
%		\begin{snugshade*}
%			\item \noindent\textsc{Organic Matter}%###############################%
%		\end{snugshade*}			
%		\begin{itemize}
%			\item The main reason for treating domestic wastewater is to remove the organic matter.  
%			\item Organics are substances containing carbon, hydrogen and oxygen, and some of which may be combined with nitrogen, sulfur or phosphorous.
%			\item About 50 percent of the solids present in wastewater are organic.  This fraction is generally of animal or vegetable life, dead animal matter, plant tissue or organisms, and also include synthetic organic compounds.
%			\item The principal organic compounds present in domestic wastewater are proteins, carbohydrates and fats together with the products of their decomposition.
%			\item Organics are subject to decay or decomposition through the activity of bacteria and other living organisms.  \hl{Since the organic fraction can be driven off at high temperatures, they are also called \textbf{volatile solids}}.\
%			\item \emph{Organics in wastewater is typically quantified in terms of oxygen required to oxidize the carbon based material present} in wastewater using the following methods:\\
%			      \begin{enumerate}[i.]
%			      	\definecolor{shadecolor}{RGB}{220,220,220}
%					%%%%%%%%%%%
%					% LEVEL 4 %
%					%%%%%%%%%%%
%			      	\begin{snugshade*}
%			      		\item \noindent\textsc{Biochemical Oxygen Demand (BOD)}%@@@@@@@@@@@@@@@@@@%
%			      	\end{snugshade*}					
%			      	\begin{itemize}
%			      		\item The BOD of wastewater is measured in terms of oxygen required for the microorganisms to consume the organic material present.
%			      		\item BOD is typically measured as BOD$_5$ which is the oxygen demand of the wastewater measured after 5 days of the initiation of the test.
%			      		\item The test involves incubating a known dilution of wastewater in a 300 ml bottle for 5 days at 20\si{\degree}C.  The dissolved oxygen (DO) content at the start and end of the incubation period is used for calculating the BOD.
%			      		\item For the test to be considered valid, the following criteria need to be met: 1) DO consumption during the test must be at least 2 mg/l, 2) DO remaining at the end of the test must be at least 1 mg/l, and 3) DO consumed in blank should be 0.2 mg/l or less
%			      		      			
%			      		\item BOD is a parameter to measure the strength of wastewater and the measurement of the wastewater treatment plant or treatment process influent and effluent BOD is standard practice to measure its performance.  Typical domestic wastewater BOD is about 200-250 mg/l.
%			      		\item The oxygen consumed by the microorganisms during the BOD test is primarily for: 1) Oxidizing the carbonaceous material (cBOD – carbonaceous BOD), and 2) Oxidizing nitrogenous constituents such as ammonia (nBOD – nitrogenous BOD).
%			      		\item Thus, BOD (Total) = cBOD + nBOD.  The cBOD and nBOD is measured by adding certain chemical inhibitors which will inhibit the bacteria responsible for consuming the nitrogenous matter, thus measuring only the cBOD as part of the BOD test.
%			      		\item Since not all of the organics is metabolized in the 5 days of the regular BOD test, certain wastewater discharge permits require reporting of the ultimate BOD value (BOD$_U$)\\
%			      	\end{itemize}
%			      	\begin{snugshade*}
%			      		\item \noindent\textsc{Chemical Oxygen Demand (COD)}%@@@@@@@@@@@@@@@@@@%
%			      	\end{snugshade*}		  
%			      	\begin{itemize}
%			      		\item The COD test involves using chemical oxidizers to measure the oxygen demand of the wastewater.
%			      		\item As the chemical oxidizers will oxidize other constituents present, including inorganic matter, the COD value of wastewater will be higher than the BOD.  
%			      		\item The COD test can be conducted rather quickly than the 5 day BOD test, it is an effective method to quantify the wastewater strength and process efficiencies and allow operators to make timely process adjustments.
%			      	\end{itemize}
%			      	\begin{snugshade*}
%			      		\item \noindent\textsc{Total Organic Carbon (TOC):}\\%@@@@@@@@@@@@@@@@@@%
%			      	\end{snugshade*}
%			      	The TOC method utilizes laboratory analytical instruments which directly measure the organic carbon content by quantifying the amount of carbon dioxide produced from the complete combustion of the organics present.
%			      \end{enumerate}
%		\end{itemize}
%		
%		
%		
%			\hl{Note: BOD measures the amount of oxygen required by the microorganisms present to consume the organic material while COD measures the chemical oxidation required to oxidize all chemicals including organics present in wastewater.  BOD value of typical domestic sewage is about 200 - 250 mg/l while the COD value ranges from 300 - 450 mg/l.  Typical BOD:COD ratio ranges from 0.5-0.8.}\\
%		
%		\pagebreak
%				\begin{snugshade*}
%			\item \noindent\textsc{Solids}
%		\end{snugshade*}	
%		Like BOD, wastewater solids is another critical parameter for establishing the wastewater strength and determining treatment process efficiencies. 
%		\begin{itemize}
%			\item The \texthl{solids can be classified as suspended or dissolved} based upon its ability to pass through a standardized filter paper.
%			\item When the wastewater is filtered:
%			      \begin{itemize}
%			      	\item the residual solids remaining on the filter paper after drying in an oven at 103\si{\degree}C is the \hl{suspended solids} portion, and 
%			      	\item the solids remaining after drying the filtrate are the \hl{dissolved solids}.
%			      \end{itemize}
%			\item Suspended solids include larger floating particles and consist of sand, grit, clay, fecal matter, paper, pieces of wood, particles of food and garbage, and similar materials.
%			\item Suspended solids can be categorized based upon its settling characteristics as:
%			      \begin{itemize}
%			      	\item \hl{Settleable}
%			      	\item \hl{Non-settleable}
%			      	      \begin{itemize}
%			      	      	\item \hl{Colloidial}-small, charged (typically negative) particles which do not settle easily.  Some of the colloidial particles are small enough to pass through the filter paper used for filtering the suspended solids
%			      	      	\item \hl{Floatable}-example oil and grease and small plastics
%			      	      \end{itemize}
%			      \end{itemize}
%			\item Dissolved solids in wastewater include organics.  However, the major elements of dissolved solids are inorganic ions such as Ca$^{+2}$, Mg$^{+2}$, Cl$^-$, SO$_4$ $^{-2}$ , HCO$_3$ $^-$, Fe$^{+2}$, PO$_4$ $^{-3}$, NO$_3$ $^-$.  These ions are part of the dissolved salts such as sodium chloride (NaCl), calcium bicarbonate (Ca(HCO$_3$)$_2$), magnesium phosphate (Mg$_3$PO$_4$) and others which are normally present in water and wastewater. 
%			      \begin{itemize}
%			      	\item Conductivity or electrical conductance (EC) measurement is typically conducted as the wastewater enters the plant as \hl{conductivity provides an indirect and simple measure of the amount of dissolved solids present.}  
%			      	\item Conductivity or electrical conductance (EC) is a measure the amount of electrical current that can be conducted by a solution.  
%			      	\item The conductance of electricity in a solution is due to the presence of dissolved inorganic ions 
%			      	\item The higher the concentration of these ions, the higher is the conductivity. 
%			      	\item \underline{Conductivity is measured in the units of µmhos/cm or µSiemens/cm.}  (Note:  mhos is the reverse of ohm which is a measure of resistance).
%			      	\item Typical wastewater conductivities range from 50 to 1500 µS/cm
%			      \end{itemize}
%			\item Both suspended and dissolved solids can be either \hl{volatile (organic)} or \hl{fixed (inorganic)}.
%			\item \hl{Total Solids is thus a sum of TSS and dissolved solids or volatile and fixed solids.}
%			      \begin{itemize}
%			      	\item The volatile solids are typically of plant or animal origin .
%			      	\item The fixed solids include sand, gravel and silt as well as the dissolved salts.
%			      \end{itemize}
%			      \begin{minipage}{0.65\textwidth}
%			      	\item The volatile or fixed fractions are quantified by incinerating the solids in a muffler furnace at 550°C which removes only the volatile solids leaving only the fixed solids behind.
%			      	\item In terms of the size of the solids, the distribution is approximately thirty percent suspended and about seventy percent dissolved solids - which includes the colloidal particles which have passed through the filter paper.\\ 
%			      	\item As primary treatment process involve settling of solids, establishing the settleable portion of the suspended solids is important.\\  
%			      	\item \hl{The settleable solids are quantified using an Imhoff cone and are reported in ml/L}.  Imhoff cone is a 1 liter, clear cone shaped container, with volume graduations (ml) at the bottom.
%			      						
%			      \end{minipage}	
%			      \begin{minipage}{0.25\textwidth}
%			      	\begin{center}
%			      		\includegraphics[scale=0.7]{ImhoffCone}\\
%			      		Imhoff Cone\\
%			      		\emph{Note the ml marking at the bottom of the cone}
%			      	\end{center}
%			      \end{minipage}
%			\item One factor which affects settleability is the conveyance time of the sewage to the treatment plant. 
%			\item The settleable component of the suspended solids will decrease as the sewage becomes more septic due to longer conveyance times.
%			\item Influent and effluent total suspended solids are measured to establish the overall treatment and individual process efficiencies.  
%			\item Volatile solids measurements before and after biological processes such as secondary treatment and digestion provide information on the process efficiency.\\
%		\end{itemize}
%		\begin{enumerate}%$$$$$$$$$$$$$$$$$$$$%
%			\definecolor{shadecolor}{RGB}{220, 220, 220}
%			\begin{snugshade*}
%				\item \noindent\textsc{Procedures for Solids Analysis}
%			\end{snugshade*}		
%			
%			\begin{enumerate}[a.]
%				\definecolor{shadecolor}{RGB}{110, 192, 221}
%				\begin{snugshade*}
%					\item \noindent\textsc{Determining wastewater suspended solids - Total(TSS) and Volatile(VSS) concentrations:}
%				\end{snugshade*}
%				 
%				%\hl{How total suspended solids concentration is established for a wastewater sample:}
%				\begin{itemize}
%					\item A known volume of wastewater sample is filtered through a pre-weighed filter paper
%					\item The suspended solids will be retained by the filter
%					\item The water with the dissolved solids will pass through the filter
%					\item The filter paper with the filter solids is rinsed with distilled water to remove 
%					\item The filter paper with the solids is dried in the oven and then weighed
%					\item The difference between the weight of the dried filter paper with the solids and the pre-weighed filter paper, measured in mg, will be the suspended solids in: mg per the original quantity of wastewater sample taken.  This value can be converted to give the suspended solids content in mg/l
%					\item A filter paper with the dried solids is incinerated in a muffler furnace
%					\item The difference in the weight of the solids, before and after incineration is the fixed solids
%					\item The difference between the weight of the solids before incineration and the fixed solids is the volatile solids
%				\end{itemize}
%				
%				
%				
%				\begin{snugshade*}
%					\item \noindent\textsc{Determining wastewater and sludge total solids (TS) and volatile solids (VS) concentrations:}
%				\end{snugshade*}
%				\begin{itemize}
%					\item A certain quantity of wastewater (by volume) or sludge (by weight) is taken in a pre-weighed dish and weighed.  Note:  the sample is not filtered.
%					\item The dish with the sample is dried in an oven
%					\item The difference in the weight of the pre-weighed dish from that of the dish with the dried sample is the total solids
%					\item The dried solids are incinerated in a muffler furnace
%					\item The difference in the weight of the solids, before and after incineration is the fixed solids
%					\item The difference between the fixed solids and the total solids is the volatile solids
%					\item Total solids of a sludge sample is reported as a \% of the sludge weight.  A 7\% sludge has 7 lbs of solids for every 100 lbs of sludge.
%				\end{itemize}
%				
%							
%				
%				
%					\hl{For sludge samples, volatile solids is typically reported as the volatile solids fraction in \% of the total solids content of the sludge.  For example, if a 8\% sludge (i.e sludge which has 8\% TS or 80,000mg/l solids), is reported to have 70\% volatile, it means that 70\% of the total solids - 0.7*8\%=5.6\% or 56,000mg/l is the sludge volatile solids content.  \emph{70\% volatile does not meet the sludge has 700,000mg/l volatile solids}}\\
%				
%			\end{enumerate}
%			
%			\begin{snugshade*}
%				\item \noindent\textsc{Summary of Wastewater Solids}
%			\end{snugshade*}
%			\begin{itemize}
%				\item Solids in wastewater can be categorized as dissolved or suspended
%				      \begin{itemize}
%				      	\item Suspended solids can be further categorized as settleable or unsettleable
%				      \end{itemize}
%				\item Solids can also be categorized as organic (aka: volatile) or inorganic (aka: fixed)
%				\item Colloidial particles are small sized particles some of which pass through the filter and accounted as part of dissolved solids
%				\item TSS - Total Suspended Solids are the solids that are captured on the filter paper upon filtration of the wastewater sample.  
%				\item Wastewater samples typically analyzed for TSS include:  plant, primary and secondary processes - influent and effluent.  TSS is reported in mg/l
%				\item TS - Total Solids are solids content of sludge.  TS of sludge is established by drying a preweighed quantity of sludge in an oven and is typically reported as \% solids - which is how many parts (by weight) of solids per 100 parts (by weight) of sludge.
%				\item Volatile solids are solids that are removed when the solids are incinerated at 550°C.  The solids that remain after incineration are fixed or non-volatile or inorganic solids.
%			\end{itemize}
%			
%			\begin{snugshade*}
%				\item \noindent\textsc{Typical influent wastewater contains:}
%			\end{snugshade*}
%			\begin{itemize}
%				\item Less than 0.1\% total solids.  Total solids concentration in typical wastewater is about 750mg/l
%				\item The total solids are 50\% organic (volatile) and 50\% inorganic (fixed)
%				\item Of the total solids, dissolved solids constitute about 70\% of the solids and the remaining 30\% solids are suspended solids
%				\item 40\% of the dissolved solids are volatile the remaining 60\% are fixed
%				\item 70\% of the suspended solids are volatile and the remaining 30\% are fixed
%			\end{itemize}
%			\newpage
%			\begin{sidewaysfigure}
%				\begin{center}
%					\includegraphics[scale=1.1]{WastewaterSolids}\\
%					Typical Wastewater Solids Concentrations
%				\end{center}
%			\end{sidewaysfigure}
%			\pagebreak
%			\end{enumerate}
%				
%			
%			\begin{snugshade*}
%				\item \noindent\textsc{Nutrients}
%			\end{snugshade*}	
%			\begin{itemize}
%				\item Plant nutrients - nitrogen and phosphorous, present in wastewater effluent discharge, promote growth of plant and algal matter in the receiving waters causing destruction of the normal aquatic life mainly due to oxygen depletion - eutrophication.
%				      
%				\item Because of the potential impacts of the presence of these nutrients in wastewater effluent on the receiving waters,  limits on the levels of these nutrients is typically stipulated in the treatment plant's wastewater discharge permit.
%				      
%				\item Typically, conventional secondary treatment processes are designed primarily remove the organics from the wastewater.  Secondary treatment process designed to additionally remove nutrients is deemed as tertiary or advanced treatment is termed as Biological Nutrient Removal (BNR).
%			\end{itemize}
%			
%			\begin{enumerate}%@@@@@@@@@@@@@@@@@@%
%				\definecolor{shadecolor}{RGB}{220,220,220}
%				\begin{snugshade*}
%					\item \noindent\textsc{Nitrogen}%@@@@@@@@@@@@@@@@@@%
%				\end{snugshade*}
%				\begin{itemize}
%					\item Forms of nitrogen:\\
%					      \begin{itemize}
%					      	\item About 60\% of nitrogen in wastewater is present as ammonia nitrogen (about 60\%).  The ammonium nitrogen is present either in the form of ammonia (NH$_3$ ) or as ammonium (NH$_4^+$ ) ion.   These two forms can rapidly change from one to the other depending on pH and temperature.  Under low pH (acidic) or neutral conditions – pH less than or equal to 7, ammonia exists mostly as ammonium.  Ammonia becomes the dominant form as the pH increases to 8 and beyond.
%					      	\item The other dominant form of nitrogen, about 40\% of the total nitrogen is as organic nitrogen
%					      	\item Nitrogen measured as Total Kjeldahl Nitrogen (TKN) which is the sum of the organic nitrogen and the ammonia nitrogen concentrations.  Total inorganic nitrogen is the total concentration of ammonia nitrogen, NO3-, and NO2-.   Table provides the concentrations and forms of nitrogen in wastewater.
%					      \end{itemize}
%					      \setlength{\arrayrulewidth}{0.7mm}
%					      \setlength{\tabcolsep}{8 pt}
%					      \renewcommand{\arraystretch}{0.8}
%					      \begin{center}
%					      	\noindent \begin{tabular}{ |p{6cm}|p{2.0cm}|p{2.5cm}|p{2.cm}|}
%					      	\hline
%					      	\multicolumn{4}{|c|}{\textbf{Forms of Nitrogen in Wastewater}} \\
%					      	\hline
%					      	%\thead{A Head} & \thead{A Second \\ Head} & \thead{A Third \\ Head} \\
%					      	%\hline%
%					      	
%					      	\hspace{1.8 cm}Forms of Nitrogen & \hspace{0.25 cm} Formula & \hspace{.4 cm} Found in & \hspace{.4 cm} Typical \newline \hspace{.2 cm}Concentration\\
%					      	\hline
%					      	\small Ammonia/Ammonium & \small NH$_3$/NH$_4^{\enspace +}$ &  \small Influent wastewater & 30-50 mg/l\\
%					      	
%					      	Total Kjeldahl Nitrogen \newline  \small (Ammonia/Ammonium + Organic Nitrogen) &  \small TKN &  \small Wastewater \newline  \small effluent  & 30-60 mg/l \\
%					      	
%					      	\small Total Inorganic Nitrogen \newline  \small (Ammonia/Ammonium + Nitrite + Nitrate) & \small TIN &  \small  Wastewater \newline  \small effluent  & 1-40 mg/l \\
%					      	
%					      	\small Nitrate  & $NO_3^{\enspace -}$ &  \small Nitrified effluent &  \small 1-35 mg/l \\
%					      	
%					      	\small Nitrate  &  $NO_2^{\enspace -}$ &  \small Partially nitrified effluent &  \small 0.1-2 mg/l \\
%					      	
%					      	\hline
%					      	\end{tabular}
%					      	
%					      \end{center}
%					      
%					      The nitrogen cycle which shows how nitrogen is converted into its various chemical forms is shown below.
%					      
%					\item Nitrogen removal as part of the secondary treatment mainly involves nitrification-denitrification processes which involve conversion of ammonia nitrogen to nitrogen gas.
%					      
%				\end{itemize}
%				\begin{snugshade*}
%					\item \noindent\textsc{Phosphorous}
%				\end{snugshade*}
%				\begin{itemize}
%					
%		
%		
%		
%	
%	
%	
%	%
%	%\begin{enumerate}
%	%	\definecolor{shadecolor}{RGB}{179, 203, 203}
%	%\begin{snugshade*}
%	%\item \noindent\textsc{Solids Analysis}%$$$$$$$$$$$$$$$$$$$$%
%	%\end{snugshade*}
%	%
%	%\vspace{0.4cm}
%	%\end{enumerate}
%	%\begin{snugshade*}
%	%\item \noindent\textsc{BOD}%$$$$$$$$$$$$$$$$$$$$%
%	%\end{snugshade*}
%	
%	
%\end{enumerate}

