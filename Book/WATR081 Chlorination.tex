% \documentclass{article}
% %\usepackage[english]{babel}%
% \usepackage{graphicx}
% \usepackage{tabulary}
% \usepackage{tabularx}
% \usepackage[normalem]{ulem}
% \usepackage{cancel}
% \usepackage{tikz} 
% \usepackage{pdflscape}
% \usepackage{colortbl}
% \usepackage{lastpage}
% \usepackage{multirow}
% \usepackage{enumerate}
% \usepackage[shortlabels]{enumitem}
% \usepackage{color,soul}
% \usepackage{pdflscape}
% \usepackage{hyperref}
% %\usepackage[table]{xcolor}
% \usepackage{rotating}
% \usepackage{amsmath}
% \usepackage{fixltx2e}
% \usepackage{framed}
% \usepackage{mdframed}
% \usepackage[T1]{fontenc}
% \usepackage[utf8]{inputenc}
% \usepackage{textcomp}
% \usepackage{siunitx}
% \usepackage{ifthen}
% \usepackage{fancyhdr}
% \usepackage{gensymb}
% \usepackage{newunicodechar}
% \usepackage[document]{ragged2e}
% \usepackage[margin=1in,top=1.1in,headheight=57pt,headsep=0.1in]
% {geometry}
% \usepackage{ifthen}
% \usepackage{fancyhdr}
% \everymath{\displaystyle}
% \usepackage[document]{ragged2e}
% \usepackage{fancyhdr}
% \everymath{\displaystyle}
% \usepackage{empheq}

% \usepackage[most]{tcolorbox}

% \usepackage{booktabs} % Required for nicer horizontal rules in tables


% \usepackage{enumitem}

% %\usepackage[table,xcdraw]{xcolor}
% \usetikzlibrary{arrows}
% \linespread{2}%controls the spacing between lines. Bigger fractions means crowded lines%
% %\pagestyle{fancy}
% %\usepackage[margin=1 in, top=1in, includefoot]{geometry}
% %\everymath{\displaystyle}
% \linespread{1.3}%controls the spacing between lines. Bigger fractions means crowded lines%
% %\pagestyle{fancy}
% \pagestyle{fancy}
% \setlength{\headheight}{56.2pt}

% \definecolor{myblue}{rgb}{.8, .8, 1}
% \newcommand*\mybluebox[1]{%
% \colorbox{myblue}{\hspace{1em}#1\hspace{1em}}}

% \chead{\ifthenelse{\value{page}=1}{\includegraphics[scale=0.3]{SCC}\\ \textbf \textbf Wastewater Constituents Analysis \& Laboratory Methods}}
% \rhead{\ifthenelse{\value{page}=1}{}{}}
% \lhead{\ifthenelse{\value{page}=1}{}{Wastewater Constituents Analysis \& Laboratory Methods}}
% \rfoot{\ifthenelse{\value{page}=1}{Module 1: WATR 048 - Spring 2019}{Module 1: WATR 048 - Spring 2019}}

% \lfoot{Shabbir Basrai}
% \cfoot{Page \thepage\ of \pageref{LastPage}}
% \renewcommand{\headrulewidth}{2pt}
% \renewcommand{\footrulewidth}{1pt}
% \begin{document}
% %\begin{empheq}[box=\mybluebox]{align}
% %a&=b\\
% %E&=mc^2 + \int_a^a x\, dx
% %\end{empheq}

% \newlist{steps}{enumerate}{1} % Defines "Steps" for enumerate as Step 1, Step 2 etc.
% \setlist[steps, 1]{label = Step \arabic*:} % Defines "Steps" for enumerate as Step 1, Step 2 etc.

% \setlist{nolistsep} % Reduce spacing between bullet points and numbered lists


%_______________________________________________________________________________________________________________________________________%
\chapterimage{Week6Chlorination.jpg} % Chapter heading image

\chapter{Chlorination}


\begin{itemize}
\item Treated wastewater effluent is disinfected prior to its discharge into a water body inorder to destroy pathogens primarily to prevent spread of waterborne disease and minimize public health problems
\item Chlorine is a very effective disinfectant and is the most widely used disinfectant for wastewater 
\item Chlorine disinfection is a practical and economical means for disinfecting large quantities of wastewaters which have been treated to various degrees. 
\item However, due to its toxicity, associated risk factors and its rising cost, use of ultraviolet light and ozone for wastewater disinfection is on the rise
\end{itemize}


\section{Forms of Chlorine}\index{Forms of Chlorine}

\begin{itemize}
	\item Due to safety issues related to the use of chlorine gas, 			\textbf{hypochlorites} are often used in lieu of chlorine
	\item Types of hypochlorites
	\begin{itemize}
	\item Sodium hypochlorite (NaOCl) comes in a liquid form which contains up to 12.5\% chlorine
	\item Calcium hypochlorite (Ca(OCl)$_2$), also known as HTH, is a solid which is mixed with water to form a hypochlorite solution. Calcium hypochlorite is 65-70\% concentrated.
	\end{itemize}
	\item Hypochlorites decompose in strength over time while in storage. Temperature, light, and physical energy can all break down hypochlorites before they are able to react with pathogens in water. 

\end{itemize} 

\section{Chlorine Properties}\index{Chlorine Properties}

\begin{itemize}
\item Chlorine is a yellowish-green gas at room temperature and atmosphric pressure
\item Chlorine gas can be pressurized and cooled to its liquid form for making it easy to ship and store. 
\item When liquid chlorine is released, it quickly turns into a gas that stays close to the ground (being heavier than air) and spreads rapidly.
	\item 	While it is not explosive or flammable, as a liquid or gas it can react violently with many substances 
	\item Chlorine is only slightly soluble in water (0.3 to 0.7\% by weight.) 
	\item Chlorine gas has a greenish-yellow color 
	\item It has a characteristic disagreeable and pungent odor, similar to chlorine-based laundry bleaches, and is detectable by smell at concentrations as low as 0.2 to 0.4 ppm
	\item It is about two and a half times as heavy as air
	\item One volume of liquid chlorine yields about 460 volumes of chlorine gas. 
	\item Liquid chlorine is amber in color and is about one and a half times as heavy as water 
	\item Chlorine is an irritant to the eyes, skin, mucous membranes, and the respiratory system 
\end{itemize}


\section{Chlorine Storage and Safety}\index{Chlorine Storage and Safety}


\subsection{Chlorine Delivery}\index{Chlorine Delivery}

\begin{itemize}
\item Typically for smaller plants chlorine gas is shipped in  pressurized steel cylinders - 150 lb or 2000 lb (ton cylinder) size
\item Larger plants may get their chlorine supply in rail tank cars
\item The daily chlorine usage is typically established based upon the weighing of the chlorine containers.
\end{itemize}


\subsection{Chlorine Leak Response}\index{Chlorine Leak Response}
\begin{itemize}
	\item Typically for smaller plants chlorine gas is shipped in  pressurized steel cylinders - 150 lb or 2000 lb (ton cylinder) size.  Larger plants may get their chlorine supply in rail tank cars.  
	\item The daily chlorine usage is typically established based upon the weighing of the chlorine containers.
	\item The withdrawal rates from a chlorine cylinder is based on the temperature of the liquid in the cylinder, and thus the pressure of the gas. 
	\item As chlorine gas is withdrawn from the cylinder, it absorbs the heat from the surroundings.
	\item For low withdrawal rates, heat will be able to be transferred from the surrounding air to the container in time so that there is no drop in temperature or pressure, 
	\item If the chlorine withdrawal is larger, the air will not be able to transfer the heat quickly enough and the temperature (and pressure) of the chlorine will drop, thus resulting in a lower feed rate. 
	\item If high enough and prolonged enough, this can even result in ice formation around the outside of the container, further decreasing the withdrawal rate. 
	\item The most effective way to increase withdrawal rate from a single container is to circulate the surrounding air with a fan. Again, never apply heat to the containers.
	\item If chlorine gas escapes from a container or system, being heavier than air, it will seek the lowest level in the building or area
	\item Only trained staff with access to proper personal protection equipment (PPE) including self-contained breathing apparatus, should handle the chlorine cylinders and address chlorine leak issues 
	\item When a leak is suspected, it is recommended that ammonia vapors be used to find the source. When ammonia vapor using a rag or brush, is directed at a leak, a white cloud will form. To produce ammonia vapor, a plastic squeeze bottle containing about 5 \% ammonia, aqua ammonia (ammonium hydroxide solution) should be used. A weaker solution such as household ammonia may not be concentrated enough to detect minor leaks
	\item All safety equipment should be located outside of the chlorine room and be easily accessed by all personnel
	\item Small leaks around valve stems can usually be corrected by tightening the packing nut or closing the valve. A leak can also be reduced by removing the chlorine as rapidly as possible
	\item If it cannot be added to the process there are several chemicals which can be used to absorb the chlorine gas. For example, chlorine can be absorbed by using 1$frac{1}{4}$ pounds of caustic soda or hydrated line, or 3 pounds of soda ash per pound of chlorine. 
	\item If the leaking container can be moved, it should be transported to an outdoors area where minimal harm will occur. Keep the leaking part the most elevated so that gaseous chlorine will leak rather than liquid chlorine.
	\item If the leak is large, all persons in the adjacent area must be warned and evacuated. Only authorized persons equipped with the proper breathing apparatus, and protective measures to the eyes and body should investigate. 
	\item As water is not an efficient absorbent for chlorine and the fact that chlorine reacts with water to form very corrosive hydrochloric acid, never apply water to a leak or consider submerging a chlorine cylinder (for example, in a pond or tank), since it will probably float.
	\item Remember to keep windward of the leak.
	\item As chlorine cylinders pressure increases with temperature, as a safety measure the chlorine cylinders are fitted with fusible plug which melts between 158$^o$ and 165$^o$ F.
	\item Keep chlorine cylinder or container emergency repair kits available. Be familiar with their use and location.
	\item Leaks at fusible plugs and cylinder valves requires special handling and emergency equipment. The chlorine supplier must be notified immediately
	\item Pin hole leaks in cylinder walls or ton tanks can usually be stopped by mechanical pressure applications (clamps, turnbuckles, etc.). This only temporary and may require your ingenuity.
	\item Leaking containers cannot be shipped.
	\item In general, daily inspection of all chlorine cylinders will avoid major problems
\end{itemize}

\subsection{Chlorine Reactions Related to Disinfection}\index{Chlorine Reactions Related to Disinfection}
\textbf{Chlorine reacts with water to form hypochlorous and hydrochloric acids}\\
Cl$_2$ \hspace{0.8cm}	+ \hspace{0.3 cm}	 H$_2$O		\hspace{0.8cm} $\iff$ 
\hspace{0.8cm} HOCl	\hspace{0.8cm}	 +	\hspace{0.8cm}	 HCl \\
chlorine \hspace{0.8cm}	water \hspace{1.8cm}		 hypochlorous acid	\hspace{0.1cm}	 hydrochloric acid\\ 
	\vspace{0.5cm}
	\begin{itemize}
		\item Hypochlorous acid dissociates in water to form the hydrogen and hypochlorite ions\\
 HOCl \hspace{1.8 cm} $\iff$ \hspace{1.8 cm} H$^+$ \hspace{1.8cm} + 	\hspace{0.8cm}OCl$^-$\\ 
hypochlorous acid  \hspace{1.9 cm}      hydrogen ion   \hspace{1.5cm}           hypochlorite ion

		\begin{itemize}
			\item Hypochlorous acid is the most effective form of chlorine available to kill microorganisms
			\item Hypochlorite ions is much less efficient disinfectant
		\end{itemize}

		\item The concentration of hypochlorous acid and hypochlorite ions in chlorinated water will depend on the water's pH
		\begin{itemize}
			\item A higher pH facilitates the formation of more hypochlorite ions and results in less hypochlorous acid in the water
		\end{itemize}
		\item A significant percentage of the chlorine is still in the form of hypochlorous acid even between pH 8 and pH 9
		\end{itemize}



\section{Chlorine Disinfection}\index{Chlorine Disinfection}

\begin{itemize}
\item When chlorine is added to a wastewater flow, it will first react or combine with certain organic and inorganic substances present, prior to acting on pathogens.  The amount of chlorine used up as part of these reactions is referred to as the \textbf{chlorine demand}\\

\item The \textbf{free chlorine} remaining after the chlorine demand is satisfied, is the strongest form of chlorine available for disinfection.  

\item Chlorine combined with ammonia (as chloramines) and organic compounds (as chloroorganic compounds), known as \textbf{combined chlorine} also exhibit disinfecting properties - albeit weaker than the free chlorine.

\item \text{Total residual chlorine} is the sum of free chlorine and combined chlorine and it is the residual chlorine concentration which represents the amount of chlorine available for disinfection 

\item \textbf{Chlorine Demand = Applied Chlorine Dose - Chlorine Residual}\\ Chlorine residual should be the basis of measuring the effectiveness of chlorine disinfection

\item Chlorine residuals are measured in the field using a colorimeteric method.  In the laboratory, chlorine residuals are measured typically using: 1) Amperometric Titration, or 2) Iodometric Titration

\item Chlorine dosage is typically established from either bench scale laboratory testing, or actual measurement of field results. 

\item Since field conditions, particularly the mixing element, are not as well controlled as laboratory tests, the actual dosage is expected to be generally higher than from that established in the laboratory. 

\item Even though residual chlorine concentration can be used for establishing the effectiveness of disinfection, the ultimate effectiveness of disinfection can be monitored by conducting bacteriological testing.

\end{itemize}

\section{Factors Affecting Chlorine Disinfection Efficiency}\index{Factors Affecting Chlorine Disinfection Efficiency}

The disinfection efficiency of chlorine depends on the following factors:\\
\begin{itemize}
	\item pH:  Disinfection is more efficient at a low pH when large quantities of hypochlorous acid are present than at a high pH when hypochlorite ions is the dominant species in the water
	\item Concentration:  Contact Time Ratio (CT):  For effective chlorine disinfection both sufficient chlorine dosages – concentration (C) as well as contact time (T) are necessary.  There may be a substantial residual but if CT factor is not adequate, disinfection may not be effective. Generally both of these factors must be worked out experimentally for a given system
	\item Temperature:  Colder temperatures are less favorable for disinfection. 
Proper contacting or mixing or agitation:  This is necessary to make sure that the chlorine applied contacts or reaches the microbial cells
	\item Organic and inorganic material present:  The chlorine used by these organic and inorganic reducing substances including metal ions, organic matter and ammonia, is defined as the chlorine demand.  So that the amount of chlorine that has to be added to wastewater for different purposes will also vary.
\item Even though residual chlorine concentration can be used for establishing the effectiveness of disinfection, the ultimate effectiveness of disinfection can be monitored by conducting bacteriological testing.
\end{itemize}
		
\section{Dechlorination}\index{Dechlorination}
\begin{itemize}
\item Dechlorination is the process of removing residual chlorine from disinfected wastewater prior to discharge into the environment
\item Dehlorination is necessary to mitigate the toxic effect of chlorine on the receiving waters.  
\item Sulfur dioxide is most commonly used for dechlorination.
\item Other chemicals used for sodium bisulfite, sodium sulfite and sodium thiosulfate.
\end{itemize}

\section{Math Problems}\index{Math Problems}

\subsection{Establishing Chlorine Dosage}\index{Establishing Chlorine Dosage}

\hl{Example Problems:}
\begin{enumerate}
\item Calculate how many pounds per day of chlorine should be used to maintain a dosage of 12 mg/l at a 5.0 MGD flow.\\
Solution:\\
$lbs/day=conc. (mg/l)*flow(MGD)*8.34$\\
$lbs/day=12*5*8.34=\boxed{500.4lbs/day}$\\

\subsection{Calculating Dosage/Residual/Demand Concentrations}\index{Calculating Dosage/Residual/Demand Concentrations}

\hl{Example Problems:}
\item If 80 pounds of chlorine are applied each day to a flow of 1.5 MGD, what is the dosage in mg/l?\\
Solution:\\
Applying the pounds formula:\\  $lbs/day=conc. (mg/l)*flow(MGD)*8.34$\\
$\implies conc. (mg/l)=\dfrac{lbs/day}{flow(MGD)*8.34}=\dfrac{80}{1.5*8.34}=\boxed{6.4mg/l}$

\item How many pounds per day of chlorine will be required to disinfect a secondary effluent flow of 1.68 MGD if the chlorine demand is found to be 8.5 mg/l and a residual of 3 mg/l is desired?
Chlorine dosage = chlorine demand + chlorine residual\\
$chlorine \enspace dosage=8.5+3=11.5mg/l$\\
$lbs/day=conc. (mg/l)*flow(MGD)*8.34=1.68*11.5*8.34=\boxed{161.2lbs/day}$\\
\end{enumerate}

