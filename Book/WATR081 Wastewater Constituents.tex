% \documentclass{article}
% %\usepackage[english]{babel}%
% \usepackage{graphicx}
% \usepackage{tabulary}
% \usepackage{tabularx}
% \usepackage[normalem]{ulem}
% \usepackage{cancel}
% \usepackage{tikz} 
% \usepackage{pdflscape}
% \usepackage{colortbl}
% \usepackage{lastpage}
% \usepackage{multirow}
% \usepackage{enumerate}
% \usepackage[shortlabels]{enumitem}
% \usepackage{color,soul}
% \usepackage{pdflscape}
% \usepackage{hyperref}
% %\usepackage[table]{xcolor}
% \usepackage{rotating}
% \usepackage{amsmath}
% \usepackage{fixltx2e}
% \usepackage{framed}
% \usepackage{mdframed}
% \usepackage[T1]{fontenc}
% \usepackage[utf8]{inputenc}
% \usepackage{textcomp}
% \usepackage{siunitx}
% \usepackage{ifthen}
% \usepackage{fancyhdr}
% \usepackage{gensymb}
% \usepackage{newunicodechar}
% \usepackage[document]{ragged2e}
% \usepackage[margin=1in,top=1.1in,headheight=57pt,headsep=0.1in]
% {geometry}
% \usepackage{ifthen}
% \usepackage{fancyhdr}
% \everymath{\displaystyle}
% \usepackage[document]{ragged2e}
% \usepackage{fancyhdr}
% \everymath{\displaystyle}
% \usepackage{empheq}

% \usepackage[most]{tcolorbox}

% \usepackage{booktabs} % Required for nicer horizontal rules in tables


% \usepackage{enumitem}

% %\usepackage[table,xcdraw]{xcolor}
% \usetikzlibrary{arrows}
% \linespread{2}%controls the spacing between lines. Bigger fractions means crowded lines%
% %\pagestyle{fancy}
% %\usepackage[margin=1 in, top=1in, includefoot]{geometry}
% %\everymath{\displaystyle}
% \linespread{1.3}%controls the spacing between lines. Bigger fractions means crowded lines%
% %\pagestyle{fancy}
% \pagestyle{fancy}
% \setlength{\headheight}{56.2pt}

% \definecolor{myblue}{rgb}{.8, .8, 1}
% \newcommand*\mybluebox[1]{%
% \colorbox{myblue}{\hspace{1em}#1\hspace{1em}}}

% \chead{\ifthenelse{\value{page}=1}{\includegraphics[scale=0.3]{SCC}\\ \textbf \textbf Wastewater Constituents Analysis \& Laboratory Methods}}
% \rhead{\ifthenelse{\value{page}=1}{}{}}
% \lhead{\ifthenelse{\value{page}=1}{}{Wastewater Constituents Analysis \& Laboratory Methods}}
% \rfoot{\ifthenelse{\value{page}=1}{Module 1: WATR 048 - Spring 2019}{Module 1: WATR 048 - Spring 2019}}

% \lfoot{Shabbir Basrai}
% \cfoot{Page \thepage\ of \pageref{LastPage}}
% \renewcommand{\headrulewidth}{2pt}
% \renewcommand{\footrulewidth}{1pt}
% \begin{document}
% %\begin{empheq}[box=\mybluebox]{align}
% %a&=b\\
% %E&=mc^2 + \int_a^a x\, dx
% %\end{empheq}

% \newlist{steps}{enumerate}{1} % Defines "Steps" for enumerate as Step 1, Step 2 etc.
% \setlist[steps, 1]{label = Step \arabic*:} % Defines "Steps" for enumerate as Step 1, Step 2 etc.

% \setlist{nolistsep} % Reduce spacing between bullet points and numbered lists


%_______________________________________________________________________________________________________________________________________%
\chapterimage{ConstituentsImg.png} % Chapter heading image

\chapter{Wastewater Constituents}
% \begin{enumerate}[1.]
% 	\definecolor{shadecolor}{RGB}{200, 200, 240}

% 	%%%%%%%%%%%
% 	% LEVEL 2 %
% 	%%%%%%%%%%%

% 	\begin{snugshade*}
% 		\item \noindent\textsc{Wastewater Constituents}%$$$$$$$$$$$$$$$$$$$$%
% 	\end{snugshade*}
% 	Solids, organic matter, nutrients, pathogens and oil \& grease are the main target constituents of wastewater treatment operations.
% 	\begin{enumerate}[A.]%___________%
% 			\definecolor{shadecolor}{RGB}{225, 235, 235}

				%%%%%%%%%%%
				% LEVEL 3 %
				%%%%%%%%%%%
		% \begin{snugshade*}
		% 	\item \noindent\textsc{Organic Matter}%###############################%
		% \end{snugshade*}

		\section{Organics}\index{Organics}		
		\begin{itemize}
			\item The main reason for treating domestic wastewater is to remove the organic matter.  
			\item Organics are substances containing carbon, hydrogen and oxygen, and some of which may be combined with nitrogen, sulfur or phosphorous.
			\item About 50 percent of the solids present in wastewater are organic.  This fraction is generally of animal or vegetable life, dead animal matter, plant tissue or organisms, and also include synthetic organic compounds.
			\item The principal organic compounds present in domestic wastewater are proteins, carbohydrates and fats together with the products of their decomposition.
			\item Organics are subject to decay or decomposition through the activity of bacteria and other living organisms.  \hl{Since the organic fraction can be driven off at high temperatures, they are also called \textbf{volatile solids}}.\
			\item \emph{Organics in wastewater is typically quantified in terms of oxygen required to oxidize the carbon based material present} in wastewater using the following methods:\\
\subsection{Biochemical Oxygen Demand (BOD)}\index{Biochemical Oxygen Demand (BOD)}

			  %     \begin{enumerate}[i.]
			  %     	\definecolor{shadecolor}{RGB}{220,220,220}
					% %%%%%%%%%%%
					% % LEVEL 4 %
					% %%%%%%%%%%%
			  %     	\begin{snugshade*}
			  %     		\item \noindent\textsc{Biochemical Oxygen Demand (BOD)}%@@@@@@@@@@@@@@@@@@%
			  %     	\end{snugshade*}					
			      	\begin{itemize}
			      		\item Oxygen is required for the consumption of organic matter by aerobic bacteria
			      		\item BOD test measures the depletion of oxygen in a wastewater sample over a five day period
			      		\item BOD measures the organic content in terms of oxygen required for the microorganisms to consume the organic material present

			      		\item BOD is typically measured as BOD$_5$ which is the oxygen demand of the wastewater measured after 5 days of the initiation of the test.
			      		\item The test involves incubating a known dilution of wastewater in a 300 ml bottle for 5 days at 20\si{\degree}C.  The dissolved oxygen (DO) content at the start and end of the incubation period is used for calculating the BOD.
			      		\item For the test to be considered valid, the following criteria need to be met: 1) DO consumption during the test must be at least 2 mg/l, 2) DO remaining at the end of the test must be at least 1 mg/l, and 3) DO consumed in blank should be 0.2 mg/l or less
			      		      			
			      		\item BOD is a parameter to measure the strength of wastewater and the measurement of the wastewater treatment plant or treatment process influent and effluent BOD is standard practice to measure its performance.  Typical domestic wastewater BOD is about 200-250 mg/l.
			      		\item The oxygen consumed by the microorganisms during the BOD test is primarily for: 1) Oxidizing the carbonaceous material (cBOD – carbonaceous BOD), and 2) Oxidizing nitrogenous constituents such as ammonia (nBOD – nitrogenous BOD).
			      		\item Thus, BOD (Total) = cBOD + nBOD.  The cBOD and nBOD is measured by adding certain chemical inhibitors which will inhibit the bacteria responsible for consuming the nitrogenous matter, thus measuring only the cBOD as part of the BOD test.
			      		\item Since not all of the organics is metabolized in the 5 days of the regular BOD test, certain wastewater discharge permits require reporting of the ultimate BOD value (BOD$_U$)\\
			      	\end{itemize}

			    \subsection{Chemical Oxygen Demand (COD)}\index{Chemical Oxygen Demand (COD)}
			      	% \begin{snugshade*}
			      	% 	\item \noindent\textsc{Chemical Oxygen Demand (COD)}%@@@@@@@@@@@@@@@@@@%
			      	% \end{snugshade*}		  
			      	\begin{itemize}
			      		\item The COD test involves using chemical oxidizers to measure the oxygen demand of the wastewater.
			      		\item As the chemical oxidizers will oxidize other constituents present, including inorganic matter, the COD value of wastewater will be higher than the BOD.  
			      		\item The COD test can be conducted rather quickly than the 5 day BOD test, it is an effective method to quantify the wastewater strength and process efficiencies and allow operators to make timely process adjustments.
			      	\end{itemize}

			    \subsection{Total Organic Carbon (TOC)}
			      	% \begin{snugshade*}
			      	% 	\item \noindent\textsc{Total Organic Carbon (TOC):}\\%@@@@@@@@@@@@@@@@@@%
			      	% \end{snugshade*}
			      	The TOC method utilizes laboratory analytical instruments which directly measure the organic carbon content by quantifying the amount of carbon dioxide produced from the complete combustion of the organics present.
			      % \end{enumerate}
		\end{itemize}
		
		
		
			\hl{Note: BOD measures the amount of oxygen required by the microorganisms present to consume the organic material while COD measures the chemical oxidation required to oxidize all chemicals including organics present in wastewater.  BOD value of typical domestic sewage is about 200 - 250 mg/l while the COD value ranges from 300 - 450 mg/l.  Typical BOD:COD ratio ranges from 0.5-0.8.}\\


\section{Solids}\index{Solids}
% 		\pagebreak
% 				\begin{snugshade*}
% 			\item \noindent\textsc{Solids}
% 		\end{snugshade*}	
		Like BOD, wastewater solids is another critical parameter for establishing the wastewater strength and determining treatment process efficiencies. 
		\begin{itemize}
			\item The \texthl{solids can be classified as suspended or dissolved} based upon its ability to pass through a standardized filter paper.
			\item When the wastewater is filtered:
			      \begin{itemize}
			      	\item the residual solids remaining on the filter paper after drying in an oven at 103\si{\degree}C is the \hl{suspended solids} portion, and 
			      	\item the solids remaining after drying the filtrate are the \hl{dissolved solids}.
			      \end{itemize}
			\item Suspended solids include larger floating particles and consist of sand, grit, clay, fecal matter, paper, pieces of wood, particles of food and garbage, and similar materials.
			\item Suspended solids can be categorized based upon its settling characteristics as:
			      \begin{itemize}
			      	\item \hl{Settleable}
			      	\item \hl{Non-settleable}
			      	      \begin{itemize}
			      	      	\item \hl{Colloidial}-small, charged (typically negative) particles which do not settle easily.  Some of the colloidial particles are small enough to pass through the filter paper used for filtering the suspended solids
			      	      	\item \hl{Floatable}-example oil and grease and small plastics
			      	      \end{itemize}
			      \end{itemize}
			\item Dissolved solids in wastewater include organics.  However, the major elements of dissolved solids are inorganic ions such as Ca$^{+2}$, Mg$^{+2}$, Cl$^-$, SO$_4$ $^{-2}$ , HCO$_3$ $^-$, Fe$^{+2}$, PO$_4$ $^{-3}$, NO$_3$ $^-$.  These ions are part of the dissolved salts such as sodium chloride (NaCl), calcium bicarbonate (Ca(HCO$_3$)$_2$), magnesium phosphate (Mg$_3$PO$_4$) and others which are normally present in water and wastewater. 
			      \begin{itemize}
			      	\item Conductivity or electrical conductance (EC) measurement is typically conducted as the wastewater enters the plant as \hl{conductivity provides an indirect and simple measure of the amount of dissolved solids present.}  
			      	\item Conductivity or electrical conductance (EC) is a measure the amount of electrical current that can be conducted by a solution.  
			      	\item The conductance of electricity in a solution is due to the presence of dissolved inorganic ions 
			      	\item The higher the concentration of these ions, the higher is the conductivity. 
			      	\item \underline{Conductivity is measured in the units of mhos/cm or Siemens/cm.}  (Note:  mhos is the reverse of ohm which is a measure of resistance).
			      	\item Typical wastewater conductivities range from 50 to 1500 S/cm
			      \end{itemize}
			\item Both suspended and dissolved solids can be either \hl{volatile (organic)} or \hl{fixed (inorganic)}.
			\item \hl{Total Solids is thus a sum of TSS and dissolved solids or volatile and fixed solids.}
			      \begin{itemize}
			      	\item The volatile solids are typically of plant or animal origin .
			      	\item The fixed solids include sand, gravel and silt as well as the dissolved salts.
			      \end{itemize}
			      \begin{minipage}{0.5\textwidth}
			      	\item The volatile or fixed fractions are quantified by incinerating the solids in a muffler furnace at 550\si{\degree} which removes only the volatile solids leaving only the fixed solids behind.
			      	\item In terms of the size of the solids, the distribution is approximately thirty percent suspended and about seventy percent dissolved solids - which includes the colloidal particles which have passed through the filter paper.\\ 
			      	\item As primary treatment process involve settling of solids, establishing the settleable portion of the suspended solids is important.\\  
			      	\item \hl{The settleable solids are quantified using an Imhoff cone and are reported in ml/L}.  Imhoff cone is a 1 liter, clear cone shaped container, with volume graduations (ml) at the bottom.
			      						
			      \end{minipage}	
			      \begin{minipage}{0.5\textwidth}
			      	\begin{center}
			      		\includegraphics[scale=0.7]{ImhoffCone}\\
			      		Imhoff Cone\\
			      		\textit{Note the ml markings at the bottom of the cone}
			      		
			      		
			      	\end{center}
		      \end{minipage}
%			      \end{minipage}
			      	\item One factor which affects settleability is the conveyance time of the sewage to the treatment plant. 			
			      	\item The settleable component of the suspended solids will decrease as the sewage becomes more septic due to longer conveyance times.
			\item Influent and effluent total suspended solids are measured to establish the overall treatment and individual process efficiencies.  
			\item Volatile solids measurements before and after biological processes such as secondary treatment and digestion provide information on the process efficiency.\\
		\end{itemize}

% 			\end{enumerate}
	\subsection{Summary of Wastewater Solids}\index{Summary of Wastewater Solids}		
% 			\begin{snugshade*}
% 				\item \noindent\textsc{Summary of Wastewater Solids}
% 			\end{snugshade*}
			\begin{itemize}
				\item Solids in wastewater can be categorized as dissolved or suspended
				      \begin{itemize}
				      	\item Suspended solids can be further categorized as settleable or unsettleable
				      \end{itemize}
				\item Solids can also be categorized as organic (aka: volatile) or inorganic (aka: fixed)
				\item Colloidial particles are small sized particles some of which pass through the filter and accounted as part of dissolved solids
				\item TSS - Total Suspended Solids are the solids that are captured on the filter paper upon filtration of the wastewater sample.  
				\item Wastewater samples typically analyzed for TSS include:  plant, primary and secondary processes - influent and effluent.  TSS is reported in mg/l
				\item TS - Total Solids are solids content of sludge.  TS of sludge is established by drying a preweighed quantity of sludge in an oven and is typically reported as \% solids - which is how many parts (by weight) of solids per 100 parts (by weight) of sludge.
				\item Volatile solids are solids that are removed when the solids are incinerated at 550C.  The solids that remain after incineration are fixed or non-volatile or inorganic solids.
			\end{itemize}
	\subsection{Wastewater Solids Content}\index{Wastewater Solids Content}			
% 			\begin{snugshade*}
% 				\item \noindent\textsc{Typical influent wastewater contains:}
% 			\end{snugshade*}
			\begin{itemize}
				\item Less than 0.1\% total solids.  Total solids concentration in typical wastewater is about 750mg/l
				\item The total solids are 50\% organic (volatile) and 50\% inorganic (fixed)
				\item Of the total solids, dissolved solids constitute about 70\% of the solids and the remaining 30\% solids are suspended solids
				\item 40\% of the dissolved solids are volatile the remaining 60\% are fixed
				\item 70\% of the suspended solids are volatile and the remaining 30\% are fixed
			\end{itemize}
			% \clearpage\thispagestyle{empty}
			\begin{figure}[!htbp]
			\vspace{2cm}
				\begin{center}
					\includegraphics[scale=0.8]{WastewaterSolids}\\
					\caption{Typical Wastewater Solids Concentrations}
				\end{center}
				\end{figure}
% % 			\end{enumerate}
				
\section{Nutrients}\index{Nutrients}	
% 			\begin{snugshade*}
% 				\item \noindent\textsc{Nutrients}
% 			\end{snugshade*}	
			\begin{itemize}
				\item Plant nutrients - nitrogen and phosphorous, present in wastewater effluent discharge, promote growth of plant and algal matter in the receiving waters causing destruction of the normal aquatic life mainly due to oxygen depletion - eutrophication.
				      
				\item Because of the potential impacts of the presence of these nutrients in wastewater effluent on the receiving waters,  limits on the levels of these nutrients is typically stipulated in the treatment plant's wastewater discharge permit.
				      
				\item Typically, conventional secondary treatment processes are designed primarily remove the organics from the wastewater.  Secondary treatment process designed to additionally remove nutrients is deemed as tertiary or advanced treatment is termed as Biological Nutrient Removal (BNR).
			\end{itemize}
	\subsection{Nitrogen}\index{Nitrogen}				
% 			\begin{enumerate}%@@@@@@@@@@@@@@@@@@%
% 				\definecolor{shadecolor}{RGB}{220,220,220}
% 				\begin{snugshade*}
% 					\item \noindent\textsc{Nitrogen}%@@@@@@@@@@@@@@@@@@%
% 				\end{snugshade*}

	\textbf{Forms of nitrogen:}\\	
% 				\begin{itemize}
% 					\item Forms of nitrogen:\\
					      \begin{itemize}
					      	\item About 60\% of nitrogen in wastewater is present as ammonia nitrogen (about 60\%).  The ammonium nitrogen is present either in the form of ammonia (NH$_3$ ) or as ammonium (NH$_4^+$ ) ion.   These two forms can rapidly change from one to the other depending on pH and temperature.  Under low pH (acidic) or neutral conditions – pH less than or equal to 7, ammonia exists mostly as ammonium.  Ammonia becomes the dominant form as the pH increases to 8 and beyond.
					      	\item The other dominant form of nitrogen, about 40\% of the total nitrogen is as organic nitrogen
					      	\item Nitrogen measured as Total Kjeldahl Nitrogen (TKN) which is the sum of the organic nitrogen and the ammonia nitrogen concentrations.  Total inorganic nitrogen is the total concentration of ammonia nitrogen, NO3-, and NO2-.   Table provides the concentrations and forms of nitrogen in wastewater.
					      \end{itemize}
					      \setlength{\arrayrulewidth}{0.7mm}
					      \setlength{\tabcolsep}{8 pt}
					      \renewcommand{\arraystretch}{0.8}
					      \begin{center}
					      \begin{figure}[!htbp]
					      	\noindent \begin{tabular}[!htbp]{ |p{6cm}|p{2.0cm}|p{2.5cm}|p{2.cm}|}
					      	\hline
					      	\multicolumn{4}{|c|}{\textbf{Forms of Nitrogen in Wastewater}} \\
					      	\hline
					      	%\thead{A Head} & \thead{A Second \\ Head} & \thead{A Third \\ Head} \\
					      	%\hline%
					      	
					      	\hspace{1.8 cm}Forms of Nitrogen & \hspace{0.25 cm} Formula & \hspace{.4 cm} Found in & \hspace{.4 cm} Typical \newline \hspace{.2 cm}Concentration\\
					      	\hline
					      	\small Ammonia/Ammonium & \small NH$_3$/NH$_4^{\enspace +}$ &  \small Influent wastewater & 30-50 mg/l\\
					      	
					      	Total Kjeldahl Nitrogen \newline  \small (Ammonia/Ammonium + Organic Nitrogen) &  \small TKN &  \small Wastewater \newline  \small effluent  & 30-60 mg/l \\
					      	
					      	\small Total Inorganic Nitrogen \newline  \small (Ammonia/Ammonium + Nitrite + Nitrate) & \small TIN &  \small  Wastewater \newline  \small effluent  & 1-40 mg/l \\
					      	
					      	\small Nitrate  & $NO_3^{\enspace -}$ &  \small Nitrified effluent &  \small 1-35 mg/l \\
					      	
					      	\small Nitrate  &  $NO_2^{\enspace -}$ &  \small Partially nitrified effluent &  \small 0.1-2 mg/l \\
					      	
					      	\hline
					      	\end{tabular}
					      	\caption{Forms of Nitrogen}
					      	\end{figure}
					      \end{center}
					      
		\subsection{Phosphorous}\index{Phosphorous}			
		\textbf{Forms of phosphorous:}\\
					      \begin{itemize}
					      	\item The principal forms are organically bound phosphorus, polyphosphates, and orthophosphates.
					      	\item Organically bound phosphorus originates from body and food waste and, upon biological decomposition of these solids, is converted to orthophosphates. 
					      	\item Polyphosphates originate from synthetic detergents and are hydrolyzed to orthophosphates. Thus, the principal form of phosphorus in wastewater is assumed to be orthophosphates, although the other forms may exist. Orthophosphates consist of the negative ions PO$_4$$^{3-}$, HPO$_4$$^{2-}$, and H$_2$PO$_4$ $^-$.  These may form chemical combinations with cations (positively charged ions).
					      \end{itemize}

\subsection{Oil and Grease}\index{Oil and Grease}	
			Fats, oil and grease in wastewater originate from homes, food establishments and industries.
			\begin{itemize}
				\item Oil and grease content of wastewater is established in the laboratory by extracting it with a solvent - \textit{n}-hexane.  The concentration of oil and grease is reported in mg/l and typical oil and grease content of wastewater ranges from 80 - 120 mg/l
				\item Presence of excessive oils and grease could potentially impact the secondary treatment process
				\item Oils and grease are removed as floatables in primary treatment and sent with the sludge to the digesters
			\end{itemize}
		


