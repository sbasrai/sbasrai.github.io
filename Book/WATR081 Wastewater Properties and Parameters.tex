% \documentclass{article}
% %\usepackage[english]{babel}%
% \usepackage{graphicx}
% \usepackage{tabulary}
% \usepackage{tabularx}
% \usepackage[normalem]{ulem}
% \usepackage{cancel}
% \usepackage{tikz} 
% \usepackage{pdflscape}
% \usepackage{colortbl}
% \usepackage{lastpage}
% \usepackage{multirow}
% \usepackage{enumerate}
% \usepackage[shortlabels]{enumitem}
% \usepackage{color,soul}
% \usepackage{pdflscape}
% \usepackage{hyperref}
% %\usepackage[table]{xcolor}
% \usepackage{rotating}
% \usepackage{amsmath}
% \usepackage{fixltx2e}
% \usepackage{framed}
% \usepackage{mdframed}
% \usepackage[T1]{fontenc}
% \usepackage[utf8]{inputenc}
% \usepackage{textcomp}
% \usepackage{siunitx}
% \usepackage{ifthen}
% \usepackage{fancyhdr}
% \usepackage{gensymb}
% \usepackage{newunicodechar}
% \usepackage[document]{ragged2e}
% \usepackage[margin=1in,top=1.1in,headheight=57pt,headsep=0.1in]
% {geometry}
% \usepackage{ifthen}
% \usepackage{fancyhdr}
% \everymath{\displaystyle}
% \usepackage[document]{ragged2e}
% \usepackage{fancyhdr}
% \everymath{\displaystyle}
% \usepackage{empheq}

% \usepackage[most]{tcolorbox}

% \usepackage{booktabs} % Required for nicer horizontal rules in tables


% \usepackage{enumitem}

% %\usepackage[table,xcdraw]{xcolor}
% \usetikzlibrary{arrows}
% \linespread{2}%controls the spacing between lines. Bigger fractions means crowded lines%
% %\pagestyle{fancy}
% %\usepackage[margin=1 in, top=1in, includefoot]{geometry}
% %\everymath{\displaystyle}
% \linespread{1.3}%controls the spacing between lines. Bigger fractions means crowded lines%
% %\pagestyle{fancy}
% \pagestyle{fancy}
% \setlength{\headheight}{56.2pt}

% \definecolor{myblue}{rgb}{.8, .8, 1}
% \newcommand*\mybluebox[1]{%
% \colorbox{myblue}{\hspace{1em}#1\hspace{1em}}}

% \chead{\ifthenelse{\value{page}=1}{\includegraphics[scale=0.3]{SCC}\\ \textbf \textbf Wastewater Constituents Analysis \& Laboratory Methods}}
% \rhead{\ifthenelse{\value{page}=1}{}{}}
% \lhead{\ifthenelse{\value{page}=1}{}{Wastewater Constituents Analysis \& Laboratory Methods}}
% \rfoot{\ifthenelse{\value{page}=1}{Module 1: WATR 048 - Spring 2019}{Module 1: WATR 048 - Spring 2019}}

% \lfoot{Shabbir Basrai}
% \cfoot{Page \thepage\ of \pageref{LastPage}}
% \renewcommand{\headrulewidth}{2pt}
% \renewcommand{\footrulewidth}{1pt}
% \begin{document}
% %\begin{empheq}[box=\mybluebox]{align}
% %a&=b\\
% %E&=mc^2 + \int_a^a x\, dx
% %\end{empheq}

% \newlist{steps}{enumerate}{1} % Defines "Steps" for enumerate as Step 1, Step 2 etc.
% \setlist[steps, 1]{label = Step \arabic*:} % Defines "Steps" for enumerate as Step 1, Step 2 etc.

% \setlist{nolistsep} % Reduce spacing between bullet points and numbered lists


%_______________________________________________________________________________________________________________________________________%
\chapterimage{PropertiesandParametersImg.png} % Chapter heading image

\chapter{Wastewater Properties and Parameters}
			
		Laboratory and field tests are conducted to measure parameters which are critical for monitoring and controlling treatment.  The following are the key parameters that are measured.	
			
\subsection{pH}\index{pH}	
			\hl{pH is a measure of the hydrogen ion (H$^+$) content or the acidity or basicity of a solution.}  pH impacts the chemical and micribiological elements of wastewater treatment processes and thus pH measurement and control is critical.
			\begin{itemize}
				\item Pure water dissociates into equal concentration of hydrogen ions and hydroxide ions:\\ 
				      $H_2O \rightarrow H^+ + OH^-$.
				\item The H$^+$ are responsible for acidic properties and the OH$^-$ ions for the basic properties.  
				\item pH is the inverse of H$^+$ concentration; pH increases when the concentration of H$^+$ decreases relative to the concentration of OH-. 
				\item pH scale ranges from 0 – 14. When the concentration of both H$^+$ and OH$^-$ are equal, as in pure water, it is considered neutral and its pH is 7.0.  \item If the pH of a sample solution is below 7.0, the sample is termed acidic and is alkaline or basic if its pH is above 7.0. 
				\item Each change of 1 pH unit represents a 10 fold change in concentration.  For example, a sample with a pH of 2.0 is 1000 times more acidic than a sample with a pH of 5.0. 
				\item pH is measured by an electrode that is sensitive only to H$^+$ or using a pH strip which is essentially an adsorbent paper which is pre-impregnated with chemicals which change color under different H$^+$ concentrations.
				\item Most organisms involved in biological wastewater treatment processes do well within a a narrow range of pH near neutral (pH of 7).			
			\end{itemize}
			
\subsection{Oxidation Reduction Potential (ORP)}\index{Oxidation Reduction Potential (ORP)}			
			\begin{itemize}
				\item ORP measurements are common in wastewater process control for monitoring conditions and process efficiency
				\item ORP is measured in milivolts (mV) using a probe\\
				\item ORP is a measure of the potential of oxidation/reduction – electron transfer, based chemical reactions to occur 
				\item If the measured ORP value (in mV), is positive it indicates an environment where oxidation will occur and if negative, an environment where reduction reactions will occur
				\item Higher positive value indicates a stronger oxidative environment and likewise, a lower (more negative) ORP value indicates a stronger reductive environment 
				\item For example, during chlorine disinfection, which is an oxidation process, the wastewater will exhibit a positive ORP.  Stronger the oxidative power of chlorine, higher will be the wastewater ORP value
				\item All living matter, including microbes depend upon respiration to generate energy and respiration involves series of chemical oxidation-reduction reactions 
				\item Bacteria grow and thrive only in specific chemical - oxidative-reductive environments which support its inbuilt metabolic pathways
				\item Aerobic bacteria need molecular oxygen as the terminal electron acceptor as part of its cellular respiration proces.  Bacteria adapted to exist in an environment where molecular oxygen is not present (anoxic and anaerobic), rely on electron acceptors such as NO$_3^-$ (denitrification), SO$_4^-$ (sulfide formation) and carbon (methane formers in anaerobic digestion)
				\item Aerobic bacteria responsible for cBOD removal in the secondary treatment process would be inhibited or wiped-out if the wastewater oxidation potential dropped and become reductive.  Likewise, if the wastewater in the sewer pipes which is normally of reductive (negative ORP)  was to become oxidative because of aeration (dissolving oxygen) it would cease the hydrogen slufide activity of the anaerobic bacteria
				\end{itemize}
		
			\setlength{\arrayrulewidth}{0.6mm}
			\setlength{\tabcolsep}{8 pt}
			\renewcommand{\arraystretch}{1.2}
			\begin{center}
			\begin{table}[!htbp]
				\begin{tabular}{ |p{9.5cm}|p{4.0cm}|}
					\hline
					\multicolumn{2}{|c|}{\textbf{Typical Wastewater Process ORPs}} \\
					\hline
					
					\hline
					\small Clorine disinfection & \small +650 to +700 mV  \\
					\small Nitrification & \small +100 to +350 mV   \\
					\small Biological phosphorous removal & \small +20 to +250mV        \\
					\small Activated sludge	cBOD degradation with free molecular oxygen & \small +50 to +250 mV  \\
					\small Denitrification                                              & \small +50 to -50 mV   \\
					\small Influent wastewater                                          & \small - 200 mV  \\ 
					\small Sulfide formation                                & \small -50 to -250 mV  \\
					\small Anaerobic Digestion: Acid formation (Acidogenesis)           & \small -100 to -225 mV \\
					\small Biological phosphorous removal & \small -100 to -250 mV \\
					\small Anaerobic Digestion: Methane production  (Methanogenesis)     & \small -75 to -400 mV  \\
					\hline
				\end{tabular}
	\end{table}			
			\end{center}
			
\subsection{Alkalinity}\index{Alkalinity}	
			\begin{itemize}
				\item \hl{Alkalinity is the ability of a water to neutralize acids.}  
				\item During certain wastewater treatment processes including anaerobic digestion, acids are generated as a result of microbiological activity.  The bacteria and other biological entities which play an active role in wastewater treatment are most effective at a neutral to slightly alkaline pH of 7 to 8.  In order to maintain these optimal pH conditions for biological activity there must be sufficient alkalinity present in the wastewater to neutralize acids generated by the active biomass.
				\item This ability to maintain the proper pH in the wastewater as it undergoes treatment is the reason why alkalinity is so important to the wastewater industry.
				\item The alkalinity is due to the presence of acid neutralizing bases in the water including the hydroxyl (OH$^-$), carbonate (CO$_3$$^-$) and bicarbonate (HCO$_3$$^-$)  ions.  These ions are of mineral origin and are also formed from carbon dioxide which comes from the atmosphere and from the microbial decomposition of organic material.  The resistance to pH change of the water will continue until all the alkalinity contributing ions are neutralized.  
				\item The pH of a water serves as a guide to the types of alkalinity present in the water but is unrelated to the alkalinity content of a water.  Important Note:  Alkalinity is a measure of the ability to neutralize acids whereas a solution is termed alkaline (or basic) if its pH greater than 7. 
				\item Alkalinity is expressed as milligrams per liter of CaCO$_3$
			\end{itemize}
			
\subsection{Dissolved Oxygen}\index{Dissolved Oxygen}	
			\begin{itemize}
				\item Dissolved oxygen (DO) is the concentration of oxygen dissolved in the wastewater sample and is typically measured in the field using a DO probe.  A titration based Winkler Test is used in the laboratory
				\item The \hl{presence of oxygen indicates an aerobic environment} where dissolved, free oxygen is available for aerobic microorganisms to live, BOD removal in the activated sludge process occurs as a result of the activity of aerobic bacteria.  The absence of DO indicates that the environment or condition is either anoxic or anaerobic.  
				\item \hl{In an anoxic environment, free oxygen is not present, but oxygen is available from its combined  forms - nitrate (NO$_3$ $^-$) and sulfate (SO$_4$ $^-$)} for the the consumption of microorganisms.  Example of an anoxic process is denitrification.  In denitrification, the anoxic bacteria in the presence of food (cBOD) consume the combined oxygen in nitrates (NO$_3$ $^-$ ) and convert it to nitrogen gas.
				\item \hl{The complete absence of oxygen including free and combined oxygen is an anaerobic environment.}
				\item Microorganisms are termed as obligate aerobes if they cannot survive without free oxygen.  Facultative aerobes are microorganisms which can survive in both aerobic and anaerobic environments.  
			\end{itemize}
			
\subsection{Microbiological testing and monitoring}\index{Microbiological testing and monitoring}	
			
			Microbes play a critical role in wastewater treatment.  
			\begin{itemize}
				\item Heterotrohic (organisms that consume organic material) microbes are responsible for the biological wastewater treatment processes - secondary treatment process, digestion and nutrient removal; and
				\item Pathogens - agents that cause disease are present in wastewater effluent.
			\end{itemize}
Microbiological testing and monitoring is conducted as part of the wastewater treatment typically for the following:
\begin{enumerate}[1.]
				
				\item Microbiological testing related to monitoring and troubleshooting biological wastewater treatment\\
				
				Microbes involved in biological wastewater treatment processes include:\\
				\begin{itemize}
					\item Fungi - Filamentous fungi occasionally bloom in activated sludge processes due to low pH or nutrient deficiency and cause problems with the settleability.
					\item Protozoa - Protozoas play a important role in the secondary treatment process.  Common protozoas in the activated sludge process include:
					      \begin{itemize}
					      	\item Amoeba
					      	\item Flagellate
					      	\item Cilliate
					      \end{itemize}
					\item Rotifers
					\item Nematodes
					\item Bacteria - Bacteria is the predominant microorganism responsible for the biological wastewater water treatment.  
				\end{itemize}
				\begin{itemize}
					\item The effectiveness of the biological wastewater treatment processes is primarily due to the presence of a microbial ecosystem with a right balance of populations of different microbial species.
					\item Methods used for monitoring the microbial composition include direct monitoring using a light microscope to see which and how many of the different microbial species are present - typically used for activated sludge process.
					\item Indirect method includes monitoring other parameters such as pH and alkalinity which are influenced by microbiological activity.
					\item The microbial monitoring ensures process stability and helps identify potential process upset conditions caused by changes to the microbial population due to other external factors - toxicty, organic loading, temperature etc.
				\end{itemize}

\item Microbiological testing related to monitoring and controlling pathogens in treated wastewater effluent\\

	
				Pathogens in wastewater belong to the following groups:
				\begin{itemize}
					\item Bacteria:  Although, bacteria is present in large numbers in feces, pathogenic or bacteria are present only because of an infection and this pathogenic bacteria can potentially spread the infection to other healthy individuals.  Disease spread by pathogenic bacteria include diarrhea, cholera and typhoid among many others.
					      
					\item Viruses: A large number of viruses may infect humans and are present in feces.  These include enteroviruses (including polioviruses), hepatitis A virus, reoviruses and diarrhea-causing viruses (especially rotavirus).
					      
					\item Protozoa:  Many species of protozoa can infect humans and cause diarrhoea and dysentery. Girardia which casues diarrheal illness is an example of a protozoan pathogen
					      
					\item Helminths:  These are parasitic worms that can infect humans and are transmitted to others through its eggs or larval forms
					      
				\end{itemize}
				
				\begin{itemize}
					\item As one of the main reasons for treating wastewater is to protect public health, microbiological/pathogen testing of the wastewater effluent and the surface water impacted by the wastewater discharge is conducted to meet the requirements of a wastewater discharge permit, to monitor the pathogen impact of treated wastewater discharge and assess the level of contamination of a public body of water.
					\item The bacteriological tests involves detection and quantification of one or more of the following bacteria:  total coliforms, fecal coliforms, E. Coli, and Enterococcus.  
					      \begin{itemize}
					      	\item The main reason why these bacteria such as coliforms and enterococcus are used \hl{as it is not practical to detect and quantify all pathogens associated with wastewater.}  
					      	\item These selected bacteria originate from feces and indicate fecal contamination and thus serve as an indicator organisms for pathogens of wastewater origin.  
					      	\item Also, they are abundant, potentially less harmful, and easy to detect.  E. coli has been shown to be a better predictor of the potential for impacts to human health and therefore many newer wastewater discharge permits require E. Coli testing in lieu of fecal coliform testing requirements.
					      \end{itemize}
					\item The microbiological test sample is always collected as a grab in a clean, sterile borosilicate glass or plastic bottle containing sodium thiosulfate. 
					      \begin{itemize}
					      	\item Sodium thiosulfate is added to remove residual chlorine which will kill coliforms during transit. 
					      	\item If the sample is not preserved or maintained under proper conditions until the test is conducted in the laboratory, the test would provide erroneous results.
					      	\item Samples must be refrigerated if they cannot be analyzed within 1 hour of collection and must be handled with care to prevent contamination and adverse conditions such as prolonged exposure to direct sunlight.
					      	\item The maximum holding time for state or federal permit reporting purposes is 6 hours. 
					      \end{itemize} 
					\item As it is not possible to exactly quantify the number of bacteria present, a statistical based - \hl{Most Probable Number (MPN)} approach is utilized.  The methods for wastewater bacteriological tests include:  multiple-tube fermentation technique, membrane filtration and quanti-tray testing. 
				\end{itemize}
			\end{enumerate}

\subsection{Specific Gravity}\index{Specific Gravity}				
			\begin{itemize}
				\item Specific gravity is a term to express the weight of a solution with respect to that of water
				\item Water weighs 1 kg/L or 8.34 lbs/gallon or 62.4 lbs/ft$^3$
				\item A solution with a specific gravity of 1.2 will weigh 1.2 times the same volume of water.  1 L of that solution will weigh ( 1.2 kg )/L  or  ( 1.2*8.34=10lbs )/gallon.
				\item Typically wastewater and the associated unthickened sludge, for all practical purposes is assumed to have a specific gravity of 1 - implying 8.34 lbs/gallon.
				\item Specific gravity is typically used for calculations related to chemicals used in wastewater treatment.
			\end{itemize}
