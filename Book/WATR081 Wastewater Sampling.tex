% \documentclass{article}
% %\usepackage[english]{babel}%
% \usepackage{graphicx}
% \usepackage{tabulary}
% \usepackage{tabularx}
% \usepackage[normalem]{ulem}
% \usepackage{cancel}
% \usepackage{tikz} 
% \usepackage{pdflscape}
% \usepackage{colortbl}
% \usepackage{lastpage}
% \usepackage{multirow}
% \usepackage{enumerate}
% \usepackage[shortlabels]{enumitem}
% \usepackage{color,soul}
% \usepackage{pdflscape}
% \usepackage{hyperref}
% %\usepackage[table]{xcolor}
% \usepackage{rotating}
% \usepackage{amsmath}
% \usepackage{fixltx2e}
% \usepackage{framed}
% \usepackage{mdframed}
% \usepackage[T1]{fontenc}
% \usepackage[utf8]{inputenc}
% \usepackage{textcomp}
% \usepackage{siunitx}
% \usepackage{ifthen}
% \usepackage{fancyhdr}
% \usepackage{gensymb}
% \usepackage{newunicodechar}
% \usepackage[document]{ragged2e}
% \usepackage[margin=1in,top=1.1in,headheight=57pt,headsep=0.1in]
% {geometry}
% \usepackage{ifthen}
% \usepackage{fancyhdr}
% \everymath{\displaystyle}
% \usepackage[document]{ragged2e}
% \usepackage{fancyhdr}
% \everymath{\displaystyle}
% \usepackage{empheq}

% \usepackage[most]{tcolorbox}

% \usepackage{booktabs} % Required for nicer horizontal rules in tables


% \usepackage{enumitem}

% %\usepackage[table,xcdraw]{xcolor}
% \usetikzlibrary{arrows}
% \linespread{2}%controls the spacing between lines. Bigger fractions means crowded lines%
% %\pagestyle{fancy}
% %\usepackage[margin=1 in, top=1in, includefoot]{geometry}
% %\everymath{\displaystyle}
% \linespread{1.3}%controls the spacing between lines. Bigger fractions means crowded lines%
% %\pagestyle{fancy}
% \pagestyle{fancy}
% \setlength{\headheight}{56.2pt}

% \definecolor{myblue}{rgb}{.8, .8, 1}
% \newcommand*\mybluebox[1]{%
% \colorbox{myblue}{\hspace{1em}#1\hspace{1em}}}

% \chead{\ifthenelse{\value{page}=1}{\includegraphics[scale=0.3]{SCC}\\ \textbf \textbf Wastewater Constituents Analysis \& Laboratory Methods}}
% \rhead{\ifthenelse{\value{page}=1}{}{}}
% \lhead{\ifthenelse{\value{page}=1}{}{Wastewater Constituents Analysis \& Laboratory Methods}}
% \rfoot{\ifthenelse{\value{page}=1}{Module 1: WATR 048 - Spring 2019}{Module 1: WATR 048 - Spring 2019}}

% \lfoot{Shabbir Basrai}
% \cfoot{Page \thepage\ of \pageref{LastPage}}
% \renewcommand{\headrulewidth}{2pt}
% \renewcommand{\footrulewidth}{1pt}
% \begin{document}
% %\begin{empheq}[box=\mybluebox]{align}
% %a&=b\\
% %E&=mc^2 + \int_a^a x\, dx
% %\end{empheq}

% \newlist{steps}{enumerate}{1} % Defines "Steps" for enumerate as Step 1, Step 2 etc.
% \setlist[steps, 1]{label = Step \arabic*:} % Defines "Steps" for enumerate as Step 1, Step 2 etc.

% \setlist{nolistsep} % Reduce spacing between bullet points and numbered lists


%_______________________________________________________________________________________________________________________________________%
\chapterimage{SamplingCoverBW.png}
\chapter{Wastewater Sampling}		
		\begin{itemize}
			\item Field or laboratory measurement of a certain parameter is critical in wastewater treatment operations to obtain information about wastewater characteristics in order to either characterize a wastewater stream, or to monitor a treatment process or for permit compliance.  
			\item A sample is a small part of the whole representing the whole.  Thus, a sample needs to be such that it truly represents the entire population – which in a wastewater operations could be either a wastewater stream, wastewater solids or a chemical used.
		\end{itemize}
		
\section{Sampling Methods}\index{Sampling Methods}
\subsection{Grab Samples}\index{Grab Samples}
				\begin{itemize}
					\item A grab sample is a sample collected at a specific spot at a site over a short period of time.  
					\item Grab sampling allows for instantaneous analysis of parameters such as pH, dissolved oxygen, chlorine residual, temperature and other parameters which change rapidly with time.
					\item A grab sample represents a snapshot of space and time of a process stream.
					\end{itemize}
\subsection{Composite Samples}\index{Composite Samples}
				\begin{itemize}
					\item A composite sample is a collection of discrete samples are combined over a certain period or space and therefore represent the average performance of a wastewater treatment plant or a process during the collection period.\\  
					\item Composite sampling can be either based on:
					      
					      1. constant time interval (time proportioned sampling)\\
					      2. constant wastewater volume interval (flow-proportioned sampling), and\\
					      3. treatment process space - includes samples taken at different depths\\
					      
					\item Composite samples are typically collected using automated samplers which can be programmed to collect samples at pre-established time intervals – for time proportional sampling.
					\item Time and space composite samples are collected by adding equal volumes of samples collected from different times or locations.  
					\item Flow proportional composite samples comprise of volume of each subsample based on flow.\\  
				\end{itemize}
				
			\begin{center}
				\includegraphics[scale=0.2]{Autosampler} \hspace{2cm} \includegraphics[scale=0.37]{Grabsampler}\\
			\end{center}
			\hspace{2.3cm} Automated Sampler \hspace{2.0cm} \parbox{\textwidth}{Grab Sampling Using a Long Handle Dipper}\\

\subsection{Sampling Precautions and Protocols}\index{Sampling Precautions and Protocols}
			\begin{itemize}
				\item Samples should represent the major portion of the process or the process stream and should be taken from places where the mixing is thorough, avoiding dead spots and areas of heavier or lighter loadings. 
				\item The collected sample is invariably exposed to conditions very different from the original source and is subject to change due to chemical and microbiological activity.  
				\item Thus, in order to ensure integrity of sample, sample preservation techniques specific to the analysis to be performed is needed.  
				      \begin{itemize}
				      	\item The preservation technique should not only allow for stabilizing the parameter to be analyzed, it should also not interfere with the analyses.  
				      	\item The common preservation techniques involve use of proper containers, temperature control, addition of chemical preservatives, and observance of the recommended maximum sample holding time.
				      \end{itemize}
			\end{itemize}
			
\subsection{Bacteriological Sampling}\index{Bacteriological Sampling}
\begin{itemize}
\item Always collected as a grab
\item A clean, sterile borosilicate glass or plastic bottle containing sodium thiosulfate is used. Sodium thiosulfate is added to remove residual chlorine which will kill coliforms during transit. If the sample is not preserved or maintained under proper conditions until the test is conducted in the laboratory, the test would provide erroneous results
\item Samples must be refrigerated if they cannot be analyzed within 1 hour of collection
\item Samples must be handled with care to prevent contamination and adverse conditions such as prolonged exposure to direct sunlight
\item Maximum holding time for state or federal permit reporting purposes is 6 hours
\end{itemize} 

\section{Data Reporting}\index{Data Reporting}	
		\begin{itemize}
			\item Arithmetic mean is typically calculated for reporting data where multiple samples have been collected and analyzed for the same process stream at different times and for reporting average value over a certain time period – daily, monthly etc.\\ \item Arithmetic mean mathematically is calculated by adding all the result values and dividing by the total number of data points.\\
		\end{itemize}
		Mathematically the arithmetic mean is represented as:\\
		$$\bar{x}=\frac{\sum_{i=1}^{n} x^i}{n} = \frac{x_1+x_2+x_3...x_n}{n}$$
		For example:\\
		Arithmetic mean of the following set of data points:  200, 304, 250, 400 is calculated as:\\
		\vspace{10pt}
		Arithmetic Mean = $\frac{200 + 302 + 250 + 400}{4}= 288$\\
		\vspace{10pt}
		For data sets for analysis such as fecal coliform could include values which vary by several orders of magnitudes, using the arithmetic mean to report the average value is not appropriate as the lower or higher values would bias the calculated mean.\\
		\vspace{10pt}
		For example, consider a data set with values:  260, 300, 500, 5,000, 320 and 200.\\
		\vspace{10pt}
		The arithmetic mean = $\frac{260+300+500+5,000+320+200}{6} = 3,444$\\
		Here the 5000 value completely skews the arithmetic mean.
		
		Therefore, for such tests, the geometric mean calculation is used for reporting the average value.\\
		
		
		Mathematically a geometric mean is represented as:\\
		$$\Bigg(\prod_{i=a}^n\Bigg)^{\frac{1}{n}}=\sqrt[n]{a_1*a_2*a_3...a_n}$$
		 
		Calculation method:\\
		1.	Find the product of all the data points (analogous to first calculating the sum of all the data points when calculating the arithmetic mean)\\
		260*300*500*5,000*320*200 = 12,480,000,000,000,000\\
		2.	Raise the product to the inverse of the number of data points\\
		(*Using the power function of a scientific calculator)\\
		Here n (\# data points) = 6 $\implies$ geometric mean = $(12,480,000,000,000,000)^{\frac{1}{6}}   = 482$


