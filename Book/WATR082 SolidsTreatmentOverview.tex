\chapterimage{SolidsTreatmentOverviewBW.jpg}
\chapter{Solids Treatment Overview}
\section{Why do we need to treat wastewater solids?}\index{Why do we need to treat wastewater solids?}

\begin{itemize}
\item Sludge is generated from the wastewater treatment processes -  settled solids and scum from primary and secondary treatment processes
\item This sludge contain organic compounds and also elements that are beneficial plant nutrients
\item However, the organic solids in the sludge are not stable (i.e. they will decay) and include pathogens.  \item Prior to disposal, sludge has to be treated – stabilized, so that its disposal or reuse does not pose a threat to public health.
\item Sludge treatment is very critical as it is an expensive process and sludge disposal is subject to strict regulatory requirement.
\item Even solids are only a small component of wastewater, the solids treatment and disposal account for a very substantial portion of wastewater treatment costs.  Typically 40 to 60\% of total wastewater treatment operations cost is attributable to sludge treatment and disposal.
\end{itemize}

\textbf{NOTE: Solids removed during Preliminary Treatment, from barscreens and grit chambers are typically not treated as part of the solids treatment process.  These solids are disposed off at a landfill}

\vspace{0.5cm}
\textbf{Typical solids treatment is comprised of the following three sequential steps:
\begin{enumerate}
\item Sludge thickening
\item Sludge stabilization
\item Sludge dewatering
\end{enumerate}}
\vspace{0.5cm}
\section{Sludge thickening}\index{Sludge thickening}
Sludge thickening involves the removal of excess water from the primary and secondary sludge increasing the solids content of the sludge and reducing the volume of sludge to be treated in the sludge stabilization process.
Sludge thickening reduces the volume of sludge that need to be handled in the sludge stabilization step thereby reducing treatment cost.  
\begin{itemize}
\item There is an upper limit of the solids concentration that can be effectively treated (stabilized) as increasing the solids concentration reduces its ability to be mixed and pumped easily.  Typically the sludge thickening process produces sludge with a solids content of less than 10\%.\\
\end{itemize}
Benefits of thickening to the sludge stabilization process include:
\begin{itemize}
\item Improved performance due to a lower volume of sludge
\item Cost savings in the construction of new facilities
\item Reduction in energy requirements as less water has to be heated
\end{itemize}
Typical methods used for sludge thickening include:
\begin{enumerate}
\item Gravity thickener - more suitable for primary sludge
\item Dissolved air floatation thickener - more suitable for lighter, fluffier floc such as the secondary sludge.
\end{enumerate}
\section{Sludge Stabilization}\index{Sludge Stabilization}
\textbf{}\\
Sludge stabilization process produces solids that are deemed safe for eventual disposal.  Federal Part 503 rule establishes requirements for the final use or disposal of sewage sludge.  The solids disposal methods may include: land application, as a crop/vegetation fertilizer, placed on a surface disposal site for final disposal and fired in an incinerator.\\
\textbf{Biosolids is the term used for stabilized sludge which meets regulatory standards for beneficial reuse}\\  

Sludge stabilization process results in the following:
\begin{enumerate}
\item Reduction in amount of solids
\item Pathogen reduction
\item Odor reduction
\item Reduction in vector attraction
\end{enumerate}
The main processes involved in sludge stabilization include:
\begin{itemize}
\item Digestion - Aerobic or anaerobic
\item Lime or alkaline stabilization
\item Composting
\item Long term storage in lagoons
\item Thermal processes
\item Incineration
\end{itemize}

\section{Sludge Dewatering}\index{Sludge Dewatering}
Solids stabilized using digestion process has only a small percentage by weight of solids -less than 5\%.  It therefore becomes necessary to dewater the stabilized sludge prior to hauling off-site for final disposal.  Like thickening, the dewatering process does not treat the sludge.  It increases the solids content to between 15 to 30 percent and the higher solids content of the stabilized sludge makes it easier to handle and reduces costs associated with elements related to accomplishing the end objectives with the sludge – land application, composting, drying, incineration or landfill.\\
Dewatering involves conditioning the sludge with a polymer and subjecting it to a physical process which include:
\begin{enumerate}
\item Belt Filter Press 
\item Centrifuge
\end{enumerate}