% \documentclass{article}
%\usepackage[english]{babel}%
% \usepackage{graphicx}
% \usepackage{tabulary}
% \usepackage{tabularx}
% \usepackage[normalem]{ulem}
% \usepackage{cancel}
% \usepackage{tikz} 
% \usepackage{pdflscape}
% \usepackage{colortbl}
% \usepackage{lastpage}
% \usepackage{multirow}
% \usepackage{enumerate}
% \usepackage{color,soul}
% \usepackage{pdflscape}
% \usepackage{hyperref}
% %\usepackage[table]{xcolor}
% \usepackage{rotating}
% \usepackage{amsmath}
% \usepackage{fixltx2e}
% \usepackage{framed}
% \usepackage{mdframed}
% \usepackage[T1]{fontenc}
% \usepackage[utf8]{inputenc}
% \usepackage{textcomp}
% \usepackage{siunitx}
% \usepackage{ifthen}
% \usepackage{fancyhdr}
% \usepackage{gensymb}
%  \usepackage{newunicodechar}
% \usepackage[document]{ragged2e}
% \usepackage[margin=1in,top=1.2in,headheight=70pt,headsep=0.1in]
% {geometry}
% \usepackage{ifthen}
% \usepackage{fancyhdr}
% \everymath{\displaystyle}
% \usepackage[document]{ragged2e}
% \usepackage{fancyhdr}
% \everymath{\displaystyle}
% %\usepackage[table,xcdraw]{xcolor}
% \usetikzlibrary{calc}
% \usetikzlibrary{arrows}
% \linespread{2}%controls the spacing between lines. Bigger fractions means crowded lines%
% %\pagestyle{fancy}
% %\usepackage[margin=1 in, top=1in, includefoot]{geometry}
% %\everymath{\displaystyle}
% \linespread{1.3}%controls the spacing between lines. Bigger fractions means crowded lines%
% %\pagestyle{fancy}
% \pagestyle{fancy}
% \setlength{\headheight}{56.2pt}


% \chead{\ifthenelse{\value{page}=1}{\includegraphics[scale=0.3]{SCC}\\ \textbf \textbf Wastewater Math:  Unit Conversions, Area \& Volume, Process Removal Efficiency and Pumping}}
% \rhead{\ifthenelse{\value{page}=1}{}{}}
% \lhead{\ifthenelse{\value{page}=1}{}{\textbf Wastewater Math: Unit Conversions, Area \& Volume, Process Removal Efficiency and Pumping}}
% \rfoot{\ifthenelse{\value{page}=1}{Module 1: WATR 048 - Spring 2019}{Module 1: WATR 048 - Spring 2019}}

% \cfoot{Page \thepage\ of \pageref{LastPage}}
% \lfoot{Shabbir Basrai}
% \renewcommand{\headrulewidth}{2pt}
% \renewcommand{\footrulewidth}{1pt}
% \begin{document}
%_______________________________________________________________________________________________________________________________________%

\chapterimage{MathCover.png} % Chapter heading image

\chapter{Wastewater Math}


% \begin{enumerate}
% \definecolor{shadecolor}{RGB}{200, 200, 240}
% \begin{snugshade*}
\section{Units and Unit Conversion}\index{Units and Unit Conversion}
% 	\item \noindent\textsc{Units and Unit Conversion}
% \end{snugshade*}

Common Units:\\

Length:  inches, ft, miles\\

Area:  ft$^2$, acres \\

Volume:  ft$^3$, gallons, acres-ft.\\

Density:  weight per volume, lbs/ft$^3$, lbs/gallon\\

Flow:  ft$^3$/min, MGD, acres-ft/day\\

		


Powers of Ten

\begin{center}
    
   
    \begin{tabular}{ | c | p{4cm} | c |p{8cm}|}
    \hline

%\hline
%\multicolumn{4}{|c|}{\textbf{ESSAYS}} \\
%\hline
%\thead{A Head} & \thead{A Second \\ Head} & \thead{A Third \\ Head} \\
%\hline%

$10^{12}$ & 1,000,000,000,000 & Tera & Like in tera byte drive - trillion\\
\hline 
$10^{9}$ & 1,000,000,000 & Giga & Like in giga byte data - billion\\
\hline
$10^{6}$ & 1,000,000 & Mega & Like in mega bytes or megawatts - million\\
\hline 
$10^{3}$ & 1,000 & Kilo & Like in kilogram \\
\hline 
$10^{0}$ & 1 &  & \\
\hline 
$10^{-3}$ & 0.001 & milli & Like in millimeter - thousandth of a meter\\
\hline 
$10^{-6}$ & 0.000001 & micro & Like in microgram - millionth of a gram \\
\hline 
$10^{-9}$ & 0.000000001 & nano & Like in nanometer - billionth of a meter\\
\hline 


    \end{tabular}
    
    \end{center}


For converting one measurement unit to another.

Step 1:  Make sure the original unit is for the same measurement as the conversion unit.  So if the original unit is for area, say ft$^2$ the conversion unit can be another area unit such as in$^2$ or acre but it cannot be gallons as gallon is a unit of volume.

Step 2: Write down the conversion formula as:

$Quantity \enspace in \enspace converted \enspace unit = Quantity \enspace (\cancel{Original \enspace Unit}) *   Conversion  \enspace Factor \enspace  \frac{Conversion \enspace unit}{\cancel{Original \enspace unit}}$


EXAMPLE PROBLEMS:\\

Problem 1\\
Convert 1000 $ft^3$ to cu. yards\\

$1000 \cancel{ft^3}*\frac{cu.yards}{27\cancel{ft^3}} = 37 cu.yards$

Problem 2\\
Convert 10 gallons/min to $ft^3$/hr\\

$\frac{10 \cancel{gallons}}{\cancel{min}}*  \frac{ft^3}{7.48 \cancel{gallons}}  * \frac{60 \cancel{min}}{hr}   = \frac{80.2ft^3}{hr}$


Problem 3\\
Convert 100,000 $ft^3$ to acre-ft.\\
$100,000 \cancel{ft^3} * \frac{acre-ft}{43,560 \cancel{ft^2-ft}} =  2.3 acre-ft$\\
\textbf{Note:} From the conversion table: acre = 43,560 $ft^2$\\
Thus, acre-ft  = 43,560 $ft^2$-ft\\



% \begin{snugshade*}
% \item \noindent\textsc{Pounds Formula}
% \end{snugshade*}
% Pounds formula is used for:
% \begin{itemize}
% \item Calculating the quantity in pounds of a particular wastewater constituent entering or leaving a wastewater treatment process
% \item Calculating the pounds of chemicals to be added\\
% \end{itemize}
% So if the concentration of a particular constituent (in mg/liter) and the volume or flow of wastewater is given, one can calculate the amount of that constituent in pounds using the following – Pounds Formula:
% $$lbs \enspace \textbf{or} \enspace \frac{lbs}{day}=concentration(\frac{mg}{l})*8.34*volume(MG) \enspace \textbf{or} \enspace flow(\frac{MG}{day}(MGD)$$

% \begin{center}
% \includegraphics[scale=0.5]{PoundsFormula}
% \end{center}

% There are three variables – (lbs, concentration and volume) and one constant (8.34) in the pounds formula.  Knowing any of the two variables in the formula, one can calculate the third (unknown) variable by rearranging the equation.

% NOTE:  Concentration is typically expressed as mg/l which is the weight of the constituent (mg) in 1 l (liter) of solution (wastewater).  As 1 l of water weighs 1 million mg, a concentration of 1 mg/l implies 1 mg of constituent per 1 million mg of water or one part per million (ppm).   \textbf{Thus, mg/l and ppm are synonymous.}\\  
% Sometimes the constituent concentration is expressed in terms of percentage.\\
% \vspace{6pt}
% For example:  sludge containing 5\% solids or a 12.5\% chlorine concentration solution.\\
% \vspace{6pt}
% As one liter of water weighs 1,000,000 mg, one percent of that weight is 10,000 mg.  So 1\% solids implies 10,000 mg of solids per liter or 10,000 mg/l or 10,000 ppm.\\
% \vspace{6pt}
% $1\% concentration = 10,000 \enspace ppm \enspace or \enspace\frac{mg}{l}$\\
% $0.1\% concentration = 1,000 \enspace ppm \enspace or \enspace \frac{mg}{l}$\\
% $0.01\% concentration = 100 \enspace ppm \enspace or \enspace \frac{mg}{l}$\\
% $10\% concentration = 100,000 \enspace ppm \enspace or \enspace \frac{mg}{l}$\\
% $5\% concentration = 50,000 \enspace ppm \enspace or \enspace \frac{mg}{l}$\\
% $12.5\% concentration = 125,000 \enspace ppm \enspace or \enspace \frac{mg}{l}$\\
% \pagebreak
% \hl{Example Problems}\\

% Problem 1\\Calculate the lbs/day of solids entering the plant given the influent flow is 5 MGD with an average solids concentration  of 250 mg/l.\\

% Solution\\

% Applying lbs formula:\\
% $\frac{lbs}{day}=5 MGD *250\frac{mg}{l}*8.34 = \boxed{10,425\frac{lbs}{day}}$
% \\
% \vspace{6pt}
% Problem 2\\Calculate the lbs of solids in the primary sludge if the sludge flow is 7500 gallons and the solids concentration is 4.5\%.\\
% Solution\\
% Applying lbs formula:\\
% $lbs \enspace solids = \frac{7500}{1,000,000}MG * 4.5*10,000 *8.34 = \boxed{2,815 \enspace lbs \enspace solids}$\\
% \textbf{Note:}\\  
% 1) 7500 gallons was converted to MG by dividing by 1,000,000\\
% $7500 \enspace gallons * \frac{1 MG}{1,000,000 \enspace gallon}$\\
% 2) 4.5\% was converted to mg/l by multiplying by 10,000 as 1\%=10,000mg/l

% \pagebreak

% \begin{snugshade*}
% 	\item \noindent\textsc{Area \& Volume}
% \end{snugshade*}

% \begin{center}
% \includegraphics[scale=0.5]{Area&VolumeFormula}
% \end{center}
% \hl{Example Problems}\\
% \begin{enumerate}

% \item The floor of a rectangular building is 20 feet long by 12 feet wide and the inside walls are 10 feet high. Find the total surface area of the inside walls of this building\\
% Solution:\\
% \begin{center}
% \begin{tikzpicture}
% 	%%% Edit the following coordinate to change the shape of your
% 	%%% cuboid
      
% 	%% Vanishing points for perspective handling
% 	\coordinate (P1) at (-7cm,1.5cm); % left vanishing point (To pick)
% 	\coordinate (P2) at (8cm,1.5cm); % right vanishing point (To pick)

% 	%% (A1) and (A2) defines the 2 central points of the cuboid
% 	\coordinate (A1) at (0em,0cm); % central top point (To pick)
% 	\coordinate (A2) at (0em,-2cm); % central bottom point (To pick)

% 	%% (A3) to (A8) are computed given a unique parameter (or 2) .8
% 	% You can vary .8 from 0 to 1 to change perspective on left side
% 	\coordinate (A3) at ($(P1)!.8!(A2)$); % To pick for perspective 
% 	\coordinate (A4) at ($(P1)!.8!(A1)$);

% 	% You can vary .8 from 0 to 1 to change perspective on right side
% 	\coordinate (A7) at ($(P2)!.7!(A2)$);
% 	\coordinate (A8) at ($(P2)!.7!(A1)$);

% 	%% Automatically compute the last 2 points with intersections
% 	\coordinate (A5) at
% 	  (intersection cs: first line={(A8) -- (P1)},
% 			    second line={(A4) -- (P2)});
% 	\coordinate (A6) at
% 	  (intersection cs: first line={(A7) -- (P1)}, 
% 			    second line={(A3) -- (P2)});

% 	%%% Depending of what you want to display, you can comment/edit
% 	%%% the following lines

% 	%% Possibly draw back faces

% 	\fill[gray!40] (A2) -- (A3) -- (A6) -- (A7) -- cycle; % face 6
% 	\node at (barycentric cs:A2=1,A3=1,A6=1,A7=1) {\tiny Floor=W*L};
	
% 	\fill[gray!50] (A3) -- (A4) -- (A5) -- (A6) -- cycle; % face 3
% 	\node at (barycentric cs:A3=1,A4=1,A5=1,A6=1) {\tiny Wall - W*H};
	
% 	\fill[gray!10, opacity=0.2] (A5) -- (A6) -- (A7) -- (A8) -- cycle; % face 4
% 	\node at (barycentric cs:A5=1,A6=1,A7=1,A8=1) {\tiny Wall - L*H};
	
% 	\fill[gray!10,opacity=0.5] (A1) -- (A2) -- (A3) -- (A4) -- cycle; % f2
% 	\node at (barycentric cs:A1=1,A2=1,A3=1,A4=1) {\tiny Wall - L*H};
	
% 	\fill[gray!40,opacity=0.2] (A1) -- (A4) -- (A5) -- (A8) -- cycle; % f5
% 	\node at (barycentric cs:A1=1,A4=1,A5=1,A8=1) {\tiny Ceiling=W*L};	
	
% 	\draw[thick,dashed] (A5) -- (A6);
% 	\draw[thick,dashed] (A3) -- (A6);
% 	\draw[thick,dashed] (A7) -- (A6);

% 	%% Possibly draw front faces

% 	%\fill[orange] (A1) -- (A8) -- (A7) -- (A2) -- cycle; % face 1
% 	\node at (barycentric cs:A1=1,A8=1,A7=1,A2=1) {\tiny Wall - W*H};
	


% 	%% Possibly draw front lines
% 	\draw[thick] (A1) -- (A2);

% 	\draw[<->] (-1.8,0.38) -- (-1.8,-1.3)node [midway, above=-1.8mm] {\hspace{-1.3cm}\tiny Height=10'};
% 	\draw[<->] (-1.6,-1.4) -- (-.3,-2.1)node [midway, above=-2.6mm] {\hspace{-1.3cm}\tiny Length=20'};
% 	\draw[<->] (2.6,-1.13) -- (0.2,-2.2)node [midway, below=.6mm] {\hspace{1.2cm}\tiny Width=12'};
% 	\draw[thick] (A3) -- (A4);
% 	\draw[thick] (A7) -- (A8);
% 	\draw[thick] (A1) -- (A4);
% 	\draw[thick] (A1) -- (A8);
% 	\draw[thick] (A2) -- (A3);
% 	\draw[thick] (A2) -- (A7);
% 	\draw[thick] (A4) -- (A5);
% 	\draw[thick] (A8) -- (A5);
	
% 	% Possibly draw points
% 	% (it can help you understand the cuboid structure)
% %	\foreach \i in {1,2,...,8}
% %	{
% %	  \draw[fill=black] (A\i) circle (0.15em)
% %	    node[above right] {\tiny \i};
% %	}
% 	% \draw[fill=black] (P1) circle (0.1em) node[below] {\tiny p1};
% 	% \draw[fill=black] (P2) circle (0.1em) node[below] {\tiny p2};
% \end{tikzpicture}\\
% \end{center}
% 2 Walls W*H + 2 Walls L*H= $2*12*10ft^2 + 2*20*10ft^2$\\
% $=240+400=\boxed{640ft^2}$\\

% 2 Walls W*H + 2 Walls L*H + Floor + Ceiling= $2*12*10ft^2 + 2*20*10ft^2 + 2*12*20ft^2$\\
% $=240+400+480=\boxed{1,120ft^2}$\\
% \pagebreak
% \item How many gallons of paint will be required to paint the inside walls of a 40 ft long x 65 ft wide x 20 ft high tank if the paint coverage is 150 sq. ft per gallon.  Note:  We are painting walls only.  Disregard the floor and roof areas.\\
% Solution:\\
% \vspace{0.3cm}
% \begin{center}
% \begin{tikzpicture}
% 	%%% Edit the following coordinate to change the shape of your
% 	%%% cuboid
      
% 	%% Vanishing points for perspective handling
% 	\coordinate (P1) at (-7cm,1.5cm); % left vanishing point (To pick)
% 	\coordinate (P2) at (8cm,1.5cm); % right vanishing point (To pick)

% 	%% (A1) and (A2) defines the 2 central points of the cuboid
% 	\coordinate (A1) at (0em,0cm); % central top point (To pick)
% 	\coordinate (A2) at (0em,-2cm); % central bottom point (To pick)

% 	%% (A3) to (A8) are computed given a unique parameter (or 2) .8
% 	% You can vary .8 from 0 to 1 to change perspective on left side
% 	\coordinate (A3) at ($(P1)!.8!(A2)$); % To pick for perspective 
% 	\coordinate (A4) at ($(P1)!.8!(A1)$);

% 	% You can vary .8 from 0 to 1 to change perspective on right side
% 	\coordinate (A7) at ($(P2)!.7!(A2)$);
% 	\coordinate (A8) at ($(P2)!.7!(A1)$);

% 	%% Automatically compute the last 2 points with intersections
% 	\coordinate (A5) at
% 	  (intersection cs: first line={(A8) -- (P1)},
% 			    second line={(A4) -- (P2)});
% 	\coordinate (A6) at
% 	  (intersection cs: first line={(A7) -- (P1)}, 
% 			    second line={(A3) -- (P2)});

% 	%%% Depending of what you want to display, you can comment/edit
% 	%%% the following lines

% 	%% Possibly draw back faces

% 	\fill[gray!40] (A2) -- (A3) -- (A6) -- (A7) -- cycle; % face 6
% 	\node at (barycentric cs:A2=1,A3=1,A6=1,A7=1) {};
	
% 	\fill[gray!50] (A3) -- (A4) -- (A5) -- (A6) -- cycle; % face 3
% 	\node at (barycentric cs:A3=1,A4=1,A5=1,A6=1) {\tiny Wall - W*H};
	
% 	\fill[gray!10, opacity=0.2] (A5) -- (A6) -- (A7) -- (A8) -- cycle; % face 4
% 	\node at (barycentric cs:A5=1,A6=1,A7=1,A8=1) {\tiny Wall - L*H};
	
% 	\fill[gray!10,opacity=0.5] (A1) -- (A2) -- (A3) -- (A4) -- cycle; % f2
% 	\node at (barycentric cs:A1=1,A2=1,A3=1,A4=1) {\tiny Wall - L*H};
	
% 	\fill[gray!40,opacity=0.2] (A1) -- (A4) -- (A5) -- (A8) -- cycle; % f5
% 	\node at (barycentric cs:A1=1,A4=1,A5=1,A8=1) {};	
	
% 	\draw[thick,dashed] (A5) -- (A6);
% 	\draw[thick,dashed] (A3) -- (A6);
% 	\draw[thick,dashed] (A7) -- (A6);

% 	%% Possibly draw front faces

% 	%\fill[orange] (A1) -- (A8) -- (A7) -- (A2) -- cycle; % face 1
% 	\node at (barycentric cs:A1=1,A8=1,A7=1,A2=1) {\tiny Wall - W*H};
	


% 	%% Possibly draw front lines
% 	\draw[thick] (A1) -- (A2);

% 	\draw[<->] (-1.8,0.38) -- (-1.8,-1.3)node [midway, above=-1.8mm] {\hspace{-1.3cm}\tiny Height=20'};
% 	\draw[<->] (-1.6,-1.4) -- (-.3,-2.1)node [midway, above=-2.6mm] {\hspace{-1.3cm}\tiny Length=45'};
% 	\draw[<->] (2.6,-1.13) -- (0.2,-2.2)node [midway, below=.6mm] {\hspace{1.2cm}\tiny Width=65'};
% 	\draw[thick] (A3) -- (A4);
% 	\draw[thick] (A7) -- (A8);
% 	\draw[thick] (A1) -- (A4);
% 	\draw[thick] (A1) -- (A8);
% 	\draw[thick] (A2) -- (A3);
% 	\draw[thick] (A2) -- (A7);
% 	\draw[thick] (A4) -- (A5);
% 	\draw[thick] (A8) -- (A5);
	
% 	% Possibly draw points
% 	% (it can help you understand the cuboid structure)
% %	\foreach \i in {1,2,...,8}
% %	{
% %	  \draw[fill=black] (A\i) circle (0.15em)
% %	    node[above right] {\tiny \i};
% %	}
% 	% \draw[fill=black] (P1) circle (0.1em) node[below] {\tiny p1};
% 	% \draw[fill=black] (P2) circle (0.1em) node[below] {\tiny p2};
% \end{tikzpicture}\\
% \end{center}
% \vspace{0.3cm}
% 2 Walls W*H + 2 Walls L*H = $2*65*20ft^2 + 2*40*20ft^2= 2,600+1,600=4,200ft^2$\\
% $\implies @150\frac{ft^2}{gal} \enspace paint \enspace coverage \enspace \rightarrow \enspace \frac{4,200\cancel{ft^2}}{150\frac{\cancel{ft^2}}{gal}}=\boxed{28 \enspace gallons}$
% \vspace{0.3cm}
% \item What is the circumference of a 100 ft diameter circular clarifier?\\
% \vspace{0.3cm}
% Solution:\\
% \vspace{0.3cm}
% $Circumference=\pi*D=3.14*100ft=\boxed{314ft}$
% \vspace{0.3cm}
% \item If the surface area of a clarifier is 5,025$ft^2$, what is its diameter?\\
% \vspace{0.3cm}
% Solution:\\
% \vspace{0.3cm}
% $Surface \enspace area=\frac{\pi}{4}*D^2 \enspace \implies 5025(ft^2)=0.785*D^2 (ft^2)$\\
% $\implies D^2=\frac{5025}{0.785} \implies D=\sqrt{6401.3}=\boxed{80ft}$
% \vspace{0.3cm}

% \item How many gallons of wastewater would 600 feet of 6-inch diameter pipe hold, approximately?\\
% \vspace{0.3cm}
% Solution:\\

% \vspace{0.3cm}
% \begin{center}
% \begin{tikzpicture}
% \draw (0,0) ellipse (0.1cm and 0.3cm);
% \draw (10,0) ellipse (0.1cm and 0.3cm);
% \draw [-] (0,-0.29) -- (10,-0.29);
% \draw [-] (0,0.29) -- (10,0.29);
% \draw [<->] (10,-0.28) -- (10,0.28) node [midway, below=-3mm] {\hspace{2.6cm}Diameter=6"};
% \draw [<->] (0,-.68) -- (10,-.68)node [midway, below] {\hspace{0.9cm}Length=600'};
% \end{tikzpicture}
% \end{center}
% \vspace{0.3cm}
% $Volume=\frac{\pi}{4}D^2*L=0.785*\Big(\frac{6}{12}\Big)^2*600\cancel{ft^3}*7.48\frac{gallons}{\cancel{ft^3}}=\boxed{881 \enspace gallons}$
% \pagebreak
% \item A 110 ft diameter digester with a 12 ft deep cone is operated at a side water depth of 20 ft.  Caluclate the volume of sludge in the digester in $ft^3$ and gallons.\\
% \vspace{0.3cm}
% Solution:\\
% \vspace{0.3cm}
% \begin{center}
% \begin{tikzpicture}
% \draw (0,0) ellipse (2cm and 0.3cm);
% \draw (0,-2.3) ellipse (2cm and 0.3cm);
% \draw (0,-.8) ellipse (2cm and 0.3cm);
% \draw [-] (2,-2.3) -- (2,0);
% \draw [<->] (-2,0) -- (2,0) node [midway, below=-0.9cm] {\hspace{0.9cm}Diameter (D)=110'}; 
% \draw [<->] (-2.6,-2.3) -- (-2.6,0) node [midway, below=-.3cm] {\hspace{-2.6cm}Cylinder Height};
% \draw [<->] (2.5,-2.3) -- (2.5,-0.8) node [midway, below=-0.2cm] {\hspace{5.2cm}Side Water Depth (SWD) =20'};
% \draw [-] (0,-4) -- (2,-2.3);
% \draw [-] (0,-4) -- (-2,-2.3);
% \draw [-] (0,-4) -- (2,-2.3);
% \draw [-] (-2,0) -- (-2,-2.3);
% \draw [<->] (2.5,-2.3) -- (2.5,-4)node [midway, below=-0.4cm] {\hspace{3.8cm}Cone Depth (CD)=12'};
% \end{tikzpicture}
% \end{center}
% $Digester \enspace volume=Volume_{cylinder}+Volume_{cone}$\\
% $\implies Digester \enspace volume=\frac{\pi}{4}D^2*SWD+\frac{1}{3}*\Bigg(\frac{\pi}{4}*D^2*CD\Bigg)$\\
% \vspace{0.3cm}
% $=0.785*110^2*20+1.05*110^2*12=\boxed{227,988ft^3}$\\
% \vspace{0.3cm}
% $227,988\cancel{ft^3}*7.48\frac{gallons}{\cancel{ft^3}}=\boxed{1,705,352 \enspace gallons}$
% \end{enumerate}
% \pagebreak
% \begin{snugshade*}
% 	\item \noindent\textsc{Process Removal Efficiency}
% \end{snugshade*}
% \begin{itemize}
% \item Process removal rate or removal efficiency is the percentage of the inlet concentration removed.  
% \item It is used for quantifying the pollutant removal during wastewater treatment and is established based upon the amount of a particular wastewater constituent entering and leaving a treatment process.

% \item $Process \enspace Removal \enspace Rate \enspace (\%) = \frac{Pollutant \enspace  In-Pollutant\enspace  Out}{Pollutant \enspace In}*100$\\

% \item If 10 units of a pollutant are entering a process and 8 units of pollutant are leaving (process removes 2 units), then the process removal rate for that pollutant is (10-8)/10*100=20\%.  In this example the process is 20\% efficient in removing that particular pollutant.

% \item The amount of pollutant can be measured in terms of concentration (mg/l) or in terms of mass loading (lbs).  The pounds formula is used for calculating the mass loadings.  
% \end{itemize}
% The above example is for calculating the removal efficiency using the inlet and outlet concentrations or mass loading.\\
% The methods below can be used for calculating either the inlet or outlet pollutant concentrations, if the removal efficiency and the corresponding inlet or outlet concentrations are given. 


% \hl{Case 1:  Calculating outlet conc. (X) given the inlet conc. and removal efficiency (RE\%):}

% \tikzstyle{block} = [rectangle, draw, fill=red!40, 
%     text width=6em, text centered, rounded corners, minimum height=3em]
% \tikzstyle{arrow} = [draw, -latex']
% \begin{figure}[!h]
% \centering
% \begin{tikzpicture}[node distance =1.5cm, auto]
%     \draw ++(0,0) node [block] (Process) {Process};
%    \node[node distance=1.9in] (dummy_in) [left of=Process] {In};
%    \node[node distance=1.9in] (dummy_out) [right of=Process] {Out};
% 	\node (Removal) [below of=Process, yshift=-0in] {$\tiny{Removal \enspace Efficiency=RE\% \enspace (Given)}$};
%     \path [arrow] (dummy_in)-- (Process)  node [above] {\hspace{-5.8cm}$A \enspace mg/l \enspace (Given) $} node [below] {\hspace{-5.8cm}$100 \enspace mg/l$};
%     \path [arrow] (Process) -- (dummy_out)  node [above] {\hspace{-4cm}$X \enspace mg/l \enspace (Unknown)$} node [below] {\hspace{-3.9cm}($100-RE\%)\enspace mg/l$};
%    \draw[arrow] (Process) -- (Removal);
% \end{tikzpicture}
% \end{figure}
% Using the fact that if the inlet concentration was 100 mg/l, the outlet concentration would be 100 minus the removal efficiency.\\
% Setup the equation as:  $\frac{Out}{In}: \enspace \frac{X \enspace mg/l}{A \enspace mg/l}=\frac{100-RE\%}{100}$\\
% Calculate X using cross multiplication - if $\frac{A}{B}=\frac{C}{D} \implies A=B*\frac{C}{D}$:\\
% $X \enspace mg/l=A \enspace mg/l*\frac{100-RE\%}{100}$\\

%  \pagebreak
% \hl{Case 2:  Calculating inlet conc. (X) given the outlet conc. and removal efficiency (RE\%):}

% \begin{figure}[!h]
% \centering
% \begin{tikzpicture}[node distance =1.5cm, auto]
%     \draw ++(0,0) node [block] (Process) {Process};
%    \node[node distance=1.9in] (dummy_in) [left of=Process] {In};
%    \node[node distance=1.9in] (dummy_out) [right of=Process] {Out};
% 	\node (Removal) [below of=Process, yshift=-0in] {$Removal \enspace Efficiency=RE\% \enspace (Given)$};
%     \path [arrow] (dummy_in)-- (Process)  node [above] {\hspace{-5.8cm}$X \enspace mg/l \enspace (Unknown)$} node [below] {\hspace{-5.8cm}$100 \enspace mg/l$};
%     \path [arrow] (Process) -- (dummy_out)  node [above] {\hspace{-4cm}$A \enspace mg/l \enspace (Given)$} node [below] {\hspace{-3.9cm}($100-RE\%)\enspace mg/l$};
%    \draw[arrow] (Process) -- (Removal);
% \end{tikzpicture}
% \end{figure}
% Using the fact that if the inlet concentration was 100 mg/l, the outlet concentration would be 100 minus the removal efficiency.\\
% Setup the equation as:  $\frac{In}{Out}: \enspace \frac{X \enspace mg/l}{A \enspace mg/l}=\frac{100}{100-RE\%}$\\
% \vspace{0.3cm}
% Calculate X using cross multiplication - if $\frac{A}{B}=\frac{C}{D} \implies A=B*\frac{C}{D}$:\\
% $X \enspace mg/l=A \enspace mg/l*\frac{100}{100-RE\%}$\\

% \vspace{0.4cm}
% \hl{Example Problems:}\\

% \begin{enumerate}

% \item What is the \% removal efficiency if the influent concentration is 10 mg/L and the effluent concentration is 2.5 mg/L?\\
% $Removal \enspace Rate (\%) = \frac{In-Out}{In}*100 \implies \frac{10-2.5}{10}*100=\boxed{75\%}$



% \item Calculate the outlet concentration if the inlet concentration is 80 mg/l and the process removal efficiency is 60\%\\
% Solution:\\

% \tikzstyle{block} = [rectangle, draw, fill=red!40, 
%     text width=6em, text centered, rounded corners, minimum height=3em]
% \tikzstyle{arrow} = [draw, -latex']
% \begin{figure}[!h]
% \centering
% \begin{tikzpicture}[node distance =1.5cm, auto]
%     \draw ++(0,0) node [block] (Process) {Process};
%    \node[node distance=1.5in] (dummy_in) [left of=Process] {In};
%    \node[node distance=1.5in] (dummy_out) [right of=Process] {Out};
% 	\node (Removal) [below of=Process, yshift=-0in] {$Removal \enspace Efficiency=60\%$};
%     \path [arrow] (dummy_in)-- (Process)  node [above] {\hspace{-4.39cm}$80mg/l$} node [below] {\hspace{-4.39cm}$100mg/l$};
%     \path [arrow] (Process) -- (dummy_out)  node [above] {\hspace{-3.cm}$Xmg/l$} node [below] {\hspace{-3cm}40mg/l};
%    \draw[arrow] (Process) -- (Removal);
% \end{tikzpicture}
% %\caption[MFCC]{Diagrama en bloques del cálculo de las MFCC para un frame.}
% %\label{MFCC}
% \end{figure}

% $\frac{Out}{In} \enspace:\enspace\frac{Actual \enspace Outlet (X)}{80}=\frac{100-60}{100}$\\
% $\implies \frac{Actual \enspace Outlet (X)}{80} =0.4$\\
% $\implies Actual \enspace  Outlet (X) = 0.4 * 80 = \boxed{32 mg/l}$\\


% \item Calculate the inlet concentration if the outlet concentration is 80 mg/l and the process removal efficiency is 60\%\\

% \tikzstyle{block} = [rectangle, draw, fill=red!40, 
%     text width=6em, text centered, rounded corners, minimum height=3em]
% \tikzstyle{arrow} = [draw, -latex']
% \begin{figure}[!h]
% \centering
% \begin{tikzpicture}[node distance =1.5cm, auto]
%     \draw ++(0,0) node [block] (Process) {Process};
%    \node[node distance=1.5in] (dummy_in) [left of=Process] {In};
%    \node[node distance=1.5in] (dummy_out) [right of=Process] {Out};
% 	\node (Removal) [below of=Process, yshift=-0in] {$Removal \enspace Efficiency=60\%$};
%     \path [arrow] (dummy_in)-- (Process)  node [above] {\hspace{-4.39cm}$Xmg/l$} node [below] {\hspace{-4.39cm}$100mg/l$};
%     \path [arrow] (Process) -- (dummy_out)  node [above] {\hspace{-3.cm}80mg/l} node [below] {\hspace{-3cm}40mg/l};
%    \draw[arrow] (Process) -- (Removal);
% \end{tikzpicture}
% \end{figure}

% $\frac{In}{Out} \enspace : \enspace \frac{Actual \enspace inlet \enspace  (X)}{80}=\frac{100}{100-60}\implies \frac{Actual \enspace inlet \enspace  (X)}{80}=2.5$\\    
% Rearranging the equation:   $Actual \enspace inlet (X)=2.5*80 = \boxed{200 mg/l}$\\

% \pagebreak


% \item If a plant removes 35\% of the influent BOD in the primary treatment and 85\% of the remaining BOD in the secondary system, what is the BOD of the raw wastewater if the BOD of the final effluent is 20mg/l\\
% Solution:\\

% \begin{figure}[!h]
% \centering
% \begin{tikzpicture}[node distance =1.5cm, auto]
%     \draw ++(0,0) node [block] (Primary) {Primary};
    
%    \node[node distance=1.9in] (dummy_in) [left of=Primary] {Influent BOD};
%    \node[node distance=1.9in] (dummy_out) [right of=Primary] {Primary BOD Out};
% 	\node (Removal) [below of=Primary, yshift=-0in] {$Removal \enspace Efficiency=35\% $};
%     \path [arrow] (dummy_in)-- (Primary)  node [above] {\hspace{-4.8cm}$X \enspace mg/l \enspace$} node [below] {};
%     \path [arrow] (Primary) -- (dummy_out)  node [above] {\hspace{-4.9cm}$0.65X \enspace mg/l$} node [below] {};
%    \draw[arrow] (Process) -- (Removal);
% \end{tikzpicture}
% \end{figure}


% \begin{figure}[!h]
% \centering
% \begin{tikzpicture}[node distance =1.5cm, auto]
%     \draw ++(0,0) node [block] (Secondary) {Secondary};
    
%    \node[node distance=1.9in] (dummy_in) [left of=Secondary] {Primary BOD Out};
%    \node[node distance=1.9in] (dummy_out) [right of=Secondary] {Secondary BOD Out};
% 	\node (Removal) [below of=Secondary, yshift=-0in] {$Removal \enspace Efficiency=85\% $};
%     \path [arrow] (dummy_in)-- (Secondary)  node [above] {\hspace{-4.8cm}$0.65X \enspace mg/l \enspace$} node [below] {\hspace{-5cm}$100 \enspace mg/l$};
%     \path [arrow] (Secondary) -- (dummy_out)  node [above] {\hspace{-4.9cm}$20 \enspace mg/l$} node [below] {\hspace{-4.9cm}$15 \enspace mg/l$};
%    \draw[arrow] (Process) -- (Removal);
% \end{tikzpicture}
% \end{figure}
% \vspace{0.3cm}
% For the Secondary process:\\
% $\frac{In}{Out}: \enspace \frac{0.65X}{20}=\frac{100}{15} \implies X \enspace mg/l=\frac{100*20}{15*0.65}=\boxed{205 \enspace mg/l}$\\

% \vspace{0.3cm}
% Alternate Solution \#1

% $\xrightarrow[
% 				\text{X}\frac{mg}{l}
% 			]
% 			{
% 			\text{Influent BOD}
% 			}
%  \boxed{Primary}
%  \xrightarrow[
%  				\text{X-0.35X=X*(1-0.35)=0.65X}\frac{mg}{l}
%  			]
%  			{
%  			\text{Primary Effluent BOD}
%  			}
%  \boxed{Secondary}
%  \xrightarrow[
% 				\text{0.65X-0.5525X=(0.65-0.5525)X=0.0975X }
% 			 ]
% 			{
% 			\text{Secondary Effluent BOD}
% 			}
% $\\
% \hspace{2.8cm}$\downarrow$ {\tiny(0.35X)BOD Removed}\hspace{3.2cm}$\downarrow$ {\tiny(0.65*0.85)X = 0.5525X BOD Removed}\\
% $\implies 0.0975X=20 \implies X=\frac{20}{0.0975}=\boxed{205\frac{mg}{l}}$\\

% \vspace{0.3cm}

% Alternate Solution \#2:\\
% $\xrightarrow[\text{X}\frac{mg}{l}]{\text{Influent BOD}}\boxed{Primary}\xrightarrow[\text{0.65X}]{\text{Primary Effluent BOD}}\boxed{Secondary}\xrightarrow[\text{(0.65*0.15)X}]{\text{Secondary Effluent BOD}}$\\
% \hspace{2.8cm}$\downarrow$ {\tiny(0.35X)BOD Removed}\hspace{2.2cm}$\downarrow$ {\tiny(0.65X*0.85)BOD Removed}\\

% Primary Effluent BOD = Influent BOD * (1-Primary BOD Removal), and\\
% Secondary Effluent BOD=[Primary Effluent BOD]*(1-Secondary BOD Removal)\\
% Secondary Eff. BOD=[Influent BOD * (1-Primary BOD Removal)]*(1-Secondary BOD Removal)\\

% Therefore, 20 = [X*(1-0.35)] * (1-0.85)= X*0.65*0.15\\
% $\implies 20 \enspace \frac{mg}{l}= 0.0975X \implies X=\frac{20}{0.0975}=\boxed{205 \enspace \frac{mg}{l}}$\\

% \end{enumerate}


% \pagebreak
% \begin{snugshade*}
% 	\item \noindent\textsc{Pumping}
% \end{snugshade*}
% For Grades I \& II, pumping rate problems include the following:
% \begin{enumerate}
% \definecolor{shadecolor}{RGB}{225, 235, 235}
% \begin{snugshade*}
% \item \noindent\textsc{Calculating volume pumped in a given time interval given the pump flow rate\\}
% \end{snugshade*}
% \textbf{Method:\\}
% \hspace{1cm}Step 1. Multiply the pump flow rate by the time interval\\
% \textbf{Make sure:}
% \begin{itemize}
% \item The time units - in the given time interval and in the pump flow rate match
% \end{itemize}

% \begin{snugshade*}
% \item \noindent\textsc{Calculating time to pump a certain volume given the pump flow rate\\}
% \end{snugshade*}
% \textbf{Method:\\}
% \hspace{1cm}Step 1. Calculate the total volume pumped\\
% \hspace{1cm}Step 2.	Divide the total volume by the pump flow rate\\
% \textbf{Make sure:}
% \begin{itemize}
% \item The volume units - in the volume that needs to be pumped and in the pump flow rate match
% \item The time unit in the pump flow rate needs to be converted to the time unit that you need the answer in
% \end{itemize}
% \end{enumerate}
% \pagebreak
% \hl{Example Problems:}\\

% \begin{enumerate}

% \item A sludge pump is set to pump 5 minutes each hour. It pumps at the rate of 35 gpm. How many gallons of sludge are pumped each day?\\
% Solution:\\
% $\frac{35 \enspace gal \enspace sludge}{\cancel{min}}*\frac{5 \enspace \cancel{min}}{\cancel{hr}} *\frac{24 \enspace \cancel{hr}}{day}=\boxed{\frac{4,200 \enspace gallons}{day}}$\\
% \vspace{0.5cm}

% \item A sludge pump operates 5 minutes each 15 minute interval.  If the pump capacity is 60 gpm, how many gallons of sludge are pumped daily?

% $\frac{60 \enspace gal \enspace sludge}{\xcancel{min}}*\frac{5 \enspace \xcancel{min}}{15 \enspace \cancel{min}}*1440\frac{\cancel{min}}{day}=\boxed{\frac {28,800 \enspace gal \enspace sludge }{day}}$\\

% \item Given the tank is 10ft wide, 12 ft long and 18 ft deep tank including 2 ft of freeboard when filled to capacity. How much time (minutes) will be required to pump down this tank to a depth of 2 ft when the tank is at maximum capacity using a 600 GPM pump\\
% Solution:\\
% \vspace{0.5cm}


% \begin{tikzpicture}

% \pgfmathsetmacro{\cubexx}{4}
% \pgfmathsetmacro{\cubeyy}{1.5}
% \pgfmathsetmacro{\cubezz}{2}
% \pgfmathsetmacro{\cubex}{4}
% \pgfmathsetmacro{\cubey}{0.5}
% \pgfmathsetmacro{\cubez}{2}
% \pgfmathsetmacro{\cubexxx}{4}
% \pgfmathsetmacro{\cubeyyy}{4}
% \filldraw [fill=cyan!10!white, draw=black] (0,-\cubey,0) -- ++(-\cubexx,0,0) -- ++(0,-\cubeyy,0) -- ++(\cubexx,0,0) -- cycle ;
% \filldraw [fill=cyan!0!white, draw=black] (0,-\cubey,0) -- ++(0,0,-\cubezz) -- ++(0,-\cubeyy,0) -- ++(0,0,\cubezz) -- cycle;
% \filldraw [fill=cyan!10!white, draw=black] (0,-\cubey,0) -- ++(0,0,-\cubezz) -- ++(0,-\cubeyy,0) -- ++(0,0,\cubezz) -- cycle;
% %\filldraw [fill=cyan!10!white, draw=black] (0,-\cubey,0) -- ++(-\cubexx,0,0) -- ++(0,0,-\cubezz) -- ++(\cubexx,0,0) -- cycle;
% %%%\draw (0,-0.5,0) -- ++(-\cubex,0,0) -- ++(0,-\cubey,-\cubez) -- ++(\cubex,0,0) -- cycle;
% \draw (-\cubex,0,0) -- ++(0,0,-\cubez) -- ++(0,-\cubey,0) -- ++(0,0,\cubez) -- cycle;
% \draw (0,-\cubey,0) -- ++(-\cubex,0,0) -- ++(0,0,-\cubez) -- ++(\cubex,0,0) -- cycle;
% \filldraw [fill=white, draw=black] (0,0,0) -- ++(-\cubex,0,0) -- ++(0,-\cubey,0) -- ++(\cubex,0,0) -- cycle ;
% \filldraw [fill=white, draw=black] (0,0,0) -- ++(0,0,-\cubez) -- ++(0,-\cubey,0) -- ++(0,0,\cubez) -- cycle;
% \filldraw [fill=white, draw=black] (0,0,0) -- ++(0,0,-\cubez) -- ++(0,-\cubey,0) -- ++(0,0,\cubez) -- cycle;
% \filldraw [fill=white, draw=black] (0,0,0) -- ++(-\cubex,0,0) -- ++(0,0,-\cubez) -- ++(\cubex,0,0) -- cycle;

% %\filldraw [fill=RoyalBlue!10!white, draw=black] (0,-1.5,0) -- ++(-\cubex,0,0) -- ++(0,-\cubey,0) -- ++(\cubex,0,0) -- cycle ;

% %\filldraw [fill=RoyalBlue!10!white, draw=black] (0,-1.5,0) -- ++(0,0,-\cubez) -- ++(0,-\cubey,0) -- ++(0,0,\cubez) -- cycle;



% %%\draw (0,-0.5,0) -- ++(-\cubex,0,0) -- ++(0,0,-\cubez) -- ++(\cubex,0,0) -- cycle;
% %%\filldraw [fill=white, draw=black] (-\cubex,0,0) -- ++(0,0,-\cubez) -- ++(0,-\cubey,0) -- ++(0,0,\cubez) -- cycle;
% %%\filldraw [fill=white, draw=black] (0,-\cubey,0) -- ++(-\cubex,0,0) -- ++(0,0,-\cubez) -- ++(\cubex,0,0) -- cycle ;

% \draw [<->] (-4,-2.3) -- (0,-2.3) node [midway, below] {12' Long};
% \draw [<->] (1,-1.3) -- (1,.2) node [midway, midway] {\hspace{4.5cm}16' Water Depth (Initial)};
% \draw [<->] (0.4,-1.62) -- (0.4,-1.1) node [midway, midway] {\hspace{-4.8cm} 2' Water Depth (Final)};
% \draw [<->] (1,.8) -- (1,.2) node [midway, midway] {\hspace{2.4cm}2' Freeboard};
% \draw [<->] (1,-1.3) -- (0,-2.3) node [midway, midway] {\hspace{2.3cm}10' Wide};
% \end{tikzpicture}\\
% Volume to be pumped=$12 \enspace ft*10 \enspace ft *(16-2)\enspace ft=1,680ft^3$\\
% \vspace{0.3cm}
% $\implies \frac{1,680\cancel{ft^3}*7.48\frac{\cancel{gal}}{\cancel{ft^3}}}{600\frac{\cancel{gal}}{min}}=\boxed{21min}$
% \end{enumerate}

% \end{enumerate}
% \pagebreak
% \begin{center}
% \phantom{A}
% \vspace{10cm}

% BLANK PAGE
% \end{center}
% \pagebreak
% \end{document}
