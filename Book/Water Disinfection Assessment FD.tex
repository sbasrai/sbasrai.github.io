Disinfection by-products are a product of:\\
Filtration\\
Disinfection\\
Sedimentation\\
Adsorption\\
Chloramine is most effective as a \rule{2cm}{0.3pt} disinfectant.\\
Primary\\
Secondary\\
Third\\
First\\
Name the two types of hypochlorites used to disinfect water.\\
Chloride and monochloride\\
Sodium and calcium\\
Ozone and hydroxide\\
Arsenic and manganese\\
Name two methods commonly used to disinfect drinking water other than chlorination.\\
Ozone and ultraviolet light\\
Soap and agitation\\
Filtration and adsorption\\
Salt and vinegar\\
In order to determine the effectiveness of disinfection, it is desirable to maintain a disinfectant residual of at least \rule{2cm}{0.3pt} mg/L entering the distribution system.\\
0.10\\
0.5\\
0.3\\
0.2\\
Secondary disinfectants are used to provide a \rule{2cm}{0.3pt} in the distribution\\
system.\\
Color\\
Chemical\\
Smell\\
Residual\\
Primary disinfectants are used to \rule{2cm}{0.3pt}microorganisms.\\
Hurt\\
Inactivate\\
Burn up\\
Evaporate\\
The quantity of chlorine remaining after primary disinfection is called a \rule{2cm}{0.3pt} residual.\\
Chlorine\\
Permaganate\\
Hot\\
Cold\\
The two most common types of chlorine disinfection by-products include:\\
TTHM and HAA5\\
TTHA of HMM5\\
Turbidity and color\\
Chloride and fluoride\\
In order to determine the effectiveness of disinfection, it is desirable to maintain a disinfectant residual of at least \rule{1cm}{0.5pt}  mg/L entering the distribution system.\\
0.10\\
0.5\\
0.3\\
0.2\\
A \rule{1cm}{0.5pt}  residual of chlorine is required throughout the system.\\
Large\\
High\\
Trace\\
Hot\\
The test used to determine the effectiveness of disinfection is called the:\\
Coliform bacteria test\\
Color test\\
Turbidity test\\
Particle test\\
Name two methods commonly used to disinfect drinking water other than chlorination.\\
Ozone and ultraviolet light\\
Soap and agitation\\
Filtration and adsorption\\
Salt and vinegar\\
Name the two types of hypochlorites used to disinfect water.\\
Chloride and monochloride\\
Sodium and calcium\\
Ozone and hydroxide\\
Arsenic and manganese\\
Free chlorine can only be obtained after \rule{1cm}{0.5pt}  point chlorination has been achieved.\\
Breakpoint\\
Fastpoint\\
Softpoint\\
Onpoint\\
The meaning of the “C” and the “T” in the term CT stands for:\\
Concentration and time\\
Color and turbidity\\
Calcium and tortellini\\
Chlorine and turbidity\\
Chloramine is most affective as a \rule{1cm}{0.5pt} disinfectant.\\
Primary\\
Secondary\\
Third\\
First\\
TTHMs and HAA5s can affect:\\
Health\\
Aesthetics\\
Color\\
Odor\\
 The multiple barrier treatment approach includes\\
Sterilization and filtration\\
Disinfection and filtration\\
Disinfection and sterilization\\
Infection and filtration\\
The maximum disinfectant residual allowed for chlorine in a water system is\\
$.02 \mathrm{mg} / \mathrm{L}$\\
$2.0 \mathrm{mg} / \mathrm{L}$\\
$3.0 \mathrm{mg} / \mathrm{L}$\\
$4.0 \mathrm{mg} / \mathrm{L}$\\
 What is the disinfectant byproduct caused by ozonation?\\
Trihalomethanes\\
Bromate\\
Chlorite\\
No DBP formation\\
 Haloacitic Acids are also known as\\
TTHM\\
$\mathrm{HOCL}$\\
Chlorite\\
HAA5\\
 What is the MCL for trihalomethanes?\\
$.10 \mathrm{mg} / \mathrm{L}$\\
$.06 \mathrm{mg} / \mathrm{L}$\\
$.08 \mathrm{mg} / \mathrm{L}$\\
$.12 \mathrm{mg} / \mathrm{L}$\\
 What is the MCL for Haloacitic Acids?\\
$100 \mathrm{ppb}$\\
$60 \mathrm{ppb}$\\
$80 \mathrm{ppb}$\\
$120 \mathrm{ppb}$\\
What is the $\mathrm{MCL}$ for bromate?\\
$.010 \mathrm{mg} / \mathrm{L}$\\
$.020 \mathrm{mg} / \mathrm{L}$\\
$.030 \mathrm{mg} / \mathrm{L}$\\
$.040 \mathrm{mg} / \mathrm{L}$\\
What is residual Chlorine?\\
Chlorine used to disinfect\\
The amount of chlorine after the demand has been satisfied\\
The amount of chlorine added before disinfection\\
Film left on DPD kit to measure residual\\
 When Chlorine reacts with natural organic matter in water it can create\\
Disinfectant by-products\\
Coliform bacteria\\
Chloroform\\
Calcium\\
 What are trihalomenthanes classified as\\
Salts\\
Inorganic compounds\\
Volatile organic compounds\\
Radio\\
 What disinfectant is used for emergency purposes and not utilized in the water treatment industry?\\
Chlorine\\
Iodine\\
Ozone\\
Chlorine Dioxide\\
 What is the disinfectant with the least killing power but that has the longest lasting residual?\\
Chlorine\\
Ozone\\
Chlorine Dioxide\\
Chloramines\\
 The active ingredient in household bleach is\\
Calcium hypochlorite\\
Calcium hydroxide\\
Sodium hypochlorite\\
Sodium hydroxide\\
Cryptosporidium is not resistant to this chemical\\
Ozone\\
Chlorine Dioxide\\
Chlorine\\
Both $A$ \& $B$\\
 If a coliform test is positive, how many repeat samples are required at a minimum?\\
None\\
1\\
3\\
Depends on the severity of the positive sample\\
 Your water system takes 75 coliform tests per month. This month there were 6 positive samples. What is the percentage of samples which tested positive? Did your system violate regulations?\\
$3 \%$ Yes\\
$5 \% \mathrm{No}$\\
$8 \%$ Yes\\
$10 \%$ No\\
  The form of Chlorine which is $100 \%$ available chlorine is?\\
Sodium Hypochlorite\\
Calcium Hypochlorite\\
Calcium Hydroxide\\
Gaseous Chlorine\\
 What is the minimum amount of chlorine residual required in the distribution system?\\
There is no minimum\\
$\mathrm{mg} / \mathrm{L}$\\
$0.2 \mathrm{mg} / \mathrm{L}$\\
$\mathrm{mg} / \mathrm{L}$\\
 What is the approximate $\mathrm{pH}$ range of sodium hypochlorite?\\
4-5\\
6-7\\
$9-11$\\
$12-14$\\
 What is the typical concentration of sodium hypochlorite utilized by water treatment professionals?\\
$5 \%$\\
$65 \%$\\
$100 \%$\\
$12.5 \%$\\
 Chlorine demand refers to\\
Chlorine in the system for a given time\\
The difference between chlorine applied and chlorine residual-usually caused by inorganics, organics, bacteria, algae, ammonia, etc.\\
Chlorine needed to produce a higher $\mathrm{pH}$\\
None of the above\\
 What is the most effective chlorine disinfectant?\\
Dichloramine\\
Trichloramine\\
Hypochlorite Ion\\
Hypochlorous acid\\
What can form when chlorine reacts with natural organic matter in source water?\\
Disinfectant by-products\\
Sulfur\\
Algae\\
Coliform bacteria\\
 What kind of solution is used to check for a gas chlorine leak?\\
Sodium hydroxide\\
Ozone\\
Ammonia\\
Calcium hypochlorite\\
 Chlorine is\\
Heavier than air\\
Lighter than air\\
Brown in color\\
not harmful to your health\\
 Chlorine demand may vary due to\\
Chlorine demand always stays the same\\
Temperature\\
$\mathrm{pH}$\\
Both B and C\\
 What effect does high turbidity have on disinfection?\\
It can increase chlorine demand\\
It has no effect\\
It gives the water a milky appearance that will clear out after some time\\
You must increase the temperature of the water\\
  What is the target chlorine:ammonia ratio?\\
$2: 1$\\
$3: 1$\\
$4: 1$\\
$5: 1$\\
 What is the MCL for Nitrates?\\
$1 \mathrm{ppm}$\\
$10 \mathrm{ppm}$\\
$5 \mathrm{ppm}$\\
None of the above\\
 What is the molecular weight of Chlorine?\\
70\\
14\\
65\\
20\\
 What disinfectant has the longest lasting residual?\\
Ozone\\
Chlorine\\
Chloramine\\
Chlorine Dioxide\\
 What are some of the early indicators of Nitrification?\\
Lowering chlorine residual\\
Excess ammonia in treated water\\
Raise in bacterial heterotrophic plate counts\\
All of the above\\
 What are THMs classified as?\\
Turbidity\\
Radiological\\
Volatile Organic Chemicals\\
Salts\\
 What method can operators employ to combat nitrification?\\
Lower residual chlorine target\\
Keep reservoir levels static\\
Minimize free ammonia in treated water\\
Increase water age\\
 How many times stronger is Chlorine compared to monochloramine?\\
250 times\\
20 times\\
1500 times\\
5 times\\
What chemicals are formed when chlorine is mixed with water?\\
Hydrogen sulfide and ammonia\\
DPD and carbon dioxide\\
Sodium hypochlorite and calcium hypochlorite\\
Hypochlorous acid and hydrochloric acid\\
  Chlorine residual is measured in the field using the\\
a. Electroconductivity method\\
b. EDTA titrimetric method\\
c. Ortho-tolidine colorimetric method\\
d. DPD colorimetric method\\
e. Differential $\mathrm{pH}$ method\\
In nitrification, bacteria consume excess ammonia in the water and produce\\
a. Chloramines\\
b. Free chlorine\\
c. Urine\\
d. Nitrite\\
e. Sodium thiosulfate\\
  Which of the following is a form of free chlorine?\\
a. Nitrite\\
b. Hypochlorous acid\\
c. Monochloramine\\
d. Hydrochloric acid\\
e. Trichloramine\\
  A distribution system operator measures a total chlorine residual of $1.25 \mathrm{mg} / \mathrm{L}$. How many points on the chlorine breakpoint curve may display this residual?\\
a. Zero\\
b. One\\
c. Two\\
d. Three\\
e. Four\\
  What is the chlorine dosage that must be applied when disinfecting a pipeline using the slug method?\\
a. $\quad 300 \mathrm{mg} / \mathrm{L}$\\
b. $\quad 100 \mathrm{mg} / \mathrm{L}$\\
c. $\quad 50 \mathrm{mg} / \mathrm{L}$\\
d. $\quad 25 \mathrm{mg} / \mathrm{L}$\\
e. $\quad 6 \mathrm{mg} / \mathrm{L}$ \\
Which of the following is a form of combined chlorine?\\
a. Hypochlorite ion\\
b. Hypochlorous acid\\
c. Monochloramine\\
d. Hydrochloric acid\\
e. Free ammonia\\
 A distribution system operator measures a total chlorine residual of $1.25 \mathrm{mg} / \mathrm{L}$, and a free chlorine residual of $1.15 \mathrm{mg} / \mathrm{L}$ : This indicates that\\
a. The system is operating with a chloramine residual\\
b. The chlorine demand is $0.10 \mathrm{mg} / \mathrm{L}$\\
c. The chlorine demand is $2.40 \mathrm{mg} / \mathrm{L}$\\
d. Chloramines are being destroyed by free chlorine\\
e. The system is operating to the right of the breakpoint on the chloramine curve\\
 Which of the following is the most desirable form of combined residual chlorine?\\
a. Hypochlorite ion\\
b. Hypochlorous acid\\
c. Monochloramine\\
d. Dichloramine\\
e. Trichloramine\\
  Of the following, which is the most effective disinfectant?\\
a. Hypochlorite ion\\
b. Hypochlorous acid\\
c. Monochloramine\\
d. Dichloramine\\
e. Trichloramine\\
  A field chlorine residual measurement shows no reading at one minute, but $2.1 \mathrm{mg} / \mathrm{L}$ after three minutes. This indicates that\\
a. The field DPD test kit needs to be returned to the laboratory for maintenance\\
b. There is no chlorine residual\\
c. There is no free chlorine residual, but there are $2.1 \mathrm{mg} / \mathrm{L}$ of chloramines\\
d. There is no combined residual, but the free chlorine residual is $2.1 \mathrm{mg} / \mathrm{L}$\\
e. The analyst should wait an additional three minutes and re-test\\
  When disinfecting a storage tank, one method calls for the bottom $6 \%$ of the tank volume to be chlorinated for at least 6 hours with an applied chlorine dosage of\\
a. $\quad 50 \mathrm{mg} / \mathrm{L}$\\
b. $\quad 25 \mathrm{mg} / \mathrm{L}$\\
c. $\quad 6 \mathrm{mg} / \mathrm{L}$\\
d. $\quad 4 \mathrm{mg} / \mathrm{L}$\\
e. $\quad 0.2 \mathrm{mg} / \mathrm{L}$ \\
Residual chlorine refers to\\
a. The amount of chlorine in the chlorinated water after several minutes\\
b. The chlorine needed to disinfect the water supply\\
c. The chlorine needed to produce floc in the water\\
d. The sludge in the bottom of the chlorine solution tank\\
e. None of the above\\
 While handling sodium hypochlorite, proper safety precautions include\\
a. Avoiding situations that could splash hypochlorite solution\\
b. Using a face shield and/or goggles to avoid eye contact\\
c. Minimizing skin contact with rubber gloves and/or protective clothing\\
d. All of the above\\
e. None of the above are necessary\\
  The fusible plug that is in all chlorine containers\\
a. Is not necessary\\
b. May be used as a tap for the chlorine source\\
c. Should be removed after the cylinders are empty\\
d. Should never be removed or tampered with\\
e. Should be removed prior to withdrawing chlorine from the container\\
 Sodium hypochlorite is a\\
a. Compound purchased in liquid solution used for disinfection\\
b. Dry neutralizing powder for treating chlorine burns\\
c. Gas delivered in 100-pound, 150-pound, or one-ton containers\\
d. Salt that is formed when hydrochloric acid is neutralized with caustic soda\\
e. None of the above\\
  The chlorine demand abruptly jumps in your source water. This may indicate that a. The water source has been contaminated\\
b. Flow rates in the distribution system have increased\\
c. The hypochlorite solution used for disinfection has deteriorated\\
d. The hypochlorite solution tank is empty\\
e. The hypochlorite ion has a higher concentration than hypochlorous acid\\
  The chemical compound typically found in chlorination tablets and granules is\\
a. Sodium hypochlorite\\
b. Sodium hydroxide\\
c. Sodium chloride\\
d. Calcium hypochlorite\\
e. Calcium hydroxide\\
The maximum rate of withdrawal of gas from a 150-pound chlorine cylinder in 24-hours is\\
a. $\quad 20$ pounds\\
b. $\quad 40$ pounds\\
c. $\quad 100$ pounds\\
d. $\quad 150$ pounds\\
e. None of the above\\
  The maximum rate of withdrawal of gas from a one-ton chlorine container in 24-hours is\\
a. $\quad 40$ pounds\\
b. $\quad 100$ pounds\\
c. $\quad 400$ pounds\\
d. One ton\\
e. Variable, depending on chlorine dosage requirements\\
  A chlorine leak can be detected by\\
a. An explosimeter\\
b. Checking the leak gauge\\
c. Applying ammonia solution\\
d. A tri-gas detector\\
e. None of the above\\
When using the continuous feed method of disinfection, a new water main should be flushed, disinfected at $50 \mathrm{mg} / \mathrm{L}$, and held at above $25 \mathrm{mg} / \mathrm{L}$ for at least\\
a. $\quad 6$ hours\\
b. $\quad 12$ hours\\
c. $\quad 24$ hours\\
d. $\quad 36$ hours\\
e. $\quad 48$ hours\\
  If you encounter a liquid chlorine leak in a one-ton container, what action should you take first, to reduce the severity of the leak?\\
a. Apply a caustic solution\\
b. Apply an acidic solution\\
c. Spray the container with water\\
d. Spray the container with an ammonia solution\\
e. Rotate the container to place the leak at the top\\
  What should the chlorine dosage be to water that has a chlorine demand of $1.5 \mathrm{mg} / \mathrm{L}$, when a free residual of $1.0 \mathrm{mg} / \mathrm{L}$ is desired?\\
a. $\quad 0.5 \mathrm{mg} / \mathrm{L}$\\
b. $\quad 1.0 \mathrm{mg} / \mathrm{L}$\\
c. $\quad 1.5 \mathrm{mg} / \mathrm{L}$\\
d. $2.5$ pounds per day\\
e. $2.5 \mathrm{mg} / \mathrm{L}$\\
  When chlorine reacts with natural organic matter in the water, it is possible to form\\
a. Disinfection by-products \\
b. Arsenic \\
c. MTBE \\
d. Coliforms\\
e. Synthetic organic compounds\\
Which of the following best describes the characteristics of chlorine when used for disinfection in drinking water?\\
a.	 Colorless, flammable, heavier than air\\
b. Greenish-yellow, nonflammable, lighter than air\\
c. Greenish-yellow, flammable, lighter than air\\
d.  Greenish-yellow, nonflammable, heavier than air\\
  Killing of pathogenic organisms in water treatment is called\\
a. Disinfection\\
b. Oxidätion\\
c. Pasteurization\\
d. Sterilization\\
Chlorine reacts with nitrogenous compounds to form\\
a. Ammonia nitrate\\
b. Free chlorine\\
c. Chlorinated hydrocarbons\\
d. Chloramines\\
  Sodium Hypochlorite is\\
a. A commercially available chlorine solution\\
b. A commercially available dry chlorine compound\\
c. Chlorine that is available in 100- and 150-pound cylinders\\
d. A reaction product of chlorine and caustic soda\\
A hypochlorinator is\\
a. Used to measure residual chlorine\\
b. Used in the treatment of iron and turbidity\\
c. Used to feed a liquid solution into a water supply\\
d. Used to measure an adequate amount of chlorine gas into the supply\\
  When calcium hypochlorite is used for disinfecting a water supply, it should be\\
a.	 Dissolved in water, allowed to settle, and the supernatant siphoned off and fed into the water system\\
b. Dissolved in water as a dry chemical then injected into the water system\\
c. Fed as a dry chemical directly into the pipeline\\
d. Fed as a dry powder into the clear well\\
The chlorine gas feed rate is usually controlled by adjusting the\\
a. water flow to the injector\\
b. valve on the chlorine cylinder\\
c.pressure in the chlorine cylinder\\
d. rotameter control valve\\
If disinfection is incomplete because the chlorine residual is in the hypochlorite ion form, what should you change to improve disinfection?\\
a. Calcium\\
b. Hardness\\
c. pH\\
d. alkalinity\\
Breakpoint chlorination is achieved when\\
a. Free ammonia can be tasted in the water\\
b. No chlorine residual is detected\\
c. The strong chlorine tasted at the plant did not persist in the distribution system\\
d. When chlorine dosage is increased, a corresponding increase in residual is detected\\
Because chlorine residual is related to the $\mathrm{pH}$ of the water, it may be said that\\
a. A higher $\mathrm{pH}$ requires a higher chiorine residual\\
b. A higher $\mathrm{pH}$ requires a lower chlorine residual\\
c. A lower pH requires a higher chlorine residual\\
d. pH  has no effect on chlorine residual\\
  As long as the temperature is steady, the pressure indicator on a chlorine cylinder will until all the chlorine has been gasified\\
a. Remain steady\\
b. Decrease slowly\\
c. Decrease rapidly\\
d. Increase slightly\\
When fresh, the typical concentration of sodium hypochlorite solution is\\
a. $\quad 1.25 \%$\\
b. $\quad 6.5 \%$\\
c. $\quad 12.5 \%$\\
d. $\quad 65 \%$\\
e. variable, depending on the manufacturer\\
Chlorine in a dry form is called:\\
a.	hypochlorite\\
b.	hypochlorous\\
c.	hydrochlorite\\
d.	hydroxide\\
Which of the following procedures is done when preparing to disconnect a chlorine cylinder?\\
a.	close the cylinder valve first to allow time for the chlorine to be drawn off\\
b.	loosen the line to the tank and then shut off the valve to the chlorine cylinder\\
c.	shut off the water supply and allow sufficient time for the chloril1e to be drawn off\\
d.	tum the chlorinator feed rate valve off then turn the valve on the chlorinator cylinder\\
A vacuum is formed in the chlorinator by the:\\
a	chlorine cylinder pressure\\
b.	pressure differential through the ejector\\
c.	chlorine feed pump\\
d.	rotameter-\\
When calcium hypochlorite is used for disinfecting a water supply, it should be be:\\
a. Dissolved in water, allowed to settle, and the supernatant siphoned off and fed into the water system\\
b. Dissolved in water as a dry chemical then injected into the water system\\
c. Fed as a dry chemical directly into the pipeline\\
d. Fed as a dry powder into the clear well\\
Because chlorine residual is related to the pH of the water, it may be said that:\\
a. A higher pH requires a higher chlorine residual\\
b. A higher pH requires a lower chlorine residual\\
c. A lower pH requires a higher chlorine residual\\
d. A lower pH has no effect on chlorine residual\\
Which of the following best describes "chlorine demand"?\\
a. The difference between the amount of chłorine added and turbidity\\
b. The difference between the amount of chlorine added and $\mathrm{pH}$\\
c. The difference between the total chlorine residual and the free chlorine residual\\
(d.) The difference between the amount of chlorine added and the amount of residual chlorine remaining after a given contact time\\
Chlorine reacts with nitrogenous compounds to form\\
a. Ammonia nitrate\\
b. Free chlorine\\
c. Chlorinated hydrocarbons\\
d. Chloramines\\
\newpage
