%\chapter*{Disinfection}
\begin{enumerate}






  \item Chlorine gas is \_\_\_ times heavier than breathing air\\
a. 2.5\\
b. 20\\
c. 60\\
d. 460\\
  \item A commonly used method to test for chlorine residual in water is called the method.\\
a. HTH\\
b. THM\\
c. VOC\\
d. *DPD\\

  \item When chlorine gas is added to water the $\mathrm{pH}$ goes down due to\\
a. chlorine gas producing caustic substances\\
b. two base materials that form\\
c. *two acids that form\\
d. caustic soda being formed in the water 

\item Disinfection by-products are a product of:
\begin{enumerate}
\item Filtration
\item Disinfection
\item Sedimentation
\item Adsorption
\end{enumerate}

\item Chloramine is most effective as a \rule{2cm}{0.3pt} disinfectant.
\begin{enumerate}
\item Primary
\item Secondary
\item Third
\item First
\end{enumerate}

\item Name the two types of hypochlorites used to disinfect water.
\begin{enumerate}
\item Chloride and monochloride
\item Sodium and calcium
\item Ozone and hydroxide
\item Arsenic and manganese
\end{enumerate}

\item Name two methods commonly used to disinfect drinking water other than chlorination.
\begin{enumerate}
\item Ozone and ultraviolet light
\item Soap and agitation
\item Filtration and adsorption
\item Salt and vinegar
\end{enumerate}

\item In order to determine the effectiveness of disinfection, it is desirable to maintain a disinfectant residual of at least \rule{2cm}{0.3pt} mg/L entering the distribution system.
\begin{enumerate}
\item 0.10
\item 0.5
\item 0.3
\item 0.2
\end{enumerate}

\item Secondary disinfectants are used to provide a \rule{2cm}{0.3pt} in the distribution
system.
\begin{enumerate}
\item Color
\item Chemical
\item Smell
\item Residual
\end{enumerate}

\item Primary disinfectants are used to \rule{2cm}{0.3pt}microorganisms.
\begin{enumerate}
\item Hurt
\item Inactivate
\item Burn up
\item Evaporate
\end{enumerate}

\item The quantity of chlorine remaining after primary disinfection is called a \rule{2cm}{0.3pt} residual.
\begin{enumerate}
\item Chlorine
\item Permaganate
\item Hot
\item Cold
\end{enumerate}

\item The two most common types of chlorine disinfection by-products include:
\begin{enumerate}
\item TTHM and HAA5
\item TTHA of HMM5
\item Turbidity and color
\item Chloride and fluoride
\end{enumerate}

\item In order to determine the effectiveness of disinfection, it is desirable to maintain a disinfectant residual of at least \rule{1cm}{0.5pt}  mg/L entering the distribution system.
\begin{enumerate}
\item 0.10
\item 0.5
\item 0.3
\item 0.2
\end{enumerate}

\item A \rule{1cm}{0.5pt}  residual of chlorine is required throughout the system.
\begin{enumerate}
\item Large
\item High
\item Trace
\item Hot
\end{enumerate}

\item The test used to determine the effectiveness of disinfection is called the:
\begin{enumerate}
\item Coliform bacteria test
\item Color test
\item Turbidity test
\item Particle test
\end{enumerate}


\item Name two methods commonly used to disinfect drinking water other than chlorination.
\begin{enumerate}
\item Ozone and ultraviolet light
\item Soap and agitation
\item Filtration and adsorption
\item Salt and vinegar
\end{enumerate}

\item Name the two types of hypochlorites used to disinfect water.
\begin{enumerate}
\item Chloride and monochloride
\item Sodium and calcium
\item Ozone and hydroxide
\item Arsenic and manganese
\end{enumerate}

\item Free chlorine can only be obtained after \rule{1cm}{0.5pt}  point chlorination has been achieved.
\begin{enumerate}
\item Breakpoint
\item Fastpoint
\item Softpoint
\item Onpoint
\end{enumerate}

\item The meaning of the “C” and the “T” in the term CT stands for:
\begin{enumerate}
\item Concentration and time
\item Color and turbidity
\item Calcium and tortellini
\item Chlorine and turbidity
\end{enumerate}

\item Chloramine is most affective as a \rule{1cm}{0.5pt} disinfectant.
\begin{enumerate}
\item Primary
\item Secondary
\item Third
\item First
\end{enumerate}

\item TTHMs and HAA5s can affect:
\begin{enumerate}
\item Health
\item Aesthetics
\item Color
\item Odor
\end{enumerate}

\item  The multiple barrier treatment approach includes\\
\begin{enumerate}
\item Sterilization and filtration\\
\item Disinfection and filtration\\
\item Disinfection and sterilization\\
\item Infection and filtration
\end{enumerate}

\item The maximum disinfectant residual allowed for chlorine in a water system is\\
\begin{enumerate}
\item $.02 \mathrm{mg} / \mathrm{L}$\\
\item $2.0 \mathrm{mg} / \mathrm{L}$\\
\item $3.0 \mathrm{mg} / \mathrm{L}$\\
\item $4.0 \mathrm{mg} / \mathrm{L}$
\end{enumerate}

\item  What is the disinfectant byproduct caused by ozonation?\\
\begin{enumerate}
\item Trihalomethanes\\
\item Bromate\\
\item Chlorite\\
\item No DBP formation
\end{enumerate}

\item  Haloacitic Acids are also known as\\
\begin{enumerate}
\item TTHM\\
\item $\mathrm{HOCL}$\\
\item Chlorite\\
\item HAA5
\end{enumerate}

\item  What is the MCL for trihalomethanes?\\
\begin{enumerate}
\item $.10 \mathrm{mg} / \mathrm{L}$\\
\item $.06 \mathrm{mg} / \mathrm{L}$\\
\item $.08 \mathrm{mg} / \mathrm{L}$\\
\item $.12 \mathrm{mg} / \mathrm{L}$
\end{enumerate}

\item  What is the MCL for Haloacitic Acids?\\
\begin{enumerate}
\item $100 \mathrm{ppb}$\\
\item $60 \mathrm{ppb}$\\
\item $80 \mathrm{ppb}$\\
\item $120 \mathrm{ppb}$
\end{enumerate}

\item What is the $\mathrm{MCL}$ for bromate?\\
\begin{enumerate}
\item $.010 \mathrm{mg} / \mathrm{L}$\\
\item $.020 \mathrm{mg} / \mathrm{L}$\\
\item $.030 \mathrm{mg} / \mathrm{L}$\\
\item $.040 \mathrm{mg} / \mathrm{L}$
\end{enumerate}

\item What is residual Chlorine?\\
\begin{enumerate}
\item Chlorine used to disinfect\\
\item The amount of chlorine after the demand has been satisfied\\
\item The amount of chlorine added before disinfection\\
\item Film left on DPD kit to measure residual
\end{enumerate}

\item  When Chlorine reacts with natural organic matter in water it can create\\
\begin{enumerate}
\item Disinfectant by-products\\
\item Coliform bacteria\\
\item Chloroform\\
\item Calcium
\end{enumerate}

\item  What are trihalomenthanes classified as\\
\begin{enumerate}
\item Salts\\
\item Inorganic compounds\\
\item Volatile organic compounds\\
\item Radio
\end{enumerate}

\item  What disinfectant is used for emergency purposes and not utilized in the water treatment industry?\\
\begin{enumerate}
\item Chlorine\\
\item Iodine\\
\item Ozone\\
\item Chlorine Dioxide
\end{enumerate}

\item  What is the disinfectant with the least killing power but that has the longest lasting residual?\\
\begin{enumerate}
\item Chlorine\\
\item Ozone\\
\item Chlorine Dioxide\\
\item Chloramines
\end{enumerate}

\item  The active ingredient in household bleach is\\
\begin{enumerate}
\item Calcium hypochlorite\\
\item Calcium hydroxide\\
\item Sodium hypochlorite\\
\item Sodium hydroxide
\end{enumerate}

\item Cryptosporidium is not resistant to this chemical\\
\begin{enumerate}
\item Ozone\\
\item Chlorine Dioxide\\
\item Chlorine\\
\item Both $A$ \& $B$
\end{enumerate}

\item  If a coliform test is positive, how many repeat samples are required at a minimum?\\
\begin{enumerate}
\item None\\
\item 1\\
\item 3\\
\item Depends on the severity of the positive sample
\end{enumerate}

\item  Your water system takes 75 coliform tests per month. This month there were 6 positive samples. What is the percentage of samples which tested positive? Did your system violate regulations?\\
\begin{enumerate}
\item $3 \%$ Yes\\
\item $5 \% \mathrm{No}$\\
\item $8 \%$ Yes\\
\item $10 \%$ No
\end{enumerate}

  \item The form of Chlorine which is $100 \%$ available chlorine is?\\
\begin{enumerate}
\item Sodium Hypochlorite\\
\item Calcium Hypochlorite\\
\item Calcium Hydroxide\\
\item Gaseous Chlorine
\end{enumerate}

\item  What is the minimum amount of chlorine residual required in the distribution system?\\
\begin{enumerate}
\item There is no minimum\\
\item $\mathrm{mg} / \mathrm{L}$\\
\item $0.2 \mathrm{mg} / \mathrm{L}$\\
\item $\mathrm{mg} / \mathrm{L}$
\end{enumerate}

\item  What is the approximate $\mathrm{pH}$ range of sodium hypochlorite?\\
\begin{enumerate}
\item 4-5\\
\item 6-7\\
\item $9-11$\\
\item $12-14$
\end{enumerate}

\item  What is the typical concentration of sodium hypochlorite utilized in water treatment?\\
\begin{enumerate}
\item $5 \%$\\
\item $65 \%$\\
\item $100 \%$\\
\item $12.5 \%$
\end{enumerate}

\item  Chlorine demand refers to\\
\begin{enumerate}
\item Chlorine in the system for a given time\\
\item The difference between chlorine applied and chlorine residual-usually caused by inorganics, organics, bacteria, algae, ammonia, etc.\\
\item Chlorine needed to produce a higher $\mathrm{pH}$\\
\item None of the above
\end{enumerate}

\item  What is the most effective chlorine disinfectant?\\
\begin{enumerate}
\item Dichloramine\\
\item Trichloramine\\
\item Hypochlorite Ion\\
\item Hypochlorous acid
\end{enumerate} 

\item What can form when chlorine reacts with natural organic matter in source water?\\
\begin{enumerate}
\item Disinfectant by-products\\
\item Sulfur\\
\item Algae\\
\item Coliform bacteria
\end{enumerate}

\item  What kind of solution is used to check for a gas chlorine leak?\\
\begin{enumerate}
\item Sodium hydroxide\\
\item Ozone\\
\item Ammonia\\
\item Calcium hypochlorite
\end{enumerate}

\item  Chlorine is\\
\begin{enumerate}
\item Heavier than air\\
\item Lighter than air\\
\item Brown in color\\
\item not harmful to your health
\end{enumerate}

\item  Chlorine demand may vary due to\\
\begin{enumerate}
\item Chlorine demand always stays the same\\
\item Temperature\\
\item $\mathrm{pH}$\\
\item Both B and C
\end{enumerate}

\item  What effect does high turbidity have on disinfection?\\
\begin{enumerate}
\item It can increase chlorine demand\\
\item It has no effect\\
\item It gives the water a milky appearance that will clear out after some time\\
\item You must increase the temperature of the water
\end{enumerate}

  \item What is the target chlorine:ammonia ratio?\\
\begin{enumerate}
\item $2: 1$\\
\item $3: 1$\\
\item $4: 1$\\
\item $5: 1$
\end{enumerate}

\item  What is the MCL for Nitrates?\\
\begin{enumerate}
\item $1 \mathrm{ppm}$\\
\item $10 \mathrm{ppm}$\\
\item $5 \mathrm{ppm}$\\
\item None of the above
\end{enumerate}

\item  What is the molecular weight of Chlorine?\\
\begin{enumerate}
\item 70\\
\item 14\\
\item 65\\
\item 20
\end{enumerate}

\item  What disinfectant has the longest lasting residual?\\
\begin{enumerate}
\item Ozone\\
\item Chlorine\\
\item Chloramine\\
\item Chlorine Dioxide
\end{enumerate}

\item  What are some of the early indicators of Nitrification?\\
\begin{enumerate}
\item Lowering chlorine residual\\
\item Excess ammonia in treated water\\
\item Raise in bacterial heterotrophic plate counts\\
\item All of the above
\end{enumerate}

\item  What are THMs classified as?\\
\begin{enumerate}
\item Turbidity\\
\item Radiological\\
\item Volatile Organic Chemicals\\
\item Salts
\end{enumerate}

\item  What method can operators employ to combat nitrification?\\
\begin{enumerate}
\item Lower residual chlorine target\\
\item Keep reservoir levels static\\
\item Minimize free ammonia in treated water\\
\item Increase water age
\end{enumerate}

\item  How many times stronger is Chlorine compared to monochloramine?\\
\begin{enumerate}
\item 250 times\\
\item 20 times\\
\item 1500 times\\
\item 5 times
\end{enumerate}

\item What chemicals are formed when chlorine is mixed with water?
\begin{enumerate}
\item Hydrogen sulfide and ammonia
\item DPD and carbon dioxide
\item Sodium hypochlorite and calcium hypochlorite
\item Hypochlorous acid and hydrochloric acid
\end{enumerate}

  \item Chlorine residual is measured in the field using the\\
a. Electroconductivity method\\
b. EDTA titrimetric method\\
c. Ortho-tolidine colorimetric method\\
d. DPD colorimetric method\\
e. Differential $\mathrm{pH}$ method\\

\item In nitrification, bacteria consume excess ammonia in the water and produce\\
a. Chloramines\\
b. Free chlorine\\
c. Urine\\
d. Nitrite\\
e. Sodium thiosulfate\\
  \item Which of the following is a form of free chlorine?\\
a. Nitrite\\
b. Hypochlorous acid\\
c. Monochloramine\\
d. Hydrochloric acid\\
e. Trichloramine\\
  \item A distribution system operator measures a total chlorine residual of $1.25 \mathrm{mg} / \mathrm{L}$. How many points on the chlorine breakpoint curve may display this residual?\\
a. Zero\\
b. One\\
c. Two\\
d. Three\\
e. Four\\
  \item What is the chlorine dosage that must be applied when disinfecting a pipeline using the slug method?\\
a. $\quad 300 \mathrm{mg} / \mathrm{L}$\\
b. $\quad 100 \mathrm{mg} / \mathrm{L}$\\
c. $\quad 50 \mathrm{mg} / \mathrm{L}$\\
d. $\quad 25 \mathrm{mg} / \mathrm{L}$\\
e. $\quad 6 \mathrm{mg} / \mathrm{L}$ \\
\item Which of the following is a form of combined chlorine?\\
a. Hypochlorite ion\\
b. Hypochlorous acid\\
c. Monochloramine\\
d. Hydrochloric acid\\
e. Free ammonia\\
 \item A distribution system operator measures a total chlorine residual of $1.25 \mathrm{mg} / \mathrm{L}$, and a free chlorine residual of $1.15 \mathrm{mg} / \mathrm{L}$ : This indicates that\\
a. The system is operating with a chloramine residual\\
b. The chlorine demand is $0.10 \mathrm{mg} / \mathrm{L}$\\
c. The chlorine demand is $2.40 \mathrm{mg} / \mathrm{L}$\\
d. Chloramines are being destroyed by free chlorine\\
e. The system is operating to the right of the breakpoint on the chloramine curve\\
 \item Which of the following is the most desirable form of combined residual chlorine?\\
a. Hypochlorite ion\\
b. Hypochlorous acid\\
c. Monochloramine\\
d. Dichloramine\\
e. Trichloramine\\

  \item Of the following, which is the most effective disinfectant?\\

a. Hypochlorite ion\\
b. Hypochlorous acid\\
c. Monochloramine\\
d. Dichloramine\\
e. Trichloramine\\
  \item A field chlorine residual measurement shows no reading at one minute, but $2.1 \mathrm{mg} / \mathrm{L}$ after three minutes. This indicates that\\
a. The field DPD test kit needs to be returned to the laboratory for maintenance\\
b. There is no chlorine residual\\
c. There is no free chlorine residual, but there are $2.1 \mathrm{mg} / \mathrm{L}$ of chloramines\\
d. There is no combined residual, but the free chlorine residual is $2.1 \mathrm{mg} / \mathrm{L}$\\
e. The analyst should wait an additional three minutes and re-test\\
  \item When disinfecting a storage tank, one method calls for the bottom $6 \%$ of the tank volume to be chlorinated for at least 6 hours with an applied chlorine dosage of\\
a. $\quad 50 \mathrm{mg} / \mathrm{L}$\\
b. $\quad 25 \mathrm{mg} / \mathrm{L}$\\
c. $\quad 6 \mathrm{mg} / \mathrm{L}$\\
d. $\quad 4 \mathrm{mg} / \mathrm{L}$\\
e. $\quad 0.2 \mathrm{mg} / \mathrm{L}$ \\
\item Residual chlorine refers to\\
a. The amount of chlorine in the chlorinated water after several minutes\\
b. The chlorine needed to disinfect the water supply\\
c. The chlorine needed to produce floc in the water\\
d. The sludge in the bottom of the chlorine solution tank\\
e. None of the above\\
 \item While handling sodium hypochlorite, proper safety precautions include
a. Avoiding situations that could splash hypochlorite solution
b. Using a face shield and/or goggles to avoid eye contact
c. Minimizing skin contact with rubber gloves and/or protective clothing
d. All of the above
e. None of the above are necessary\\
  \item The fusible plug that is in all chlorine containers\\
a. Is not necessary\\
b. May be used as a tap for the chlorine source\\
c. Should be removed after the cylinders are empty\\
d. Should never be removed or tampered with\\
e. Should be removed prior to withdrawing chlorine from the container\\
 \item Sodium hypochlorite is a
a. Compound purchased in liquid solution used for disinfection\\
b. Dry neutralizing powder for treating chlorine burns\\
c. Gas delivered in 100-pound, 150-pound, or one-ton containers\\
d. Salt that is formed when hydrochloric acid is neutralized with caustic soda\\
e. None of the above\\
  \item The chlorine demand abruptly jumps in your source water. This may indicate that a. The water source has been contaminated
b. Flow rates in the distribution system have increased\\
c. The hypochlorite solution used for disinfection has deteriorated\\
d. The hypochlorite solution tank is empty\\
e. The hypochlorite ion has a higher concentration than hypochlorous acid\\
  \item The chemical compound typically found in chlorination tablets and granules is\\
a. Sodium hypochlorite\\
b. Sodium hydroxide\\
c. Sodium chloride\\
d. Calcium hypochlorite\\
e. Calcium hydroxide\\ 

\item The maximum rate of withdrawal of gas from a 150-pound chlorine cylinder in 24-hours is\\
a. $\quad 20$ pounds\\
b. $\quad 40$ pounds\\
c. $\quad 100$ pounds\\
d. $\quad 150$ pounds\\
e. None of the above\\
  \item The maximum rate of withdrawal of gas from a one-ton chlorine container in 24-hours is\\
a. $\quad 40$ pounds\\
b. $\quad 100$ pounds\\
c. $\quad 400$ pounds\\
d. One ton\\
e. Variable, depending on chlorine dosage requirements\\
  \item A chlorine leak can be detected by\\
a. An explosimeter\\
b. Checking the leak gauge\\
c. Applying ammonia solution\\
d. A tri-gas detector\\
e. None of the above\\

\item When using the continuous feed method of disinfection, a new water main should be flushed, disinfected at $50 \mathrm{mg} / \mathrm{L}$, and held at above $25 \mathrm{mg} / \mathrm{L}$ for at least\\
a. $\quad 6$ hours\\
b. $\quad 12$ hours\\
c. $\quad 24$ hours\\
d. $\quad 36$ hours\\
e. $\quad 48$ hours\\
  \item If you encounter a liquid chlorine leak in a one-ton container, what action should you take first, to reduce the severity of the leak?\\
a. Apply a caustic solution\\
b. Apply an acidic solution\\
c. Spray the container with water\\
d. Spray the container with an ammonia solution\\
e. Rotate the container to place the leak at the top\\
  \item What should the chlorine dosage be to water that has a chlorine demand of $1.5 \mathrm{mg} / \mathrm{L}$, when a free residual of $1.0 \mathrm{mg} / \mathrm{L}$ is desired?\\
a. $\quad 0.5 \mathrm{mg} / \mathrm{L}$\\
b. $\quad 1.0 \mathrm{mg} / \mathrm{L}$\\
c. $\quad 1.5 \mathrm{mg} / \mathrm{L}$\\
d. $2.5$ pounds per day\\
e. $2.5 \mathrm{mg} / \mathrm{L}$\\
  \item When chlorine reacts with natural organic matter in the water, it is possible to form\\
a. Disinfection by-products \\
b. Arsenic \\
c. MTBE \\
d. Coliforms\\
e. Synthetic organic compounds\\
\item Which of the following best describes the characteristics of chlorine when used for disinfection in drinking water?\\
a.	 Colorless, flammable, heavier than air\\
b. Greenish-yellow, nonflammable, lighter than air\\
c. Greenish-yellow, flammable, lighter than air\\
d.  Greenish-yellow, nonflammable, heavier than air\\
  \item Killing of pathogenic organisms in water treatment is called\\
a. Disinfection\\
b. Oxidätion\\
c. Pasteurization\\
d. Sterilization\\
\item Chlorine reacts with nitrogenous compounds to form\\
a. Ammonia nitrate\\
b. Free chlorine\\
c. Chlorinated hydrocarbons\\
d. Chloramines\\
  \item Sodium Hypochlorite is\\
a. A commercially available chlorine solution\\
b. A commercially available dry chlorine compound\\
c. Chlorine that is available in 100- and 150-pound cylinders\\
d. A reaction product of chlorine and caustic soda\\
\item A hypochlorinator is\\
a. Used to measure residual chlorine\\
b. Used in the treatment of iron and turbidity\\
c. Used to feed a liquid solution into a water supply\\
d. Used to measure an adequate amount of chlorine gas into the supply\\
  \item When calcium hypochlorite is used for disinfecting a water supply, it should be\\
a.	 Dissolved in water, allowed to settle, and the supernatant siphoned off and fed into the water system\\
b. Dissolved in water as a dry chemical then injected into the water system\\
c. Fed as a dry chemical directly into the pipeline\\
d. Fed as a dry powder into the clear well\\
\item The chlorine gas feed rate is usually controlled by adjusting the\\
a. water flow to the injector\\
b. valve on the chlorine cylinder\\
c.pressure in the chlorine cylinder\\
d. rotameter control valve\\
\item If disinfection is incomplete because the chlorine residual is in the hypochlorite ion form, what should you change to improve disinfection?\\
a. Calcium\\
b. Hardness\\
c. pH\\
d. alkalinity\\
\item Breakpoint chlorination is achieved when\\
a. Free ammonia can be tasted in the water\\
b. No chlorine residual is detected\\
c. The strong chlorine tasted at the plant did not persist in the distribution system\\
d. When chlorine dosage is increased, a corresponding increase in residual is detected\\

\item Because chlorine residual is related to the $\mathrm{pH}$ of the water, it may be said that\\
a. A higher $\mathrm{pH}$ requires a higher chiorine residual\\
b. A higher $\mathrm{pH}$ requires a lower chlorine residual\\
c. A lower pH requires a higher chlorine residual\\
d. pH  has no effect on chlorine residual\\


  \item As long as the temperature is steady, the pressure indicator on a chlorine cylinder will until all the chlorine has been gasified\\
a. Remain steady\\
b. Decrease slowly\\
c. Decrease rapidly\\
d. Increase slightly\\

\item When fresh, the typical concentration of sodium hypochlorite solution is\\
a. $\quad 1.25 \%$\\
b. $\quad 6.5 \%$\\
c. $\quad 12.5 \%$\\
d. $\quad 65 \%$\\
e. variable, depending on the manufacturer\\

\item Chlorine in a dry form is called:\\
a.	hypochlorite\\
b.	hypochlorous\\
c.	hydrochlorite\\
d.	hydroxide\\

\item Which of the following procedures is done when preparing to disconnect a chlorine cylinder?\\
a.	close the cylinder valve first to allow time for the chlorine to be drawn off\\
b.	loosen the line to the tank and then shut off the valve to the chlorine cylinder\\
c.	shut off the water supply and allow sufficient time for the chloril1e to be drawn off\\
d.	tum the chlorinator feed rate valve off then turn the valve on the chlorinator cylinder\\


\item A vacuum is formed in the chlorinator by the:\\
a	chlorine cylinder pressure\\
b.	pressure differential through the ejector\\
c.	chlorine feed pump\\
d.	rotameter-\\

\item When calcium hypochlorite is used for disinfecting a water supply, it should be be:\\
a. Dissolved in water, allowed to settle, and the supernatant siphoned off and fed into the water system\\
b. Dissolved in water as a dry chemical then injected into the water system\\
c. Fed as a dry chemical directly into the pipeline\\
d. Fed as a dry powder into the clear well\\

\item Because chlorine residual is related to the pH of the water, it may be said that:
a. A higher pH requires a higher chlorine residual\\
b. A higher pH requires a lower chlorine residual\\
c. A lower pH requires a higher chlorine residual\\
d. A lower pH has no effect on chlorine residual\\

\item Which of the following best describes "chlorine demand"?\\
a. The difference between the amount of chłorine added and turbidity\\
b. The difference between the amount of chlorine added and $\mathrm{pH}$\\
c. The difference between the total chlorine residual and the free chlorine residual\\
(d.) The difference between the amount of chlorine added and the amount of residual chlorine remaining after a given contact time\\


\item Chlorine reacts with nitrogenous compounds to form\\
a. Ammonia nitrate\\
b. Free chlorine\\
c. Chlorinated hydrocarbons\\
d. Chloramines\\

\item When two ton cylinders are feeding gas and one of them is frosted, what might be the problem?\\
A. The feed rate is too high\\
B. The line on the frosted tank is clogged\\
C. The valve on the unfrosted tank\\
D. The injector is clogged\\

\item There is low vacuum on the system and the flow rate is low when the rate valve is wide open, what is the problem?\\
A. The feed rate is too high\\
B. The injector is clogged\\
C. There is a clogged feed line\\
D. The rotameter is clogged\\



\end{enumerate}
\newpage

