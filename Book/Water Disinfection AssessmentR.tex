%\chapter*{Disinfection}
\begin{enumerate}[1.]
\item Chlorine gas is times heavier than breathing air\\
a. 2.5\\
b. 20\\
c. 60\\
d. 460\\
\item A commonly used method to test for chlorine residual in water is called the method.\\
a. HTH\\
b. THM\\
c. VOC\\
d. DPD\\
\item When chlorine gas is added to water the $\mathrm{pH}$ goes down due to\\
a. chlorine gas producing caustic substances\\
b. two base materials that form\\
c. *two acids that form\\
d. caustic soda being formed in the water\\
\item Disinfection by-products are a product of:\\
a. Filtration\\
b. Disinfection\\
c. Sedimentation\\
d. Adsorption\\
\item Chloramine is most effective as a disinfectant.\\
a. Primary\\
b. Secondary\\
c. Third\\
d. First\\
\item Name the two types of hypochlorites used to disinfect water.\\
a. Chloride and monochloride\\
b. Sodium and calcium\\
c. Ozone and hydroxide\\
d. Arsenic and manganese\\
\item Name two methods commonly used to disinfect drinking water other than chlorination.\\
a. Ozone and ultraviolet light\\
b. Soap and agitation\\
c. Filtration and adsorption\\
d. Salt and vinegar\\
\item In order to determine the effectiveness of disinfection, it is desirable to maintain a disinfectant residual of at least $\mathrm{mg} / \mathrm{L}$ entering the distribution system.\\
a. 0.10\\
b. 0.5\\
c. 0.3 d. 0.2\\
\item Secondary disinfectants are used to provide a in the distribution system.\\
a. Color\\
b. Chemical\\
c. Smell\\
d. Residual\\
\item Primary disinfectants are used to microorganisms.\\
a. Hurt\\
b. Inactivate\\
c. Burn up\\
d. Evaporate\\
\item The quantity of chlorine remaining after primary disinfection is called a residual.\\
a. Chlorine\\
b. Permaganate\\
c. Hot\\
d. Cold\\
\item The two most common types of chlorine disinfection by-products include:\\
a. TTHM and HAA5\\
b. TTHA of HMM5\\
c. Turbidity and color\\
d. Chloride and fluoride\\
\item In order to determine the effectiveness of disinfection, it is desirable to maintain a disinfectant residual of at least $\mathrm{mg} / \mathrm{L}$ entering the distribution system.\\
a. 0.10\\
b. 0.5\\
c. 0.3\\
d. 0.2\\
\item A\_\_ residual of chlorine is required throughout the system.\\
a. Large\\
b. High\\
c. Trace\\
d. Hot\\
\item The test used to determine the effectiveness of disinfection is called the:\\
a. Coliform bacteria test\\
b. Color test\\
c. Turbidity test\\
d. Particle test\\
\item Name two methods commonly used to disinfect drinking water other than chlorination.\\
a. Ozone and ultraviolet light\\
b. Soap and agitation\\
c. Filtration and adsorption\\
d. Salt and vinegar\\
\item Name the two types of hypochlorites used to disinfect water.\\
a. Chloride and monochloride\\
b. Sodium and calcium\\
c. Ozone and hydroxide\\
d. Arsenic and manganese\\
\item Free chlorine can only be obtained after point chlorination has been achieved.\\
a. Breakpoint\\
b. Fastpoint\\
c. Softpoint\\
d. Onpoint\\
\item The meaning of the " $\mathrm{C}$ " and the " $\mathrm{T}$ " in the term $\mathrm{CT}$ stands for:\\
a. Concentration and time\\
b. Color and turbidity\\
c. Calcium and tortellini\\
d. Chlorine and turbidity\\
\item Chloramine is most affective as a disinfectant.\\
a. Primary\\
b. Secondary\\
c. Third\\
d. First\\
\item TTHMs and HAA5s can affect:\\
a. Health\\
b. Aesthetics\\
c. Color\\
d. Odor\\
\item The multiple barrier treatment approach includes\\
a. Sterilization and filtration\\
b. Disinfection and filtration\\
c. Disinfection and sterilization\\
d. Infection and filtration\\
\item The maximum disinfectant residual allowed for chlorine in a water system is\\
a. $.02 \mathrm{mg} / \mathrm{L}$\\
b. $2.0 \mathrm{mg} / \mathrm{L}$\\
c. $3.0 \mathrm{mg} / \mathrm{L}$\\
d. $4.0 \mathrm{mg} / \mathrm{L}$\\
\item What is the disinfectant byproduct caused by ozonation?\\
a. Trihalomethanes\\
b. Bromate\\
c. Chlorite\\
d. No DBP formation\\
\item Haloacitic Acids are also known as\\
a. TTHM\\
b. HOCL\\
c. Chlorite\\
d. HAA5\\
\item What is the MCL for trihalomethanes?\\
a. $.10 \mathrm{mg} / \mathrm{L}$\\
b. $.06 \mathrm{mg} / \mathrm{L}$\\
c. $.08 \mathrm{mg} / \mathrm{L}$\\
d. $.12 \mathrm{mg} / \mathrm{L}$\\
\item What is the MCL for Haloacitic Acids?\\
a. $100 \mathrm{ppb}$\\
b. $60 \mathrm{ppb}$\\
c. $80 \mathrm{ppb}$\\
d. $120 \mathrm{ppb}$\\
\item What is the MCL for bromate?\\
a. $.010 \mathrm{mg} / \mathrm{L}$\\
b. $.020 \mathrm{mg} / \mathrm{L}$\\
c. $.030 \mathrm{mg} / \mathrm{L}$\\
d. $.040 \mathrm{mg} / \mathrm{L}$\\
\item What is residual Chlorine?\\
a. Chlorine used to disinfect\\
b. The amount of chlorine after the demand has been satisfied\\
c. The amount of chlorine added before disinfection\\
d. Film left on DPD kit to measure residual\\
\item When Chlorine reacts with natural organic matter in water it can create\\
a. Disinfectant by-products\\
b. Coliform bacteria\\
c. Chloroform\\
d. Calcium\\
\item What are trihalomenthanes classified as\\
a. Salts\\
b. Inorganic compounds\\
c. Volatile organic compounds\\
d. Radio\\
\item What disinfectant is used for emergency purposes and not utilized in the water treatment industry?\\
a. Chlorine\\
b. Iodine\\
c. Ozone\\
d. Chlorine Dioxide\\
\item What is the disinfectant with the least killing power but that has the longest lasting residual?\\
a. Chlorine\\
b. Ozone\\
c. Chlorine Dioxide\\
d. Chloramines\\
\item The active ingredient in household bleach is\\
a. Calcium hypochlorite\\
b. Calcium hydroxide\\
c. Sodium hypochlorite\\
d. Sodium hydroxide\\
\item Cryptosporidium is not resistant to this chemical\\
a. Ozone\\
b. Chlorine Dioxide\\
c. Chlorine\\
d. Both $A \& B$\\
\item If a coliform test is positive, how many repeat samples are required at a minimum?\\
a. None\\
b. 1\\
c. 3\\
d. Depends on the severity of the positive sample\\
\item Your water system takes 75 coliform tests per month. This month there were 6 positive samples. What is the percentage of samples which tested positive? Did your system violate regulations?\\
a. $3 \%$ Yes\\
b. $5 \% \mathrm{No}$\\
c. $8 \%$ Yes\\
d. $10 \% \mathrm{No}$\\
\item The form of Chlorine which is $100 \%$ available chlorine is?\\
a. Sodium Hypochlorite\\
b. Calcium Hypochlorite\\
c. Calcium Hydroxide\\
d. Gaseous Chlorine\\
\item What is the minimum amount of chlorine residual required in the distribution system?\\
a. There is no minimum\\
b. $\mathrm{mg} / \mathrm{L}$\\
c. $0.2 \mathrm{mg} / \mathrm{L}$\\
d. $\mathrm{mg} / \mathrm{L}$\\
\item What is the approximate $\mathrm{pH}$ range of sodium hypochlorite?\\
a. 4-5\\
b. 6-7\\
c. $9-11$\\
d. $12-14$\\
\item What is the typical concentration of sodium hypochlorite utilized in water treatment?\\
a. $5 \%$\\
b. $65 \%$\\
c. $100 \%$\\
d. $12.5 \%$\\
\item Chlorine demand refers to\\
a. Chlorine in the system for a given time\\
b. The difference between chlorine applied and chlorine residual-usually caused by inorganics, organics, bacteria, algae, ammonia, etc.\\
c. Chlorine needed to produce a higher $\mathrm{pH}$\\
d. None of the above\\
\item What is the most effective chlorine disinfectant?\\
a. Dichloramine\\
b. Trichloramine\\
c. Hypochlorite Ion\\
d. Hypochlorous acid\\
\item What can form when chlorine reacts with natural organic matter in source water?\\
a. Disinfectant by-products\\
b. Sulfur\\
c. Algae\\
d. Coliform bacteria\\
\item What kind of solution is used to check for a gas chlorine leak?\\
a. Sodium hydroxide\\
b. Ozone\\
c. Ammonia\\
d. Calcium hypochlorite\\
\item Chlorine is\\
a. Heavier than air\\
b. Lighter than air\\
c. Brown in color\\
d. not harmful to your health\\
\item Chlorine demand may vary due to\\
a. Chlorine demand always stays the same\\
b. Temperature\\
c. $\mathrm{pH}$\\
d. Both B and C\\
\item What effect does high turbidity have on disinfection?\\
a. It can increase chlorine demand\\
b. It has no effect\\
c. It gives the water a milky appearance that will clear out after some time\\
d. You must increase the temperature of the water\\
\item What is the target chlorine:ammonia ratio?\\
a. $2: 1$\\
b. $3: 1$\\
c. $4: 1$\\
d. $5: 1$\\
\item What is the MCL for Nitrates?\\
a. $1 \mathrm{ppm}$\\
b. $10 \mathrm{ppm}$\\
c. $5 \mathrm{ppm}$\\
d. None of the above\\
\item What is the molecular weight of Chlorine?\\
a. 70\\
b. 14\\
c. 65\\
d. 20\\
\item What disinfectant has the longest lasting residual?\\
a. Ozone\\
b. Chlorine\\
c. Chloramine\\
d. Chlorine Dioxide\\
\item What are some of the early indicators of Nitrification?\\
a. Lowering chlorine residual\\
b. Excess ammonia in treated water\\
c. Raise in bacterial heterotrophic plate counts\\
d. All of the above\\
\item What are THMs classified as?\\
a. Turbidity\\
b. Radiological\\
c. Volatile Organic Chemicals\\
d. Salts\\
\item What method can operators employ to combat nitrification?\\
a. Lower residual chlorine target b. Keep reservoir levels static\\
c. Minimize free ammonia in treated water\\
d. Increase water age\\
\item How many times stronger is Chlorine compared to monochloramine?\\
a. 250 times\\
b. 20 times\\
c. 1500 times\\
d. 5 times\\
\item What chemicals are formed when chlorine is mixed with water?\\
a. Hydrogen sulfide and ammonia\\
b. DPD and carbon dioxide\\
c. Sodium hypochlorite and calcium hypochlorite\\
d. Hypochlorous acid and hydrochloric acid\\
\item Chlorine residual is measured in the field using the\\
a. Electroconductivity method\\
b. EDTA titrimetric method\\
c. Ortho-tolidine colorimetric method\\
d. DPD colorimetric method\\
e. Differential $\mathrm{pH}$ method\\
\item In nitrification, bacteria consume excess ammonia in the water and produce\\
a. Chloramines\\
b. Free chlorine\\
c. Urine\\
d. Nitrite\\
e. Sodium thiosulfate\\
\item Which of the following is a form of free chlorine?\\
a. Nitrite\\
b. Hypochlorous acid\\
c. Monochloramine\\
d. Hydrochloric acid\\
e. Trichloramine\\
\item A distribution system operator measures a total chlorine residual of $1.25 \mathrm{mg} / \mathrm{L}$. How many points on the chlorine breakpoint curve may display this residual?\\
a. Zero\\
b. One\\
c. Two\\
d. Three\\
e. Four\\
\item What is the chlorine dosage that must be applied when disinfecting a pipeline using the slug method?\\
a. $300 \mathrm{mg} / \mathrm{L}$\\
b. $\quad 100 \mathrm{mg} / \mathrm{L}$\\
c. $50 \mathrm{mg} / \mathrm{L}$\\
d. $25 \mathrm{mg} / \mathrm{L}$\\
e. $6 \mathrm{mg} / \mathrm{L}$\\
\item Which of the following is a form of combined chlorine?\\
a. Hypochlorite ion\\
b. Hypochlorous acid\\
c. Monochloramine\\
d. Hydrochloric acid\\
e. Free ammonia\\
\item A distribution system operator measures a total chlorine residual of $1.25 \mathrm{mg} / \mathrm{L}$, and a free chlorine residual of $1.15 \mathrm{mg} / \mathrm{L}$ : This indicates that\\
a. The system is operating with a chloramine residual\\
b. The chlorine demand is $0.10 \mathrm{mg} / \mathrm{L}$\\
c. The chlorine demand is $2.40 \mathrm{mg} / \mathrm{L}$\\
d. Chloramines are being destroyed by free chlorine\\
e. The system is operating to the right of the breakpoint on the chloramine curve\\
\item Which of the following is the most desirable form of combined residual chlorine?\\
a. Hypochlorite ion\\
b. Hypochlorous acid\\
c. Monochloramine\\
d. Dichloramine\\
e. Trichloramine\\
\item Of the following, which is the most effective disinfectant?\\
a. Hypochlorite ion\\
b. Hypochlorous acid\\
c. Monochloramine\\
d. Dichloramine\\
e. Trichloramine\\
\item A field chlorine residual measurement shows no reading at one minute, but $2.1 \mathrm{mg} / \mathrm{L}$ after three minutes. This indicates that\\
a. The field DPD test kit needs to be returned to the laboratory for maintenance\\
b. There is no chlorine residual\\
c. There is no free chlorine residual, but there are $2.1 \mathrm{mg} / \mathrm{L}$ of chloramines\\
d. There is no combined residual, but the free chlorine residual is $2.1 \mathrm{mg} / \mathrm{L}$\\
e. The analyst should wait an additional three minutes and re-test\\
\item When disinfecting a storage tank, one method calls for the bottom $6 \%$ of the tank volume to be chlorinated for at least 6 hours with an applied chlorine dosage of a. $50 \mathrm{mg} / \mathrm{L}$\\
b. $25 \mathrm{mg} / \mathrm{L}$\\
c. $6 \mathrm{mg} / \mathrm{L}$\\
d. $4 \mathrm{mg} / \mathrm{L}$\\
e. $\quad 0.2 \mathrm{mg} / \mathrm{L}$\\
\item Residual chlorine refers to\\
a. The amount of chlorine in the chlorinated water after several minutes\\
b. The chlorine needed to disinfect the water supply\\
c. The chlorine needed to produce floc in the water\\
d. The sludge in the bottom of the chlorine solution tank\\
e. None of the above\\
\item While handling sodium hypochlorite, proper safety precautions include:\\
a. Avoiding situations that could splash hypochlorite solution.\\
b. Using a face shield and/or goggles to avoid eye contact.\\
c. Minimizing skin contact with rubber gloves and/or protective clothing\\
d. All of the above e. None of the above are necessary\\
\item The fusible plug that is in all chlorine containers\\
a. Is not necessary\\
b. May be used as a tap for the chlorine source\\
c. Should be removed after the cylinders are empty\\
d. Should never be removed or tampered with\\
e. Should be removed prior to withdrawing chlorine from the container\\
\item Sodium hypochlorite is a a. Compound purchased in liquid solution used for disinfection\\
b. Dry neutralizing powder for treating chlorine burns\\
c. Gas delivered in 100-pound, 150-pound, or one-ton containers\\
d. Salt that is formed when hydrochloric acid is neutralized with caustic soda\\
e. None of the above\\
\item The chlorine demand abruptly jumps in your source water. This may indicate that a. The water source has been contaminated $b$. Flow rates in the distribution system have increased\\
c. The hypochlorite solution used for disinfection has deteriorated\\
d. The hypochlorite solution tank is empty\\
e. The hypochlorite ion has a higher concentration than hypochlorous acid\\
\item The chemical compound typically found in chlorination tablets and granules is\\
a. Sodium hypochlorite\\
b. Sodium hydroxide\\
c. Sodium chloride\\
d. Calcium hypochlorite\\
e. Calcium hydroxide\\
\item The maximum rate of withdrawal of gas from a 150-pound chlorine cylinder in 24-hours is\\
a. 20 pounds\\
b. 40 pounds\\
c. 100 pounds\\
d. $\quad 150$ pounds\\
e. None of the above\\
\item The maximum rate of withdrawal of gas from a one-ton chlorine container in 24-hours is\\
a. 40 pounds\\
b. $\quad 100$ pounds\\
c. 400 pounds\\
d. One ton\\
e. Variable, depending on chlorine dosage requirements\\
\item A chlorine leak can be detected by\\
a. An explosimeter\\
b. Checking the leak gauge\\
c. Applying ammonia solution\\
d. A tri-gas detector\\
e. None of the above\\
\item When using the continuous feed method of disinfection, a new water main should be flushed, disinfected at $50 \mathrm{mg} / \mathrm{L}$, and held at above $25 \mathrm{mg} / \mathrm{L}$ for at least\\
a. 6 hours\\
b. 12 hours\\
*c. 24 hours\\
d. 36 hours\\
e. 48 hours\\
\item If you encounter a liquid chlorine leak in a one-ton container, what action should you take first, to reduce the severity of the leak?\\
a. Apply a caustic solution\\
b. Apply an acidic solution\\
c. Spray the container with water\\
d. Spray the container with an ammonia solution\\
e. Rotate the container to place the leak at the top\\
\item What should the chlorine dosage be to water that has a chlorine demand of $1.5 \mathrm{mg} / \mathrm{L}$, when a free residual of $1.0 \mathrm{mg} / \mathrm{L}$ is desired?\\
a. $\quad 0.5 \mathrm{mg} / \mathrm{L}$\\
b. $\quad 1.0 \mathrm{mg} / \mathrm{L}$\\
c. $\quad 1.5 \mathrm{mg} / \mathrm{L}$\\
d. 2.5 pounds per day\\
e. $2.5 \mathrm{mg} / \mathrm{L}$\\
\item When chlorine reacts with natural organic matter in the water, it is possible to form\\
a. Disinfection by-products\\
b. Arsenic\\
c. MTBE\\
d. Coliforms e. Synthetic organic compounds\\
\item Which of the following best describes the characteristics of chlorine when used for disinfection in drinking water?\\
a. Colorless, flammable, heavier than air\\
b. Greenish-yellow, nonflammable, lighter than air\\
c. Greenish-yellow, flammable, lighter than air\\
d. Greenish-yellow, nonflammable, heavier than air\\
\item Killing of pathogenic organisms in water treatment is called\\
a. Disinfection\\
b. Oxidätion\\
c. Pasteurization\\
d. Sterilization\\
\item Chlorine reacts with nitrogenous compounds to form\\
a. Ammonia nitrate\\
b. Free chlorine\\
c. Chlorinated hydrocarbons\\
*d. Chloramines\\
\item Sodium Hypochlorite is\\
a. A commercially available chlorine solution\\
b. A commercially available dry chlorine compound\\
c. Chlorine that is available in 100- and 150-pound cylinders\\
d. A reaction product of chlorine and caustic soda\\
\item A hypochlorinator is\\
a. Used to measure residual chlorine\\
b. Used in the treatment of iron and turbidity\\
c. Used to feed a liquid solution into a water supply\\
d. Used to measure an adequate amount of chlorine gas into the supply\\

\item When calcium hypochlorite is used for disinfecting a water supply, it should be\\
a. Dissolved in water, allowed to settle, and the supernatant siphoned off and fed into the water system\\
b. Dissolved in water as a dry chemical then injected into the water system\\
c. Fed as a dry chemical directly into the pipeline\\
d. Fed as a dry powder into the clear well\\
\item The chlorine gas feed rate is usually controlled by adjusting the\\
a. water flow to the injector\\
b. valve on the chlorine cylinder\\
c.pressure in the chlorine cylinder\\
d. rotameter control valve\\
\item If disinfection is incomplete because the chlorine residual is in the hypochlorite ion form, what should you change to improve disinfection?\\
a. Calcium\\
b. Hardness\\
c. $\mathrm{pH}$\\
d. alkalinity\\
\item Breakpoint chlorination is achieved when\\
a. Free ammonia can be tasted in the water\\
b. No chlorine residual is detected\\
c. The strong chlorine tasted at the plant did not persist in the distribution system\\
d. When chlorine dosage is increased, a corresponding increase in residual is detected\\
\item Because chlorine residual is related to the $\mathrm{pH}$ of the water, it may be said that\\
a. A higher $\mathrm{pH}$ requires a higher chiorine residual\\
b. A higher $\mathrm{pH}$ requires a lower chlorine residual\\
c. A lower $\mathrm{pH}$ requires a higher chlorine residual\\
d. $\mathrm{pH}$ has no effect on chlorine residual\\
\item As long as the temperature is steady, the pressure indicator on a chlorine cylinder will until all the chlorine has been gasified\\
a. Remain steady\\
b. Decrease slowly\\
c. Decrease rapidly\\
d. Increase slightly\\
\item When fresh, the typical concentration of sodium hypochlorite solution is\\
a. $1.25 \%$\\
b. $6.5 \%$\\
c. $12.5 \%$\\
d. $65 \%$\\
e. variable, depending on the manufacturer\\
\item Chlorine in a dry form is called:\\
a. hypochlorite\\
b. hypochlorous\\
c. hydrochlorite\\
d. hydroxide\\
\item Which of the following procedures is done when preparing to disconnect a chlorine cylinder?\\
a. close the cylinder valve first to allow time for the chlorine to be drawn off\\
b. loosen the line to the tank and then shut off the valve to the chlorine cylinder\\
c. shut off the water supply and allow sufficient time for the chlorille to be drawn off\\
d. tum the chlorinator feed rate valve off then turn the valve on the chlorinator cylinder\\
\item A vacuum is formed in the chlorinator by the:\\
a chlorine cylinder pressure\\
b. pressure differential through the ejector\\
c. chlorine feed pump\\
d. rotameter-\\
\item When calcium hypochlorite is used for disinfecting a water supply, it should be be:\\
a. Dissolved in water, allowed to settle, and the supernatant siphoned off and fed into the water system\\
b. Dissolved in water as a dry chemical then injected into the water system\\
c. Fed as a dry chemical directly into the pipeline\\
d. Fed as a dry powder into the clear well\\
\item Because chlorine residual is related to the $\mathrm{pH}$ of the water, it may be said that: a. A higher $\mathrm{pH}$ requires a higher chlorine residual\\
b. A higher $\mathrm{pH}$ requires a lower chlorine residual\\
c. A lower $\mathrm{pH}$ requires a higher chlorine residual\\
d. A lower $\mathrm{pH}$ has no effect on chlorine residual\\

\item Which of the following best describes "chlorine demand"?\\
a. The difference between the amount of chłorine added and turbidity\\
b. The difference between the amount of chlorine added and $\mathrm{pH}$\\
c. The difference between the total chlorine residual and the free chlorine residual\\
(d.) The difference between the amount of chlorine added and the amount of residual chlorine remaining after a given contact time\\
\item When two ton cylinders are feeding gas and one of them is frosted, what might be the problem?\\
a. The feed rate is too high\\
b. The line on the frosted tank is clogged\\
c. The valve on the unfrosted tank\\
d. The injector is clogged\\
\item There is low vacuum on the system and the flow rate is low when the rate valve is wide open, what is the problem?\\
a. The feed rate is too high\\
b. The injector is clogged\\
c. There is a clogged feed line\\
d. The rotameter is clogged 
\item If ammonia vapor is passed over a chlorine leak in a cylinder valve, the presence of the leak is indicated by a\\
a. Yellow cloud\\
b. White cloud\\
c. Gray cloud\\
d. Brown cloud
 \item When chlorine is used as a disinfectant in water there reaches a point when the amount of chlorine added is reflected identically with the amount of free residual measured on your DPD\\
a) chloramination\\
*b) breakpoint\\
c) ozone\\
d) liftoff\\
 \item Chlorine gas is times than air\\
a) 2.5 , lighter\\
b) 4.5 , heavier\\
c) 3.5 , lighter\\
*d) 2.5, heavier\\
  \item Which disinfection method provides a residual safeguard?\\
a) ozonation\\
*b) chlorination\\
c) membrane filtration\\
d) ultraviolet radiation\\
  \item Which is the most effective disinfectant when chlorine is added to water?\\
a) hydrogen lon\\
b) calcium dioxide\\
*c) hypochlorous acid\\
d) haloacetic acid\\
  \item When chlorine reacts with organics in the water it has the tendency to produce\\
a) chloramines\\
*b) trihalomethanes and haloacetic acids\\
c) macrofloc\\
d) apparent color\\

\item Which type of flow meter should be used for chlorinators which require very low flow\\
a.  Rotameter\\
b.  Loss of head meter\\
*c.  Positive displacement meter\\
d.  Magnetic meter\\
\item The device used to regulate and control the rate of feed of chlorine gas is called a\\
a.  Pump regulator\\
b.  Rotarneter\\
c.  Gas regulator\\
*d.  Chlorinator\\
\item Disinfection by products can be formed by the reaction of disinfectants with\\
a.  Disinfection by products\\
b.  Dissolved solids\\
c.  Chemicals\\
*d.  Natural organic matter\\
\item The best way to determine how much chlorine is in a chlorine cylinder is to\\
a.  Shake the cylinder\\
b.  Measure the internal pressure\\
c. Measure the cylinder residual  d\\
d.  Weigh the cylinder\\
\item The fusible plugs on a chlorine container are designed to melt and release chlorine when the container reaches which temperature?\\
*a.  158° to 165°F\\
b.  158° to 165°C\\
c.  212° to 220°f\\
d.  100° to 105°C\\

\item The disinfection is incomplete because the chlorine residual is in the hypochlorite ion form, what should you change to improve disinfection?\\
a.  Calcium\\
b.  Hardness\\
*c.  pH\\
d. Total alkalinity\\

 \item Which is the primary drawback for facilities that use ultraviolet light to disinfect water?\\
a. It does not inactivate all microorganisms\\
b. It has the potential to produce trihalomethanes\\
c. Dissolved colloids can shield microorganisms from the UV light\\
*d. There is potential for the light bulbs to be coated with light-obscuring material, preventing the UV light from killing microorganisms \\

\item Potassium permanganate is most effective in\\
a. color removal.\\
b. control of biological growth.\\
c. control of trihalomethanes formation potential.\\
*d. removing iron.\\

\item Chlorine is advantageous over chloramines in that chlorine\\
*a. is a much stronger oxidant.\\
b. has long history of use.\\
c. has simple feeding.\\
d. has a persistent residual.\\

  \item Which oxidant has the potential of producing $\mathrm{ClO}_{3}$ by-products?\\
*a. Chlorine dioxide\\
b. Chlorine\\
c. Chloramines\\
d. Calcium hypochlorite\\

  \item Which is the working strength of sodium hydroxide solution in a chlorine neutralization tank?\\
a. $15 \%$\\
*b. $20 \%$\\
c. $25 \%$\\
d. $30 \%$\\

  \item Water treatment plants are increasingly using hypochlorination because it is\\
*a. relatively safe compared to using gaseous chlorine.\\
b. very inexpensive compared to other disinfectants.\\
c. a stronger oxidant than gaseous chlorine.\\
d. not going to form trihalomethanes.\\

 \item A single chlorine cylinder is delivering $48 \mathrm{lb} / \mathrm{d}$ of chlorine to the water process, causing the cylinder to form a little frost. Which would be the best solution to this problem?\\
a. Install a fan to improve air circulation\\
b. Heat the cylinder immediately below the valve with heat tape\\
c. Heat the valve only\\
*d. Add another cylinder and feed from both\\


  \item The pressure-reducing and shutoff valve on a vaporizer will shut off when there is a/an\\
*a. loss of electrical power.\\
b. high water level.\\
c. high water temperature.\\
d. over-pressurization of the vaporization system. 

  \item When chlorine gas is added to water the $\mathrm{pH}$ goes down due to\\
a. chlorine gas producing caustic substances\\
b. two base materials that form\\
*c. *two acids that form\\
d. caustic soda being formed in the water 

\item Taste and odor from Phenolic compounds are \rule{1.5cm}{0.5pt} by chlorinatiqn.\\
*a. Increased\\
b. Decreased\\
c. Not affected\\

\item When certain organic precursors (humic and fulvic acids) and chlorine combine together in water during disinfection, by products can be formed these by products are called\\
a.	Nitrate\\
b.	Arsenic\\
*c.	Trihalomethanes\\
d.	Trichloroethylene\\

\item When should ammonia be added to the water when making disinfectant chloramines for secondary disinfection?\\
a.	Before chlorine addition\\
*b.	After chlorine addition\\
c.	During filtration\\
d.	Before filtration\\


\item Which of the following substances is the most effective disinfection residual?\\
a.	Trichloramine\\
b.	Hypochlorous acid\\
c.	Chloramine\\
d.	Hypochlorite  ion
\end{enumerate}
\newpage

