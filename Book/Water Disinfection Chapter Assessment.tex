\chapterimage{QuizCover} % Chapter heading image

\chapter*{Chapter 6 Assessment}
% \textbf{Multiple Choice}
\section*{Chapter 6 Assessment}
\begin{enumerate}[1.]

\item The two most common types of chlorine disinfection by-products include:\\
a. TTHM and HAA5\\
b. TTHA of HMM5\\
c. Turbidity and color\\
d. Chloride and fluoride\\
\item Chlorine gas is times heavier than breathing air\\
a. 2.5\\
b. 20\\
c. 60\\
d. 460\\
\item A commonly used method to test for chlorine residual in water is called the method.\\
a. HTH\\
b. THM\\
c. VOC\\
d. DPD\\
\item When chlorine gas is added to water the pH goes down due to\\
a. chlorine gas producing caustic substances\\
b. two base materials that form\\
c. two acids that form\\
d. caustic soda being formed in the water\\
\item Disinfection by-products are a product of:\\
a. Filtration\\
b. Disinfection\\
c. Sedimentation\\
d. Adsorption\\
\item Chloramine is most effective as a disinfectant.\\
a. Primary\\
b. Secondary\\
c. Third\\
d. First\\
\item Name two methods commonly used to disinfect drinking water other than chlorination.\\
a. Ozone and ultraviolet light\\
b. Soap and agitation\\
c. Filtration and adsorption\\
d. Salt and vinegar\\
\item In order to determine the effectiveness of disinfection, it is desirable to maintain a disinfectant residual of at least $\mathrm{mg} / \mathrm{L}$ entering the distribution system.\\
a. 0.10\\
b. 0.5\\
c. 0.3\\
d. 0.2\\
\item Secondary disinfectants are used to provide a in the distribution system.\\
a. Color\\
b. Chemical\\
c. Smell\\
d. Residual\\
\item Primary disinfectants are used to microorganisms.\\
a. Hurt\\
b. Inactivate\\
c. Burn up\\
d. Evaporate\\
\item The quantity of chlorine remaining after primary disinfection is called a residual.\\
a. Chlorine\\
b. Permaganate\\
c. Hot\\
d. Cold\\
\item The two most common types of chlorine disinfection by-products include:\\
a. TTHM and HAA5\\
b. TTHA of HMM5\\
c. Turbidity and color\\
d. Chloride and fluoride\\
\item In order to determine the effectiveness of disinfection, it is desirable to maintain a disinfectant residual of at least $\mathrm{mg} / \mathrm{L}$ entering the distribution system.\\
a. 0.10\\
b. 0.5\\
c. 0.3\\
d. 0.2\\
\item A \rule{1.5cm}{0.5pt} residual of chlorine is required throughout the distribution system.\\
a. Large\\
b. High\\
c. Trace\\
d. Hot\\
\item The test used to determine the effectiveness of disinfection is called the:\\
a. Coliform bacteria test\\
b. Color test\\
c. Turbidity test\\
d. Particle test\\
\item Name two methods commonly used to disinfect drinking water other than chlorination.\\
a. Ozone and ultraviolet light\\
b. Soap and agitation\\
c. Filtration and adsorption\\
d. Salt and vinegar\\
\item Name the two types of hypochlorites used to disinfect water.\\
a. Chloride and monochloride\\
b. Sodium and calcium\\
c. Ozone and hydroxide\\
d. Arsenic and manganese\\
\item Free chlorine can only be obtained after \rule{1.5cm}{0.5pt} chlorination has been achieved.\\
a. Breakpoint\\
b. Fastpoint\\
c. Softpoint\\
d. Onpoint\\
\item The meaning of the " C" and the " T " in the term CT stands for:\\
a. Concentration and time\\
b. Color and turbidity\\
c. Calcium and tortellini\\
d. Chlorine and turbidity\\
\item Chloramine is most affective as a disinfectant.\\
a. Primary\\
b. Secondary\\
c. Third\\
d. First\\
\item TTHMs and HAA5s can affect:\\
a. Health\\
b. Aesthetics\\
c. Color\\
d. Odor\\

\end{enumerate}



