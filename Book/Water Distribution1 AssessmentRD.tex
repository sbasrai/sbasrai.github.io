%\chapterimage{QuizCover} % Chapter heading image
%\chapter*{Chapter Assessment}
%\textbf{Multiple Choice}
\begin{enumerate}[1.]
\item What is the reason for keeping adequate, reliable records in a treatment plant?\\
a. *to record the plant's effectiveness and because of requirements by regulatory agencies\\
b. to maintain records for cold cases\\
c. in case the IRS wishes to check files for due diligence\\
d. because of homeland security issues and files being available to the public\\
\item Which statement about displacement meters is not correct:\\
a. The most common type of water service meter is the displacement type\\
b. Displacement meters are accurate at low flows\\
c. Excess sediment can cause the meter to stop registering\\
d. *Displacement meters have little head loss due to friction\\
e. Displacement meters operated at a rate in excess of its stated capacity can result in excessive wear\\
\item A fire hydrant should be closed slowly to avoid:\\
a. Excessive wear\\
b. *Water hammer\\
c. Excessive head loss\\
d. Injury to operator\\
\item The minimum separation between municipal water mains and sanitary sewers for installation in a common trench shall be:\\
a. 5 feet horizontal separation\\
b. $* 10$ feet horizontal separation\\
c. 15 feet horizontal separation\\
d. 25 feet horizontal separation\\
\item To properly disinfect a water main after new construction, you should:\\
a. *apply $50 \mathrm{mg} / \mathrm{l}$ chlorine for 24 hours.\\
b. clean the pipe out' with a pig and then disinfect at $10 \mathrm{mg} / 1$ for 24 hours\\
c. use a $10 \%$ solution of calcium chloride\\
$\mathrm{d}$ don't use them main for one week\\
\item When using a dry-barrel fire hydrant, the valve:\\
a. should never be opened completely\\
b. be opened only during the hours of 8AM to 5PM\\
c. be opened to the desired amount of flow\\
d. be opened all the way\\
\item The primary reason for dry barrel-fire hydrants is to:\\
a. allow easy maintenance\\
b. prevent water hammer\\
c. *keep the hydrant from freezing\\
d. keep the barrel from rusting\\
\item A centrifugal pump should not be run empty except momentarily because:\\
a. a serious counter pressure could develop and damage the pump casing. b. it is a waste of energy to run a pump without water.\\
c. the excessive end thrust of the shaft would damage the thrust bearing.\\
d. *the parts lubricated by water could be damaged.\\
\item Pipes of dissimilar metal should not be connected together because of problems due:\\
a. to scale formation\\
b. *corrosion\\
c. water hammer\\
d. the venturi effect\\
\item Which type of valve will prevent the collapse of a pipe?\\
a. Pressure-relief valve\\
b. Needle valve\\
c. Pinch valve\\
d. *Air-and-vacuum relief valve\\
\item The correct protective methods for backflow-prevention devices in order of decreasing effectiveness are\\
a. air gap, VB, RPZ, and DCVA.\\
b. air gap, VB, DCVA, and RPZ.\\
c. air gap, RPZ, VB, and DCVA.\\
d. *air gap, RPZ, DCVA, and VB.\\
\item The $\mathrm{C}$-value is a measure of a pipe's wall\\
a. smoothness.\\
b. smoothness giving even flow.\\
c. smoothness that retards turbulent flow.\\
d. *roughness that retards flow due to friction.\\
\item Which one of the following is a type of joint for ductile iron piping?\\
a. Expansion joint\\
b. *Push-on joint\\
c. Bell and spigot with rubber o-ring\\
d. Rubber gasket joint\\
\item Water hammer can be described as\\
a. particle waves.\\
b. *acoustic waves.\\
c. rogue waves.\\
d. longitudinal waves.\\
\item Which thrust control is easy to use, especially in locations where existing utilities or structures are numerous?\\
a. *Restraining fittings\\
b. Tie rods\\
c. Thrust anchors\\
d. Thrust blocks
The backfill material for a pipe installation should contain enough to allow for thorough compaction.\\
a. moisture\\
b. *sand\\
c. gravel\\
d. mixed sizes\\
\item Thrust from a water surge almost always acts pushes against. to the inside surface that it\\
a. *vertically\\
b. horizontally\\
c. perpendicular\\
d. vertically and horizontally\\
\item The breaking of a buried pipe when it is unevenly supported is called\\
a. stress breakage.\\
b. shear breakage.\\
c. *beam breakage.\\
d. flexural breakage.\\
\item Compression fittings used with copper or plastic tubing seal by means of a\\
a. *beveled sleeve.\\
b. compression ring.\\
c. compressed beveled gasket.\\
d. compressed o-rings located at either end of the fitting's beveled neck.\\
\item Which should be installed at a dead-end water main?\\
a. Vacuum valve\\
b. Air valve\\
c. *Blowoff valve\\
d. Water quality sampling station\\
\item First draw samples for the analysis of lead and copper water must be collected from taps where the water has stood motionless in the plumbing for at least\\
a. 4 hours.\\
b. 6 hours.\\
c. 8 hours.\\
d. $* 24$ hours.\\
\item According to AWWA Standard C651, disinfection of water mains requires 24-hour exposure to which minimum free chlorine residual?\\
a. $10 \mathrm{mg} / \mathrm{L}$\\
b. $* 25 \mathrm{mg} / \mathrm{L}$\\
c. $50 \mathrm{mg} / \mathrm{L}$\\
d. $100 \mathrm{mg} / \mathrm{L}$\\
\item The tensile strength of a pipe is its ability to\\
a. *Stretch or pull without breakage\\
b. Resist internal pressure without breakage\\
c. Resist external pressure without breakage\\
d. Twist or bend without breakage\\
e. Resist heating without breakage\\
\item The lowest point of the inside of a pipe is known as the\\
a Pervert\\
b. Soffit\\
c. *Invert\\
d. Curb stop\\
e. None of the above\\
\item A lightweight type of pipe that has a very smooth interior, is essentially corrosion-free, and which is difficult to locate when buried is\\
a. *Polyvinyl chloride\\
b. Cast iron\\
c. Ductile iron\\
d. Concrete cylinder\\
e. Steel\\
\item An example of a pipe material that is relatively easy to locate underground is\\
a. ABS\\
b. PVC\\
c. Polyethylene\\
d. *Reinforced concrete cylinder\\
e. Asbestos-cement\\
\item \rule{1.5cm}{0.5pt} is a type of valve typically found in a storage tank of a water distribution system it closes to prevent the storage tank from overflowing when a pre-set level is reached\\
a. Ball valve\\
b. Altitude valve\\
c. Gate valve\\
d. Spring valve\\
\item \_\_\_ is a valve which opens by lifting a round or rectangular gate/ wedge out of the path of the fluid are designed to fully open or closed service\\
a. Ball valve\\
b. Spring valve\\
c. Altitude valve\\
d. Gate valve\\
\item A \_ \_ is a form of quarter turn valve which uses a hollow perforated and pivoting to control flow through it and is a pivoted 90 degrees by the valve handle.\\
a. Gate valve\\
b. Spring valve\\
c. Ball valve d. d. Altitude valve\\
\item The sudden closure of a check valve will result in\\
a. water hammer\\
b. flow reversal\\
c. cavitation\\
d. water aeration\\
\item A \_\_\_ located at the bottom end of suction pipe on a pump this valve opens when the pump operates to allow water to enter the suction pipe but closes when the pump shuts off water from flowing out of the suction pipe\\
a. Check valve\\
b. Foot valve\\
c. Spring valve\\
d. Ball valve\\
\item A valve that automatically shuts off flow into an elevated storage tank when the water level in the tank reaches a preset level is termed a(n)\\
a. Gate valve\\
b. Air/ vacuum relief valve\\
c. Wet-barrel hydrant\\
d. Altitude valve\\
e. Angle valve\\
\item A normally buried valve located on a street water main and leading to a water service is known as\\
a. Check valve\\
b. Gate valve\\
c. Corporation stop\\
d. Altitude valve\\
e. Butterfly valve\\
\item The risk of pipeline damage from water hammer can be reduced by\\
a. Installation of gate valves\\
b. Air release valves\\
c. Repair of defective pipes\\
d. Trimming pump impellers\\
e. Rapid closing of pump discharge valves\\
\item The proper location for air relief valves is\\
a. At low points along a pipeline\\
*b. At high points along a pipeline\\
c. At the bottom of surge tanks\\
d. At the mid-line of water storage reservoirs\\
e. At the springline of a pipeline\\
\item When fully open, which of the following will have the highest friction loss?\\
a Gate valve\\
b. Butterfly valve\\
c. Globe valve\\
d. Ball valve\\
e. All will have about the same friction loss.\\
\item A nutating disc is found in certain:\\
a. Centrifugal pumps\\
b. Positive displacement pumps\\
c. Main line valves\\
d. Chemical feeder\\
*e. Water meters\\
\item The drain hole in a fire hydrant is designed to\\
a. Release air upon closing the valve\\
b. Relieve vacuum upon opening the valve\\
c. Allow access for interior inspection\\
d. Relieve excess water. pressure when closing the valve\\
*e. Remove water from the riser to prevent freezing\\
\item A main break may cause low pressure in the distributions system, which in turn may result in\\
a. Contamination of the system by backsiphonage\\
b. "ice" formation in the pipes\\
c. Increase in chlorine residual\\
d. Water hammer\\
\item Check valves are used to prevent\\
a. Excessive pump pressure\\
b. Priming\\
c. Water from flowing in two directions\\
d. Water hammer
\item To protect stored water from contamination, a ground storage reservoir should\\
a. Be totally airtight\\
b. Have both the overflow pipe and vent screened\\
c. Have cathodic protection\\
d. Have its interior surface coated with an AWWA-approved paint system\\
\item The least amount of head loss in a pipeline would be caused by a fully open\\
a. Angle valve\\
b. Check valve\\
c. Gate valve\\
d. Globe valve\\
\item The variation in water demand during the course of a day is termed\\
a. Seasonal variation\\
b. Fire flow requirements\\
c. Emergency storage variation\\
d. The straight line equalization method\\
e. Diurnal variation\\
\item The maximum momentary load placed on a water supply system is known as\\
a. Average daily flow\\
b. Average daily demand\\
c. Rated capacity\\
d. System float\\
e. Peak demand\\
\item Elevated storage tanks are used primarily to\\
a. Eliminate the need for continuous pumping\\
b. Minimize variations in the system water pressures\\
c. Reduce auxiliary power requirements\\
d. Provide a considerable amount of water for storage\\
e. Protect against backflows\\
\item Because pipe materials come into contact with drinking water, they must conform with\\
a. Primary drinking water standards\\
b. Secondary drinking water standards\\
c. Surface water treatment rule\\
*d. ANSI/NSF Standard 61\\
e. All of the above\\
\item Pipe with a " C " factor of 140 is regarded as having $a(n)$\\
a. Extremely smooth interior\\
b. Extremely rough interior\\
c. Extremely high corrosion resistance\\
d. Extremely low corrosion resistance\\
e. A purple color\\
\item If possible, a water main leak should be repaired under pressure to\\
a. Prevent contamination of the water line\\
b. Prevent flooding of basements\\
c. Save repair time\\
d. Use fewer materials\\
e. All of the above\\
\item An system for the prevention of corrosion is called\\
a. Water hammer\\
b. Reverse osmosis\\
c. Diurnal variation\\
d. A foot valve\\
e. Cathodic protection\\
\item What category of meters is exemplified by propeller and turbine types?\\
a. Differential pressure\\
b. Positive displacement\\
c. Mass flow\\
d. Velocity\\
\item The hydraulic grade line in a pipeline is normally determined by\\
a. Reading pressure gauges\\
b. Checking for backflow\\
c. Opening fire hydrants on each loop of the system\\
d. Using a leak detector\\
e. A venturi meter\\
\item The slope of the hydraulic grade line is due to\\
a. Well elevations\\
b. Elevations of storage facilities\\
c. Pumping\\
d. Backflows\\
e. Friction loss\\
\item A venturi is a device used to\\
a. Increase water flow\\
b. Decrease water flow\\
c. Regulate water flow\\
d. Stop or start water flow\\
e. Measure water flow\\
\item The most commonly used meter on small diameter domestic service is the\\
a. Venturi meter\\
b. Propeller meter\\
c. Orifice plate meter\\
d. Compound meter\\
e. Nutating disc meter\\
\item The valve type most commonly used for isolation in a water distribution system is the\\
a. Gate valve\\
b. Air relief valve\\
c. Globe valve\\
d. Ball valve\\
e. Butterfly valve\\
\item Which of the following is a device used to measure flow?\\
a. Baffle\\
b. Diversion box\\
c. Stop logs\\
d. Weir\\
e. None of the above\\
\item A compound meter is a device which\\
a. Is installed to allow automated meter reading\\
b. Can be installed to measure water use by as many as 12 separate customers\\
c. Provides accurate readings over a wide range of flows\\
d. Electronically records peak flows, as a demand meter does for electricity\\
e. Is a typical residential water flow meter\\
\item Magnetic flow meters and ultrasonic flow meters are well suited to measure flow rates of water with a large concentration of suspended solids, because they have\\
a. The best accuracy of any meters\\
b. No parts within the flow stream\\
c. Easily accessed cleanout ports\\
d. Simple recalibration procedures\\
e. All of the above\\
\item The most common valve in a water distribution system is the\\
a. Gate valve\\
b. Air relief valve\\
c. Globe valve\\
d. Ball valve\\
e. Butterfly valve\\
\item An example of a pressure-differential type water meter is a\\
a. Venturi meter\\
b. Propeller meter\\
c. Nutating disk meter\\
d. Magnetic flow meter\\
\item An abnormal flow condition caused by a difference in water pressures is known as:\\
a. Backflow\\
b. Reverse osmosis\\
c. Peak demand\\
d. Fire flow\\
e. Minimum daily requirement\\
\item "Backflow Device" is a term used to describe a device that\\
a. connects three inlet lines with one outlet line\\
b. lets air into valve vaults\\
c. prevents flow of potentially contaminated source into a drinking water supply\\
d. tests for oxygen deficiency in valve vaults\\
e. prevents backflow of water through an out-of-service pump\\
\item A cross-connection means\\
a. Four pipelines tied together\\
b. A T-shaped tool\\
c. A connection between potable water and "unapproved" water supplies\\
d. A backflow caused by negative pressure\\
e. A connection between two or more pressure zones\\
\item Egress is normally required (per OSHA guidelines) for trenches of what minimum depth?\\
a. 4 feet\\
b. 5 feet\\
c. 6 feet\\
d. 7 feet\\
e. 8 feet\\
\item A backflow prevention device that can be used in any cross-connection situation is a\\
a. Pressure vacuum breaker\\
b. Single check valve\\
c. Double check valve\\
d. Reduced pressure zone device\\
e. Atmospheric vacuum breaker\\
\item A backflow prevention device that is designed for intermittent use in situations where there is no backpressure, such as toilet flush valves and lawn sprinkler systems is a\\
a. Pressure vacuum breaker\\
b. Single check valve\\
c. Double check valve\\
d. Reduced pressure zone device\\
e. Atmospheric vacuum breaker\\
\item Two hydraulic conditions can induce backflow. These are backsiphonage and\\
a. Peak flow\\
b. Diurnal flow\\
c. Faulty solenoid valves\\
d. Back pressure\\
e. Fire flow\\
\item From a sanitary standpoint. the pressure in a distribution system should never be allowed to fall to zero because:\\
a. low pressure allows bacteria to multiply\\
b. ground water may e:oter and back siphonage may occur\\
c. the chlorine residual will drop faster\\
d. the main may collapse\\
\item The primary purpose of pressure-reducing valves between water system pressure zones is to\\
a. Minimize surge\\
b. Reduce downstream pressure\\
c. Control flows\\
d. Reduce upstream pressure\\
\item An example of a pipe material that is difficult to locate underground is\\
a. Mortar lined and coated steel\\
b. Reinforced concrete cylinder\\
c. Ductile iron\\
*d. Asbestos-cement\\
e. Steel\\
\item Sleeve-type and "victaulic" couplings are the most common forms of\\
*a. Mechanical couplings\\
b. Welded joints\\
c. Asbestos-cement pipe fittings\\
d. PVC pipe fittings\\
e. Flanged joints\\
\item A typical installation site for a compound meter is\\
a. Any small commercial business\\
b. A common single location with as many as 12 separate customers\\
c. A large industrial user\\
d. Any location that requires the electronic monitoring of peak flows e. A typical residential water flow meter\\
\item An example of a pressure-differential type water meter is a:\\
a. Venturi meter b. Propeller meter c. Nutating disk meter d. Magnetic flow meter e. Ultrasonic flow meter\\
\item When closing a hydrant, it should be\\
a. Closed rapidly to minimize water loss\\
*b. Closed slowly to reduce surges\\
c. Closed using a standard valve key\\
d. Closed using a standard pipe wrench\\
e. Closed at the street valve and left slightly open at the hydrant valve\\
\item Dry-barrel fire hydrants have their operating valves\\
*a. In the base\\
b. In the head\\
c. Either of the above, depending on the manufacturer\\
d. In the street several feet away from the riser\\
e. None of the above\\
\item An example of a valve that has a 90 degree travel is a\\
a. Butterfly valve\\
b. Plug valve\\
c. Ball valve\\
d. All of the above\\
e. None of the above\\
\item The valve type most commonly found on the discharge of a pump or well, and installed to prevent reverse flows is the\\
a. Gate valve\\
b. Check valve\\
c. Globe valve\\
d. Butterfly valve\\
e. Ball or Plug valve\\
\item Features that impact the " $\mathrm{K}$ " factor for measuring friction in pipelines include\\
a. Pipe length\\
b. Pipe type\\
c. Number of valves\\
d. Type of valves\\
e. All of the above\\
\item A completely fail-safe means of backflow prevention is\\
a. Atmospheric vacunm breaker\\
b. Pressure vacuum breaker\\
c. Air gap\\
d. Check valve\\
e. Double check valve\\
\item Back-siphonage is defined as:\\
a. Back flow that occurs when a vacuum exists.\\
b. Increase in pressure.\\
c. Interconnection between the plumbing systems in the building and water supply.\\
d. Open end of a water supply through which water is discharged in the plumbing fixture.\\
\item A venturi tube increases the velocity and decreases the pressure as water flows through it, This type of tube is used to measure the: .\\
a. Amount of chlorine in the water.\\
b. Amount of turbidity in the water.\\
c. Rate of aeration.\\
d. Rate of water flowing through it.\\
\item A venturi meter measures flow of a fluid in a pipe based upon the:\\
a. Difference in pressure between a constricted and a fill size portion of the pipe,\\
b. Electronic measurement\\
c. Velocity of the fluid past a given point.\\
d. Weight of the fluid\\
\item Valves are provided in a distribution system to\\
a. Detect any safety hazards.\\
b. Detect weak links in the system.\\
c. Isolate small areas for maintenance and emergency conditions.\\
d. Reduce costs of maintenance.\\
\item A connection that is made into a main that is under pressure is called a:\\
a. Cross connection\\
b. Dry Tap\\
c. Wet Tap\\
d. Valve Box\\
\item Because it permits flow in only one direction, which valve would help you determine the direction of the fluid flow?\\
a. Butterfly valve\\
b. ${ }^{*}$ Check Valve\\
c. Pressure valve\\
d. Gate valve\\
\item The size of water mains, pumping stations, and storage tanks is primarily determined by:\\
a. Maximum day demand during a $24 \mathrm{hr}$. period during the previous year.\\
b. Population served\\
c. Per-capita water use\\
d. Fire protection requirement\\
\item Firefighting may cause low pressure in an area of the distribution system. This low pressure might lead to:\\
a. contamination of the system by back-siphonage\\
b. ice formation in the pipes\\
c. loss of chlorine residual\\
d. None of the above\\
\item The problem caused by dissolved carbon dioxide in the water of the distribution system is b. Corrosion\\
c. Excessive encrustation\\
d. Tastes and odors\\
a. increased trihalomethanes (THMs)\\
\item The peak capacity of water mains is often reduced by\\
a. High pressure\\
b. Looping\\
*c. Tuberculation\\
d. Vacuum breakers\\
\item When using the $A W W A$ spray method for disinfecting the interior walls of water tanks, the minimum applied chlorine dose is\\
a. $5 \mathrm{ppm}$\\
b. $50 \mathrm{ppm}$\\
c. $10 \mathrm{ppm}$\\
d. $200 \mathrm{ppm}$\\
\item Water should be delivered with a minimum working pressure of:\\
a. $45 \mathrm{psi}$\\
a. $100 \mathrm{psi}$\\
b. $35 \mathrm{psi}$\\
c. $50 \mathrm{psi}$\\
d. $15 \mathrm{psi}$
\item Thrust blocks are installed to\\
a. boost flexible joints.\\
b. boost water pressure.\\
c. minimize corrosion\\
d. prevent movement of pipes \& joints.\\
\item Distribution system pressure (even during fire fighting demands) should not be allowed to drop below psi.\\
a. 0\\
b. 5\\
c. 20\\
d. 40\\
\item Whenever possible the end of a distribution system should be to prevent taste and odor problems.\\
a. inspected\\
b. looped.\\
c. plugged\\
d. capped\\
\item The three common types of plastic pipes are listed as PVC, PE, \& PB. These names refer to the:\\
a. Chemical resistance of the pipe\\
b. Composition of the pipe\\
c. Pressure for which the pipe is designed\\
d. Types of appropriate application\\
\item An invert of a pipe is located:\\
a). According to the pipe manufacturers specifications\\
b. At the inside bottom of the pipe\\
c. At the inside cross section\\
d. At the outside bottom of the pipe\\
\item An Altitude valve is a device used to:\\
a. turn water flow off or on\\
b. allow two or more pumps to alternate operation\\
c. prevent backflow due to a cross connection\\
d. regulate the water surface level in a water storage tank\\
e. none of the above\\
\item The type of corrosion caused by the use of dissimilar metal in a water system is\\
a. Caustic corrosion\\
b. Galvanic corrosion\\
c. Oxygen corrosion\\
d. Tubercular corrosion\\
item The best way to protect the water supply from contamination by cross-connection is:\\
a. A double check valve\\
b. A vacuum breaker\\
c. An air gap\\
d. A reduced pressure zone device\\
\item The positive side of the cathodic protection system is the:\\
a. Tank\\
b. Cathode\\
c. Rectifier\\
d. Sacrificial anode\\
\item A flow meter on a fire line would probably be a:\\
a. Venturi meter\\
b. Nutating disk meter\\
c. Oscillating piston meter\\
d. Compound meter\\
\item When filling a main, the water velocity should never exceed:\\
a. $1 \mathrm{ft} / \mathrm{sec}$\\
b. $2.5 \mathrm{ft} / \mathrm{sec}$\\
c. $10 \mathrm{ft} / \mathrm{sec}$\\
d. $20 \mathrm{ft} / \mathrm{sec}$\\
\item When two storage tanks that serve the same area have different overflow elevations, what type of valve should be included on the lower tank?\\
a. Check valve\\
b. Altitude valve\\
c. Air relief valve\\
d. Ball valve\\
\item Water hammer is caused by:\\
a. Opening a valve too slowly\\
b. Closing a valve too quickly\\
c. Excessive hardness\\
d. High pressure on the suction side of a pump\\
\item Comprehensive maps of medium to large systems generally have scales ranging from\\
a. 250-500 feet to 1 inch.\\
b. $500-1,000$ feet to 1 inch.\\
c. $1,000-1,500$ feet to 1 inch.\\
d. $1,500-2,000$ feet to 1 inch.\\
\item Sectional maps generally have scales ranging from\\
a. 50-100 feet to 1 inch.\\
b. $100-200$ feet to 1 inch.\\
c. $200-250$ feet to 1 inch. d. 250-400 feet to $1 \mathrm{inch}$.\\
\item A comprehensive map should be\\
a. compact enough to fit in a folder.\\
b. as large as possible.\\
c. as detailed as possible.\\
d. written in technical language so that only engineers can read it.\\
\item On a plan and profile drawing, what does the abbreviation EL mean? a. English language\\
b. Estimated length\\
c. Electric\\
d. Elevation\\
\item What type of map is also referred to as a wall map?\\
Comprehensive map\\
\item What type of map, commonly called a plat, is a series of maps covering sections of the water system?\\
\item When comparing friction loss in various types of pipes, a larger Hazen-Williams ' $C$ ' value indicates the pipe\\
a. is rougher inside\\
b. is rougher outside.\\
d. is able to withstand a higher pressure.\\
c. *is smoother inside.\\
\item What is the recommended minimum contact time water mains with the chlorine slug method?\\
a. 3 hours\\
b. 6 hours\\
c. 10 hours\\
d. 12 hours\\
\item A potable water supply discharges into an irrigation water storage tank. The 3 -inch potable supply line should be terminated\\
*a. Above the tank overflow by at least two pipe diameters\\
b. Above the tank outlet by at least two pipe diameters\\
c. Below the tank outlet by at least two pipe diameters\\
d. Level with the tank outlet\\
e. Level with the tank overflow\\
\item has been implicated in more waterborne disease outbreaks than any other factor.\\
a) improper treatment\\
*b) main breaks\\
c) improper or inadequate flushing\\
d) backflow\\
\item A physical link between a potable water supply and one of unknown or questionable quality is\\
*a) a cross connection\\
b) a Tier 1 violation\\
c) a Boil Water Advisory\\
d) a backflow prevention assembly\\
\item The best cross connection device is\\
*a) air gap\\
b) double check\\
c) atmospheric vacuum breaker\\
d) barometric loop\\
\item \rule{2cm}{0.5pt}corrosion is the corrosivity due to dissimilar metals.\\
a) saline\\
b) hydroxyl\\
c) excessive\\
*d) galvanic\\
\item The two types of backflow are\\
*a) backsiphonage and backpressure\\
b) backpressure and cavitation\\
c) air gap and rpz\\
d) dynamic and backsiphonage\\
\item This device is approved to protect against backflow and backsiphonage in high hazard applications.\\
a) double check valve assembly\\
b) vacuum pressure breaker\\
c) a hose bib\\
*d) reduced pressure zone assembly\\
\item Coupon testing is a viable indicator of\\
a) treatment optimization\\
b) the speed at which macrofloc is formed\\
*c) the corrosive or scale forming tendencies of your water\\
d) the super saturation level of dissolved oxygen in your water\\
\item An atmospheric vacuum breaker backflow prevention device protects against\\
a) backflow\\
b) backsiphonage and backpressure\\
c) neither\\
*d) backsiphonage\\
\item An approved air gap separation must be\\
a) 12 inches or 3 times the diameter whichever is greater\\
*b) $2 \frac{1}{2}$ times the inside diameter or a minimum of 1 inch\\
c) $.785 \times D^{\prime} \times D^{\prime}$\\
d) a barometric loop\\
\item Cathodic protection refers to\\
a) personal protective equipment\\
b) thermal electric protection\\
*c) corrosion\\
d) filtration\\
\item Which of the following physical factors does not influence the rate of corrosion\\
a.  Higher flow velocities\\
b.  Higher water temperatures\\
*c.  Higher system pressures\\
d.  Higher alkalinity levels\\
\item The most serious potential problem that water distribution systems can experience during high flows such as during fire fighting is:\\
*a.  Back-siphonage caused by negative or low pressures\\
b.  Movement of buried pipes caused by surge\\
c.  Loss of chlorine residual in system\\
d.  Temporary dirty-water complaints\\
\item A new section of pipeline must be disinfected:\\
a.  at the factory\\
b.  after delivery to utility storage yards\\
*c.  after installation and prior to potl!ble use\\
d. just before delivery to the site\\
\item Which water quality complaint is the most common for most utilities?\\
a. Appearance of the water\\
*b. Taste and odors\\
c. Stained laundry and plumbing fixtures\\
d. Illness caused by the water\\
\item When using the continuous feed method of disinfection, a new water main should be flushed, disinfected at $50 \mathrm{mg} / \mathrm{L}$, and held at above $25 \mathrm{mg} / \mathrm{L}$ for at least\\
a. 6 hours\\
b. 12 hours\\
c. 24 hours\\
d. 36  hours\\
e. 48 hours\\
\item \rule{2cm}{0.5pt}corrosion is the corrosivity due to dissimilar metals.\\
a) saline\\
b) hydroxyl\\
c) excessive\\
*d) galvanic\\
\item If a water system collects at least 40 samples per month for the analyses of total coliforms, which percent of total coliform positive samples are acceptable for the system to remain in compliance with the maximum contaminant level for total coliforms?\\
a. No more than $2 \%$\\
b. No more than $3 \%$\\
c. No more than $4 \%$\\
*d. No more than $5 \%$\\
\item Water systems are required to achieve at least \rule{1.5cm}{0.5pt} removal and/or inactivation of viruses between a point where the raw water is not subject to recontamination by surface water runoff and a point downstream before or at the first customer.\\
a. 2 log
b. 2.5 log
c. 3 log
*d. 4 log
\item Where are the sampling points located for required sampling of organics (except trihalomethanes) in a community water system?\\
a. Representative points within the distribution system\\
b. $75 \%$ at locations representative of population distribution and $25 \%$ at the farthest points in the distribution system\\
*c. Entry points to the distribution system\\
d. Entry points to the distribution system and representative points within the distribution system
\item Where are the sampling point(s) located for required sampling of natural radionuclides in a community water system?\\
a. Consumer's faucet\\
b. Representative points within the distribution system\\
*c. Each entry point to the system\\
d. $75 \%$ at locations representative of population distribution and $25 \%$ at the farthest points in the distribution system\\
\item Continuous chlorine residual monitoring is required where the water enters the distribution system under the Surface Water Treatment Rule when the\\
*a. population served is $>3,300$ people.\\
b. population served is $>10,000$ people.\\
c. number of taps is $>1,000$.\\
d. number of taps is $>2,500$.\\
\item Which of the following physical factors does not influence the rate of corrosion\\
a.	Higher flow velocities\\
b.	Higher water temperatures\\
c.	Higher system pressures\\
d.	Higher alkalinity levels\\
\item Small and medium-size utilities are considered to have optimal corrosion control if they meet the lead and copper action levels for\\
a.	One sampling period\\
b.	Two consecutive sampling periods\\
c.	Three consecutive sampling periods\\
d.	Four consecutive sampling periods\\


\item Surge tanks in the water system are used for?\\
*a. Eliminating water hammer and prevent pressure spikes\\
b. Eliminating tastes and odors\\
c. Eliminating excessive chlorine dosage\\
d. Eliminating Iron and Manganese

\item Which type of pipe joint for connecting water pipes has the advantage of moderate deflection and the gaskets used to absorb vibration and pipe movement?\\
a. flanged joint\\
b. victaulic joint\\
*c. dresser coupling\\
d. restrained joint\\
\
\item Which type of joint for connecting water pipes has the disadvantage of having a grove that weakens the pipe wall?\\
a. flanged joint\\
*b. victaulic joint\\
c. dressler coupling\\
d. restrained joint\\
\end{enumerate}
