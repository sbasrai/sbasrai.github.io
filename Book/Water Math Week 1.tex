\chapterimage{MathCover.png}
\chapter{Water Math - Week 1}



\begin{table}[H]
\begin{tabular}{| m{1cm} | m{1cm} | m{12cm} |}
\hline
\multicolumn{3}{|l|}{\textbf{Expected   Range of Knowledge for Math}}                                                                      \\ \hline
\multicolumn{3}{|l|}{\textit{Water   Distribution System Operator License Exams}}                                                          \\ \hline
\multicolumn{1}{l|}{} & \multicolumn{1}{l|}{D1-D5} & Ability to calculate   flow rates for a storage facility                     \\ \cline{2-3} 
\multicolumn{1}{l|}{} & \multicolumn{1}{l|}{D1-D5} & Ability to calculate   the volume of a storage facility                      \\ \cline{2-3} 
\multicolumn{1}{l|}{} & \multicolumn{1}{l|}{D1-D5} & Knowledge of unit   conversions                                              \\ \cline{2-3} 
\multicolumn{1}{l|}{} & \multicolumn{1}{l|}{D1-D5} & Ability to calculate   flow rates                                            \\ \cline{2-3} 
\multicolumn{1}{l|}{} & \multicolumn{1}{l|}{D1-D5} & Ability to calculate   pipe volumes                                          \\ \cline{2-3} 
\multicolumn{1}{l|}{} & \multicolumn{1}{l|}{D1-D5} & Ability to calculate   the area of a pipe cross-section                      \\ \cline{2-3}
\multicolumn{1}{l|}{} & \multicolumn{1}{l|}{D1-D5} & Ability to calculate   the volume of a trench                                \\ \cline{2-3}  
\multicolumn{1}{l|}{} & \multicolumn{1}{l|}{D1-D5} & Ability to calculate   the surface area of a valve face                      \\ \cline{2-3} 
\multicolumn{1}{l|}{} & \multicolumn{1}{l|}{D1-D5} & Ability to calculate   the volume of a cylinder, rectangle, and square       \\ \cline{2-3} 
\multicolumn{1}{l|}{} & \multicolumn{1}{l|}{D1-D5} & Ability to calculate   the volume of a pipe                                  \\ \cline{2-3} 
\multicolumn{1}{l|}{} & \multicolumn{1}{l|}{D1-D5} & Ability to calculate   the volume of a well, storage reservoir, pipe, trench \\ \cline{2-3} 
\multicolumn{1}{l|}{} & \multicolumn{1}{l|}{D1-D5} & Ability to calculate   the well draw down                                    \\ \cline{2-3} 
\multicolumn{1}{l|}{} & \multicolumn{1}{l|}{D1-D5} & Ability to calculate   total force on a valve                                \\ \cline{2-3} 
\multicolumn{1}{l|}{} & \multicolumn{1}{l|}{D1-D5} & Ability to convert   pressure to feet of head                                \\ \cline{2-3} 
\multicolumn{1}{l|}{} & \multicolumn{1}{l|}{D1-D5} & Ability to convert   units of volume, area, and time                         \\ \cline{2-3} 
\multicolumn{1}{l|}{} & \multicolumn{1}{l|}{D1-D5} & Ability to convert   units of volume, area, pressure, and time               \\ \cline{2-3} 
\multicolumn{1}{l|}{} & \multicolumn{1}{l|}{D1-D5} & Ability to convert   units of volume, pressure and area                      \\ \cline{2-3} 
\multicolumn{1}{l|}{} & \multicolumn{1}{l|}{D1-D5} & Ability to convert   water units                                             \\ \cline{2-3} 
\multicolumn{1}{l|}{} & \multicolumn{1}{l|}{D2-D5} & Ability to calculate   pipe capacity                                         \\ \cline{2-3} 
\multicolumn{1}{l|}{} & \multicolumn{1}{l|}{D2-D5} & Ability to calculate   the velocity of water                                 \\ \cline{2-3} 
\multicolumn{1}{l|}{} & \multicolumn{1}{l|}{D2-D5} & Ability to calculate   thrust block size                                     \\ \cline{2-3} 
\multicolumn{1}{l|}{} & \multicolumn{1}{l|}{D2-D5} & Ability to convert a   pressure reading to depth of water                    \\ \cline{2-3} 
\multicolumn{1}{l|}{} & \multicolumn{1}{l|}{D2-D5} & Ability to convert a   scale to actual distance                              \\ \cline{2-3} 
\end{tabular}
\end{table}
\newpage






\begin{table}[H]
\begin{tabular}{| m{1cm} | m{1cm} | m{12cm} |}
\hline
\multicolumn{3}{|l|}{\textbf{Expected   Range of Knowledge for Math}}                                                                      \\ \hline
\multicolumn{3}{|l|}{\textit{Water   Distribution System Operator License Exams (Continued)}}                                                          \\ \hline
\multicolumn{1}{l|}{} & \multicolumn{1}{l|}{D3-D5} & Ability to calculate   brake-horsepower                                      \\ \cline{2-3} 
\multicolumn{1}{l|}{} & \multicolumn{1}{l|}{D3-D5} & Ability to calculate   pump efficiency                                       \\ \cline{2-3} 
\multicolumn{1}{l|}{} & \multicolumn{1}{l|}{D3-D5} & Ability to calculate   specific yield of a well                              \\ \cline{2-3} 
\multicolumn{1}{l|}{} & \multicolumn{1}{l|}{D3-D5} & Ability to calculate   the cost of water production                          \\ \cline{2-3} 
\multicolumn{1}{l|}{} & \multicolumn{1}{l|}{D4-D5} & Ability to calculate a water loss rate                                       \\ \cline{2-3} 
\multicolumn{1}{l|}{} & \multicolumn{1}{l|}{D4-D5} & Ability to calculate the cost of pumping   water                             \\ \cline{2-3} 
\multicolumn{1}{l|}{} & \multicolumn{1}{l|}{D4-D5} & Ability to calculate the hydraulic gradient                                  \\ \cline{2-3} 
\multicolumn{1}{l|}{} & \multicolumn{1}{l|}{D4-D5} & Ability to calculate water production costs                                  \\ \hline
\multicolumn{3}{|l|}{Water   Treatment Operator License Exams}                                                                    \\ \hline
\multicolumn{1}{l|}{} & \multicolumn{1}{l|}{T1-T4} & Ability to calculate   flow rates and water velocity                         \\ \cline{2-3} 
\multicolumn{1}{l|}{} & \multicolumn{1}{l|}{T1-T4} & Ability to calculate   the volume of water in a storage facility             \\ \cline{2-3} 
\multicolumn{1}{l|}{} & \multicolumn{1}{l|}{T1-T4} & Ability to calculate   well head pressure                                    \\ \cline{2-3} 
\multicolumn{1}{l|}{} & \multicolumn{1}{l|}{T1-T4} & Ability to convert   common water units (e.g. gallons per minute to MGD)     \\ \cline{2-3} 
\multicolumn{1}{l|}{} & \multicolumn{1}{l|}{T1-T4} & Ability to convert   head pressure to water elevation                        \\ \cline{2-3} 
\multicolumn{1}{l|}{} & \multicolumn{1}{l|}{T1-T4} & Ability to convert   units of length, volume, flow and pressure              \\ \cline{2-3} 
\multicolumn{1}{l|}{} & \multicolumn{1}{l|}{T1-T4} & Ability to determine   water level in a storage tank, reservoir, or well     \\ \cline{2-3} 
\multicolumn{1}{l|}{} & \multicolumn{1}{l|}{T1-T4} & Ability to calculate   a chemical dosage                                     \\ \cline{2-3} 
\multicolumn{1}{l|}{} & \multicolumn{1}{l|}{T1-T4} & Ability to calculate   a chemical solution concentration                     \\ \cline{2-3} 
\multicolumn{1}{l|}{} & \multicolumn{1}{l|}{T1-T4} & Ability to calculate   chlorine demand and chlorine residual                 \\ \cline{2-3} 
\multicolumn{1}{l|}{} & \multicolumn{1}{l|}{T1-T4} & Ability to convert   common water units, (gallons per minute to MGD, etc...) \\ \cline{2-3} 
\multicolumn{1}{l|}{} & \multicolumn{1}{l|}{T1-T4} & Ability to determine   water level in a storage tank, reservoir or well      \\ \cline{2-3} 
\multicolumn{1}{l|}{} & \multicolumn{1}{l|}{T3-T4} & Ability to perform   blending calculations                                   \\ \cline{2-3} 
\multicolumn{1}{l|}{} & \multicolumn{1}{l|}{T3-T4} & Ability to calculate   a dilution factor                                     \\ \cline{2-3} 
\multicolumn{1}{l|}{} & \multicolumn{1}{l|}{T3-T4} & Ability to mix   chemicals and prepare reagents                              \\ \cline{2-3} 
\multicolumn{1}{l|}{} & \multicolumn{1}{l|}{T3-T4} & Ability to perform   dilutions                                               \\ \cline{2-3} 
\multicolumn{1}{l|}{} & \multicolumn{1}{l|}{T3-T4} & Ability to calculate   a coagulant dose from a jar test                      \\ \cline{2-3} 
\multicolumn{1}{l|}{} & \multicolumn{1}{l|}{T3-T4} & Ability to calculate   a filter-aid dosage                                   \\ \cline{2-3} 
\multicolumn{1}{l|}{} & \multicolumn{1}{l|}{T3-T4} & Ability to calculate   a filtration rate                                     \\ \cline{2-3} 
\multicolumn{1}{l|}{} & \multicolumn{1}{l|}{T3-T4} & Ability to calculate   filter loading rate                                   \\ \cline{2-3} 
\multicolumn{1}{l|}{} & \multicolumn{1}{l|}{T3-T4} & Ability to calculate   percent or log removal of contaminants from water     \\ \cline{2-3} 
\multicolumn{1}{l|}{} & \multicolumn{1}{l|}{T3-T4} & Ability to calculate   the cost of water treatment operations                \\ \cline{2-3} 
\end{tabular}
\end{table}
\newpage










\section{Fractions}\index{Fractions}
\begin{itemize}
\item A fraction is defined as part of whole.  If in a class there are 20 male students and 30 male students, the fraction of male students is $\dfrac{20}{50} or \dfrac{2}{5}$.
\item It is composed of three items: two numbers and a line.
\item The number on the top is called the numerator, the number on the bottom is called the denominator, and the line in between them means to divide. 
$$
\text { Divide } \longrightarrow \dfrac{3}{4} \quad \begin{aligned}
&\text { Numerator } \\
&\text { Denominator }
\end{aligned}
$$
\item A proper fraction is a fraction that has no whole number part and its numerator is smaller than its denominator. An improper fraction is a fraction that has a larger numerator than denominator and it represents a number greater than one.\\
Proper Fraction Examples: $\dfrac{1}{2}$, $\dfrac{5}{8}$, $\dfrac{11}{12}$\\
\vspace{0.2cm}
Improper Fraction Examples: $\dfrac{12}{2}$, $\dfrac{5}{2}$
\item Any whole number can be expressed as a fraction by placing a "1" in the denominator. For example:

2 is the same as $\dfrac{2}{1}$ and 45 is the same as $\dfrac{45}{1}$

\item Only fractions with the same denominator can be added/subtracted, and only the numerators are added/subtracted. For example:
$$
\dfrac{1}{8}+\dfrac{3}{8}=\dfrac{4}{8}  \enspace  \text {and},  \enspace \dfrac{7}{8}-\dfrac{3}{8}=\dfrac{4}{8}
$$

\item A fraction combined with a whole number is called a mixed number. For example:
$$
4 \dfrac{1}{8}, \enspace 16 \dfrac{2}{3}, \enspace  8 \dfrac{3}{4}, \enspace  45  \dfrac{1}{2} \text { and, } 12\dfrac{17}{32}
$$
These numbers are read, four and one eighth, sixteen and two thirds, eight and three fourths, forty-five and one half, and twelve and seventeen thirty seconds.\\
Mized numbers 

\item A fraction can be changed by multiplying the numerator and denominator by the same number. This does not change the value of the fraction, only how it looks. For instance:
$$
\dfrac{1}{2} \text { is the same as } \dfrac{1}{2} \times \dfrac{2}{2} \text { which is } \dfrac{2}{4}
$$

\item Steps to convert $\dfrac{17}{4}$ to a mixed number:
\begin{enumerate}[Step 1.]
\item How many times can 4 fit into 17? 4 because 4×4=16.  Thus, 4 becomes the whole number part
\item How much is left over in the numerator? 1 because $17-16=1$.  Thus, 1 becomes the numerator of the fractional part
\item $\dfrac{17}{4} = 4\dfrac{1}{4}$
\end{enumerate}
\vspace{0.2cm}
\item To turn a mixed number into an improper fraction, multiply the whole number part by the denominator and add the numerator. This becomes the new numerator over the original denominator.

Example: Converting 1.5 feet to fraction\\
$1.5ft=1\dfrac{1}{2}$\\
\vspace{0.2cm}
$1\dfrac{1}{2}=\dfrac{1*2+1}{2}=\dfrac{2+1}{2}=\dfrac{3}{2}$
\vspace{0.2cm}
\item A mixed value - say a circumference is given in feet and fraction of feet (say $7 \enspace 3/4$), needs to be converted to a fraction for calculation purposes.
\end{itemize}

% \begin{tcolorbox}[
% colframe=blue!25,
% colback=blue!10,
% coltitle=blue!20!black,  
% title= Practice Problems]
% \begin{enumerate}
% \item Convert 22$\dfrac{1}{4}$ into a fraction
% \item Express 10ft, 6in as a fraction
% \item Express 10ft, 6in as decimal
% \item Add $\dfrac{3}{4}+\dfrac{1}{7}$
% \item Multiply $\dfrac{4}{9}*\dfrac{3}{16}$
% \end{enumerate}
% \end{tcolorbox}


\section{Decimals \& Powers of Ten}\index{Decimals \& Powers of Ten}
\begin{itemize}
\item A decimal is composed of two sets of numbers: the numbers to the left of the decimal are whole numbers, and numbers to the right of the decimal are parts of whole numbers, a fraction of a number.\\

\item The term used to express the fraction component is dependent on the number of characters to the right of the decimal.

\begin{itemize}
  \item The first character after the decimal point is tenths: $0.1$ - tenths

  \item The second character is hundredths: $0.01$ - hundredths

  \item The third character is thousandths: $0.001$ - thousandths
\end{itemize}

\item Powers of 10 notation enables us to work with these very large and small quantities efficiently.
\item In water math, the most common application of this concept is related to parts per million (ppm) or parts per billion (ppb).
\item 1 million - 1,000,000 can be represented as 10$^6$.  Likewise, 1 billion - 1,000, 000,000 can be represented as 10$^9$
\item The sequence of powers of ten can also be extended to negative powers.
\item 1 part per million (1/1,000,000) can be written as 10$^-6$
\end{itemize}



\begin{table}[ht]
\begin{tabular}{|l|l|l|l|l|}
\hline
\multicolumn{1}{|c|}{\textbf{Name}} & \multicolumn{1}{c|}{\textbf{Power}} & \multicolumn{1}{c|}{\textbf{Number}} & \multicolumn{1}{c|}{\textbf{SI symbol}} & \multicolumn{1}{c|}{\textbf{SI prefix}} \\ \hline
one                                 & $10^0$& 1                                    &                                         &                                         \\ \hline
ten                                 & $10^1$                                   & 10                                   & da (D)                                  & deca                                    \\ \hline
hundred                             & $10^2$                                   & 100                                  & h (H)                                   & hecto                                   \\ \hline
thousand                            & $10^3$                                  & 1,000                                & k (K)                                   & kilo                                    \\ \hline
million                             & $10^6$                                   & 1,000,000                            & M                                       & mega                                    \\ \hline
billion                             & $10^9$                                  & 1,000,000,000                        & G                                       & giga                                    \\ \hline
tenth                               & $10^{-1}$                                 & 0.1                                  & d                                       & deci                                    \\ \hline
hundredth                           & $10^{-2}$                                  & 0.01                                 & c                                       & centi                                   \\ \hline
thousandth                          & $10^{-3} $                                 & 0.001                                & m                                       & milli                                   \\ \hline
millionth                           &$10^{-6} $                               & 0.000 001                            & $\mu$                                      & micro                                   \\ \hline
billionth                           & $10^{-9} $                               & 0.000 000 001                        & n                                       & nano                                    \\ \hline
\end{tabular}
\end{table}

% \begin{tcolorbox}[
% colframe=blue!25,
% colback=blue!10,
% coltitle=blue!20!black,  
% title= Practice Problems]
% \begin{enumerate}
% \item Write the equivalent of 10,000,000 as a power of ten
% \item Find the product of $3.4564*10^2$
% \item Find the product of $534.567*10^{-2}$
% \vspace{0.2cm}
% \item Find the value of $\dfrac{165.93}{10^{-2}}$
% \vspace{0.2cm}
% \item Find the value of $0.023*10^4$
% \end{enumerate}
% \end{tcolorbox}

\section{Rounding and Significant Digits}\index{Rounding and Significant Digits}

\begin{itemize}
\item Significant digits (also called Significant Figures) are digits which give us useful information about the accuracy of a measurement and are related to rounding.
\item This concept is used to determine the direction to round a number (answer). The basic idea is that no answer can be more accurate than the least accurate piece of data used to calculate the answer.\\
\item Significant digits is the count of the numerals in a measured quantity (counting from the left) whose values are considered as known exactly, plus one more whose value could be one more or one less.\\
\item Rules for determining the number of significant digits:
\begin{enumerate}
\item All nonzero digits are significant:\\
1.234 g has 4 significant figures, and 1.2 g has 2 significant figures.
\item Zeroes between nonzero digits are significant:
1002 kg has 4 significant figures, 3.07 mL has 3 significant figures.
\item Zeroes to the left of the first nonzero digits are not significant; such zeroes merely indicate the position of the decimal point:
\SI{0.001}{\celsius} has only 1 significant figure, 0.012 g has 2 significant figures.
\item Zeroes to the right of a decimal point in a number are significant:
0.023 mL has 2 significant figures, 0.200 g has 3 significant figures.
\item When a number ends in zeroes that are not to the right of a decimal point, the zeroes are not necessarily significant:
190 miles may be 2 or 3 significant figures, 50,600 calories may be 3, 4, or 5 significant figures. The potential ambiguity in the last rule can be avoided by the use of standard exponential, or ”scientific,” notation. For example, depending on whether 3, 4, or 5 significant figures is correct, we could write 50,600 calories as: $5.06*10^4$ calories (3 significant figures) $5.060*10^4$ calories (4 significant figures), or
$5.0600*10^4)$ calories (5 significant figures).
\end{enumerate}
\item Examples of significant figures:
% Please add the following required packages to your document preamble:
% \usepackage[normalem]{ulem}
% \useunder{\uline}{\ul}{}
% Please add the following required packages to your document preamble:
% \usepackage[normalem]{ulem}
% \useunder{\uline}{\ul}{}
\begin{table}[h]
\begin{tabular}{|p{16cm}|}
\hline
\scriptsize{1000 has   one significant digit: only the 1 is interesting (only it tells us anything   specific); we don't know anything for sure about the hundreds, tens, or units   places; the zeroes may just be placeholders; they may have rounded something   off to get this value.                                    } \\ \hline
\scriptsize{1000.0 has five significant   digits: the ".0" tells us something interesting about the presumed   accuracy of the measurement being made; namely, that the measurement is   accurate to the tenths place, but that there happen to be zero tenths.                                                               } \\ \hline
\scriptsize{0.00035 has two significant   digits: only the 3 and 5 tell us something; the other zeroes are   placeholders, only providing information about relative size.                                                                                                                                                    } \\ \hline
\scriptsize{0.000350 has three significant   digits: the last zero tells us that the measurement was made accurate to that   last digit, which just happened to have a value of zero.                                                                                                                                         } \\ \hline
\scriptsize{1006 has four significant   digits: the 1 and 6 are interesting, and we have to count the zeroes, because   they're between the two interesting numbers.                                                                                                                                                          } \\ \hline
\scriptsize{560 has two significant   digits: the last zero is just a placeholder.                                                                                                                                                                                                                                            } \\ \hline
\scriptsize{560. : notice that   "point" after the zero! This has three significant digits, because   the decimal point tells us that the measurement was made to the nearest unit,   so the zero is not just a placeholder.                                                                                                  } \\ \hline
\scriptsize{560.0 has four significant   digits: the zero in the tenths place means that the measurement was made   accurate to the tenths place, and that there just happen to be zero tenths;   the 5 and 6 give useful information, and the other zero is between   significant digits, and must therefore also be counted.} \\ \hline
\end{tabular}
\end{table}
\item Addition and Subtraction\\
\begin{itemize}
\item When you are adding or subtracting a bunch of numbers and need to be concerned with significant figures, first add (or subtract) the numbers given in their entire format, and then round the final answer. When rounding the final answer after adding or subtracting, the answer must be written with the same significant figures as the least accurate decimal place given.\\
\textbf{Example:} 13.214 + 234.6 + 7.0350 + 6.38\\
\begin{itemize}
\item 13.214 + 234.6 + 7.0350 + 6.38 = 261.2290\\
\item 234.6 is only accurate to the tenths place making it the least accurate number. My answer must be rounded to the same place as the least accurate number:\\
\item 261.2290 rounds to 261.2 (one decimal place)\\
\end{itemize}
\end{itemize}
\item Multiplication and Division\\
\begin{itemize}
\item When multiplying or dividing multiple numbers you would do these calculations as normal. When the answer must be written in the appropriate significant figure your answer must round to the same number of significant figures as the least number of significant figures.\\
\textbf{Example 1:}  Simplify, and round, to the appropriate number of significant digits \\
\begin{center}
16.235 x 0.217 x 5\\
\end{center}
\begin{enumerate}[Step 1.]
\item First off, 5 has only one significant figure, thus the final answer needs to be rounded to one significant digit\\
\item 16.235 x 0.217 x 5 = 17.614975\\
\item To round 17.614975 to one digit. I'll start with the 1 in the tens place. Immediately to its right is a 7, which is greater than 5, so 1 is rounded up to 2, and then replacing the 7 with a zero, and dropping the decimal point and everything after it.
\item 17.614975 rounds to 20\\
\end{enumerate}
\textbf{Example 2:}  Simplify, and round, to the appropriate number of significant figures\\
\begin{center}
0.00435 x 4.6
\end{center}
\begin{enumerate}[Step 1.]
\item 4.6 has only 2 significant figures, so the final answer should be rounded to two significant figures.
\item 0.00435 x 4.6 = 0.02001\\
\item 0.02001 would round to 0.020, which has 2 significant figures (0.020). The answer cannot be 0.02, because that value would have only one significant figure.\\
\end{enumerate}
\end{itemize}
\end{itemize}

\begin{itemize}
\item \emph{A number is rounded off by dropping one or more numbers from the right and adding zeroes, if necessary, to maintain the decimal point.} 
\item \emph{If the last figure dropped is 5 or more, increase the last retained figure by 1. If the last digit dropped is less than 5, do not increase the last retained figure.}
\end{itemize}


\section{Averages}\index{Averages}
\begin{itemize}
\item Also known as \emph{arithmetic mean}, this value is arrived at by adding the quantities in a series and dividing the total by the number in the series.
\end{itemize}
Example 1: Find the average of the following series of numbers: 12,8,6,21,4,5 , 9 , and 12.\\
Adding the numbers together we get 77.\\
There are 8 numbers in this set.\\
Divide 77 by 8.\\

$\dfrac{77}{8}=9.6$ is the average of the set\\

Example 2:  Find the average of the set of daily turbidity data - 0.3,0.4,0.3,0.1,and 0.8\\
The total is 1.9.\\
There are 5 numbers in the set.\\
Therefore:
$$
\dfrac{1.9}{5}=0.38, \text { rounding off }=0.4
$$






% \section*{Practice Problems - Averages}

% \begin{tcolorbox}[
% colframe=blue!25,
% colback=blue!10,
% coltitle=blue!20!black,  
% title= Practice Problems]
% \begin{enumerate}
% \item Find the average of the following set of numbers:\\
% $
% \begin{aligned}
% &0.2 \\
% &0.2 \\
% &0.1 \\
% &0.3 \\
% &0.2 \\
% &0.4 \\
% &0.6 \\
% &0.1 \\
% &0.3
% \end{aligned}
% $

% \item The chemical used for each day during a week is given below. Based on these data, what was the average lb/day chemical used during the week?\\

% \begin{tabular}{|l|l|}
% \hline
% Monday & 92 lb/day\\
% \hline
% Tuesday & 93 lb/day \\
% \hline
% Wednesday & 98 lb/day\\
% \hline
% Thursday & 93 lb/day \\
% \hline
% Friday & 89 lb/day\\
% \hline
% Saturday & 93 lb/day \\
% \hline
% Sunday & 97 lb/day\\
% \hline
% \end{tabular}

% \item The average chemical use at a plant is 77 lb/day. If the chemical inventory is 2800 lbs, how many days supply is this?
% \end{enumerate}
% \end{tcolorbox}






