\chapterimage{MathCover.png}
\chapter{Water Math - Week 4}



\begin{table}[H]
\begin{tabular}{| m{1cm} | m{1cm} | m{12cm} |}
\hline
\multicolumn{3}{|l|}{\textbf{Expected   Range of Knowledge for Math}}                                                                      \\ \hline
\multicolumn{3}{|l|}{\textit{Water   Distribution System Operator License Exams}}                                                          \\ \hline
\multicolumn{1}{l|}{} & \multicolumn{1}{l|}{D1-D5} & Ability to calculate   flow rates for a storage facility                     \\ \cline{2-3} 
\multicolumn{1}{l|}{} & \multicolumn{1}{l|}{D1-D5} & Ability to calculate   the volume of a storage facility                      \\ \cline{2-3} 
\multicolumn{1}{l|}{} & \multicolumn{1}{l|}{D1-D5} & Knowledge of unit   conversions                                              \\ \cline{2-3} 
\multicolumn{1}{l|}{} & \multicolumn{1}{l|}{D1-D5} & Ability to calculate   flow rates                                            \\ \cline{2-3} 
\multicolumn{1}{l|}{} & \multicolumn{1}{l|}{D1-D5} & Ability to calculate   pipe volumes                                          \\ \cline{2-3} 
\multicolumn{1}{l|}{} & \multicolumn{1}{l|}{D1-D5} & Ability to calculate   the area of a pipe cross-section                      \\ \cline{2-3}
\multicolumn{1}{l|}{} & \multicolumn{1}{l|}{D1-D5} & Ability to calculate   the volume of a trench                                \\ \cline{2-3}  
\multicolumn{1}{l|}{} & \multicolumn{1}{l|}{D1-D5} & Ability to calculate   the surface area of a valve face                      \\ \cline{2-3} 
\multicolumn{1}{l|}{} & \multicolumn{1}{l|}{D1-D5} & Ability to calculate   the volume of a cylinder, rectangle, and square       \\ \cline{2-3} 
\multicolumn{1}{l|}{} & \multicolumn{1}{l|}{D1-D5} & Ability to calculate   the volume of a pipe                                  \\ \cline{2-3} 
\multicolumn{1}{l|}{} & \multicolumn{1}{l|}{D1-D5} & Ability to calculate   the volume of a well, storage reservoir, pipe, trench \\ \cline{2-3} 
\multicolumn{1}{l|}{} & \multicolumn{1}{l|}{D1-D5} & Ability to calculate   the well draw down                                    \\ \cline{2-3} 
\multicolumn{1}{l|}{} & \multicolumn{1}{l|}{D1-D5} & Ability to calculate   total force on a valve                                \\ \cline{2-3} 
\multicolumn{1}{l|}{} & \multicolumn{1}{l|}{D1-D5} & Ability to convert   pressure to feet of head                                \\ \cline{2-3} 
\multicolumn{1}{l|}{} & \multicolumn{1}{l|}{D1-D5} & Ability to convert   units of volume, area, and time                         \\ \cline{2-3} 
\multicolumn{1}{l|}{} & \multicolumn{1}{l|}{D1-D5} & Ability to convert   units of volume, area, pressure, and time               \\ \cline{2-3} 
\multicolumn{1}{l|}{} & \multicolumn{1}{l|}{D1-D5} & Ability to convert   units of volume, pressure and area                      \\ \cline{2-3} 
\multicolumn{1}{l|}{} & \multicolumn{1}{l|}{D1-D5} & Ability to convert   water units                                             \\ \cline{2-3} 
\multicolumn{1}{l|}{} & \multicolumn{1}{l|}{D2-D5} & Ability to calculate   pipe capacity                                         \\ \cline{2-3} 
\multicolumn{1}{l|}{} & \multicolumn{1}{l|}{D2-D5} & Ability to calculate   the velocity of water                                 \\ \cline{2-3} 
\multicolumn{1}{l|}{} & \multicolumn{1}{l|}{D2-D5} & Ability to calculate   thrust block size                                     \\ \cline{2-3} 
\multicolumn{1}{l|}{} & \multicolumn{1}{l|}{D2-D5} & Ability to convert a   pressure reading to depth of water                    \\ \cline{2-3} 
\multicolumn{1}{l|}{} & \multicolumn{1}{l|}{D2-D5} & Ability to convert a   scale to actual distance                              \\ \cline{2-3} 
\end{tabular}
\end{table}
\newpage






\begin{table}[H]
\begin{tabular}{| m{1cm} | m{1cm} | m{12cm} |}
\hline
\multicolumn{3}{|l|}{\textbf{Expected   Range of Knowledge for Math}}                                                                      \\ \hline
\multicolumn{3}{|l|}{\textit{Water   Distribution System Operator License Exams (Continued)}}                                                          \\ \hline
\multicolumn{1}{l|}{} & \multicolumn{1}{l|}{D3-D5} & Ability to calculate   brake-horsepower                                      \\ \cline{2-3} 
\multicolumn{1}{l|}{} & \multicolumn{1}{l|}{D3-D5} & Ability to calculate   pump efficiency                                       \\ \cline{2-3} 
\multicolumn{1}{l|}{} & \multicolumn{1}{l|}{D3-D5} & Ability to calculate   specific yield of a well                              \\ \cline{2-3} 
\multicolumn{1}{l|}{} & \multicolumn{1}{l|}{D3-D5} & Ability to calculate   the cost of water production                          \\ \cline{2-3} 
\multicolumn{1}{l|}{} & \multicolumn{1}{l|}{D4-D5} & Ability to calculate a water loss rate                                       \\ \cline{2-3} 
\multicolumn{1}{l|}{} & \multicolumn{1}{l|}{D4-D5} & Ability to calculate the cost of pumping   water                             \\ \cline{2-3} 
\multicolumn{1}{l|}{} & \multicolumn{1}{l|}{D4-D5} & Ability to calculate the hydraulic gradient                                  \\ \cline{2-3} 
\multicolumn{1}{l|}{} & \multicolumn{1}{l|}{D4-D5} & Ability to calculate water production costs                                  \\ \hline
\multicolumn{3}{|l|}{Water   Treatment Operator License Exams}                                                                    \\ \hline
\multicolumn{1}{l|}{} & \multicolumn{1}{l|}{T1-T4} & Ability to calculate   flow rates and water velocity                         \\ \cline{2-3} 
\multicolumn{1}{l|}{} & \multicolumn{1}{l|}{T1-T4} & Ability to calculate   the volume of water in a storage facility             \\ \cline{2-3} 
\multicolumn{1}{l|}{} & \multicolumn{1}{l|}{T1-T4} & Ability to calculate   well head pressure                                    \\ \cline{2-3} 
\multicolumn{1}{l|}{} & \multicolumn{1}{l|}{T1-T4} & Ability to convert   common water units (e.g. gallons per minute to MGD)     \\ \cline{2-3} 
\multicolumn{1}{l|}{} & \multicolumn{1}{l|}{T1-T4} & Ability to convert   head pressure to water elevation                        \\ \cline{2-3} 
\multicolumn{1}{l|}{} & \multicolumn{1}{l|}{T1-T4} & Ability to convert   units of length, volume, flow and pressure              \\ \cline{2-3} 
\multicolumn{1}{l|}{} & \multicolumn{1}{l|}{T1-T4} & Ability to determine   water level in a storage tank, reservoir, or well     \\ \cline{2-3} 
\multicolumn{1}{l|}{} & \multicolumn{1}{l|}{T1-T4} & Ability to calculate   a chemical dosage                                     \\ \cline{2-3} 
\multicolumn{1}{l|}{} & \multicolumn{1}{l|}{T1-T4} & Ability to calculate   a chemical solution concentration                     \\ \cline{2-3} 
\multicolumn{1}{l|}{} & \multicolumn{1}{l|}{T1-T4} & Ability to calculate   chlorine demand and chlorine residual                 \\ \cline{2-3} 
\multicolumn{1}{l|}{} & \multicolumn{1}{l|}{T1-T4} & Ability to convert   common water units, (gallons per minute to MGD, etc...) \\ \cline{2-3} 
\multicolumn{1}{l|}{} & \multicolumn{1}{l|}{T1-T4} & Ability to determine   water level in a storage tank, reservoir or well      \\ \cline{2-3} 
\multicolumn{1}{l|}{} & \multicolumn{1}{l|}{T3-T4} & Ability to perform   blending calculations                                   \\ \cline{2-3} 
\multicolumn{1}{l|}{} & \multicolumn{1}{l|}{T3-T4} & Ability to calculate   a dilution factor                                     \\ \cline{2-3} 
\multicolumn{1}{l|}{} & \multicolumn{1}{l|}{T3-T4} & Ability to mix   chemicals and prepare reagents                              \\ \cline{2-3} 
\multicolumn{1}{l|}{} & \multicolumn{1}{l|}{T3-T4} & Ability to perform   dilutions                                               \\ \cline{2-3} 
\multicolumn{1}{l|}{} & \multicolumn{1}{l|}{T3-T4} & Ability to calculate   a coagulant dose from a jar test                      \\ \cline{2-3} 
\multicolumn{1}{l|}{} & \multicolumn{1}{l|}{T3-T4} & Ability to calculate   a filter-aid dosage                                   \\ \cline{2-3} 
\multicolumn{1}{l|}{} & \multicolumn{1}{l|}{T3-T4} & Ability to calculate   a filtration rate                                     \\ \cline{2-3} 
\multicolumn{1}{l|}{} & \multicolumn{1}{l|}{T3-T4} & Ability to calculate   filter loading rate                                   \\ \cline{2-3} 
\multicolumn{1}{l|}{} & \multicolumn{1}{l|}{T3-T4} & Ability to calculate   percent or log removal of contaminants from water     \\ \cline{2-3} 
\multicolumn{1}{l|}{} & \multicolumn{1}{l|}{T3-T4} & Ability to calculate   the cost of water treatment operations                \\ \cline{2-3} 
\end{tabular}
\end{table}
\newpage


\section{Density}\index{Density}
\begin{itemize}
\item Density is defined as the weight of a substance per a unit of its volume. For example, pounds per cubic foot or pounds per gallon.

\item Here are a few key facts about density:
\begin{itemize}

\item Density is measured in units of lb/ft3, lb/gal, or mg/L. Density of water = 62.4 lb/ft3 = 8.34 lb/gal.
\end{itemize}
\end{itemize}

\section{Specific Gravity}\index{Specific Gravity}
\begin{itemize}
\item Specific gravity is the ratio of the density of a substance (liquid or solid) to the density water.
\item It is the ratio of the weight of the substance of a certain volume to the weight of water of the same volume.

\item Any substance with a density greater than that of water will have a specific gravity greater than 1.0. Any substance with a density less than that of water will have a specific gravity less than 1.0. 

\item Specific gravity examples:
\begin{itemize}

\item Specific gravity of water = 1.0 
\item Specific gravity of concrete = 2.5 (depending on ingredients)
\item Specific gravity of alum (liquid @ 60°F) = 1.33 
\item Specific gravity of hydrogen peroxide (35\%) = 1.132
\end{itemize}

\item Specific gravity is used in two ways:
\begin{enumerate}
\item To calculate the total weight of a \% solution (either as a single gallon or a drum volume).\\
Total Weight = Drum Vol X SG X 8.34
\item To calculate the “active ingredient” weight of a single gallon or a drum.\\

Active Ingredient Weight within Drum = Drum Volume X SG X 8.34 X \% solution as a decimal. (i.e., Total Weight X \% solution as a decimal)\\

NOTE: Both ways start with solving for the total weight (Drum Vol X SG X 8.34). When solving for “active ingredient” weight, you have to then multiply by \% solution as a decimal.

\end{enumerate}
\end{itemize}

\textbf{Example:} What is the weight of 5 gallons of a 40\% ferric chloride solution given its specific gravity of 1.43?
$$(8.34 * 1.43) \enspace lbs/gal*5 \enspace gallons = \boxed{59.6 \enspace lbs}$$

The weight of active ferric chloride in the drum will be 59.6*0.4=23.84 lbs (as ferric chloride is 40\% strength)

% \begin{tcolorbox}[
% colframe=blue!25,
% colback=blue!10,
% coltitle=blue!20!black,  
% title= Practice Problems]
% \begin{enumerate}
% \item What is the specific gravity of a 1 ft$^3$ concrete block which weighs 145 lbs?

% \item What is the specific gravity of a chlorine solution if 1 (one) gallon weighs 10.2lbs?

% \item How much does each gallon of zinc orthophosphate weigh (pounds) if it has a specific gravity of 1.46?

% \item How much does a 55 gallon drum of 25\% caustic soda weigh (pounds) if the specific gravity is 1.28?

% \end{enumerate}

% \end{tcolorbox}


\section{Concentration}\index{Concentration}
\begin{itemize}
\item Concentration is typically expressed as mg/l which is the weight of the constituent (mg) in 1 liter of water.
\item As 1 liter of water weighs 1 million mg, a concentration of 1 mg/l implies 1 mg of constituent per 1 million mg of water or one part per million (ppm).   \texthl{Thus, mg/l and ppm are synonymous.}
\item Sometimes the constituent concentration is expressed in terms of percentage.\\
\vspace{6pt}
\textbf{Example:} 12.5\% chlorine concentration solution.\\
\vspace{0.2cm}
100\% would mean 1,000,000 mg/l or 1,000,000 ppm\\
\vspace{0.2cm}
$\implies$1\% would be $\dfrac{1,000,000}{100}\textrm{mg/l} = \textrm{10,000 mg/l or 10,000 ppm}$\\
\vspace{0.2cm}
$\implies$12.5\% chlorine concentration is 125,000 mg/l or 125,000 ppm.
\vspace{6pt}

$1\% \enspace concentration = 10,000 \enspace ppm \enspace or \enspace\dfrac{mg}{l}$\\
$0.1\% \enspace concentration = 1,000 \enspace ppm \enspace or \enspace \dfrac{mg}{l}$\\
$0.01\% \enspace concentration = 100 \enspace ppm \enspace or \enspace \dfrac{mg}{l}$\\
$10\% \enspace concentration = 100,000 \enspace ppm \enspace or \enspace \dfrac{mg}{l}$\\
$5\% \enspace concentration = 50,000 \enspace ppm \enspace or \enspace \dfrac{mg}{l}$\\
$12.5\% \enspace concentration = 125,000 \enspace ppm \enspace or \enspace \dfrac{mg}{l}$\\
\end{itemize}

\vspace{0.3cm}
Above concepts are used for chemicals such as fluoride and hypochlorites - the strength of the product as used is commonly expressed as a percentage.
\vspace{0.3cm}

\textbf{Example 1:} A chlorine solution was made to have a $4 \%$ concentration. It is often desirable to determine this concentration in $\mathrm{mg} / \mathrm{L}$. This is relatively simple: the $4 \%$ is four percent of a million.

To find the concentration in $\mathrm{mg} / \mathrm{L}$ when it is expressed in percent, do the following:

\begin{enumerate}
  \item Change the percent to a decimal.
\end{enumerate}
$$
4 \% \div 100=0.04
$$

\begin{enumerate}
  \setcounter{enumi}{2}
  \item Multiply times a million.
\end{enumerate}
$$
0.04 \times 1,000,000=40,000 \mathrm{mg} / \mathrm{L}
$$
We get the million because a liter of water weighs $1,000,000 \mathrm{mg} .1 \mathrm{mg}$ in 1 liter is 1 part in a million parts ( $\mathrm{ppm}) .1 \%=10,000 \mathrm{mg} / \mathrm{L}$.


\textbf{Example 2:} How much $65 \%$ calcium hypochlorite is required to obtain 7 pounds of pure chlorine?\\
$65 \%$ implies that in every lb of calcium hypochlorite has $65 \%$ lbs of available chlorine.\\
\vspace{0.2cm}
Therefore, $\dfrac{0.65 \textrm{ lbs available chlorine}}{\textrm{lb of calcium hypochlorite}} $ or conversely $\dfrac{\textrm{lb of calcium hypochlorite}}{0.65 \textrm{ lbs available chlorine}}$\\
\vspace{0.2cm}
$\implies{\textrm{lbs calcium hypchlorite required}}=\dfrac{\textrm{lb of calcium hypochlorite}}{0.65 \cancel{\textrm{ lbs available chlorine}}}*\dfrac{7\cancel{\textrm{ lb of available chlorine}}}{}$\\
\vspace{0.2cm}
$=\boxed{10.8 \textrm{ lbs of calcium hypochlorite with } 65\%\textrm{available chlorine is required}}$

% \begin{tcolorbox}[
% colframe=blue!25,
% colback=blue!10,
% coltitle=blue!20!black,  
% title= Practice Problems]
% \begin{enumerate}
% \item What is the concentration in mg/l of  4.5\% solution of that substance.
% \item How many lbs of salt is needed to make 5 gallons of a 2,500mg/l solution
% \end{enumerate}
% \end{tcolorbox}


\section{Pounds Formula}\index{Pounds Formula}
\begin{itemize}
\item Pounds formula: 
$$lbs \enspace \textbf{or} \enspace \dfrac{lbs}{day}=Concentration\Big(\dfrac{mg}{l}\Big)*8.34*volume(MG) \enspace \textbf{or} \enspace Flow (MGD)$$\\
\item So if the concentration of a particular constituent (in mg/liter) and the volume or flow of wastewater is given, one can calculate the amount of that constituent or using this formula.\\
\texthl{Important notes:}\\
\begin{enumerate}
\item \texthl{The unit of the constituent loading rate will be in lbs per the unit of time the flow is expressed in.  So if the flow is in MG per day the calculated loading rate will be in lbs/day.  Likewise if the flow value used is in MG per minute, the calculated loading rate will be in lbs/min.}
\item \texthl{If volume is used, the calculated value will be the mass of the constituent in that volume.  If flow is used, the calculated value will be the mass of the constituent in that flow.}
\item \texthl{For the Pound Formula to work, the volume or flow needs to be expressed in MG.  Volume or flows in other units - gallons, $ft^3$ etc. needs to be converted to MG.}
\end{enumerate}

\item The formula assumes that all of the material found in water (TSS, BOD, MLSS, Chlorine, etc.) weighs the same as water, that is, $8.34$ pounds per gallon.
\item In the Pounds Formula, there are three variables – lbs, concentration and volume, and one constant - 8.34.  Knowing any of the two variables in the formula, one can calculate the third (unknown) variable by rearranging the equation.\\
\begin{figure}[h]
\begin{tikzpicture}
    \newcommand{\R}{3}

\path[help lines,step=.2] (0,0) grid (16,6);
\path[help lines,line width=.6pt,step=1] (0,0) grid (16,6);
%\foreach \x in {0,1,2,3,4,5,6,7,8,9,10,11,12,13,14,15,16}
%\node[anchor=north] at (\x,0) {\x};
%\foreach \y in {0,1,2,3,4,5,6}
%\node[anchor=east] at (0,\y) {\y};
%-------------CIRCLE-----------------------------------
\draw[black,fill=gray!10] (8,3) circle (\R);
\draw[black, very thick, rotate=0](5,3) -- (11,3);
\draw (8,4.5) node[text width=3cm,align=center]
  {\scriptsize{lbs or lbs/day}};
\draw (6.4,2) node[text width=3cm,align=center]
  {\scriptsize{Concentration\\mg/l}};
\draw (9.7,2) node[text width=3cm,align=center]
  {\scriptsize{Volume(MG)\\Flow(MGD)}};
  \draw (8,1)node[text width=3cm,align=center]
  {\scriptsize{8.34}};
\draw[black, very thick, rotate=0](6.4,0.5) -- (8,3);
\draw[black, very thick, rotate=0](9.6,0.5) -- (8,3);
  \node [circle split,draw,double,fill=red!20] at (4,3)
  {
    % No \nodepart has been used, yet. So, the following is put in the
    % ``text'' node part by default.
    $\div$
    \nodepart{lower} % Ok, end ``text'' part, start ``output'' part
    $=$
  };
  
    \node [circle split,draw,double,fill=red!20] at (5.8,-0.2)
  {
    % No \nodepart has been used, yet. So, the following is put in the
    % ``text'' node part by default.
    \scriptsize{$X$}
    \nodepart{lower} % Ok, end ``text'' part, start ``output'' part
    \tiny{$Multiply$}
  };
  
    \node [circle split,draw,double,fill=red!20] at (10,-0.2)
  {
    % No \nodepart has been used, yet. So, the following is put in the
    % ``text'' node part by default.
    \scriptsize{$X$}
    \nodepart{lower} % Ok, end ``text'' part, start ``output'' part
    \tiny{$Multiply$}
  };
\end{tikzpicture}
\caption{Davidson Pie}
\end{figure}
\vspace{0.2cm}
\item Davidson Pie provides a pictorial reference for calculating any unknown variable.  If for example, if Concentration is unknown, it can be calculated as follows: \\$$Concentration\Big(\dfrac{mg}{l}\Big)=\dfrac{lbs \enspace \textbf{or} \enspace \dfrac{lbs}{day}}{8.34*Volume(MG) \enspace \textbf{or} \enspace Flow (MGD)}$$\\
\vspace{0.2cm}
\item Likewise, if Volume (or Flow) is the unknown variable. it can be calculated as:  \\$$Volume (MG) \enspace or \enspace Flow(MGD)=\dfrac{lbs \enspace \textbf{or} \enspace \dfrac{lbs}{day}}{Concentration\Big(\dfrac{mg}{l}\Big)* \enspace 8.34  }$$
\vspace{0.2cm}
\item Pounds formula is used for:
\begin{itemize}
\item Calculating the quantity in pounds of a particular wastewater constituent entering or leaving a wastewater treatment process
\item Calculating the pounds of chemicals to be added\\
\end{itemize}
\end{itemize}


\textbf{Example 1:} If a 5 MGD flow is to be dosed with 25 mg/l of a certain chemical, calculate the lbs/day that chemical required.\\

Solution\\

Applying lbs formula:\\
$\dfrac{lbs}{day}=5 MGD *250\dfrac{mg}{l}*8.34 = \boxed{1,042\dfrac{lbs}{day}}$
\\
\vspace{6pt}
\textbf{Example 2:} Calculate the lbs of chemical in 7,500 gallons of 4.5\% active solution of that chemical.\\
Solution\\
Applying lbs formula:\\
$lbs chemical = \dfrac{7500}{1,000,000}MG * 4.5*10,000 *8.34 = \boxed{2,815 \enspace lbs \enspace chemical}$\\
\textbf{Note:}\\  
1) 7500 gallons was converted to MG by dividing by 1,000,000\\
$7500 \enspace gallons * \dfrac{1 MG}{1,000,000 \enspace gallon}$\\
2) 4.5\% was converted to mg/l by multiplying by 10,000 as 1\%=10,000mg/l

% \begin{tcolorbox}[
% colframe=blue!25,
% colback=blue!10,
% coltitle=blue!20!black,  
% title= Practice Problems]

% \begin{enumerate}

% \item A water treatment plant operates at the rate of 75 gallons per minute. They dose soda ash at
% 14 mg/L. How many pounds of soda ash will they use in a day?

% \item A water treatment plant is producing 1.5 million gallons per day of potable water, and
% uses 38 pounds of soda ash for pH adjustment. What is the dose of soda ash at that plant?

% \item A water treatment plant produces 150,000 gallons of water every day. It uses an
% average of 2 pounds of permanganate for iron and manganese removal. What is the dose of the
% permanganate? 

% \item A water treatment plant uses 8 pounds of chlorine daily and the dose is 17 mg/l. How
% many gallons are they producing?

% \item An operator mixes 40 lb of lime in a 100-gal tank containing 80 gal of water. What is the percent of lime in the slurry?

% \end{enumerate}
% \end{tcolorbox}

\section{Chemicals Related Math Problems}\index{Chemicals Related Math Problems}
\subsection{Chemical Dosing}\index{Chemical Dosing}

\begin{itemize}
\item Use lbs formula to calculate the lbs of chemicals required\\
\item Using the calculated lbs chemical required value, calculate the amount of that chemical at the concentration available
\end{itemize}

\textbf{Example 1:} If a 5 MGD flow is to be dosed with 25 mg/l of a certain chemical, calculate the lbs/day that chemical required.\\

Solution\\

Applying lbs formula:\\
$\dfrac{lbs}{day}=5 MGD *250\dfrac{mg}{l}*8.34 = \boxed{1,042\dfrac{lbs}{day}}$
\\
\vspace{6pt}
\textbf{Example 2:} Calculate the lbs of chemical in 7,500 gallons of 4.5\% active solution of that chemical.\\
Solution\\
Applying lbs formula:\\
$lbs chemical = \dfrac{7500}{1,000,000}MG * 4.5*10,000 *8.34 = \boxed{2,815 \enspace lbs \enspace chemical}$\\

\subsection{Chlorine dosing problems}\index{Chlorine dosing problems}
\textbf{Example 4:} Determine the chlorinator setting (lb/day) required to treat a flow of $4 \mathrm{MGD}$ with a chlorine dose of $5 \mathrm{mg} / \mathrm{L}$.

Chlorine feed rate $(\mathrm{lb} /$ day $)=$ Chlorine $(\mathrm{mg} / \mathrm{L}) \times$ Flow $(\mathrm{MGD}) \times 8.34 \mathrm{lb} / \mathrm{gal}$

Chlorine feed rate $(\mathrm{lb} /$ day $)=5 \mathrm{mg} / \mathrm{L} \times 4 \mathrm{MGD} \times 8.34 \mathrm{lb} / \mathrm{gal}$

Chlorine feed rate $(\mathrm{lb} /$ day $)=167 \mathrm{lb} /$ day

\textbf{Example 5:} A pipeline that is 12 inches in diameter and $1400 \mathrm{ft}$ long is to be treated with a chlorine dose of $48 \mathrm{mg} / \mathrm{L}$. How many lb of chlorine will this require?

First determine the gallon volume of the pipeline:

Volume $(\mathrm{gal})=0.785 \times \mathrm{D}^{2} \times$ length $(\mathrm{ft}) \times 7.48 \mathrm{gal} / \mathrm{cu} \mathrm{ft}$

Volume $(\mathrm{gal})=0.785 \times(1 \mathrm{ft})^{2} \times 1400 \mathrm{ft} \times 7.48 \mathrm{gal} / \mathrm{cu} \mathrm{ft}$ Volume $(\mathrm{gal})=8221 \mathrm{gal}$

Next calculate the amount of chlorine required:

Chlorine feed rate $(\mathrm{lb} /$ day $)=$ Chlorine $(\mathrm{mg} / \mathrm{L})$ x Flow $($ MGD) $\times 8.34 \mathrm{lb} / \mathrm{gal}$

Chlorine feed rate $(\mathrm{lb} /$ day $)=48 \mathrm{mg} / \mathrm{L} \times 0.008221 \mathrm{MGD} \times 8.34 \mathrm{lb} / \mathrm{gal}$

Chlorine feed rate $(\mathrm{lb} /$ day $)=3.3 \mathrm{lb}$

\textbf{Example 6:} A water sample is tested and found to have a chlorine demand of $1.7 \mathrm{mg} / \mathrm{L}$. If the desired chlorine residual is $0.9 \mathrm{mg} / \mathrm{L}$, what is the desired chlorine dose (in $\mathrm{mg} / \mathrm{L}$ )?

Chlorine Dose $(\mathrm{mg} / \mathrm{L})=$ Chlorine Demand $+$ Chlorine Residual

Chlorine Dose $(\mathrm{mg} / \mathrm{L})=1.7 \mathrm{mg} / \mathrm{L}+0.9 \mathrm{mg} / \mathrm{L}$

Chlorine $\operatorname{Dose}(\mathrm{mg} / \mathrm{L})=2.6 \mathrm{mg} / \mathrm{L}$

\textbf{Example 7:}\\
The chlorine dosage for water is $2.7 \mathrm{mg} / \mathrm{L}$. If the chlorine residual after a 30-minute contact time is found to be $0.7 \mathrm{mg} / \mathrm{L}$, what is the chlorine demand (in $\mathrm{mg} / \mathrm{L}$ )?

Chlorine Demand $=$ Chlorine Dose $-$ Chlorine Residual

Chlorine Demand $=2.7 \mathrm{mg} / \mathrm{L}-0.7 \mathrm{mg} / \mathrm{L}$

Chlorine Demand $=2.0 \mathrm{mg} / \mathrm{L}$

\textbf{Example 8:} How many gallons per day of bleach solution (SG 1.2)containing 12.5\% available chlorine is required to disinfect a 10 MGD flow of water given the required chlorine dosage of 7 mg/l.\\
\begin{enumerate}
\item Calculate the lbs of chlorine required using the lbs formula:\\
\vspace{0.5cm}
=$10 MGD \enspace * \enspace 7 \dfrac{mg}{l} \enspace * \enspace 8.34\enspace=\enspace 583.8 \enspace lbs \enspace chlorine \enspace per \enspace day$\\
\vspace{0.5cm}
\item Calculate the gallons of bleach which will provide the 583.8 lbs chlorine\\
\vspace{0.5cm}
Applying the lbs formula - note that 8.34 * SG will give the actual lbs/gal of bleach.  If SG is not provided, use only 8.34 lbs per gallon:\\
\vspace{0.5cm}
$583.8 \dfrac{lbs \enspace bleach}{day}\enspace=\enspace x \dfrac{gal}{day} \enspace * \enspace 8.34 * 1.2 \dfrac{lbs \enspace bleach}{gal} \enspace * \enspace 0.0125 \dfrac{lbs \enspace chlorine}{lb \enspace bleach} \enspace $\\
\vspace{0.5cm}
$ \implies x \dfrac{gal}{day}\enspace = \enspace \dfrac{583.8}{8.34*1.2*0.125} \enspace = \boxed{467 \dfrac{gal}{day}}$
\end{enumerate}
\vspace{0.3cm}
\textbf{The above problem can be solved directly using the formula below given in the SWRCB Water Treatment Exam Formula Sheet.}\\
\vspace{0.3cm}
 $\textrm{GPD}=\dfrac{\textrm{(MGD)}*\textrm{(ppm or mg/l)}*8.34 \enspace \textrm{lbs/gal}}{\textrm{\% \enspace purity}*\textrm{Chemical \enspace Wt. (lbs/gal)}}$ 
 \vspace{0.3cm}
 $\textrm{GPD}=\dfrac{10*7*8.34}{0.125*(1.2*8.34)}=\boxed{467 \dfrac{\textrm{gal}}{\textrm{day}}}$ 

% \begin{tcolorbox}[
% colframe=blue!25,
% colback=blue!10,
% coltitle=blue!20!black,  
% title= Practice Problems]
% \begin{enumerate}

%   \item Determine the chlorinator setting in pounds per day if a water plant produces $300 \mathrm{gpm}$ and the desired chlorine dose is $2.0 \mathrm{mg} / \mathrm{L}$.

%   \item The finished water chlorine demand is $1.2 \mathrm{mg} / \mathrm{L}$ and the target residual is $2.0 \mathrm{mg} / \mathrm{L}$. If the plant flow is $5.6 \mathrm{mgd}$, how many pounds per day of $65 \%$ hypochlorite solution will be required?

%   \item Fluoride is added to finished water at a dose of $4 \mathrm{mg} / \mathrm{L}$. Find the feed rate setting for a fluoride saturator in gal/min if the water plant produces $5 \mathrm{mgd}$.

%   \item If chlorine costs $\$ 0.21$ per pound, what is the daily cost to chlorinate a $5 \mathrm{mgd}$ flow rate at a dosage of $2.6 \mathrm{mg} / \mathrm{L}$ ?

%   \item One gallon of sodium hypochlorite laundry bleach, with $5.25 \%$ available chlorine, contains how many pounds of active chlorine?


% \end{enumerate}
% \end{tcolorbox}


\subsection{Blending and Dilution Calculations}\index{Blending and Dilution Calculations}
\begin{itemize}
\item Blending and dilution calculations apply to the following scenarios:
\begin{itemize}
\item Blending involves mixing two streams - each with a different concentration of contaminant/chemical, to obtain a certain volume or flow containing the target concentration of contaminant/chemical.  For example: \textit{Finding the correct blend of two source water streams - one with 15 mg/L of iron and other containing  4 mg/L of iron to get a 100 gpm product water containing 8 mg/l of iron.} \textbf{OR}\\
\textit{Calculating the actual combined TDS concentration obtained by mixing two known flows with known TDS concentrations.}
\item Dilution involves makedown of a higher concentration of a chemical to a lower concentration using water as a dilutant.   For example: \textit{How much initial volume of a 4\% polymer solution is needed to make 3500 gallons of polymer at 0.25\% concentration?}\\
\end{itemize}
\item These type of problems are solved using C*V relationship where:
\begin{itemize}
 \item C is the concentration expressed in ppm or mg/l or as \% purity.
 \item V is either the volume or flow.
\item The product - C*V - $\dfrac{\textrm{\textrm{mass}}}{\textrm{volume}/\textrm{flow}}*\textrm{volume/flow} = \textrm{mass}$  
\end{itemize}
\item For blended streams, the sum of the mass from each of the two source streams will equal to the mass in the target stream:

\item Thus, \textbf{for blending calculations}, if:\\

C$_1$ and V$_1$ is the concentration and volume respectively of the one of the sources streams and\\
\vspace{0.2cm}
 C$_2$ and V$_2$ is the concentration and volume respectively of the second source stream, and \\
 \vspace{0.2cm}
C$_3$ and and V$_3$ is the concentration and volume respectively of the target stream\\
\vspace{0.3cm}
The sum of the mass from each of the two source streams will equal to the mass in the target stream:\\
\vspace{0.3cm}
\textbf{C$_1$ * V$_1$ + C$_2$ * V$_2$ =  C$_3$ * V$_3$.}\\
\vspace{0.3cm}
This equation can be manipulated algebraically to calculate anyone of the unknown values in the equation.\\
\vspace{0.2cm}
Also, any of the three volume variables can be expressed as the sum or difference of the other two - , or V$_1$ + V$_2$ = V$_3$ or V$_1$ = V$_3$ - V$_2$ or V$_2$ = V$_T$ - V$_1$\\

\item \textbf{For dilution}, the mass of the target chemical will remain the same, as only water is added to the source (concentrated chemical).
\item Thus, for dilution calculations, if:\\
\vspace{0.2cm}
C$_1$ and V$_1$ is the concentration and volume respectively of the concentrated product used for the dilution, and\\
\vspace{0.2cm}
C$_2$ and V$_2$ is the concentration and volume of the resultant product after dilution with water\\
\vspace{0.2cm}
The mass of the target chemical in the volume of the concentrated product used for dilution will remain the same in the final diluted product:\\
\vspace{0.3cm}
\textbf{C$_1$ * V$_1$ =  C$_2$ * V$_2$.}\\

\end{itemize}

\textbf{Example Problem \#1:} Two wells are used to satisfy demand during the summer months. One well produces water that contains 22 mg/L of Arsenic. The other well produces water that contains 3 mg/L of Arsenic. If the total demand for water is 400 gpm and the target Arsenic concentration in the finished water is 8 mg/L, what is the highest pumping rate possible for the first well?\\
\vspace{0.3cm}
\textbf{Solution:}\\
C$_1$ * V$_1$ + C$_2$ * V$_2$ =  C$_3$ * V$_3$\\
\vspace{0.3cm}
Thus 22 * V$_{22}$ + 3 * V$_3$ =  8 * V$_8$\\
\vspace{0.3cm}

V$_{22}$ + V$_3$ = V$_8$ = 400 gpm\\
\vspace{0.3cm}
As we want to solve for V$_{22}$, we can express V$_3$ as: V$_3$ = 400-V$_{22}$\\
\vspace{0.3cm}
Thus, 22 * V$_{22}$ + 3 * (400-V$_{22}$) =  8 * 400=3,200\\
\vspace{0.3cm}
22V$_{22}$ + 1200-3V$_{22}$ =  3,200\\
\vspace{0.3cm}
V$_{22}$(22-3) =  2,000\\
\vspace{0.3cm}
V$_{22}$ = $ \dfrac{2,000}{19}=\boxed{105.3 \enspace gpm}$\\
\vspace{0.3cm}
Also, V$_3$=400-105.3=294.7\\
\vspace{0.3cm}

NOTE:  If one does not want to utilize algebraic manipulation, one may memorize the following formula:\\
\vspace{0.3cm}
$V_{1/2}=\dfrac{\lvert C_3 - C_{2/1}\rvert*V_3}{C_1-C_2}$\\
\vspace{0.3cm}
Applying the formula above to Example Problem \#2:\\
\vspace{0.3cm}
$V_{22}=\dfrac{\lvert 8 - 3\rvert*400}{22-3}=\boxed{105.3 \enspace gpm}$\\
\vspace{0.3cm}
$V_{3}=\dfrac{\lvert 8 - 22\rvert*400}{22-3}=\boxed{294.7 \enspace gpm}$\\
\vspace{0.3cm}
\textbf{Example Problem \#2:}  How many gallons of a 4\% polymer solution is required to make a 3,500 gallon batch of 0.25\% polymer solution.\\

\textbf{Solution:}\\
\vspace{0.3cm}
Here, we are adding water - which has zero percent of polymer concentration to the 4\% polymer to make a 0.25\% polymer solution.\\
\vspace{0.3cm}
C$_1$ * V$_1$ = C$_2$ * V$_2$\\
\vspace{0.3cm}
C$_{4\%}$ * V$_{4\%}$ =  C$_{0.25\%}$ * V$_{0.25\%}$\\
\vspace{0.3cm}
4 * V$_{4\%}$ =  0.25 * 3,500\\
\vspace{0.3cm}
$\implies V_{4\%} = \dfrac{0.25 \enspace * \enspace 3500}{4}= \boxed{219 \enspace\textrm{gal}} $\\
\vspace{0.3cm}
Take 219 gallons of the 4\% polymer and dilute to 3,500 gallons to give a 0.25\% polymer solution.\\

% \begin{tcolorbox}[
% colframe=blue!25,
% colback=blue!10,
% coltitle=blue!20!black,  
% title= Practice Problems]
% \begin{enumerate}
% \item Ferric chloride is being added as a coagulant to the raw water entering a plant. Sampling
% shows that the concentration of ferric in the raw water is 25 ppm. A quick check of the chemical
% metering pump shows that it is operating at a flow rate of 4.3 gpm. If the flow through the water
% plant is 800 gpm, what is the concentration of raw chemical in the dosing tank?

% \item A water plant is fed by two different wells. The first well produces water at a rate of 600
% gpm and contains arsenic at 0.5 mg/L. The second well produces water at a rate of 350 gpm and
% contains arsenic at 12.5 mg/L. What is the arsenic concentration of the blended water?
% \end{enumerate}
% \end{tcolorbox}