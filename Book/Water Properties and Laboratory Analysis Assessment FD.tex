Which of the following is an indicator organism?\\
Giardia\\
Cryptosporidium\\
Hepatitis\\
E. Coli\\
	What is the primary origin of coliform bacteria in water supplies?\\
	Natural algae growth\\
	Industrial solvents\\
	Animal or human feces\\
	Acid rain\\
	What ls the term for water samples collected at regular intervals and combined in equal volume with each other?\\
	Time grab samples\\
	Time flow samples\\
Proportional time composite samples\\
	What is the basis for the number of samples that must be collected for utilities monitoring for lead and copper that are in compliance or have installed corrosion control'?\\
	Size of distribution system\\
	Population\\
	Amount of water produced\\
	Number of raw water sources\\
	Where should bacteriological samples be collected in the distribution system?\\
	Uniformly distributed throughout the system based on area\\
	At locations that are representative of conditions within the system\\
	Always from extreme locations in the system but occasionally at other locations\\
	Uniformly throughout the system based on population density\\
	The	quantity of oxygen. that can remain dissolved in water is related to\\
	Temperature\\
	pH\\
	Turbidity\\
	Alkalinity\\
	In coliform analysis using the presence-absence test, a sample should be incubated for	\\
	24 hours at 25°C\\
	36 hours at 35°C\\
	24 and 36 hours at 25°0\\
	24 and 48 hours at 35°C\\
A major source of error when obtaining water quality information is improper:\\
Sampling\\
Preservation\\
Tests of samples\\
Reporting of data\\
What is commonly used as an indicator of potential contamination in drinking water samples?\\
Viruses\\
Coliform bacteria\\
Intestinal parasites\\
Pathogenic organisms\\
The type of organisms that can cause disease are said to be \rule{2cm}{0.3pt}\\
microorganisms.\\
Bad\\
Pathogenic\\
Undesirable\\
Sick\\
Four types of aesthetic contaminants in water include the following:\\
Odor, turbidity, color, hydrogen sulfide gas\\
Pathogens, microorganisms, arsenic, disinfection by-products\\
Odor, color, turbidity, hardness\\
Color, pathogens, metals, organics\\
What is the purpose of adding fluoride to drinking water?\\
Increase tooth decay\\
Reduce tooth decay\\
Make teeth white\\
Government conspiracy\\
The test used to determine the effectiveness of disinfection is called the:\\
Coliform bacteria test\\
Color test\\
Turbidity test\\
Particle test\\
Turbidity is measured as:\\
mg/L\\
mL\\
gpm\\
NTU\\
Giardia and cryptosporidium are a type of:\\
Mineral\\
Organism\\
Color\\
Bird\\
Chronic contaminants are those that can cause sickness after:\\
Prolonged exposure\\
Low levels or low exposure\\
A positive total coliform test indicates that:\\
Disease-causing organisms may be present in the water supply\\
The water is safe to consume\\
The water supply has high iron levels\\
There is nothing to be concerned about\\
What is the purpose of the bacteriological site sampling plan?\\
To have a map showing where BacT samples are drawn\\
In case of a positive Bac T sample, the operator will know where to take the\\
four repeat samples\\
The state will know where you are taking your repeat samples\\
All of the above\\
To ensure that the water supplied by a public water system meets state requirements, the water system operator must regularly collect samples and:\\
Have water analyzed at an approved water testing laboratory\\
Determine a sampling schedule based on state requirements\\
Send all analyses results to the state\\
All of the above\\
Samples taken for routine bacteriological testing should be preserved by:\\
Freezing\\
Boiling\\
DPD preservative\\
Refrigeration\\
How many coliform samples are required per month for a water system serving a population between 25 and 100?\\
1\\
2\\
3\\
4\\
Before taking a bacteriological (BacT) water sample from a faucet, you should:\\
Wash hands thoroughly\\
Remove the faucet aerator\\
Flush water until you’re sure water is from the main, not the service line\\
All of the above\\
Monthly BacT samples should be taken from:\\
The well pump house\\
The distribution system\\
The treatment plant\\
An outside hose spigot\\
If your BacT sample test is positive, how long do you have to collect four repeat samples and deliver them to the lab?\\
12 hours\\
24 hours\\
48 hours\\
72 hours\\
\rule{2cm}{0.3pt}is a measure of the capacity of water to neutralize acids.\\
Concentration\\
Alkalinity\\
pH\\
Conductivity\\
The DPD method is used to determine the \rule{2cm}{0.3pt} of a water sample.\\
Dissolved oxygen content\\
Conductivity\\
pH\\
Free chlorine residual\\
What color does N,N-diethyl-p-phenylenediamine (DPD) turn in the presence of\\
chlorine?\\
Brown\\
Green\\
Blue\\
Pink\\
 The presence-absence (P-A) test used for microbiological testing is also commonly referred to as\\
Multiple Tube Fermentation\\
Membrane Filtration\\
Confirmed Test\\
Colilert\\
 When testing for coliform bacteria with the multiple tube fermentation (MFT) method what is the best indicator for a positive test?\\
Color change\\
Gas bubble formation\\
Formation of a cyst\\
Formation of turbidity\\
 Coliform bacteria share many characteristics with pathogenic organisms. Which of the following is not true?\\
They survive longer in water\\
They grow in the intestines\\
There are less coliform than pathogenic organisms\\
They are still present in water without fecal contamination\\
 What is the second step in the multiple tube fermentation test?\\
Presumptive test\\
Negative test\\
Completed\\
Confirmed\\
What is the removal and deactivation requirement for Giardia?\\
$2 \mathrm{log}$\\
$3 \mathrm{log}$\\
$4 \mathrm{log}$\\
There is no requirement\\
 The multiple barrier approach to water treatment includes removal through which method?\\
Filtration\\
Coagulation\\
Disinfection\\
a and c\\
 A pH reading of 7 is considered\\
Slightly acidic\\
Acidic\\
Basic\\
Neutral\\
EDTA titration is used to determine the \rule{2cm}{0.3pt} of a water sample.\\
Hardness\\
Conductivity\\
Alkalinity\\
Free chlorine residual\\
 A higher than normal turbidity reading could signify\\
A change in water quality\\
Nothing. Keep operating as normal\\
Microbiological contamination\\
Both $A$ \& $C$\\
 What is the ingredient used during the second multiple tube fermentation test?\\
Colilert\\
MMO/MUG\\
Brilliant Green Bile\\
Chlorine\\
When collecting a distribution system sample for bacteriological testing, the person collecting the sample should allow the water to run before filling the sample bottle.\\
A minimum of five minutes.\\
1 hr.\\
30 min\\
only a few seconds\\
Black stains on plumbing fixtures might be attributed to\\
calcium.\\
copper.\\
magnesium.\\
manganese.\\
The multiple tube fermentation test consists of three distinct tests. These tests, in the order performed, are the:\\
preliminary, confirmed, and completed tests.\\
preliminary, presumptive and confirmed tests.\\
presumptive, confirmed, and completed tests.\\
prespumtive, preliminary, and completed tests.\\
What should the sample volume be when testing for total coliform bacteria?\\
l00mL\\
250mL\\
500mL\\
1,000mL\\
$\mathrm{pH}$ is a measure of :\\
a. conductivity\\
b. water's ability to neutralize acid\\
c. hydrogen ion activity\\
d. dissolved solids\\
 Sodium Thiosulfate is used to\\
a. Buffer chlorine solutions\\
b. Neutralize chlorine residuals\\
c. Detect chlorine leaks\\
d. Sterilize sample bottles\\
  The presence of total coliforms in drinking water indicates\\
a. The presence of pathogens.\\
b. The absence of an adequate chlorine residual\\
c. The existence of an urgent public health problem\\
d. The potential presence of pathogens\\
A primary health risk associated with microorganisms in drinking water is\\
a. Cancer\\
b. Acute gastrointestinal diseases\\
c. Birth defects\\
d. Nervous system disorders\\
  After 5 years use, a portion of cast iron pipe shows a white scale about $1 / 2$ inch thick lining the inside. This means\\
a. Red water will soon become a problem\\
b. The water has been corrosive\\
c. The water is chemically unstable and is depositing\\
d. Water should flow easier since the lining is smooth\\
  Hardness in water is caused by\\
a. Dissolved minerals\\
b. High $\mathrm{pH}$.\\
c. Low turbidity\\
d. Alkalinity\\
  The meniscus on calibrated glassware is read at the\\
a. Bottom of curvature for mercury but the top for water\\
b. Extreme point of contact between the liquid and glass, i.e., where gas, liquid, and air all meet at one point\\
c. Mid-height of the curvature so that beginning and ending readings will results in zero error\\
d. Top of curvature for mercury but at the bottom for most other liquids including water\\
  An unknown substance is found on the bottom of the water within a drinking water reservoir. Which of the following statements is true of this substance?\\
a. It has a specific gravity less than $1.0$\\
b. It has a specific gravity equal to $1.0$\\
c. It has a specific gravity greater than $1.0$\\
d. It has no specific gravity\\
e. None of the above\\
  The term "Chain of Custody" refers to\\
a. A large accessory to a come-along\\
b. An attachment to a pipe-cutter\\
c. Employee labor laws\\
d. Procedures and documentation required for water quality sampling\\
e. Procedures and documentation required for chemical application\\
  Water samples to be analyzed for taste and odor must be\\
a. Analyzed in the field\\
b. Collected in glass sample containers\\
c. Dechlorinated with sodium thiosulfate\\
d. Preserved with dilute hydrochloric acid\\
e. None of the above\\
  Bacteriological samples for a distribution system must be collected in accordance with\\
a. The Surface Water Treatment Rule\\
b. OSHA requirements\\
c. An approved sample siting plan\\
d. FLSA requirements\\
e. ANSI/NSF Standard 61\\
  Trihalomethanes are classified as\\
a. Metals\\
b. Inorganic constituents\\
c. Secondary drinking water standards\\
d. Radiological contaminants\\
e. Volatile organic compounds\\
 The multiple tube fermentation analysis consists of\\
a. Positive, negative, and neutral tests\\
b. Presumptive, confirmed, and completed tests\\
c. Preliminary, presumptive, and confirmed tests\\
d. Preliminary, confirmed, and completed tests\\
e. Presence or absence testing\\
  Which of the following is NOT a characteristic of coliform organisms?\\
a. Intestinal origin\\
b. Will produce carbon dioxide from lactose\\
c. Heartier in a water environment than pathogenic organisms\\
d. Far less numerous than pathogenic organisms\\
e. Able to survive with or without oxygen\\
  A bacteriological test that measures only the presence or absence of coliforms is\\
a. ColiLert (MMO/MUG)\\
b. Multiple tube fermentation\\
c. Most probable number (MPN)\\
d. Membrane filtration\\
e. Presumptive test\\
  After collection, if stored at $4^{\circ} \mathrm{C}$, bacteriological samples must be processed within\\
a. 1 hour\\
b. 6 hours\\
c. 24 hours\\
d. 48 hours\\
e. 72 hours\\
  Sample bottles which are furnished by a certified laboratory for collection of bacteriological samples\\
a. Should be rinsed with the water to be sampled before use\\
b. Should be placed in boiling water for at least 10 minutes before use\\
c. Should be rinsed with a chlorine solution before use\\
d. Should be rinsed with distilled water before use\\
e. Are ready to use\\
The standard indicator of potential fecal contamination of a water supply is\\
a. Cryptosporidium\\
b. $\mathrm{pH}$\\
c. Alkalinity\\
d. Hardness\\
e. Coliform Presence - Absence\\
Where should bacteriological samples be collected?\\
a. At different locations on each sampling cycle, to make sure the entire system is sampled\\
b. Only from public locations, such as drinking fountains and restrooms\\
c. Only from locations owned by consumers\\
d. Only from specially constructed sampling stations\\
e. From several sampling locations around the entire distribution system, in accordance with a DHS-approved sample siting plan\\
Storage of bacteriological samples during transport to a laboratory is best accomplished using\\
a. A clean storage box specifically designed to hold sample containers\\
b. An ice chest packed with ice\\
c. An insulated storage box with "blue ice".\\
d. An insulated storage box with "dry ice"\\
e. No particular sample storage requirements apply, as long as the samples can be delivered to a laboratory prior to the end of the work day\\
  Sodium thiosulfate is added in the laboratory to bacteriological sample bottles to:\\
a. Thoroughly disinfect the sample bottle\\
b. -Complete the cleaning and sterilization process\\
c. Neutralize any residual chlorine present in the sample at the time of collection\\
d. Counteract the effects of sunlight on the water sample\\
e. Prevent further growth of bacteria in water samples following collection\\
  Radiological contaminant concentrations in drinking water are measured in\\
a. Milligrams per liter\\
b. Micrograms per liter\\
c. Nanograms per liter\\
d. Picograms per liter\\
e. None of the above\\
  Which of the following is NOT a characteristic of coliform organisms?\\
a. Intestinal origin\\
b. Will produce carbon dioxide from lactose\\
c. Heartier in a water environment than pathogenic organisms\\
d. Far less numerous than pathogenic organisms\\
e. Able to survive with or without oxygen\\
A water supply is found to have a calcium carbonate concentration of 50 mg/L. This water would be considered\\
a.	soft water\\
b.	hard water\\
c.	potable water\\
d.	non-potable water\\
Cathodic protection refers to protection against\\
a.	contamination\\
b.	corrosion\\
c.	hardness\\
d.  alkalinity\\
An operator uses \rule{1.5cm}{0.3pt} to test for residual chlorine\\
a. DPD\\
b. Cresol red\\
c. Methyl orange\\
d. Sulfuric acid\\
The meniscus on calibrated glassware is read at the:\\
a. Bottom of curvature for mercury but the top for water\\
b. Extreme point of contact between the liquid and glass, i.e., where gas, liquid, and air all meet at one point\\
c. Mid-height of the curvature so that beginning and ending readings will results in zero error\\
d. Top of curvature for mercury but at the bottom for most other liquids including water\\
The type of corrosion caused by the use of dissimilar metal in a water system is\\
a. Caustic corrosion\\
b. Galvanic corrosion\\
c. Oxygen corrosion\\
d. Tubercular corrosion\\
Which of the following can cause tastes and odors in a water supply?\\
a. Dissolved zinc\\
b. Algae\\
c. High pH\\
d. Low pH\\
The primary health risk associated with volatile organic chemicals.(VOCs) is\\
a. Cancer\\
b. Acute respiratory diseases\\
c. "Blue baby" syndrome\\
d. Reduced IQ. in children \\
Lead in drinking water can result in\\
a. Impaired mental functioning in children\\
b. Prostate cancer in men\\
c. Stomach and intestinal disorders\\
d. Reduced white blood cell count\\
Sodium thiosulfate is used to\\
a. Buffer chlorine solutions\\
b. Neutralize chlorine residuals\\
c. Raise pH\\
d. Sterilize sample bottles\\
Cathodic protection means protection against\\
a. contamination\\
b. corrosion\\
c. hardness\\
d. infiltration\\
A water supply is found to have a calcium carbonate concentration of 50 mg/l. This water would be considered\\
a. soft water\\
b. hard water\\
c. potable water\\
d. non-potable water\\
The main characteristic of raw water that enables algae to grow is\\
a. Presence of copper sulfate\\
b. Low pH\\
c. High hardness\\
d. Presence of nutrients\\











