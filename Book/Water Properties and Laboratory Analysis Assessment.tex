%\chapterimage{QuizCover} % Chapter heading image

%\chapter*{Properties and Laboratory Analysis}
%\textbf{Multiple Choice}
\begin{enumerate}[1.]










  \item Hard water contains an abundance of\\
a. sodium\\
b. iron\\
c. lead\\
d. *calcium carbonate\\

  \item A specific class of bacteria that only inhibit the intestines of warm-blooded animals is referred to as?\\
a. Eutrophic\\
b. Grazing\\
c. Salmonella\\
d. *Fecal coliform\\
e. pathogenic\\


\item Water with a $\mathrm{pH}$ of 8.0 is considered to be\\
a. acidic\\
b. *basic or alkaline\\
c. neutral\\
d. undrinkable\\

  \item Over which water quality indicator do operators have the greatest control?\\
a. alkalinity\\
b. $\mathrm{pH}$\\
c. temperature\\
d. *turbidity\\
  \item Which piece of laboratory equipment is used to titrate a chemical reagent?\\
a. graduated cylinder\\
b. *burette\\
c. pipet\\
d. Buchner funnel\\
  \item Which $\mathrm{pH}$ range is generally accepted as most palatable (drinkable)?\\
a. *6.5 to 8.5\\
b. 4.5 to 6.5\\
c. 8.5 to 9.5\\
d. 9.5 and above\\
e. all of the above 

\item Which of the following conditions is favorable for the rapid growth of algal?\\
a. *moderate to high dissolved oxygen and nutrients\\
b. high $\mathrm{pH}$ and water hardness\\
c. low temperatures and low dissolved oxygen\\
d. high alkalinity and water hardness\\

  \item Which of the following is the name given for a turbidity meter that has reflected or scattered light off suspended particles as a measurement?\\
a. HACH colorimeter\\
b. spectrophotometer\\
c. Wheaton bridge\\
d. *Nephelometer\\

  \item Water hardness is the measure of the concentrations of and dissolved in the water sample.\\
a. iron , manganese\\
b. nitrates, nitrites\\
c. sulfates, bicarbonates\\
d. *calcium \& magnesium carbonates\\
e. ferric chlorides and polymers\\

  \item The electrical potential required to transfer electrons from one compound or element to another is commonly referred to as\\
a. *oxidation-reduction potential (ORP)\\
b. voltage potential $(\mathrm{OHM} / \mathrm{P})$\\
c. resistance-impedance potential\\
d. microMho differential 

\item Water has physical, chemical, and biological characteristics. Which of the following is a
physical characteristic?
a. Coliform
b. *Turbidity
c. Hardness
d. All the above

  \item Tastes and odors in surface water are most often caused by:

a. clays

b. hardness

c. *algae

d. coliform bacteria

  \item Which of the following elements cause hardness in water?

a. sodium and potassium

b. *calcium and magnesium

c. iron and manganese

d. turbidity and suspended solids

  \item When measuring for free chlorine residual, which method is the quickest and simplest?\\
a. DPD color comparater\\
b. Orthotolidine method\\
c. Amperometric titration\\
d. 1, 2 nitrotoluene di-amine method


  \item Which water quality parameter requires a grab sample because it cannot be collected as a composite sample?\\
a. $\mathrm{pH}$\\
b. Iron\\
c. Nitrate\\
d. Zinc



    \item If a water sample is not analyzed immediately for chlorine residual, it is acceptable if it is analyzed within\\
a. 10 minutes.\\
b. 15 minutes.\\
c. 20 minutes.\\
d. 30 minutes.

  \item The volume of a sample for coliform compliance is\\
a. $100 \mathrm{~mL}$.\\
b. $200 \mathrm{~mL}$.\\
c. $300 \mathrm{~mL}$.\\
d. 0 ; there is no volume compliance for coliforms.

\item Which of the following is an indicator organism?
\begin{enumerate}
\item Giardia
\item Cryptosporidium
\item Hepatitis
\item E. Coli
\end{enumerate}

\item 	What is the primary origin of coliform bacteria in water supplies?
\begin{enumerate}
\item 	Natural algae growth
\item 	Industrial solvents
\item 	Animal or human feces
\item 	Acid rain
\end{enumerate}

\item 	What ls the term for water samples collected at regular intervals and combined in equal volume with each other?
\begin{enumerate}
\item 	Time grab samples
\item 	Time flow samples
\item Proportional time composite samples
\end{enumerate}

\item 	What is the basis for the number of samples that must be collected for utilities monitoring for lead and copper that are in compliance or have installed corrosion control'?
\begin{enumerate}
\item 	Size of distribution system
\item 	Population
\item 	Amount of water produced
\item 	Number of raw water sources
\end{enumerate}

\item 	Where should bacteriological samples be collected in the distribution system?
\begin{enumerate}
\item 	Uniformly distributed throughout the system based on area
\item 	At locations that are representative of conditions within the system
\item 	Always from extreme locations in the system but occasionally at other locations
\item 	Uniformly throughout the system based on population density
\end{enumerate}
 
\item 	The	quantity of oxygen. that can remain dissolved in water is related to
\begin{enumerate}
\item 	Temperature
\item 	pH
\item 	Turbidity
\item 	Alkalinity
\end{enumerate}

\item 	In coliform analysis using the presence-absence test, a sample should be incubated for	
\begin{enumerate}
\item 	24 hours at 25°C
\item 	36 hours at 35°C
\item 	24 and 36 hours at 25°0
\item 	24 and 48 hours at 35°C
\end{enumerate}

\item A major source of error when obtaining water quality information is improper:
\begin{enumerate}
\item Sampling
\item Preservation
\item Tests of samples
\item Reporting of data
\end{enumerate}

\item What is commonly used as an indicator of potential contamination in drinking water samples?
\begin{enumerate}
\item Viruses
\item Coliform bacteria
\item Intestinal parasites
\item Pathogenic organisms
\end{enumerate}

\item The type of organisms that can cause disease are said to be \rule{2cm}{0.3pt}
microorganisms.
\begin{enumerate}
\item Bad
\item Pathogenic
\item Undesirable
\item Sick
\end{enumerate}

\item Four types of aesthetic contaminants in water include the following:
\begin{enumerate}
\item Odor, turbidity, color, hydrogen sulfide gas
\item Pathogens, microorganisms, arsenic, disinfection by-products
\item Odor, color, turbidity, hardness
\item Color, pathogens, metals, organics
\end{enumerate}

\item What is the purpose of adding fluoride to drinking water?
\begin{enumerate}
\item Increase tooth decay
\item Reduce tooth decay
\item Make teeth white
\item Government conspiracy
\end{enumerate}

\item The test used to determine the effectiveness of disinfection is called the:
\begin{enumerate}
\item Coliform bacteria test
\item Color test
\item Turbidity test
\item Particle test
\end{enumerate}

\item Turbidity is measured as:
\begin{enumerate}
\item mg/L
\item mL
\item gpm
\item NTU
\end{enumerate}

\item Giardia and cryptosporidium are a type of:
\begin{enumerate}
\item Mineral
\item Organism
\item Color
\item Bird
\end{enumerate}

\item Chronic contaminants are those that can cause sickness after:
\begin{enumerate}
\item Prolonged exposure
\item Low levels or low exposure
\end{enumerate}

\item A positive total coliform test indicates that:
\begin{enumerate}
\item Disease-causing organisms may be present in the water supply
\item The water is safe to consume
\item The water supply has high iron levels
\item There is nothing to be concerned about
\end{enumerate}

\item What is the purpose of the bacteriological site sampling plan?
\begin{enumerate}
\item To have a map showing where BacT samples are drawn
\item In case of a positive Bac T sample, the operator will know where to take the
four repeat samples
\item The state will know where you are taking your repeat samples
\item All of the above
\end{enumerate}
\item To ensure that the water supplied by a public water system meets state requirements, the water system operator must regularly collect samples and:
\begin{enumerate}
\item Have water analyzed at an approved water testing laboratory
\item Determine a sampling schedule based on state requirements
\item Send all analyses results to the state
\item All of the above
\end{enumerate}
\item Samples taken for routine bacteriological testing should be preserved by:
\begin{enumerate}
\item Freezing
\item Boiling
\item DPD preservative
\item Refrigeration
\end{enumerate}

\item How many coliform samples are required per month for a water system serving a population between 25 and 100?
\begin{enumerate}
\item 1
\item 2
\item 3
\item 4
\end{enumerate}

\item Before taking a bacteriological (BacT) water sample from a faucet, you should:
\begin{enumerate}
\item Wash hands thoroughly
\item Remove the faucet aerator
\item Flush water until you’re sure water is from the main, not the service line
\item All of the above
\end{enumerate}

\item Monthly BacT samples should be taken from:
\begin{enumerate}
\item The well pump house
\item The distribution system
\item The treatment plant
\item An outside hose spigot
\end{enumerate}

\item If your BacT sample test is positive, how long do you have to collect four repeat samples and deliver them to the lab?
\begin{enumerate}
\item 12 hours
\item 24 hours
\item 48 hours
\item 72 hours
\end{enumerate}

\item \rule{2cm}{0.3pt}is a measure of the capacity of water to neutralize acids.
\begin{enumerate}
\item Concentration
\item Alkalinity
\item pH
\item Conductivity
\end{enumerate}

\item The DPD method is used to determine the \rule{2cm}{0.3pt} of a water sample.
\begin{enumerate}
\item Dissolved oxygen content
\item Conductivity
\item pH
\item Free chlorine residual
\end{enumerate}

\item What color does N,N-diethyl-p-phenylenediamine (DPD) turn in the presence of
chlorine?
\begin{enumerate}
\item Brown
\item Green
\item Blue
\item Pink
\end{enumerate}

\item  The presence-absence (P-A) test used for microbiological testing is also commonly referred to as\\
\begin{enumerate}
\item Multiple Tube Fermentation\\
\item Membrane Filtration\\
\item Confirmed Test\\
\item Colilert
\end{enumerate}

\item  When testing for coliform bacteria with the multiple tube fermentation (MFT) method what is the best indicator for a positive test?\\
\begin{enumerate}
\item Color change\\
\item Gas bubble formation\\
\item Formation of a cyst\\
\item Formation of turbidity
\end{enumerate}

\item  Coliform bacteria share many characteristics with pathogenic organisms. Which of the following is not true?\\
\begin{enumerate}
\item They survive longer in water\\
\item They grow in the intestines\\
\item There are less coliform than pathogenic organisms\\
\item They are still present in water without fecal contamination
\end{enumerate}

\item  What is the second step in the multiple tube fermentation test?\\
\begin{enumerate}
\item Presumptive test\\
\item Negative test\\
\item Completed\\
\item Confirmed
\end{enumerate}

\item What is the removal and deactivation requirement for Giardia?\\
\begin{enumerate}
\item $2 \mathrm{log}$\\
\item $3 \mathrm{log}$\\
\item $4 \mathrm{log}$\\
\item There is no requirement
\end{enumerate}

\item  The multiple barrier approach to water treatment includes removal through which method?\\
\begin{enumerate}
\item Filtration\\
\item Coagulation\\
\item Disinfection\\
\item a and c
\end{enumerate}

\item  A pH reading of 7 is considered\\
\begin{enumerate}
\item Slightly acidic\\
\item Acidic\\
\item Basic\\
\item Neutral
\end{enumerate}

\item EDTA titration is used to determine the \rule{2cm}{0.3pt} of a water sample.
\begin{enumerate}
\item Hardness
\item Conductivity
\item Alkalinity
\item Free chlorine residual
\end{enumerate}

\item  A higher than normal turbidity reading could signify\\
\begin{enumerate}
\item A change in water quality\\
\item Nothing. Keep operating as normal\\
\item Microbiological contamination\\
\item Both $A$ \& $C$
\end{enumerate}

\item  What is the ingredient used during the second multiple tube fermentation test?\\
\begin{enumerate}
\item Colilert\\
\item MMO/MUG\\
\item Brilliant Green Bile\\
\item Chlorine
\end{enumerate}

\item When collecting a distribution system sample for bacteriological testing, the person collecting the sample should allow the water to run before filling the sample bottle.
\begin{enumerate}
\item A minimum of five minutes.
\item 1 hr.
\item 30 min
\item only a few seconds
\end{enumerate}

\item Black stains on plumbing fixtures might be attributed to
\begin{enumerate}
\item calcium.
\item copper.
\item magnesium.
\item manganese.
\end{enumerate}

\item The multiple tube fermentation test consists of three distinct tests. These tests, in the order performed, are the:
\begin{enumerate}
\item preliminary, confirmed, and completed tests.
\item preliminary, presumptive and confirmed tests.
\item presumptive, confirmed, and completed tests.
\item prespumtive, preliminary, and completed tests.
\end{enumerate}

\item What should the sample volume be when testing for total coliform bacteria?
\begin{enumerate}
\item l00mL
\item 250mL
\item 500mL
\item 1,000mL
\end{enumerate}

\item $\mathrm{pH}$ is a measure of :\\
a. conductivity\\
b. water's ability to neutralize acid\\
c. hydrogen ion activity\\
d. dissolved solids\\
\item  Sodium Thiosulfate is used to\\
a. Buffer chlorine solutions\\
b. Neutralize chlorine residuals\\
c. Detect chlorine leaks\\
d. Sterilize sample bottles\\
  \item The presence of total coliforms in drinking water indicates\\
a. The presence of pathogens.\\
b. The absence of an adequate chlorine residual\\
c. The existence of an urgent public health problem\\
d. The potential presence of pathogens\\
\item A primary health risk associated with microorganisms in drinking water is\\
a. Cancer\\
b. Acute gastrointestinal diseases\\
c. Birth defects\\
d. Nervous system disorders\\
  \item After 5 years use, a portion of cast iron pipe shows a white scale about $1 / 2$ inch thick lining the inside. This means\\
a. Red water will soon become a problem\\
b. The water has been corrosive\\
c. The water is chemically unstable and is depositing\\
d. Water should flow easier since the lining is smooth\\
  \item Hardness in water is caused by\\
a. Dissolved minerals\\
b. High $\mathrm{pH}$.\\
c. Low turbidity\\
d. Alkalinity\\
  \item The meniscus on calibrated glassware is read at the\\
a. Bottom of curvature for mercury but the top for water\\
b. Extreme point of contact between the liquid and glass, i.e., where gas, liquid, and air all meet at one point\\
c. Mid-height of the curvature so that beginning and ending readings will results in zero error\\
d. Top of curvature for mercury but at the bottom for most other liquids including water\\
  \item An unknown substance is found on the bottom of the water within a drinking water reservoir. Which of the following statements is true of this substance?\\
a. It has a specific gravity less than $1.0$\\
b. It has a specific gravity equal to $1.0$\\
c. It has a specific gravity greater than $1.0$\\
d. It has no specific gravity\\
e. None of the above\\
  \item The term "Chain of Custody" refers to\\
a. A large accessory to a come-along\\
b. An attachment to a pipe-cutter\\
c. Employee labor laws\\
d. Procedures and documentation required for water quality sampling\\
e. Procedures and documentation required for chemical application\\
  \item Water samples to be analyzed for taste and odor must be\\
a. Analyzed in the field\\
b. Collected in glass sample containers\\
c. Dechlorinated with sodium thiosulfate\\
d. Preserved with dilute hydrochloric acid\\
e. None of the above\\
  \item Bacteriological samples for a distribution system must be collected in accordance with\\
a. The Surface Water Treatment Rule\\
b. OSHA requirements\\
c. An approved sample siting plan\\
d. FLSA requirements\\
e. ANSI/NSF Standard 61\\
  \item Trihalomethanes are classified as\\
a. Metals\\
b. Inorganic constituents\\
c. Secondary drinking water standards\\
d. Radiological contaminants\\
e. Volatile organic compounds\\
 \item The multiple tube fermentation analysis consists of\\
a. Positive, negative, and neutral tests\\
b. Presumptive, confirmed, and completed tests\\
c. Preliminary, presumptive, and confirmed tests\\
d. Preliminary, confirmed, and completed tests\\
e. Presence or absence testing\\
  \item Which of the following is NOT a characteristic of coliform organisms?\\
a. Intestinal origin\\
b. Will produce carbon dioxide from lactose\\
c. Heartier in a water environment than pathogenic organisms\\
d. Far less numerous than pathogenic organisms\\
e. Able to survive with or without oxygen\\
  \item A bacteriological test that measures only the presence or absence of coliforms is\\
a. ColiLert (MMO/MUG)\\
b. Multiple tube fermentation\\
c. Most probable number (MPN)\\
d. Membrane filtration\\
e. Presumptive test\\
  \item After collection, if stored at $4^{\circ} \mathrm{C}$, bacteriological samples must be processed within\\
a. 1 hour\\
b. 6 hours\\
c. 24 hours\\
d. 48 hours\\
e. 72 hours\\
  \item Sample bottles which are furnished by a certified laboratory for collection of bacteriological samples\\
a. Should be rinsed with the water to be sampled before use\\
b. Should be placed in boiling water for at least 10 minutes before use\\
c. Should be rinsed with a chlorine solution before use\\
d. Should be rinsed with distilled water before use\\
e. Are ready to use\\

\item The standard indicator of potential fecal contamination of a water supply is\\
a. Cryptosporidium\\
b. $\mathrm{pH}$\\
c. Alkalinity\\
d. Hardness\\
e. Coliform Presence - Absence\\

\item Where should bacteriological samples be collected?\\
a. At different locations on each sampling cycle, to make sure the entire system is sampled\\
b. Only from public locations, such as drinking fountains and restrooms\\
c. Only from locations owned by consumers\\
d. Only from specially constructed sampling stations\\
e. From several sampling locations around the entire distribution system, in accordance with a DHS-approved sample siting plan\\
\item Storage of bacteriological samples during transport to a laboratory is best accomplished using\\
a. A clean storage box specifically designed to hold sample containers\\
b. An ice chest packed with ice\\
c. An insulated storage box with "blue ice".\\
d. An insulated storage box with "dry ice"\\
e. No particular sample storage requirements apply, as long as the samples can be delivered to a laboratory prior to the end of the work day\\
  \item Sodium thiosulfate is added in the laboratory to bacteriological sample bottles to:\\
a. Thoroughly disinfect the sample bottle\\
b. -Complete the cleaning and sterilization process\\
c. Neutralize any residual chlorine present in the sample at the time of collection\\
d. Counteract the effects of sunlight on the water sample\\
e. Prevent further growth of bacteria in water samples following collection\\
  \item Radiological contaminant concentrations in drinking water are measured in\\
a. Milligrams per liter\\
b. Micrograms per liter\\
c. Nanograms per liter\\
d. Picograms per liter\\
e. None of the above\\
  \item Which of the following is NOT a characteristic of coliform organisms?\\
a. Intestinal origin\\
b. Will produce carbon dioxide from lactose\\
c. Heartier in a water environment than pathogenic organisms\\
d. Far less numerous than pathogenic organisms\\
e. Able to survive with or without oxygen\\

\item A water supply is found to have a calcium carbonate concentration of 50 mg/L. This water would be considered\\
a.	soft water\\
b.	hard water\\
c.	potable water\\
d.	non-potable water\\

\item Cathodic protection refers to protection against\\
a.	contamination\\
b.	corrosion\\
c.	hardness\\
d.  alkalinity

\item An operator uses \rule{1.5cm}{0.3pt} to test for residual chlorine\\
a. DPD\\
b. Cresol red\\
c. Methyl orange\\
d. Sulfuric acid\\

\item The meniscus on calibrated glassware is read at the:\\
a. Bottom of curvature for mercury but the top for water\\
b. Extreme point of contact between the liquid and glass, i.e., where gas, liquid, and air all meet at one point\\
c. Mid-height of the curvature so that beginning and ending readings will results in zero error\\
d. Top of curvature for mercury but at the bottom for most other liquids including water

\item The type of corrosion caused by the use of dissimilar metal in a water system is\\
a. Caustic corrosion\\
b. Galvanic corrosion\\
c. Oxygen corrosion\\
d. Tubercular corrosion\\

\item Which of the following can cause tastes and odors in a water supply?\\
a. Dissolved zinc\\
b. Algae\\
c. High pH\\
d. Low pH\\



\item The primary health risk associated with volatile organic chemicals.(VOCs) is\\
a. Cancer\\
b. Acute respiratory diseases\\
c. "Blue baby" syndrome\\
d. Reduced IQ. in children 

\item Lead in drinking water can result in\\
a. Impaired mental functioning in children\\
b. Prostate cancer in men\\
c. Stomach and intestinal disorders\\
d. Reduced white blood cell count\\

Sodium thiosulfate is used to\\
a. Buffer chlorine solutions\\
b. Neutralize chlorine residuals\\
c. Raise pH
d. Sterilize sample bottles\\

\item Cathodic protection means protection against\\
a. contamination\\
b. corrosion\\
c. hardness\\
d. infiltration\\

\item A water supply is found to have a calcium carbonate concentration of 50 mg/l. This water would be considered\\
a. soft water\\
b. hard water\\
c. potable water\\
d. non-potable water\\


\item The main characteristic of raw water that enables algae to grow is\\
a. Presence of copper sulfate\\
b. Low pH\\
c. High hardness\\
d. Presence of nutrients\\







\end{enumerate}


