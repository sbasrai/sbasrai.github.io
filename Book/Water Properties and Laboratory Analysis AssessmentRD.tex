%\chapterimage{QuizCover} % Chapter heading image
%\chapter*{Properties and Laboratory Analysis}
%\textbf{Multiple Choice}
\begin{enumerate}[1.]
\item Water with a $\mathrm{pH}$ of 8.0 is considered to be\\
a. acidic\\
b. *basic or alkaline\\
c. neutral\\
d. undrinkable\\
\item Over which water quality indicator do operators have the greatest control?\\
a. alkalinity\\
b. $\mathrm{pH}$\\
c. temperature\\
*d. turbidity\\
\item Which $\mathrm{pH}$ range is generally accepted as most palatable (drinkable)?\\
a. $* 6.5$ to 8.5\\
b. 4.5 to 6.5\\
c. 8.5 to 9.5\\
d. 9.5 and above\\
e. all of the above\\
\item Which of the following conditions is favorable for the rapid growth of algal?\\
a. *moderate to high dissolved oxygen and nutrients\\
b. high $\mathrm{pH}$ and water hardness\\
c. low temperatures and low dissolved oxygen\\
d. high alkalinity and water hardness\\

\item Water has physical, chemical, and biological characteristics. Which of the following is a physical characteristic?\\
a. Coliform\\
b. *Turbidity\\
c. Hardness\\
d. All the above\\
\item Tastes and odors in surface water are most often caused by:\\
a. clays\\
b. hardness\\
c. *algae\\
d. coliform bacteria\\
\item Which of the following elements cause hardness in water?\\
a. sodium and potassium\\
b. *calcium and magnesium\\
c. iron and manganese\\
d. turbidity and suspended solids\\
\item When measuring for free chlorine residual, which method is the quickest and simplest?\\
*a. DPD color comparater\\
b. Orthotolidine method\\
c. Amperometric titration\\
d. 1, 2 nitrotoluene di-amine method\\
\item Which water quality parameter requires a grab sample because it cannot be collected as a composite sample?\\
a. $\mathrm{pH}$\\
b. Iron\\
c. Nitrate\\
d. Zinc\\
\item If a water sample is not analyzed immediately for chlorine residual, it is acceptable if it is analyzed within\\
a. 10 minutes.\\
*b. 15 minutes.\\
c. 20 minutes.\\
d. 30 minutes.\\
\item The volume of a sample for coliform compliance is\\
*a. $100 \mathrm{~mL}$.\\
b. $200 \mathrm{~mL}$.\\
c. $300 \mathrm{~mL}$.\\
d. 0 ; there is no volume compliance for coliforms.\\
\item Which of the following is an indicator organism?\\
a. Giardia\\
b. Cryptosporidium\\
c. Hepatitis\\
*d. E. Coli\\
\item What is the primary origin of coliform bacteria in water supplies?\\
a. Natural algae growth\\
b. Industrial solvents\\
c. Animal or human feces\\
d. Acid rain\\
\item What ls the term for water samples collected at regular intervals and combined in equal volume with each other?\\
a. Time grab samples\\
b. Time flow samples\\
c. Proportional time composite samples\\
\item What is the basis for the number of samples that must be collected for utilities monitoring for lead and copper that are in compliance or have installed corrosion control'?\\
a. Size of distribution system\\
b. Population\\
c. Amount of water produced\\
d. Number of raw water sources\\
\item Where should bacteriological samples be collected in the distribution system?\\
a. Uniformly distributed throughout the system based on area\\
b. At locations that are representative of conditions within the system\\
c. Always from extreme locations in the system but occasionally at other locations\\
d. Uniformly throughout the system based on population density\\
\item The quantity of oxygen. that can remain dissolved in water is related to\\
*a. Temperature\\
b. $\mathrm{pH}$\\
c. Turbidity\\
d. Alkalinity\\
\item In coliform analysis using the presence-absence test, a sample should be incubated for\\
a. 24 hours at $25^{\circ} \mathrm{C}$\\
b. 36 hours at $35^{\circ} \mathrm{C}$\\
c. 24 and 36 hours at $25^{\circ} 0$\\
*d. 24 and 48 hours at $35^{\circ} \mathrm{C}$\\
\item A major source of error when obtaining water quality information is improper:\\
*a. Sampling\\
b. Preservation\\
c. Tests of samples\\
d. Reporting of data\\
\item What is commonly used as an indicator of potential contamination in drinking water samples?\\
a. Viruses\\
*b. Coliform bacteria\\
c. Intestinal parasites\\
d. Pathogenic organisms\\
\item The type of organisms that can cause disease are said to be microorganisms.\\
a. Bad\\
*b. Pathogenic\\
c. Undesirable\\
d. Sick\\
\item Four types of aesthetic contaminants in water include the following:\\
a. Odor, turbidity, color, hydrogen sulfide gas\\
b. Pathogens, microorganisms, arsenic, disinfection by-products\\
*c. Odor, color, turbidity, hardness\\
d. Color, pathogens, metals, organics\\
\item What is the purpose of adding fluoride to drinking water?\\
a. Increase tooth decay\\
*b. Reduce tooth decay\\
c. Make teeth white\\
d. Government conspiracy\\
\item The test used to determine the effectiveness of disinfection is called the:\\
a. Coliform bacteria test\\
b. Color test\\
c. Turbidity test\\
d. Particle test\\
\item Turbidity is measured as:\\
a. $\mathrm{mg} / \mathrm{L}$\\
b. $\mathrm{mL}$\\
c. $\mathrm{gpm}$\\
d. NTU\\
\item Giardia and cryptosporidium are a type of:\\
a. Mineral\\
b. Organism\\
c. Color\\
d. Bird\\
\item Chronic contaminants are those that can cause sickness after:\\
a. Prolonged exposure\\
b. Low levels or low exposure\\
\item A positive total coliform test indicates that:\\
a. Disease-causing organisms may be present in the water supply\\
b. The water is safe to consume\\
c. The water supply has high iron levels\\
d. There is nothing to be concerned about\\
\item What is the purpose of the bacteriological site sampling plan?\\
a. To have a map showing where BacT samples are drawn\\
b. In case of a positive Bac $\mathrm{T}$ sample, the operator will know where to take the four repeat samples\\
c. The state will know where you are taking your repeat samples\\
d. All of the above\\
\item To ensure that the water supplied by a public water system meets state requirements, the water system operator must regularly collect samples and:\\
a. Have water analyzed at an approved water testing laboratory\\
b. Determine a sampling schedule based on state requirements\\
c. Send all analyses results to the state\\
d. All of the above\\
\item Samples taken for routine bacteriological testing should be preserved by:\\
a. Freezing\\
b. Boiling\\
c. DPD preservative\\
d. Refrigeration\\
\item How many coliform samples are required per month for a water system serving a population between 25 and 100 ?\\
a. 1\\
b. 2\\
c. 3\\
d. 4\\
\item Before taking a bacteriological (BacT) water sample from a faucet, you should:\\
a. Wash hands thoroughly b. Remove the faucet aerator\\
c. Flush water until you're sure water is from the main, not the service line\\
d. All of the above\\
\item Monthly BacT samples should be taken from:\\
a. The well pump house\\
b. The distribution system\\
c. The treatment plant\\
d. An outside hose spigot\\
\item If your BacT sample test is positive, how long do you have to collect four repeat samples and deliver them to the lab?\\
a. 12 hours\\
b. 24 hours\\
c. 48 hours\\
d. 72 hours\\
\item \_\_ is a measure of the capacity of water to neutralize acids.\\
a. Concentration\\
b. Alkalinity\\
c. $\mathrm{pH}$\\
d. Conductivity\\
\item The DPD method is used to determine the of a water sample.\\
a. Dissolved oxygen content\\
b. Conductivity\\
c. $\mathrm{pH}$\\
d. Free chlorine residual\\
\item What color does N,N-diethyl-p-phenylenediamine (DPD) turn in the presence of chlorine?\\
a. Brown\\
b. Green\\
c. Blue\\
d. Pink\\
\item The presence-absence ( $\mathrm{P}-\mathrm{A})$ test used for microbiological testing is also commonly referred to as\\
a. Multiple Tube Fermentation\\
b. Membrane Filtration\\
c. Confirmed Test\\
d. Colilert\\
\item When testing for coliform bacteria with the multiple tube fermentation (MFT) method what is the best indicator for a positive test?\\
a. Color change\\
b. Gas bubble formation\\
c. Formation of a cyst d. Formation of turbidity\\
\item Coliform bacteria share many characteristics with pathogenic organisms. Which of the following is not true?\\
a. They survive longer in water\\
b. They grow in the intestines\\
c. There are less coliform than pathogenic organisms\\
d. They are still present in water without fecal contamination\\
\item What is the second step in the multiple tube fermentation test?\\
a. Presumptive test\\
b. Negative test\\
c. Completed\\
d. Confirmed\\
\item What is the removal and deactivation requirement for Giardia?\\
a. $2 \log$\\
b. $3 \log$\\
c. $4 \log$\\
d. There is no requirement\\
\item The multiple barrier approach to water treatment includes removal through which method?\\
a. Filtration\\
b. Coagulation\\
c. Disinfection\\
d. a and c\\
\item A pH reading of 7 is considered\\
a. Slightly acidic\\
b. Acidic\\
c. Basic\\
d. Neutral\\
\item EDTA titration is used to determine the of a water sample.\\
a. Hardness\\
b. Conductivity\\
c. Alkalinity\\
d. Free chlorine residual\\
\item A higher than normal turbidity reading could signify\\
a. A change in water quality\\
b. Nothing. Keep operating as normal\\
c. Microbiological contamination\\
d. Both $A \& C$\\
\item What is the ingredient used during the second multiple tube fermentation test?\\
a. Colilert\\
b. MMO/MUG\\
c. Brilliant Green Bile \\
d. Chlorine\\
\item When collecting a distribution system sample for bacteriological testing, the person collecting the sample should allow the water to run before filling the sample bottle.\\
a. A minimum of five minutes.\\
b. $1 \mathrm{hr}$.\\
c. $30 \mathrm{~min}$\\
d. only a few seconds\\
\item Black stains on plumbing fixtures might be attributed to\\
a. calcium.\\
b. copper.\\
c. magnesium.\\
d. manganese.\\
\item The multiple tube fermentation test consists of three distinct tests. These tests, in the order performed, are the:\\
a. preliminary, confirmed, and completed tests.\\
b. preliminary, presumptive and confirmed tests.\\
c. presumptive, confirmed, and completed tests.\\
d. prespumtive, preliminary, and completed tests.\\
\item What should the sample volume be when testing for total coliform bacteria?\\
a. $100 \mathrm{~mL}$\\
b. $250 \mathrm{~mL}$\\
c. $500 \mathrm{~mL}$\\
d. $1,000 \mathrm{~mL}$\\
\item $\mathrm{pH}$ is a measure of :\\
a. conductivity\\
b. water's ability to neutralize acid\\
c. hydrogen ion activity\\
d. dissolved solids\\
\item Sodium Thiosulfate is used to\\
a. Buffer chlorine solutions\\
b. Neutralize chlorine residuals\\
c. Detect chlorine leaks\\
d. Sterilize sample bottles\\
\item The presence of total coliforms in drinking water indicates\\
a. The presence of pathogens.\\
b. The absence of an adequate chlorine residual\\
c. The existence of an urgent public health problem\\
d. The potential presence of pathogens\\
\item A primary health risk associated with microorganisms in drinking water is\\
a. Cancer\\
b. Acute gastrointestinal diseases\\
c. Birth defects\\
d. Nervous system disorders\\
\item After 5 years use, a portion of cast iron pipe shows a white scale about $1 / 2$ inch thick lining the inside. This means\\
a. Red water will soon become a problem\\
b. The water has been corrosive\\
c. The water is chemically unstable and is depositing\\
d. Water should flow easier since the lining is smooth\\
\item Hardness in water is caused by\\
a. Dissolved minerals\\
b. High $\mathrm{pH}$.\\
c. Low turbidity\\
d. Alkalinity\\
\item An unknown substance is found on the bottom of the water within a drinking water reservoir. Which of the following statements is true of this substance?\\
a. It has a specific gravity less than 1.0\\
b. It has a specific gravity equal to 1.0\\
c. It has a specific gravity greater than 1.0\\
d. It has no specific gravity\\
e. None of the above\\
\item The term "Chain of Custody" refers to\\
a. A large accessory to a come-along\\
b. An attachment to a pipe-cutter\\
c. Employee labor laws\\
d. Procedures and documentation required for water quality sampling\\
e. Procedures and documentation required for chemical application\\
\item Water samples to be analyzed for taste and odor must be\\
a. Analyzed in the field\\
b. Collected in glass sample containers\\
c. Dechlorinated with sodium thiosulfate\\
d. Preserved with dilute hydrochloric acid e. None of the above\\
\item Bacteriological samples for a distribution system must be collected in accordance with\\
a. The Surface Water Treatment Rule\\
b. OSHA requirements\\
c. An approved sample siting plan\\
d. FLSA requirements\\
e. ANSI/NSF Standard 61\\
\item Trihalomethanes are classified as\\
a. Metals\\
b. Inorganic constituents\\
c. Secondary drinking water standards\\
d. Radiological contaminants\\
e. Volatile organic compounds\\
\item The multiple tube fermentation analysis consists of\\
a. Positive, negative, and neutral tests\\
b. Presumptive, confirmed, and completed tests\\
c. Preliminary, presumptive, and confirmed tests\\
d. Preliminary, confirmed, and completed tests\\
e. Presence or absence testing\\
\item A bacteriological test that measures only the presence or absence of coliforms is\\
*a. ColiLert (MMO/MUG)\\
b. Multiple tube fermentation\\
c. Most probable number (MPN)\\
d. Membrane filtration\\
e. Presumptive test\\
\item After collection, if stored at $4^{\circ} \mathrm{C}$, bacteriological samples must be processed within\\
a. 1 hour\\
b. 6 hours\\
*c. 24 hours\\
d. 48 hours\\
e. 72 hours\\
\item Sample bottles which are furnished by a certified laboratory for collection of bacteriological samples\\
a. Should be rinsed with the water to be sampled before use b. Should be placed in boiling water for at least 10 minutes before use\\
c. Should be rinsed with a chlorine solution before use\\
d. Should be rinsed with distilled water before use\\
e. Are ready to use\\
\item The standard indicator of potential fecal contamination of a water supply is\\
a. Cryptosporidium\\
b. $\mathrm{pH}$\\
c. Alkalinity\\
d. Hardness\\
e. Coliform Presence - Absence\\
\item Where should bacteriological samples be collected?\\
a. At different locations on each sampling cycle, to make sure the entire system is sampled\\
b. Only from public locations, such as drinking fountains and restrooms\\
c. Only from locations owned by consumers\\
d. Only from specially constructed sampling stations\\
e. From several sampling locations around the entire distribution system, in accordance with a DHS-approved sample siting plan\\
\item Storage of bacteriological samples during transport to a laboratory is best accomplished using\\
a. A clean storage box specifically designed to hold sample containers\\
b. An ice chest packed with ice\\
c. An insulated storage box with "blue ice".\\
d. An insulated storage box with "dry ice"\\
e. No particular sample storage requirements apply, as long as the samples can be delivered to a laboratory prior to the end of the work day\\
\item Sodium thiosulfate is added in the laboratory to bacteriological sample bottles to:\\
a. Thoroughly disinfect the sample bottle\\
b. -Complete the cleaning and sterilization process\\
c. Neutralize any residual chlorine present in the sample at the time of collection\\
d. Counteract the effects of sunlight on the water sample\\
e. Prevent further growth of bacteria in water samples following collection\\
\item Radiological contaminant concentrations in drinking water are measured in\\
a. Milligrams per liter\\
b. Micrograms per liter\\
c. Nanograms per liter\\
d. Picograms per liter\\
e. None of the above\\
\item Which of the following is NOT a characteristic of coliform organisms?\\
a. Intestinal origin\\
b. Will produce carbon dioxide from lactose\\
c. Heartier in a water environment than pathogenic organisms\\
d. Far less numerous than pathogenic organisms\\
e. Able to survive with or without oxygen\\
\item A water supply is found to have a calcium carbonate concentration of $50 \mathrm{mg} / \mathrm{l}$. This water would be considered\\
a. soft water\\
b. hard water\\
c. potable water\\
d. non-potable water\\
\item Cathodic protection refers to protection against\\
a. contamination\\
*b. corrosion\\
c. hardness\\
d. alkalinity\\
\item An operator uses to test for residual chlorine\\
a. DPD\\
b. Cresol red\\
c. Methyl orange\\
d. Sulfuric acid\\
\item The meniscus on calibrated glassware is read at the:\\
a. Bottom of curvature for mercury but the top for water\\
b. Extreme point of contact between the liquid and glass, i.e., where gas, liquid, and air all meet at one point\\
c. Mid-height of the curvature so that beginning and ending readings will results in zero error\\
d. Top of curvature for mercury but at the bottom for most other liquids including water\\
\item The type of corrosion caused by the use of dissimilar metal in a water system is\\
a. Caustic corrosion\\
b. Galvanic corrosion\\
c. Oxygen corrosion\\
d. Tubercular corrosion\\
\item Which of the following can cause tastes and odors in a water supply?\\
a. Dissolved zinc\\
b. Algae\\
c. High $\mathrm{pH}$\\
d. Low $\mathrm{pH}$\\
\item The primary health risk associated with volatile organic chemicals (VOCs) is\\
a. Cancer\\
b. Acute respiratory diseases\\
c. "Blue baby" syndrome d. Reduced IQ. in children\\
\item Lead in drinking water can result in\\
a. Impaired mental functioning in children\\
b. Prostate cancer in men\\
c. Stomach and intestinal disorders\\
d. Reduced white blood cell count\\
Sodium thiosulfate is used to\\
a. Buffer chlorine solutions\\
b. Neutralize chlorine residuals\\
c. Raise pH d. Sterilize sample bottles\\
\item Cathodic protection means protection against\\
a. contamination\\
b. corrosion\\
c. hardness\\
d. infiltration\\

\item Under no circumstances should a composite sample be collected for which type of analysis?\\
*a. Bacteriological\\
b. Total dissolved solids\\
c. Alkalinity\\
d. Turbidity\\
\item The number of monthly distribution system chlorine residual samples required is\\
a. based on water withdrawal permit limit.\\
b. based on system size.\\
*c. based on population.\\
d. different for each state.\\
\item Which is (are) the ideal indicator for pathogens?\\
a. Salmonella species\\
*b. Coliform group bacteria\\
c. Gram-negative cocci\\
d. Gram-negative coccobacilli\\
\item When one substance is dissolved in another and will not settle out, which is the product called?\\
a. An emulsion\\
b. A compound\\
c. A suspension\\
*d. A solution
\item Acids, bases, and salts lacking carbon are\\
a. ketones.\\
b. aldehydes.\\
c. organic compounds.\\
*d. inorganic compounds.\\
\item Which type of sample should always be collected for determining the presence of coliform bacteria?\\
a. Time composite.\\
*b. Grab sample.\\
c. Proportional.\\
d. Composite.\\
\item Samples to be tested for coliforms can be refrigerated for up to hours before analysis, but should be done as soon as possible.\\
a. 4\\
b. 6\\
*c. 8\\
d. 12\\
\item When a water sample is acidified, the final pH of the water must be\\
*a. <2.0.\\
b. $<2.5$.\\
c. $<3.0$.\\
d. $<3.5$.\\
\item Which chemical is used to remove residual chlorine from water?\\
*a. $\mathrm{Na}_{2} \mathrm{~S}_{2} \mathrm{O}_{3}$\\
b. $\mathrm{Na}_{2} \mathrm{SiO}_{3}$\\
c. $\mathrm{Na}_{2} \mathrm{SiF}_{6}$\\
d. $\mathrm{NaOCl}$\\
\item When a sample is collected, which causes its quality to begin to change?\\
a. $\mathrm{CO}_{2}$\\
b. Dissolved gases\\
*c. Biological activity\\
d. pH\\
\item Chemical analysis for synthetic organic compounds should not be collected in containers made of\\
a. polytetrafluoroethylene.\\
b. stainless steel.\\
*c. polypropylene.\\
d. borosilicate.\\
\item How much acid per $100 \mathrm{~mL}$ should be used to preserve a sample for later hardness analyses?\\
a. $\quad 0.1 \mathrm{~mL}$\\
b. $\quad 0.2 \mathrm{~mL}$\\
*c. $\quad 0.5 \mathrm{~mL}$\\
d. $\quad 1.0 \mathrm{~mL}$\\
\item Conductivity measurements can assist the laboratory analyst in\\
a. measuring the electrical strength, which is directly proportional to the number of free electrons.\\
b. estimating the concentration of calcium carbonate.\\
c. evaluating variations in the concentration of suspended particles.\\
*d. determining the degree of mineralization of the water.
\item Water that is to be analyzed for inorganic metals should be filtered for before\\
a. dissolved metals; analyses\\
b. suspended metals; analyses\\
*c. dissolved metals; preserving\\
d. suspended metals; preserving\\
\item In routine water quality sampling, which one of the following is an early warning sign that conditions are becoming more conducive to sulfate-reducing bacteria?\\
a. Increase in ferrous iron\\
b. Increase in ferric iron\\
*c. Dramatic decline in dissolved oxygen\\
d. Increase in sulfides
\item pH, by definition is\\
a) the ability of particles to stick together\\
b) the ability to cause color to turn insoluble\\
c) causes a water molecule to bring in a third hydrogen atom\\
*d) the hydrogen ion concentration in water\\
\item One method of determining if your finished water has the likelihood to be corrosive is\\
a) Van der Waals formula\\
b) Zeta potential\\
*c) Langeliers Saturation Index\\
d) Hydrological Cycles\\
\item The Langeliers Saturation Index provides an indication of\\
a) the solubility of iron and manganese\\
b) the $\mathrm{pH}$ necessary to settle out color\\
c) the rate at which particles will settle\\
*d) the likelihood that your source water is corrosive\\
\item The most common operational complaint received by a water operator is\\
a) water rates are too high\\
*b) taste and odor\\
c) your uniforms aren't stylish enough\\
d) improper treatment techniques\\
\item The two main substances that cause water hardness are\\
a) benzene and cadmium\\
b) manganese and calcium\\
c) calcium and copper\\
*d) magnesium and calcium\\
\item Heterotrophic Plate Counts measure\\
a) all pathogens in the sample\\
*b) all bacteria in the sample\\
c) all giardia lamblia in the sample\\
d) percent of sludge in the sample\\
\item Total Coliform samples have a hold time.\\
a) 12 hour\\
b) 24 hour\\
*c) 30 hour\\
d) 36 hour\\
\item Extremely soft water can cause problems with pipes and fittings because it is\\
*a) corrosive\\
b) scale forming\\
c) full of suspended solids\\
d) toxic\\
\item Which, surface water or groundwater, usually contain a higher level of pathogens?\\
*a) surface water\\
b) groundwater\\
c) both are equal\\
d) neither\\
\item High nitrate levels in the water can cause\\
a) rickets\\
b) cholera\\
*c) blue baby syndrome\\
d) dysentery\\
\item Hard water can cause problems. Which of these is NOT a problem caused by hard water?\\
a) scale formation in pipes\\
*b) toxic substances occurring because of corrosion\\
c) white scale on laundry fixtures, sinks, cooking utensil, etc.\\
d) buildup on water heater heating elements\\
\item Sources of taste and odor issues include\\
*a) raw water\\
b) distribution systems\\
c) consumer plumbing\\
d) all of the above\\
\item Algae has a profound effect on our surface waters. During the day algae and at night it\\
a) produces carbon dioxide, produces oxygen\\
b) secrets sludge, produces toxins\\
*c) produces oxygen, produces carbon dioxide\\
d) sleeps soundly, parties hardy\\
\item Which type of bottle should be used and how should it be cleaned for taste and odor sample\\
a.  Plastic cleaned with detergent and rinsed with de-ionized water\\
b.  New glass bottles only\\
c.  Glass bottle cleaned with detergent and rinsed with distilled water C\\
d.  New glass or plastic bottles only rinsed with de-ionized water\\
\item Which of the following is a strong acid\\
*a.  pH=2\\
b.  pH=4\\
c.  pH=5\\
d.  pH=13\\
\item Which is the correct order (on the average) from smallest to largest for the microorganisms below?\\
a.	Giardia cysts, bacteria, viruses\\
*b.	Viruses, bacteria, Giardia cysts\\
c.	Bacteria, Giardia cysts, viruses\\
d.	Giardia cysts, viruses, bacteria\\
\item Which radioactive nuclide is a gas?\\
a. Radium\\
b. Uranium\\
*c. Radon\\
d. Thorium
\item Which type of adverse effects does the secondary contaminant color have?\\
*a. Unappealing appearance and indication that dissolved organics may be present\\
b. Added total dissolved solids and scale, indication of contamination, and tastes\\
c. Taste, scale, corrosion, and hardness\\
d. Undesirable taste and appearance\\
\item Precursors to the formation of trihalomethanes (THMs) would most likely come from which source?\\
a. Domestic and commercial activities\\
b. Wastewater treatment plants and industrial waters\\
*c. Humic materials\\
d. Reactions that occur within the treatment plant
\item Which gas occurs mainly in groundwater, is heavier than air, and is odoriferous?\\
*a. Hydrogen sulfide\\
b. Carbon dioxide\\
c. Radon\\
d. Methane\\
\item Which adverse effects does the secondary contaminant iron have?\\
a. Unappealing to drink, undesirable taste, and possible indication of corrosion\\
*b. Discolored laundry brown and changed taste of water, coffee, tea, and other beverages\\
c. Undesirable metallic taste and possible indication of corrosion\\
d. Added total dissolved solids and scale, indication of sewage contamination, and tastes
\item Which species causes typhoid?\\
*a. Salmonella\\
b. Shigella\\
c. Klebsiella\\
d. Pseudomonas\\
\item In humans, Salmonella will cause\\
a. hemochromatosis.\\
b. cholera.\\
*c. gastroenteritis.\\
d. dysentery.\\
\item The freezing point of water is \rule{1.5cm}{0.5pt}.\\
a.	0°F\\
*b.	32°C\\
c.	32°F\\
d.	100°C\\
\item An unknown material is found on the bottom of the water within a drinking water reservoir. Which of the following statements is true of this substance?\\
a.	It has a specific gravity less than 1.0\\
b.	It has a specific gravity equal to 1.0\\
*c.	It has a specific gravity greater than 1.0\\
d.	It has no specific gravity\\
\item The electrical potential required to transfer electrons from one compound or element to another is commonly referred to as\\
a. *oxidation-reduction potential (ORP)\\
b. voltage potential $(\mathrm{OHM} / \mathrm{P})$\\
c. resistance-impedance potential\\
d. microMho differential
\item Which of the following is the name given for a turbidity meter that has reflected or scattered light off suspended particles as a measurement?\\
a. HACH colorimeter\\
b. spectrophotometer\\
c. Wheaton bridge\\
*d. Nephelometer\\
\item Which piece of laboratory equipment is used to titrate a chemical reagent?\\
a. graduated cylinder\\
*b. burette\\
c. pipet\\
d. Buchner funnel\\
\item What is the grain per gallon (gpg) hardness of water that has a total hardness of $228 \mathrm{mg} / \mathrm{L}$ ?\\
*a. 14\\
c. 18\\
e. 133.3\\
b. 3898.8\\
d. 39\\
\item Water hardness is the measure of the concentrations of and dissolved in the water sample.\\
a. iron , manganese\\
b. nitrates, nitrites\\
c. sulfates, bicarbonates\\
d. *calcium \& magnesium carbonates\\
e. ferric chlorides and polymers\\
\item A specific class of bacteria that only inhibit the intestines of warm-blooded animals is referred to as?\\
a. Eutrophic\\
b. Pathogenic\\
c. Salmonella\\
d. *Fecal coliform\\
\item When sampling for bacteria in a distribution system, the bacteria sample bottle is prepared with which chemical inside the bottle before it is sterilized?\\
a. copper sulfate\\
b. chlorine tablets\\
c. *sodium thiosulfate\\
d. hydrochloric acid\\
\item Hard water contains an abundance of\\
a. sodium\\
b. iron\\
c. lead\\
d. *calcium carbonate\\
\item Microorganisms from smallest to largest are as follows:\\
a. Viruses, protozoans, and bacteria\\
b. Bacteria, viruses, and protozoans\\
*c. Viruses, bacteria, and protozoans\\
d. Protozoans, bacteria, and viruses\\
\item 	What ls the term for water samples collected at regular intervals and combined in equal volume with each other?\\
a.  Time grab samples\\
b.  Time flow samples\\
*c. Proportional time composite samples\\
d. Proportional flow composite samples\\
\item Which of the following conditions would increase turbidity?\\
a.	decrease in pH\\
b.	increase in alkalinity\\
c.	 increase in phosphate concentration\\
*d.	increase in organic matter\\
\item Water that is to be analyzed for inorganic metals should be acidified with\\
a. dilute hydrochloric acid.\\
b. concentrated hydrochloric acid.\\
c. dilute nitric acid.\\
*d. concentrated nitric acid.
\item A solution used to determine the concentration of another solution is called a\\
a. saturated solution.\\
*b. standardized solution.\\
c. concentrated solution.\\
d. dilute solution.\\
\item Which method would you use to concentrate and retrieve low numbers of bacteria from a large quantity of water?\\
a. Colilert\\
b. Colisure\\
*c. Membrane filtration\\
d. MPN (Most Probable Number)\\
\item A typical coliform colony in the membrane filter method has the following characteristics:\\
a. Blue with lustrous surface sheen\\
*b. Pink to dark red with green metallic surface sheen\\
c. Pink or yellow with lustrous to metallic surface sheen depending on species\\
d. Yellow with silver metallic to lustrous surface sheen\\
\item In the presumptive phase of the Most Probable Number test, how long does it take for the coliforms to produce gas?\\
a. 12 to 24 hours\\
*b. 24 to 48 hours\\
c. 24 to 36 hours\\
d. 36 to 48 hours\\
\item Which gas is radioactive, occurs in many groundwater supplies, and is colorless and odorless?\\
a. Neon\\
b. Argon\\
c. Boron\\
*d. Radon\\
\item How often should the Langelier Index of the raw and treated water be calculated?\\
a. Every 8 hours\\
b. Every 12 hours\\
*c. Every day\\
d. Every week\\
\item Which type of hardness is considered permanent hardness?\\
a. Carbonate hardness\\
*b. Noncarbonated hardness\\
c. Calcium hardness\\
d. Magnesium hardness\\
\item Which organisms are prokaryotes and release odorous compounds such as geosmin and 2-methylisoborneal?\\
a. Methylomonas\\
b. Chlorobium\\
c. Clostridium\\
*d. Cyanobacteria
\item Which is the most common radionuclide in water?\\
*a. Radium\\
b. Uranium\\
c. Radon\\
d. Thorium\\
\item The polarity of water causes the hydrogen part of one water molecule to be weakly bonded to the oxygen of another water molecule. This bonding is called\\
a. ionic.\\
b. covalent.\\
c. van der Waals.\\
*d. hydrogen bonding.
\item The process whereby water moves with the air currents in the atmosphere is called\\
a. transpiration.\\
b. evaporation.\\
c. interception.\\
*d. advection.\\
\item The most common forms of manganese found in nature are\\
a. sulfides, inosilicates, and native metal.\\
b. sulfides and phosphates.\\
*c. oxides, carbonates, and hydroxides.\\
d. phyllosilicates, tectosilicates, and inosilicates.\\
\item Which has the USEPA chosen to suggest assaying water for as indicators for the inactivation of viruses?\\
a. Total coliforms\\
b. Fecal coliforms\\
c. E-coli\\
*d. Coliphages
\item When ingested Giardia cysts mature and multiply, which problem do they cause?\\
*a. Interfere with nutrient absorption\\
b. Destroy the lining of the stomach\\
c. Secrete a toxin, which causes diarrhea\\
d. Kill millions of cells in the large intestine, which causes diarrhea\\
\item Consistency is maintained by providing certified laboratories with a known concentration of a contaminant. Who oversees the analytical results of the samples provided to the certified laboratories?\\
*a. USEPA and state primacy programs\\
b. Primacy agency for each state\\
c. American Chemical Society\\
d. US Department of Public Health\\
\item Which type of adverse effect does the secondary contaminant zinc have?\\
a. Unappealing taste and possible indication of contamination\\
b. Added total dissolved solids and stained laundry\\
c. A laxative effect and undesirable metallic taste\\
*d. Undesirable taste and a milky appearance\\
\item Which statement is true concerning dissolved oxygen?\\
a. Dissolved oxygen will increase the absorption of carbon dioxide\\
b. High dissolved oxygen has adverse health effects\\
c. Most consumers do not prefer high levels of dissolved oxygen in the water\\
*d. Dissolved oxygen may oxidize iron and manganese\\
\item Zebra mussels multiply and spread so fast because\\
a. the female produces over 250,000 eggs in a single season.\\
b. the larvae are free swimming for 2 or 3 months before attaching themselves.\\
*c. they commonly adhere to boats so they move around and disperse eggs.\\
d. they can live out of water for as long as 3 months.\\
\item Which is the unit of gamma radiation?\\
a. Muon\\
b. Neutrino\\
c. X-ray\\
*d. Photon\\
\item Which of the following causes lung cancer?\\
a. Uranium\\
b. Radium\\
*c. Radon\\
d. Thorium\\
\item Which genera of cyanobacteria primarily release neurotoxins?\\
a. Nodularia\\
b. Microcystis\\
*c. Anabaena\\
d. Cylindrospermopsis\\
\item If artificial radionuclides are present in the environment, which element is most likely to be found?\\
a. Potassium 20\\
b. Carbon 14\\
c. Argon 40\\
*d. Tritium\\
\item Beta radiation consists of $a(n)$\\
*a. electron.\\
b. neutron.\\
c. proton.\\
d. muon.\\
\item Which naturally occurring radionuclide is most prevalent in drinking water?\\
a. Gamma\\
*b. Alpha\\
c. Beta\\
d. Muson\\
\item Enteric protozoans\\
a. usually have four stages in their life cycles.\\
b. are easy to culture.\\
*c. have cysts that can penetrate filters.\\
d. can be found in very high densities in most lakes.\\
\item Which is the size range of Cryptosporidium?\\
*a. 4 to $6 \mu \mathrm{m}$\\
b. 6 to $8 \mu \mathrm{m}$\\
c. $\quad 10$ to $12 \mu \mathrm{m}$\\
d. 12 to $18 \mu \mathrm{m}$\\
\item Prolonged Entamoeba histolytica disease can cause amoebic abscesses which usually occur in the\\
*a. liver.\\
b. urinary tract.\\
c. small intestine.\\
d. kidney.\\
\item The quantity of dissolved oxygen in water is a function of\\
a.	pH, alkalinity temperature and total dissolved solids\\
b.	Temperature and alkalinity\\
c.	pH and temperature\\
*d.	Temperature, pressure and salinity\\
\item Which type of solution contains 1 gram equivalent weight of a reactant compound per liter of solution?\\
a.	Molar solution\\
b.	Molal solution\\
c.	Normal solution\\
d.	Percentage strength solution\\
\item What laboratory device sterilizes laboratory apparatus and microbial media by using pressurized steam?\\
a.	Muffle furnace\\
b.	Aspirator\\
*c.	Autoclave\\
d.	Membrane filter\\

\item What level of threshold odor number (TON) will result in a customer complaint?\\
a. 2 TON\\
*b. 3 TON\\
c. 10 TON\\
4. 12 TON

\item What is the secondary MCL for odor?\\
a. 2 TON\\
*b. 3 TON\\
c. 10 TON\\
4. 12 TON

\item .  What element is found in all acids?\\
a. Fluoride (F)\\
b. Sulfur (S)\\
c. Hydrogen (H)\\
d. Calcium (Ca)
\end{enumerate}



