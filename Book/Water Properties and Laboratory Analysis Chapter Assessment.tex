\chapterimage{QuizCover} % Chapter heading image

\chapter*{Chapter 3 Assessment}
% \textbf{Multiple Choice}
\section*{Chapter 3 Assessment}
\begin{enumerate}[1.]

\item Hard water contains an abundance of\\
a. sodium\\
b. iron\\
c. lead\\
d. calcium carbonate\\
\item A specific class of bacteria that only inhibit the intestines of warm-blooded animals is referred to as?\\
a. Eutrophic\\
b. Grazing\\
c. Salmonella\\
d. Fecal coliform\\
e. pathogenic\\
\item Water with a pH of 8.0 is considered to be\\
a. acidic\\
b. basic or alkaline\\
c. neutral\\
d. undrinkable\\
\item Over which water quality indicator do operators have the greatest control?\\
a. alkalinity\\
b. pH\\
c. temperature\\
d. turbidity\\
\item Which piece of laboratory equipment is used to titrate a chemical reagent?\\
a. graduated cylinder\\
b. burette\\
c. pipet\\
d. Buchner funnel\\
\item Which pH range is generally accepted as most palatable (drinkable)?\\
a. 6.5 to 8.5\\
b. 4.5 to 6.5\\
c. 8.5 to 9.5\\
d. 9.5 and above\\
e. all of the above\\
\item Which of the following conditions is favorable for the rapid growth of algae?\\
a. plant nutrients\\
b. high $\mathrm{pH}$ and water hardness\\
c. low temperatures and low dissolved oxygen\\
d. high alkalinity and water hardness\\
\item Which of the following is the name given for a turbidity meter that has reflected or scattered light off suspended particles as a measurement?\\
a. Hach colorimeter\\
b. spectrophotometer\\
c. Wheaton bridge\\
d. Nephelometer\\
\item Water hardness is the measure of the concentrations of and dissolved in the water sample.\\
a. iron, manganese\\
b. nitrates, nitrites\\
c. sulfates, bicarbonates\\
d. calcium \& magnesium carbonates\\
e. ferric chlorides and polymers\\
\item The electrical potential required to transfer electrons from one compound or element to another is commonly referred to as\\
a. oxidation-reduction potential (ORP)\\
b. voltage potential $(\mathrm{OHM} / \mathrm{P})$\\
c. resistance-impedance potential\\
d. microMho differential\\
\item Water has physical, chemical, and biological characteristics. Which of the following is a physical characteristic?\\
a. Coliform\\
b. Turbidity\\ 
c. Hardness\\
d. All the above\\
\item Tastes and odors in surface water are most often caused by:\\
a. clays\\
b. hardness\\
c. algae\\
d. coliform bacteria\\
\item Which of the following elements cause hardness in water?\\
a. sodium and potassium\\
b. calcium and magnesium\\
c. iron and manganese\\
d. turbidity and suspended solids\\
\item When measuring for free chlorine residual, which method is the quickest and simplest?\\
a. DPD color comparator\\
b. Orthotolidine method\\
c. Amperometric titration\\
d. 1, 2 nitrotoluene di-amine method\\


\end{enumerate}


