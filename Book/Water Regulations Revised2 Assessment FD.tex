 What does the acronym MCL stand for?\\

Minimum contaminant level\\
Micron contaminant level\\
Maximum contaminant level\\
Milligrams counted last


 How long do sanitary surveys have to be retained for records?\\

3 years\\
5 years\\
7 years\\
10 years


The most severe water system violation that requires the fastest public notification\\

Tier I\\
Tier II\\
Tier III\\
Tier IV


 The primacy agency may grant a variance or exemption as long as\\

The agency is using the Best Available Technology\\
There is no threat to public health\\
There is never a scenario for a variance or exemption\\
Both A. and B.


 A public water system that serves at least 25 people six months out of the year\\

Nontransient noncommunity\\
Transient noncommunity\\
Community public water system\\
None of the above


 Regulations based on the aesthetic quality of drinking water\\

Primary Standards\\
Secondary Standards\\
Microbiological Standards\\
Radiological Standards


 The lowest reportable limit for a water sample\\

0.5mg/l\\
Zero\\
Public health goal\\
Detection Level for reporting


 Primary Standards are based on\\

Color and Taste\\
Aesthetic quality\\
Public Health\\
Odor


A disease causing microorganism\\

Pathogen\\
Colilert\\
Pathological\\
Turbidity


 According to Surface Water Treatment Rule, what is the combined inactivation and removal for Giardia?\\

1.0 Logs\\
2.0 Logs\\
3.0 Logs\\
4.0 Logs


 What is the equivalency expressed as a percentage for the SWTR inactivation and removal of viruses?\\

$99.9 \%$\\
$99.99 \%$\\
$99.0 \%$\\
$99.999 \%$


 A water agency that takes more than 40 coliform samples must fall under what percentile?\\

$10 \%$\\
$7 \%$\\
$5 \%$\\
No positive samples allowable


The National Primary Drinking Water Regulations apply to drinking water contaminants that may have adverse effects on\\
a.	Water color\\
b.	Water taste\\
c.	Water odor\\
d.	Human health\\

Which of the following is considered an acute risk to health?\\
a.	Two Tier 2 violations\\
b.	One Tier 2 violation\\
c.	Two Tier 1 violations\\
d.	One Tier 1 violation\\

Records on turbidity analyses should be kept for a minimum of\\
a.	5 years\\
b.	7 years\\
c.	l0 years\\
d.	25 years\\

Records on bacteriological analyses should be kept for a minimum of\\
a.	5 years\\
b.	7 years\\
c.	10 years\\
d.	25 years\\

Differecne between primary and secondary standard substances:\\
a.	Primary standards refer to substances that are carcinogenic, secondary standards do not.\\
b.	Primary standards refer to substances that are thought to pose a threat to human health, secondary standards do not.\\
c.	Primary standards refer to substances that, if not.put in check, will eventually kill humans, secondary standards do not.\\
d.	Secondary qualities are aesthetic qualities and will only make some people sick, while primary standards refer to substances that will make everyone sick and may possibly cause death.\\

The SDWA defines a public water system that supplies piped water for human consumption as one that has\\
a.	10 service connection or serves 20 or more people for 60 or more days per year\\
b.	15 service connections or serves 20 or more people for 90 or more days per year\\
c.	10 service connections or serves 25 or more people for 30 or more days per year\\
d.	15 service connections or serves 25 or more people for 60 or more days per year\\

According to the USEPA regulations, the owner or operator of a public water system that fails to comply with applicable\\
monitoring requirements shall give notice to the public within\\
a.	1 week of the violation in a letter hand-delivered to customers\\
b.	45 days of the violation by posting a notice at the town hall\\
c. 	3 months of the violation in a daily newspaper in the area served by the system 
d.  1 year of the violation by including the notice with the water-bill ·\\

What US agency establishes drinking water standards?\\
a.	AWWA\\
b.	USEPA\\
c.	NIOSH\\
d.	NSF\\

If a water supply exceeds the MCL, whose responsibility is it to notify the consumer?\\
a.	the testing lab\\
b. 	the supplier\\
c.	the DOH\\
d.	the USEPA\\

According to the Lead and Copper Rule. the action for the 90th percentile lead level is:\\
a.	0. 005 mg/1\\
b.	0. 015 mg/l\\
c.	0. 030 mg/l\\
d.	0.050 mg/l\\


The term "maximum contaminant level goal (MCLG)" means the:\\ 
a. Maximum allowable level of a given contaminant in drinking water\\
b. Level of a contaminant .in drinking water below which there are no known or suspected adverse health effects with a margin of safety\\
c. Level of a contaminant in drinking water that will trigger a Tier 1 violation\\
d. Minimum detectable level of a given contaminant\\

The maximum contaminant level goal (MCLG) of known or probable carcinogens is:\\
a. Set by the state\\
b. The same number as the maximum contaminant level (MCL)\\
c. Zero\\
d. The minimum detectable level of a given contaminant\\

The difference between Tier 1 and Tier 2 violations is:\\
a. Tier 1 violations·potentially impose·direct and adverse health effects;-Tier 2 violations do not pose a a direct threat to public health.
b. Tier 1 violations require public notification; Tier 2 violations do not require public notification\\
c. Tier 1 violations are acute; Tier 2 violations are not acute\\
d. Tier 1 violations have legal consequences; Tier 2 violations do not\\

The Safe Drinking Water Act requires \rule{1.5cm}{0.1mm} to develop a comprehensive coliform monitoring plan\\
a. Large public water systems (serving >50,000 people)\\
b. Large and medium public water systems (serving >3,300 people)\\
c. Small and medium public water systems (serving >25 and <3,300 people)\\
d. All public water systems\\

Final determination of vulnerability is made by:
a. Private contractor/consultants\\
b. The primacy agency\\
c. The water supplier\\
d. All of the above

The most important factor to consider in locating a well site from the health point of view is\\
a. Anticipated yield\\
b. Availability of electric power\\
c. Distance from other wells\\
d. Vulnerability\\

Trihalomethanes are classified as:\\
a. Metals\\
b. Inorganic constituents\\
c. Secondary drinking water standards\\
d. Radiological contaminants\\
e. Volatile organic compounds\\


  The primary health risk associated with volatile organic chemicals.(VOCs) is\\

a. Cancer\\

b. Acute respiratory diseases\\

c. "Blue baby" syndrome\\

d. Reduced IQ in children\\

The term "primacy" means the\\

a. Authority by the states to supersede USEPA drinking water regulations\\

b. Authority by the USEPA to supersede state drinking water regulations\\

c. Requirements for states to maintain drinking water regulations more stringent than USEPA regulations \\
d. Primary authority for implementation and enforcement of drinking water regulations


The Safe Drinking Water Act requires to develop a comprehensive coliform monitoring plan\\
a. Large public water systems (serving $>50,000$ people)\\
b. Large and medium public water systems (serving $>3,300$ people)\\
c. Small and medium public water systems (serving $>25$ and $<3,300$ people)\\
d. All public water systems\\

Contaminant monitoring requirements can depend on\\

a. The results of a vulnerability assessment\\

b. The size of the water system\\

c. Previous maximum contaminant level (MCL) violations\\

d.  All of the above\\

For public water systems using surface water and groundwater under the influence of surface water, turbidity must be measure at least\\

a. Every 4 hours\\

b. Daily\\

c. Weekly\\

d. Monthly\\

The difference between Tier 1 and Tier 2 violations is\\

a. Tier1-violations potentially impose-direct and adverse health effects; Tier 2 violations do not pose a direct threat to public health\\

b. Tier 1 violations require public notification; Tier 2 violations do not require public notification\\

c. Tier 1 violations are acute; Tier 2 violations are not acute\\

d. Tier 1 violations have legal consequences; Tier 2 violations do not\\


The maximum contaminant level goal (MCLG) of known or probable carcinogens is\\
a. Set by the state\\
b. The same number as the maximum contaminant level $(\mathrm{MCL})$\\
c. Zero\\
d. The minimum detectable level of a given contaminant\\

  All of the following diseases may be transmitted by contaminated water, except for:\\


a. Cryptosporidiosis\\

b. Giardiasis\\

c. Cholera\\

d. Typhoid\\

e. Tuberculosis\\


The maximum disinfectant residual allowed in a distribution system is\\

a. $\quad 0.2 \mathrm{mg} / \mathrm{L}$\\

b. $\quad 2.0 \mathrm{mg} / \mathrm{L}$\\

c. $\quad 2.0 \mu \mathrm{g} / \mathrm{L}$\\

d. $\quad 4.0 \mathrm{mg} / \mathrm{L}$\\

e. There is no maximum disinfectant residual standard\\

What steps must be taken when a single routine sample tests positive for total coliform?
a. Immediately notify the Department of Health Services

b. Immediately notify customers

c. Re-test a new sample taken from the original sample point

d. Re-test a new sample taken from the original sample point, plus at points immediately upstream and downstream

e. Flush the system around the original sample point to re-establish disinfectant levels

 For drinking water distribution systems with over 40 routine coliform samples per month, the maximum amount of coliform-positive samples permitted is\\
a. 2\\
b. 2 \%\\
c. 5\\
d. 5 \%

e. variable, depending on the size of the system

  The regulation that establishes standards for microbiological quality in drinking water is
a. The Disinfection By-Product Rule

b. Secondary Drinking Water Standards

c. The Total Coliform Rule

d. The Lead and Copper Rule

e. Maximum Contaminant Level


  Primary and secondary drinking water standards are normally established with a

a. Maximum contaminant level

b. Minimum contaminant level

c. Public health goal

d. Maximum contaminant level goal

e. Minimum contaminant level goal

The presence of coliform bacteria in a distribution system

a. Is positive proof that pathogenic organisms are present

b. Indicates that chlorine demand has increased dramatically 
c. Indicates that pathogenic organisms may be present also

d. Requires the use of brilliant green bile as a secondary disinfectant

e. Has no particular significance

The regulation that establishes standards for microbiological quality in drinking water is

a. The Disinfection By-Product Rule

b. Secondary Drinking Water Standards

c. The Total Coliform Rule

d. The Lead and Copper Rule

e. Maximum Contaminant Level

For public water systems using surface water and groundwater under the influence of surface water, turbidity must be measured at least\\
a. Every 4 hours\\
b. Daily\\
c. Weekly.\\
d. Monthly\\

Contaminant monitoring requirements can depend on\\
a. The results of a vulnerability assessment\\
b. The size of the water system\\
c. Previous maximum contaminant level (MCL) violations\\
d. All of the above\\

The term "primacy" means the\\
a. Authority by the states to supersede USEPA drinking water regulations\\
b. Authority by the USEPA to supersede state drinking water regulations\\
c. Requirements for states to maintain drinking water regulations more stringent than USEPA regulations \\
d. Primary authority for implementation and enforcement of drinking water regulations\\

According to the Lead and Copper Rule. the action for the 90th percentile lead level is:\\
a. 0.005 mg/l\\
b. 0.015 mg/l\\
c. 0.030 mg/l\\
d. 0.050 mg/l\\



The difference between Tier 1 and Tier 2 violations is\\
a.	Tier 1-violations potentially impose-direct and adverse health effects; Tier 2 violations do not pose a direct threat to public health\\
b. Tier 1 violations require public notification; Tier 2 violations do not require public notification\\
c. Tier 1 violations are acute; Tier 2 violations are not acute\\
d. Tier 1 violations have legal consequences; Tier 2 violations do not\\


For public water systems using surface water and groundwater under the influence of surface water, turbidity must be measure at least\\
a. Every 4 hours\\
b. Daily\\
c. Weekly.\\
d. Monthly\\

Contaminant monitoring requirements can depend on\\
a. The results of a vuilnerability assessment\\
b. The size of the water system\\
c. Previous maximum contaminant level (MCL) violations\\
d. All of the above\\

The Safe Drinking Water Act requires to develop a comprehensive coliform monitoring plan
a. Large public water systems (serving $>50,000$ people)\\
b. Large and medium public water systems (serving $>3,300$ people)\\
c. Small and medium public water systems (serving $>25$ and $<3,300$ people)\\
d. All public water systems\\








