\chapterimage{QuizCover} % Chapter heading image

\chapter*{Chapter 4 Assessment}

\section*{Chapter 4 Assessment}
% \textbf{Multiple Choice}
\begin{enumerate}[1.]
\item Primary drinking water standards are set to protect the public from illnesses as a direct result in drinking water that exceeds maximum set levels. Secondary standards were set to alert the public to\\
a. the incidences of local cancer numbers\\
b. dissolved solids in water\\
c. immediate health concerns\\
d. radiological conditions concerning drinking water\\
e. aesthetic issues with drinking water\\
\item A positive fecal coliform test must be reported to the primacy agency within\\
a. 8 hours.\\
b. 12 hours.\\
c. 24 hours.\\
d. 48 hours.\\
\item Which agency sets legal limits on the concentration levels of harmful contaminants in potable water distributed to customers?\\
a. National Primary Drinking Water Regulations\\
b. United States Environmental Protection Agency\\
c. United States Public Health Service\\
d. Occupational Health and Safety Organization\\
\item Which may be substituted for the analysis of residual disinfectant concentration, when total coliforms are also sampled at the same sampling point?\\
a. Heterotrophic plate count (HPC)\\
b. Fecal coliforms\\
c. Giardia lamblia\\
d. Combined chlorine\\
\item What does the acronym MCL stand for?\\
a. Minimum contaminant level\\
b. Micron contaminant level\\
c. Maximum contaminant level\\
d. Milligrams counted last\\
\item How long do sanitary surveys have to be retained for records?\\
a. 3 years\\
b. 5 years\\
c. 7 years\\
d. 10 years\\
\item The most severe water system violation that requires the fastest public notification\\
a. Tier I\\
b. Tier II\\
c. Tier III\\
d. Tier IV
\item The primacy agency may grant a variance or exemption as long as\\
a. The agency is using the Best Available Technology\\
b. There is no threat to public health\\
c. There is never a scenario for a variance or exemption\\
d. Both A. and B.\\
\item A public water system that serves at least 25 people six months out of the year\\
a. Nontransient noncommunity\\
b. Transient noncommunity\\
c. Community public water system\\
d. None of the above\\
\item Regulations based on the aesthetic quality of drinking water\\
a. Primary Standards\\
b. Secondary Standards\\
c. Microbiological Standards\\
d. Radiological Standards\\
\item The lowest reportable limit for a water sample\\
a. $0.5 \mathrm{mg} / 1$\\
b. Zero\\
c. Public health goal\\
d. Detection Level for reporting\\
\item Primary Standards are based on\\
a. Color and Taste\\
b. Aesthetic quality\\
c. Public Health\\
d. Odor\\
\item A disease causing microorganism\\
a. Pathogen\\
b. Colilert\\
c. Pathological\\
d. Turbidity\\
\item According to Surface Water Treatment Rule, what is the combined inactivation and removal for Giardia?\\
a. $1.0 \log s$\\
b. $2.0 \log \mathrm{s}$\\
c. $3.0 \log s$\\
d. 4.0 Logs\\
\item What is the equivalency expressed as a percentage for the SWTR inactivation and removal of viruses?\\
a. $99.9 \%$\\
b. $99.99 \%$\\
c. $99.0 \%$\\
d. $99.999 \%$\\
\item A water agency that takes more than 40 coliform samples must fall under what percentile?\\
a. $10 \%$\\
b. $7 \%$\\
c. $5 \%$\\
d. No positive samples allowable\\
\item The National Primary Drinking Water Regulations apply to drinking water contaminants that may have adverse effects on\\
a. Water color\\
b. Water taste\\
c. Water odor\\
d. Human health\\
\item Which of the following is considered an acute risk to health?\\
a. Two Tier 2 violations\\
b. One Tier 2 violation\\
c. Two Tier 1 violations\\
d. One Tier 1 violation\\
\item Records on turbidity analyses should be kept for a minimum of\\
a. 5 years\\
b. 7 years\\
c. 10 years\\
d. 25 years\\
\item Records on bacteriological analyses should be kept for a minimum of\\
a. 5 years\\
b. 7 years\\
c. 10 years\\
d. 25 years\\

\end{enumerate}


