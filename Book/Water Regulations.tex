\chapterimage{Water1.png} % Chapter heading image

\chapter{Water Regulations}

\section{Wastewater Regulations}\index{Wastewater Regulations}

\begin{itemize}
\item The 1972 Clean Water Act (CWA) addresses pollution of the many factors can cause pollution and adversely affect the quality of the waters of the United States, including municipal and industrial wastewater discharges, polluted runoff from urban and rural areas, and habitat destruction.\\
\item CWA established the National Pollutant Discharge Elimination System (NPDES) permit program to regulate discharge of pollutants.
\item NPDES permit program:
\begin{itemize}
\item Applies to sources that discharge pollutants to waters of the United States.
\item Requires all facilities discharging “pollutants” into any body of water in the USA to obtain and comply with a \hl{NPDES permit}.
\item \hl{Establishes} \textul{discharge limits}, \textul{monitoring} and \textul{reporting} \hl{requirements}\\
\item In California, the responsibility of implementing the federal NPDES program is delegated to the State of California through the State Water Resources Control Board (State Water Board or SWRCB) and finally to the nine Regional Water Quality Control Boards (Regional Water Boards or RWQCB), collectively known as Water Boards. 
\item The RWQCB issues the NPDES permit.
\end{itemize}
\end{itemize}











\section{Drinking Water Regulations}\index{Drinking Water Regulations}
\begin{itemize}
\item Drinking water sources which include surface and groundwater sources have inherent vulnerabilities to contamination and regulations have been established to protect public health and safety.

\item Federal Safe Drinking Water Act (SDWA)
\begin{itemize}
\item SDWA enacted in 1974 established national enforceable standards for drinking water quality and to guarantee that water suppliers monitor water to ensure that it meets national standards. \\

\item SDWA gives individual states the opportunity to set and enforce their own drinking water standards if the standards are at a minimum as stringent as EPA's national standards.

\item Water treatment standards are set and enforced by the state’s nine regional water quality control boards in consultation with the California Department of Public Health. The nine regional boards are part of the State Water Board.

\item Water treatment standards:
\begin{enumerate}
\item For contaminants – chemicals and microorganisms - found in drinking water and known to present adverse health effects to humans, SDWA established National Primary Drinking Water Regulations (NPDWR)

\begin{itemize}
\item NPDWR are legally enforceable drinking water standards.  
\item The NPDWR standard can be either:
\begin{enumerate}
\item Maximum Contaminant Levels (MCLs) - the maximum permissible level of a contaminant in water which is delivered to any user of a public water system,  or 
\item Treatment technique which is a drinking water treatment requirement typically used when setting an MCL would be too difficult or when compliance with an MCL would be too costly.
\end{enumerate}

\item The 90 Primary Contaminants identified by the EPA are grouped into four major categories:
\begin{enumerate}
\item Inorganic chemicals
\begin{itemize}
\item These contaminants are mostly heavy metals. 
\item They may enter the water supply naturally through ground water formations or from mining runoff and industrial discharges.
\item Nitrates are the only chemical contaminant that represent an immediate health risk. Pregnant mothers and infants under 18 months can develop a condition known as “Blue Baby Syndrome”. The presence of nitrates in the bloodstream reduces oxygen uptake that gives the skin a blue tint.
\end{itemize}
\item Organic chemicals
\begin{itemize}
\item These contaminants include herbicides and insecticides that are primarily used in agriculture applications, organic solvents used in industrial applications, organic by-products of industrial processes, and chemical by- products from chlorination of drinking water.
\item Runoff from agricultural spraying or improper application techniques can be a major source of these contaminants in a surface water supply.
\item Industrial discharges, accidental spills and improper disposal of hazardous wastes can also become sources of contamination.
\item These compounds are grouped together under the headings of Volatile Organic Compounds or VOC’s and Synthetic Organic Compounds or SOC’s. 

\item There are currently 21 regulated VOC’s and 30 SOC’s that must be analyzed.
\end{itemize}
\item Radioactive chemicals
\begin{itemize}
\item Most radioactive substances occur naturally in ground water and in some surface supplies. 
\item Some man-made substances may also enter drinking water supplies from processing facilities, mining areas, and nuclear power plants. 
\end{itemize}

\item Bacteriological Contaminants
\begin{itemize}
\item The coliform group of bacteria represents the indicator organisms used in determining bacteriological contamination. Their presence indicates the possibility that some pathogenic (disease causing) organisms may also be present. 
\item The MCL is exceeded when 5\% of the required monthly routine (M/R) samples indicate the presence of Coliform bacteria. 
\item The presence of coliform in any sample will require three repeat samples be taken. These repeat samples must be taken within 24 hrs of notification of positive results.
\end{itemize}
\end{enumerate}
\item Turbidity - Measure of the cloudiness of water. 
\begin{itemize}
\item Although turbidity does not represent a health risk by itself, it can shield harmful bacteria from disinfection processes.
\item Turbidity is measured in Nephelometric Turbidity Units (NTU). 
\item The device used to measure NTU’s is called a nephelometer or turbidimeter.
\end{itemize}
\end{itemize}
\item SDWA also established National Secondary Drinking Water Regulations (NSDWRs) (or secondary standards) are non-enforceable guidelines regulating contaminants that may cause cosmetic effects (such as skin or tooth discoloration) or aesthetic effects (such as taste, odor, or color) in drinking water.
\begin{itemize}
\item NSDWRs are recommended standards and water systems are not required to comply with the established standard. However, states may choose to adopt them as enforceable standards.
\item While secondary standards are not federally enforceable, EPA requires a special notice for exceedance of the fluoride secondary standard of 2.0 mg/L.
\end{itemize}
\item Unregulated Contaminant Monitoring Rule (UCMR) is established by the EPA to collect data for contaminants that are suspected to be present in drinking water and do not have health-based standards set under the Safe Drinking Water Act (SDWA).
\begin{itemize}

\item Data are collected through UCMR to support the Administrator's determination of whether to regulate particular contaminants in the interest of protecting public health.\\

\item The UCMR program was developed in coordination with the Contaminant Candidate List (CCL) a list of contaminants that:

\item Are not regulated by the National Primary Drinking Water Regulations
\item Are known or anticipated to occur at PWSs May warrant regulation under the SDWA
\end{itemize}
\end{enumerate}
\end{itemize}



\item Surface Water Rules
\begin{itemize}
\item Any system that uses surface water must provide treatment of the supply. 
\item The minimum acceptable level of treatment is filtration and disinfection. \item Infiltration galleries may now be considered surface supplies because they are groundwater that is under the influence of surface water.
\item The concerns about contamination by Giardia and Cryptosporidium bacteria have created the need for higher free chlorine residuals and longer disinfection contact times.
\item Removal of Cryptosporidium is based on a 3-log reduction of the numbers found in raw water. 
\item An LRV of 1 is equivalent to 90\% removal of a target pathogen, an LRV of 2 is equivalent to 99\% removal and an LRV of 3 is equivalent to 99.9\% removal and so on.
\end{itemize}

\item Disinfection and Disinfection Byproducts Rule
\begin{itemize}
\item Systems that use chlorination may create TTHMs and halo acetic acids (HA5 ) as a by-product of disinfection.
 
\item If the creation of these by-products causes the system to exceed the MCL for Total TTHMs (0.1 mg/l or 100 ppb), the system will be required to change to a different means of disinfection. 
\item Total chlorine residuals are also limited to a maximum of 4.0 mg/l. 
\item The D-DBP rule currently only applies to systems serving a population over 10,000.

\item https://www.epa.gov/dwreginfo/surface-water-treatment-rules
\end{itemize}


\item Groundwater Rules
\begin{itemize}
\item California’s groundwater has gone unregulated at the state level for decades. In fact the state was one of the last to enact any laws pertaining to how this resource can be pumped and used. However, a new era of groundwater management began Sept. 17, 2014 after Gov. Jerry Brown’s signing of historic legislation, the Sustainable Groundwater Management Act (SGMA), which empowers local officials to halt the trend of critically overdraft basins.\\

\item The law, which went into effect in 2015, sets a timeline to identify responsible local agencies – which will work as a team – the means to reverse overdrafted conditions in certain areas and to ensure the 127 “high and medium priority” groundwater basins or sub-basins not in overdraft reach sustainability by 2040.\\

\item The Ground Water Rule (GWR) was signed by the EPA Administrator Stephen L. Johnson on October 11, 2006. EPA published the GWR in the Federal Register on November 08, 2006.  The GWR provides protection against microbial pathogens in public water systems using ground water sources.

\item The GWR applies to public water systems that use ground water as a source of drinking water. The rule also applies to any system that delivers surface and ground water to consumers where the ground water is added to the distribution system without treatment. 

\url{https://www.watereducation.org/aquapedia-background/groundwater-law}

\item Climate change projections are for higher temperatures and extreme droughts by the end of the 21st century. This will alter the natural recharge of groundwater, including decreased inflow from runoff, increased evaporative losses, and warmer and shorter winter seasons, impacts that are likely to exacerbate already existing groundwater overdraft in many basins. Additionally, the imported surface water that can be delivered from the Central Valley Project (CVP) and State Water Project (SWP) to areas reliant on this water for groundwater recharge and consumptive use is projected to be less reliable and more expensive. Yet groundwater is a critical water supply source during drought when it compensates for reduced surface water supplies. The need for proactive adaptation strategies to address the extreme droughts projected under climate change are frequently discussed, yet there are limited examples of such groundwater management strategies.\\ 
This paper therefore explores: \\
1) How groundwater management agencies are planning for drought \\
2) What new approaches are currently being used that show promise for addressing the more 
extreme droughts projected under climate change? \\
\item First, the paper provides a review of the research on drought and groundwater management including strategies currently used to address drought. Second, case studies illustrate newer and varied approaches being used to reduce drought impacts. Highlighted are the different approaches used by groundwater managers to both increase storage and develop drought reserves. These strategies can help to reduce vulnerability to the extreme droughts projected under climate change. Two additional case studies discuss the limits of a drought reserve strategy and indicate that more is needed under climate change to address the range of basin conditions and the varied needs of communities reliant on groundwater. 
\item Several overall groundwater management trends are noted:
\item A shift from voluntary to mandatory requirements for the sustainable management of 
groundwater after the 2014 passage of SGMA; \\
\item An increase in the use of recycled water from 190,000 AF in 1976 to 714,000 AF in 2016 that 
can be used for groundwater recharge to enhance storage; \\
\item An increase in the development of groundwater drought reserves; \\
\end{itemize}
\end{itemize}

\section{Recycled Water Regulations}\index{Recycled Water Regulations}
\begin{itemize}
\item The principal state regulatory agencies involved in water recycling in California are the California Department of Public Health (CDPH), the California State Water Resources Control Board (SWRCB), and the nine Regional Water Quality Control Boards (RWQCBs). 
\item In 1991, the SWRCB and RWQCBs were brought together with five other state environmental protection agencies under the newly crafted California Environmental Protection Agency (Cal/EPA).51
\item The nine semi-autonomous RWQCBs are divided by regional boundaries based on major 
watersheds. 
\item Each RWQCB makes water quality planning and regulatory decisions for its region. 
\item The SWRCB is generally responsible for setting statewide water quality policy and considering petitions contesting RWQCB actions. The SWRCB also makes water rights determinations.\\

\item CDPH has statutory authority in two areas with respect to direct potable reuse. It 
regulates public water systems (drinking water purveyors) and develops and adopts water 
recycling criteria.

\item The CDPH also permits operation of water treatment and distribution, and monitors drinking water quality.\\

\item Title 22 of California’s Code of Regulations refers to state guidelines for how treated and recycled water is discharged and used.\\

\item Title 22 of California’s Code of Regulations refers to state guidelines for how treated and recycled water is discharged and used.\\

\item State discharge standards for recycled water and its reuse are regulated by the 1969 Porter-Cologne Water Quality Control Act and the State Water Resources Control Board’s 2019 Water Recycling Policy.\\

\item Title 22 lists 40 specific uses allowed with disinfected tertiary recycled water (such as irrigating parks), 24 specific uses allowed with disinfected secondary recycled water (such as irrigating animal feed and other unprocessed crops), and seven specific uses allowed with undisinfected secondary recycled water (such industrial uses).\\

\item The State Water Board governs the permitting of recycled water projects, develops uniform water recycling criteria and reviews and approves Title 22 engineering reports for recycled water use.\\
\end{itemize}

1.	What is an MCL?\\
2.	Why is turbidity a Primary Contaminant?\\
3.	What is a nephelometer?\\
4.	How much is the “Water Conservation Fee”?\\
5.	How long must bacteriological and chemical sampling results be kept?\\

BASIC SAMPLE TEST QUESTIONS\\
1.	A public water system is any system that serves a population greater than or equal to:\\
A.	25\\
B.	50\\
C.	100\\

2.	What is the maximum total chlorine residual allowed by the Disinfectant-Disinfection By-Products Rule?\\
A.	2 mg/l\\
B.	4 mg/l\\
C.	6 mg/l\\
D.	8 mg/l\\

3.	What type of contaminant is iron?\\
A.	Primary Inorganic\\
B.	Primary Organic\\
C.	Secondary\\

4.	Which Primary Contaminant is sometimes added to water supplies to prevent tooth decay?\\
A.	Iron\\
B.	Arsenic\\
C.	Fluoride\\
D.	Mercury\\

5.	The failure of a public water system to comply with the NM Drinking Water Regulations must be reported to NMED within:\\
A.	12 Hours\\
B.	48 Hours\\
C.	4 Days\\
D.	One week\\

6.	The Environmental Protection Agency’s “National Interim Primary Drinking Water Regulations” require that analysis for inorganic chemicals in ground water supplies be repeated at \rule{1cm}{0.15mm} intervals.\\

a.	One-year\\
b.	Five-year\\
c.	Three-year\\
d.	None of the above\\




ADVANCED STUDY QUESTIONS\\
1.	Which Primary inorganic contaminant poses an immediate health risk?\\
2.	When you get a positive Total Coliform sample result, what is the minimum number of retakes required?\\
3.	What are the action levels for lead and copper?\\
4.	If bacteriological retakes are done this month, what is the minimum numbers of samples that must be turned in next months?\\
5.	If a 3-log removal is required for Giardia Lamblia, what percentage of organisms can survive and still meet the requirement?\\

ADVANCED SAMPLE STUDY QUESTIONS\\
1.	The MCL for Total Trihalomethanes is:\\
A.	0.01 mg/l\\
B.	0.1 mg/l\\
C.	0.2 mg/l\\
D.	2.0 mg/l\\

2.	SDWA sampling results must be reported to:\\
A.	New Mexico Water Association\\
B.	American Water Works Association\\
C.	New Mexico Environment Department\\

3.	Groundwater systems must sample for inorganic chemicals every:\\
A.	Month\\
B.	Day\\
C.	Year\\
D.	Three years\\
4.	The SDWA Compliance Cycle for the Standardized Monitoring Rule consists of three:\\
A.	Years\\
B.	Compliance Periods\\
C.	Quarters\\
D.	Months\\

6.	The Environmental Protection Agency’s “National Interim Primary Drinking Water Regulations” require that analysis for inorganic chemicals in ground water supplies be repeated at Intervals.\\

a.	One-year\\
b.	Five-year\\
c.	Three-year\\
d.	None of the above\\




















































\begin{table}[ht]
\begin{center}
\begin{tabular}{|l|c|c|}
\hline
Contaminant                               & USEPA   MCL  (mg/L) & California   MCL  (mg/L) \\
\hline
Aluminum                                  & Not   Established   & 1                        \\
\hline
Antimony                                  & 0.006               & 0.006                    \\
\hline
Arsenic                                   & 0.010               & 0.010                    \\
\hline
Asbestos                                  & 7 MFL1              & 7 MFL1                   \\
\hline
Barium                                    & 2                   & 1                        \\
\hline
Beryllium                                 & 0.004               & 0.004                    \\
\hline
Cadmium                                   & 0.005               & 0.005                    \\
\hline
Chromium,   Total                         & 0.1                 & 0.05                     \\
\hline
Chromium,   Hexavalent                    & Not   Established   & 0.0102                   \\
\hline
Cyanide                                   & 0.2                 & 0.15                     \\
\hline
Fluoride                                  & 4.0                 & 2.0                      \\
\hline
Mercury                                   & 0.002               & 0.002                    \\
\hline
Nickel                                    & Remanded            & 0.1                      \\
\hline
Nitrate   (as   Nitrogen)                 & 10                  & 10                       \\
\hline
Nitrite   (as   Nitrogen)                 & 1                   & 1                        \\
\hline
Total   Nitrate/Nitrite   (as   Nitrogen) & 10                  & 10                       \\
\hline
Perchlorate                               & Not   Established   & 0.006                    \\
\hline
Selenium                                  & 0.05                & 0.05                     \\
\hline
Thallium                                  & 0.002               & 0.002                   \\
\hline
\end{tabular}
\caption{Drinking Water Standards for Inorganic Contaminants}
\end{center}
\end{table}

\small{\begin{enumerate}
\item MFL = million fibers per liter, with fiber length > 10 microns.
\item Hexavalent Chromium MCL was withdrawn in September 2017 and is no longer in effect.
\end{enumerate}}


\begin{table}[ht]
\begin{center}
\begin{tabular}{|l|c|c|}
\hline
Contaminant  & USEPA   MCL  (mg/L) & California   MCL  (mg/L) \\
\hline
Benzene                                                                    & 0.005  & 0.001  \\ \hline
Carbon   Tetrachloride                                                     & 0.005 & 0.0005\\ \hline
1,2-Dichlorobenzene                                                        & 0.6    & 0.6  \\ \hline
1,4-Dichlorobenzene                                                        & 0.075  & 0.005 \\ \hline
1,1-Dichloroethane                                                         & Not   Established & 0.005   \\ \hline
1,2-Dichloroethane                                                         & 0.005                                                                                                   & 0.0005  \\ \hline
1,1-Dichloroethylene                                                       & 0.007   & 0.006 \\ \hline
cis-1,2-   Dichloroethylene                                                & 0.07  & 0.006   \\ \hline
trans-1,2-   Dichloroethylene                                              & 0.1    & 0.01 \\ \hline
Dichloromethane                                                            & 0.005    & 0.005 \\ \hline
1,3-Dichloropropene                                                        & Not   Established  & 0.0005 \\ \hline
1,2-Dichloropropane                                                        & 0.005  & 0.005   \\ \hline
Ethylbenzene                                                               & 0.7  & 0.3     \\ \hline
Methyl-tert-butyl   ether   (MTBE)                                         & Not   Established & 0.013 \\ \hline
Monochlorobenzene                                                          & 0.1 & 0.07\\ \hline
Styrene                                                                    & 0.1  & 0.1  \\ \hline
1,1,2,2-Tetrachloroethane & Not   Established & 0.001\\ \hline
Tetrachloroethylene                                                        & 0.005   & 0.005  \\ \hline
Toluene                                                                    & 1 & 0.15 \\ \hline
1,2,4   Trichlorobenzene                                                   & 0.07   & 0.005  \\ \hline
1,1,1-Trichloroethane                                                      & 0.200 & 0.200\\ \hline
1,1,2-Trichloroethane                                                      & 0.005  & 0.005 \\ \hline
Trichloroethylene                                                          & 0.005 & 0.005\\ \hline
Trichlorofluoromethane                                                     & Not   Established & 0.15 \\ \hline
1,1,2-Trichloro-   1,2,2Trifluoroethane                                    & Not   Established & 1.2   \\ \hline
Vinyl   chloride                                                           & 0.002              & 0.0005 \\ \hline
Xylenes                                                                    & 10                & 1.750 \\ \hline
\end{tabular}
\caption{Drinking Water Standards for Volatile Organic Compounds}
\end{center}
\end{table}


\begin{table}[ht]
\begin{center}
\begin{tabular}{|l|c|c|}
\hline
Contaminant  & USEPA   MCL  (mg/L) & California   MCL  (mg/L) \\
\hline
Uranium                                                                & 30 ug/L                                                          & 20 pCi/L   \\ \hline
Combined   Radium   -   226+228                                        & 5   pCi/L                                                         & 5   pCi/L                         \\ \hline
Gross   Alpha   particle   activity   (excluding   radon   \& uranium) & 15 pCi/L                                                          & 15 pCi/L                                                       \\ \hline
Gross   Beta   particle   activity                                     & 4 millirem/year1  & 4 millirem/year1                        \\ \hline
Strontium-90                                                           & 8   pCi/L2                                                                                                                                    & 8   pCi/L2                                                       \\ \hline
Tritium                                                                & 20,000   pCi/L3                                                  & 20,000   pCi/L3                                                  \\ \hline
\end{tabular}
\caption{Drinking Water Standards for Radionuclides}
\end{center}
\end{table}
















































\textbf{Reporting Requirements????}\\
\begin{itemize}
\item LEAD AND COPPER RULE A representative sampling survey must be conducted for lead and copper that may be present at the customers’ tap. Most of the lead and copper found this way comes from the customers’ plumbing. The system will be responsible for treating the water to stabilize the corrosive qualities that cause the leeching of lead and copper from plumbing. Sampling for lead and copper requires taking a “first draw” sample from a customer’s tap, after water has been standing in the plumbing for at least 6 hours but no longer than 18 hours. If the 90th percentile results exceed the action levels for either metal, the system must take steps to stabilize the system water through chemical addition of lime or another form of alkalinity.\\
NITRATES Nitrates are the only chemical contaminant that represent an immediate health risk. Pregnant mothers and infants under 18 months can develop a condition known as “Blue Baby Syndrome”. The presence of nitrates in the bloodstream reduces oxygen uptake that gives the skin a blue tint.\\

\item FLUORIDE Fluoride is added to water to help prevent tooth decay. The optimum dosage for fluoride is 0.8-1.2 mg/l. However, at higher concentrations, fluoride can create stains on teeth and lead to brittle bones in older individuals. The average ambient air temperature for the system is used to determine the optimum dosage for fluoride. \\
TURBIDITY Turbidity is clay, silt or mud in the water. Although turbidity does not represent a health risk by itself, it can shield harmful bacteria from disinfection processes. Turbidity is measured in Nephelometric Turbidity Units (NTU). The device used to measure NTU\\

\item BACTERIOLOGICAL CONTAMINANTS  \\
The coliform group of bacteria represents the indicator organisms used in determining bacteriological contamination. Their presence indicates the possibility that some pathogenic (disease causing) organisms may also be present. The MCL is exceeded when 5\% of the required monthly routine (M/R) samples indicate the presence of Coliform bacteria. The presence of coliform in any sample will require three repeat samples be taken. These repeat samples must be taken within 24 hrs of notification of positive results.  \\
The regulations state that, when repeats are required, a minimum of five (5) samples are now required for the month. This means that any small system that would normally only take one sample per month, will have to take four (4) repeats when they get a positive test result. If any system has to take repeat samples, it must also take a minimum of five (5) samples the following month.  \\
\item SECONDARY CONTAMINANTS There are certain substances in water that, although they do not present serious health hazards, can cause temporary physical discomfort and make the water unsuitable for use. Each state may determine which of these standards are included in their regulations. Chlorides can make the water taste salty. This is also known as brackish water. Sulphates can cause minor gastro-intestinal problems. Iron and manganese can result in red or black water problems. The pH of the treated water can also create some digestive problems if it is very high or very low.
\end{itemize}
MONITORING AND REPORTING The public water systems are responsible for monitoring their water quality and reporting violations of the SDWA standards to the public. The New Mexico Environment Department is currently collecting and submitting samples to the laboratory for all public water supplies. The program is funded through a “Water Conservation Fee” of 3 cents per 1000 gallons paid by each system. However, the systems will still be responsible for the results of testing and any public notification that may be required. Systems must retain copies of chemical analysis records for 10 years and bacteriological tests results for 5 years. SAMPLING SCHEDULES Samples used in testing for chemical and biological contaminants must be collected periodically. Samples for inorganic chemical analysis must be submitted once every year for surface supplies and once every three years for ground water supplies. Sampling for organic compounds is done quarterly for the initial set of samples. Surface water plants must also collect four TTHM samples quarterly during this initial period. After that, samples are collected yearly for surface water and every three years for ground water as long as no VOC’s or SOC’s are detected. If they are found, the source (well or surface supply) must be sampled every quarter. Radiological samples are taken every four years. Under the new Standardized Monitoring Rule, most chemical contaminants are monitored in a cycle of 3/6/9 years. Each three (3) year period is referred to as a compliance period. Bacteriological sampling schedules vary from state to state. A minimum of one sample per month is normally required for the smallest systems. As the population served increases so does the number of samples required. Whenever compliance samples are submitted it is important to maintain a “chain of custody” that identifies who handled the sample from the time it was taken until it was tested.  \\
BACTERIOLOGICAL VIOLATIONS When a positive BAC-T sample is reported repeat samples are required. If the repeats come back negative there is no violation. If more than 5\% of the monthly samples are positive for Total Coliform (TC), including repeats, there is a non-acute violation that requires public notification. This means that any system taking less than 40 samples per month can only have 1 total coliform positive sample per month. If a monthly routine sample is positive for TC and for fecal or E. Coli; and any repeat is positive for TC, OR if any of the repeats are positive for fecal coliform, or E. Coli, an acute violation has occurr0-ed that requires notification through the electronic media. This sometimes triggers a “Boil Order” advisory.  \\
PUBLIC NOTIFICATION The water system will be required to notify the public any time maximum contaminant levels are exceeded. These violations of the standards fall into two categories: acute violations and non-acute violations. A non-acute violation occurs when an MCL is exceeded but the situation does not present an immediate health risk to the public. In this case, notification must be placed on or with the billing notice within 45 days and must run in the newspaper within 14 days. In addition, all new customers must be sent notice of violations when they connect to the system. Acute violations are violations that could result in an immediate danger to the public health and therefore require immediate notification through television and radio stations within 72 hours. This is in addition to the newspaper and/or billing notifications. Public notification must continue until the problem is corrected. Notification must also be given to the NMED within 48 hours any time a system fails to comply with the NM Drinking Water Regulations.  \\
ACTION PLANS FOR VIOLATIONS If a water supply exceeds the primary standards the water system must either provide adequate treatment to remove the contaminants or locate a new source of supply that meets these requirements.  \\
VARIANCES AND EXEMPTIONS A system that is found to exceed the MCL for a primary contaminant may not be able to correct the problem for financial or technical reasons. Depending on the circumstances, the system may be granted a variance orexemption. The fact that a variance or exemption has been granted does not mean that the system is no longer required to notify the public of the problem. Notification must continue on a monthly basis until the system meets the standard. \\
Variances \\
A variance may be granted to a water system when its supply is found to exceed maximum standards and no technology is available to economically remove these contaminants. Variances may be extended at the discretion of the state regulatory agency if no treatment methods are made available during the period the variance is granted. \\
Exemptions \\
When a system is unable to financially provide the necessary treatment to reduce contaminant levels to acceptable limits, an exemption can be granted to the water system. Exemptions are granted by state regulatory agencies only in cases where a serious health hazard is not present. \\
OTHER NEW REGULATIONS \\
The 1986 amendments to the SDWA included a number of new rules regarding treatment and operations of public water supplies. The major changes are identified below with a brief description of the rule and its implications. \\
SURFACE WATER RULE \\
Any system that uses surface water must provide treatment of the supply. The minimum acceptable level of treatment is filtration and disinfection. Infiltration galleries may now be considered surface supplies because they are groundwater that is under the influence of surface water. The concerns about contamination by Giardia and Cryptosporidium bacteria have created the need for higher free chlorine residuals and longer disinfection contact times. \\
The “CT” calculation is used to determine the necessary contact time at any given concentration. The formula is C x T = A, where C is the chlorine concentration, T is the contact time in minutes, and A is a temperature-based constant. Removal of Cryptosporidium is based on a 3-log reduction of the numbers found in raw water. A 3-log removal or deactivation would mean that 0.1% of the bacteria may survive or 99.9% were removed. A 4- log removal or deactivation would mean that 0.01% of the organisms may survive or 99.99% were removed \\
DISINFECTION AND DISINFECTION BY-PRODUCTS RULE \\
Systems that use chlorination may create TTHMs and halo acetic acids (HA5 ) as a by-product of disinfection.   \\

\begin{itemize}
\item The SWB along with the nine Reional Water Quality Control Boards are responsible for regulation of the state's drinking water and for protecting the waters of the state including drinking water sources
\item In 2014 the repsonsibility of regulating the water quality was transferred to SWB
\item SWB has the responsibility for regulating all Public Water System
\item A public water system is defined as a system that provides water for human consumption to 15 or more connections or regularly serves 25 or more.\\

\item The CPUC shares regulatory responsibility for ensuring the quality of water supplied by investor owned water utilityes and is responsiible for overseeing their rate structure. 
\end{itemize}

Unregulated Drinking Water Contaminants\\
This list of contaminants which, at the time of publication, are not subject to any proposed or promulgated national primary drinking water regulation (NPDWRs), are known or anticipated to occur in public water systems, and may require regulations under the Safe Drinking Water Act (SDWA).\\

The Contaminant Candidate List (CCL) is a list of drinking water contaminants that are known or anticipated to occur in public water systems and are not currently subject to EPA drinking water regulations.\\



AB 685 affirms California’s commitment to ensuring affordable, accessible, acceptable and safe water sufficient to protect the health and
dignity of all its residents.\\

With the passage of AB 685 in 2012, California became one of the first states in the United States
to recognize the human right to water. California now has a comprehensive law guaranteeing the right
to safe, affordable water without discrimination, prioritizing water for personal and domestic use and
delineating the responsibilities of public officials at the state level. AB 685 specifically charges relevant
California agencies with fulfillment of the law’s mandate by considering the human right to water in
policy, programming, and budgetary activities.\\

AB 685 identifies a specific list
of factors—safety, affordability, and accessibility—that agencies must consider when revising,
adopting, or establishing policies, regulations, and
grant criteria related to domestic water use.\\




Title 22 of California’s Code of Regulations refers to state guidelines for how treated and recycled water is discharged and used.\\

State discharge standards for recycled water and its reuse are regulated by the 1969 Porter-Cologne Water Quality Control Act and the State Water Resources Control Board’s 2019 Water Recycling Policy.\\

Title 22 lists 40 specific uses allowed with disinfected tertiary recycled water (such as irrigating parks), 24 specific uses allowed with disinfected secondary recycled water (such as irrigating animal feed and other unprocessed crops), and seven specific uses allowed with undisinfected secondary recycled water (such industrial uses).\\

Other allowed uses of the disinfected recycled water include irrigation of food crops and residential landscaping, supply of recreational impoundments for unrestricted body contact, air conditioning, commercial laundry, decorative fountains, and flushing toilets in commercial buildings.\\

The State Water Board governs the permitting of recycled water projects, develops uniform water recycling criteria and reviews and approves Title 22 engineering reports for recycled water use.\\



PHGs are necessary guides for making decisions about the levels of chemical contaminants in drinking water, but these guidance levels are just one element that SWRCB must consider when maintaining the quality of drinking water.   By law, SWRCB must set the state’s regulatory standards, known as Primary Maximum Contaminant Levels (MCLs), as close as possible to the PHG levels that OEHHA establishes. However, SWRCB must also consider the cost and technological feasibility of treating or preventing chemical contamination. \\

The Calderon‐Sher Safe Drinking Water Act requires OEHHA to develop a PHG for each drinking water contaminant that is regulated with an MCL. OEHHA must also develop a PHG before SWRCB can establish an MCL for a contaminant for the first time. SWRCB must review a primary MCL at least every five years and amend it, if necessary, to make it as close to the corresponding PHG as is feasible. SWRCB could amend an MCL if the PHG evaluation indicates that the contaminant is more or less toxic than was previously believed, or if new technology is available to reduce concentrations to levels closer to the PHG. \\

Is Water Safe to Drink if Contaminant Levels Exceed Public Health Goals?\\
 As long as drinking water complies with all MCLs, it is considered safe to drink, even if some contaminants exceed PHG levels. A PHG represents a health‐protective level for a contaminant that SWRCB and California’s public water systems should strive to achieve if it is feasible to do so. However, a PHG is not a boundary line between a “safe” and “dangerous” level of a contaminant, and drinking water can still be considered acceptable for public consumption even if it contains contaminants at levels exceeding the PHG.\\
How Can the Public Learn More About Contaminants in the Water? \\
California law requires that public water systems inform consumers about the quality of their drinking water through the following reports:\\
Annual Consumer confidence Reports\\
 Public water systems are required to send each customer an annual consumer confidence report that describes the source of the water supply and any contaminants detected in it. The report must list the current level of a contaminant as well as its PHG and primary MCL. The report must also disclose if an MCL was exceeded and include a plainly worded statement of associated health concerns.\\

Exceedance Reports\\
 Water systems with more than 10,000 service connections are legally required to prepare an exceedance report every three years if one or more chemical contaminants exceed PHG levels. The report provides information on health risks posed by the contaminants as well as the costs and technology needed to reduce the contaminants to the PHG level. The report must also explain what action, if any, the local water supplier has planned to address the contamination. The water supplier must hold a public hearing on the report.\\
Other Notification Requirements\\
When a contaminant in a public drinking water source exceeds the primary MCL, the water supplier must notify its customers in accordance with SWRCB requirements. In instances where there is an imminent threat to human health, the water supplier would have to provide immediate notice to customers. The law requires SWRCB to approve the content of such notices.\\

The Safe Drinking Water Act (SDWA) is the principal federal law. The SDWA authorizes the United States Environmental Protection Agency (EPA) to create and enforce regulations to achieve the SDWA goals.\\

Federal requirements:\\
The Safe Drinking Water Act is the principal federal law governing public water systems.[1] These systems provide drinking water through pipes or other constructed conveyances to at least 15 service connections, or serve an average of at least 25 people for at least 60 days a year. As of 2017 there are over 151,000 public water systems.\\
\begin{itemize}
\item Approximately 52,000 Community Water Systems serve the majority of the U.S. population
\item Approximately 85,000 systems are non-transient, non-community water systems (such as schools, factories, office buildings, and hospitals that operate their own systems)
\item Approximately 18,000 systems are transient, non-community water systems (such as rural gas stations or campgrounds).
\end{itemize}
Eight percent of the Community Water Systems—large municipal water systems—provide water to 82 percent of the US population.\\
The SDWA authorized the EPA to promulgate regulations regarding water supply. The major regulations are in Title 40 of the Code of Federal Regulations: 40 CFR Parts 141, 142, and 143. Parts 141, 142, and 143 regulate primary contaminants, implementation by states, and secondary contaminants. Primary contaminants are those with health impacts. State implementation allows states to be the primary regulators of the water supplies (rather than EPA) provided they meet certain requirements. Secondary contaminants generally cause aesthetic problems and are not directly harmful.\\
The SDWA also contains provisions that require water supplies to develop emergency plans, water supply operators to be licensed, and watersheds to be protected. The Act does not cover private wells.\\
\textbf{National Primary Drinking Water Regulations}\\

Types of water systems\\
Part 141 regulates public water systems based on size (population served) and type of water consumers. Larger water systems and water systems serving year-round residents (cities) have more requirements than smaller water systems or those serving different people each day (e.g., a shopping mall). In 2009, public water systems on commercial airlines were included\\
Control of contaminants\\
The drinking water standards are organized into six classes of contaminants: Microorganisms, Disinfectants, Disinfection Byproducts, Inorganic Chemicals, Organic Chemicals and Radionuclides. The standards specify either Maximum Contaminant Levels (MCLs) or Treatment Techniques (enforceable procedures).\\ 
The most recent major standard-setting rules include:
\begin{itemize}
\item Ground Water Rule (2006)
\item Long Term 2 Enhanced Surface Water Treatment Rule (2006) for control  Cryptosporidium and other pathogens. 
\item Stage 2 Disinfectants and Disinfection Byproducts Rule (2006)
\item Lead and Copper Rule (last revised 2021).
\end{itemize}
Monitoring and reporting\\
Testing is required to determine compliance with maximum contaminant levels. The regulations specify when and how samples are to be taken and analyzed. For example:
\begin{itemize}
\item The Information Collection Rule required large public water systems to collect samples in the late 1990s to provide data for designing new regulations or revising regulations related to pathogen contamination in surface water and disinfection byproduct production.
\item The Unregulated Contaminant Monitoring Rules require certain water systems to test for contaminants which do not yet have drinking water limits. The resulting information is used to prioritize the regulation of new contaminants. Section 141.40 includes the latest list of proposed contaminants. In 2012, the third set of contaminants (UCMR3) replaced the previous set (UCMR2).\\ 
\end{itemize}
The regulations specify who must be notified and the manner of the notification. One such provision is Subpart O, Consumer Confidence Reports. These reports are a summary of the water supplies sources and water quality testing results. The reports must be sent to all customers annually.[15][16] Subpart Q regulates how violations must be reported.\\
\textbf{National Primary Drinking Water Regulations implementation}\\
EPA issued the implementation regulations in Part 142 pursuant to the Public Health Service Act and the SDWA. Oversight of public water systems is managed by "primacy" agencies, which are either state government agencies, Indian tribes or EPA regional offices.[18] All state and territories, except Wyoming and the District of Columbia, have received primacy approval from EPA, to supervise the PWS in their respective jurisdictions.Generally, a primacy agency must incorporate the requirements of the National Primary Drinking Water Regulations in its own regulations. States may be more stringent, but not less stringent, than the federal rules. Federal funding is available to primacy agencies that implement or enforce some or all of the federal requirements.\\
\textbf{National Secondary Drinking Water Regulations}\\
The relatively short Secondary Regulations at Part 143 provide guidance for aesthetic characteristics, including taste, color, and odor, but do not actually regulate public water systems. "The regulations are not Federally enforceable but are intended as guidelines for the States.  Although not federally enforceable, some states regulate the secondary contaminants.\\
The guidelines include recommendations for maximum concentrations for 15 contaminants, when to sample, and how to analyze the samples. Some contaminants in the Secondary Regulations are also regulated in the Primary Regulations. This generally occurs when a contaminant is a nuisance at a low level, but toxic at a higher concentration\\
textbf{Compliance}\\
Municipalities throughout the United States, from the largest cities to the smallest towns, sometimes fail to meet EPA standards. The EPA may fine the jurisdiction responsible for the violation, but this does not always motivate the municipality to take corrective action. In such cases, non-compliance with EPA may continue for many months or years after the initial violation. This could result from the fact that the city simply doesn't have the financial resources necessary to replace aging water pipes or upgrade their purification equipment. In rare cases, the source water used by the municipality could be so polluted that water purification processes can't do an adequate job. This can occur when a town is downstream from a large sewage treatment plant or large-scale agricultural operations. Citizens who live in such places—especially young children, the elderly, or people of any age with autoimmune deficiencies—may suffer serious health complications as a long-term result of drinking water from their own taps\\]

\textbf{California}\\
Timeline of existing federal water and state drinking water quality regulations:
\begin{itemize}
\item National Interim Primary Drinking Water Regulations (NIPDWR)
\begin{itemize}
\item Promulgated 1975-1981
\item Contained 7 contaminants
\item Targeted: trihalomethanes, arsenic, and radionuclides
\item Established 22 drinking water standards
\end{itemize}
\item Phase 1 standards
\begin{itemize}
\item Promulgated 1987
\item Contained 8 contaminants
\item Targeted: VOCs
\end{itemize}
\item Phase 2 standards
\begin{itemize}
\item Promulgated 1991
\item Contained 36 contaminants
\item Targeted: VOCs, SOCs, and IOCs
\end{itemize}
\item Phase 5 standards
\begin{itemize}
\item Promulgated 1992
\item Contained 23 contaminants
\item Targeted: VOCs, SOCs, and IOCs
\end{itemize}
\item Surface Water Treatment Rule (SWTR)
\begin{itemize}
\item Promulgated 1989
\item Contained 5 contaminants
\item Targeted: Microbiological and Turbidity
\end{itemize}
\item Stage 1 Disinfectant/Disinfection By-product(D/DBP) Rule
\begin{itemize}
\item Promulgated 1998
\item Contained 14 contaminants
\item Targeted: DBPs and precursors
\end{itemize}
\item Interim Enhanced Surface Water Treatment Rule (IESWTR)
\begin{itemize}
\item Promulgated 1998
\item Contained 2 contaminants
\item Targeted: Microbiological and Turbidity
\end{itemize}
\item Radionuclide Rule
\begin{itemize}
\item Promulgated 2000
\item Contained 4 contaminants
\item Targeted: Radionuclides
\end{itemize}
\item Arsenic Rule
\begin{itemize}
\item Promulgated 2001
\item Contained 1 contaminant
\item Targeted: Arsenic
\end{itemize}
\item Filter Backwash Recycling Rule
\begin{itemize}
\item Promulgated 2001
\item Contained -
\item Targeted: Microbiological and Turbidity
\end{itemize}
\end{itemize}
