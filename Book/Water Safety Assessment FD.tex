What federal law is designed to protect the safety and health of operators?\\
A.	OSHA\\
B.	FMLA\\
C.	FLSA\\
D.	ADEA

What are the two most important safety concerns when entering a confined space?\\
A.	Corrosive chemicals and falls\\
B.	Bad odors and claustrophobia\\
C.	Extreme air temperatures and slippery surfaces\\
D.	Oxygen deficiency and hazardous gases

Which document provides a profile of hazardous substances?\\
A.	CERCLA\\
B.	SARA\\
C.	CFR\\
D.	MSDS

What is the purpose of a pump guard?\\
A.	Allows operators to turn off pump in emergency situations\\
B.	Notifies operators of excessive temperatures\\
c.	Allows operators to pump against a closed discharge valve\\
D.	Protects operators from rotating parts

Atmosphere is considered oxygen deficient when the oxygen level is below\\
A.	21.5\%\\
B.	20\%\\
C.	19.5\%\\
D.  17\%


Employee hazards include\\
A. Noxious or toxic gases or vapors\\
B. Oxygen deficiency\\
C. Physical injuries\\
D. All of the above\\

Before entering a permit-required confined space, you must:\\
A. Check the atmosphere with a calibrated gas detector.\\
B. Make notification that personnel are entering the space.\\
C. Lock out and tag out all equipment.\\
D. All of the above.
	 
When making a sulfuric acid dilution, the appropriate method is:\\
A. Add the water to the acid.\\
B. Add the acid to the water.\\
C. Add both at the same time.\\
D. None of the above.

When manually lifting any object, be sure to\\
A. Hold it at arm's length.\\
B. Keep your back bent and hold it low.\\
C. Keep it close to your body and use leg strength.\\
D. Keep your knees locked and bend at the waist.


What is the proper slope of a ladder?\\
A.	Every 4 feet up the ladder is 1 foot out from the wall.\\
B.	Every 5 feet up the ladder is 1 foot out from the wall. \\
C.	Every 6 feet up the ladder is 1 foot out from the wall.\\
D.	Every 7 feet up the ladder is 1 foot out from the wall.


When working on a chemical feed pump, what of the following is not required?\\
A.	Nitrile gloves.\\
B.	Safety glasses.\\
C.	Leather work gloves.\\
D.	Full face shield.

When must the atmosphere of a confined space be tested?\\
A.	Only before a worker enters\\
B.	Never, if adequate ventilation exists\\
C.	Continuously\\
D.	Only if welding or painting is being performed

Some gases in a confined space can be:\\
A.	Colorless\\
B.	Odorless\\
C.	Deadly\\
D.	All of the above

Why should you contact other area companies with underground utilities before starting an underground repair job?\\
a.	 To determine if there have been recent excavations in that location\\
b.	 To ask these companies to mark the location of their utilities in the area of the repair job\\
c.	 To see if they also have excavating to do in the area\\
d.	 To see if they will help route traffic while you are doing the repair job\\
The only acceptable breathing device to wear while handling chlorine leaks is the\\
a. Activated carbon canister type\\
b. Potassium tetroxide canister type\\
c. Self-contained breathing apparatus\\
d.	Oxygen supply apparatus\\
It is essential to ventilate a vault before entry in order to\\
a. Remove excessive moisture\\
b. Equalize temperature and pressure\\
c. Eliminate foul odors\\
d.	 Remove dangerous gasses\\
Permit-required confined space entry requires\\
a. Bright orange jackets, rubber boots, and gloves\\
b. Safety harness and a lifeline\\
c. Tool belts with flashlight attached\\
d. Utility belts with a full complement of tools\\
During a confined space entry, how often must the confined space be monitored for hazardous atmospheres?\\
a.  Continuously\\
b.  Every five minutes\\
c. Before entry only\\
d. Before entry and then once per hour during entry\\
  Which of the following is the most likely to be a fuel involved in a Class A fire?\\
a. Butane\\
b. Magnesium\\
c. Electrical equipment\\
d. Gasoline\\
e. Paper and/or fabrics\\
  In an occupied trench where exits (i.e., ladders) are required, what is the maximum allowed travel distance between an occupant and the nearest exit?\\
i. 25 feet\\
b. 50 feet\\
c. 100 feet\\
d. At the discretion of the safety officer\\
e. None of the above\\
  Standard first aid procedures direct that the first step to control bleeding is to\\
a. Apply a tight tourniquet\\
b. Apply pressure directly to the wound\\
c. Let it bleed until natural clotting takes place\\
d. Wash wound and bandage\\
e. None of the above\\
  When excavating materials that will not stand in a vertical position, the most suitable form of shoring is\\
a. Air shores\\
b. Hydraulic shores\\
c. Screw jacks\\
d. Solid sheeting\\
e. Cleats\\
  A potable water supply discharges into an irrigation water storage tank. The 3-inch potable supply line should be terminated\\
a. Above the tank overflow by at least two pipe diameters\\
b. Above the tank outlet by at least two pipe diameters\\
c. Below the tank outlet by at least two pipe diameters\\
d. Level with the tank outlet\\
e. Level with the tank overflow\\
  Which of the following gases is toxic at the lowest concentration?\\
a. Carbon dioxide\\
b. Hydrogen sulfide\\
c. Methane\\
d. Nitrogen\\
e. Oxygen\\
  Entry into an atmosphere with high concentrations of chlorine gas requires\\
a. A self-contained breathing apparatus\\
b. An approved and uncontaminated canister mask\\
c. Forced ventilation of the work area\\
d. Atmospheric testing with ammonia solution prior to entry\\
e. Rubber gloves and a full-face shield

Shoring is normally required (per OSHA guidelines) for trenches of what minimum depth?\\
a. $\quad 4$-feet\\
b. $\quad 5$-feet\\
c. $\quad 6$-feet\\
d. $\quad 7$-feet\\
e. $\quad 8$-feet\\

  First aid for first-degree burns is to\\
a. Bandage tightly\\
b. Cover liberally with salve\\
c. Pack in ice\\
d. Submerge the burned area in cold water\\
e. All of the above\\
  What information must be on a warning tag attached to a locked-out switch?\\
a. Directions for removing the tag\\
c. Signature of the person who locked out the switch and who will remove it\\
d. Time to unlock the switch\\
e. None of the above\\
  A confined space that contains a material that has the potential for engulfing an entrant is\\
a. A transition zone\\
b. A permit space\\
c. Prohibited by OSHA\\
d. Required to undergo atmospheric testing with ammonia solution prior to entry\\
e. S Required to use a complete "A" suit for personal protective equipment\\
 What condition must exist for an area to be considered a confined space?\\
a. Limited or restricted means of entry or exit\\
b. Is large enough for a person to enter and perform work\\
c. Is not designated for continuous occupancy\\
d. All of the above\\
e. None of the above\\
Which of the following is the most likely to be a fuel involved in a Class C fire?\\
a. Butane\\
b. Magnesium\\
c. Paper and/or fabrics\\
d. Gasoline\\
e. Electrical equipment\\
  Which of the following is the most likely to be a fuel involved in a Class B fire?\\
a. Wood\\
b. Magnesium\\
c. Electrical equipment\\
d. Gasoline\\
e. Paper and/or fabrics\\

The angle of repose is the angle of the slope of a\\
a. Sewer\\
b. Graded and/or cut ground elevation\\
c. Trench excavation\\
d. Unsupported loose soil\\
e. Filled and compacted ground elevation\\

At least 48 hours prior to conducting excavations in locations where other utilities may be present, whom should you notify?\\
a. WARN\\
b. USA\\
c. AWWA\\
d. DHS\\
e. EPA\\

Which of the following compounds emits a "rotten egg" odor?\\
a. Hydrogen sulfide\\
b. Chorine dioxide\\
c. Chloramines\\
d. Hydrochloric acid\\
e. Hypochlorous acid\\

Where is the best place to store a self -contained breathing apparatus (SCBA)?\\
a.	inside a cabinet in the chlorinator room\\
b.	in  an unlocked cabinet outside the chlorinator room\\
c.	locked in a cabinet in the office\\
d.	locked in a cabinet just outside the chlorinator room\\

Which of the following is a hazard when handling hydrofluosilicic acid?\\
a.	fire\\
b.	explosion\\
c.	corrosion\\
d.	inhalation\\

Which of the following chemical substances ii most likely to cause corrosion or deterioration of metal and concrete surfaces\\
a.	carbon dioxide\\
b.	ethanol\\
c.	methane\\
d.	hydrogen sulfide\\


An employee ls caught in a room where ch1orine gas is leaking.  He has no SCBA, he should\\
a.	lay down on the floor and quickly crawl out of the room \\
b.	walk out of the room quickly\\
c.	pull shirt over mouth and face and quickly walk out of the room\\
d.	keep mouth closed, head as high as possible, and quickly walk out of the room holding breath.\\

It is essential to ventilate a vault before entry in order to\\
a. Rennove exçessive moisture\\
b. Equalize temperature and pressure\\
c. Eliminate foul odors\\
d. Remove dangerous gasses\\

A portable ladder must extend at least feet above the upper surface of an excavated trench.\\
a. 1\\
b. 3\\
c. 4\\
d. 4.5\\

A trench must be shored if it is \rule{1.5cm}{0.5mm} feet deep or more.\\
a. 3\\
b. 4\\
c. 5\\
d. 6\\

When employees are working in a trench $5 \mathrm{ft}$ deep or more, an adequate means of exit, such as a ladder or steps, must be located no mote than ft away from them.\\
a. 5\\
b. 10\\
c. 25\\
d. 40\\


Permit-required confined space entry requires\\
a. Bright orange jackets, rubber boots, and gloves\\
b. Safety harness and a lifeline\\
c. Tool belts with flashlight attached\\
d. Utility belts with a full complement of tools\\


