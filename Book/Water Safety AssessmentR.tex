%\chapterimage{QuizCover} % Chapter heading image

% \chapter*{Safety}
% \textbf{Multiple Choice}
\begin{enumerate}[1.]

\item What is the most important aspect in maintaining a high degree of safety awareness in the water treatment facility?\\
*a. Making sure carelessness or negligence is stressed\\
b. driver's training away from the workplace\\
c. reading the MSDS postings each day\\
d. maintaining batteries in flashlights and emergency storage areas\\
\item Which is the approximate angle of repose for average soils when using the sloping method for the prevention of cave-ins? (Note: horizontal to vertical distance, respectively)\\
a. $0.5: 1.0$\\
b. $1.0: 1.0$\\
*c. $1.5: 1.0$\\
d. $2.0: 1.0$\\
\item What federal law is designed to protect the safety and health of operators?\\
*a. OSHA\\
b. FMLA\\
c. FLSA\\
d. ADEA\\
\item What are the two most important safety concerns when entering a confined space?\\
a. Corrosive chemicals and falls\\
b. Bad odors and claustrophobia\\
c. Extreme air temperatures and slippery surfaces\\
*d. Oxygen deficiency and hazardous gases\\
\item Which document provides a profile of hazardous substances?\\
a. CERCLA\\
b. SARA\\
c. CFR\\
*d. SDS\\
\item What is the purpose of a pump guard?\\
a. Allows operators to turn off pump in emergency situations\\
b. Notifies operators of excessive temperatures\\
c. Allows operators to pump against a closed discharge valve\\
*d. Protects operators from rotating parts\\
\item Atmosphere is considered oxygen deficient when the oxygen level is below\\
a. $21.5 \%$\\
b. $20 \%$\\
*c. $19.5 \%$\\
d. $17 \%$\\
\item Employee hazards include\\
a. Noxious or toxic gases or vapors\\
b. Oxygen deficiency\\
c. Physical injuries\\
*d. All of the above\\
\item Before entering a permit-required confined space, you must:\\
a. Check the atmosphere with a calibrated gas detector.\\
b. Make notification that personnel are entering the space.\\
c. Lock out and tag out all equipment.\\
*d. All of the above.\\
\item When making a sulfuric acid dilution, the appropriate method is:\\
a. Add the water to the acid.\\
*b. Add the acid to the water.\\
c. Add both at the same time.\\
d. None of the above.\\
\item When manually lifting any object, be sure to\\
a. Hold it at arm's length.\\
b. Keep your back bent and hold it low.\\
*c. Keep it close to your body and use leg strength.\\
d. Keep your knees locked and bend at the waist.\\
\item What is the proper slope of a ladder?\\
*a. Every 4 feet up the ladder is 1 foot out from the wall.\\
b. Every 5 feet up the ladder is 1 foot out from the wall.\\
c. Every 6 feet up the ladder is 1 foot out from the wall.\\
d. Every 7 feet up the ladder is 1 foot out from the wall.\\
\item When working on a chemical feed pump, what of the following is not required?\\
a. Nitrile gloves.\\
b. Safety glasses.\\
*c. Leather work gloves.\\
d. Full face shield.\\
\item When must the atmosphere of a confined space be tested?\\
a. Only before a worker enters\\
b. Never, if adequate ventilation exists\\
c. Continuously\\
d. Only if welding or painting is being performed\\
\item Some gases in a confined space can be:\\
a. Colorless\\
b. Odorless\\
c. Deadly\\
d. All of the above\\
\item Why should you contact other area companies with underground utilities before starting an underground repair job?\\
a. To determine if there have been recent excavations in that location b. To ask these companies to mark the location of their utilities in the area of the repair job\\
c. To see if they also have excavating to do in the area\\
d. To see if they will help route traffic while you are doing the repair job\\
\item The only acceptable breathing device to wear while handling chlorine leaks is the\\
a. Activated carbon canister type\\
b. Potassium tetroxide canister type\\
c. Self-contained breathing apparatus\\
d. Oxygen supply apparatus\\
\item It is essential to ventilate a vault before entry in order to\\
a. Remove excessive moisture\\
b. Equalize temperature and pressure\\
c. Eliminate foul odors\\
d. Remove dangerous gasses\\
\item Permit-required confined space entry requires\\
a. Bright orange jackets, rubber boots, and gloves\\
b. Safety harness and a lifeline\\
c. Tool belts with flashlight attached\\
d. Utility belts with a full complement of tools\\
\item During a confined space entry, how often must the confined space be monitored for hazardous atmospheres?\\
a. Continuously\\
b. Every five minutes\\
c. Before entry only\\
d. Before entry and then once per hour during entry\\
\item Which of the following is the most likely to be a fuel involved in a Class A fire?\\
a. Butane\\
b. Magnesium\\
c. Electrical equipment\\
d. Gasoline\\
e. Paper and/or fabrics\\
\item In an occupied trench where exits (i.e., ladders) are required, what is the maximum allowed travel distance between an occupant and the nearest exit?\\
i. 25 feet\\
b. 50 feet\\
c. 100 feet\\
d. At the discretion of the safety officer\\
e. None of the above\\
\item Standard first aid procedures direct that the first step to control bleeding is to\\
a. Apply a tight tourniquet\\
*b. Apply pressure directly to the wound\\
c. Let it bleed until natural clotting takes place\\
d. Wash wound and bandage\\
e. None of the above\\
\item When excavating materials that will not stand in a vertical position, the most suitable form of shoring is\\
a. Air shores\\
b. Hydraulic shores\\
c. Screw jacks\\
*d. Solid sheeting\\
e. Cleats\\

\item Which of the following gases is toxic at the lowest concentration?\\
a. Carbon dioxide\\
*b. Hydrogen sulfide\\
c. Methane\\
d. Nitrogen\\
e. Oxygen\\
\item Entry into an atmosphere with high concentrations of chlorine gas requires\\
*a. A self-contained breathing apparatus\\
b. An approved and uncontaminated canister mask\\
c. Forced ventilation of the work area\\
d. Atmospheric testing with ammonia solution prior to entry\\
e. Rubber gloves and a full-face shield\\
\item Shoring is normally required (per OSHA guidelines) for trenches of what minimum depth?\\
*a. 4-feet\\
b. 5-feet\\
c. 6-feet\\
d. 7-feet\\
e. 8 -feet\\
\item First aid for first-degree burns is to\\
a. Bandage tightly\\
b. Cover liberally with salve\\
c. Pack in ice\\
*d. Submerge the burned area in cold water\\
e. All of the above\\
\item What information must be on a warning tag attached to a locked-out switch?\\
a. Directions for removing the tag\\
*c. Signature of the person who locked out the switch and who will remove it\\
d. Time to unlock the switch\\
e. None of the above\\
\item A confined space that contains a material that has the potential for engulfing an entrant is\\
a. A transition zone\\
*b. A permit space\\
c. Prohibited by OSHA\\
d. Required to undergo atmospheric testing with ammonia solution prior to entry\\
e. Required to use a complete "A" suit for personal protective equipment\\
\item What condition must exist for an area to be considered a confined space?\\
a. Limited or restricted means of entry or exit\\
b. Is large enough for a person to enter and perform work\\
c. Is not designated for continuous occupancy\\
*d. All of the above\\
e. None of the above\\
\item Which of the following is the most likely to be a fuel involved in a Class C fire?\\
a. Butane\\
b. Magnesium\\
c. Paper and/or fabrics\\
d. Gasoline\\
*e. Electrical equipment\\
\item Which of the following is the most likely to be a fuel involved in a Class B fire?\\
a. Wood\\
b. Magnesium\\
c. Electrical equipment\\
*d. Gasoline\\
e. Paper and/or fabrics\\
\item The angle of repose is the angle of the slope of a\\
a. Sewer\\
*b. Graded and/or cut ground elevation\\
c. Trench excavation\\
d. Unsupported loose soil\\
e. Filled and compacted ground elevation\\
\item At least 48 hours prior to conducting excavations in locations where other utilities may be present, whom should you notify?\\
a. WARN\\
*b. USA\\
c. AWWA\\
d. DHS\\
e. EPA\\
\item Which of the following compounds emits a "rotten egg" odor?\\
*a. Hydrogen sulfide\\
b. Chorine dioxide\\
c. Chloramines\\
d. Hydrochloric acid\\
e. Hypochlorous acid\\
\item Where is the best place to store a self-contained breathing apparatus (SCBA)?\\
a. inside a cabinet in the chlorinator room\\
*b. in an unlocked cabinet outside the chlorinator room\\
c. locked in a cabinet in the office\\
d. locked in a cabinet just outside the chlorinator room\\
\item Which of the following is a hazard when handling hydrofluosilicic acid?\\
a. fire\\
b. explosion\\
c. corrosion\\
*d. inhalation\\
\item Which of the following chemical substances ii most likely to cause corrosion or deterioration of metal and concrete surfaces\\
a. carbon dioxide\\
b. ethanol\\
c. methane\\
*d. hydrogen sulfide\\
\item An employee ls caught in a room where ch1orine gas is leaking. He has no SCBA, he should\\
a. lay down on the floor and quickly crawl out of the room\\
b. walk out of the room quickly\\
c. pull shirt over mouth and face and quickly walk out of the room\\
*d. keep mouth closed, head as high as possible, and quickly walk out of the room holding breath.\\
\item It is essential to ventilate a vault before entry in order to\\
a. Rennove exçessive moisture\\
b. Equalize temperature and pressure\\
c. Eliminate foul odors\\
*d. Remove dangerous gasses\\
\item A portable ladder must extend at least feet above the upper surface of an excavated trench.\\
a. 1\\
*b. 3\\
c. 4\\
d. 4.5\\
\item A trench must be shored if it is feet deep or more.\\
a. 3\\
*b. 4\\
c. 5\\
d. 6\\
\item When employees are working in a trench 5 ft deep or more, an adequate means of exit, such as a ladder or steps, must be located no mote than \rule{2cm}{2pt} ft away from them.\\
a. 5\\
b. 10\\
*c. 25\\
d. 40\\

\item What should a supervisor do if an employee is performing work in\\
an unsafe manner?\\
a.	Discuss the incident vvith fhe employee clnriug the next performance appraisal\\
*b.	Stop the work immediately and train the employee to perform the work safely\\
c.	Call OSHA immediately to investigate the incident. \\
d.	Give the employee a written warning that the work was performed unsafely\\

  \item Water treatment personnel should only use self-contained breathing apparatus equipment that has been approved by\\
a. the Occupational Safety and Health Administration (OSHA).\\
b. their state's Department of Public Health.\\
*c. the National Institute of Occupational Safety and Health (NIOSH).\\
d. the American Standards and Testing Methods.\\
  \item If a substantial chlorine leak incident occurs, which agency should be called for actual hands-on assistance?\\
a. The Occupational Safety and Health Administration (OSHA)\\
*b. The Chemical Transportation Emergency Center\\
c. The Transportation Emergency Institute\\
d. The Chlorine Institute\\
  \item In regards to safety, wet activated carbon will remove which from the air?\\
a. Organic gases and hydrogen sulfide\\
*b. Oxygen\\
c. Carbon dioxide\\
d. Carbon monoxide\\
  \item An employee's average airborne exposure in any 8-hour shift in a 40-hour workweek that should not be exceeded is called\\
a. Short Term Exposure Limit (STEL).\\
*b. Time-Weighted Average (TWA).\\
c. Threshold Limit Value (TLV).\\
d. Recommended Exposure Limits (REL).\\
  \item Sites are required to do a site assessment under the process safety management (PSM) regulations (OSHA) if the facility in a single process has more than how many pounds of chlorine?\\
a. $1,000 \mathrm{lb}$\\
*b. $1,500 \mathrm{lb}$\\
c. $2,000 \mathrm{lb}$\\
d. $4,000 \mathrm{lb}$\\

 \item Which statement concerning contact with chlorine is true?\\
a. If chlorine contacts the eyes, flush for 15 minutes with water, then neutralize with appropriate electrolytes that are safe for the eyes\\
b. Flush the eyes and give a sedative to the person that contacted chlorine, as it usually leads to excited behavior\\
c. Apply an appropriate ointment to the area of the skin that liquid chlorine came in contact with\\
*d. Chlorine inhalation may lead to delayed reactions such as pulmonary edema\\

  \item Which gas is commonly called swamp gas?\\
a. Hydrogen sulfide\\
*b. Methane\\
c. Carbon monoxide\\
d. Radon\\

  \item Self-contained breathing apparatus (SCBA) units\\
a. are very different than the units used by SCUBA divers.\\
*b. are fitted with a low-air-pressure alarm that sounds, alerting the wearer to leave the contaminated site.\\
c. should not be stored in storage rooms far away from the chlorine location, but two units should be stored in the chlorine feed room and two in the chlorine room.\\
*d. should have all straps rolled up so they can properly fit in their cases.\\

  \item In permit entry confined space, who is responsible for knowing the behavioral effects of exposure?\\
a. Authorized entrant, entry supervisor and the standby attendant\\
b. Entry supervisor and authorized attendant\\
*c. Authorized attendant\\
d. Standby attendant\\

  \item Which is the highest recommended stacking height for bags of powdered activated carbon and granular activated carbon?\\
a. 5 ft\\
*b. 6 ft\\
c. 8 ft\\
d. 10 ft\\

 \item Which national law authorizes the government to clean up contaminants from hazardous waste sites or contaminants caused by chemical spills that could possibly threaten the environment?\\
a. The Toxic Substances Control Act (TSCA)\\
*b. The Comprehensive Environmental Response, Compensation, and Liability Act (Superfund)\\
c. The Resource Conservation and Recovery Act (RCRA)\\
d. The Safe Drinking Water Act\\


  \item During a pumping test for a public water supply well, which must be held constant?\\
a. The pumping water level\\
b. The pump's amperage\\
c. The drawdown\\
*d. The pumping rate\\

  \item Which type of wells are commonly used near the shore of a lake or near a river?\\
a. Monitoring wells\\
b. Bedrock wells\\
c. Gravel wall wells\\
*d. Radial wells\\

  \item Chemical splash goggles (no face shield) are required PPE when working with potential exposure to 3 to 20 \% sodium hypochlorite solutions at temperatures below 100Deg. F only when\\
a. handling the material.\\
b. an initial line break occurs.\\
*c. inspecting the dome and no product is flowing.\\
d. loading that is remotely activated.\\

  \item In permit entry confined space, who is responsible for knowing the conditions within that confined space?\\
a. Standby attendant\\
*b. Entry supervisor\\
c. Authorized entrant\\
d. Authorized attendant\\

\item Introducing water into a strange tank containing ammonia vapors can cause\\
a. a rapid exothermic reaction.\\
b. a rapid endothermic reaction.\\
*c. the tank to collapse.\\
d. an explosion.\\

  \item The IDLH (Immediately Dangerous to Life and Health) represents the maximum concentration from which, if respiratory equipment failed, one could not escape within without a respirator and without experiencing any escape impairing or irreversible health effects.\\
a. 10 minutes\\
b. 20 minutes\\
*c. 30 minutes\\
d. 60 minutes\\

  \item Gaseous ammonia may be fatal when it reaches levels of\\
a. 400 to 700 ppm.\\
b. 800 to 1,100 ppm.\\
c. 1,100 to 1,700 ppm.\\
*d. 2,000 to 3,000 ppm.\\

  \item Which mixture may be produced if ammonia hydroxide is accidentally unloaded into a sodium hypochlorite tank?\\
a. Cyanide\\
b. Strychnine\\
*c. An explosive mixture of nitrogen trichloride\\
d. An explosive mixture of nitroglycerin\\

  \item Which national law regulates the storage, transportation, treatment, and disposal of solid and hazardous wastes?\\
a. The Toxic Substances Control Act (TSCA)\\
b. The Comprehensive Environmental Response, Compensation, and Liability Act (Superfund)\\
*c. The Resource Conservation and Recovery Act (RCRA)\\
d. The Safe Drinking Water Act\\

  \item Sites using chlorine in a single process are required by the US Environmental Protection Agency to write a risk management plan (RMP), if chlorine exceeds\\
a. 2,000 lb.\\
*b. $2,500 \mathrm{lb}$.\\
c. $4,000 \mathrm{lb}$.\\
d. $6,000 \mathrm{lb}$.\\


  \item Which plan is a mutual aid program for chlorine incidents that occur during transportation or at user locations?\\
*a. North American Chlorine Emergency Plan\\
b. Chemical Transportation Emergency Plan\\
c. Chlorine Institute Plan\\
d. Transportation Emergency Assistance Plan\\

\item What is the only type of self-contained breathing apparatus that should be used at water plants?\\
a.	Negative--pressure\\
b.	Zero-pressure\\
*c.	Positive-pressure\\
d.	Air-pressure\\

\end{enumerate}


