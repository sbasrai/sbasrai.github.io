\documentclass{article}
%\usepackage[english]{babel}%
\usepackage{graphicx}
\usepackage{tabulary}
\usepackage{tabularx}
\usepackage[table,xcdraw]{xcolor}
\usepackage{pdflscape}
\usepackage{lastpage}
\usepackage{multirow}
\usepackage{cancel}
\usepackage{amsmath}
\usepackage[table]{xcolor}
\usepackage{fixltx2e}
\usepackage[T1]{fontenc}
\usepackage[utf8]{inputenc}
\usepackage{ifthen}
\usepackage{fancyhdr}
\usepackage[document]{ragged2e}
\usepackage[margin=1in,top=1.2in,headheight=57pt,headsep=0.1in]
{geometry}
\usepackage{ifthen}
\usepackage{fancyhdr}
\everymath{\displaystyle}
\usepackage[document]{ragged2e}
\usepackage{fancyhdr}
\everymath{\displaystyle}
\linespread{2}%controls the spacing between lines. Bigger fractions means crowded lines%
%\pagestyle{fancy}
%\usepackage[margin=1 in, top=1in, includefoot]{geometry}
%\everymath{\displaystyle}
\linespread{1.3}%controls the spacing between lines. Bigger fractions means crowded lines%
%\pagestyle{fancy}
\pagestyle{fancy}
\setlength{\headheight}{56.2pt}


\chead{\ifthenelse{\value{page}=1}{\includegraphics[scale=0.3]{BassettCTCLogo}\\ \textbf \textbf Water Sources}}
\rhead{\ifthenelse{\value{page}=1}{Shabbir Basrai}{Shabbir Basrai}}
\lhead{\ifthenelse{\value{page}=1}{}{\textbf Water Sources Full Text}}


\cfoot{}
\lfoot{Page \thepage\ of \pageref{LastPage}}
\rfoot{}
\renewcommand{\headrulewidth}{2pt}
\renewcommand{\footrulewidth}{1pt}
\begin{document}
Source water refers to bodies of water (such as rivers, streams, lakes, reservoirs, springs, and ground water) that provide water to public drinking-water supplies and private wells. Water sources can include:

Surface water (for example, a lake, river, or reservoir)
Ground water (for example, an aquifer)
Recycled waterexternal icon (also called reused water)

The California State Water Project (SWP):
 is a multi-purpose water storage and delivery system that extends more than 705 miles -- two-thirds the length of California. A collection of canals, pipelines, reservoirs, and hydroelectric power facilities delivers clean water to 27 million Californians, 750,000 acres of farmland, and businesses throughout our state.\\

Planned, built, operated and maintained by DWR, the SWP is the nation’s largest state-owned water and power generator and user-financed water system. The project is considered an engineering marvel that has helped fuel California’s population boom and economic prosperity since its initial construction.\\

For the last 20 years, the California State Water Project’s average water is 34 percent for agricultural and 66 percent for residential, municipal, and industrial.\\

The State Water Project also plays an important role in efforts to combat climate change. Not only does it help California manage its water supply during extremes such as flooding and drought, it is also a major source of hydroelectric power deliveries for the State's power grid.\\

Benefits of the SWP:\\
The primary purpose of the SWP is water supply delivery and flood control, but it provides many additional benefits, including:\\

Power generation\\
Recreation activities\\
Environmental stewardship\\

DWR manages the California State Water Project (SWP) to ensure adequate water supplies are available under various hydrologic and legal conditions while maintaining operational flexibility. \\

DWR also develops, plans and implements the operation of the SWP in coordination with environmental and regulatory agencies to meet fish, water, and environmental requirements for the Feather River and Sacramento-San Joaquin Delta.\\

Additionally, the SWP coordinates closely with other water storage and water users that utilize the Sacramento-San Joaquin Delta watershed. The other agency that operates in a similar fashion to move water throughout California is the federal Bureau of Reclamation’s Central Valley Project (CVP).  \\

Learn more about DWR's operations and maintenance.\\

The SWP is operated in a manner that protects endangered and threatened species under the State and federal endangered species acts.\\

DWR does this in part through compliance with a permit granted by the California Department of Fish and Wildlife (DFW), called the Incidental Take Permit.\\

DWR also conducts water quality monitoring for the SWP. This program is currently managed by the Division of Operations and Maintenance,  Environmental Assessment Branch. Initially, this program sought to monitor eutrophication (an increase in chemical nutrients) and salinity in the SWP. Over time, the water quality program expanded to include parameters of concern for drinking water, recreation, and wildlife.\\

While the majority of the SWP was being constructed in the 1960s, public agencies and local water districts signed long-term water supply contracts with DWR. Today, the 29 public agencies and local water districts are collectively known as the SWP long-term water contractors or simply, SWP water contractors.\\

The water supply contracts (which expire in 2035) sets forth the maximum amount of SWP water a contractor may request annually (see Table A amounts, below). However, the amount of SWP water available for delivery will vary yearly, based on a number of factors, including:\\

Hydrologic conditions\\
Current reservoir storage\\
Delivery requests from the SWP water contractor\\
Learn more about State Water Project management.\\


A watershed is an area of land that contributes water to a given location, such as a reservoir, a
confluence of two streams, or the ocean. Within a watershed, water from rain or snow flows
down the slope, through the soil, or via groundwater flow – and usually by a combination of
these routes – to reach the stream and contribute to the flow of the stream. Watersheds are
important sources of drinking water, as well as a habitat for many aquatic species. Healthy
watersheds with intact native vegetation and wetlands provide important functions such as
water purification, flood control, nutrient recycling, and groundwater recharge. Such valuable
functions are sometimes referred to as “ecosystem services” (Revenga et al. 1998).\\
Californians rely on both surface and ground water sources for their domestic water supply.
Unfortunately, the watersheds that yield water to these sources face many sources of
degradation including sedimentation and pollution from residential and industrial
development, timber harvests, agricultural production, land clearing, and mining (Revenga et al.
1998, Bolund and Hunhammar 1999). While these types of degradation can affect both surface
and groundwater, this report focuses on surface drinking water sources and watersheds.
However, protecting and/or restoring native vegetation in these watersheds can also improve
groundwater supply by maintaining or increasing groundwater recharge rates.\\
Mapping the watersheds that supply drinking water to people is a crucial first step to ensure
they remain healthy and protected. Previous studies have listed sources or mapped a subset of
the watersheds (e.g., the watersheds for one city), but none of these explicitly mapped all
watersheds that supply drinking water to Californians1
. To fill this data gap, we have generated
the most comprehensive map to date of surface water sources and the watersheds that supply
80\% of Californian’s drinking water. In addition, we have analyzed the current land uses and
protections in these watersheds. Finally, we identified which watersheds supply drinking water
to 30 of the largest cities in the state. This document presents the results of this effort as well
as a description of the methods we used.\\

The watersheds that supply drinking water to 80\% of California’s resident cover almost 157
million acres and span 8 states (Figure 1). These watersheds drain lands that include highly
protected areas (e.g., wilderness areas) and those that have been developed (e.g., downtown
Sacramento). From the map, it is clear that the Sierra Nevada mountains are an important
source of water for the state of California, providing snowmelt for the many lakes and rivers
that drain into the Sacramento and San Joaquin rivers. These rivers, in turn, supply water to the
Sacramento-San Joaquin Delta, a water source that serves roughly 25 million Californians via
the State Water Project. In addition, most southern California cities obtain some of their
drinking water from the Colorado River, which originates in the mountains of Wyoming and
Colorado, and then passes through and drains portions of Utah, New Mexico, Arizona, and
Nevada until it reaches Lake Havasu, on the border between Arizona and California. There, it is
diverted into the Metropolitan Water District’s Colorado River Aqueduct which carries the
water 242 miles to southern California. \\
% Please add the following required packages to your document preamble:
% \usepackage[table,xcdraw]{xcolor}
% If you use beamer only pass "xcolor=table" option, i.e. \documentclass[xcolor=table]{beamer}
\begin{table}[]
\begin{tabular}{lll}
\multicolumn{3}{c}{{\color[HTML]{3166FF} \textbf{World Water Distribution}}}                                            \\
\multicolumn{2}{c}{{\color[HTML]{FD6864} Location}}                      & {\color[HTML]{FE996B} Percent (\%) of Total} \\
Land areas                    &                                          & 2.8                                          \\
                              & Freshwater lakes                         & 0.009                                        \\
                              & Saline lakes and inland seas             & 0.008                                        \\
                              & Rivers (average instantaneous volume)    & 0.0001                                       \\
                              & Soil moisture                            & 0.005                                        \\
                              & Groundwater (above depth of 4000 meters) & 0.61                                         \\
                              & Ice caps and glaciers                    & 2.14                                         \\
Atmosphere (water vapor)      &                                          & 0.001                                        \\
Oceans                        &                                          & 97.3                                         \\
Total all locations (rounded) &                                          & 100                                         
\end{tabular}
\end{table}
Eighty percent of the people of California rely on an area that is 1.5 times the size of the state
(157 million acres) to collect, filter, and deliver drinking water to their homes. In general, these
watersheds are relatively well protected and support natural vegetation. Two thirds of the
watersheds have some form of protection, while only 7% have been developed or converted to
agriculture. However, this analysis highlights the following significant threats to drinking water
quality and supply:
\begin{itemize}
\item Land management. Roughly half of the watersheds that supply drinking water are
public lands that are designated for multiple uses, including intensive logging and
mining. There are even fewer land use controls on private lands. Along the
Sacramento, San Joaquin, and Russian rivers, people live, work, and grow crops directly
adjacent to the rivers. If not managed correctly, these activities can pollute the drinking
water for millions of Californians, increasing treatment costs that are then passed on to
ratepayers.
\item Urban growth. Only 2% of the watersheds that supply drinking water are currently
developed for urban and suburban uses, but the population of California is expected to
increase by 37% by 2050, with 13.8 million new people needing places to live (Pitkin and
Myers 2012). Conversion of farmland and natural habitats to urban and suburban
growth will increase impervious surfaces and reduce the ability of soils and plants to
naturally filter the water.
\item Climate change. All of the recent General Circulation Models (GCMs) that project future
climate indicate significant increases in temperature, and many predict a drop in
average annual precipitation, for the watersheds that supply California’s drinking water
(Christensen et al. 2007). The precipitation projections are especially dry for the
Colorado River watershed and southern California (Seager et al. 2007, Seager and Vecchi
2010). In addition, recent studies indicate that a greater proportion of the rain that
does fall will come in big events, increasing the probability of floods and decreasing the
ability of water managers to store the water for drier times (Field et al. 2012).
\item Wildfires. As the climate warms and gets potentially drier, the area burned by wildfires
is predicted to increase (Westerling et al. 2011). These trends are already being
observed throughout the western U.S. (Westerling et al. 2006, Williams et al. 2010).
Intense wildfires can remove vegetation and increase sedimentation for years after the
fire (George et al. 2004). If the fire is large enough and close to drinking water diversion
infrastructure, this can cause significant damage by filling in reservoirs and clogging
filtration systems.
\end{itemize}

For example, forward-thinking and innovative management
practices, including the use of advanced rainwater capture systems, account
for at least 20\% of Singapore’s water supply, where the total annual precipitation
averages 2150 mm (84.6 in.), equivalent to 2150 L/m2 or 52.73 gal/ft2. Another 40\%
of Singapore’s water supply is imported from Malaysia. The use of gray water or
sullage (i.e., wastewater generated in homes and offices that is non-toilet water) adds
30\% to the total, with desalinization producing the remaining 10\% of the supply
necessary to meet the location’s total demand. Unfortunately, Singapore is the exception
today, not the trend. For this reason, among other contributing factors, major
portions of the globe remain either uninhabited or sparsely populated.\\

The other key concern (and the main focus of this text) is water quality. Obviously,
having a sufficient quantity of freshwater available does little good if the water is
unsafe for consumption or for other uses. There is another issue; namely, it is rather
easy to determine the quantity of a substance such as water. We can say there is
too little, enough, or too much. A quantity is a metric that indicates or signifies a
number. In the case of water, quantity can be expressed as the number of gallons or
acre-feet of water available or not available. Trying to quantify the quality of water
is an entirely different matter, though. Quality is a characteristic that often is a judgment
call, but the problem with judging quality is that it can be subjective.\\

The bottom line: Obviously, having a sufficient quantity of freshwater available
does little good if the water is unsafe for consumption or for other uses.\\



\end{document}