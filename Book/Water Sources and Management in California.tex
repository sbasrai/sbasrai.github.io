\documentclass{article}
%\usepackage[english]{babel}%
\usepackage{graphicx}
\usepackage{tabulary}
\usepackage{tabularx}
\usepackage[table,xcdraw]{xcolor}
\usepackage{pdflscape}
\usepackage{lastpage}
\usepackage{multirow}
\usepackage{cancel}
\usepackage{amsmath}
\usepackage[table]{xcolor}
\usepackage{fixltx2e}
\usepackage[T1]{fontenc}
\usepackage[utf8]{inputenc}
\usepackage{ifthen}
\usepackage{fancyhdr}
\usepackage[document]{ragged2e}
\usepackage[margin=1in,top=1.2in,headheight=57pt,headsep=0.1in]
{geometry}
\usepackage{ifthen}
\usepackage{fancyhdr}
\everymath{\displaystyle}
\usepackage[document]{ragged2e}
\usepackage{fancyhdr}
\everymath{\displaystyle}
\linespread{2}%controls the spacing between lines. Bigger fractions means crowded lines%
%\pagestyle{fancy}
%\usepackage[margin=1 in, top=1in, includefoot]{geometry}
%\everymath{\displaystyle}
\linespread{1.3}%controls the spacing between lines. Bigger fractions means crowded lines%
%\pagestyle{fancy}
\pagestyle{fancy}
\setlength{\headheight}{56.2pt}


\chead{\ifthenelse{\value{page}=1}{\includegraphics[scale=0.3]{BassettCTCLogo}\\ \textbf \textbf Water Sources}}
\rhead{\ifthenelse{\value{page}=1}{Shabbir Basrai}{Shabbir Basrai}}
\lhead{\ifthenelse{\value{page}=1}{}{\textbf Water Sources and Management in California}}


\cfoot{}
\lfoot{Page \thepage\ of \pageref{LastPage}}
\rfoot{Module 8}
\renewcommand{\headrulewidth}{2pt}
\renewcommand{\footrulewidth}{1pt}
\begin{document}

California's interconnected water system:
\begin{itemize}
\item Serves over 30 million people
\item Irrigates over 5,680,000 acres (2,300,000 ha) of farmland.
\item Manages over 40 million acre-feet (49 km$_3$) of water per year.
\end{itemize}

While parts of Northern California receive 100 inches or more of precipitation per year, the state’s southern, drier areas receive less precipitation – and just a few inches of rain annually in the desert regions. That means 75 percent of California’s available water is in the northern third of the state (north of Sacramento), while 80 percent of the urban and agricultural water demands are in the southern two-thirds of the state.

California's limited water supply comes from two main sources: 
\begin{enumerate}
\item Surface water, or water that travels or gathers on the ground, like rivers, streams, and lakes
\begin{itemize}
\item The freshwater is usually found in deposits of gravel, silt, and sand. Below these deposits lies a layer of deep sediment, a relic of the era when the Pacific Ocean covered the area.
\item California has ten major drainage basins defined for convenience of water management. 
\item These basins are divided from one another by the crests of mountains. From north to south the basins are: North Coast, Sacramento River, North Lahontan, San Francisco Bay, San Joaquin River, Central Coast, Tulare Lake, South Lahontan, South Coast, and Colorado River regions.
\item Many of the drainage basins are extremely altered, with hydroelectric power generation happening in much of the upper portion of these watersheds.
\item The Central Valley watershed, which incorporates the Sacramento River, San Joaquin River and Tulare Lake regions:
\begin{itemize}
\item Is the largest in California, draining over a third of the state – 60,000 square miles (160,000 km2) – 
\item Produces nearly half the total runoff.
\item The Sierra Nevada snowpack feeds Central Valley river systems and is a critical source of water in the state's long dry season when little if any precipitation falls. 
\item Up to 30 percent of California's water supply is from snowpack, and the majority of California's hydroelectricity is also generated from the Sierra Nevada snowpack.
\item Much of California's extensive reservoir and aqueduct system is designed to store and capture runoff from the Central Valley watershed. As this infrastructure ages, dam removal in California has become more widespread--a process that has been largely successful.[3] The Sacramento and San Joaquin Rivers converge at the Sacramento–San Joaquin River Delta, a large fresh-water estuary where much of the state's water supply is withdrawn. The Central Valley watershed provides most of the water for Northern and Central California, as well as a significant chunk of Southern California's usage.
\item In February 2022 issued a draft environmental impact statement saying there were significant benefits to a plan to demolish four massive dams on Northern California’s Klamath River to save imperiled migratory salmon, setting the stage for the largest dam demolition project in U.S. history.
\item The aging dams near the Oregon-California border were built before current environmental regulations and essentially cut the 253-mile-long (407-kilometer-long) river in half for migrating salmon, whose numbers have plummeted. The project on California’s second-largest river would be at the vanguard of a push to demolish dams in the U.S. as the structures age and become less economically viable and as concerns grow about their environmental impact, particularly on fish.
\item The North Coast watershed receives the highest annual precipitation of any California watershed. It incorporates many large river systems such as the Klamath, Smith, Trinity, and Eel, and produces over a third of the runoff in the state. With the notable exceptions of the Trinity Dam complex that transfers water from the Trinity River into the Sacramento River and Scott Dam that transfers water from the Eel River into the Russian River, most of the North Coast watersheds are relatively undeveloped, some have federal Wild and Scenic status that protect them from development; the northern coastal rivers provide water for salmonid habitat, carbon-sequestering forests, and local communities; some are within the influence of tribal water and fishing rights. Water flowing in these watersheds and into the Pacific Ocean is critical for sensitive, threatened, and endangered salmonids. There have been proposals to create additional inter-basin transfers from North Coast rivers to increase water supplies in the rest of California, but these projects have been rejected due to presumed environmental harm.\\

\item The Colorado River originates more than 1,000 miles (1,600 km) from California in the Rocky Mountains of Colorado and Wyoming and forms the state's southeastern border in the Mojave Desert. Unlike the other California watersheds, essentially all of the water flowing in the Colorado originates outside the state. The Colorado is a critical source of irrigation and urban water for southern California, providing between 55 and 65 percent of the total supply.\\

\item The Central and South Coast watersheds include the most populous regions of California – the San Francisco Bay Area, Los Angeles and San Diego – but have relatively little natural runoff, requiring the importation of water from other parts of the state.\\

\item Rivers of the Lahontan watersheds in eastern California are part of the high desert Great Basin and do not drain to the Pacific. Most of the water is used locally in eastern California and western Nevada for irrigation. The Owens River of the South Lahontan region, however, is a principal source of water for Los Angeles.\\
\end{itemize}
\end{itemize}
\item Groundwater:
\begin{itemize}
\item It is the water is pumped out from the ground from its 450 known groundwater reservoirs which holds over 850,000,000 acre-feet (1,050 km3) of water.
\item The largest groundwater reservoirs are found in the Central Valley
\item Over half of the groundwater is unavailable due to poor quality and the high cost of pumping the water from the ground.
\item During a normal year, 30\% of the state's water supply comes from groundwater (underground water). In times of intense drought, groundwater consumption can rise to 60\% or more.
\item In 2014, the Sustainable Groundwater Management Act was introduced to regulate usage of groundwater sources statewide. This legislation regulates management of groundwater through local agencies in their own respective groundwater basin regions.
\item Groundwater sustainability agencies are created by the legislation, and they are required to develop groundwater sustainability plans that control overdraft and recharge.
\item California's groundwater is also overseen by the Groundwater Ambient Monitoring and Assessment Program, which evaluates groundwater quality and contamination across the state under the direction of the California State Water Resources Control Board.[7]
\end{itemize}
\end{enumerate}

Surface water is concentrated mostly in the northern part of the state, groundwater is more evenly distributed.

Water and water rights are among the state's divisive political issues. Due to the lack of reliable dry season rainfall, water is limited in the most populous U.S. state. An ongoing debate is whether the state should increase the redistribution of water to its large agricultural and urban sectors, or increase conservation and preserve the natural ecosystems of the water sources.\\




California has also begun producing a small amount of desalinated water, water that was once sea water, but has been purified.\\

Groundwater is a critical element of the California water supply. 

The largest groundwater reservoirs are found in the Central Valley.[4] The majority of the supply there is in the form of runoff that seeps into the aquifer. 



The large quantity of water beneath the surface has given rise to the misconception that groundwater is a sort of renewable resource that can be limitlessly tapped. Calculations assuming that groundwater usage is sustainable if the rate of removal equals the rate of recharge are often incorrect as a result of ignoring changes in water consumption and water renewal.[8]

While the volume of groundwater in California is very large, aquifers can be over drafted when groundwater is removed more rapidly than it is replenished. In 1999, it was estimated that the average, annual overdrafting was around 2,200,000 acre-feet (2.7 km3) across the state, with 800,000 acre-feet (0.99 km3) in the Central Valley.[9][page needed] Since then, overdrafting had significantly increased. Satellite measurements found that in just the combined Sacramento and San Joaquin River basins, including the Central Valley, overdrafting between 2011 and 2014 was 12,000,000 acre-feet (15 km3) of water per year.\\

Major rivers of California
 Each region incorporates watersheds from many rivers of similar clime. \\





The California State Water Project (SWP):
\begin{itemize} 
\item It is a multi-purpose water storage and delivery system that extends more than 705 miles -- two-thirds the length of California. 
\item A collection of canals, pipelines, reservoirs, and hydroelectric power facilities delivers clean water to 27 million Californians, 750,000 acres of farmland, and businesses throughout our state.
\item Planned, built, operated and maintained by DWR, the SWP is the nation’s largest state-owned water and power generator and user-financed water system.
\item For the last 20 years, the California State Water Project’s average water is 34 percent for agricultural and 66 percent for residential, municipal, and industrial.\\
\item The State Water Project also plays an important role in efforts to combat climate change. Not only does it help California manage its water supply during extremes such as flooding and drought, it is also a major source of hydroelectric power deliveries for the State's power grid.\\

Benefits of the SWP:\\
The primary purpose of the SWP is water supply delivery and flood control, but it provides many additional benefits, including:\\

Power generation\\
Recreation activities\\
Environmental stewardship\\

DWR manages the California State Water Project (SWP) to ensure adequate water supplies are available under various hydrologic and legal conditions while maintaining operational flexibility. \\

DWR also develops, plans and implements the operation of the SWP in coordination with environmental and regulatory agencies to meet fish, water, and environmental requirements for the Feather River and Sacramento-San Joaquin Delta.\\

Additionally, the SWP coordinates closely with other water storage and water users that utilize the Sacramento-San Joaquin Delta watershed. The other agency that operates in a similar fashion to move water throughout California is the federal Bureau of Reclamation’s Central Valley Project (CVP).\\

Learn more about DWR's operations and maintenance.\\

The SWP is operated in a manner that protects endangered and threatened species under the State and federal endangered species acts.\\

DWR does this in part through compliance with a permit granted by the California Department of Fish and Wildlife (DFW), called the Incidental Take Permit.\\

DWR also conducts water quality monitoring for the SWP. This program is currently managed by the Division of Operations and Maintenance,  Environmental Assessment Branch. Initially, this program sought to monitor eutrophication (an increase in chemical nutrients) and salinity in the SWP. Over time, the water quality program expanded to include parameters of concern for drinking water, recreation, and wildlife.\\

While the majority of the SWP was being constructed in the 1960s, public agencies and local water districts signed long-term water supply contracts with DWR. Today, the 29 public agencies and local water districts are collectively known as the SWP long-term water contractors or simply, SWP water contractors.\\

\item The water supply contracts (which expire in 2035) sets forth the maximum amount of SWP water a contractor may request annually.
However, the amount of SWP water available for delivery will vary yearly, based on a number of factors, including:\\
\end{itemize}
A watershed is an area of land that contributes water to a given location, such as a reservoir, a
confluence of two streams, or the ocean. Within a watershed, water from rain or snow flows
down the slope, through the soil, or via groundwater flow – and usually by a combination of
these routes – to reach the stream and contribute to the flow of the stream. Watersheds are
important sources of drinking water, as well as a habitat for many aquatic species. Healthy
watersheds with intact native vegetation and wetlands provide important functions such as
water purification, flood control, nutrient recycling, and groundwater recharge. Such valuable
functions are sometimes referred to as “ecosystem services” (Revenga et al. 1998).\\
Californians rely on both surface and ground water sources for their domestic water supply.
Unfortunately, the watersheds that yield water to these sources face many sources of
degradation including sedimentation and pollution from residential and industrial
development, timber harvests, agricultural production, land clearing, and mining (Revenga et al.
1998, Bolund and Hunhammar 1999). While these types of degradation can affect both surface
and groundwater, this report focuses on surface drinking water sources and watersheds.
However, protecting and/or restoring native vegetation in these watersheds can also improve
groundwater supply by maintaining or increasing groundwater recharge rates.\\
Mapping the watersheds that supply drinking water to people is a crucial first step to ensure
they remain healthy and protected. Previous studies have listed sources or mapped a subset of
the watersheds (e.g., the watersheds for one city), but none of these explicitly mapped all
watersheds that supply drinking water to Californians1
. To fill this data gap, we have generated
the most comprehensive map to date of surface water sources and the watersheds that supply
80\% of Californian’s drinking water. In addition, we have analyzed the current land uses and
protections in these watersheds. Finally, we identified which watersheds supply drinking water
to 30 of the largest cities in the state. This document presents the results of this effort as well
as a description of the methods we used.\\

The watersheds that supply drinking water to 80\% of California’s resident cover almost 157 million acres and span 8 states (Figure 1). These watersheds drain lands that include highly protected areas (e.g., wilderness areas) and those that have been developed (e.g., downtown Sacramento). From the map, it is clear that the Sierra Nevada mountains are an important source of water for the state of California, providing snowmelt for the many lakes and rivers that drain into the Sacramento and San Joaquin rivers. These rivers, in turn, supply water to the Sacramento-San Joaquin Delta, a water source that serves roughly 25 million Californians via the State Water Project. In addition, most southern California cities obtain some of their drinking water from the Colorado River, which originates in the mountains of Wyoming and Colorado, and then passes through and drains portions of Utah, New Mexico, Arizona, and Nevada until it reaches Lake Havasu, on the border between Arizona and California. There, it is diverted into the Metropolitan Water District’s Colorado River Aqueduct which carries the
water 242 miles to southern California. \\

\begin{table}[]
\begin{tabular}{lll}
\multicolumn{3}{c}{{\color[HTML]{3166FF} \textbf{World Water Distribution}}}                                            \\
\multicolumn{2}{c}{{\color[HTML]{FD6864} Location}}                      & {\color[HTML]{FE996B} Percent (\%) of Total} \\
Land areas                    &                                          & 2.8                                          \\
                              & Freshwater lakes                         & 0.009                                        \\
                              & Saline lakes and inland seas             & 0.008                                        \\
                              & Rivers (average instantaneous volume)    & 0.0001                                       \\
                              & Soil moisture                            & 0.005                                        \\
                              & Groundwater (above depth of 4000 meters) & 0.61                                         \\
                              & Ice caps and glaciers                    & 2.14                                         \\
Atmosphere (water vapor)      &                                          & 0.001                                        \\
Oceans                        &                                          & 97.3                                         \\
Total all locations (rounded) &                                          & 100                                         
\end{tabular}
\end{table}
Eighty percent of the people of California rely on an area that is 1.5 times the size of the state (157 million acres) to collect, filter, and deliver drinking water to their homes. In general, these watersheds are relatively well protected and support natural vegetation. Two thirds of the watersheds have some form of protection, while only 7\% have been developed or converted to agriculture. However, this analysis highlights the following significant threats to drinking water quality and supply:
\begin{itemize}
\item Land management. Roughly half of the watersheds that supply drinking water are public lands that are designated for multiple uses, including intensive logging and mining. There are even fewer land use controls on private lands. Along the Sacramento, San Joaquin, and Russian rivers, people live, work, and grow crops directly adjacent to the rivers. If not managed correctly, these activities can pollute the drinking water for millions of Californians, increasing treatment costs that are then passed on to ratepayers.
\item Urban growth. Only 2\% of the watersheds that supply drinking water are currently developed for urban and suburban uses, but the population of California is expected to increase by 37\% by 2050, with 13.8 million new people needing places to live (Pitkin and Myers 2012). Conversion of farmland and natural habitats to urban and suburban growth will increase impervious surfaces and reduce the ability of soils and plants to naturally filter the water.  
\item Climate change. All of the recent General Circulation Models (GCMs) that project future climate indicate significant increases in temperature, and many predict a drop in average annual precipitation, for the watersheds that supply California’s drinking water (Christensen et al. 2007). The precipitation projections are especially dry for the Colorado River watershed and southern California (Seager et al. 2007, Seager and Vecchi 2010). In addition, recent studies indicate that a greater proportion of the rain that does fall will come in big events, increasing the probability of floods and decreasing the ability of water managers to store the water for drier times (Field et al. 2012).
\item Wildfires. As the climate warms and gets potentially drier, the area burned by wildfires is predicted to increase (Westerling et al. 2011). These trends are already being observed throughout the western U.S. (Westerling et al. 2006, Williams et al. 2010).  Intense wildfires can remove vegetation and increase sedimentation for years after the fire (George et al. 2004). If the fire is large enough and close to drinking water diversion infrastructure, this can cause significant damage by filling in reservoirs and clogging filtration systems.
\end{itemize}

For example, forward-thinking and innovative management practices, including the use of advanced rainwater capture systems, account for at least 20\% of Singapore’s water supply, where the total annual precipitation averages 2150 mm (84.6 in.), equivalent to 2150 L/m2 or 52.73 gal/ft2. Another 40\% of Singapore’s water supply is imported from Malaysia. The use of gray water or sullage (i.e., wastewater generated in homes and offices that is non-toilet water) adds 30\% to the total, with desalinization producing the remaining 10\% of the supply necessary to meet the location’s total demand. Unfortunately, Singapore is the exception today, not the trend. For this reason, among other contributing factors, major portions of the globe remain either uninhabited or sparsely populated.\\

The other key concern (and the main focus of this text) is water quality. Obviously,
having a sufficient quantity of freshwater available does little good if the water is
unsafe for consumption or for other uses. There is another issue; namely, it is rather
easy to determine the quantity of a substance such as water. We can say there is
too little, enough, or too much. A quantity is a metric that indicates or signifies a
number. In the case of water, quantity can be expressed as the number of gallons or
acre-feet of water available or not available. Trying to quantify the quality of water
is an entirely different matter, though. Quality is a characteristic that often is a judgment
call, but the problem with judging quality is that it can be subjective.\\

The bottom line: Obviously, having a sufficient quantity of freshwater available
does little good if the water is unsafe for consumption or for other uses.\\

Most of the precipitation falls in the winter as rain and snow. Although the climate is variable, the state receives about 200 million acre-feet of precipitation per year on average. An acre-foot of water equals about 326,000 gallons, or enough water to cover an acre of land (the size of a football field) 1 foot deep. One acre-foot is enough water to meet the annual indoor and outdoor needs of two households.\\

About 60 percent of all precipitation evaporates or is transpired by trees and vegetation. What’s left is roughly 75 million acre-feet per average year that flows into waterways and groundwater aquifers and ultimately becomes available to use in homes, as irrigation for farmland, by industry and in the environment.\\

There’s a catch. \\

Despite the geographic and hydrologic challenges, California has more irrigated acreage than any other state, thanks to massive water projects that include dams, reservoirs, aqueducts and canals to deliver water to users, especially in the central and southern portions of the state. Water also is moved east to west such as through San Francisco’s Hetch Hetchy system.\\

Water fuels the economy of California, and managing it properly is of paramount importance.\\

The resource also has been a source of decades-long political wars. Besides satisfying the needs of a growing population, demands for more water also come from the agricultural industry, businesses, manufacturers and developers. These needs must be balanced against demands for protecting water quality and for protecting fisheries, wildlife and recreational interests.\\

The fundamental controversy is one of distribution, as conflicts between these competing interests continue to be exacerbated by continued population growth and periods of drought.\\

Everything depends on the development and management of water: Capturing it behind dams, storing it in reservoirs, and rerouting it in canals stretching hundreds of miles across the state. California has 1,400 dams, two of the largest water storage and transport systems in the world – the Central Valley Project (CVP) and State Water Project (SWP) – and some of the largest reservoirs in the country.\\

This post gives an overview of the organization of water utilities in California. For this purpose, water utilities are defined as entities that own and operate public drinking water systems at the retail level. There are few sources of information about who provides water utility service in the state, and this post offers only an overview of the topic. The data used for the analysis in this post was derived from a proprietary database I have of all water utilities in California.\\

Water utilities in California may be organized as one of six general types: cities, county districts, special districts, public utilities, mutual water companies or mobile home parks. The distribution of water systems by number and population served are shown in the figures below. As you can see, cities, special districts and public utilities dominate in terms of the population served, but there are large numbers of mutual water companies and mobile home parks. That occurs because most mutuals and mobile home parks are relatively small.\\

Cities.  There are 482 incorporated cities in California, and 285 of those (roughly 60 percent) own and operate water utilities. Not surprisingly, the largest city water system is owned by Los Angeles and serves approximately 4.1 million people. Other large systems are owned by San Diego (1.27 million people) and San Francisco (800,000 people), which operates its utility through the San Francisco Public Utilities Commission. Together, cities provide water service to almost 20 million Californians, slightly more than half of all state residents.\\

County Districts.  There are 129 county districts in California. These may take several different forms, such as county service areas, county water works districts and county maintenance districts, but all are owned and operated by county governments. The districts are operated as separate enterprises, although many receive financial assistance from the county general fund due to their small, uneconomical size. These districts tend to lie in rural areas, with the majority serving fewer than 1,000 people.\\

Special Districts.  California has 537 special districts that own and operate public water systems, serving a total of 10.6 million people. That makes special districts the second-largest category of water providers in the state, behind cities. Special districts may be of many different types, with each type governed by its own statute. In addition, there are a number of special act districts, each of which was formed directly by the Legislature pursuant to a statute that applies only to that district. Special districts are governed by an independent board elected by registered voters within the district (with the exception of California water districts, which hold landowner elections). Some special districts provide multiple public services, while many focus on water service alone. The number and population served by special districts are shown in the figures below.  Districts included in the “Other” category include resort improvement districts, resource conservation districts, sanitary districts, water conservation districts, water storage districts and special act districts.\\

Public Utilities.  There are 138 public utilities in California, including eight Class A utilities with more than 10,000 customers. Public utilities are organized as private corporations, with their stock owned by shareholders for investment purposes. Of the 138 companies operating in California, four are publicly traded, two are owned by investment funds, and the remainder are closely-held businesses. Public utilities are subject to comprehensive regulation by the California Public Utilities Commission regarding water supplies, capital improvements, service quality and water rates.\\

Many of the larger public utilities own more than one water system, so that the 138 companies own a total of 255 separate systems. Together, public utilities serve almost 5.5 million people in California, 15 percent of the state’s population. The largest public utility is California Water Service Company, which serves approximately 1.8 million people through 48 systems.\\

Mutual Water Companies.  Mutual water companies are private organizations owned and controlled by their customers. They are often formed in connection with real estate subdivisions, and some operate through the organization of a homeowners’ association. I have written extensively about mutual water companies in prior posts. In California, there are approximately 1,200 mutual water companies, of which 1,100 operate as water-specific organizations and 100 perform multiple functions as homeowners’ associations. While most mutual water companies are small, some provide service to more than 10,000 people. Combined, mutual water companies serve approximately 1.3 million residents in California.\\

Mobile Home Parks.  Many mobile home parks provide water service to their tenants, since municipal and private utility companies have difficulty in serving individual mobile homes. Some mobile home parks purchase water from larger utilities for resale to their tenants, while others own and operate surface water diversions or groundwater wells. Currently, there are 467 mobile home parks operating public water systems in California, serving a combined population of 75,000. There are special rules governing mobile home parks, and they are subject to limited oversight by the California Public Utilities Commission.\\


The Metropolitan Water District of Southern California is a regional wholesaler and the largest supplier of treated water in the United States. The name is usually shortened to "Met," "Metropolitan," or "MWD." It is a cooperative of fourteen cities, eleven municipal water districts, and one county water authority, that provides water to 19 million people in a 5,200-square-mile (13,000 km2) service area. It was created by an act of the California Legislature in 1928, primarily to build and operate the Colorado River Aqueduct. Metropolitan became the first (and largest) contractor to the State Water Project in 1960.\\

Metropolitan owns and operates an extensive range of capital facilities including the Colorado River Aqueduct which runs from an intake at Lake Havasu on the California-Arizona border to its endpoint at the Lake Mathews reservoir in Riverside County. It also imports water supplies from northern California via the 444-mile (715 km) California Aqueduct as a contractor to the State Water Project. In 1960, Metropolitan became the first (and largest) contractor to the State Water Project. Metropolitan's extensive water system includes three major reservoirs, six smaller reservoirs, 830 miles (1,340 km) of large-scale pipes, about 400 connections to member agencies, 16 hydroelectric facilities and five water treatment plants.\\

It serves parts of Los Angeles, Orange, San Diego, Riverside, San Bernardino and Ventura counties. The district covers the coastal and most heavily populated portions of Southern California; however large portions of San Diego, San Bernardino and Riverside counties are located outside of its service area.

The Metropolitan headquarters is in downtown Los Angeles, adjacent to historic Union Station.\\

History


In the early 20th century, Southern California cities were faced with a growing population and shrinking local groundwater supplies. The Metropolitan Water District of Southern California was established in 1928 under an act of the California Legislature to build and operate the 242-mile Colorado River Aqueduct (389 km) that would bring water to southern coastal areas. Southland residents voted for a major bond in the depths of the Great Depression to fund the construction effort through the desert to deliver essential water supplies and generate badly needed jobs.\\

The post-World War II boom and 1950s dry spells prompted a huge expansion of the Metropolitan service area as new cities began seeking additional reliable water supplies.\\

In 1960, Metropolitan, along with 30 other public agencies, signed a long-term contract that made possible the construction of the State Water Project, including reservoirs, pumping plants and the 444-mile California Aqueduct (715 km), which serves urban and agricultural agencies from the San Francisco Bay to Southern California. As the largest of the now 29 agencies, Metropolitan contracts with the state Department of Water Resources, which owns and operates the State Water Project, for slightly less than half of all supplies delivered to Metropolitan.\\

Metropolitan is governed by a board of 38 directors whose powers and functions are specified in the 1927 authorization act.[1] This board was in charge of issuing bonds and financing their repayment by selling water to member agencies. In the early years, revenue from water sales was too low, so Metropolitan also collected taxes that ranged from 0.25 to 0.50 percent of assessed value. Ninety percent of the cost of the aqueduct has been paid for by the taxpayers. In 1929 the district was set up with an area of 600 square miles (1,600 km2) and served a population of around 1,600,000 in 13 cities.\\

During the aqueduct's first five years of service from 1941 to 1946 it delivered an average of about 27,000 acre-feet (33,000,000 m3) of water, using less than 2\% of its capacity. Only one pump at each lift, operating from one to six months out of the year, was needed to meet all the demands made on the system. At this time, due to availability of ground water, less than 10\% of the Colorado River Aqueduct's capacity was used, only 178,000 acre-feet (220,000,000 m3) of water.\\

The San Diego County Water Authority joined Metropolitan as its first wholesale member agency in 1946. SDCWA was formed in 1944 to facilitate joining Metropolitan, received its first deliveries in 1947 and was buying half of Metropolitan's water by 1949. The SDCWA annexation broke two traditions at Metropolitan: Member agencies had previously been cities (SDCWA was a water wholesaler) in the south coast basin (SDCWA was south of the basin. 
The next break came in 1950, when Pomona MWD (now Three Valleys MWD)joined Metropolitan. Since Pomona was a largely agricultural member agency at the time, Metropolitan was no longer selling water only for domestic use.
The territory served by Pomona district has urbanized rapidly, with agriculture having disappeared almost entirely by 1970.\\

In 1952, Metropolitan began a 200 million dollar program to bring the Colorado River Aqueduct to its full capacity of 1,212,000 acre-feet (1.495×109 m3) annually. The Colorado River Aqueduct added six pumps to the original three at each of its five pumping stations. CRA pumping expanded from about 16,500 acre-feet (20,400,000 m3) of water in 1950 to about 1,029,000 acre-feet (1.269×109 m3) by 1960. On August 9, 1962, the Metropolitan set an all-time delivery record of 1,316,000,000 gallons of water in just a 24-hour period.\\

Metropolitan's additional supplies and easier rules of entry facilitated an expansion through annexation of large areas of low populations: The eight MWDs that joined from 1946 to 1955 added 200 percent to Metropolitan's service area but only 75 percent to Metropolitan's population served.  By 1965, Metropolitan had 13 cities and 13 municipal water districts as members. It covered more than 4,500 square miles (12,000 km2) and served some 10,000,000 people.\\

By 2008 Metropolitan had 14 cities and 12 municipal water districts (San Fernando joined in 1973; MWDOC and Coastal MWD merged in 2001) and provided water to nearly 10,000,000 people.[5] As of 2021, Metropolitan with 26 member agencies and cities served nearly 19 million people in the counties of Los Angeles, Ventura, Orange, San Diego, Riverside, and San Bernardino.\\

Colorado River Drought Contingency Plan\\
In 2019 the Metropolitan Water District played a crucial role in the development of the Colorado River Drought Contingency Plan (DCP). The Drought Contingency Plan aims to implement legislation to reduce the risk of declining levels in the Colorado River reservoirs, particularly by incentivizing agencies to store additional water in Lake Powell and Lake Mead.[6] In 2018, the Imperial Irrigation District elected to not execute the DCP and the Metropolitan Water District agreed to provide the full portion of water storage contributions to Lake Mead.[7] By the end of 2020, MWD will have nearly stored 1 million acre-feet in Lake Mead and contributing to 12 feet (3.7 m) of Lake Mead's elevation.\\

Water sources\\

Colorado River Aqueduct\\
The State Water Project moves water from the western Sierra Nevada through the Sacramento-San Joaquin River Delta before delivering supplies—via the California Aqueduct to Southern California. Once in the south coastal plain, deliveries are split between the SWP's West Branch, storing water in Castaic Lake for delivery to the west side of the Los Angeles metropolitan area, and the East Branch, which delivers water to the Inland Empire and the south and east parts of the Los Angeles Basin.\\

The Colorado River Aqueduct begins at Lake Havasu, just north of Parker Dam, and travels 242 miles (389 km) west to Lake Mathews in southwest Riverside County. Water is first pumped 125 miles (201 km) uphill through a series of five pumping plants approaching Chiriaco Summit, then flows 117 miles (188 km) downhill towards Los Angeles.\\

Metropolitan contracts for about 2 MAF/Y (million acre feet per year) from the State Water Project (SWP) and 1.35 MAF/Y from the Colorado River Aqueduct (CRA), but actual delivery amounts depend on a conditions including hydrology, infrastructure and regulatory conditions . Between 1984 and 2004 the actual deliveries were 0.7 MAF/Y from the SWP and 1.2 MAF/Y from the CRA. The SWP allotment is rarely met, if at all, due to restrictions on the amount of water that can be pumped from the Delta. A minimum freshwater flow has to pass through the Delta in order to prevent salinity intrusion from San Francisco Bay, and the removal of freshwater from the Delta has also threatened multiple species, such as native chinook salmon.\\

The Inland Feeder project[9] added a direct tunnel and pipeline connection from Silverwood Lake to Diamond Valley Lake and was completed in 2010.\\

Reservoirs
The Metropolitan Water District of Southern California reservoirs store fresh water for use in Los Angeles, Orange, Ventura, Riverside, San Bernardino, and San Diego counties. These reservoirs were built specifically to preserve water during times of drought, and are in place for emergencies uses such as earthquake, floods or other events.\\

Metropolitan maintains three major water reservoirs. One is Lake Mathews located in southwest Riverside, California, with a capacity of 182,000 acre-feet (224,000,000 m3) of water. Another is Lake Skinner located south of Hemet in Riverside County, its capacity is 44,000 acre-feet (54,000,000 m3) of water. Diamond Valley Lake is their third and newest reservoir, with a capacity of 810,000 acre-feet (1.00×109 m3) of water. This capacity is over twice as large as that of Castaic Lake, the next largest reservoir in Southern California maintained by the state Department of Water Resources.\\

Metropolitan partly funded the Brock Reservoir project with \$28.6 million. In return for their contribution, California can each use 100,000 acre-feet (120,000,000 m3) of water starting in 2016.\\

Purification and treatment\\

Interior of zeolite building at F. E. Weymouth plant\\
Metropolitan operates five treatment plants:\\

Robert B. Diemer Treatment Plant[10] in Yorba Linda[11] began operation in 1963 and has a treatment capacity of 520 million gallons a day\\
Joseph Jensen Treatment Plant[12] in Granada Hills started operation in 1972 and is believed to be the largest treatment plant west of the Mississippi River with a treatment capacity of 750 million gallons a day
Henry J. Mills Treatment Plant[13] in Riverside became operational in 1978 and has a treatment capacity of 220 million gallons a day\\
Robert A. Skinner Treatment Plant[14] in Winchester started operation in 1976 and Metropolitan with a treatment capacity of 630 million gallons a day\\
F. E. Weymouth Treatment Plant,[15] a 58,800-square-foot (5,460 m2) facility in La Verne began operation in 1941 and has a treatment capacity of 520 million gallons a day\\
They collectively filter water for more than 19 million Southern Californians.[15] Metropolitan employs over 2,100 people to maintain and do research at these facilities,[citation needed] including scientists specializing in chemistry, microbiology, and limnology (the study of lakes and rivers).\\

Metropolitan's water treatment plants each use a conventional 5-step treatment process as follows:[16]

Disinfection/Pre-Treatment: Water entering the plant is disinfected using ozone as the primary disinfectant. Chlorine is used as an ozone back-up disinfectant.
Coagulation: Chemical coagulants (either alum or ferric chloride and polymer) are injected into the water and mixed with flash jet mixers.
Flocculation: The water travels into the mixing and settling basins, where large mechanical mixers (flocculators) gently agitate the water. This further mixes the water with the coagulant chemicals and allows sufficient time for the larger suspended particles in the water to bind together and form “floc.”
Sedimentation: The floc particles, which are heavier than the surrounding water, settle to the bottom of the basin, forming a layer of material that is later removed.
Filtration: Settled water from the sedimentation basins is treated with a filter aid polymer and enters the filters, which consist of layers of anthracite coal and sand filter media. The filters remove virtually all of the suspended particles that did not settle during the sedimentation process.
Following the conventional treatment process, chlorine and ammonia are added to the water to form chloramines and maintain a disinfectant residual in the distribution system. Sodium hydroxide is added as a corrosion control measure to adjust the pH level and protect pipes and plumbing fixtures. Also, fluoride is added to help prevent dental caries in children as recommended by the U.S. Department of Health and Human Services.

Every year trained scientists and technicians perform more than 320,000 analytical tests on more than 50,000 samples.[citation needed] Metropolitan Water District has various EPA Environmental Protection Agency approved methods used to for the detection of bacteria, viruses, protozoan parasites, chemical contaminants and toxins.

Future Expansion
Regional Recycled Water Program
In partnership with the Sanitation Districts of Los Angeles County and the Metropolitan Water District, The Regional Recycled Water Program will introduce purified and treated wastewater that will replenish groundwater basins across Los Angeles and Orange Counties that aims to potentially accommodate direct potable reuse demands in the near future.[17] The program includes 60 miles (97 km) of new pipelines to convey the treated water across four regional groundwater basins, an industrial facility, and two MWD treatment plants.

The program calls for a water treatment facility that would be the one of the largest in the nation, producing 150 million gallons per day or 168 thousand acre-feet per year of purified water.[18] However, before the full-scale facility is developed, a 0.5 million gallon per day demonstration facility, The Advanced Purification Center, in Carson will take its place and vigorously test, treat, and operate to ensure the highest quality standards of wastewater treatment are met prior to the development of the new facility.[18] The construction and application of a membrane bioreactors in the demonstration facility cost nearly $17 million dollars and the total cost of building the full-scale program will be $3.4 billion, resulting in an annual operation cost of $129 million, and water cost of $1,830 per acre-foot.[18] The full scale treatment facility would serve 500,000 homes daily and deliver a purified source of water to the four regional groundwater basins: Central, West Coast, Main San Gabriel, and Orange County.

Metropolitan list of member agencies:\\
City of Anaheim\\
City of Beverly Hills\\
City of Burbank\\
City of Compton\\
City of Fullerton\\
City of Glendale\\
City of Long Beach\\
City of Los Angeles\\
City of Pasadena\\
City of San Fernando\\
City of San Marino\\
City of Santa Ana\\
City of Santa Monica\\
City of Torrance\\
Calleguas Municipal Water District\\
Central Basin Municipal Water District\\
Eastern Municipal Water District\\
Foothill Municipal Water District\\
Inland Empire Utilities Agency (IEUA)\\
Las Virgenes Municipal Water District\\
Municipal Water District of Orange County\\
San Diego County Water Authority\\
Three Valleys Municipal Water District\\
Upper San Gabriel Valley Municipal Water District\\
West Basin Municipal Water District\\
Western Municipal Water District of Riverside County\\

\end{document}