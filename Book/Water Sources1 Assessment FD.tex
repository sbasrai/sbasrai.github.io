Which one of the following best defines the term aquifer?\\
A low lying area where water pools\\
Water-bearing stratum of rock, sand, or gravel\\
Impervious stratum near the ground surface\\
Treated water leaving the water system\\
The height to which water will rise in wells located in an artesian aquifer is called the\\
Pumping water level\\
Water table\\
Piezometric surface\\
Drawdown\\
Radius of influence\\
What percentage of all the earth's water is readily available as a potential drinking water supply in the form of lakes, rivers, and near-surface groundwater?\\
97\%\\
50\%\\
2\%\\
1\%\\
0.34\%\\
To  prevent the entry of surface contamination into a well is the purpose of\\
The well casing\\
The water table\\
The louvers or slots\\
Well development\\
The  annular grout seal	\\
An aquifer that is located underneath an aquiclude is called\\
An unconfined aquifer\\
A confined aquifer\\
A water table\\
Unreachable groundwater\\
An Artesian spring\\
The process by which water changes from the gas to the liquid phase is termed\\
Condensation	·\\
Evaporation\\
Percolation\\
Precipitation\\
Runoff\\
The free surface of the water in an unconfined aquifer is known as the\\
Pumping water level\\
Artesian spring\\
Water table\\
Drawdown\\
Percolation\\
The transfer of liquid water from plants and animals on the surface of the earth into water vapor in the atmosphere is called\\
Transpiration\\
Evaporation\\
Condensation\\
Runoff\\
Percolation\\
The elevation of water in the casing of an operating well is called the\\
Piezometric surface\\
Water table\\
Pumping water level\\
Drawdown\\
Radius of influence\\
An aquifer under pressure is often termed\\
Unconfined\\
Pacific\\
Artesian\\
Alluvial\\
Elevated\\
An aquifer is usually composed of\\
Sand and gravel \\
Clays and silts\\
Bedrock\\
Large voids in the soil, resembling underground lakes\\
None of the above\\
Which of the following best defines the term specific capacity?\\
Amount of water a given volume of saturated rock or sediment will yield to gravity\\
Amount of water a given volume of saturated rock or sediment will yield to pumping\\
Rate at which water would flow in an aquifer if the aquifer were an open conduit\\
Amount of water a well will produce for each foot of drawdown\\
The most common type of well used for public water supply systems is a\\
Jetted well\\
Driven well\\
Drilled well\\
Bored well\\
Which of the following best defines the term static water level?\\
Water level in a well after a pump has operated for a period of time\\
Water level in a well when the well is not in operation\\
Water level in a well measured from the ground surface to the drawdown water level\\
Waterlevel in a well measured from the natural water level to the drawdown water level\\
The residual drawdown of a well is defined as\\
Water level in a well after a pump has operated over a period of time\\
Measured distance from the ground to the pumping level\\
Water level below the normal level that persists after a well pump has been off for a period of time\\
Measured distance between the water level and the top of the screen\\
A well is located in an aquifer with a water table elevation 20 feet below the ground surface. After operating for three hours, the water level in the well stabilizes at 50 feet below the ground surface. The pumping water level is:\\
20 feet\\
30 feet\\
50 feet\\
70 feet\\
100 feet\\
What percentage of all the earth's water is readily available as a potential drinking water supply in the form of lakes, rivers, and near-surface groundwater?\\
97\%\\
50\%\\
2\%\\
1\%\\
0.34\%\\
To  prevent the entry of surface contamination into a well is the purpose of\\
The well casing\\
The water table\\
The louvers or slots\\
Well development\\
The  annular grout seal	\\
The process by which water changes from the gas to the liquid phase is termed\\
Condensation	·\\
Evaporation\\
Percolation\\
Precipitation\\
Runoff\\
The free surface of the water in an unconfined aquifer is known as the\\
Pumping water level\\
Artesian spring\\
Water table\\
Drawdown\\
Percolation\\
The transfer of liquid water from plants and animals on the surface of the earth into water vapor in the atmosphere is called\\
Transpiration\\
Evaporation\\
Condensation\\
Runoff\\
Percolation\\
The term for the combined processes which transfer liquid water on the earth's surface into water in the gas phase in the atmosphere is\\
Percolation\\
Evapotranspiration\\
Sublimation\\
Overdraft\\
Precipitation\\
A primary advantage of using surface water as a water source includes:\\
Usually higher in turbidity\\
Generally softer than groundwater\\
Easily contaminated with microorganisms\\
Can be variable in quality\\
Which source of water has the greatest natural protection from bacterial contamination?\\
Shallow well\\
Deep well\\
Surface water\\
Spring\\
A water-bearing formation in the soil is referred to as\\
An aquitard or aquiclude\\
An aquifer\\
An aqueduct\\
The drawdown\\
The static water level\\
An operating well will drain the water from a volume of soil around the well during pumping. This volume is referred to as the\\
Pumping water level\\
Radius of influence\\
Drawdown\\
Cone of depression\\
Recharge zone\\
One acre is 43,560 square feet. If this acre is covered with one foot of water, it contains\\
1 acre-foot\\
43,560 cubic feet\\
325,829 gallons\\
All of the above\\
None of the above\\
The safe yield of an aquifer is\\
Determined by the Department of Health Services\\
Variable, depending on rainfall\\
The average amount of water that can be withdrawn each year without causing a long-term drop in the water table\\
The difference between the static water level and the pumping water level\\
All of the above\\
The movement of water from the surface of the earth into the soil is called\\
Condensation\\
Evaporation\\
Evapotranspiration\\
Runoff\\
None of the above\\
The freezing point of water is\\
0\degree{F}\\
32\degree{C}\\
32\degree{F}\\
0\degree{C}\\
100\degree{F}\\
The movement of water from the atmosphere to the surface of the earth is called\\
Condensation\\
Evaporation\\
Evapotranspiration\\
Runoff\\
Precipitation\\
The movement of water on the surface of the earth is called\\
Percolation\\
Evaporation\\
Evapotranspiration\\
Runoff\\
Infiltration\\
A formation in the soil that resists water movement (such as a clay layer) is called\\
An aquitard or aquiclude\\
An aquifer\\
An aqueduct\\
The drawdown\\
Another term for the percolation that transports water from the surface into an aquifer is\\
Artesian springs\\
Recharge\\
Extraction\\
Overdraft\\
Runoff\\
Water that is safe to drink is called \rule{2cm}{0.3pt} water.\\
Potable\\
Palatable\\
Good\\
Clear\\
Groundwaters generally have consistent water quality that include\\
having a higher total dissolved solids content than surface water*\\
having a lower mineral content than surface waters\\
having lower pH values than surface waters\\
having a higher amount of bacteria than surface waters\\
What is the middle layer of a stratified lake called?\\
Thermocline\\
Benthic Zone\\
Epilimnion\\
Hypolimnion\\
 What is the conversion of liquid water to gaseous water known as?\\
Advection\\
Condensation\\
Precipitation\\
Evaporation\\
 Water weighs\\
$7.48 \mathrm{lbs} / \mathrm{gal}$\\
$8.34 \mathrm{lbs} / \mathrm{gal}$\\
$62.4 \mathrm{lbs} / \mathrm{ft}^{3}$\\
Both B. and C.\\
 What is the static level of an unconfined aquifer also known as?\\
Drawdown\\
Water Table\\
Pumping Water Level\\
Aquitard\\
A water bearing geologic formation that accumulates water due to its porousness\\
Aquifer\\
Lake\\
Aquiclude\\
Well\\
 What kind of stream flows continuously throughout the year?\\
Ephemeral\\
Perennial\\
Intermittent\\
Stratified\\
 The surface to atmosphere movement of water is known as\\
Precipitation\\
Percolation\\
Stratification\\
Evapotranspiration\\
 An aquifer that is underneath a layer of low permeability is known as\\
Confined aquifer\\
Water Table aquifer\\
Unconfined aquifer\\
Unreachable groundwater\\
 What is the middle layer of a stratified lake known as?\\
Hypolimnion\\
Benthic Zone\\
Thermocline\\
Epilimnion\\
 The amount of water that can be pulled from a aquifer without depleting\\
Drawdown\\
Safe yield\\
Overdraft\\
Subsidence\\
  The primary origin of coliforms in water supplies is\\
a. Natural algae growth\\
b. Industrial solvents\\
c. Fecal contamination by warm-blooded animals\\
d. Acid raid\\
A primary source of volatile organic chemical (VOC) contamination of water supplies is\\
a. Agricultural pesticides\\
b.Industrial solvents\\
c. Acid rain\\
d. Agricultural fertilizers\\
The term "surface runoff" refers to\\
a. Rainwater that soaks into the ground\\
b. Rain that returns to the atmosphere from the earth's surface\\
c. Surface water that overflows the banks of rivers\\
d. Water that flows into rivers after a rainfall\\
  A disease that can be transferred by water is\\
a. Gonorrhea\\
b. Malaria\\
c. Mumps\\
d. Typhoid\\
  Final determination of vulnerability is made by\\
a. Private contractor/consultants\\
b. The primacy agency\\
c. The water supplier\\
d. All of the above\\
  To prevent the entry of surface contamination into a well is the purpose of\\
a. The well casing\\
b. The water table\\
c. The louvers or slots\\
d. Well development\\
e. The annular grout seal\\
Potable water may be defined as\\
a. Water high in organic content\\
b. Any water that occasionally may be polluted from another source\\
c. Any water that, according to recognized standards, is safe for consumption\\
a. Water that indicates a septic condition\\
e. Water that has been transported from outside the service area\\
An operating well will drain the water from a volume of soil around the well during pumping. This volume is referred to as the\\
a.	Pumping water level\\
b.	Radius of influence\\
c.	Drawdown\\
d. Cone of depression\\
e.	Recharge zone\\
A well screen must be installed in\\
a.	deep wells\\
b.	consolidated materials\\
c.	shallow wells\\
d.	unconsolidated materials\\
A well is acidified in order to\\
a	disinfect\\
b.	increase yield\\
c.	remove objectionable gases\\
d.	 remove disinfection by-products\\
The amount of water that a well will produce for each foot of drawdown is called:\\
a.	specific head\\
b.	static yield\\
c.	yield/feet\\
d.	specific capacity\\
Surging a well to loosen scale deposits on the screen refers to:\\
a.	turning the pumps on and off as fast as possible to cause a water hammer\\
b.	pumping the water in and out of a well\\
c.	sending shock waves through the aquifer to cause a surge of water\\
d.	using a water jet to surge around the well casing.\\
A well is acidized in order to\\
a. Disinfect the water\\
b. Increase yield\\
c. Remove objectionable gasses\\
d. Remove disinfection by-products\\
To prevent the entry of surface contamination into a well is the purpose of\\
a, The well casing\\
b. The water table\\
c. The louvers or slots\\
d. Well development\\
e. The annular grout seal\\
The variation in water demand during the course of a day is termed\\
a. Seasonal variation\\
b. Fire flow requirements\\
c. Emergency storage variation\\
d. The straight line equalization method\\
e. Diurnal variation\\
The maximum momentary load placed on a water supply system is known as\\
a. Average daily flow\\
b. Average daily demand\\
c. Rated capacity\\
d. A System float\\
d. Peak demand\\
The term aquifer refers to:\\
a. A special type of aqueduct.\\
b. A natural source of water.\\
c. A potable water.\\
d. Water bearing strata.\\
The use of a well supply as a source normally results in:\\
a. Water that is high in nitrates\\
b. Water of consistent quality\\
c. Water very high in mineral content\\
d. Water that is considered "soft".\\
Maximum Safe Yield of a water source is defined as:\\
a) Where the state health department has approved the source of use.\\
b) The quantity of water that can be taken from a source of supply over a period of years without depleting the source permanently - beyond it's ability to replenish in wet years.\\
c) Water that is free of bacteria.\\
d) Quantity of water that may be treated in the plant.\\
Movement of water through the ground is called:\\
a) Hydraulic subsidence\\
b) Runoff\\
c. Percolation\\
d. Infiltration\\
A primary source of volatile organic chemical (VOC) contamination of water supplies is\\
a. Agricultural pesticides\\
b. Industrial solvents\\
c. Acld rain\\
d. Agricultural fertilizers\\
Surging a well to loosen scale deposits on the screen refers to:\\
a. turning the pumps on and off as fast as possible to cause a water hammer\\
b. pumping the water in and out of a well\\
c. sending shock waves through the aquifer to cause a surge of water\\
d. using a water jet to surge around the well casing.\\
A sanitary well seal is used to:\\
a. seal the clear well\\
b. seal the top of the well casing\\
c. seal the water tower\\
d. seal a break in the distribution system\\
The amount of water that a well will produce for each foot of drawdown is called:\\
a. specific head\\
b. static yield\\
c. yield/feet\\
d. specific capacity\\
After replacing a repaired pump back into a well, the operator should first:\\
a put the seal on tight to avoid contamination\\
b. add chlorine to disinfect the well and surrounding aquifer\\
c. start the pump to make sure that it will pump water\\
d. open the valve to let the pressure off the line \\
The amount of water in a water-bearing formation depends on the\\
a. Depth of the well\\
b. Size of the pump\\
c. Thickness and permeability of the formation\\
d. Type of well casing\\

