\chapterimage{WaterChapterImage1}
\chapter{Water Treatment}
Purpose of water treatment – to provide safe drinking water that does not contain objectionable taste, odor or color; to provide adequate quantities of water for domestic, commercial, industrial and fire protection needs.

All water produced by public water systems must be drinking water quality, even though only about 1% of water produced is used for drinking and cooking.

Schematic of conventional water treatment:
\begin{enumerate}
\item Water is withdrawn from a lake, reservoir or river at the intake
\item It is screened to remove debris
\item Water then enters the flash mixing tank where coagulants and other chemicals are added
\item Then it is divided into the flocculation basin
\item After flocculation, the water enters the settling basins where solids are removed
\item Filtration then removes particles that are too small to settle by gravity
\item The water is disinfected using some form of chlorine
\item Other chemicals such as fluoride, phosphate corrosion inhibitors or pH adjustment chemicals may be added
\item After a minimum detention time, the water may be pumped to the distribution systems Other processes may occur, such as pre-oxidation or activated carbon treatment.
\end{enumerate}



Typical drinking water treatment processes\\
Treatment for drinking water production involves the removal of contaminants and/or inactivation of any potentially harmful microbes from raw water to produce water that is pure enough for human consumption without any short term or long term risk of any adverse health effect.\\

Elimination of hazardous chemicals from the water, many treatment procedures have been applied. The selection of wastewater treatment systems is contingent on a number of factors: (
\begin{enumerate}
\item The degree to which a method is necessary to raise the waste water quality to a permissible level; 
\item The control method's flexibility;
\item The process's cost; and
\item The process's environmental compatibility.\\
\end{enumerate}
The processes involved in removing the contaminants include physical processes such as settling and filtration, chemical processes such as disinfection and coagulation, and biological processes such as slow sand filtration.\\
A combination selected from the following processes (depending on the season and contaminants and chemicals present in the raw water) is used for municipal drinking water treatment worldwide.\\
\begin{enumerate}
\item Chemical\\

Chemical approaches are used in addition to physical and biological measures to reduce the discharge of pollutants and waste water into water bodies. Different chemical procedures for the conversion into final products or the removal of pollutants are used for the safe disposal of contaminants.\\
\begin{itemize}
\item Pre-chlorination for algae control and arresting biological growth.
\item Aeration along with pre-chlorination for removal of dissolved iron when present with relatively small amounts of manganese.
\item Disinfection for killing bacteria, viruses and other pathogens, using chlorine, ozone and ultra-violet light.
\end{itemize}
\item Physical\\
Physical techniques of water/waste water treatment rely on physical phenomena to complete the removal process, rather than biological or chemical changes.\\
Most common physical techniques are:\\
\begin{itemize}
\item Sedimentation is one of the most important main wastewater treatment procedures. Gravity settling is a method of separating particles from a fluid. The particle in suspension remains stable in quiescent conditions due to the decrease in water velocity throughout the water treatment process, following which the particles settle by gravitational force. For solids separation that is the removal of suspended solids trapped in the floc.
\item Filtration is the technique of removing pollutants based on their particle size is known as filtration. Pollutant removal from waste water permits water to be reused for a variety of purposes. The types of filters used in the procedure differ depending on the contaminants present in the water. Particle filtration and Membrane filtration are the two main forms of waste water filtration.
\item Dissolved air flotation (Degasification) is the process of removing dissolved gases from a solution . The law of Henry's law states that the amount of dissolved gas in a liquid is proportionate to the partial pressure of the gas. Degasification is a low-cost method of removing carbon dioxide gas from waste water that raises the pH of the water by removing the gas. 
\end{itemize}
\item Physico-chemical\\
Also referred to as "Conventional" Treatment”
\begin{itemize}
\item Coagulation for flocculation. The addition of coagulants destabilizes colloidal suspensions by neutralizing their charges, resulting in the aggregation of smaller particles during the coagulation process.
\item Coagulant aids, also known as polyelectrolytes – to improve coagulation and for more robust floc formation.
\item Polyelectrolytes or also known in the field as polymers, usually consist of either a positive or negative charge. The nature of the polyelectrolyte used is purely based on the source water characteristics of the treatment plant.
\item These will usually be used in conjunction with a primary coagulant such as ferric chloride, ferric sulfate, or alum.
\end{itemize}
\item Chemical Precipitation\\
Chemical precipitation is a common process used to reduce heavy metals concentrations in wastewater. The dissolved metal ions are transformed to an insoluble phase by a chemical interaction with a precipitant agent such as lime. In industrial applications stronger alkalis may be used to effect complete precipitation. In drinking water treatment, the common-ion effect is often used to help reduce water hardness.
\item Flotation\\
Flotation uses bubble attachment to separate solids or dispersed liquids from a liquid phase.

\item Membrane Filtration\\
Membrane filtration has gotten a lot of attention for inorganic effluent treatment since it can remove not only suspended solids and organic components, but also inorganic pollutants such heavy metals. For heavy metal removal, several forms of membrane filtration, such as ultrafiltration, nanofiltration, and reverse osmosis, can be used depending on the particle size that can be maintained. 
\item Ion Exchange\\
Ion exchange is a reversible ion exchange process in which an insoluble substance (resin) takes ions from an electrolytic solution and releases additional ions of the same charge in a chemically comparable amount without changing the resin's structure.
\item Electrochemical Treatment Techniques\\
\begin{itemize}
\item Electrodialysis (ED)
\item Membrane electrolysis (ME)
\item Electrochemical precipitation (EP)
\end{itemize}
\item Adsorption\\
Adsorption is a mass transfer process in which a substance is transported from the liquid phase to the surface of a solid/liquid (adsorbent) and becomes physically and chemically bonded (adsorbate). Adsorption can be classified into two forms based on the type of attraction between the adsorbate and the adsorbent: physical and chemical adsorption, commonly known as physisorption and chemisorptions.\\
\textbf{Activated Carbon}\\
Activated carbons (ACs) are effective adsorbents for a wide variety of contaminants. The adsorptive removal of color, aroma, taste, and other harmful organics and inorganics from drinking water and wastewater is one of their industrial applications.\\
Both a high surface area and a large pore size can improve the efficiency of activated carbon. Activated carbon was utilized by a number of studies to remove heavy metals and other types of contaminants from wastewater. The cost of activated carbon is rising due to a shortage of commercial activated carbon (AC). Because of its high surface area, porosity, and flexibility, activated carbon has a lot of potential in wastewater treatment.\\ 
\item Biological Treatment\\
This is the method by which dissolved and suspended organic chemical components are eliminated through biodegradation, in which an optimal amount of microorganism is given to re-enact the same natural self-purification process. Through two distinct biological process, such as biological oxidation and biosynthesis, microorganisms can degrade organic materials in wastewater. Microorganisms involved in wastewater treatment produce end products such as minerals, carbon dioxide, and ammonia during the biological oxidation process. The minerals (products) remained in the wastewater and were discharged with the effluent. Microorganisms use organic materials in wastewater to generate new microbial cells with dense biomass that is eliminated by sedimentation throughout the biosynthesis process.\\
\textbf{Bioremediation}\\
Phytoremediation, Rhizofiltration, Bioaugmentation, and Biostimulation are all biological treatment method in which microorganisms breakdown or transform hazardous contaminants in wastewater to a less toxic or non-toxic state. Both autotrophs and heterotrophs may be involved. Autotrophs can fix carbon and use inorganic chemicals in wastewater to make organic compounds such as fats, proteins, and carbohydrates. Heterotrophs feed on the soluble and emulsified organic organic substances present in the wastewater to develop and reproduce.\\
The sewage treatment processes of trickling filters and activated sludge both depend on maintained populations of heterotrophic organisms to do much of the work in removing contaminants. In potable water production, the hypogeal zone (the Schmutzdecke) of Slow sand filtration is a naturally developed biofilm that metabolizes organic matter, adsorbs soluble components and entraps particulates.
\end{enumerate}
% Please add the following required packages to your document preamble:
% \usepackage[table,xcdraw]{xcolor}
% If you use beamer only pass "xcolor=table" option, i.e. \documentclass[xcolor=table]{beamer}
% \usepackage[normalem]{ulem}
% \useunder{\uline}{\ul}{}
\begin{table}[]
\begin{tabular}{
>{\columncolor[HTML]{F8F9FA}}l 
>{\columncolor[HTML]{F8F9FA}}l lll}
{\color[HTML]{202122} \textbf{Constituent}}                & {\color[HTML]{202122} \textbf{Unit processes}}                                             &  &  &  \\
{\color[HTML]{202122} \textbf{Turbidity and particles}}    & {\color[HTML]{202122} {\ul Coagulation/ flocculation, sedimentation, granular filtration}} &  &  &  \\
{\color[HTML]{202122} \textbf{Major dissolved inorganics}} & {\color[HTML]{202122} {\ul Softening, aeration, membranes}}                                &  &  &  \\
{\color[HTML]{202122} \textbf{Minor dissolved inorganics}} & {\color[HTML]{202122} {\ul Membranes}}                                                     &  &  &  \\
{\color[HTML]{202122} Pathogens}                           & {\color[HTML]{202122} Sedimentation, filtration, disinfection}                             &  &  &  \\
{\color[HTML]{202122} Major dissolved organics}            & {\color[HTML]{202122} Membranes, adsorption}                                               &  &  & 
\end{tabular}
\end{table}

We live on a planet with a surface that is three-fourths covered with water, so we
recognize the irony inherent in the fact that many areas of the world face critical
shortages of drinking water. Most of Earth’s water is seawater, of course—far
too saline for human consumption. Of the little “fresh” water that remains, most
is trapped in polar ice caps, where harnessing it for use is difficult. Much of the
accessible natural supply of potable water is stressed by a growing world population,
which increases the basic demand for this natural resource while reducing\\

Developing world cities with private water-management companies have been plagued by lapses in service, soaring costs, corruption and worse. In Manila, where the water system is controlled by Suez, San Francisco-based Bechtel and the prominent Ayala family, water is only reliably available for a few hours a day, and rate increases have been so severe that the poorest families must choose each month between paying for water and two days’ worth of food. In 2001 the government of Ghana agreed to privatise local water systems as a condition for an IMF loan. To attract investors, the government doubled water rates, setting off protests in a country where the average annual income is less than \$400 a year and the water bill (for those fortunate enough to have running water) can run upwards of \$110. In Cochabamba, the third-largest city in Bolivia, water rates shot up by 35 per cent after a consortium led by Bechtel took over the city’s water system in 1999; some residents found themselves paying 20 per cent of their income on water. An initial round of peaceful street protests led to riots in which six people were killed. Eventually, the Bolivian government voided Bechtel’s contract and told the company’s officials it could not guarantee their safety if they stayed in town. Privatisation has also spawned protests (and, in some cases, even dominated elections) in Paraguay, where police turned water cannons on anti-privatisation protesters, Panama, Brazil, Peru, Colombia, India, Pakistan, Hungary and South Africa.
Louma (2004)\\

It is common practice to treat wastewater to the point where it is cleaner than the local waterways into which it is ultimately released. Eventually, it arrives at the ocean, with absolutely no downstream use—this is referred to as one-and-done usage. Why waste such a valuable resource? Why not reuse it? But we do already reuse it to some extent through de facto water recycling, as shown in Figure 11.1.\\

INTRODUCTION
We live on a planet with a surface that is three-fourths covered with water, so we recognize the irony inherent in the fact that many areas of the world face critical shortages of drinking water. Most of Earth’s water is seawater, of course—far too saline for human consumption. Of the little “fresh” water that remains, most is trapped in polar ice caps, where harnessing it for use is difficult. Much of the accessible natural supply of potable water is stressed by a growing world population, which increases the basic demand for this natural resource while reducing
the supply further through contamination. Major population centers in developing nations (those without established waste treatment or water treatment infrastructure) often suffer from epidemics of waterborne disease. In these areas, raw sewage can directly contaminate the rivers and streams used for drinking, washing, and cooking. In other cases, unchecked industrialization leads to water contamination through improperly disposed of chemical and nuclear wastes. The drinking water purveyor must ensure that the drinking water supplied is safe for human consumption. In fact, the primary reason for the development and installation of a public water system is the protection of public health. Basically, a properly operated water system serves as a line of defense between disease and the public. Properly operated water treatment and supply systems are defined as those that:\\
\begin{itemize}
\item Remove or inactivate pathogenic microorganisms including bacteria, viruses, and protozoa.
\item Reduce or remove chemicals that can be detrimental to health.
\item Provide quality water, thus discouraging the customer from seeking better tasting or better looking water that may be contaminated.
\end{itemize}
This last point is critical, but one often overlooked in the operation and management of public water systems. When the water produced by a system is objectionable because of odor, taste, or appearance, customers will seek another source for their drinking water. Ironically, these alternative sources, although they look, taste, and smell fine (“better than city water”), could contain microorganisms or chemicals that are harmful. This chapter discusses the drinking water practitioner’s most important function: ensuring that water delivered to the public is properly treated and arrives as the clean, wholesome, safe product that it must be. Moreover, it also covers the innovative approach taken by the Hampton Roads Sanitation District (HRSD) to replacing one-and-done usage with one-and-redone usage.\\

\textbf{CONVENTIONAL WATER TREATMENT}\\
A typical water treatment plant treats stream or river water (turbid surface water with organics) and processes the raw water using various unit processes, including:\\
(1) screening, (2) coagulation, (3) flocculation, (4) sedimentation or settling, (5) filtra-
tion, (6) hardness treatment, (7) disinfection, and (8) fluoridation (see Figure 11.2). This chapter provides a brief overview of each of these unit processes, which constitute a typical drinking water treatment system for surface water supplies, in addition to a brief discussion of alternative approaches.\\

\textbf{Screening}
Screening (the first important step in treating water containing large solids) is defined as the process whereby relatively large and suspended debris is removed from the water before it enters the plant. River water (the source of water used in our discussion) frequently contains suspended and floating debris varying in size from small rags to logs. Removing these solids is important, not only because these items have no place in potable water but also because this river trash may cause damage to downstream equipment (e.g., clog and damage pumps), may increase chemical requirements, may impede hydraulic flow in open channels or pipes, or may hinder the treatment process (Pankratz, 1995). The most important criteria used in the selection of a particular screening system for water treatment technology are the screen opening size and flow rate. Other important criteria include costs related to operation and equipment, plant hydraulics, debris handling requirements, and operator qualifications and availability. Large surface water treatment plants may employ a variety of screening devices, such as trash screens (or trash rake), traveling water screens, drum screens, bar screens, or passive intake screens. Each of these screening devices is briefly discussed in the following sections.\\
 

Trash Screens (Rakes)\\
Trash screens or trash rakes are used to remove rough or large debris retained on a trash rack. They protect pumping equipment and may be used as a preliminary screening device to protect finer screens—drum or traveling water screens, for example. A trash screen consists of one or more stationary trash rakes and a screen raking device. Trash rack bar spacings range from 1.5 to 4 inches and are mostly constructed of steel bars. Those constructed of high-density polyethylene polymers are beginning to replace many of the older steel bar models—these synthetic screens are lighter and less prone to microbial growth, corrosion, and ice. Raking mechanisms are available for use in a variety of intake configurations, including installation on vertical building and dam walls. Rakes are typically mounted on fixed structures designed to clean a single trash rack, suspended from an overhead gantry, wheelmounted to traverse the entire width of an intake structure and clean individual sections of a wide trash rack, or suspended from an overhead gantry.\\
Traveling Water Screens\\
Traveling water screens (sometimes called bandscreens) are placed in a channel of flowing water to remove floating or suspended debris. These automatically cleaned screening devices protect pumping or other downstream equipment from debris in surface water intakes. Consisting of a continuous series of wire mesh panels bolted to basket frames, or trays, and attached to two matched strands of roller chain, the traveling water screen operates in a vertical path over a sprocket assembly through the flow. As raw water passes through the revolving baskets, debris is collected and retained on the upstream face of the wire mesh panels. The debris-laden baskets are lifted out of the flow and above the operating flow, where a high-pressure water spray directed outward removes the impinged debris. This process can be continuous or intermittent. For intermittent operation, the screen activates when a specified headloss or time elapsed has occurred. When located on a river, traveling water screens may be subject to large fluctuations in flow conditions, debris loading, water depths, and salinity. Depending upon application, the size of the traveling water screen is determined by considering such factors as maximum and average flow; maximum, minimum, and average water levels; wire mesh size; velocity through mesh; basket or channel width; number of screens; type of service; and/or starting and operating headloss requirements.\\
Drum Screens\\
A drum screen (or cylinder screen) has very few moving parts and is mounted on a horizontal axis with a series of wire mesh panels attached or mounted on the periphery of its cylinder. The cylinder slowly rotates on its axis. Because of its simplicity of construction, the maintenance requirements and operating costs of a drum screen installation are usually lower than those of a traveling water screen.\\
Bar Screens\\
Primarily used in wastewater treatment applications, bar screens are also employed in some water treatment facilities. A bar screen consists of straight steel bars welded at both ends to two horizontal steel members and is automatically cleaned by one
or more power operated rakes. As a rake is operated up the face of the bar rack, it removes accumulated debris (usually large solid objects and rags) and elevates in and out of the flow. At the top of the operating cycle of the rake, the debris is swept from the rake into a debris receptacle by a wiper mechanism. When installed in a waterway, the bar screen assembly normally is placed at an angle of 60 to 80 degrees from the horizontal.\\
Passive Intake Screens\\
Passive intake screens (stationary screening cylinders) have no moving parts and require no debris handling or debris removal equipment. Passive intake screens are placed in a surface water body in such a manner so as to take advantage of natural ambient currents and controlled through-screen velocities to minimize debris buildup. Usually mounted on a horizontal axis and oriented parallel to the natural current flow within the water body, current flow action works to keep the screen clean. Maximum intake velocity of about 0.5 foot per second (fps) is typical and works to minimize debris impingement on the screen surface.\\


\textbf{Coagulation}\\
Coagulation, the second step in water purification, is a unit process that has been used for several years in the treatment of raw water. Basically, coagulation works to settle very fine material of suspended solids.
Note: Chemicals employed for coagulation are expected to be safe for drinking water when used according to the American Water Works Association (AWWA) coagulation standards (e.g., Coagulation, Nos. 42402 to 42407).\\

Coagulants\\
Typically, after screening, raw water is pumped into large settling basins, also known as clarifiers or sedimentation tanks. Within the confines of the settling basin, the screened raw water is allowed to sit for some predetermined time. Although screened, the raw water still contains impurities that may be either dissolved or suspended. The settling basin provides the most convenient way to remove the suspended matter, as it lets the force of gravity do the work. Within the basin, when flow and turbulence are minimal (quiescent conditions), particles more dense than water settle to the bottom of the tank. This process is called sedimentation, and the layer of accumulated solids at the bottom of the tank is called sludge (or biosolids in some wastewater treatment unit processes). The size and density of the suspended particles have a direct bearing on the speed at which they will settle toward the bottom of the basin. The larger or heavier particles will, of course, settle faster than smaller or lighter particles. The forces opposing the downward force of gravity include buoyancy and drag (friction). The particle-settling rate is also affected by the temperature and viscosity of the water.\\
Note: The nature of the sedimentation process also varies with the concentration of suspended solids and their tendency to interact with one another.\\
In the sedimentation process just described, not all suspended solids or particles can be completely removed from water, even when given very long detention times. Very small particles called colloids (e.g., bacteria, fine clays, silts) will not settle out of suspension by gravity without some help. This is where coagulants come into play. If we rapidly mix chemical coagulants in the water and then slowly stir the mixture before allowing sedimentation to occur, the colloidal particles will settle. Colloids or finer particles must be chemically coagulated to produce larger floc that is removable in subsequent settling and filtration.
The coagulation process (along with flocculation) works to neutralize or reduce the natural repelling electrical force of particles in water, keeping them apart and in suspension. Particles in water usually carry a negative electrical charge. Because all of these particles carry this same negative electrical charge, they repel each other—in the same way that like poles of a magnet do. The object of coagulation (and subsequently flocculation) is to turn the small particles into larger flocs, either as precipitates or suspended particles. These flocs are then conditioned for ready removal in subsequent processes. Stated another way, in this text we define coagulation as a method to alter the colloids so they will be able to approach and adhere to each other to form larger floc particles.\\
Types of Coagulants:\\
Two types of coagulants are used in the coagulation process: coagulants and coagulant aids. Generally, the types of coagulants and aids available are defined by the plant process scheme. To determine optimum chemical dosages for treatment, jar tests are normally used.\\
Jar Tests\\
Jar tests are widely used to simulate a full-scale coagulation and flocculation process to determine optimum chemical dosages—the cost-effective dose of a coagulant for the time and intensity of agitation selected. Such tests have been used for many years by the water treatment industry; the test conditions are intended to reflect the normal operation of a chemical treatment facility and allow evaluation of the type and quantity of sludge and physical properties of the floc. The test can be used to:\\
\begin{itemize}
\item Select the most effective chemical.
\item Select the optimum dosage.
\item Determine the value of a flocculant aid and the proper dose.
\end{itemize}
The testing procedure requires a series of samples to be placed in testing jars and mixed at 100 rpm. Varying amounts of the process chemical or specified amounts of several flocculants are added (one volume/sample container). The mix is continued
for one minute. The mixing slows to 30 rpm to provide gentle agitation, then the floc is allowed to settle. The flocculation period and settling process are observed carefully to determine the floc strength, settleability, and clarity of the supernatant liquor (the water that remains above the settled floc). The supernatant can then be tested to determine the efficiency of the chemical addition for removal of total suspended solids (TSS), biochemical oxygen demand (BOD), and phosphorus. The equipment required for the jar test includes a six-position, variable-speed paddle mixer; six 2-quart wide-mouth jars; an interval timer; and assorted glassware, pipettes, graduates, and so forth.\\

Coagulation Chemicals\\
Several different chemicals can be used for coagulation. Commonly used metal coagulants are those based on aluminum (aluminum sulfate) and those based on iron (ferric sulfate). The most common coagulant is aluminum sulfate (alum, Al2(SO4)3). Other common coagulation chemicals are provided in Table 11.1.\\

Coagulant Aids\\
Coagulation problems often occur because of slow-settling precipitates or fragile flocs that are easily fragmented under hydraulic forces in basins and filters (Hammer and Hammer, 1996). A coagulant aid is a chemical added during coagulation to improve coagulation; to build stronger, more settleable floc; to overcome the effect of temperature drops that slow coagulation; to reduce the amount of coagulant needed; and to reduce the amount of sludge produced (AWWA, 1995). Coagulant aids benefit
flocculation by improving the settling qualities and toughness of flocs. Polymers are the most widely used materials. Synthetic polymers are water-soluble, high-molecular-weight organic compounds with multiple electrical charges along a molecular chain of carbon atoms. In drinking water treatment, polymers are extensively used as coagulant aids to build large floc prior to sedimentation and filtration. Other coagulant aids are activated silica, adsorbent weighting agents, and oxidants.\\

Coagulation Process Operation\\
The common coagulation unit process operation involves the addition of coagulant chemicals by rapid mixing—detention time in the rapid mix tank is typically on the order of minutes (Masters, 1991). During this mixing process, polymer (or some other coagulant aid) is added and blended into the destabilized water prior to flocculation. The removal of impurities by coagulation depends on their nature and concentration, the use of both coagulants and coagulants aids, and characteristics of the water, including pH, temperature, and ionic strength. Because of the complex nature of coagulation reactions, chemical treatment is based on empirical data derived from jar testing or other laboratory tests and field studies (Viessman and Hammer, 1998).\\

\textbf{Flocculation}\\
The destabilized particles and chemical precipitates resulting from coagulation are designed to enhance their settling qualities and thus their removal from water; however, even after coagulation has taken place, these particles and chemical precipitates may still settle very slowly (too slowly). To speed up the settling process, flocculation is employed.
Note: Flocculation is the clumping together of the fine particles formed by coagulation. Although the two terms are often used interchangeably, flocculation and coagulation are actually distinct concepts.
Flocculation is the most important factor affecting particle-removal efficiency. In water treatment operations, flocculation is a slow mixing process in which the coagulated particles are brought into contact so they will collide, stick together, and grow (agglomerate) to a size that will readily settle. Enough mixing must be provided (e.g., gentle agitation for approximately half an hour) to bring the floc particles into contact with each other and to keep the floc from settling in the flocculation basin. (The heavier the floc and the higher the suspended solids concentration, the more mixing is required to keep the floc in suspension.) The most common type of mixer or flocculator is the paddle type, which uses redwood slats mounted horizontally on motor-driven shafts. Rotating slowly at about one revolution per minute, the paddles provide gentle agitation that promotes floc growth. The rate of agglomeration or flocculation depends on the number of particles present, the relative volume that they occupy, and the velocity gradient in the basin.  Note: The statement that the rate of agglomeration or flocculation depends on velocity gradient refers to the fact that too much mixing can shear the floc particles, tearing them apart again; the floc then becomes smaller and more finely dispersed, a situation we are obviously trying to avoid. For this reason, the velocity gradient must be controlled within a relatively narrow range.
The theory of flocculation is complex and beyond the needs of this text, but on an elemental level we can say that flocculation is generally accomplished by slowly rotating, large-diameter mixers. Current practice incorporates dispersion of the coagulant (flash mixing), flocculation, and sedimentation in a single unit called a contact clarifier.
Note: Flocculation is the principal mechanism for removing turbidity from water.

\textbf{Sedimentation}\\
In a conventional water treatment plant, the process of coagulation and flocculation precedes the sedimentation process for better results and improved utilization of the settling basins. Sedimentation is then followed by the filtration process. Filtration occasionally may be preceded only by coagulation, in which case filtration is provided after only a few minutes of contact, adding additional stress to the filters. Lack of sedimentation results in less reliable operation of filters when water quality suddenly changes characteristics (DeZuane, 1997).
Sedimentation (also known as clarification) is the gravity-induced removal of particulate matter, chemical floc, precipitates from suspension, and other settleable solids. Simply stated, sedimentation separates the liquid from the solids. The process takes place in a rectangular, square, or round tank called a settling or sedimentation tank or basin. Flow patterns within such basins may be rectilinear flow in rectangular basins, radial flow in center-feed settling tanks or square settling tanks, or radial flow or spiral flow in peripheral-feed settling tanks.
Sedimentation, in the conventional water treatment process, is typically the step between flocculation and filtration. Design criteria are based on empirical data from the performance of full-scale sedimentation tanks. The common criteria for sizing settling basins are detention time (typically from 1 to 10 hr), overflow rate, weir loading, and, with rectangular tanks, horizontal velocity.
In water treatment, the majority of settling basins are essentially upflow clarifiers where the water rises vertically for discharge through effluent channels. More specifically, in the idealized sedimentation tank, water flows horizontally through the basin and then rises vertically, overflowing the weir of a discharge channel at the tank
surface. Floc settles downward, opposite the upflow of water, and is removed from the bottom by a continuous mechanical sludge removal apparatus. The particles with a settling velocity greater than the overflow rate are removed (settled) while lighter flocs are carried out in the effluent. The effluent is then filtered.
Note: Sedimentation tanks, either circular or rectangular, are designed for slow, uniform water movement with a minimum of short-circuiting.\\

textbf{Filtration}\\
Even after chemical coagulation and sedimentation by gravity, not all of the suspended solids or impurities are removed from water. Nonsettleable floc particles (about 5% of the suspended solids) may still remain in the water, and with only that small percentage left we might ask, “Isn’t this good enough?” No, it isn’t. This remaining floc would cause problems (including noticeable turbidity), and particles shield microorganisms from the subsequent disinfection processes. The goal of water treatment is to produce potable water that is perceptually crystal clear and that satisfies the Safe Drinking Water Act (SDWA) requirement of 0.5 NTU for turbidity. To accomplish this, an additional treatment step is required that follows coagulation, flocculation, and sedimentation.
Filtration (sometimes called a polishing process) involves the removal of suspended particles from water by passing it through a layer or bed of a porous granular material—sand, for example. As water flows through the filter bed, the suspended particles become trapped within the pore spaces of the filter material (or filter media). When purifying a surface water source (as in the discussion here), filtration is a very important process, even though filtration is only one step in the overall treatment process.
Note: Filtration is the process that occurs naturally as surface waters migrate (percolate) through the porous layers of soil to recharge groundwater. This natural filtration removes most suspended matter and microorganisms and is the reason why many wells produce water that does not require any further treatment.\\
The Surface Water Treatment Rule (SWTR) specifies certain filtration technologies. The most common treatment filter systems include rapid gravity filters (either built onsite or packaged plants) and pressure filters. Other types include direct filtration, slow sand filters, and diatomaceous earth (DE) filters. The SWTR also allows the use of alternative filtration technologies, such as cartridge filters.\\
Filtration treatment unit processes most commonly used in water purification systems include slow or rapid sand filtration, diatomaceous earth filtration, and package filtration systems. Slow and rapid filter systems refer to the rate of flow per unit of surface area. Filters are also classified by the type of granular material used in them. Sand, anthracite coal, coal–sand, multilayered, mixed bed, and diatomaceous earth are examples of different filtering media. Filtration systems may also be classified by the direction the water flows through the medium: downflow, upflow, fine-tocoarse, coarse-to-fine. Finally, filters are commonly distinguished by whether they are gravity or pressure filters. Gravity filters rely only on the force of gravity to move the water down through the grains and typically use upflow for washing (backwashing) the filter media to remove the collected foreign material. Gravity filters are free surface filters commonly used for municipal applications. Pressure filters are completely enclosed in a shell so most of the water pressure in the lines leading to the filter is not lost and can be used to push the water through the filter.\\

Rapid Filter Systems\\
Slow sand filtration has been used in the United States since 1872. It is still used in many older plants but is not commonly used today in most modern water treatment plants because of various problems associated with this technique. One of the problems is related to the tiny size of the pore spaces in the fine sand, which slows down the water’s progress through the filter bed. These filter types also have problems with suspended particles clogging the surface, requiring the filter to be manually scraped clean. These units take up a considerable amount of land area because slow filtration rates require a greater filter surface area to produce the necessary filtered water qualities.
In modern water treatment plants, the rapid filter has largely replaced the slow sand filter. The rapid filter consists of a layer of carefully sieved silica sand ranging from 0.6 to 0.75 m in depth on top of a bed of graded gravels. The pore openings between the grains of sand are often greater than the size of the floc particles that are to be removed, so much of the filtration is accomplished by means other than simple straining.
Note: The ideal filter medium is coarse enough for large pore openings to retain large quantities of floc, yet sufficiently fine to prevent the passage of suspended solids. It must have adequate depth to allow relatively long filter runs and be graded to permit effective cleaning during backwash.\\
Adsorption, continued flocculation, and sedimentation in the pore spaces are also important removal mechanisms. When the filter becomes clogged with particles (which occurs approximately once a day, depending on the turbidity of the water), the filter is shut down for a short period of time and cleaned by forcing water backward through the sand for 10 to 15 minutes. After cleaning, the sand settles back in place and operation resumes.\\
Other Common Filter Types\\
Rapid flow filters are the most common type used for treating water supplies, primarily because they are the most reliable, but other types of filters are sometimes used to clarify water, including pressure filters and diatomaceous earth filters. A pressure filter is similar to a rapid filter in that the water flows through a granular filter bed; however, instead of being open to the atmosphere and using the force of gravity, the pressure filter is enclosed in a cylindrical steel tank and the water is pumped through the bed under pressure. They are not as reliable as rapid filters, because pressure may force solids through the bed in the effluent. Because of this problem, they are seldom employed in municipal water treatment works but instead are used for filtering water for industrial use or for swimming pools. Diatomaceous earth filters contain a thin layer of a natural, powdery material formed from the shells of diatoms; they are also used primarily for industrial or swimming pool aapplications because they are not as reliable as rapid sand filters.\\

The unit processes described thus far—screening, coagulation, flocculation, sedimentation, and filtration—together comprise a type of treatment called clarification. Along with removing turbidity and suspended solids, clarification also removes many microorganisms from the water; however, clarification by itself is not sufficient to ensure the complete removal of pathogenic bacteria and viruses.
Earlier it was stated that one of the primary goals of water treatment is to treat raw water to the point where it is possible to deliver to the consumer a water product that is perceptually crystal clear. Obviously, the consumer does not want to drink a glass full of mud, a glass full of slime, a glass full of metal-colored, foul-smelling water— or even a glass of water that looks like it was dipped from a creek. Would you? The point is, when the water has been treated to the point of crystal clarity, the treatment process must be taken a step further—to the point where the water is completely free of disease-causing microorganisms. To accomplish this, the final treatment process in water treatment plants occurs—disinfection, which destroys or inactivates pathogens.\\

\textbf{Hardness treatment}\\
Two commonly used methods to reduce hardness are the lime-soda process and ion exchange. The lime-soda process is applicable for large facilities, whereas ion exchange is normally employed in smaller water works. The lime-soda process will not remove all of the hardness and is usually operated to produce a residual hardness of about 100 mg/L as CaCO3. Greater reductions are not economical and may have adverse health consequences as well (McGhee, 1991). The discussion in this text focuses on ion exchange. Ion exchange is accomplished by charging a resin with sodium ions and allowing the resin to exchange the sodium ions for calcium or magnesium ions. Common resins include zeolites—natural and manmade minerals that will collect from a solution certain ions (sodium or KMnO4), and either exchange these ions (in the case in water softening) or use the ions to oxidize a substance (in the case of iron or manganese removal). The negative side of using ion exchange is that, even though the process softens water by removing all (or nearly all) of the hardness and adds sodium ions to the water, the water may be more corrosive than before. The addition of sodium ions to the water may also increase the health risk of those with high blood pressure.

\textbf{Disinfection}\\
At the turn of the last century, 35,000 people per 1,000,000 people did not reach 20 years of age. Today, however, the rate of births exceeds the rate of deaths, and the average lifespan is much longer. Curbing waterborne disease through disinfection
has made a significant contribution to birth rates outpacing death rates worldwide. The Safe Drinking Water Act requires that public water supplies be disinfected, and the U.S. Environmental Protection Agency (USEPA) sets standards and establishes processes for the treatment and distribution of disinfected water to ensure that no significant risks to human health occur. The USEPA Science Advisory Board has ranked pollutants in drinking water as one of the highest health risks meriting the Agency’s attention because of large-scale population exposure to contaminants, including lead, disinfectants and disinfection byproducts (DBPs), and disease-causing organisms.
Disinfectants are used by virtually all surface water systems in the United States and by an unknown percentage of systems that rely on groundwater. For nearly a century, chlorine has been the most widely used and most cost-effective disinfectant; however, disinfection treatments can produce a wide variety of byproducts, many of which have been shown to cause cancer or other toxic effects. Recently, concern has been raised over water quality deterioration, a problem that can grow dramatically during distribution unless systems are properly designed and operated. Disinfection is an integral part of water treatment, but filtration prior to disinfection is necessary to reduce pathogen levels and make disinfection more reliable by removing turbidity and other interfering constituents.
To solve the disinfectant and disinfection byproducts problem, we need innovative upgrades for the existing techniques, as well as new approaches to address these problems. Areas of interest include:\\
\begin{itemize}
\item Alternatives to chlorine disinfection for removing pathogenic microorganisms, including innovative applications of ultraviolet (UV) radiation and processes that improve overall effectiveness while using reduced amounts of disinfectant
\item Development of innovative unit processes, particularly for small systems, for removal of organic and inorganic contaminants (such as arsenic), particulates, and pathogens, such as cyst-like organisms and emerging pathogens such as caliciviruses, microsplorida (septata and enterocytozoan), hepatitis A virus, Mycobacterium avium–intracellulare complex (MAC), Helicobacter pylori, Legionella pneumophila, adenovirus 40/41/1-39, and Toxoplasma gondii
\item Development of efficient, cost-effective treatment processes for removing disinfection byproduct precursors (e.g., trihalomethanes, haloacetic acids), for ozonation (bromate, aldehydes), for chlorination (chloropicrin, haloacetonitriles), and for chloramination (organic chloramines, cyanogen chloride)
\item Improved methods for controlling pathogens through coagulation/settling, filtration, or other cost-effective means
\item Drinking water contamination control between the treatment plant and the user, especially considering potential chemical leaching from distribution system materials and surfaces (e.g., lead, copper, iron, and other pipe materials; protective coatings) as a result of instability, interaction with microorganisms, disinfection agents, and water treatment chemicals
\end{itemize}


Key Disinfection Terms\\
Before moving on to a discussion of the major disinfection methods used in treating water for human consumption, it is necessary to first define a few pertinent terms related to disinfection in general. To begin with, let’s establish the distinction between primary and secondary disinfection:
\begin{itemize}
\item Primary disinfection—Initial killing of Giardia cysts, bacteria, and viruses
\item Secondary disinfection—Maintenance of a disinfectant residual that prevents regrowth of microorganisms in the water distribution system between treatment and consumer
\end{itemize}
Other terms the reader should understand include
\begin{itemize}
\item Disinfection—Inactivation of virtually all recognized pathogenic microorganisms, but not necessarily all microbial life (which would be considering pasteurization or sterilization).
\item Disinfectant—(1) Any oxidant, including but not limited to, chlorine, chlorine dioxide, chloramine, and ozone, added to water in any part of the treatment or distribution process that is intended to kill or inactivate pathogenic microorganisms. (2) A chemical or physical process that kills pathogenic organisms in water; chlorine is often used to disinfect sewage treatment effluent, water supplies, wells, and swimming pools.
\item Disinfectant time—The time required for water to move from one point of disinfectant application (or the previous point of residual disinfectant measurement) to a point before or at the point where the residual disinfectant is measured.
\item Disinfectant contact time (T in C*T calculation)—The time (in minutes) required for water to move from the point of disinfectant application or the previous point of disinfection residual measurement to a point before or at
the point where residual disinfectant concentration (C) is measured. Where only one C is measured, T is the time (in minutes) required for water to move from the point of disinfectant application to a point before or at where residual disinfectant concentration (C) is measured. Where more than one C is measured, T is defined as follows:
\item For the first measurement of C, the time (in minutes) required for water to move from the first or only point of disinfectant application to a point before or at the point where the first C is measured
\item For subsequent measurements of C, the time in minutes that water takes to move from the previous C measurement point to the C measurement point for which the particular T is being calculated
\item Disinfection byproduct—A compound formed by the reaction of a disinfectant such as chlorine with organic material in the water supply.
\item Presence or absence of coliforms—Presence of coliform bacteria in water is an indication that the water may be contaminated by pathogenic organisms. Absence of coliform bacteria is considered to be sufficient evidence that pathogens are absent—if the source is good, a chlorine residual level is maintained and the supply has a good history.
\item Sterilization—The destruction of all microorganisms. Sterilizing potable water requires the application of a much higher dose of chemical disinfectants, which would greatly increase operating costs and would create taste problems for the consumer. Excessive application of disinfectants also generates excessive levels of unwanted disinfection byproducts. For these reasons, current treatment practices are used for turbidity removal and subsequent disinfection to the extent necessary to eliminate known diseasecausing organisms sufficient to protect public health.
Note: Sterilization should not be confused with disinfection.
\item Waterborne disease—Caused by pathogenic organisms in water.
\end{itemize}
Disinfection Methods\\
Although chlorination is the best known and the most common disinfection method, other methods are available and can be used in various situations. The three general types of disinfection are
\begin{itemize}
\item Heat treatment—Probably one of the first methods employed to disinfect water was to boil it. For small quantities of water, boiling water is still a good emergency procedure to use.
\item Radiation treatment—Uses ultraviolet radiation to disinfect water.
\item Chemical treatment—Employs the use of chemicals to disinfect water. Examples of chemical disinfectants include oxidizing agents such as chlorine, ozone, bromine, iodine, and potassium permanganate; metal ions such as silver, copper, and mercury; and acids and alkalis.
\end{itemize}
Obviously, several different disinfectants are available for use in treating water, and several of these are discussed in detail in subsequent sections. For now, it is important to understand that, even though several choices are available, whichever disinfectant is chosen must meet certain criteria—more specifically, the disinfectant chosen must be effective for disinfecting water (and wastewater) and must possess certain desirable characteristics.\\
Desirable Characteristics of a Disinfectant:
\begin{enumerate}
\item It must act in a reasonable time.
\item It must act as temperature or pH changes.
\item It must be nontoxic.
\item It must not add unpleasant taste or odor.
\item It must be readily available.
\item It must be safe and easy to handle and apply.
\item It must be easy to determine the concentration of.
\item It must be able to provide residual protection.
\item Pathogenic organisms must be more sensitive to the disinfectant than are nonpathogens.
\item It must be capable of being applied continually.
\item The amount applied must be sufficient to produce a safe water.
\end{enumerate}
In addition to the desirable characteristics of a disinfectant listed above, the disinfectant chosen must be able to kill off or deactivate pathogenic microorganisms by one of several possible methods, including: (1) damaging the cell wall, (2) altering the ability to pass food and waste through the cell membrane, (3) altering the cell protoplasm,
(4) inhibiting the cells’ conversion of food to energy, or (5) inhibiting reproduction.\\

Chlorination\\
For the past several decades, chlorine dispensed as a solid (calcium hypochlorite), liquid (sodium hypochlorite), or gas (elemental chlorine, Cl2) has been the disinfectant of choice, particularly in the United States. Chlorine (sometimes referred to as the workhorse of disinfection) has proven its worth both because of its effectiveness and because it is relatively inexpensive; it also provides a chlorine residual in the water distribution system, ensuring that the water remains disease free.\\
Gaseous chlorine (Cl2), 2.5 times as heavy as air, is a greenish-yellow toxic gas. One volume of liquid chlorine confined in a container under pressure yields about 450 volumes of gas. Large water treatment works usually use chlorine gas, supplied in liquid form, in high-strength, high-pressure steel cylinders. The liquid immediately vaporizes in the form of gas when released from these pressurized containers. Chlorine gas is lethal at concentrations as low as 0.1% air by volume. In nonlethal concentrations, it irritates the eyes, nasal membranes, and respiratory tract.\\
Sodium hypochlorite is most commonly used in smaller systems, because it is simpler to use and has less extensive safety requirements than gaseous chlorine; in the form used, it is less toxic. Recently, many larger water facilities that have used chlorine for disinfection are beginning to substitute sodium hypochlorite for chlorine because of regulatory pressure.
Note: The Occupational Safety and Health Administration’s Process Safety Management Standard (29 CFR 1910.119) and USEPA’s Risk Management Program (Clean Air Act, Section 112(r)(7)) have come to be known in the industry as the “chlorine killers,” because of their effect on industrial processes. The USEPA is attempting to steer industry away from the use of chlorine. Although the Agency cannot absolutely outlaw this substance from use, it is following the path of simply regulating it to death. In an effort to avoid having to comply with strict (in some cases, unworkable) regulations, many water treatment and wastewater facilities in the United States are substituting some other chemical product that is not regulated (at least for the moment) such as sodium hypochlorite. Sodium hypochlorite provides 5 to 15% available chlorine (common laundry bleach is a 5% solution of sodium hypochlorite). Usually diluted with water before application as a disinfectant, it is very corrosive and should be handled and stored with care and kept away from equipment that can be damaged by corrosion. Sodium hypochlorite solution is more costly per pound of available chlorine and does not provide the same level of protection of chlorine gas.\\
Calcium hypochlorite is a white solid in granular, powdered, or tablet form containing 65% available chlorine. In packaged form, calcium hypochlorite is stable— more stable than solutions of sodium hypochlorite, which deteriorate over time; however, calcium hypochlorite is hygroscopic, which means it readily absorbs moisture. It reacts slowly with moisture in the air to form chlorine gas. It is a corrosive material with a strong odor and requires proper handling. Some practical difficulty is involved in dissolving calcium hypochlorite. It must be kept away from organic materials such as wood, cloth, and petroleum products. Reactions between it and organic materials can generate enough heat to cause a fire or explosion.\\

Chlorine Use\\
Whatever form of chlorine is used for disinfection (elemental chlorine, sodium hypochlorite, or calcium hypochlorite), it may be added to the incoming flow (prechlorination) to assist with the oxidation of inorganics or to arrest biological action that may produce undesirable gases in the bottom of clarifiers. More often, however, chlorine is added just prior to filtration to keep algae from growing at the medium surface and to prevent large populations of bacteria from developing within the filter medium. Safe and effective application of chlorine requires specialized equipment and considerable care and skill on the part of the plant operator. Various means of feeding chlorine have been developed, but probably one of the widest used and safest types of chlorine feed devices is the all-vacuum chlorinator. Mounted directly on the chlorine cylinder, the gaseous chlorine is always under a partial vacuum in the line that carries it to the point of application. In a typical vacuum chlorine feed system, the vacuum is formed by water flowing through the ejector unit at high velocity.\\
Hypochlorites are usually applied to water in liquid form by means of positive displacement-type pumps, which deliver a specific amount of liquid on each stroke of a piston or flexible diaphragm. Chlorine, when added to water, reacts with various substances or impurities in the water (e.g., organic materials, sulfides, ferrous iron, nitrites), which creates a chlorine demand. Chlorine demand is a measure of the amount of chlorine that will combine with impurities and is therefore available to act as a disinfectant. Chlorine combines with ammonia or other nitrogen compounds to form chlorine compounds that have some disinfectant properties. These compounds are called combined available chlorine residual. In the context used here, “available” means available to act as a disinfectant. The uncombined chlorine that remains in the water after combined residual is formed is called free available chlorine residual. Free chlorine is a much more effective disinfectant than combined chlorine.\\

Factors Affecting Successful Chlorination\\
The factors important to successful chlorination are:
\begin{itemize}
\item Concentration of free chlorine
\item Contact time
\item Temperature
\item pH
\item Turbidity
\end{itemize}
The effectiveness of chlorination is directly related to the contact time with and concentration of free available chlorine. At lower chlorine concentrations, contact times must be increased. Maintaining a lower pH will also increase the effectiveness of disinfection. The higher the temperature, the faster the disinfection rate. Chlorine (or any other disinfectant for that matter) is effective only if it comes into contact with the organisms to be killed. Good contact between chlorine and microorganisms is prevented whenever high turbidity levels exist. For this and aesthetic reasons, turbidity should be reduced where necessary through the coagulation and sedimentation methods previously discussed.\\
Chlorination Byproducts\\
A serious disadvantage of chlorination is the potential formation of byproducts. Chlorine, for example, can mix with the organic compounds in water (such as decaying vegetation) to form trihalomethanes (THMs). One THM, chloroform, is a suspected carcinogen. Other common trihalomethanes are similar to chloroform and may cause cancer.
At the present time, about 90% of U.S. water utilities use chlorine to disinfect water. Although chlorine has virtually eliminated the risks of waterborne disease such as typhoid fever, cholera, and dysentery, recent studies have shown risks associated with byproducts of chlorine—a reason why water utilities already have been looking at alternative methods for disinfecting water.
Several approaches for reducing harmful chlorination byproducts have been used. For example, one approach is to remove more of the organics before any chlorination takes place. This can be accomplished (to a degree) by not chlorinating the incoming
raw water before coagulation and filtration, thus reducing the formation of THMs. Aeration or adsorption on activated carbon will remove organic materials at higher concentrations or those not removed by other techniques. Another approach is to reevaluate the amount of chlorine used—the same degree of disinfection might be possible with lower chlorine dosages. Changing the point in treatment where chlorine is added is another approach commonly employed; rather than adding chlorine as chemical feed during coagulation, sedimentation, or filtration, it can instead be added after filtration. Another current approach is using alternative disinfection methods.\\
Note: Because of OSHA’s Process Safety Management (PSM) standard and USEPA’s Risk Management Program (RMP), many facilities currently using elemental chlorine have used or are actively pursuing the use of alternative disinfection methods. We further reemphasize that the problem of THMs is also helping spur interest in alternatives to chlorination as the preferred method of disinfection.\\

Alternative Disinfection Methods\\
Currently, several alternative disinfection methods are available for use in treating water, but the following discussion focuses on two of these alternatives: ozonation and ultraviolet (UV) radiation. These commonly used alternatives (especially in small water treatment systems) are also increasingly being substituted for existing chlorination systems at larger plants because of regulatory pressure.
Note: Before discussing the ozonation and ultraviolet disinfection alternatives, it is important to point out that neither one of these two alternative disinfectants is an easy solution to problems created by chlorination. It is true that each has the advantages of not creating THMs and not being covered by the requirements under the PSM standard and RMP, but each has uncertainties and known disadvantages that have restricted their more widespread use. In addition, ozonation and ultraviolet irradiation cannot be used as disinfectants by themselves. Both require secondary disinfectant (usually chlorine) to maintain a residual in the distribution system.\\
Ozonation\\
Ozone (O3), a gas at ordinary temperature and pressures, is a very powerful disinfectant that breaks up molecules in water; it is even more effective against some viruses and cysts than chlorine. It has the added advantage of leaving no taste or odor and is unaffected by pH or the ammonia content of the water. When ozone reacts with reduced inorganic compounds and with organic material, an oxygen atom instead of a chloride atom is added to the organics, the end result being an environmentally acceptable compound. But, because ozone is unstable and cannot be stored, it must be produced onsite. Ozonation usually costs more than chlorination.\\
Ultraviolet\\
Ultraviolet (UV) light is electromagnetic radiation just beyond the blue end of the light spectrum, outside the range of visible light. It has a much higher energy level than visible light, and in large doses it inactivates both bacteria and viruses. UV energy is absorbed by genetic material in the microorganisms, interfering with their ability to reproduce and survive, as long as the radiation contacts the microorganisms without interference from turbidity. The big advantage of UV disinfection over chlorine and
ozone is that UV does not involve chemical use. Generally, UV light used for disinfecting water is generated by a series of submerged, low-pressure mercury lamps. Continuing advances in UV germicidal lamp technology are making UV disinfection a more reliable and economical option for disinfection in many plants.\\

\textbf{NONCONVENTIONAL WATER TREATMENT TECHNOLOGIES}
Stage 1 of the USEPA’s Disinfectants and Disinfection Byproduct Rule and the Interim Enhanced Surface Water Treatment Rule, designed to significantly lower THM byproducts of chlorine disinfection in water, has driven (along with the regulatory requirements of the PSM standard and RMP) many water and wastewater treatment utilities to find and use alternative disinfection methodologies. Although ozonation and ultraviolet irradiation might be suitable disinfection alternatives, switching from chlorine to chlorine dioxide (a chemical that has been proven to form fewer THMs) might also be another viable disinfection alternative. Whichever disinfection alternative is ultimately selected, remember that the selection is driven not only by regulatory requirements but also by site-specific requirements.
The disinfection issues covered to this point are important—the overall ramifications of regulatory pressure and environmental impact cannot be overstated—but other issues besides disinfection must be considered when deciding which water treatment methodology to employ. Most of the time, clarification by coagulation, flocculation, sedimentation, and filtration removes suspended impurities and turbidity from drinking water, and disinfection (the final step in the process) produces potable water, free of harmful pathogens. Simply put, the water treatment processes discussed in the previous sections of this part of the text are usually sufficient to render most natural surface water (such as a river) potable. In some instances, however, the water supply may contain materials that are not removed by conventional water treatment processes, and other treatment processes may be required to remove many of the dissolved organic and inorganic substances. Examples would include groundwater with excessive dissolved solids and surface waters containing organic compounds from domestic or industrial wastewaters or organics occurring naturally such as humic acid or products of algae blooms. Additional processes are available for removing these contaminants.
Note: These additional water treatment processes involve sophisticated equipment and require highly skilled operators; therefore, they are quite expensive (Peavy et al., 1985).
Additional water unit treatment processes may be used in addition to clarification or applied separately, depending on the source and quality of the raw water. Let’s take a closer look at groundwater. The question is—does a typical groundwater source require treatment beyond conventional means? The answer is that groundwater does not normally require processing by the unit treatment steps listed above, other than disinfection, because groundwater is filtered naturally by the layers of soil from which it is withdrawn. Disinfection is only applied (in many cases) as a precautionary step required by law for public water systems. Groundwater is usually free
of bacteria or other microorganisms; however, that all groundwater comes into contact with soil and rock is a cause for concern. With such contact, groundwater may become contaminated by high levels of dissolved minerals that must be removed.

\textbf{FLUORIDATION}\\
Fluoride, when added to drinking water supplies in small concentrations (about 1.0 mg/L), can be beneficial. In some locations, common practice is to mix a 4% solution of sodium fluoride and feed that into the flow of the water system. The amount that is fed depends on the air temperature and on the fluoride levels in the raw water. Experience has shown that drinking water containing a proper amount of fluoride can reduce tooth decay by 65% in children. Fluoride combines chemically with tooth enamel when permanent teeth are forming, and the result is teeth that are harder, stronger, and more resistant to decay. The USEPA sets the upper limits for fluoride in drinking water supplies based on ambient temperatures; for example, people drink more water in warmer climates, so fluoride concentrations should be lower in these areas.\\

\textbf{WATER TREATMENT OF ORGANIC AND INORGANIC CONTAMINANTS}\\
Manmade compounds that contain carbon—synthetic organic chemicals (SOCs)— are, from time to time, detected in U.S. water supplies. Some of these are volatile organic chemicals (VOCs), such as the solvent trichloroethylene. The problem with VOCs in a water supply (i.e., any water supply used by the public) is twofold. They are easily absorbed through the skin and they volatize into gases that can then be inhaled by those taking a shower or a bath or while washing dishes. How do water supplies become contaminated by organic compounds? Good question. Basically, sources of organic contaminants are usually improperly disposed wastes, pesticides, industrial effluents, and leaking fuel oil tanks (gasoline in particular).
Water supplies may also contain inorganic contaminants consisting mainly of substances occurring naturally in the ground, such as sulfate, fluoride, arsenic, barium, radium, selenium, and radon. Metallic substances from industrial sources can contaminate surface waters. The inorganic ion nitrate (from fertilizers and feedlot runoff in agricultural areas) occurs frequently in groundwater supplies. Another source of inorganic chemical contamination in drinking water supplies is corrosion or deterioration of water supply equipment, such as plumbing systems, which release metal and nonmetal substances into the water, including lead, cadmium, zinc, copper, iron, and plumbing cement. Inorganic contaminants can be treated by corrosion controls and removal techniques. Corrosion controls reduce the presence of corrosion byproducts (e.g., lead) at the consumer’s tap. Removal technologies, coagulation and filtration, reverse osmosis, and ion exchange are used to treat source water that iscontaminated with metals or radioactive substances. The following sections discuss processes for removing inorganic and organic dissolved solids from water intended for potable use. Keep in mind that (with some modifications) these same processes may act as tertiary treatment for wastewater.\\
 

\textbf{Aeration}\\
Aeration (air stripping) is a physical treatment process in which air is thoroughly mixed with water—a technique effective for removing dissolved gases and highly volatile odorous compounds. Contact with air and oxygen can improve water quality in a number of ways. When aeration is a first step in processing well water, for example, it may achieve any or all of the following: removal of hydrogen sulfide, reduction of dissolved carbon dioxide, and addition of dissolved oxygen for oxidation of iron and manganese (the oxygen in the air reacts with the iron and manganese to form an insoluble precipitate—rust). One of the most common uses of aeration is for taste and odor control. Sedimentation and filtration are then necessary to clarify the water.
Note: Aeration is rarely effective in processing surface waters, simply because the odor-producing substances are generally nonvolatile.\\
Several methods to aerate the water are available. The method selected depends primarily on the type and concentration of material to be removed from the water and on the available pressure. Aeration in water treatment can be accomplished using spray nozzles, cascade systems, multiple-tray aerators, diffused-air aerators, and mechanical aerators.\\


\textbf{Oxidation}\\
Simply stated, oxidation is a reaction in which a substance loses electrons, thus increasing its charge. A substance that oxidizes another is referred to as an oxidizing agent or oxidizer. In water treatment, oxidation is used to remove or destroy undesirable tastes or odors, to aid in removal of iron and manganese, and to help improve clarification and color removal in source water. Chlorine dioxide, potassium permanganate, and ozone are strong oxidants capable of destroying many odorous compounds. Because they do not produce THMs, these chemicals are favored over heavy chlorination.
Note: Atmospheric oxygen, through aeration, can be used to oxidize the organic substances responsible for undesirable tastes and odors, but the process is usually too slow to be of value. If dissolved gases such as hydrogen sulfide are the cause of taste and odor problems, aeration will effectively remove them through oxidation and stripping.

\textbf{Adsorption}\\
When we speak of adsorption, we are referring primarily to a surface phenomenon—the adsorption that results when one substance attaches itself to the surface of another. The two most common adsorptive media used in water treatment are activated carbon and activated alumina. These adsorptive materials are generally most effective for taste and odor control and for removal of organic pollutants; however, the most important applications of adsorption in water treatment are the removal of arsenic and organic pollutants.
Adsorption of organic materials using activated carbon has been a common practice in water treatment for many years. Activated carbon is manufactured from carbonaceous material such as wood, coal, and petroleum residues. A char is made by burning the material in the absence of air, and it is then oxidized at higher temperatures to create a very porous structure. This activation step provides irregular channels and pores in the solid mass, resulting in a very large surface-area-to-mass ratio. This large surface area gives activated carbon its effectiveness as an adsorbing agent. The larger the surface area of an adsorber, the greater its power. Each activated carbon contains a huge number of pores and crevices into which organic molecules enter and are adsorbed onto the activated carbon surface.
Activated carbon has a particularly strong attraction for organic molecules such as the aromatic solvents benzene, toluene, and nitrobenzene; the chlorinated aromatics polychlorinated biphenyls (PCBs), chlorobenzenes, and chloronaphthalene; phenol and chlorophenols; the polynuclear aromatics acenaphthene and benzopyrenes; pesticides and herbicides; chlorinated aliphatics such as carbon tetrachloride and chloroalkyl ethers; and high-molecular-weight hydrocarbons such as dyes, gasoline, amines, and humics.
Two forms of activated carbon are used in water treatment: powdered and granular. Powdered activated carbon is often used for taste and odor control. Its effectiveness depends on the source of the undesirable tastes and odors. It is also effective in removing the organic precursors that react with chlorine to form harmful THM compounds after disinfection.
Powdered activated carbon is a finely ground, insoluble black powder that can be added at any point in the treatment process ahead of the filters. It is fed to water either as a dry powder or as a wet slurry. Although adsorption is nearly instantaneous, a contact time of 15 minutes or more is desirable before sedimentation or filtration. Activated carbon media must periodically be replaced with new or regenerated activated carbon. Replacement cycles can vary from 1 to 3 years for taste and odor treatment to as little as 4 or 5 weeks for removal of organics. The activated carbon regeneration process involves (1) removing the spent carbon as a slurry, (2) dewatering the slurry, (3) feeding the carbon into a special furnace where regeneration occurs (i.e., the organics are driven from the carbon surface by heat), and (4) returning it to use.
Activated alumina (a highly porous and granular form of aluminum oxide) is also an adsorptive medium used in water treatment. It is used primarily to remove arsenic and excess fluoride ions. Water is percolated through a column of alumina media, and a combination of adsorption and ion exchange performs the actual removal of arsenic and fluoride ions. Like the regeneration process used to restore used activated carbon to full potency, activated alumina also requires periodic regeneration, accomplished by passing a caustic soda solution through the media. Excess caustic soda is neutralized by rinsing the activated alumina with an acid. Disposal of these wash waters, laden with toxic arsenic and fluoride ions, must be done in accordance with applicable laws.
Note: Powdered activated carbon is much more difficult to regenerate than granular activated carbon. Granular activated carbon is sometimes used in the filter bed itself, combining both filtration and adsorption in one treatment unit. The major problem associated with granular activated carbon systems is suspended solids in the water plugging up the bed.\\
 

\textbf{Demineralization}\\
Demineralization refers to the removal of dissolved solids (inorganic mineral substances) from water. Dissolved solids contain both cations and anions and therefore require two types of ion exchange resins. Cation exchange resins used for demineralization purposes have hydrogen exchange sites and are divided into strong acid and weak acid classes. The anion exchange resins commonly used contain hydroxide ions and are divided into strong and weak base classes. Demineralization is commonly used in industry in waste treatment for removal of arsenic, barium, cadmium, chromium, fluoride, sulfate, and zinc. Some general advantages of using ion exchange to remove these contaminants are the low capital investment required and the mechanical simplicity of the process. In addition, the ion exchange process can be used to recover valuable chemicals for reuse, or harmful ones for disposal. For example, it is often used to recover chromic acid from metal finishing waste for reuse in chrome-plating baths. It also has some application in the removal of radioactivity. The major disadvantages are the high chemical requirements needed to regenerate the resins and to dispose of chemical wastes from the regeneration process. These factors make ion exchange more suitable for small systems than for large ones.\\

\textbf{Membrane processes}\\
Membrane processes used in water treatment are primarily demineralization processes. Demineralization of water can be accomplished using thin, microporous membranes. Electrodialysis and reverse osmosis are the most common membrane processes. Before we briefly discuss these two membrane processes, you need a basic understanding of osmosis. During osmosis, two solutions containing different concentrations of minerals are separated by a semipermeable membrane. Water tends to migrate through the membrane from the side of the more dilute solution to the side of the more concentrated solution. This is osmosis, and it continues until the build-up of hydrostatic pressure on the more concentrated solution is sufficient to stop the net flow. In reverse osmosis, the flow of water through the semipermeable membrane is reversed by applying external pressure to offset the hydrostatic pressure. This results in a concentration of minerals on one side of the membrane and pure water on the other side. Reverse osmosis can treat for a wide variety of health and aesthetic contaminants in water. Effectively designed, reverse osmosis equipment can treat aesthetic contaminants that cause unpleasant taste, color, and odor problems, such as a salty or soda taste caused by chlorides or sulfates. Reverse osmosis can also be effective for treating arsenic, asbestos, atrazine, fluoride, lead, mercury, nitrate, and radium. When used with appropriate carbon prefiltering, additional treatment can also be provided for such “volatile” contaminants as benzene, trichloroethylene, trihalomethanes, and radon. Some reverse osmosis equipment is also capable of treating for Cryptosporidium. Reverse osmosis can be expected to play a major role in water
treatment for years to come. Reverse osmosis (also called ultrafiltration) is the most common process for reducing the salinity of brackish groundwater. In operation, a semipermeable membrane (the most essential element in the reverse osmosis method of demineralization)
separates salty water of two different concentrations. Concentrations have a natural tendency to become equalized by a flow of water from the dilute side to the concentrated side (osmosis). But high pressure applied to the high concentration side of the membrane can reverse this direction of flow. Freshwater diffuses through the membrane, leaving a more concentrated salt solution behind. The performance of reverse osmosis units is highly dependent on a number of water quality parameters. Suspended solids, dissolved organics, hydrogen sulfide, iron, and strong oxidizing agents (chlorine, ozone, and permanganate) are harmful to membranes.
Electrodialysis is the demineralization of water using the principles of osmosis—but it uses ion-selective membranes and an electric field to separate anions and cations in solution. In the past, electrodialysis was most often used for purifying brackish water, but it is now finding a role in industrial waste treatment as well. For example, metals salts from plating rinses are sometimes removed in this way.

\textbf{A PARADIGM SHIFT IN PROGRESS}\\
Water shortage is the lack of adequate accessible water resources to meet water needs within a locality. More than 1.2 billion people lack access to clean drinking water (United Nations, 2017). For localities where access to drinking water is readily available, an issue that is not necessarily recognized at this time is the one-and-done scenario discussed earlier—that is, safe drinking water quality water is drawn from a tap and used for a variety of purposes and that is that. After being used, this water is poured down drains or flushed down toilets—out of sight and out of mind. But, a significant paradigm shift is beginning to occur. The idea of toilet-to-tap reuse is not palatable to many people, but we need water. We cannot live without water. Fortunately, we can clean used water and reuse it, a task that Mother Nature often can do for us naturally. We have no other choice. Regions where water is readily accessible today may not be able to brag about that in the future. Population growth, overuse, misuse, abuse, and other events and actions affect water use and have detrimental impacts on water quality. We need to change the one-and-done scenario to a one-and-redone scenario by using technology to purify used water. The use of advanced treatment and purification of used water (wastewater) to drinking water quality is a paradigm change in progress.\\


\textbf{ADVANCED TREATMENT OF WASTEWATER TO DRINKING WATER QUALITY}\\
Advanced technologies and processes used for wastewater treatment and purification provided at indirect potable reuse (IPR) plants varies (see Figure 11.3) but are typically focused on providing multiple barriers for the removal of pathogens and organics. Nitrogen and TDS removal is provided at some locations where necessary. Table 11.2 shows most of the indirect potable reuse projects that have been implemented in the United States. The table has been sorted according to the type of potable reuse application (i.e., direct aquifer injection, aquifer recharge with surface spreading, and surface water augmentation). The first five projects shown in this table are direct injection
projects that match the proposed HRSD concept. Water extracted from direct injection and surface spreading projects that recharge groundwater is not typically treated again prior to distribution into the potable water system; however, water from surface water augmentation projects is typically treated again at water treatment plants because of water treatment requirements stipulated by the USEPA’s Surface Water Treatment Rule (SWTR). For example, Fairfax County’s Griffith Water Treatment Plant provides coagulation, sedimentation, ozone oxidation, biological activated carbon filtration, and chlorine disinfection for water extracted from the Occoquan Reservoir that is augmented by the Upper Occoquan Service Authority’s indirect potable reuse plant.
As shown in Table 11.2, the treatment provided for indirect potable reuse projects is typically a combination of multiple barriers for the removal of pathogens and organics. Multiple barriers for pathogens are typically provided through a combination of coagulation, flocculation, sedimentation, lime clarification, filtration (granular or membrane), and disinfection (chlorine, ultraviolet, or ozone). Multiple barriers for organics removal are typically provided through a combination of advanced
treatment processes (e.g., reverse osmosis, granular activated carbon, ozone in combination biological activated carbon), although conventional treatment processes (e.g., coagulation, softening) also provide removal at some locations. All potable reuse plants listed in Table 11.2 include a robust organics removal process of granular activated carbon (GAC), granular media filtration (GMF), biological activated carbon (BAC), reverse osmosis (RO), microfiltration (MF), ultraviolet advanced oxidation process (UVAOP), membrane bioreactor (MBR), or soil aquifer treatment (SAT), which are effective barriers to bulk and trace organics and represent the backbone of the potable treatment process. SAT is the controlled application of wastewater to earthen basins in permeable soils at a rate typically measured in terms of meters of liquid per week. The purpose of a soil aquifer treatment system is to provide a receiver aquifer capable of accepting liquid intended to recharge shallow groundwater, and system design and operating criteria are developed to achieve that goal. However, there are several alternatives with respect to the utilization or final fate of the treated water (USEPA, 2006):\\
\begin{itemize}
\item Groundwater recharge
\item Recovery of treated water for subsequent reuse or discharge
\item Recharge of adjacent surface streams
\item Seasonal storage of treated water beneath the site with seasonal recovery for agriculture
\end{itemize}
The SAT process typically includes application of the reclaimed water using spreading basins and subsequent percolation through the vadose zone. SAT provides significant removal of both pathogens and organics through biological activity and natural filtration. However, because some aquifers are confined, it is not possible to utilize the SAT for treatment through the vadose zone to recharge them. On the other hand, movement of reclaimed water through the aquifer after direct injection will provide significant treatment benefits, including excellent removal of pathogens. Advanced water treatment plants based on reverse osmosis and granular activated carbon are often utilized at locations where SAT treatment through the vadose zone is not feasible, because it is possible for these processes to be implemented at most locations.\\

\begin{table}[]
\begin{tabular}{lllll}
Project                                                            & Type of Potable Reuse Application                              & Year & Capacity (mgd) & Advanced Treatment Processes                                       \\
Hueco Bolton Recharge Project; El Paso, TX                         & Groundwater recharge via direct injection and spreading basins & 1985 & 10             & Lime + GMF + ozone + BAC + C                                       \\
West Basin Water Recycling Plant; Carson, CA                       & Groundwater recharge via direct injection                      & 1993 & 12.5           & MF + RO + UVAOP                                                    \\
Scottsdale Water Campus; Scottsdale, AZ                            & Groundwater recharge via direct injection                      & 1999 & 20             & MF + RO + Cl                                                      \\
Los Alamitos Seawater Intrusion Barrier; Long Beach, CA            & Groundwater recharge via direct injection                      & 2006 & 3              & MF + RO + UV disinfection                                          \\
Groundwater Replenishment                                          & Groundwater recharge via direct injection and spreading basins & 2008 & 70             & MF + RO + UVAOP + SAT (spreading basins for a portion of the flow) \\
Montebello Forebay, Groundwater Recharge District, Los Angeles, CA & Groundwater recharge via spreading basins                      & 1962 & 44             & GMF + C                                                            \\
Chino Basin Groundwater Recharge Project; Chico, CA                & Groundwater recharge via spreading basins                      & 2007 & 18             & GMF + O$_2$ + SAT (spreading basins)                                 
\\
Hueco Bolton Recharge Project; El Paso, TX                         & Groundwater recharge via direct injection and spreading basins & 1985 & 10             & Lime + GMF + ozone + BAC + C                                       \\
West Basin Water Recycling Plant; Carson, CA                       & Groundwater recharge via direct injection                      & 1993 & 12.5           & MF + RO + UVAOP                                                    \\
Scottsdale Water Campus; Scottsdale, AZ                            & Groundwater recharge via direct injection                      & 1999 & 20             & MF + RO + Cl$_2$ \\
Los Alamitos Seawater Intrusion Barrier; Long Beach, CA            & Groundwater recharge via direct injection                      & 2006 & 3              & MF + RO + UV disinfection                                          \\
Groundwater Replenishment                                          & Groundwater recharge via direct injection and spreading basins & 2008 & 70             & MF + RO + UVAOP + SAT (spreading basins for a portion of the flow) \\
Montebello Forebay, Groundwater Recharge District, Los Angeles, CA & Groundwater recharge via spreading basins                      & 1962 & 44             & GMF + C                                                            \\
Chino Basin Groundwater Recharge Project; Chico, CA                & Groundwater recharge via spreading basins                      & 2007 & 18             & GMF + O$_2$ + SAT (spreading basins)
\end{tabular}
\end{table}

\textbf{CONVENTIONAL WATER TREATMENT}\\
A typical water treatment plant treats stream or river water (turbid surface water with organics) and processes the raw water using various unit processes, including:\\
(1) screening, (2) coagulation, (3) flocculation, (4) sedimentation or settling, (5) filtra-
tion, (6) hardness treatment, (7) disinfection, and (8) fluoridation (see Figure 11.2). This chapter provides a brief overview of each of these unit processes, which constitute a typical drinking water treatment system for surface water supplies, in addition to a brief discussion of alternative approaches.\\

\textbf{Screening}
Screening (the first important step in treating water containing large solids) is defined as the process whereby relatively large and suspended debris is removed from the water before it enters the plant. River water (the source of water used in our discussion) frequently contains suspended and floating debris varying in size from small rags to logs. Removing these solids is important, not only because these items have no place in potable water but also because this river trash may cause damage to downstream equipment (e.g., clog and damage pumps), may increase chemical requirements, may impede hydraulic flow in open channels or pipes, or may hinder the treatment process (Pankratz, 1995). The most important criteria used in the selection of a particular screening system for water treatment technology are the screen opening size and flow rate. Other important criteria include costs related to operation and equipment, plant hydraulics, debris handling requirements, and operator qualifications and availability. Large surface water treatment plants may employ a variety of screening devices, such as trash screens (or trash rake), traveling water screens, drum screens, bar screens, or passive intake screens. Each of these screening devices is briefly discussed in the following sections.\\
 

Trash Screens (Rakes)\\
Trash screens or trash rakes are used to remove rough or large debris retained on a trash rack. They protect pumping equipment and may be used as a preliminary screening device to protect finer screens—drum or traveling water screens, for example. A trash screen consists of one or more stationary trash rakes and a screen raking device. Trash rack bar spacings range from 1.5 to 4 inches and are mostly constructed of steel bars. Those constructed of high-density polyethylene polymers are beginning to replace many of the older steel bar models—these synthetic screens are lighter and less prone to microbial growth, corrosion, and ice. Raking mechanisms are available for use in a variety of intake configurations, including installation on vertical building and dam walls. Rakes are typically mounted on fixed structures designed to clean a single trash rack, suspended from an overhead gantry, wheelmounted to traverse the entire width of an intake structure and clean individual sections of a wide trash rack, or suspended from an overhead gantry.\\
Traveling Water Screens\\
Traveling water screens (sometimes called bandscreens) are placed in a channel of flowing water to remove floating or suspended debris. These automatically cleaned screening devices protect pumping or other downstream equipment from debris in surface water intakes. Consisting of a continuous series of wire mesh panels bolted to basket frames, or trays, and attached to two matched strands of roller chain, the traveling water screen operates in a vertical path over a sprocket assembly through the flow. As raw water passes through the revolving baskets, debris is collected and retained on the upstream face of the wire mesh panels. The debris-laden baskets are lifted out of the flow and above the operating flow, where a high-pressure water spray directed outward removes the impinged debris. This process can be continuous or intermittent. For intermittent operation, the screen activates when a specified headloss or time elapsed has occurred. When located on a river, traveling water screens may be subject to large fluctuations in flow conditions, debris loading, water depths, and salinity. Depending upon application, the size of the traveling water screen is determined by considering such factors as maximum and average flow; maximum, minimum, and average water levels; wire mesh size; velocity through mesh; basket or channel width; number of screens; type of service; and/or starting and operating headloss requirements.\\
Drum Screens\\
A drum screen (or cylinder screen) has very few moving parts and is mounted on a horizontal axis with a series of wire mesh panels attached or mounted on the periphery of its cylinder. The cylinder slowly rotates on its axis. Because of its simplicity of construction, the maintenance requirements and operating costs of a drum screen installation are usually lower than those of a traveling water screen.\\
Bar Screens\\
Primarily used in wastewater treatment applications, bar screens are also employed in some water treatment facilities. A bar screen consists of straight steel bars welded at both ends to two horizontal steel members and is automatically cleaned by one
or more power operated rakes. As a rake is operated up the face of the bar rack, it removes accumulated debris (usually large solid objects and rags) and elevates in and out of the flow. At the top of the operating cycle of the rake, the debris is swept from the rake into a debris receptacle by a wiper mechanism. When installed in a waterway, the bar screen assembly normally is placed at an angle of 60 to 80 degrees from the horizontal.\\
Passive Intake Screens\\
Passive intake screens (stationary screening cylinders) have no moving parts and require no debris handling or debris removal equipment. Passive intake screens are placed in a surface water body in such a manner so as to take advantage of natural ambient currents and controlled through-screen velocities to minimize debris buildup. Usually mounted on a horizontal axis and oriented parallel to the natural current flow within the water body, current flow action works to keep the screen clean. Maximum intake velocity of about 0.5 foot per second (fps) is typical and works to minimize debris impingement on the screen surface.\\


\textbf{Coagulation}\\
Coagulation, the second step in water purification, is a unit process that has been used for several years in the treatment of raw water. Basically, coagulation works to settle very fine material of suspended solids.
Note: Chemicals employed for coagulation are expected to be safe for drinking water when used according to the American Water Works Association (AWWA) coagulation standards (e.g., Coagulation, Nos. 42402 to 42407).\\

Coagulants\\
Typically, after screening, raw water is pumped into large settling basins, also known as clarifiers or sedimentation tanks. Within the confines of the settling basin, the screened raw water is allowed to sit for some predetermined time. Although screened, the raw water still contains impurities that may be either dissolved or suspended. The settling basin provides the most convenient way to remove the suspended matter, as it lets the force of gravity do the work. Within the basin, when flow and turbulence are minimal (quiescent conditions), particles more dense than water settle to the bottom of the tank. This process is called sedimentation, and the layer of accumulated solids at the bottom of the tank is called sludge (or biosolids in some wastewater treatment unit processes). The size and density of the suspended particles have a direct bearing on the speed at which they will settle toward the bottom of the basin. The larger or heavier particles will, of course, settle faster than smaller or lighter particles. The forces opposing the downward force of gravity include buoyancy and drag (friction). The particle-settling rate is also affected by the temperature and viscosity of the water.\\
Note: The nature of the sedimentation process also varies with the concentration of suspended solids and their tendency to interact with one another.\\
In the sedimentation process just described, not all suspended solids or particles can be completely removed from water, even when given very long detention times. Very small particles called colloids (e.g., bacteria, fine clays, silts) will not settle out of suspension by gravity without some help. This is where coagulants come into play. If we rapidly mix chemical coagulants in the water and then slowly stir the mixture before allowing sedimentation to occur, the colloidal particles will settle. Colloids or finer particles must be chemically coagulated to produce larger floc that is removable in subsequent settling and filtration.
The coagulation process (along with flocculation) works to neutralize or reduce the natural repelling electrical force of particles in water, keeping them apart and in suspension. Particles in water usually carry a negative electrical charge. Because all of these particles carry this same negative electrical charge, they repel each other—in the same way that like poles of a magnet do. The object of coagulation (and subsequently flocculation) is to turn the small particles into larger flocs, either as precipitates or suspended particles. These flocs are then conditioned for ready removal in subsequent processes. Stated another way, in this text we define coagulation as a method to alter the colloids so they will be able to approach and adhere to each other to form larger floc particles.\\
Types of Coagulants:\\
Two types of coagulants are used in the coagulation process: coagulants and coagulant aids. Generally, the types of coagulants and aids available are defined by the plant process scheme. To determine optimum chemical dosages for treatment, jar tests are normally used.\\
Jar Tests\\
Jar tests are widely used to simulate a full-scale coagulation and flocculation process to determine optimum chemical dosages—the cost-effective dose of a coagulant for the time and intensity of agitation selected. Such tests have been used for many years by the water treatment industry; the test conditions are intended to reflect the normal operation of a chemical treatment facility and allow evaluation of the type and quantity of sludge and physical properties of the floc. The test can be used to:\\
\begin{itemize}
\item Select the most effective chemical.
\item Select the optimum dosage.
\item Determine the value of a flocculant aid and the proper dose.
\end{itemize}
The testing procedure requires a series of samples to be placed in testing jars and mixed at 100 rpm. Varying amounts of the process chemical or specified amounts of several flocculants are added (one volume/sample container). The mix is continued
for one minute. The mixing slows to 30 rpm to provide gentle agitation, then the floc is allowed to settle. The flocculation period and settling process are observed carefully to determine the floc strength, settleability, and clarity of the supernatant liquor (the water that remains above the settled floc). The supernatant can then be tested to determine the efficiency of the chemical addition for removal of total suspended solids (TSS), biochemical oxygen demand (BOD), and phosphorus. The equipment required for the jar test includes a six-position, variable-speed paddle mixer; six 2-quart wide-mouth jars; an interval timer; and assorted glassware, pipettes, graduates, and so forth.\\

Coagulation Chemicals\\
Several different chemicals can be used for coagulation. Commonly used metal coagulants are those based on aluminum (aluminum sulfate) and those based on iron (ferric sulfate). The most common coagulant is aluminum sulfate (alum, Al2(SO4)3). Other common coagulation chemicals are provided in Table 11.1.\\

Coagulant Aids\\
Coagulation problems often occur because of slow-settling precipitates or fragile flocs that are easily fragmented under hydraulic forces in basins and filters (Hammer and Hammer, 1996). A coagulant aid is a chemical added during coagulation to improve coagulation; to build stronger, more settleable floc; to overcome the effect of temperature drops that slow coagulation; to reduce the amount of coagulant needed; and to reduce the amount of sludge produced (AWWA, 1995). Coagulant aids benefit
flocculation by improving the settling qualities and toughness of flocs. Polymers are the most widely used materials. Synthetic polymers are water-soluble, high-molecular-weight organic compounds with multiple electrical charges along a molecular chain of carbon atoms. In drinking water treatment, polymers are extensively used as coagulant aids to build large floc prior to sedimentation and filtration. Other coagulant aids are activated silica, adsorbent weighting agents, and oxidants.\\

Coagulation Process Operation\\
The common coagulation unit process operation involves the addition of coagulant chemicals by rapid mixing—detention time in the rapid mix tank is typically on the order of minutes (Masters, 1991). During this mixing process, polymer (or some other coagulant aid) is added and blended into the destabilized water prior to flocculation. The removal of impurities by coagulation depends on their nature and concentration, the use of both coagulants and coagulants aids, and characteristics of the water, including pH, temperature, and ionic strength. Because of the complex nature of coagulation reactions, chemical treatment is based on empirical data derived from jar testing or other laboratory tests and field studies (Viessman and Hammer, 1998).\\

\textbf{Flocculation}\\
The destabilized particles and chemical precipitates resulting from coagulation are designed to enhance their settling qualities and thus their removal from water; however, even after coagulation has taken place, these particles and chemical precipitates may still settle very slowly (too slowly). To speed up the settling process, flocculation is employed.
Note: Flocculation is the clumping together of the fine particles formed by coagulation. Although the two terms are often used interchangeably, flocculation and coagulation are actually distinct concepts.
Flocculation is the most important factor affecting particle-removal efficiency. In water treatment operations, flocculation is a slow mixing process in which the coagulated particles are brought into contact so they will collide, stick together, and grow (agglomerate) to a size that will readily settle. Enough mixing must be provided (e.g., gentle agitation for approximately half an hour) to bring the floc particles into contact with each other and to keep the floc from settling in the flocculation basin. (The heavier the floc and the higher the suspended solids concentration, the more mixing is required to keep the floc in suspension.) The most common type of mixer or flocculator is the paddle type, which uses redwood slats mounted horizontally on motor-driven shafts. Rotating slowly at about one revolution per minute, the paddles provide gentle agitation that promotes floc growth. The rate of agglomeration or flocculation depends on the number of particles present, the relative volume that they occupy, and the velocity gradient in the basin.  Note: The statement that the rate of agglomeration or flocculation depends on velocity gradient refers to the fact that too much mixing can shear the floc particles, tearing them apart again; the floc then becomes smaller and more finely dispersed, a situation we are obviously trying to avoid. For this reason, the velocity gradient must be controlled within a relatively narrow range.
The theory of flocculation is complex and beyond the needs of this text, but on an elemental level we can say that flocculation is generally accomplished by slowly rotating, large-diameter mixers. Current practice incorporates dispersion of the coagulant (flash mixing), flocculation, and sedimentation in a single unit called a contact clarifier.
Note: Flocculation is the principal mechanism for removing turbidity from water.

\textbf{Sedimentation}\\
In a conventional water treatment plant, the process of coagulation and flocculation precedes the sedimentation process for better results and improved utilization of the settling basins. Sedimentation is then followed by the filtration process. Filtration occasionally may be preceded only by coagulation, in which case filtration is provided after only a few minutes of contact, adding additional stress to the filters. Lack of sedimentation results in less reliable operation of filters when water quality suddenly changes characteristics (DeZuane, 1997).
Sedimentation (also known as clarification) is the gravity-induced removal of particulate matter, chemical floc, precipitates from suspension, and other settleable solids. Simply stated, sedimentation separates the liquid from the solids. The process takes place in a rectangular, square, or round tank called a settling or sedimentation tank or basin. Flow patterns within such basins may be rectilinear flow in rectangular basins, radial flow in center-feed settling tanks or square settling tanks, or radial flow or spiral flow in peripheral-feed settling tanks.
Sedimentation, in the conventional water treatment process, is typically the step between flocculation and filtration. Design criteria are based on empirical data from the performance of full-scale sedimentation tanks. The common criteria for sizing settling basins are detention time (typically from 1 to 10 hr), overflow rate, weir loading, and, with rectangular tanks, horizontal velocity.
In water treatment, the majority of settling basins are essentially upflow clarifiers where the water rises vertically for discharge through effluent channels. More specifically, in the idealized sedimentation tank, water flows horizontally through the basin and then rises vertically, overflowing the weir of a discharge channel at the tank
surface. Floc settles downward, opposite the upflow of water, and is removed from the bottom by a continuous mechanical sludge removal apparatus. The particles with a settling velocity greater than the overflow rate are removed (settled) while lighter flocs are carried out in the effluent. The effluent is then filtered.
Note: Sedimentation tanks, either circular or rectangular, are designed for slow, uniform water movement with a minimum of short-circuiting.\\

textbf{Filtration}\\
Even after chemical coagulation and sedimentation by gravity, not all of the suspended solids or impurities are removed from water. Nonsettleable floc particles (about 5% of the suspended solids) may still remain in the water, and with only that small percentage left we might ask, “Isn’t this good enough?” No, it isn’t. This remaining floc would cause problems (including noticeable turbidity), and particles shield microorganisms from the subsequent disinfection processes. The goal of water treatment is to produce potable water that is perceptually crystal clear and that satisfies the Safe Drinking Water Act (SDWA) requirement of 0.5 NTU for turbidity. To accomplish this, an additional treatment step is required that follows coagulation, flocculation, and sedimentation.
Filtration (sometimes called a polishing process) involves the removal of suspended particles from water by passing it through a layer or bed of a porous granular material—sand, for example. As water flows through the filter bed, the suspended particles become trapped within the pore spaces of the filter material (or filter media). When purifying a surface water source (as in the discussion here), filtration is a very important process, even though filtration is only one step in the overall treatment process.
Note: Filtration is the process that occurs naturally as surface waters migrate (percolate) through the porous layers of soil to recharge groundwater. This natural filtration removes most suspended matter and microorganisms and is the reason why many wells produce water that does not require any further treatment.\\
The Surface Water Treatment Rule (SWTR) specifies certain filtration technologies. The most common treatment filter systems include rapid gravity filters (either built onsite or packaged plants) and pressure filters. Other types include direct filtration, slow sand filters, and diatomaceous earth (DE) filters. The SWTR also allows the use of alternative filtration technologies, such as cartridge filters.\\
Filtration treatment unit processes most commonly used in water purification systems include slow or rapid sand filtration, diatomaceous earth filtration, and package filtration systems. Slow and rapid filter systems refer to the rate of flow per unit of surface area. Filters are also classified by the type of granular material used in them. Sand, anthracite coal, coal–sand, multilayered, mixed bed, and diatomaceous earth are examples of different filtering media. Filtration systems may also be classified by the direction the water flows through the medium: downflow, upflow, fine-tocoarse, coarse-to-fine. Finally, filters are commonly distinguished by whether they are gravity or pressure filters. Gravity filters rely only on the force of gravity to move the water down through the grains and typically use upflow for washing (backwashing) the filter media to remove the collected foreign material. Gravity filters are free surface filters commonly used for municipal applications. Pressure filters are completely enclosed in a shell so most of the water pressure in the lines leading to the filter is not lost and can be used to push the water through the filter.\\

Rapid Filter Systems\\
Slow sand filtration has been used in the United States since 1872. It is still used in many older plants but is not commonly used today in most modern water treatment plants because of various problems associated with this technique. One of the problems is related to the tiny size of the pore spaces in the fine sand, which slows down the water’s progress through the filter bed. These filter types also have problems with suspended particles clogging the surface, requiring the filter to be manually scraped clean. These units take up a considerable amount of land area because slow filtration rates require a greater filter surface area to produce the necessary filtered water qualities.
In modern water treatment plants, the rapid filter has largely replaced the slow sand filter. The rapid filter consists of a layer of carefully sieved silica sand ranging from 0.6 to 0.75 m in depth on top of a bed of graded gravels. The pore openings between the grains of sand are often greater than the size of the floc particles that are to be removed, so much of the filtration is accomplished by means other than simple straining.
Note: The ideal filter medium is coarse enough for large pore openings to retain large quantities of floc, yet sufficiently fine to prevent the passage of suspended solids. It must have adequate depth to allow relatively long filter runs and be graded to permit effective cleaning during backwash.\\
Adsorption, continued flocculation, and sedimentation in the pore spaces are also important removal mechanisms. When the filter becomes clogged with particles (which occurs approximately once a day, depending on the turbidity of the water), the filter is shut down for a short period of time and cleaned by forcing water backward through the sand for 10 to 15 minutes. After cleaning, the sand settles back in place and operation resumes.\\
Other Common Filter Types\\
Rapid flow filters are the most common type used for treating water supplies, primarily because they are the most reliable, but other types of filters are sometimes used to clarify water, including pressure filters and diatomaceous earth filters. A pressure filter is similar to a rapid filter in that the water flows through a granular filter bed; however, instead of being open to the atmosphere and using the force of gravity, the pressure filter is enclosed in a cylindrical steel tank and the water is pumped through the bed under pressure. They are not as reliable as rapid filters, because pressure may force solids through the bed in the effluent. Because of this problem, they are seldom employed in municipal water treatment works but instead are used for filtering water for industrial use or for swimming pools. Diatomaceous earth filters contain a thin layer of a natural, powdery material formed from the shells of diatoms; they are also used primarily for industrial or swimming pool aapplications because they are not as reliable as rapid sand filters.\\

The unit processes described thus far—screening, coagulation, flocculation, sedimentation, and filtration—together comprise a type of treatment called clarification. Along with removing turbidity and suspended solids, clarification also removes many microorganisms from the water; however, clarification by itself is not sufficient to ensure the complete removal of pathogenic bacteria and viruses.
Earlier it was stated that one of the primary goals of water treatment is to treat raw water to the point where it is possible to deliver to the consumer a water product that is perceptually crystal clear. Obviously, the consumer does not want to drink a glass full of mud, a glass full of slime, a glass full of metal-colored, foul-smelling water— or even a glass of water that looks like it was dipped from a creek. Would you? The point is, when the water has been treated to the point of crystal clarity, the treatment process must be taken a step further—to the point where the water is completely free of disease-causing microorganisms. To accomplish this, the final treatment process in water treatment plants occurs—disinfection, which destroys or inactivates pathogens.\\

\textbf{Hardness treatment}\\
Two commonly used methods to reduce hardness are the lime-soda process and ion exchange. The lime-soda process is applicable for large facilities, whereas ion exchange is normally employed in smaller water works. The lime-soda process will not remove all of the hardness and is usually operated to produce a residual hardness of about 100 mg/L as CaCO3. Greater reductions are not economical and may have adverse health consequences as well (McGhee, 1991). The discussion in this text focuses on ion exchange. Ion exchange is accomplished by charging a resin with sodium ions and allowing the resin to exchange the sodium ions for calcium or magnesium ions. Common resins include zeolites—natural and manmade minerals that will collect from a solution certain ions (sodium or KMnO4), and either exchange these ions (in the case in water softening) or use the ions to oxidize a substance (in the case of iron or manganese removal). The negative side of using ion exchange is that, even though the process softens water by removing all (or nearly all) of the hardness and adds sodium ions to the water, the water may be more corrosive than before. The addition of sodium ions to the water may also increase the health risk of those with high blood pressure.

\textbf{Disinfection}\\
At the turn of the last century, 35,000 people per 1,000,000 people did not reach 20 years of age. Today, however, the rate of births exceeds the rate of deaths, and the average lifespan is much longer. Curbing waterborne disease through disinfection
has made a significant contribution to birth rates outpacing death rates worldwide. The Safe Drinking Water Act requires that public water supplies be disinfected, and the U.S. Environmental Protection Agency (USEPA) sets standards and establishes processes for the treatment and distribution of disinfected water to ensure that no significant risks to human health occur. The USEPA Science Advisory Board has ranked pollutants in drinking water as one of the highest health risks meriting the Agency’s attention because of large-scale population exposure to contaminants, including lead, disinfectants and disinfection byproducts (DBPs), and disease-causing organisms.
Disinfectants are used by virtually all surface water systems in the United States and by an unknown percentage of systems that rely on groundwater. For nearly a century, chlorine has been the most widely used and most cost-effective disinfectant; however, disinfection treatments can produce a wide variety of byproducts, many of which have been shown to cause cancer or other toxic effects. Recently, concern has been raised over water quality deterioration, a problem that can grow dramatically during distribution unless systems are properly designed and operated. Disinfection is an integral part of water treatment, but filtration prior to disinfection is necessary to reduce pathogen levels and make disinfection more reliable by removing turbidity and other interfering constituents.
To solve the disinfectant and disinfection byproducts problem, we need innovative upgrades for the existing techniques, as well as new approaches to address these problems. Areas of interest include:\\
\begin{itemize}
\item Alternatives to chlorine disinfection for removing pathogenic microorganisms, including innovative applications of ultraviolet (UV) radiation and processes that improve overall effectiveness while using reduced amounts of disinfectant
\item Development of innovative unit processes, particularly for small systems, for removal of organic and inorganic contaminants (such as arsenic), particulates, and pathogens, such as cyst-like organisms and emerging pathogens such as caliciviruses, microsplorida (septata and enterocytozoan), hepatitis A virus, Mycobacterium avium–intracellulare complex (MAC), Helicobacter pylori, Legionella pneumophila, adenovirus 40/41/1-39, and Toxoplasma gondii
\item Development of efficient, cost-effective treatment processes for removing disinfection byproduct precursors (e.g., trihalomethanes, haloacetic acids), for ozonation (bromate, aldehydes), for chlorination (chloropicrin, haloacetonitriles), and for chloramination (organic chloramines, cyanogen chloride)
\item Improved methods for controlling pathogens through coagulation/settling, filtration, or other cost-effective means
\item Drinking water contamination control between the treatment plant and the user, especially considering potential chemical leaching from distribution system materials and surfaces (e.g., lead, copper, iron, and other pipe materials; protective coatings) as a result of instability, interaction with microorganisms, disinfection agents, and water treatment chemicals
\end{itemize}


Key Disinfection Terms\\
Before moving on to a discussion of the major disinfection methods used in treating water for human consumption, it is necessary to first define a few pertinent terms related to disinfection in general. To begin with, let’s establish the distinction between primary and secondary disinfection:
\begin{itemize}
\item Primary disinfection—Initial killing of Giardia cysts, bacteria, and viruses
\item Secondary disinfection—Maintenance of a disinfectant residual that prevents regrowth of microorganisms in the water distribution system between treatment and consumer
\end{itemize}
Other terms the reader should understand include
\begin{itemize}
\item Disinfection—Inactivation of virtually all recognized pathogenic microorganisms, but not necessarily all microbial life (which would be considering pasteurization or sterilization).
\item Disinfectant—(1) Any oxidant, including but not limited to, chlorine, chlorine dioxide, chloramine, and ozone, added to water in any part of the treatment or distribution process that is intended to kill or inactivate pathogenic microorganisms. (2) A chemical or physical process that kills pathogenic organisms in water; chlorine is often used to disinfect sewage treatment effluent, water supplies, wells, and swimming pools.
\item Disinfectant time—The time required for water to move from one point of disinfectant application (or the previous point of residual disinfectant measurement) to a point before or at the point where the residual disinfectant is measured.
\item Disinfectant contact time (T in C*T calculation)—The time (in minutes) required for water to move from the point of disinfectant application or the previous point of disinfection residual measurement to a point before or at
the point where residual disinfectant concentration (C) is measured. Where only one C is measured, T is the time (in minutes) required for water to move from the point of disinfectant application to a point before or at where residual disinfectant concentration (C) is measured. Where more than one C is measured, T is defined as follows:
\item For the first measurement of C, the time (in minutes) required for water to move from the first or only point of disinfectant application to a point before or at the point where the first C is measured
\item For subsequent measurements of C, the time in minutes that water takes to move from the previous C measurement point to the C measurement point for which the particular T is being calculated
\item Disinfection byproduct—A compound formed by the reaction of a disinfectant such as chlorine with organic material in the water supply.
\item Presence or absence of coliforms—Presence of coliform bacteria in water is an indication that the water may be contaminated by pathogenic organisms. Absence of coliform bacteria is considered to be sufficient evidence that pathogens are absent—if the source is good, a chlorine residual level is maintained and the supply has a good history.
\item Sterilization—The destruction of all microorganisms. Sterilizing potable water requires the application of a much higher dose of chemical disinfectants, which would greatly increase operating costs and would create taste problems for the consumer. Excessive application of disinfectants also generates excessive levels of unwanted disinfection byproducts. For these reasons, current treatment practices are used for turbidity removal and subsequent disinfection to the extent necessary to eliminate known diseasecausing organisms sufficient to protect public health.
Note: Sterilization should not be confused with disinfection.
\item Waterborne disease—Caused by pathogenic organisms in water.
\end{itemize}
Disinfection Methods\\
Although chlorination is the best known and the most common disinfection method, other methods are available and can be used in various situations. The three general types of disinfection are
\begin{itemize}
\item Heat treatment—Probably one of the first methods employed to disinfect water was to boil it. For small quantities of water, boiling water is still a good emergency procedure to use.
\item Radiation treatment—Uses ultraviolet radiation to disinfect water.
\item Chemical treatment—Employs the use of chemicals to disinfect water. Examples of chemical disinfectants include oxidizing agents such as chlorine, ozone, bromine, iodine, and potassium permanganate; metal ions such as silver, copper, and mercury; and acids and alkalis.
\end{itemize}
Obviously, several different disinfectants are available for use in treating water, and several of these are discussed in detail in subsequent sections. For now, it is important to understand that, even though several choices are available, whichever disinfectant is chosen must meet certain criteria—more specifically, the disinfectant chosen must be effective for disinfecting water (and wastewater) and must possess certain desirable characteristics.\\
Desirable Characteristics of a Disinfectant:
\begin{enumerate}
\item It must act in a reasonable time.
\item It must act as temperature or pH changes.
\item It must be nontoxic.
\item It must not add unpleasant taste or odor.
\item It must be readily available.
\item It must be safe and easy to handle and apply.
\item It must be easy to determine the concentration of.
\item It must be able to provide residual protection.
\item Pathogenic organisms must be more sensitive to the disinfectant than are nonpathogens.
\item It must be capable of being applied continually.
\item The amount applied must be sufficient to produce a safe water.
\end{enumerate}
In addition to the desirable characteristics of a disinfectant listed above, the disinfectant chosen must be able to kill off or deactivate pathogenic microorganisms by one of several possible methods, including: (1) damaging the cell wall, (2) altering the ability to pass food and waste through the cell membrane, (3) altering the cell protoplasm,
(4) inhibiting the cells’ conversion of food to energy, or (5) inhibiting reproduction.\\

Chlorination\\
For the past several decades, chlorine dispensed as a solid (calcium hypochlorite), liquid (sodium hypochlorite), or gas (elemental chlorine, Cl2) has been the disinfectant of choice, particularly in the United States. Chlorine (sometimes referred to as the workhorse of disinfection) has proven its worth both because of its effectiveness and because it is relatively inexpensive; it also provides a chlorine residual in the water distribution system, ensuring that the water remains disease free.\\
Gaseous chlorine (Cl2), 2.5 times as heavy as air, is a greenish-yellow toxic gas. One volume of liquid chlorine confined in a container under pressure yields about 450 volumes of gas. Large water treatment works usually use chlorine gas, supplied in liquid form, in high-strength, high-pressure steel cylinders. The liquid immediately vaporizes in the form of gas when released from these pressurized containers. Chlorine gas is lethal at concentrations as low as 0.1% air by volume. In nonlethal concentrations, it irritates the eyes, nasal membranes, and respiratory tract.\\
Sodium hypochlorite is most commonly used in smaller systems, because it is simpler to use and has less extensive safety requirements than gaseous chlorine; in the form used, it is less toxic. Recently, many larger water facilities that have used chlorine for disinfection are beginning to substitute sodium hypochlorite for chlorine because of regulatory pressure.
Note: The Occupational Safety and Health Administration’s Process Safety Management Standard (29 CFR 1910.119) and USEPA’s Risk Management Program (Clean Air Act, Section 112(r)(7)) have come to be known in the industry as the “chlorine killers,” because of their effect on industrial processes. The USEPA is attempting to steer industry away from the use of chlorine. Although the Agency cannot absolutely outlaw this substance from use, it is following the path of simply regulating it to death. In an effort to avoid having to comply with strict (in some cases, unworkable) regulations, many water treatment and wastewater facilities in the United States are substituting some other chemical product that is not regulated (at least for the moment) such as sodium hypochlorite. Sodium hypochlorite provides 5 to 15\% available chlorine (common laundry bleach is a 5% solution of sodium hypochlorite). Usually diluted with water before application as a disinfectant, it is very corrosive and should be handled and stored with care and kept away from equipment that can be damaged by corrosion. Sodium hypochlorite solution is more costly per pound of available chlorine and does not provide the same level of protection of chlorine gas.\\
Calcium hypochlorite is a white solid in granular, powdered, or tablet form containing 65\% available chlorine. In packaged form, calcium hypochlorite is stable— more stable than solutions of sodium hypochlorite, which deteriorate over time; however, calcium hypochlorite is hygroscopic, which means it readily absorbs moisture. It reacts slowly with moisture in the air to form chlorine gas. It is a corrosive material with a strong odor and requires proper handling. Some practical difficulty is involved in dissolving calcium hypochlorite. It must be kept away from organic materials such as wood, cloth, and petroleum products. Reactions between it and organic materials can generate enough heat to cause a fire or explosion.\\

Chlorine Use\\
Whatever form of chlorine is used for disinfection (elemental chlorine, sodium hypochlorite, or calcium hypochlorite), it may be added to the incoming flow (prechlorination) to assist with the oxidation of inorganics or to arrest biological action that may produce undesirable gases in the bottom of clarifiers. More often, however, chlorine is added just prior to filtration to keep algae from growing at the medium surface and to prevent large populations of bacteria from developing within the filter medium. Safe and effective application of chlorine requires specialized equipment and considerable care and skill on the part of the plant operator. Various means of feeding chlorine have been developed, but probably one of the widest used and safest types of chlorine feed devices is the all-vacuum chlorinator. Mounted directly on the chlorine cylinder, the gaseous chlorine is always under a partial vacuum in the line that carries it to the point of application. In a typical vacuum chlorine feed system, the vacuum is formed by water flowing through the ejector unit at high velocity.\\
Hypochlorites are usually applied to water in liquid form by means of positive displacement-type pumps, which deliver a specific amount of liquid on each stroke of a piston or flexible diaphragm. Chlorine, when added to water, reacts with various substances or impurities in the water (e.g., organic materials, sulfides, ferrous iron, nitrites), which creates a chlorine demand. Chlorine demand is a measure of the amount of chlorine that will combine with impurities and is therefore available to act as a disinfectant. Chlorine combines with ammonia or other nitrogen compounds to form chlorine compounds that have some disinfectant properties. These compounds are called combined available chlorine residual. In the context used here, “available” means available to act as a disinfectant. The uncombined chlorine that remains in the water after combined residual is formed is called free available chlorine residual. Free chlorine is a much more effective disinfectant than combined chlorine.\\

Factors Affecting Successful Chlorination\\
The factors important to successful chlorination are:
%\begin{itemize}
%\item Concentration of free chlorine
%\item Contact time
%\item Temperature
%\item pH
%\item Turbidity
%\end{itemize}
The effectiveness of chlorination is directly related to the contact time with and concentration of free available chlorine. At lower chlorine concentrations, contact times must be increased. Maintaining a lower pH will also increase the effectiveness of disinfection. The higher the temperature, the faster the disinfection rate. Chlorine (or any other disinfectant for that matter) is effective only if it comes into contact with the organisms to be killed. Good contact between chlorine and microorganisms is prevented whenever high turbidity levels exist. For this and aesthetic reasons, turbidity should be reduced where necessary through the coagulation and sedimentation methods previously discussed.\\
Chlorination Byproducts\\
A serious disadvantage of chlorination is the potential formation of byproducts. Chlorine, for example, can mix with the organic compounds in water (such as decaying vegetation) to form trihalomethanes (THMs). One THM, chloroform, is a suspected carcinogen. Other common trihalomethanes are similar to chloroform and may cause cancer.
At the present time, about 90\% of U.S. water utilities use chlorine to disinfect water. Although chlorine has virtually eliminated the risks of waterborne disease such as typhoid fever, cholera, and dysentery, recent studies have shown risks associated with byproducts of chlorine—a reason why water utilities already have been looking at alternative methods for disinfecting water.
Several approaches for reducing harmful chlorination byproducts have been used. For example, one approach is to remove more of the organics before any chlorination takes place. This can be accomplished (to a degree) by not chlorinating the incoming
raw water before coagulation and filtration, thus reducing the formation of THMs. Aeration or adsorption on activated carbon will remove organic materials at higher concentrations or those not removed by other techniques. Another approach is to reevaluate the amount of chlorine used—the same degree of disinfection might be possible with lower chlorine dosages. Changing the point in treatment where chlorine is added is another approach commonly employed; rather than adding chlorine as chemical feed during coagulation, sedimentation, or filtration, it can instead be added after filtration. Another current approach is using alternative disinfection methods.\\
Note: Because of OSHA’s Process Safety Management (PSM) standard and USEPA’s Risk Management Program (RMP), many facilities currently using elemental chlorine have used or are actively pursuing the use of alternative disinfection methods. We further reemphasize that the problem of THMs is also helping spur interest in alternatives to chlorination as the preferred method of disinfection.\\

Alternative Disinfection Methods\\
Currently, several alternative disinfection methods are available for use in treating water, but the following discussion focuses on two of these alternatives: ozonation and ultraviolet (UV) radiation. These commonly used alternatives (especially in small water treatment systems) are also increasingly being substituted for existing chlorination systems at larger plants because of regulatory pressure.
Note: Before discussing the ozonation and ultraviolet disinfection alternatives, it is important to point out that neither one of these two alternative disinfectants is an easy solution to problems created by chlorination. It is true that each has the advantages of not creating THMs and not being covered by the requirements under the PSM standard and RMP, but each has uncertainties and known disadvantages that have restricted their more widespread use. In addition, ozonation and ultraviolet irradiation cannot be used as disinfectants by themselves. Both require secondary disinfectant (usually chlorine) to maintain a residual in the distribution system.\\
Ozonation\\
Ozone (O3), a gas at ordinary temperature and pressures, is a very powerful disinfectant that breaks up molecules in water; it is even more effective against some viruses and cysts than chlorine. It has the added advantage of leaving no taste or odor and is unaffected by pH or the ammonia content of the water. When ozone reacts with reduced inorganic compounds and with organic material, an oxygen atom instead of a chloride atom is added to the organics, the end result being an environmentally acceptable compound. But, because ozone is unstable and cannot be stored, it must be produced onsite. Ozonation usually costs more than chlorination.\\
Ultraviolet\\
Ultraviolet (UV) light is electromagnetic radiation just beyond the blue end of the light spectrum, outside the range of visible light. It has a much higher energy level than visible light, and in large doses it inactivates both bacteria and viruses. UV energy is absorbed by genetic material in the microorganisms, interfering with their ability to reproduce and survive, as long as the radiation contacts the microorganisms without interference from turbidity. The big advantage of UV disinfection over chlorine and
ozone is that UV does not involve chemical use. Generally, UV light used for disinfecting water is generated by a series of submerged, low-pressure mercury lamps. Continuing advances in UV germicidal lamp technology are making UV disinfection a more reliable and economical option for disinfection in many plants.\\

\textbf{NONCONVENTIONAL WATER TREATMENT TECHNOLOGIES}
Stage 1 of the USEPA’s Disinfectants and Disinfection Byproduct Rule and the Interim Enhanced Surface Water Treatment Rule, designed to significantly lower THM byproducts of chlorine disinfection in water, has driven (along with the regulatory requirements of the PSM standard and RMP) many water and wastewater treatment utilities to find and use alternative disinfection methodologies. Although ozonation and ultraviolet irradiation might be suitable disinfection alternatives, switching from chlorine to chlorine dioxide (a chemical that has been proven to form fewer THMs) might also be another viable disinfection alternative. Whichever disinfection alternative is ultimately selected, remember that the selection is driven not only by regulatory requirements but also by site-specific requirements.
The disinfection issues covered to this point are important—the overall ramifications of regulatory pressure and environmental impact cannot be overstated—but other issues besides disinfection must be considered when deciding which water treatment methodology to employ. Most of the time, clarification by coagulation, flocculation, sedimentation, and filtration removes suspended impurities and turbidity from drinking water, and disinfection (the final step in the process) produces potable water, free of harmful pathogens. Simply put, the water treatment processes discussed in the previous sections of this part of the text are usually sufficient to render most natural surface water (such as a river) potable. In some instances, however, the water supply may contain materials that are not removed by conventional water treatment processes, and other treatment processes may be required to remove many of the dissolved organic and inorganic substances. Examples would include groundwater with excessive dissolved solids and surface waters containing organic compounds from domestic or industrial wastewaters or organics occurring naturally such as humic acid or products of algae blooms. Additional processes are available for removing these contaminants.
Note: These additional water treatment processes involve sophisticated equipment and require highly skilled operators; therefore, they are quite expensive (Peavy et al., 1985).
Additional water unit treatment processes may be used in addition to clarification or applied separately, depending on the source and quality of the raw water. Let’s take a closer look at groundwater. The question is—does a typical groundwater source require treatment beyond conventional means? The answer is that groundwater does not normally require processing by the unit treatment steps listed above, other than disinfection, because groundwater is filtered naturally by the layers of soil from which it is withdrawn. Disinfection is only applied (in many cases) as a precautionary step required by law for public water systems. Groundwater is usually free
of bacteria or other microorganisms; however, that all groundwater comes into contact with soil and rock is a cause for concern. With such contact, groundwater may become contaminated by high levels of dissolved minerals that must be removed.

\textbf{FLUORIDATION}\\
Fluoride, when added to drinking water supplies in small concentrations (about 1.0 mg/L), can be beneficial. In some locations, common practice is to mix a 4\% solution of sodium fluoride and feed that into the flow of the water system. The amount that is fed depends on the air temperature and on the fluoride levels in the raw water. Experience has shown that drinking water containing a proper amount of fluoride can reduce tooth decay by 65\% in children. Fluoride combines chemically with tooth enamel when permanent teeth are forming, and the result is teeth that are harder, stronger, and more resistant to decay. The USEPA sets the upper limits for fluoride in drinking water supplies based on ambient temperatures; for example, people drink more water in warmer climates, so fluoride concentrations should be lower in these areas.\\

\textbf{WATER TREATMENT OF ORGANIC AND INORGANIC CONTAMINANTS}\\
Manmade compounds that contain carbon—synthetic organic chemicals (SOCs)— are, from time to time, detected in U.S. water supplies. Some of these are volatile organic chemicals (VOCs), such as the solvent trichloroethylene. The problem with VOCs in a water supply (i.e., any water supply used by the public) is twofold. They are easily absorbed through the skin and they volatize into gases that can then be inhaled by those taking a shower or a bath or while washing dishes. How do water supplies become contaminated by organic compounds? Good question. Basically, sources of organic contaminants are usually improperly disposed wastes, pesticides, industrial effluents, and leaking fuel oil tanks (gasoline in particular).
Water supplies may also contain inorganic contaminants consisting mainly of substances occurring naturally in the ground, such as sulfate, fluoride, arsenic, barium, radium, selenium, and radon. Metallic substances from industrial sources can contaminate surface waters. The inorganic ion nitrate (from fertilizers and feedlot runoff in agricultural areas) occurs frequently in groundwater supplies. Another source of inorganic chemical contamination in drinking water supplies is corrosion or deterioration of water supply equipment, such as plumbing systems, which release metal and nonmetal substances into the water, including lead, cadmium, zinc, copper, iron, and plumbing cement. Inorganic contaminants can be treated by corrosion controls and removal techniques. Corrosion controls reduce the presence of corrosion byproducts (e.g., lead) at the consumer’s tap. Removal technologies, coagulation and filtration, reverse osmosis, and ion exchange are used to treat source water that iscontaminated with metals or radioactive substances. The following sections discuss processes for removing inorganic and organic dissolved solids from water intended for potable use. Keep in mind that (with some modifications) these same processes may act as tertiary treatment for wastewater.\\
 

\textbf{Aeration}\\
Aeration (air stripping) is a physical treatment process in which air is thoroughly mixed with water—a technique effective for removing dissolved gases and highly volatile odorous compounds. Contact with air and oxygen can improve water quality in a number of ways. When aeration is a first step in processing well water, for example, it may achieve any or all of the following: removal of hydrogen sulfide, reduction of dissolved carbon dioxide, and addition of dissolved oxygen for oxidation of iron and manganese (the oxygen in the air reacts with the iron and manganese to form an insoluble precipitate—rust). One of the most common uses of aeration is for taste and odor control. Sedimentation and filtration are then necessary to clarify the water.
Note: Aeration is rarely effective in processing surface waters, simply because the odor-producing substances are generally nonvolatile.\\
Several methods to aerate the water are available. The method selected depends primarily on the type and concentration of material to be removed from the water and on the available pressure. Aeration in water treatment can be accomplished using spray nozzles, cascade systems, multiple-tray aerators, diffused-air aerators, and mechanical aerators.\\


\textbf{Oxidation}\\
Simply stated, oxidation is a reaction in which a substance loses electrons, thus increasing its charge. A substance that oxidizes another is referred to as an oxidizing agent or oxidizer. In water treatment, oxidation is used to remove or destroy undesirable tastes or odors, to aid in removal of iron and manganese, and to help improve clarification and color removal in source water. Chlorine dioxide, potassium permanganate, and ozone are strong oxidants capable of destroying many odorous compounds. Because they do not produce THMs, these chemicals are favored over heavy chlorination.
Note: Atmospheric oxygen, through aeration, can be used to oxidize the organic substances responsible for undesirable tastes and odors, but the process is usually too slow to be of value. If dissolved gases such as hydrogen sulfide are the cause of taste and odor problems, aeration will effectively remove them through oxidation and stripping.

\textbf{Adsorption}\\
When we speak of adsorption, we are referring primarily to a surface phenomenon—the adsorption that results when one substance attaches itself to the surface of another. The two most common adsorptive media used in water treatment are activated carbon and activated alumina. These adsorptive materials are generally most effective for taste and odor control and for removal of organic pollutants; however, the most important applications of adsorption in water treatment are the removal of arsenic and organic pollutants.
Adsorption of organic materials using activated carbon has been a common practice in water treatment for many years. Activated carbon is manufactured from carbonaceous material such as wood, coal, and petroleum residues. A char is made by burning the material in the absence of air, and it is then oxidized at higher temperatures to create a very porous structure. This activation step provides irregular channels and pores in the solid mass, resulting in a very large surface-area-to-mass ratio. This large surface area gives activated carbon its effectiveness as an adsorbing agent. The larger the surface area of an adsorber, the greater its power. Each activated carbon contains a huge number of pores and crevices into which organic molecules enter and are adsorbed onto the activated carbon surface.
Activated carbon has a particularly strong attraction for organic molecules such as the aromatic solvents benzene, toluene, and nitrobenzene; the chlorinated aromatics polychlorinated biphenyls (PCBs), chlorobenzenes, and chloronaphthalene; phenol and chlorophenols; the polynuclear aromatics acenaphthene and benzopyrenes; pesticides and herbicides; chlorinated aliphatics such as carbon tetrachloride and chloroalkyl ethers; and high-molecular-weight hydrocarbons such as dyes, gasoline, amines, and humics.
Two forms of activated carbon are used in water treatment: powdered and granular. Powdered activated carbon is often used for taste and odor control. Its effectiveness depends on the source of the undesirable tastes and odors. It is also effective in removing the organic precursors that react with chlorine to form harmful THM compounds after disinfection.
Powdered activated carbon is a finely ground, insoluble black powder that can be added at any point in the treatment process ahead of the filters. It is fed to water either as a dry powder or as a wet slurry. Although adsorption is nearly instantaneous, a contact time of 15 minutes or more is desirable before sedimentation or filtration. Activated carbon media must periodically be replaced with new or regenerated activated carbon. Replacement cycles can vary from 1 to 3 years for taste and odor treatment to as little as 4 or 5 weeks for removal of organics. The activated carbon regeneration process involves (1) removing the spent carbon as a slurry, (2) dewatering the slurry, (3) feeding the carbon into a special furnace where regeneration occurs (i.e., the organics are driven from the carbon surface by heat), and (4) returning it to use.
Activated alumina (a highly porous and granular form of aluminum oxide) is also an adsorptive medium used in water treatment. It is used primarily to remove arsenic and excess fluoride ions. Water is percolated through a column of alumina media, and a combination of adsorption and ion exchange performs the actual removal of arsenic and fluoride ions. Like the regeneration process used to restore used activated carbon to full potency, activated alumina also requires periodic regeneration, accomplished by passing a caustic soda solution through the media. Excess caustic soda is neutralized by rinsing the activated alumina with an acid. Disposal of these wash waters, laden with toxic arsenic and fluoride ions, must be done in accordance with applicable laws.
Note: Powdered activated carbon is much more difficult to regenerate than granular activated carbon. Granular activated carbon is sometimes used in the filter bed itself, combining both filtration and adsorption in one treatment unit. The major problem associated with granular activated carbon systems is suspended solids in the water plugging up the bed.\\
 

\textbf{Demineralization}\\
Demineralization refers to the removal of dissolved solids (inorganic mineral substances) from water. Dissolved solids contain both cations and anions and therefore require two types of ion exchange resins. Cation exchange resins used for demineralization purposes have hydrogen exchange sites and are divided into strong acid and weak acid classes. The anion exchange resins commonly used contain hydroxide ions and are divided into strong and weak base classes. Demineralization is commonly used in industry in waste treatment for removal of arsenic, barium, cadmium, chromium, fluoride, sulfate, and zinc. Some general advantages of using ion exchange to remove these contaminants are the low capital investment required and the mechanical simplicity of the process. In addition, the ion exchange process can be used to recover valuable chemicals for reuse, or harmful ones for disposal. For example, it is often used to recover chromic acid from metal finishing waste for reuse in chrome-plating baths. It also has some application in the removal of radioactivity. The major disadvantages are the high chemical requirements needed to regenerate the resins and to dispose of chemical wastes from the regeneration process. These factors make ion exchange more suitable for small systems than for large ones.\\

\textbf{Membrane processes}\\
Membrane processes used in water treatment are primarily demineralization processes. Demineralization of water can be accomplished using thin, microporous membranes. Electrodialysis and reverse osmosis are the most common membrane processes. Before we briefly discuss these two membrane processes, you need a basic understanding of osmosis. During osmosis, two solutions containing different concentrations of minerals are separated by a semipermeable membrane. Water tends to migrate through the membrane from the side of the more dilute solution to the side of the more concentrated solution. This is osmosis, and it continues until the build-up of hydrostatic pressure on the more concentrated solution is sufficient to stop the net flow. In reverse osmosis, the flow of water through the semipermeable membrane is reversed by applying external pressure to offset the hydrostatic pressure. This results in a concentration of minerals on one side of the membrane and pure water on the other side. Reverse osmosis can treat for a wide variety of health and aesthetic contaminants in water. Effectively designed, reverse osmosis equipment can treat aesthetic contaminants that cause unpleasant taste, color, and odor problems, such as a salty or soda taste caused by chlorides or sulfates. Reverse osmosis can also be effective for treating arsenic, asbestos, atrazine, fluoride, lead, mercury, nitrate, and radium. When used with appropriate carbon prefiltering, additional treatment can also be provided for such “volatile” contaminants as benzene, trichloroethylene, trihalomethanes, and radon. Some reverse osmosis equipment is also capable of treating for Cryptosporidium. Reverse osmosis can be expected to play a major role in water
treatment for years to come. Reverse osmosis (also called ultrafiltration) is the most common process for reducing the salinity of brackish groundwater. In operation, a semipermeable membrane (the most essential element in the reverse osmosis method of demineralization)
separates salty water of two different concentrations. Concentrations have a natural tendency to become equalized by a flow of water from the dilute side to the concentrated side (osmosis). But high pressure applied to the high concentration side of the membrane can reverse this direction of flow. Freshwater diffuses through the membrane, leaving a more concentrated salt solution behind. The performance of reverse osmosis units is highly dependent on a number of water quality parameters. Suspended solids, dissolved organics, hydrogen sulfide, iron, and strong oxidizing agents (chlorine, ozone, and permanganate) are harmful to membranes.
Electrodialysis is the demineralization of water using the principles of osmosis—but it uses ion-selective membranes and an electric field to separate anions and cations in solution. In the past, electrodialysis was most often used for purifying brackish water, but it is now finding a role in industrial waste treatment as well. For example, metals salts from plating rinses are sometimes removed in this way.
\end{document}