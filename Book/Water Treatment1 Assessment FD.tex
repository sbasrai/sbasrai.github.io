What is the recommended loading rate for copper sulfate for algae control at an alkalinity greater than 50 mg/L?\\
0.9 lb of copper sulfate per acre of surface area\\
1.9 lb of copper sulfate per acre of surface area\\
2-4 lb of copper sulfate per acre of surface area\\
4 lb of copper sulfate per acre of surface area\\
If ammonia vapor is passed over a chlorine leak in a cylinder valve, the presence of the leak is indicated by a\\
Yellow cloud\\
White cloud\\
Gray cloud\\
Brown cloud\\
What is the recommended minimum contact time water mains with the chlorine slug method?\\
3 hours\\
6 hours\\
10 hours\\
12 hours\\
The basic goal for water treatment is to \rule{2cm}{0.3pt}.\\
Protect public health\\
Make it clear\\
Make it taste good\\
Get stuff out\\
Greensand can be operated in either \rule{2cm}{0.5pt} regeneration or \rule{2cm}{0.5pt} regeneration modes.\\
Continuous or intermittent\\
Fast or slow\\
Hot or cold\\
Constant or unusual\\
The two most common types of chlorine disinfection by-products include:\\
TTHM and HAA5\\
TTHA of HMM5\\
Turbidity and color\\
Chloride and fluoride\\
GAC contactors are used to reduce the amount of \rule{2cm}{0.5pt} contaminants in water.\\
Inorganic\\
Turbidity\\
Particle\\
Organic\\
List the five types of surface water filtration systems.\\
Bag filtration, cartridge filtration, fine filtration, coarse filtration, media filtration\\
Conventional treatment, direct filtration, slow sand filtration, diatomaceous earth filtration, membrane filtration\\
Turbidity filtration, color filtration, bag filtration, fine filtration, media filtration\\
None of the above\\
Describe two primary methods used to control taste and odor?\\
Oxidation and adsorption\\
Filtration and sedimentation\\
Mixing and coagulation\\
Sedimentation and clarification\\
The adsorption process is used to remove:\\
Organics or inorganics\\
Bugs or salts\\
Organisms or dirt\\
Color or particles\\
The solid that adsorbs a contaminant is called the:\\
Adsorbent\\
Adsorbate\\
Sorbet\\
Rock\\
What is a method of reducing hardness?\\
Softening\\
Hardening\\
Lightning\\
Flashing\\
Bag and cartridge filters are used to remove which two pathogenic microorganisms?\\
Viruses and giardia\\
Giardia and cryptosporidium\\
Viruses and bacteria\\
None of the above\\
The process of cleaning a filter by pumping water up through the filter media is called \rule{2cm}{0.3pt} the filter.\\
Backwashing\\
Rewashing\\
Purging\\
Lifting\\
In a typical water treatment plant, alum would be added into the \rule{2cm}{0.3pt} mixer.\\
Speed\\
Large\\
Slow\\
Flash\\
When comparing conventional treatment with direct filtration, what process unit is in the conventional treatment plant that is not in the direct filtration plant?\\
Filter\\
Clarifier\\
Mixer\\
Detention\\
List the basic processes, in the proper order, for a conventional treatment plant.\\
Coagulation, flocculation, sedimentation, filtration\\
Flocculation, coagulation, sedimentation, filtration\\
Filtration, coagulation, flocculation, sedimentation\\
Coagulation, sedimentation, flocculation, filtration\\
The four most common oxidants include:\\
Chlorine, potassium permanganate, ozone, chlorine dioxide\\
Chlorides, soap, air, coagulants\\
Air, chemicals, sodium, chloride\\
Flocculants, coagulants, sediments, granules\\
 When operating a filter, one of the operational concerns is the difference between the pressure or head on top of the filter and the pressure or head at the bottom of the filter. This difference is called \rule{2cm}{0.3pt} pressure.\\
Different\\
Differential\\
High\\
Low\\
 What type of polymer is used to improve the efficiency of the sedimentation\\
process?\\
Cationic\\
Nonionic\\
Anionic\\
All of the above\\
A(n) \rule{2cm}{0.3pt} polymer is commonly used as a coagulant.\\
Anionic\\
Cationic\\
Nonionic\\
Ionic\\
A(n) \rule{2cm}{0.3pt} polymer is used to enhance flocculation.\\
Anionic\\
Cationic\\
Nonionic\\
Ionic\\
Al$_2$(SO$_4$)$_3$ • 18H$_2$O is the chemical formula for:\\
Alum\\
Iron\\
Manganese\\
Lead\\
Particles that are less than 1 $\mu$m in size and will not settle easily and are called:\\
Light particles\\
Colloidal particles\\
Colored particles\\
Flat particles\\
The sedimentation portion of water treatment is also called a(n):\\
Clarifier\\
Filter\\
Adsorber\\
Water treater\\
Slowly agitating coagulated materials is the process of:\\
Flocculation\\
Coagulation\\
Sedimentation\\
Filtration\\
The process of decreasing the stability of colloids in water is called:\\
Flocculation\\
Coagulation\\
Sedimentation\\
Clarification\\
The chemical oxidation process in water treatment is typically used to aid in the\\
removal of :\\
Organic contaminants\\
Inorganic contaminants\\
Large contaminants\\
None of the above\\
Flocculation, sedimentation, filtration, and adsorption are \rule{2cm}{0.3pt}\\
processes.\\
Physical\\
Chemical\\
Biological\\
Mechanical\\
Oxidation, coagulation, and disinfection are \rule{2cm}{0.3pt} processes.\\
Physical\\
Chemical\\
Biological\\
Mechanical\\
A precipitate can be formed after which one of the following processes:\\
Oxidation\\
Flocculation\\
Filtration\\
Adsorption\\
Water that is safe to drink is called \rule{1cm}{0.5pt}  water.\\
Potable\\
Palatable\\
Good\\
Clear\\
The type of organisms that can cause disease are said to be \rule{1cm}{0.5pt} microorganisms.\\
Bad\\
Pathogenic\\
Undesirable\\
Sick\\
The basic goal for water treatment is to \rule{1cm}{0.5pt}.\\
Protect public health\\
Make it clear\\
Make it taste good\\
Get stuff out\\
Four types of aesthetic contaminants in water include the following:\\
Odor, turbidity, color, hydrogen sulfide gas\\
Pathogens, microorganisms, arsenic, disinfection by-products\\
What does mg/L stand for?\\
Microorganisms/Liter\\
Milligrams/Loser\\
Milligrams/Liter\\
None of the above\\
Disinfection by-products are a product of:\\
Filtration\\
Disinfection\\
Sedimentation\\
Adsorption\\
Acute contaminants are those that can cause sickness after:\\
Prolonged exposure\\
Low levels or low exposure\\
Chronic contaminants are those that can cause sickness after:\\
Prolonged exposure\\
Low levels or low exposure\\
TTHMs and HAA5s can affect:\\
Health\\
Aesthetics\\
Color\\
Odor\\
Oxidation, coagulation, and disinfection are \rule{1cm}{0.5pt}  processes.\\
Physical\\
Chemical\\
Biological\\
Mechanical\\
Flocculation, sedimentation, filtration, and adsorption are \rule{1cm}{0.5pt} processes.\\
Physical\\
Chemical\\
Biological\\
Mechanical\\
A precipitate can be formed after which one of the following processes:\\
Oxidation\\
Flocculation\\
Filtration\\
Adsorption\\
Giardia and cryptosporidium are a type of:\\
Mineral\\
Organism\\
Color\\
Bird\\
14. The chemical oxidation process in water treatment is typically used to aid in the\\
removal of :\\
Organic contaminants\\
Inorganic contaminants\\
Large contaminants\\
None of the above\\
The process of decreasing the stability of colloids in water is called:\\
Flocculation\\
Coagulation\\
Sedimentation\\
Clarification\\
Slowly agitating coagulated materials is the process of:\\
Flocculation\\
Coagulation\\
Sedimentation\\
Filtration\\
The sedimentation portion of water treatment is also called a(n):\\
Clarifier\\
Filter\\
Adsorber\\
Water treater\\
Particles that are less than 1 $\mu\text{m}$ in size and will not settle easily and are called:\\
Light particles\\
Colloidal particles\\
Colored particles\\
Flat particles\\
One micrometer is also equal to:\\
0.1 mm\\
0.0001 mm\\
0.001 mm\\
1 m\\
Particles less than 0.45 $\mu\text{m}$ in size are considered to be:\\
Dissolved\\
Really little\\
Colored particles\\
Flat particles\\
Turbidity is measured as:\\
Mg/L\\
mL\\
gpm\\
NTU\\
Al2(SO4)3 • 18H20 is the chemical formula for:\\
Alum\\
Iron\\
Manganese\\
Lead\\
A(n) \rule{1cm}{0.5pt}  polymer is commonly used as a coagulant.\\
Anionic\\
Cationic\\
Nonionic\\
Ionic\\
A(n) \rule{1cm}{0.5pt}  polymer is used to enhance flocculation.\\
Anionic\\
Cationic\\
Nonionic\\
Ionic\\
The concentration of a chemical added to the water is measured in:\\
mL\\
mg\\
mg/L\\
Liters\\
The quantity of chlorine remaining after primary disinfection is called a\\
\rule{1cm}{0.5pt}  residual.\\
Chlorine\\
Permaganate\\
Hot\\
Cold\\
Primary disinfectants are used to \rule{1cm}{0.5pt}  microorganisms.\\
Hurt\\
Inactivate\\
Burn up\\
Evaporate\\
Secondary disinfectants are used to provide a \rule{1cm}{0.5pt}  in the distribution system.\\
Color\\
Chemical\\
Smell\\
Residual\\
What type of polymer is used to improve the efficiency of the sedimentation\\
process?\\
Cationic\\
Nonionic\\
Anionic\\
All of the above\\
When operating a filter, one of the operational concerns is the difference between the pressure or head on top of the filter and the pressure or head at the bottom of the filter. This difference is called \rule{1cm}{0.5pt}  pressure.\\
Different\\
Differential\\
High\\
Low\\
List the basic processes, in the proper order, for a conventional treatment plant.\\
Coagulation, flocculation, sedimentation, filtration\\
Flocculation, coagulation, sedimentation, filtration\\
Filtration, coagulation, flocculation, sedimentation\\
Coagulation, sedimentation, flocculation, filtration\\
The four most common oxidants include:\\
Chlorine, potassium permanganate, ozone, chlorine dioxide\\
Chlorides, soap, air, coagulants\\
Air, chemicals, sodium, chloride\\
Flocculants, coagulants, sediments, granules\\
When comparing conventional treatment with direct filtration, what process unit is in the conventional treatment plant that is not in the direct filtration plant?\\
Filter\\
Clarifier\\
Mixer\\
Detention\\
In a typical water treatment plant, alum would be added into the \rule{1cm}{0.5pt}  mixer.\\
Speed\\
Large\\
Slow\\
Flash\\
The process of cleaning a filter by pumping water up through the filter media is called \rule{1cm}{0.5pt}  the filter.\\
Backwashing\\
Rewashing\\
Purging\\
Lifting\\
Bag and cartridge filters are used to remove which two pathogenic microorganisms?\\
Viruses and giardia\\
Giardia and cryptosporidium\\
Viruses and bacteria\\
None of the above\\
List the four types of membrane filtration processes commonly used in water\\
treatment.\\
MF, UF, NF, and RO\\
MNF, UOF, NOF, and ROO\\
CFM, FM, FN, and OR\\
None of the above\\
What is a method of reducing hardness?\\
Softening\\
Hardening\\
Lightning\\
Flashing\\
Adsorption of a substance involves its accumulation onto the surface of a:\\
Solid\\
Rock\\
Pellet\\
Snow ball\\
The solid that adsorbs a contaminant is called the:\\
Adsorbent\\
Adsorbate\\
Sorbet\\
Rock\\
The adsorption process is used to remove:\\
Organics or inorganics\\
Bugs or salts\\
Organisms or dirt\\
Color or particles\\
Describe two primary methods used to control taste and odor?\\
Oxidation and adsorption\\
Filtration and sedimentation\\
Mixing and coagulation\\
Sedimentation and clarification\\
List the five types of surface water filtration systems.\\
Bag filtration, cartridge filtration, fine filtration, coarse filtration, media filtration\\
Conventional treatment, direct filtration, slow sand filtration, diatomaceous\\
earth filtration, membrane filtration\\
Turbidity filtration, color filtration, bag filtration, fine filtration, media filtration\\
None of the above\\
GAC contactors are used to reduce the amount of \rule{1cm}{0.5pt}  contaminants in water.\\
Inorganic\\
Turbidity\\
Particle\\
Organic\\
Greensand can be operated in either \rule{1cm}{0.5pt}  regeneration or \rule{1cm}{0.5pt} regeneration modes.\\
Continuous or intermittent\\
Fast or slow\\
Hot or cold\\
Constant or unusual\\
 What is the cause of taste and odor problems in raw surface water?\\
Copper sulfate\\
Blue-green algae\\
Oxygen\\
Lake turnover\\
 What chemical reduces blue-green algae growth?\\
Chlorine\\
Caustic Soda\\
Copper Sulfate\\
Alum\\
What is the purpose of adding fluoride to drinking water?\\
Increase tooth decay\\
Reduce tooth decay\\
Make teeth white\\
Government conspiracy\\
The optimal coagulant dose is determined by a\\
Chlorine Test\\
Flocculation test\\
Jar Test\\
Coagulation test\\
 The most common primary coagulant is\\
Alum\\
Cationic polymer\\
Fluoride\\
Anionic polymer\\
 Bacteria and Viruses belong to a particle size known as\\
Suspended\\
Dissolved\\
Strained\\
Colloidal\\
 The purpose of coagulation is to\\
Increase filter run times\\
Increase sludge\\
Increase particle size\\
Destabilize colloidal particles\\
 The purpose of flocculation\\
Destabilize colloidal particles\\
Increase particle size\\
Decrease sludge\\
Decrease filter run times\\
 Primary coagulant aids used in treatment process are\\
Poly-aluminum chloride\\
Aluminum sulfate\\
Ferric chloride\\
All of the Above\\
 How do water agencies monitor the effectiveness of their filtration process?\\
Alkalinity\\
Conductivity\\
Turbidity\\
$\mathrm{pH}$\\
Flocculation is used to enhance\\
Number of particle collisions to increase floc\\
Charge neutralization\\
Dispersion of chemicals in water\\
Settling speed of floc\\
 If there is a problem with floc formation, what would you consider changing?\\
Adjust coagulant dose\\
Stay the course\\
Adjust mixing intensity\\
Both $A$ \& $C$\\
 Which step in the treatment process is the shortest?\\
Filtration\\
Sedimentation\\
Flocculation\\
Coagulation\\
 To lower the $\mathrm{pH}$ for enhanced coagulation the operator will add\\
Chlorine\\
Sulfuric acid\\
Lime\\
Caustic Soda\\
 The flocculation process lasts how long?\\
Seconds\\
5-10 minutes\\
15-45 minutes\\
Over an hour\\
 The function of a flocculation basin is to\\
Settle colloidal particles\\
Destabilize colloidal particles\\
Mix chemicals\\
Allow suspended particles to grow\\
The treatment process that involves coagulation, flocculation, sedimentation, and filtration is known as\\
Direct filtration\\
Slow sand Filtration\\
Conventional treatment\\
Pressure filtration\\
 Sedimentation produces waste known as\\
Backwash water\\
Sludge\\
Waste water\\
Mud\\
 What kind of process is the sedimentation step?\\
Physical\\
Chemical\\
Biological\\
Direct\\
 The weirs at the effluent of a sedimentation basin are also called\\
Effluent weirs\\
Baffling\\
Launders\\
Spokes\\
 Sedimentation is used in water treatment plants to\\
Settle pathogenic material\\
Destabilize particles\\
Disinfect water\\
Reduce loading on Filters\\
 Scouring is a term that describes conditions in a sedimentation tank which\\
Could impact the rest of treatment process\\
Higher flow rates in the sludge zone\\
Re-suspends settle sludge\\
All of the above\\
The four zones in a Sedimentation basin include\\
Inlet, sedimentation, sludge, outlet\\
Inlet, filter, waste, outlet\\
Inlet, top, bottom, outlet\\
Surface, sedimentation, sludge, outlet\\
The removal and inactivation requirement for Giardia is?\\
$99.9 \%$\\
$99.99 \%$\\
$99.00 \%$\\
$90 \%$\\
Short circuiting in a sedimentation basin could be caused by\\
Surface wind\\
Ineffective weir placement, or weirs covered in algae\\
Poor baffling in sedimentation inlet zone\\
All of the Above\\
How much solids should be removed during sedimentation?\\
$95 \%$ or more\\
$80-95 \%$\\
$70-80 \%$\\
$60-70 \%$\\
The type of basin that includes coagulation and flocculation is\\
Rectangular\\
Triangular\\
Up-Flow\\
None of the above\\
Recarbonation basins are used to stabilize water after\\
Filtration\\
Disinfection\\
Softening\\
Coagulation\\
Which of the following is an effective way for removing iron water?\\
	adding baffles\\
	adding sodium chloride\\
	aeration and filtration\\
	flash mixing\\
How can iron bacteria be controlled in a water distribution system?\\
a.	by aeration\\
b.	filtration\\
c.	chlorination\\
d.	precipitation\\
Which of the following is a hazard when handling hydrofluosilicic acid?\\
a.	fire\\
b.	explosion\\
c.	corrosion\\
d.	inhalation\\
Trihalomenthane may be partially removed from water by:\\
a.	fluoridation\\
b.	chlorination\\
c.	oxidation\\
d.	ultraviolet radiation\\
Which of the following forms of iron is most soluble in water?\\
a. Ferric (Fe$^{+3}$)\\
b. Ferric hydroxide [Fe(OH$_3$)]\\
c) Ferrous (Fe$^{+2}$)\\
d. Ferrous oxide (FeO)\\
Two fundamental treatment requirements for public water systems using surface sources are\\
a. Coagalat1on and sedimentation\\
b. Lime softening and disinfection\\
c. Filtration and aeration \\
d. Disinfection and filtration\\
A zeolite softening unit will replace calcium and magnesium ions with \rule{1.5cm}{0.3mm} ions.\\
a. Fluoride\\
b. Iron\\
c. Sodium\\
d. Sulfur\\
One use of polyphosphates is to:\\
a. Control algae\\
b. Improve taste\\
c. Sequester iron and manganese\\
d. Kill bacteria\\
An acceptable means of corrosion control for relatively small systems is\\
a. Activated carbon\\
b. Lime-soda ash softening\\
c. pH control\\
d. zeolite softening\\
Which of the following chemicals will most likely keep iron in suspension?\\
a. Chlorine\\
b. Fluoride\\
c. Polyphosphate\\
d. Lime inhibitor\\
Lead in drinking water can result in\\
a. Impaired mental functioning in children\\
b. Prostate cancer in men\\
c. Stomach and intestinal disorders\\
d. Reduced white blood cell count\\
  If raw water turbidity changed from 10 to 300 turbidity units and the finished water turbidity had increased from $0.1$ to 1.0 turbidity units, the unit process having the most impact to correct this situation is\\
a. Coagulation\\
b. Sedimentation\\
c. Filtration\\
d. Disinfection\\
The problem caused by dissolved carbon dioxide in the water of the distribution system is\\
a. increased Trihalomethanes\\
b. Corrosion\\
c. Excessive encrustation\\
d. Tastes and odors\\
The presence of the coliform group of bacteria in water indicates\\
a. Contamination\\
b. Inadequate disinfection\\
c. Improper sampling\\
d. Taste and odor problems\\
  The granular filtration process is designed to reduce\\
a. Calcium and magnesium sulfates\\
b. True color\\
c. Total dissolved solids\\
d. Turbidity\\
The presence of the coliform group of bacteria in water indicates\\
a. Contamination\\
b. Inadequate disinfection\\
c. Improper sampling\\
d. Taste and odor problems\\
Aeration in water treatment plants is used to\\
a. Lower the $\mathrm{pH}$\\
b. Reduce concentrations of dissolved gasses\\
c. Reduce turbidity\\
d. Stabilize chlorine residuals\\
What can the operator do if iron fouling appears to be a problem in an ion exchange softener?\\
a. Decrease the strength of the brine used in the regeneration stage\\
b. Increase backwash flow rates\\
c. Inçrease duration of backwash stage\\
d. Increase duration of service stage\\
  At what $\mathrm{pH}$ would a chlorinated water have the highest concentration of hypochlorous acid?\\
a. 5\\
b. 7\\
c. 9\\
d. 11\\
One use of polyphosphates is to\\
a. Control algae\\
b. Improve taste\\
c. Sequester iron and manganese\\
d. Kill bacteria\\
Which of the following can cause tastes and odors in a water supply?\\
a. Dissolved zinc\\
b. Algae\\
c.  High pH\\
d.  Low pH\\
What happens when lime is fed to water for corrosion control?\\
a. Alkalinity is decreased\\
b. CO2 does not change\\
c. Turbidity is decreased\\
d.  pH is increased\\
The main characteristic of raw water that enables algae to grow is\\
a. Presence of copper sulfate\\
b. Low pH\\
c. High hardness\\
d. Presence of nutrients\\
The type of corrosion caused by the use of dissimilar metal in a water system is\\
a. Caustic corrosion\\
b. Galvanic corrosion\\
c. Oxygen corrosion\\
d. Tubercular corrosion\\
  A zeolite softening unit will replace calcium and magnesium ions with ions.\\
a. Fluoride\\
b. Iron\\
c. Sodium\\
d. Sulfur\\
Two fundamental treatment requirements for public water systems using surface sources are\\
a. Coagulation and sedimentation\\
b. Lime softening and disinfection\\
c. Filtration and aeration\\
d.  Disinfection and filtration\\
A method used to soften water is\\
a. Aeration\\
b. Sedimentation\\
c. Ion exchange\\
d. Adsorption\\
The main characteristic of raw water that enables algae to grow is\\
a. Presence of copper sulfate\\
b. Low pH\\
c. High hardness\\
d. Presence of nutrients\\
What happens when lime is fed to water for corrosion control?\\
a. Alkalinity is decreased\\
b. $\mathrm{CO}_{2}$ does not change\\
c. Turbidity is decreased\\
d. $\mathrm{pH}$ is increased\\
Which of the following chemicals will most likely keep iron in suspension?\\
a. Chlorine\\
b. Fluoride\\
c. Polyphosphate\\
d. Lime inhibitor\\
If raw water turbidity changed from 10 to 300 turbidity units and the finished water turbidity had increased from $0.1$ to 1.0 turbidity units, the unit process having the most impact to correct this situation is\\
a. Coagulation\\
b. Sedimentation\\
c. Filtration\\
d. Disinfection\\
The granular filtration process is designed to reduce\\
a. Calcium and magnesium sulfates\\
b. True color\\
c. Total dissolved solids\\
d. TurbidityAeration in water treatment plants is used to\\
a. Lower the $\mathrm{pH}$\\
b. Reduce concentrations of dissolved gasses\\
c. Reduce turbidity\\
d. Stabilize chlorine residuals\\
What can the operator do if iron fouling appears to be a problem in an ion exchange softener?\\
a. Decrease the strength of the brine used in the regeneration stage\\
b. Increase backwash flow rates\\
c. Inçrease duration of backwash stage\\
d. Increase duration of service stage\\
Trihalomenthane may be partially removed from water by:\\
a. fluoridation\\
b. chlorination\\
c. oxidation\\
d. ultraviolet radiation\\
Temporary cloudiness in a freshly drawn sample of tap water may be caused by:\\
a. air\\
b. chlorine\\
c. hardness\\
d. silica\\
Two fundamental treatment requirements for public water systems using surface sources are\\
a. Coagulation and sedimentation\\
b. Lime softening and disinfection\\
c. Filtration and aeration\\
d. Disinfection and filtration\\
A zeolite softening unit will replace calcium and magnesium ions with ions.\\
a. Fluoride\\
b. Iron\\
c. Sodium\\
d. Sulfur\\
What happens when lime is fed to water for corrosion control?\\
a. Alkalinity is decreased\\
b. CO$_2$ does not change\\
c. Turbidity is decreased\\
d. pH is increased\\

