%\chapterimage{QuizCover} 
% \chapter*{Treatment}
\begin{enumerate}











  \item What is the purpose of coagulation and flocculation?\\
a. control corrosion\\
b. to kill disease causing organisms\\
c. to remove leaves, sticks, and fish debris\\
d. *to remove particulate impurities and suspended matter\\
  \item How are filter production (capacity) rates measured?\\
a. Mgd/sq.ft.\\
b. *Gpm/sq.ft.\\
c. Gpm\\
d. Mgd\\
  \item Why should a filter be drained if it is going to be out-of-service for a prolonged period?\\
a. to allow the media to dry out\\
b. to save water\\
c. to prevent the filter from floating on groundwater levels\\
d. *to avoid algal growth\\

  \item Which of the following are commonly used coagulation chemicals?\\
a. hypochlorites and free chlorine\\
b. sodium and potassium chlorides\\
c. *alum and polymers\\
d. bleach and HTH\\
  \item How can an operator tell if a filter is NOT completely cleaned after backwashing?\\
a. *the initial headloss is on the high side\\
b. the backwash rate was too slow\\
c. mudballs are NOT present\\
d. backwashing pumping rate is too low\\

  \item Flocculation is defined as\\
a. *the gathering of fine particles after coagulation by gentle mixing\\
b. clumps of bacteria\\
c. the capacity of water to neutralize acids\\
d. a high molecular weight of compounds that have negative charges\\

  \item A multi-barrier water filtration plant that contains a flash mix, a coagulation/flocculation zone, sedimentation, filtration and a clear well is considered to be a\\
a. community special treatment plant\\
b. direct filtration plant\\
c. reverse osmosis plant\\
d. *conventional filtration plant\\
e. traditional plant 

\item The filtration unit process usually\\
a. is located at the beginning of a filtration plant\\
b. *follows the coagulation/flocculation/sedimentation processes\\
c. is located after the clear well area\\
$\mathrm{d}$. is located on the plant effluent line after the clearwell\\

  \item Filters are generally backwashed when the loss-of-head indicator registers a certain set value, such as 6-ft, or upon a certain time, say 48-hours, or upon a rise in\\
a. alkalinity\\
b. a jar-test result\\
c. *turbidity\\
d. temperature\\

\item What is a method of reducing hardness?\\
a.  *Softening\\
b.  Hardening\\
c.  Lightning\\
d.  Flashing\\

\item The solid that adsorbs a contaminant is called the:\\
a.  *Adsorbent\\
b.  Adsorbate\\
c.  Sorbet\\
d.  Rock\\

\item The adsorption process is used to remove:\\
a.  *Organics or inorganics\\
b.  Bugs or salts\\
c.  Organisms or dirt\\
d.  Color or particles\\

\item Describe two primary methods used to control taste and odor?\\
a.  *Oxidation and adsorption\\
b.  Filtration and sedimentation\\
c.  Mixing and coagulation\\
d.  Sedimentation and clarification\\

\item What is the recommended loading rate for copper sulfate for algae control at an alkalinity greater than 50 mg/L?
\begin{enumerate}
\item 0.9 lb of copper sulfate per acre of surface area
\item 1.9 lb of copper sulfate per acre of surface area
\item 2-4 lb of copper sulfate per acre of surface area
\item.4 lb of copper sulfate per acre of surface area
\end{enumerate}

\item If ammonia vapor is passed over a chlorine leak in a cylinder valve, the presence of the leak is indicated by a
\begin{enumerate}
\item Yellow cloud
\item White cloud
\item Gray cloud
\item Brown cloud
\end{enumerate}

\item What is the recommended minimum contact time water mains with the chlorine slug method?
\begin{enumerate}
\item 3 hours
\item 6 hours
\item 10 hours
\item 12 hours
\end{enumerate}

\item The basic goal for water treatment is to \rule{2cm}{0.3pt}.
\begin{enumerate}
\item Protect public health
\item Make it clear
\item Make it taste good
\item Get stuff out
\end{enumerate}

\item Greensand can be operated in either \rule{2cm}{0.5pt} regeneration or \rule{2cm}{0.5pt} regeneration modes.
\begin{enumerate}
\item Continuous or intermittent
\item Fast or slow
\item Hot or cold
\item Constant or unusual
\end{enumerate}

\item The two most common types of chlorine disinfection by-products include:
\begin{enumerate}
\item TTHM and HAA5
\item TTHA of HMM5
\item Turbidity and color
\item Chloride and fluoride
\end{enumerate}

\item GAC contactors are used to reduce the amount of \rule{2cm}{0.5pt} contaminants in water.
\begin{enumerate}
\item Inorganic
\item Turbidity
\item Particle
\item Organic
\end{enumerate}

\item List the five types of surface water filtration systems.
\begin{enumerate}
\item Bag filtration, cartridge filtration, fine filtration, coarse filtration, media filtration
\item Conventional treatment, direct filtration, slow sand filtration, diatomaceous earth filtration, membrane filtration
\item Turbidity filtration, color filtration, bag filtration, fine filtration, media filtration
\item None of the above
\end{enumerate}

\item Describe two primary methods used to control taste and odor?
\begin{enumerate}
\item Oxidation and adsorption
\item Filtration and sedimentation
\item Mixing and coagulation
\item Sedimentation and clarification
\end{enumerate}

\item The adsorption process is used to remove:
\begin{enumerate}
\item Organics or inorganics
\item Bugs or salts
\item Organisms or dirt
\item Color or particles
\end{enumerate}

\item The solid that adsorbs a contaminant is called the:
\begin{enumerate}
\item Adsorbent
\item Adsorbate
\item Sorbet
\item Rock
\end{enumerate}

\item What is a method of reducing hardness?
\begin{enumerate}
\item Softening
\item Hardening
\item Lightning
\item Flashing
\end{enumerate}


\item Bag and cartridge filters are used to remove which two pathogenic microorganisms?
\begin{enumerate}
\item Viruses and giardia
\item Giardia and cryptosporidium
\item Viruses and bacteria
\item None of the above
\end{enumerate}

\item The process of cleaning a filter by pumping water up through the filter media is called \rule{2cm}{0.3pt} the filter.
\begin{enumerate}
\item Backwashing
\item Rewashing
\item Purging
\item Lifting
\end{enumerate}

\item In a typical water treatment plant, alum would be added into the \rule{2cm}{0.3pt} mixer.
\begin{enumerate}
\item Speed
\item Large
\item Slow
\item Flash
\end{enumerate}

\item When comparing conventional treatment with direct filtration, what process unit is in the conventional treatment plant that is not in the direct filtration plant?
\begin{enumerate}
\item Filter
\item Clarifier
\item Mixer
\item Detention
\end{enumerate}

\item List the basic processes, in the proper order, for a conventional treatment plant.
\begin{enumerate}
\item Coagulation, flocculation, sedimentation, filtration
\item Flocculation, coagulation, sedimentation, filtration
\item Filtration, coagulation, flocculation, sedimentation
\item Coagulation, sedimentation, flocculation, filtration
\end{enumerate}

\item The four most common oxidants include:
\begin{enumerate}
\item Chlorine, potassium permanganate, ozone, chlorine dioxide
\item Chlorides, soap, air, coagulants
\item Air, chemicals, sodium, chloride
\item Flocculants, coagulants, sediments, granules
\end{enumerate}

\item  When operating a filter, one of the operational concerns is the difference between the pressure or head on top of the filter and the pressure or head at the bottom of the filter. This difference is called \rule{2cm}{0.3pt} pressure.
\begin{enumerate}
\item Different
\item Differential
\item High
\item Low
\end{enumerate}

\item  What type of polymer is used to improve the efficiency of the sedimentation
process?
\begin{enumerate}
\item Cationic
\item Nonionic
\item Anionic
\item All of the above
\end{enumerate}

\item A(n) \rule{2cm}{0.3pt} polymer is commonly used as a coagulant.
\begin{enumerate}
\item Anionic
\item Cationic
\item Nonionic
\item Ionic
\end{enumerate}


\item A(n) \rule{2cm}{0.3pt} polymer is used to enhance flocculation.
\begin{enumerate}
\item Anionic
\item Cationic
\item Nonionic
\item Ionic
\end{enumerate}

\item Al$_2$(SO$_4$)$_3$ • 18H$_2$O is the chemical formula for:
\begin{enumerate}
\item Alum
\item Iron
\item Manganese
\item Lead
\end{enumerate}

\item Particles that are less than 1 $\mu$m in size and will not settle easily and are called:
\begin{enumerate}
\item Light particles
\item Colloidal particles
\item Colored particles
\item Flat particles
\end{enumerate}

\item The sedimentation portion of water treatment is also called a(n):
\begin{enumerate}
\item Clarifier
\item Filter
\item Adsorber
\item Water treater
\end{enumerate}

\item Slowly agitating coagulated materials is the process of:
\begin{enumerate}
\item Flocculation
\item Coagulation
\item Sedimentation
\item Filtration
\end{enumerate}

\item The process of decreasing the stability of colloids in water is called:
\begin{enumerate}
\item Flocculation
\item Coagulation
\item Sedimentation
\item Clarification
\end{enumerate}

\item The chemical oxidation process in water treatment is typically used to aid in the
removal of :
\begin{enumerate}
\item Organic contaminants
\item Inorganic contaminants
\item Large contaminants
\item None of the above
\end{enumerate}

\item Flocculation, sedimentation, filtration, and adsorption are \rule{2cm}{0.3pt}
processes.
\begin{enumerate}
\item Physical
\item Chemical
\item Biological
\item Mechanical
\end{enumerate}

\item Oxidation, coagulation, and disinfection are \rule{2cm}{0.3pt} processes.
\begin{enumerate}
\item Physical
\item Chemical
\item Biological
\item Mechanical
\end{enumerate}

\item A precipitate can be formed after which one of the following processes:
\begin{enumerate}
\item Oxidation
\item Flocculation
\item Filtration
\item Adsorption
\end{enumerate}

\item Water that is safe to drink is called \rule{1cm}{0.5pt}  water.
\begin{enumerate}
\item Potable
\item Palatable
\item Good
\item Clear
\end{enumerate}

\item The type of organisms that can cause disease are said to be \rule{1cm}{0.5pt} microorganisms.
\begin{enumerate}
\item Bad
\item Pathogenic
\item Undesirable
\item Sick
\end{enumerate}

\item The basic goal for water treatment is to \rule{1cm}{0.5pt}.
\begin{enumerate}
\item Protect public health
\item Make it clear
\item Make it taste good
\item Get stuff out
\end{enumerate}

\item Four types of aesthetic contaminants in water include the following:
\begin{enumerate}
\item Odor, turbidity, color, hydrogen sulfide gas
\item Pathogens, microorganisms, arsenic, disinfection by-products
\end{enumerate}

\item What does mg/L stand for?
\begin{enumerate}
\item Microorganisms/Liter
\item Milligrams/Loser
\item Milligrams/Liter
\item None of the above
\end{enumerate}

\item Disinfection by-products are a product of:
\begin{enumerate}
\item Filtration
\item Disinfection
\item Sedimentation
\item Adsorption
\end{enumerate}

\item Acute contaminants are those that can cause sickness after:
\begin{enumerate}
\item Prolonged exposure
\item Low levels or low exposure
\end{enumerate}

\item Chronic contaminants are those that can cause sickness after:
\begin{enumerate}
\item Prolonged exposure
\item Low levels or low exposure
\end{enumerate}

\item TTHMs and HAA5s can affect:
\begin{enumerate}
\item Health
\item Aesthetics
\item Color
\item Odor
\end{enumerate}

\item Oxidation, coagulation, and disinfection are \rule{1cm}{0.5pt}  processes.
\begin{enumerate}
\item Physical
\item Chemical
\item Biological
\item Mechanical
\end{enumerate}

\item Flocculation, sedimentation, filtration, and adsorption are \rule{1cm}{0.5pt} processes.
\begin{enumerate}
\item Physical
\item Chemical
\item Biological
\item Mechanical
\end{enumerate}

\item A precipitate can be formed after which one of the following processes:
\begin{enumerate}
\item Oxidation
\item Flocculation
\item Filtration
\item Adsorption
\end{enumerate}

\item Giardia and cryptosporidium are a type of:
\begin{enumerate}
\item Mineral
\item Organism
\item Color
\item Bird
\end{enumerate}

14. The chemical oxidation process in water treatment is typically used to aid in the
removal of :
\begin{enumerate}
\item Organic contaminants
\item Inorganic contaminants
\item Large contaminants
\item None of the above
\end{enumerate}

\item The process of decreasing the stability of colloids in water is called:
\begin{enumerate}
\item Flocculation
\item Coagulation
\item Sedimentation
\item Clarification
\end{enumerate}

\item Slowly agitating coagulated materials is the process of:
\begin{enumerate}
\item Flocculation
\item Coagulation
\item Sedimentation
\item Filtration
\end{enumerate}

\item The sedimentation portion of water treatment is also called a(n):
\begin{enumerate}
\item Clarifier
\item Filter
\item Adsorber
\item Water treater
\end{enumerate}

\item Particles that are less than 1 $\mu\text{m}$ in size and will not settle easily and are called:
\begin{enumerate}
\item Light particles
\item Colloidal particles
\item Colored particles
\item Flat particles
\end{enumerate}

\item One micrometer is also equal to:
\begin{enumerate}
\item 0.1 mm
\item 0.0001 mm
\item 0.001 mm
\item 1 m
\end{enumerate}

\item Particles less than 0.45 $\mu\text{m}$ in size are considered to be:
\begin{enumerate}
\item Dissolved
\item Really little
\item Colored particles
\item Flat particles
\end{enumerate}

\item Turbidity is measured as:
\begin{enumerate}
\item Mg/L
\item mL
\item gpm
\item NTU
\end{enumerate}

\item Al2(SO4)3 • 18H20 is the chemical formula for:
\begin{enumerate}
\item Alum
\item Iron
\item Manganese
\item Lead
\end{enumerate}

\item A(n) \rule{1cm}{0.5pt}  polymer is commonly used as a coagulant.
\begin{enumerate}
\item Anionic
\item Cationic
\item Nonionic
\item Ionic
\end{enumerate}

\item A(n) \rule{1cm}{0.5pt}  polymer is used to enhance flocculation.
\begin{enumerate}
\item Anionic
\item Cationic
\item Nonionic
\item Ionic
\end{enumerate}

\item The concentration of a chemical added to the water is measured in:
\begin{enumerate}
\item mL
\item mg
\item mg/L
\item Liters
\end{enumerate}

\item The quantity of chlorine remaining after primary disinfection is called a
\rule{1cm}{0.5pt}  residual.
\begin{enumerate}
\item Chlorine
\item Permaganate
\item Hot
\item Cold
\end{enumerate}

\item Primary disinfectants are used to \rule{1cm}{0.5pt}  microorganisms.
\begin{enumerate}
\item Hurt
\item Inactivate
\item Burn up
\item Evaporate
\end{enumerate}

\item Secondary disinfectants are used to provide a \rule{1cm}{0.5pt}  in the distribution system.
\begin{enumerate}
\item Color
\item Chemical
\item Smell
\item Residual
\end{enumerate}

\item What type of polymer is used to improve the efficiency of the sedimentation
process?
\begin{enumerate}
\item Cationic
\item Nonionic
\item Anionic
\item All of the above
\end{enumerate}

\item When operating a filter, one of the operational concerns is the difference between the pressure or head on top of the filter and the pressure or head at the bottom of the filter. This difference is called \rule{1cm}{0.5pt}  pressure.
\begin{enumerate}
\item Different
\item Differential
\item High
\item Low
\end{enumerate}

\item List the basic processes, in the proper order, for a conventional treatment plant.
\begin{enumerate}
\item Coagulation, flocculation, sedimentation, filtration
\item Flocculation, coagulation, sedimentation, filtration
\item Filtration, coagulation, flocculation, sedimentation
\item Coagulation, sedimentation, flocculation, filtration
\end{enumerate}

\item The four most common oxidants include:
\begin{enumerate}
\item Chlorine, potassium permanganate, ozone, chlorine dioxide
\item Chlorides, soap, air, coagulants
\item Air, chemicals, sodium, chloride
\item Flocculants, coagulants, sediments, granules
\end{enumerate}

\item When comparing conventional treatment with direct filtration, what process unit is in the conventional treatment plant that is not in the direct filtration plant?
\begin{enumerate}
\item Filter
\item Clarifier
\item Mixer
\item Detention
\end{enumerate}

\item In a typical water treatment plant, alum would be added into the \rule{1cm}{0.5pt}  mixer.
\begin{enumerate}
\item Speed
\item Large
\item Slow
\item Flash
\end{enumerate}

\item The process of cleaning a filter by pumping water up through the filter media is called \rule{1cm}{0.5pt}  the filter.
\begin{enumerate}
\item Backwashing
\item Rewashing
\item Purging
\item Lifting
\end{enumerate}

\item Bag and cartridge filters are used to remove which two pathogenic microorganisms?
\begin{enumerate}
\item Viruses and giardia
\item Giardia and cryptosporidium
\item Viruses and bacteria
\item None of the above
\end{enumerate}

\item List the four types of membrane filtration processes commonly used in water
treatment.
\begin{enumerate}
\item MF, UF, NF, and RO
\item MNF, UOF, NOF, and ROO
\item CFM, FM, FN, and OR
\item None of the above
\end{enumerate}

\item What is a method of reducing hardness?
\begin{enumerate}
\item Softening
\item Hardening
\item Lightning
\item Flashing
\end{enumerate}

\item Adsorption of a substance involves its accumulation onto the surface of a:
\begin{enumerate}
\item Solid
\item Rock
\item Pellet
\item Snow ball
\end{enumerate}

\item The solid that adsorbs a contaminant is called the:
\begin{enumerate}
\item Adsorbent
\item Adsorbate
\item Sorbet
\item Rock
\end{enumerate}

\item The adsorption process is used to remove:
\begin{enumerate}
\item Organics or inorganics
\item Bugs or salts
\item Organisms or dirt
\item Color or particles
\end{enumerate}

\item Describe two primary methods used to control taste and odor?
\begin{enumerate}
\item Oxidation and adsorption
\item Filtration and sedimentation
\item Mixing and coagulation
\item Sedimentation and clarification
\end{enumerate}



\item List the five types of surface water filtration systems.
\begin{enumerate}
\item Bag filtration, cartridge filtration, fine filtration, coarse filtration, media filtration
\item Conventional treatment, direct filtration, slow sand filtration, diatomaceous
earth filtration, membrane filtration
\item Turbidity filtration, color filtration, bag filtration, fine filtration, media filtration
\item None of the above
\end{enumerate}

\item GAC contactors are used to reduce the amount of \rule{1cm}{0.5pt}  contaminants in water.
\begin{enumerate}
\item Inorganic
\item Turbidity
\item Particle
\item Organic
\end{enumerate}

\item Greensand can be operated in either \rule{1cm}{0.5pt}  regeneration or \rule{1cm}{0.5pt} regeneration modes.
\begin{enumerate}
\item Continuous or intermittent
\item Fast or slow
\item Hot or cold
\item Constant or unusual
\end{enumerate}

\item  What is the cause of taste and odor problems in raw surface water?\\
\begin{enumerate}
\item Copper sulfate\\
\item Blue-green algae\\
\item Oxygen\\
\item Lake turnover
\end{enumerate}

\item  What chemical reduces blue-green algae growth?\\
\begin{enumerate}
\item Chlorine\\
\item Caustic Soda\\
\item Copper Sulfate\\
\item Alum
\end{enumerate}


\item What is the purpose of adding fluoride to drinking water?
\begin{enumerate}
\item Increase tooth decay
\item Reduce tooth decay
\item Make teeth white
\item Government conspiracy
\end{enumerate}

\item The optimal coagulant dose is determined by a\\
\begin{enumerate}
\item Chlorine Test\\
\item Flocculation test\\
\item Jar Test\\
\item Coagulation test
\end{enumerate}

\item  The most common primary coagulant is\\
\begin{enumerate}
\item Alum\\
\item Cationic polymer\\
\item Fluoride\\
\item Anionic polymer
\end{enumerate}

\item  Bacteria and Viruses belong to a particle size known as\\
\begin{enumerate}
\item Suspended\\
\item Dissolved\\
\item Strained\\
\item Colloidal
\end{enumerate}

\item  The purpose of coagulation is to\\
\begin{enumerate}
\item Increase filter run times\\
\item Increase sludge\\
\item Increase particle size\\
\item Destabilize colloidal particles
\end{enumerate}

\item  The purpose of flocculation\\
\begin{enumerate}
\item Destabilize colloidal particles\\
\item Increase particle size\\
\item Decrease sludge\\
\item Decrease filter run times
\end{enumerate}

\item  Primary coagulant aids used in treatment process are\\
\begin{enumerate}
\item Poly-aluminum chloride\\
\item Aluminum sulfate\\
\item Ferric chloride\\
\item All of the Above
\end{enumerate}

\item  How do water agencies monitor the effectiveness of their filtration process?\\
\begin{enumerate}
\item Alkalinity\\
\item Conductivity\\
\item Turbidity\\
\item $\mathrm{pH}$
\end{enumerate}


\item Flocculation is used to enhance\\
\begin{enumerate}
\item Number of particle collisions to increase floc\\
\item Charge neutralization\\
\item Dispersion of chemicals in water\\
\item Settling speed of floc
\end{enumerate}

\item  If there is a problem with floc formation, what would you consider changing?\\
\begin{enumerate}
\item Adjust coagulant dose\\
\item Stay the course\\
\item Adjust mixing intensity\\
\item Both $A$ \& $C$
\end{enumerate}

\item  Which step in the treatment process is the shortest?\\
\begin{enumerate}
\item Filtration\\
\item Sedimentation\\
\item Flocculation\\
\item Coagulation
\end{enumerate}

\item  To lower the $\mathrm{pH}$ for enhanced coagulation the operator will add\\
\begin{enumerate}
\item Chlorine\\
\item Sulfuric acid\\
\item Lime\\
\item Caustic Soda
\end{enumerate}

\item  The flocculation process lasts how long?\\
\begin{enumerate}
\item Seconds\\
\item 5-10 minutes\\
\item 15-45 minutes\\
\item Over an hour
\end{enumerate}

\item  The function of a flocculation basin is to\\
\begin{enumerate}
\item Settle colloidal particles\\
\item Destabilize colloidal particles\\
\item Mix chemicals\\
\item Allow suspended particles to grow
\end{enumerate}

\item The treatment process that involves coagulation, flocculation, sedimentation, and filtration is known as\\
\begin{enumerate}
\item Direct filtration\\
\item Slow sand Filtration\\
\item Conventional treatment\\
\item Pressure filtration
\end{enumerate}

\item  Sedimentation produces waste known as\\
\begin{enumerate}
\item Backwash water\\
\item Sludge\\
\item Waste water\\
\item Mud
\end{enumerate}

\item  What kind of process is the sedimentation step?\\
\begin{enumerate}
\item Physical\\
\item Chemical\\
\item Biological\\
\item Direct
\end{enumerate}

\item  The weirs at the effluent of a sedimentation basin are also called\\
\begin{enumerate}
\item Effluent weirs\\
\item Baffling\\
\item Launders\\
\item Spokes
\end{enumerate}

\item  Sedimentation is used in water treatment plants to\\
\begin{enumerate}
\item Settle pathogenic material\\
\item Destabilize particles\\
\item Disinfect water\\
\item Reduce loading on Filters
\end{enumerate}

\item  Scouring is a term that describes conditions in a sedimentation tank which\\
\begin{enumerate}
\item Could impact the rest of treatment process\\
\item Higher flow rates in the sludge zone\\
\item Re-suspends settle sludge\\
\item All of the above
\end{enumerate}

The four zones in a Sedimentation basin include\\
\begin{enumerate}
\item Inlet, sedimentation, sludge, outlet\\
\item Inlet, filter, waste, outlet\\
\item Inlet, top, bottom, outlet\\
\item Surface, sedimentation, sludge, outlet
\end{enumerate}

\item The removal and inactivation requirement for Giardia is?\\
\begin{enumerate}
\item $99.9 \%$\\
\item $99.99 \%$\\
\item $99.00 \%$\\
\item $90 \%$
\end{enumerate}

\item Short circuiting in a sedimentation basin could be caused by\\
\begin{enumerate}
\item Surface wind\\
\item Ineffective weir placement, or weirs covered in algae\\
\item Poor baffling in sedimentation inlet zone\\
\item All of the Above
\end{enumerate}

\item How much solids should be removed during sedimentation?\\
\begin{enumerate}
\item $95 \%$ or more\\
\item $80-95 \%$\\
\item $70-80 \%$\\
\item $60-70 \%$
\end{enumerate}

\item The type of basin that includes coagulation and flocculation is\\
\begin{enumerate}
\item Rectangular\\
\item Triangular\\
\item Up-Flow\\
\item None of the above
\end{enumerate}

\item Recarbonation basins are used to stabilize water after
\begin{enumerate}
\item Filtration
\item Disinfection
\item Softening
\item Coagulation
\end{enumerate}

\item Which of the following is an effective way for removing iron water?
\begin{enumerate}
\item 	adding baffles
\item 	adding sodium chloride
\item 	aeration and filtration
\item 	flash mixing
\end{enumerate}

\item How can iron bacteria be controlled in a water distribution system?\\
a.	by aeration\\
b.	filtration\\
c.	chlorination\\
d.	precipitation

\item Which of the following is a hazard when handling hydrofluosilicic acid?\\
a.	fire\\
b.	explosion\\
c.	corrosion\\
d.	inhalation\\

\item Trihalomenthane may be partially removed from water by:\\
a.	fluoridation\\
b.	chlorination\\
c.	oxidation\\
d.	ultraviolet radiation\\

\item Which of the following forms of iron is most soluble in water?\\
a. Ferric (Fe$^{+3}$)\\
b. Ferric hydroxide [Fe(OH$_3$)]\\
c) Ferrous (Fe$^{+2}$)\\
d. Ferrous oxide (FeO)\\

\item Two fundamental treatment requirements for public water systems using surface sources are\\
a. Coagalat1on and sedimentation\\
b. Lime softening and disinfection\\
c. Filtration and aeration \\
d. Disinfection and filtration

\item A zeolite softening unit will replace calcium and magnesium ions with \rule{1.5cm}{0.3mm} ions.\\
a. Fluoride\\
b. Iron\\
c. Sodium\\
d. Sulfur\\

\item One use of polyphosphates is to:\\
a. Control algae\\
b. Improve taste\\
c. Sequester iron and manganese\\
d. Kill bacteria

\item An acceptable means of corrosion control for relatively small systems is\\
a. Activated carbon\\
b. Lime-soda ash softening\\
c. pH control\\
d. zeolite softening



\item Which of the following chemicals will most likely keep iron in suspension?\\


a. Chlorine\\

b. Fluoride\\

c. Polyphosphate\\

d. Lime inhibitor\\


\item Lead in drinking water can result in\\


a. Impaired mental functioning in children\\

b. Prostate cancer in men\\

c. Stomach and intestinal disorders\\

d. Reduced white blood cell count\\

  \item If raw water turbidity changed from 10 to 300 turbidity units and the finished water turbidity had increased from $0.1$ to 1.0 turbidity units, the unit process having the most impact to correct this situation is\\
a. Coagulation\\
b. Sedimentation\\
c. Filtration\\
d. Disinfection\\

\item The problem caused by dissolved carbon dioxide in the water of the distribution system is\\
a. increased Trihalomethanes\\
b. Corrosion\\
c. Excessive encrustation\\
d. Tastes and odors\\

\item The presence of the coliform group of bacteria in water indicates\\

a. Contamination\\

b. Inadequate disinfection\\

c. Improper sampling\\

d. Taste and odor problems\\


  \item The granular filtration process is designed to reduce\\
a. Calcium and magnesium sulfates\\

b. True color\\

c. Total dissolved solids\\

d. Turbidity\\


\item The presence of the coliform group of bacteria in water indicates\\

a. Contamination\\

b. Inadequate disinfection\\

c. Improper sampling\\

d. Taste and odor problems\\

\item Aeration in water treatment plants is used to\\

a. Lower the $\mathrm{pH}$\\

b. Reduce concentrations of dissolved gasses\\

c. Reduce turbidity\\

d. Stabilize chlorine residuals


\item What can the operator do if iron fouling appears to be a problem in an ion exchange softener?\\

a. Decrease the strength of the brine used in the regeneration stage\\

b. Increase backwash flow rates\\

c. Inçrease duration of backwash stage\\

d. Increase duration of service stage\\


  \item At what $\mathrm{pH}$ would a chlorinated water have the highest concentration of hypochlorous acid?\\
a. 5\\
b. 7\\
c. 9\\
d. 11\\

\item One use of polyphosphates is to\\



a. Control algae\\

b. Improve taste\\

c. Sequester iron and manganese\\

d. Kill bacteria\\


\item Which of the following can cause tastes and odors in a water supply?\\
a. Dissolved zinc\\
b. Algae\\
c.  High pH\\
d.  Low pH\\

\item What happens when lime is fed to water for corrosion control?\\


a. Alkalinity is decreased\\

b. CO2 does not change\\

c. Turbidity is decreased\\

d.  pH is increased\\

\item The main characteristic of raw water that enables algae to grow is\\

a. Presence of copper sulfate\\

b. Low pH\\

c. High hardness\\

d. Presence of nutrients\\


\item The type of corrosion caused by the use of dissimilar metal in a water system is\\

a. Caustic corrosion\\

b. Galvanic corrosion\\

c. Oxygen corrosion\\

d. Tubercular corrosion\\

  \item A zeolite softening unit will replace calcium and magnesium ions with ions.\\
a. Fluoride\\
b. Iron\\
c. Sodium\\
d. Sulfur\\

\item Two fundamental treatment requirements for public water systems using surface sources are\\

a. Coagulation and sedimentation\\

b. Lime softening and disinfection\\

c. Filtration and aeration\\

d.  Disinfection and filtration\\


\item A method used to soften water is\\
a. Aeration\\
b. Sedimentation\\
c. Ion exchange\\
d. Adsorption\\

\item The main characteristic of raw water that enables algae to grow is\\
a. Presence of copper sulfate\\
b. Low pH\\
c. High hardness\\
d. Presence of nutrients\\

\item What happens when lime is fed to water for corrosion control?\\
a. Alkalinity is decreased\\
b. $\mathrm{CO}_{2}$ does not change\\
c. Turbidity is decreased\\
d. $\mathrm{pH}$ is increased\\

\item Which of the following chemicals will most likely keep iron in suspension?\\
a. Chlorine\\
b. Fluoride\\
c. Polyphosphate\\
d. Lime inhibitor

\item If raw water turbidity changed from 10 to 300 turbidity units and the finished water turbidity had increased from $0.1$ to 1.0 turbidity units, the unit process having the most impact to correct this situation is\\
a. Coagulation\\
b. Sedimentation\\
c. Filtration\\
d. Disinfection

\item The granular filtration process is designed to reduce\\
a. Calcium and magnesium sulfates\\
b. True color\\
c. Total dissolved solids\\
d. Turbidity\item Aeration in water treatment plants is used to\\
a. Lower the $\mathrm{pH}$\\
b. Reduce concentrations of dissolved gasses\\
c. Reduce turbidity\\
d. Stabilize chlorine residuals\\

\item What can the operator do if iron fouling appears to be a problem in an ion exchange softener?\\
a. Decrease the strength of the brine used in the regeneration stage\\
b. Increase backwash flow rates\\
c. Inçrease duration of backwash stage\\
d. Increase duration of service stage\\

\item Trihalomenthane may be partially removed from water by:\\
a. fluoridation\\
b. chlorination\\
c. oxidation\\
d. ultraviolet radiation\\

\item Temporary cloudiness in a freshly drawn sample of tap water may be caused by:\\
a. air\\
b. chlorine\\
c. hardness\\
d. silica\\

\item Two fundamental treatment requirements for public water systems using surface sources are\\
a. Coagulation and sedimentation\\
b. Lime softening and disinfection\\
c. Filtration and aeration\\
d. Disinfection and filtration\\

\item A zeolite softening unit will replace calcium and magnesium ions with ions.\\
a. Fluoride\\
b. Iron\\
c. Sodium\\
d. Sulfur\\

\item What happens when lime is fed to water for corrosion control?
a. Alkalinity is decreased\\
b. CO$_2$ does not change\\
c. Turbidity is decreased\\
d. pH is increased\\

\item Which two chemicals are used to remove turbidity?\\
A. Soda Ash and lime\\
B. Copper sulphate and caustic soda\\
C. Alum and lime\\

\item Which of the following is considered to be a\\
coagulant aid?\\
A. Lime\\
B. Polymer\\
C. Bentonite\\
D. All of the above\\
\item Alum precipitates as\\
A. Aluminum carbonate\\
B. Aluminum sulphate\\
C. Aluminum hydroxide\\

\item Turbidity removal with alum is best accomplished at what pH?\\
A. 3.5\\
B. 5.0\\
C. 6.5\\

\item Which of the following will not lower the pH?\\
A. Alum\\
B. Carbonic acid\\
C. Ferric chloride\\
D. Sodium carbonate\\

\item  Liquid fluoride is delivered as:\\
A. Sodium Fluoride\\
B. Hydrofluorosilicic acid\\
C. Sodium Silicofluoride\\
D. Hydrofluoric acid\\


\item An upflow clarifier will have which of the following processes?\\
A. Coagulation\\
B. Flocculation\\
C. Sedimentation\\
D. All of the above\\

\item Sludge that rises to the surface of a sedimentation\\
basin is caused by:\\
A. Not removing sludge often enough\\
B. Removing sludge too often\\
C. pH is too low\\
D. Surface loading rate is too low\\

\item Pin floc leaving a sedimentation basin may indicate a\\
problem with:\\
A. Coagulation\\
B. Flocculation\\
C. Sedimentation\\
D. Disinfection\\


\item What is the backwash rate for a rapid sand filter?\\
A. 2 gpm/sq.ft.\\
B. 15 gpm/sq.ft.\\
C. 20 gpm/sq.ft.\\
D. 25 gpm/sq.ft.\\

\item What is the maximum run time for a gravity filter?\\
A. 8 hours\\
B. 20 hours\\
C. 48 hours\\
D. 100 hours\\

\item During backwash, the filter bed should expand:\\
A. 5-10\%\\
B. 15-20\%\\
C. 30-50\%\\
D. 60-80\%\\

\item If the backwash time is too short, what may result?\\
A. Too much freeboard\\
B. Mudballs\\
C. Loss of filter media\\
D. Filter breakthrough\\

\item If the filtration rate is too high, what may result?\\
A. Filter breakthrough\\
B. Mudballs\\
C. Reduction in operating costs\\
D. Lower headloss\\

\item Solids removed from a filter are most commonly removed by what method? \\
a. Adsorption \\
b. Straining \\
c. Deactivation \\
d. Flocculation \\

\item What is a typical filtration rate for slow sand filters? \\
a. 2.0-6.0 GPM/sq. ft. \\
b. 6.0-10.0 GPM/sq. ft. \\
c. 1.0-2.0 GPM/sq. ft. \\
d. 0.5-0.10 GPM/sq. ft. \\

\item In a typical conventional treatment plant, the finished water turbidity for an individual filter should be less than \\
a. 1.0 NTUs \\
b. 0.3 NTUs \\
c. 5.0 NTUs \\
d. 3.0 NTUs \\

\item A filter running under normal conditions will see head loss in a filter \\
a. Remain constant \\
b. Increase slowly \\
c. Rapidly increase \\
d. Decrease slowly \\

\item A filter must be washed if this condition is met \\
a. Head Loss \\
b. Turbidity break through \\
c. Maximum Filter run time \\
d. All of the Above \\

\item Filter performance is measured by the removal of \\
a. Oxygen \\
b. Head loss \\
c. Turbidity \\
d. Chlorine \\

\item What is the biologically active layer of a slow sand filter called? \\
a. Mixed Media \\
b. Duel Media \\
c. Sludge Layer \\
d. Schmutzdecke \\

\item The pressure drop in a filter is called \\
a. Turbidity breakthrough \\
b. Head Loss \\
c. Filtration \\
d. Backwash \\

\item What is the most common reason for putting a filter into the wash cycle? \\
a. Head loss \\
b. Filter run time \\
c. Turbidity breakthrough \\
d. Water level decrease \\

\item Formation of mud balls and excessive boiling during a wash is an indicator of \\
a. Proper backwash rate \\
b. Too low backwash rate \\
c. Excessive backwash rate \\
d. Improper chemical dose \\

\item Important processes which occur during filtration are \\
a. Sedimentation \\
b. Adsorption \\
c. Straining \\
d. All of the Above \\

\item Typical filtration rates for a conventional treatment plant are \\
a. 0.2-0.6 GPM/sq.ft. \\
b. 2.0-10.0 GPM/sq.ft. \\
c. 10.0-20.0 GPM/sq.ft. \\
d. 200-400 GPM/sq.ft. \\

\end{enumerate}

