%\chapterimage{QuizCover} 
% \chapter*{Treatment}
\begin{enumerate}
\item What is the purpose of coagulation and flocculation?\\
a. control corrosion\\
b. to kill disease causing organisms\\
c. to remove leaves, sticks, and fish debris\\
d. *to remove particulate impurities and suspended matter\\
\item How are filter production (capacity) rates measured?\\
a. Mgd/sq.ft.\\
b. $* \mathrm{Gpm} / \mathrm{sq} . \mathrm{ft}$.\\
c. $\mathrm{Gpm}$\\
d. Mgd\\
\item Why should a filter be drained if it is going to be out-of-service for a prolonged period?\\
a. to allow the media to dry out\\
b. to save water\\
c. to prevent the filter from floating on groundwater levels\\
d. *to avoid algal growth\\
\item Which of the following are commonly used coagulation chemicals?\\
a. hypochlorites and free chlorine\\
b. sodium and potassium chlorides\\
c. *alum and polymers\\
d. bleach and HTH\\
\item How can an operator tell if a filter is NOT completely cleaned after backwashing?\\
a. *the initial headloss is on the high side\\
b. the backwash rate was too slow\\
c. mudballs are NOT present\\
d. backwashing pumping rate is too low\\
\item Flocculation is defined as\\
a. *the gathering of fine particles after coagulation by gentle mixing\\
b. clumps of bacteria\\
c. the capacity of water to neutralize acids\\
d. a high molecular weight of compounds that have negative charges\\
\item A multi-barrier water filtration plant that contains a flash mix, a coagulation/flocculation zone, sedimentation, filtration and a clear well is considered to be a\\
a. community special treatment plant\\
b. direct filtration plant\\
c. reverse osmosis plant\\
d. *conventional filtration plant\\
e. traditional plant\\
\item The filtration unit process usually\\
a. is located at the beginning of a filtration plant\\
b. *follows the coagulation/flocculation/sedimentation processes\\
c. is located after the clear well area\\
d. is located on the plant effluent line after the clearwell\\
\item Filters are generally backwashed when the loss-of-head indicator registers a certain set value, such as 6-ft, or upon a certain time, say 48-hours, or upon a rise in\\
a. alkalinity\\
b. a jar-test result\\
c. *turbidity\\
d. temperature\\
\item What is a method of reducing hardness?\\
a. *Softening\\
b. Hardening\\
c. Lightning\\
d. Flashing\\
\item The solid that adsorbs a contaminant is called the:\\
a. *Adsorbent\\
b. Adsorbate\\
c. Sorbet\\
d. Rock\\
\item The adsorption process is used to remove:\\
a. *Organics or inorganics\\
b. Bugs or salts\\
c. Organisms or dirt\\
d. Color or particles\\
\item Describe two primary methods used to control taste and odor?\\
a. *Oxidation and adsorption\\
b. Filtration and sedimentation\\
c. Mixing and coagulation\\
d. Sedimentation and clarification\\
\item What is the recommended loading rate for copper sulfate for algae control at an alkalinity greater than $50 \mathrm{mg} / \mathrm{L}$ ?\\
a. $0.9 \mathrm{lb}$ of copper sulfate per acre of surface area\\
b. $1.9 \mathrm{lb}$ of copper sulfate per acre of surface area\\
c. 2-4 lb of copper sulfate per acre of surface area\\
d. $.4 \mathrm{lb}$ of copper sulfate per acre of surface area\\
\item The basic goal for water treatment is to\\
a. Protect public health\\
b. Make it clear\\
c. Make it taste good\\
d. Get stuff out\\
\item Greensand can be operated in either regeneration or regeneration modes.\\
a. Continuous or intermittent\\
b. Fast or slow\\
c. Hot or cold\\
d. Constant or unusual\\
\item The two most common types of chlorine disinfection by-products include:\\
a. TTHM and HAA5\\
b. TTHA of HMM5\\
c. Turbidity and color\\
d. Chloride and fluoride\\
\item GAC contactors are used to reduce the amount of contaminants in water.\\
a. Inorganic\\
b. Turbidity\\
c. Particle\\
d. Organic\\
\item List the five types of surface water filtration systems.\\
a. Bag filtration, cartridge filtration, fine filtration, coarse filtration, media filtration\\
b. Conventional treatment, direct filtration, slow sand filtration, diatomaceous earth filtration, membrane filtration\\
c. Turbidity filtration, color filtration, bag filtration, fine filtration, media filtration\\
d. None of the above\\
\item Describe two primary methods used to control taste and odor?\\
a. Oxidation and adsorption\\
b. Filtration and sedimentation\\
c. Mixing and coagulation\\
d. Sedimentation and clarification\\
\item The adsorption process is used to remove:\\
a. Organics or inorganics\\
b. Bugs or salts\\
c. Organisms or dirt\\
d. Color or particles\\
\item The solid that adsorbs a contaminant is called the:\\
a. Adsorbent\\
b. Adsorbate\\
c. Sorbet\\
d. Rock\\
\item What is a method of reducing hardness?\\
a. Softening\\
b. Hardening\\
c. Lightning\\
d. Flashing\\
\item Bag and cartridge filters are used to remove which two pathogenic microorganisms?\\
a. Viruses and giardia\\
b. Giardia and cryptosporidium\\
c. Viruses and bacteria\\
d. None of the above\\
\item The process of cleaning a filter by pumping water up through the filter media is called the filter.\\
a. Backwashing\\
b. Rewashing\\
c. Purging\\
d. Lifting\\
\item In a typical water treatment plant, alum would be added into the mixer.\\
a. Speed\\
b. Large\\
c. Slow\\
d. Flash\\
\item When comparing conventional treatment with direct filtration, what process unit is in the conventional treatment plant that is not in the direct filtration plant?\\
a. Filter\\
b. Clarifier\\
c. Mixer\\
d. Detention\\
\item List the basic processes, in the proper order, for a conventional treatment plant.\\
a. Coagulation, flocculation, sedimentation, filtration\\
b. Flocculation, coagulation, sedimentation, filtration\\
c. Filtration, coagulation, flocculation, sedimentation\\
d. Coagulation, sedimentation, flocculation, filtration 31. The four most common oxidants include:\\
a. Chlorine, potassium permanganate, ozone, chlorine dioxide\\
b. Chlorides, soap, air, coagulants\\
c. Air, chemicals, sodium, chloride\\
d. Flocculants, coagulants, sediments, granules\\
\item When operating a filter, one of the operational concerns is the difference between the pressure or head on top of the filter and the pressure or head at the bottom of the filter. This difference is called pressure.\\
a. Different\\
b. Differential\\
c. High\\
d. Low\\
\item What type of polymer is used to improve the efficiency of the sedimentation process?\\
a. Cationic\\
b. Nonionic\\
c. Anionic\\
d. All of the above\\
\item A(n)\_ polymer is commonly used as a coagulant.\\
a. Anionic\\
b. Cationic\\
c. Nonionic\\
d. Ionic\\
\item $\mathrm{A}(\mathrm{n}) \ldots$ polymer is used to enhance flocculation.\\
a. Anionic\\
b. Cationic\\
c. Nonionic\\
d. Ionic\\
\item $\mathrm{Al}_{2}\left(\mathrm{SO}_{4}\right)_{3} \cdot 18 \mathrm{H}_{2} \mathrm{O}$ is the chemical formula for:\\
a. Alum\\
b. Iron\\
c. Manganese\\
d. Lead\\
\item Particles that are less than $1 \mu \mathrm{m}$ in size and will not settle easily and are called:\\
a. Light particles\\
b. Colloidal particles\\
c. Colored particles\\
d. Flat particles\\
\item The sedimentation portion of water treatment is also called a(n):\\
a. Clarifier\\
b. Filter\\
c. Adsorber\\
d. Water treater\\
\item Slowly agitating coagulated materials is the process of:\\
a. Flocculation\\
b. Coagulation\\
c. Sedimentation\\
d. Filtration\\
\item The process of decreasing the stability of colloids in water is called:\\
a. Flocculation\\
b. Coagulation\\
c. Sedimentation\\
d. Clarification\\
\item The chemical oxidation process in water treatment is typically used to aid in the removal of :\\
a. Organic contaminants\\
b. Inorganic contaminants\\
c. Large contaminants\\
d. None of the above\\
\item Flocculation, sedimentation, filtration, and adsorption are processes.\\
a. Physical\\
b. Chemical\\
c. Biological\\
d. Mechanical\\
\item Oxidation, coagulation, and disinfection are\\
a. Physical\\
b. Chemical\\
c. Biological\\
d. Mechanical\\
\item A precipitate can be formed after which one of the following processes:\\
a. Oxidation\\
b. Flocculation\\
c. Filtration\\
d. Adsorption\\
\item Water that is safe to drink is called water.\\
a. Potable\\
b. Palatable\\
c. Good\\
d. Clear\\
\item The type of organisms that can cause disease are said to be microorganisms.\\
a. $\mathrm{Bad}$\\
b. Pathogenic\\
c. Undesirable\\
d. Sick\\
\item The basic goal for water treatment is to\\
a. Protect public health\\
b. Make it clear\\
c. Make it taste good\\
d. Get stuff out\\
\item Four types of aesthetic contaminants in water include the following:\\
a. Odor, turbidity, color, hydrogen sulfide gas\\
b. Pathogens, microorganisms, arsenic, disinfection by-products\\
\item What does $\mathrm{mg} / \mathrm{L}$ stand for?\\
a. Microorganisms/Liter\\
b. Milligrams/Loser\\
c. Milligrams/Liter\\
d. None of the above\\
\item Disinfection by-products are a product of:\\
a. Filtration\\
b. Disinfection\\
c. Sedimentation\\
d. Adsorption\\
\item Acute contaminants are those that can cause sickness after:\\
a. Prolonged exposure\\
b. Low levels or low exposure\\
\item Chronic contaminants are those that can cause sickness after:\\
a. Prolonged exposure\\
b. Low levels or low exposure\\
\item TTHMs and HAA5s can affect:\\
a. Health\\
b. Aesthetics\\
c. Color\\
d. Odor\\
\item Oxidation, coagulation, and disinfection are processes.\\
a. Physical\\
b. Chemical\\
c. Biological\\
d. Mechanical\\
\item Flocculation, sedimentation, filtration, and adsorption are processes.\\
a. Physical\\
b. Chemical\\
c. Biological\\
d. Mechanical\\
\item A precipitate can be formed after which one of the following processes:\\
a. Oxidation\\
b. Flocculation\\
c. Filtration\\
d. Adsorption\\
\item Giardia and cryptosporidium are a type of:\\
a. Mineral\\
b. Organism\\
c. Color\\
d. Bird\\
\item The chemical oxidation process in water treatment is typically used to aid in the removal of :\\
a. Organic contaminants\\
b. Inorganic contaminants\\
c. Large contaminants\\
d. None of the above\\
\item The process of decreasing the stability of colloids in water is called:\\
a. Flocculation\\
b. Coagulation\\
c. Sedimentation\\
d. Clarification\\
\item Slowly agitating coagulated materials is the process of:\\
a. Flocculation\\
b. Coagulation\\
c. Sedimentation\\
d. Filtration\\
\item The sedimentation portion of water treatment is also called a(n):\\
a. Clarifier\\
b. Filter\\
c. Adsorber\\
d. Water treater\\
\item Particles that are less than $1 \mu \mathrm{m}$ in size and will not settle easily and are called:\\
a. Light particles\\
b. Colloidal particles\\
c. Colored particles\\
d. Flat particles\\
\item One micrometer is also equal to:\\
a. $0.1 \mathrm{~mm}$\\
b. $0.0001 \mathrm{~mm}$\\
c. $0.001 \mathrm{~mm}$\\
d. $1 \mathrm{~m}$\\
\item Particles less than $0.45 \mu \mathrm{m}$ in size are considered to be:\\
a. Dissolved\\
b. Really little\\
c. Colored particles\\
d. Flat particles\\
\item Turbidity is measured as:\\
a. $\mathrm{Mg} / \mathrm{L}$\\
b. $\mathrm{mL}$\\
c. $\mathrm{gpm}$\\
d. NTU\\
\item $\mathrm{Al} 2(\mathrm{SO} 4) 3 \cdot 18 \mathrm{H} 20$ is the chemical formula for:\\
a. Alum\\
b. Iron\\
c. Manganese\\
d. Lead\\
\item $\mathrm{A}(\mathrm{n}) \_$polymer is commonly used as a coagulant.\\
a. Anionic\\
b. Cationic\\
c. Nonionic\\
d. Ionic\\
\item A(n)\_polymer is used to enhance flocculation.\\
a. Anionic\\
b. Cationic\\
c. Nonionic\\
d. Ionic\\
\item The concentration of a chemical added to the water is measured in:\\
a. $\mathrm{mL}$\\
b. $\mathrm{mg}$\\
c. $\mathrm{mg} / \mathrm{L}$\\
d. Liters\\
\item The quantity of chlorine remaining after primary disinfection is called a residual.\\
a. Chlorine\\
b. Permaganate\\
c. Hot\\
d. Cold\\
\item Primary disinfectants are used to microorganisms.\\
a. Hurt\\
b. Inactivate\\
c. Burn up\\
d. Evaporate\\
\item Secondary disinfectants are used to provide a in the distribution system.\\
a. Color\\
b. Chemical\\
c. Smell\\
d. Residual\\
\item What type of polymer is used to improve the efficiency of the sedimentation process?\\
a. Cationic\\
b. Nonionic\\
c. Anionic\\
d. All of the above\\
\item When operating a filter, one of the operational concerns is the difference between the pressure or head on top of the filter and the pressure or head at the bottom of the filter. This difference is called pressure.\\
a. Different\\
b. Differential\\
c. High\\
d. Low\\
\item List the basic processes, in the proper order, for a conventional treatment plant.\\
a. Coagulation, flocculation, sedimentation, filtration\\
b. Flocculation, coagulation, sedimentation, filtration\\
c. Filtration, coagulation, flocculation, sedimentation\\
d. Coagulation, sedimentation, flocculation, filtration\\
\item The four most common oxidants include:\\
a. Chlorine, potassium permanganate, ozone, chlorine dioxide\\
b. Chlorides, soap, air, coagulants\\
c. Air, chemicals, sodium, chloride\\
d. Flocculants, coagulants, sediments, granules\\
\item When comparing conventional treatment with direct filtration, what process unit is in the conventional treatment plant that is not in the direct filtration plant?\\
a. Filter\\
b. Clarifier\\
c. Mixer\\
d. Detention\\
\item In a typical water treatment plant, alum would be added into the mixer.\\
a. Speed\\
b. Large\\
c. Slow d. Flash\\
\item The process of cleaning a filter by pumping water up through the filter media is called \_ the filter.\\
a. Backwashing\\
b. Rewashing\\
c. Purging\\
d. Lifting\\
\item Bag and cartridge filters are used to remove which two pathogenic microorganisms?\\
a. Viruses and giardia\\
b. Giardia and cryptosporidium\\
c. Viruses and bacteria\\
d. None of the above\\
\item List the four types of membrane filtration processes commonly used in water treatment.\\
a. MF, UF, NF, and RO\\
b. MNF, UOF, NOF, and ROO\\
c. CFM, FM, FN, and OR\\
d. None of the above\\
\item What is a method of reducing hardness?\\
a. Softening\\
b. Hardening\\
c. Lightning\\
d. Flashing\\
\item Adsorption of a substance involves its accumulation onto the surface of a:\\
a. Solid\\
b. Rock\\
c. Pellet\\
d. Snow ball\\
\item The solid that adsorbs a contaminant is called the:\\
a. Adsorbent\\
b. Adsorbate\\
c. Sorbet\\
d. Rock\\
\item The adsorption process is used to remove:\\
a. Organics or inorganics\\
b. Bugs or salts\\
c. Organisms or dirt\\
d. Color or particles\\
\item Describe two primary methods used to control taste and odor?\\
a. Oxidation and adsorption\\
b. Filtration and sedimentation\\
c. Mixing and coagulation\\
d. Sedimentation and clarification\\
\item List the five types of surface water filtration systems.\\
a. Bag filtration, cartridge filtration, fine filtration, coarse filtration, media filtration\\
b. Conventional treatment, direct filtration, slow sand filtration, diatomaceous earth filtration, membrane filtration\\
c. Turbidity filtration, color filtration, bag filtration, fine filtration, media filtration\\
d. None of the above\\
\item GAC contactors are used to reduce the amount of contaminants in water.\\
a. Inorganic\\
b. Turbidity\\
c. Particle\\
d. Organic\\
\item Greensand can be operated in either regeneration or regeneration modes.\\
a. Continuous or intermittent\\
b. Fast or slow\\
c. Hot or cold\\
d. Constant or unusual\\
\item What is the cause of taste and odor problems in raw surface water?\\
a. Copper sulfate\\
b. Blue-green algae\\
c. Oxygen\\
d. Lake turnover\\
\item What chemical reduces blue-green algae growth?\\
a. Chlorine\\
b. Caustic Soda\\
c. Copper Sulfate\\
d. Alum\\
\item What is the purpose of adding fluoride to drinking water?\\
a. Increase tooth decay\\
b. Reduce tooth decay\\
c. Make teeth white\\
d. Government conspiracy\\
\item The optimal coagulant dose is determined by a\\
a. Chlorine Test\\
b. Flocculation test\\
c. Jar Test\\
d. Coagulation test\\
\item The most common primary coagulant is\\
a. Alum\\
b. Cationic polymer\\
c. Fluoride\\
d. Anionic polymer\\
\item Bacteria and Viruses belong to a particle size known as\\
a. Suspended\\
b. Dissolved\\
c. Strained\\
d. Colloidal\\
\item The purpose of coagulation is to\\
a. Increase filter run times\\
b. Increase sludge\\
c. Increase particle size\\
d. Destabilize colloidal particles\\
\item The purpose of flocculation\\
a. Destabilize colloidal particles\\
b. Increase particle size\\
c. Decrease sludge\\
d. Decrease filter run times\\
\item Primary coagulant aids used in treatment process are\\
a. Poly-aluminum chloride\\
b. Aluminum sulfate\\
c. Ferric chloride\\
d. All of the Above\\
\item How do water agencies monitor the effectiveness of their filtration process?\\
a. Alkalinity\\
b. Conductivity\\
c. Turbidity\\
d. $\mathrm{pH}$\\
\item Flocculation is used to enhance\\
a. Number of particle collisions to increase floc\\
b. Charge neutralization\\
c. Dispersion of chemicals in water\\
d. Settling speed of floc\\
\item If there is a problem with floc formation, what would you consider changing?\\
a. Adjust coagulant dose\\
b. Stay the course\\
c. Adjust mixing intensity\\
d. Both $A \& C$\\
\item Which step in the treatment process is the shortest?\\
a. Filtration\\
b. Sedimentation\\
c. Flocculation\\
d. Coagulation\\
\item To lower the $\mathrm{pH}$ for enhanced coagulation the operator will add\\
a. Chlorine\\
b. Sulfuric acid\\
c. Lime\\
d. Caustic Soda\\
\item The flocculation process lasts how long?\\
a. Seconds\\
b. 5-10 minutes\\
c. 15-45 minutes\\
d. Over an hour\\
\item The function of a flocculation basin is to\\
a. Settle colloidal particles\\
b. Destabilize colloidal particles\\
c. Mix chemicals\\
d. Allow suspended particles to grow\\
\item The treatment process that involves coagulation, flocculation, sedimentation, and filtration is known as\\
a. Direct filtration\\
b. Slow sand Filtration\\
c. Conventional treatment\\
d. Pressure filtration\\
\item Sedimentation produces waste known as\\
a. Backwash water\\
b. Sludge\\
c. Waste water\\
d. Mud\\
\item What kind of process is the sedimentation step?\\
a. Physical\\
b. Chemical\\
c. Biological\\
d. Direct\\
\item The weirs at the effluent of a sedimentation basin are also called\\
a. Effluent weirs\\
b. Baffling\\
c. Launders\\
d. Spokes\\
\item Sedimentation is used in water treatment plants to\\
a. Settle pathogenic material\\
b. Destabilize particles\\
c. Disinfect water\\
d. Reduce loading on Filters\\
\item Scouring is a term that describes conditions in a sedimentation tank which\\
a. Could impact the rest of treatment process\\
b. Higher flow rates in the sludge zone\\
c. Re-suspends settle sludge\\
d. All of the above\\
\item The four zones in a Sedimentation basin include\\
a. Inlet, sedimentation, sludge, outlet\\
b. Inlet, filter, waste, outlet\\
c. Inlet, top, bottom, outlet\\
d. Surface, sedimentation, sludge, outlet\\
\item The removal and inactivation requirement for Giardia is?\\
a. $99.9 \%$\\
b. $99.99 \%$\\
c. $99.00 \%$\\
d. $90 \%$\\
\item Short circuiting in a sedimentation basin could be caused by\\
a. Surface wind\\
b. Ineffective weir placement, or weirs covered in algae\\
c. Poor baffling in sedimentation inlet zone\\
d. All of the Above\\
\item How much solids should be removed during sedimentation?\\
a. $95 \%$ or more\\
b. $80-95 \%$\\
c. $70-80 \%$\\
d. $60-70 \%$\\
\item The type of basin that includes coagulation and flocculation is\\
a. Rectangular\\
b. Triangular\\
c. Up-Flow\\
d. None of the above\\
\item Recarbonation basins are used to stabilize water after\\
a. Filtration\\
b. Disinfection\\
c. Softening\\
d. Coagulation\\
\item Which of the following is an effective way for removing iron water? a. adding baffles\\
b. adding sodium chloride\\
c. aeration and filtration\\
d. flash mixing\\
\item How can iron bacteria be controlled in a water distribution system?\\
a. by aeration\\
b. filtration\\
c. chlorination\\
d. precipitation\\
\item Which of the following is a hazard when handling hydrofluosilicic acid?\\
a. fire\\
b. explosion\\
c. corrosion\\
d. inhalation\\
\item Trihalomenthane may be partially removed from water by:\\
a. fluoridation\\
b. chlorination\\
c. oxidation\\
d. ultraviolet radiation\\
\item Which of the following forms of iron is most soluble in water?\\
a. Ferric $\left(\mathrm{Fe}^{+3}\right)$\\
b. Ferric hydroxide $\left[\mathrm{Fe}\left(\mathrm{OH}_{3}\right)\right]$\\
c) Ferrous $\left(\mathrm{Fe}^{+2}\right)$\\
d. Ferrous oxide $(\mathrm{FeO})$\\
\item Two fundamental treatment requirements for public water systems using surface sources are\\
a. Coagalat1on and sedimentation\\
b. Lime softening and disinfection\\
c. Filtration and aeration\\
d. Disinfection and filtration\\
\item A zeolite softening unit will replace calcium and magnesium ions with ions.\\
a. Fluoride\\
b. Iron\\
c. Sodium\\
d. Sulfur\\
\item One use of polyphosphates is to:\\
a. Control algae\\
b. Improve taste\\
c. Sequester iron and manganese\\
d. Kill bacteria 124. An acceptable means of corrosion control for relatively small systems is\\
a. Activated carbon\\
b. Lime-soda ash softening\\
c. $\mathrm{pH}$ control\\
d. zeolite softening\\
\item Which of the following chemicals will most likely keep iron in suspension?\\
a. Chlorine\\
b. Fluoride\\
c. Polyphosphate\\
d. Lime inhibitor\\
\item Lead in drinking water can result in\\
a. Impaired mental functioning in children\\
b. Prostate cancer in men\\
c. Stomach and intestinal disorders\\
d. Reduced white blood cell count\\
\item If raw water turbidity changed from 10 to 300 turbidity units and the finished water turbidity had increased from 0.1 to 1.0 turbidity units, the unit process having the most impact to correct this situation is\\
a. Coagulation\\
b. Sedimentation\\
c. Filtration\\
d. Disinfection\\
\item The problem caused by dissolved carbon dioxide in the water of the distribution system is\\
a. increased Trihalomethanes\\
b. Corrosion\\
c. Excessive encrustation\\
d. Tastes and odors\\
\item The presence of the coliform group of bacteria in water indicates\\
a. Contamination\\
b. Inadequate disinfection\\
c. Improper sampling\\
d. Taste and odor problems\\
\item The granular filtration process is designed to reduce\\
a. Calcium and magnesium sulfates\\
b. True color\\
c. Total dissolved solids\\
d. Turbidity\\
\item The presence of the coliform group of bacteria in water indicates\\
a. Contamination\\
b. Inadequate disinfection\\
c. Improper sampling\\
d. Taste and odor problems\\
\item Aeration in water treatment plants is used to\\
a. Lower the $\mathrm{pH}$\\
b. Reduce concentrations of dissolved gasses\\
c. Reduce turbidity\\
d. Stabilize chlorine residuals\\
\item What can the operator do if iron fouling appears to be a problem in an ion exchange softener?\\
a. Decrease the strength of the brine used in the regeneration stage\\
b. Increase backwash flow rates\\
c. Inçrease duration of backwash stage\\
d. Increase duration of service stage\\
\item At what $\mathrm{pH}$ would a chlorinated water have the highest concentration of hypochlorous acid?\\
a. 5\\
b. 7\\
c. 9\\
d. 11\\
\item One use of polyphosphates is to\\
a. Control algae\\
b. Improve taste\\
c. Sequester iron and manganese\\
d. Kill bacteria\\
\item What happens when lime is fed to water for corrosion control?\\
a. Alkalinity is decreased\\
b. CO2 does not change\\
c. Turbidity is decreased\\
d. $\mathrm{pH}$ is increased\\
\item The main characteristic of raw water that enables algae to grow is\\
a. Presence of copper sulfate\\
b. Low $\mathrm{pH}$\\
c. High hardness\\
d. Presence of nutrients\\
\item The type of corrosion caused by the use of dissimilar metal in a water system is\\
a. Caustic corrosion\\
b. Galvanic corrosion\\
c. Oxygen corrosion\\
d. Tubercular corrosion\\
\item A zeolite softening unit will replace calcium and magnesium ions with ions.\\
a. Fluoride\\
b. Iron\\
c. Sodium\\
d. Sulfur\\
\item Two fundamental treatment requirements for public water systems using surface sources are\\
a. Coagulation and sedimentation\\
b. Lime softening and disinfection\\
c. Filtration and aeration\\
d. Disinfection and filtration\\
\item A method used to soften water is\\
a. Aeration\\
b. Sedimentation\\
c. Ion exchange\\
d. Adsorption\\
\item The main characteristic of raw water that enables algae to grow is\\
a. Presence of copper sulfate\\
b. Low $\mathrm{pH}$\\
c. High hardness\\
d. Presence of nutrients\\
\item What happens when lime is fed to water for corrosion control?\\
a. Alkalinity is decreased\\
b. $\mathrm{CO}_{2}$ does not change\\
c. Turbidity is decreased\\
d. $\mathrm{pH}$ is increased\\
\item Which of the following chemicals will most likely keep iron in suspension?\\
a. Chlorine\\
b. Fluoride\\
c. Polyphosphate\\
d. Lime inhibitor\\
\item If raw water turbidity changed from 10 to 300 turbidity units and the finished water turbidity had increased from 0.1 to 1.0 turbidity units, the unit process having the most impact to correct this situation is\\
a. Coagulation\\
b. Sedimentation\\
c. Filtration\\
d. Disinfection\\
\item The granular filtration process is designed to reduce\\
a. Calcium and magnesium sulfates\\
b. True color\\
c. Total dissolved solids\\
d. Turbidity\\
\item Aeration in water treatment plants is used to\\
a. Lower the $\mathrm{pH}$\\
b. Reduce concentrations of dissolved gasses\\
c. Reduce turbidity\\
d. Stabilize chlorine residuals\\
\item What can the operator do if iron fouling appears to be a problem in an ion exchange softener?\\
a. Decrease the strength of the brine used in the regeneration stage\\
b. Increase backwash flow rates\\
c. Inçrease duration of backwash stage\\
d. Increase duration of service stage\\
\item Trihalomenthane may be partially removed from water by:\\
a. fluoridation\\
b. chlorination\\
c. oxidation\\
d. ultraviolet radiation\\
\item Temporary cloudiness in a freshly drawn sample of tap water may be caused by:\\
a. air\\
b. chlorine\\
c. hardness\\
d. silica\\
\item Two fundamental treatment requirements for public water systems using surface sources are\\
a. Coagulation and sedimentation\\
b. Lime softening and disinfection\\
c. Filtration and aeration\\
d. Disinfection and filtration\\
\item A zeolite softening unit will replace calcium and magnesium ions with ions.\\
a. Fluoride\\
b. Iron\\
c. Sodium\\
d. Sulfur\\
\item What happens when lime is fed to water for corrosion control? a. Alkalinity is decreased\\
b. $\mathrm{CO}_{2}$ does not change\\
c. Turbidity is decreased\\
d. $\mathrm{pH}$ is increased\\
\item Which two chemicals are used to remove turbidity?\\
a. Soda Ash and lime\\
b. Copper sulphate and caustic soda\\
c. Alum and lime\\
\item Which of the following is considered to be a coagulant aid?\\
a. Lime\\
b. Polymer\\
c. Bentonite\\
d. All of the above\\
\item Alum precipitates as\\
a. Aluminum carbonate\\
b. Aluminum sulphate\\
c. Aluminum hydroxide\\
\item Turbidity removal with alum is best accomplished at what $\mathrm{pH}$ ?\\
a. 3.5\\
b. 5.0\\
c. 6.5\\
\item Which of the following will not lower the $\mathrm{pH}$ ?\\
a. Alum\\
b. Carbonic acid\\
c. Ferric chloride\\
d. Sodium carbonate\\
\item Liquid fluoride is delivered as:\\
a. Sodium Fluoride\\
b. Hydrofluorosilicic acid\\
c. Sodium Silicofluoride\\
d. Hydrofluoric acid\\
\item An upflow clarifier will have which of the following processes?\\
a. Coagulation\\
b. Flocculation\\
c. Sedimentation\\
d. All of the above\\
\item Sludge that rises to the surface of a sedimentation basin is caused by:\\
a. Not removing sludge often enough\\
b. Removing sludge too often\\
c. $\mathrm{pH}$ is too low\\
d. Surface loading rate is too low\\
\item Pin floc leaving a sedimentation basin may indicate a problem with:\\
a. Coagulation\\
b. Flocculation\\
c. Sedimentation\\
d. Disinfection\\
\item What is the backwash rate for a rapid sand filter?\\
a. 2 gpm/sq.ft.\\
b. $15 \mathrm{gpm} / \mathrm{sq} . \mathrm{ft}$.\\
c. $20 \mathrm{gpm} / \mathrm{sq} . \mathrm{ft}$.\\
d. $25 \mathrm{gpm} / \mathrm{sq} . \mathrm{ft}$.\\
\item What is the maximum run time for a gravity filter?\\
a. 8 hours\\
b. 20 hours\\
c. 48 hours\\
d. 100 hours\\
\item During backwash, the filter bed should expand:\\
a. $5-10 \%$\\
b. $15-20 \%$\\
c. $30-50 \%$\\
d. $60-80 \%$\\
\item If the backwash time is too short, what may result?\\
a. Too much freeboard\\
b. Mudballs\\
c. Loss of filter media\\
d. Filter breakthrough\\
\item If the filtration rate is too high, what may result?\\
a. Filter breakthrough\\
b. Mudballs\\
c. Reduction in operating costs\\
d. Lower headloss\\
\item Solids removed from a filter are most commonly removed by what method?\\
a. Adsorption\\
b. Straining\\
c. Deactivation\\
d. Flocculation\\
\item What is a typical filtration rate for slow sand filters?\\
a. 2.0-6.0 GPM/sq. ft.\\
b. 6.0-10.0 GPM/sq. ft.\\
c. $1.0-2.0 \mathrm{GPM} / \mathrm{sq}$. ft.\\
*d. $0.5-0.10 \mathrm{GPM} / \mathrm{sq}$. ft.\\
\item In a typical conventional treatment plant, the finished water turbidity for an individual filter should be less than\\
a. 1.0 NTUs\\
*b. 0.3 NTUs\\
c. $5.0 \mathrm{NTUs}$\\
d. 3.0 NTUs\\
\item A filter running under normal conditions will see head loss in a filter\\
a. Remain constant\\
b. Increase slowly\\
c. Rapidly increase\\
d. Decrease slowly\\
\item A filter must be washed if this condition is met\\
a. Head Loss\\
b. Turbidity break through\\
c. Maximum Filter run time\\
d. All of the Above\\
\item Filter performance is measured by the removal of\\
a. Oxygen\\
b. Head loss\\
c. Turbidity\\
d. Chlorine\\
\item What is the biologically active layer of a slow sand filter called?\\
a. Mixed Media\\
b. Duel Media\\
c. Sludge Layer\\
d. Schmutzdecke\\
\item The pressure drop in a filter is called\\
a. Turbidity breakthrough\\
b. Head Loss\\
c. Filtration\\
d. Backwash\\
\item What is the most common reason for putting a filter into the wash cycle?\\
a. Head loss\\
b. Filter run time\\
c. Turbidity breakthrough\\
d. Water level decrease\\
\item Formation of mud balls and excessive boiling during a wash is an indicator of\\
a. Proper backwash rate\\
b. Too low backwash rate\\
c. Excessive backwash rate\\
d. Improper chemical dose\\
\item Important processes which occur during filtration are\\
a. Sedimentation\\
b. Adsorption\\
c. Straining\\
d. All of the Above\\
\item Typical filtration rates for a conventional treatment plant are\\
a. 0.2-0.6 GPM/sq.ft.\\
b. 2.0-10.0 GPM/sq.ft.\\
c. 10.0-20.0 GPM/sq.ft.\\
d. 200-400 GPM/sq.ft.

 \item Detention time in flocculation basins are usually designed to provide for\\
a. 5 to 15 minutes.\\
*b. 15 to 45 minutes.\\
c. 45 to 60 minutes.\\
d. 60 to 90 minutes.\\
  \item Alum works best in a pH range of\\
a. less than 4.0.\\
b. 4.0 to 5.5 .\\
*c. 5.8 to 8.5 .\\
d. Greater than 9.0.\\
  \item Which statement is true concerning colloidal particles?\\
*a. Colloidal particles are so small that gravity has little effect on them\\
b. The zeta potential between colloidal particles is balanced by covalent bonding\\
c. Electrical phenomenon of colloidal particles predominate and control their behavior\\
d. The surface area of colloidal particles is very small compared to their mass\\
  \item Which natural electrical force keeps colloidal particles apart in water treatment?\\
a. van der Waals forces\\
b. Ionic forces\\
*c. Zeta potential\\
d. Quantum forces\\
  \item The zeta potential measures the number of excess all particulate matter.\\
*a. electrons\\
b. ions\\
c. cations\\
d. protons\\
\end{enumerate}
\section{Sample Questions for Level II, Answers on Page 157}
\begin{enumerate}[label=TII-\arabic*]
  \item Low temperature water can be compensated for when using alum by\\
a. increasing the pH.\\
b. decreasing the pH.\\
*c. increasing the alum dosage.\\
d. decreasing the alum dosage.\\
  \item Which is the optimal pH range for the removal of particulate matter, when using alum as a coagulant?\\
a. 4.5 to 5.7\\
b. 5.8 to 6.5\\
*c. 6.5 to 7.2\\
d. 7.3 to 8.1\\
  \item Which forces will pull particles together once they have been destabilized in the coagulation-flocculation process?\\
*a. van der Waals forces\\
b. Zeta potential\\
c. Ionic forces\\
d. Quantum forces\\
  \item Which is a common mistake that operators make in regards to flocculation units?\\
a. Excessive flocculation time\\
b. Lack of food-grade NSF-approved grease on the flocculator bearings\\
*c. Keeping the mixing energy the same in all flocculation units\\
d. Too short a flocculation time\\
  \item Ferric sulfate has which advantage over aluminum sulfate (alum)?\\
a. Less staining characteristics\\
b. Less cost\\
*c. More dense floc\\
d. Not as corrosive\\
\end{enumerate}
\section{Sample Questions for Level III, Answers on Page 157}
\begin{enumerate}[label=TIII-\arabic*]
  \item How much alkalinity as CaCO$_{3}$ will dry-basis alum consume?\\
*a. 0.5 mg/l\\
b. 0.8 mg/l\\
c. 1.2 mg/l\\
d. 1.5 mg/l\\
  \item Natural zeolites that have become exhausted with use are regenerated by immersing them in a strong solution of which chemical?\\
*a. NaCl\\
b. NaOH\\
c. HCl\\
d. H$_2$SO$_4$ \\
\item The zeta potential on a particular sample of water is -2 . The degree of coagulation is best described as\\
a. poor.\\
b. fair.\\
*c. excellent.\\
d. maximum.\\
  \item Which is a disadvantage of using static mixers?\\
a. They do not provide good mixing\\
b. They are not economical\\
*c. They increase head loss\\
d. They require too much maintenance\\
  \item Which is the usual effective pH range of iron salt coagulants?\\
*a. 3.5 to 9.0\\
b. 6.5 to 8.8\\
c. 3.0 to 9.5\\
d. 4.2 to 9.0\\
\end{enumerate}
\section{Sample Questions for Level IV, Answers on Page 158}
\begin{enumerate}[label=TIV-\arabic*]
  \item Which is the minimum recommended number of flocculation basins?\\
a. 2\\
*b. 3\\
c. 4\\
d. 5\\
  \item Which type of polymer(s) is (are) sometimes formulated with regulated substances from the following list?\\
a. Polyethylene\\
b. Divinylbenzene\\
c. Polypropylene and polyethylene\\
*d. Nonionic and anionic polymers\\
  \item Which is the most probable solution if rotifers are visible in the finished water?\\
a. Superchlorinate the water plant\\
*b. Optimize coagulation, flocculation, and filtration\\
c. Use aeration followed by lime softening before the settling process\\
d. Use oxygen deprivation\\
  \item The best addition for water that is highly colored due to organic matter would be\\
a. the addition of lime.\\
b. lime addition with increase in the coagulant being used.\\
c. a small increase in a nonionic polymer.\\
*d. the addition of an acid to lower pH before coagulation.\\
  \item If the activation process of silica is not carefully controlled,\\
a. the silica could splash due to high heat of reactants.\\
*b. it could inhibit floc formation.\\
c. it could corrode and destroy the metal and rubber in the flocculators.\\
d. it could deposit silica on the flocculators and the gears, bringing it eventually to a grinding halt.\\
\end{enumerate}
\section{Monitor, Evaluate, \& Adjust Treatment Processes-Clarification and Sedimentation}
\section{Sample Questions for Level I, Answers on Page 158}
\begin{enumerate}[label=TI\arabic*]

  \item In solid-contact basins with fairly constant water quality parameters, how often should the solids concentration be determined?\\
a. At least once per week\\
b. At least every other day\\
c. At least once per month\\
*d. At least twice per day\\
  \item The definition of decant is\\
a. to draw off a liquid layer from a vessel of any size without disturbing any layer(s) above or below.\\
b. to draw off the sediment at the bottom of a vessel of any size without disturbing the overlying liquid layer(s).\\
c. to remove the precipitate at the bottom of any size vessel.\\
*d. to draw off the liquid from a vessel of any size without stirring up bottom sediment.\\
  \item How often should sedimentation basins with mechanical sludge removal equipment be drained and inspected?\\
a. Twice a year\\
*b. Once a year\\
c. Every other year\\
d. Every three years\\
  \item Which is the most important reason to reduce turbidity?\\
a. To reduce taste and odor problems\\
*b. To remove pathogens\\
c. To reduce corrosion\\
d. To determine the efficiency of coagulation and filtration\\
\end{enumerate}
\section{Sample Questions for Level II, Answers on Page 159}
\begin{enumerate}[label=TII-\arabic*]
  \item If enteric disease-causing protozoans have been found in the effluent of a water plant, which is the most probable solution?\\
a. Where possible, use powdered activated carbon (PAC) throughout water plant; backwashing filters will remove the PAC\\
b. Use PAC only in the sedimentation basin; backwashing the filters will remove the PAC\\
*c. Use multibarrier approach—coagulation, flocculation, sedimentation, and filtration\\
d. Superchlorinate the water plant \\
\item Which is the major cause of short circuiting in a sedimentation basin?\\
a. Open basins that are subject to algal growths and thick slime growths on the side of the basin\\
b. Basins without a wind break\\
*c. Poor inlet baffling\\
d. Density currents\\
  \item Conventional sedimentation has a removal of Cryptosporidium oocysts.\\
*a. less than 0.5-log\\
b. 0.5-log\\
c. 1.0-log\\
d. 2.0-log\\
  \item In solids-contact basins, the weir loading normally should not exceed weir length.\\
a. 1 gpm/ft\\
b. 2 gpm/ft\\
c. 5 gpm/ft\\
*d. 10 gpm/ft\\
  \item Dissolved-air flotation is particularly good for removing\\
a. sulfides.\\
b. inorganics.\\
c. manganese and iron.\\
*d. algae.\\
\end{enumerate}
\section{Sample Questions for Level III, Answers on Page 159}
\begin{enumerate}[label=TIII-\arabic*]
  \item Which determines whether or not colloidal-sized particles in suspension repel each other, stay in suspension, or agglomerate and eventually settle?\\
a. Number of collisions\\
b. Flow and temperature of the water\\
c. Types of chemical bonding\\
*d. Magnitude of the charges\\
  \item If the sludge in a sedimentation basin becomes too thick, which could happen?\\
a. Gases from decomposition will rise through the settled sludge accelerating normal floc settling\\
b. Abundant trihalomethanes and haloacetic acids will form\\
*c. Solids can become resuspended or taste and odors can develop\\
d. The sludge will compact at the bottom of the basin making it very difficult to remove\\
  \item In basins using tube and plate settlers, which parameter must be much better than conventional treatment basins?\\
a. Metals must be oxidized before reaching the tubes and plates\\
b. Floc rate must be 2 to 3 times slower than conventional basins\\
c. Floc rate must be 3 to 4 times faster than conventional basins\\
*d. Floc must have good settling characteristics \\
\item Which type of sedimentation basins have the flow of water admitted at an angle?\\
a. Rectangular settling basins\\
b. Square settling basins\\
c. Center-feed settling basins\\
*d. Spiral-flow basins\\
  \item At which minimum angle must self-cleaning tube settlers be placed?\\
*a. 50 degrees\\
b. 60 degrees\\
c. 65 degrees\\
d. 70 degrees\\
\end{enumerate}
\section{Sample Questions for Level IV, Answers on Page 160}
\begin{enumerate}[label=TIV-\arabic*]
  \item If nematodes are interfering with the disinfectant, which is the most probable solution?\\
*a. Optimize the settling process\\
b. Use chloramines\\
c. Decrease detention time of finished water in clearwell and tanks\\
d. Use oxygen deprivation\\
  \item At which angle should the parallel inclined plates be installed when using the shallow-depth sedimentation method?\\
a. $35^{\circ}$\\
*b. $45^{\circ}$\\
c. $50^{\circ}$\\
d. $60^{\circ}$\\
  \item Why do solids-contact basins have much shorter detention times than conventional treatment basins?\\
a. Because chemical reactions take place throughout the basin\\
b. Because the settling zone water moves upward, while at the same time the mixing zone moves upward\\
c. Because the gentle upward flow of the water throughout the basin is conducive for producing larger settleable floc\\
*d. Because of the recycled materials from the sludge blanket, the chemical reactions occur more quickly and completely in the mixing area\\
  \item The pulsating energy in a pulsator clarifier helps to\\
*a. maintain a uniform sludge blanket layer.\\
b. mix the coagulants with the raw water.\\
c. mix the coagulant aids with the primary coagulant and water to help in flocculation.\\
d. raise the sludge blanket over the weir for wasting.\\
  \item Pulsator clarifiers are used to treat water that is\\
a. low in temperature, usually $<10^{\circ} \mathrm{C}$.\\
*b. high in color and low in turbidity.\\
c. low in color and high in turbidity.\\
d. high in organic acids.\\
\end{enumerate}
\section{Monitor, Evaluate, \& Adjust Treatment Processes-Filtration}
\section{Sample Questions for Level I, Answers on Page 160}
\begin{enumerate}[label=TI\arabic*]
  \item Which is the filtration flow rate through a manganese greensand pressure filter?\\
a. 1 to $2 \mathrm{gpm} / \mathrm{ft}^{2}$\\
*b. 2 to $3 \mathrm{gpm} / \mathrm{ft}^{2}$\\
c. 3 to $5 \mathrm{gpm} / \mathrm{ft}^{2}$\\
d. 5 to $8 \mathrm{gpm} / \mathrm{ft}^{2}$\\
  \item When a filter is ripening,\\
a. it is in need of a backwash.\\
b. turbidity is just starting to break through.\\
*c. it is becoming more efficient in particle removal.\\
d. it is beginning to grow algae in the filter bed, walls, and troughs.\\
  \item Virgin greensand can be regenerated by soaking the filter bed for several hours in a solution of chlorine containing\\
a. $50 \mathrm{mg} / \mathrm{L} \mathrm{Cl}_{2}$.\\
b. $\quad 75 \mathrm{mg} / \mathrm{L} \mathrm{Cl}_{2}$.\\
*c. $\quad 100 \mathrm{mg} / \mathrm{L} \mathrm{Cl}_{2}$.\\
d. $200 \mathrm{mg} / \mathrm{L} \mathrm{Cl}_{2}$.\\
  \item Which role does the action of straining of suspended particles play during filtration?\\
*a. Minor\\
b. Fair\\
c. Good\\
d. Major\\
  \item The turbidity of settled water before it is applied to the filters (post sedimentation process) should always be kept below\\
*a. 1 to $2 \mathrm{ntu}$.\\
b. 2 to $4 \mathrm{ntu}$.\\
c. 5 ntu.\\
d. 8 to $10 \mathrm{ntu}$.\\
\end{enumerate}
\section{Sample Questions for Level II, Answers on Page 160}
\begin{enumerate}[label=TII-\arabic*]
  \item Which is the best process for the removal of turbidity?\\
a. Anion exchange\\
*b. Coagulation, flocculation, sedimentation, and filtration\\
c. Chemical oxidation\\
d. Granular activated carbon\\
  \item If filter run times between backwashes are long, for example one week, because high quality (low turbidity) water is being applied to the filters, which problem could still arise?\\
a. Mudball formation\\
b. Air binding and formation of mudballs\\
c. Extended backwashing due to media becoming too compacted\\
*d. Floc breakthrough \\
\item Gravel displacement in a filter bed from backwash rates with too high of a velocity could eventually cause\\
a. compaction of the filter media.\\
b. loss of media into the backwash troughs.\\
*c. a sand boil.\\
d. bed shrinkage.\\
  \item Virgin greensand\\
a. does not require regeneration.\\
*b. requires regeneration with potassium permanganate - 1 hour soak with 60 grams KMnO4\\
c. requires regeneration with manganese dioxide - 2 hour soak with 25\% by weight solution of MnO2\\
d. requires regeneration with manganese hydroxide - 4 hour soak with 200 grams Mn(OH)2)\\
  \item Which conventional treatment step is eliminated by direct filtration?\\
a. Oxidation\\
b. Aeration\\
c. Flocculation\\
*d. Sedimentation\\
\end{enumerate}
\section{Sample Questions for Level III, Answers on Page 161}
\begin{enumerate}[label=TIII-\arabic*]
  \item If crustaceans have clogged the water treatment plant's filters, which is the most probable solution?\\
a. Shut down the filters and physically remove them\\
b. Shut down one filter at a time and drain; once the crustaceans have died, physically remove them and then repeat process on the other filters\\
c. Backwash filters using a very high concentration of ozone in the water\\
*d. Use a disinfectant that targets the specific organisms in question\\
  \item Which organism can escape coagulation and thus pass through a granular filter?\\
a. Giardia\\
b. Entamoeba\\
*c. Cryptosporidium\\
d. Naegleria\\
  \item Reverse osmosis membranes will compact faster with\\
a. higher iron content.\\
b. higher chlorine contact.\\
*c. higher pressure.\\
d. higher pH.\\
  \item Which would immediately occur if newly installed manganese greensand was not skimmed of the fines after backwashing and stratification steps were completed?\\
a. Uneven flow through the bed\\
b. Cracks would develop in the bed\\
c. Mudball formation\\
*d. Shorter filter runs \\
\item When a filter is operated at normal flow rates, its ability to trap flocculated particles in suspension is a function of\\
a. effective size multiplied by uniformity coefficient.\\
b. effective size multiplied by uniformity coefficient divided by media size.\\
*c. media depth and media size.\\
d. media depth and uniformity coefficient.\\
\end{enumerate}
\section{Sample Questions for Level IV, Answers on Page 161}
\begin{enumerate}[label=TIV-\arabic*]
  \item Which is the best solution if iron bacteria are causing corrosion problems in the filters?\\
a. Protect the metal parts of the filters by adding zinc orthophosphate\\
b. Optimize the settling process\\
*c. Superchlorinate\\
d. Add lime at 50 mg/L to one filter at a time\\
  \item A conventional water treatment plant with dual media filters has very cold water in the winter and warm water in the summer. Which should the operator do to compensate for this temperature change?\\
a. Use more coagulants in the summer per million gallons\\
*b. Sustain the same bed expansion without media loss by reducing or increasing backwash flow rate\\
c. Increase summer bed expansion and increase winter backwash flow rates\\
d. Increase bed expansion in the winter compared to summer in order to remove turbidity\\
  \item How are reverse osmosis membranes cleaned once they become fouled?\\
a. They are soaked in high purity industrial soap for at least 24 hours\\
*b. They are cleaned with an acid wash\\
c. They are cleaned with an acid, then with an industrial soap for 24 hours\\
d. They are cleaned first with a high purity industrial soap and then soaked in an acid solution for 3 days\\
  \item Which membrane process is used to treat brackish water or seawater?\\
a. Microfiltration\\
b. Nanofiltration\\
*c. Reverse osmosis\\
d. Ultrafiltration\\
  \item The amount of reject water from a reverse osmosis unit is dependent on the number of stages in which the membranes are configured and the\\
*a. feed pressure.\\
b. amount of cations.\\
c. amount of cations and anions.\\
d. pH of the water.\\
\end{enumerate}
\section{Monitor, Evaluate, and Adjust Treatment Processes-Residuals Disposal}
\section{Sample Questions, General, Answers on Page 162}
\begin{enumerate}[label=TG-\arabic*]
  \item In the precipitative softening plant, which percentage of solids sludge is produced?\\
a. 1\%\\
*b. 5\%\\
c. 10\%\\
d. 30\%\\
  \item Which sludge disposal method is most economical for lime-soda ash softening plants?\\
a. Disposal into the sewage system\\
b. Sand drying beds\\
*c. Lagoons\\
d. Landfill the sludge\\
  \item Current regulations require water treatment wastes to be monitored\\
*a. daily.\\
b. weekly.\\
c. monthly.\\
d. quarterly.\\
  \item Which process is used to concentrate sludge?\\
a. Sand bed\\
b. Solar lagoon\\
*c. Thickener\\
d. Centrifuge\\
  \item Which process is used to dewater sludge?\\
a. Wash water basin\\
c. Thickener\\
*b. Sand bed\\
c. Thickener\\
d. Reclamation basin\\
\end{enumerate}
\section{Sample Questions for Level III, Answers on Page 162}
\begin{enumerate}[label=TIII-\arabic*]
  \item Which is the total concentration of dissolved solids in the wastewater from the regeneration of ion exchange units?\\
a. $\quad 10,000$ to $20,000 mg/l$\\
b. 20,000 to 30,000 mg/l\\
*c. 35,000 to 45,000 mg/l\\
d. 45,000 to 60,000 mg/l\\
  \item Which is the usual range of percent sludge solids, if the sludge is allowed to accumulate and compact at the bottom of a sedimentation basin?\\
*a. 2-4 \%\\
b. 4-7 \%\\
c. 7-12 \%\\
d. 12-15 \% \\
\item Which should be determined first before an in-ground sedimentation tank is drained?\\
a. The solids content\\
b. The hazardous metals content\\
c. Sludge volume and volume of process area to make sure it will be large enough\\
*d. Water table level\\
  \item Which sludge dewatering process is best for alum sludges (which are difficult to dewater) when the cakes are very dry, filtrate is clear, and solids capture is very high?\\
a. Centrifuge\\
b. Vacuum filters\\
*c. Filter press\\
d. Belt filter press\\
  \item Which sludge dewatering process requires a precoat of diatomaceous earth and its use has declined due to other newer methods?\\
a. Centrifuge\\
*b. Vacuum filters\\
c. Filter press\\
d. Belt filter press\\
 \item Which precipitates can foul a cation exchange resin?\\
a. Sodium chloride and potassium chloride\\
b. Chlorate and borate\\
c. Sulfates\\
*d. Iron and manganese\\
  \item Which process works best for sequestering manganese?\\
a. Sodium silicate alone\\
b. Sodium silicate and chlorine\\
c. Polyphosphates alone\\
*d. Polyphosphates and chlorine\\
  \item When should polyphosphates used for sequestration of iron and manganese from a well be injected into the process?\\
a. Right after disinfection\\
b. Immediately after aeration to remove unwanted gases\\
c. Right after clarification\\
*d. Right after the water leaves the well\\
  \item Recarbonation is\\
*a. adding CO2 to the water.\\
b. adding bicarbonate to the water.\\
c. adding acid to precipitate the excess lime.\\
d. adding caustic soda.\\
  \item In the ion-exchange softening process, once the resin can no longer soften water it must be\\
a. renewed.\\
b. re-catalyzed.\\
*c. regenerated.\\
d. recharged.\\
\end{enumerate}
\section{Sample Questions for Level II, Answers on Page 164}
\begin{enumerate}[label=TII-\arabic*]
  \item Ion exchange processes can typically be used for direct groundwater treatment as long as turbidity and levels are not excessive.\\
a. calcium carbonate\\
*b. iron\\
c. carbon dioxide\\
d. sodium sulfate \\
\item Softened water has a high pH and a high concentration of CaCO$_{3}$. Therefore, stabilization is essential in order to prevent the CaCO$_{3}$ from precipitating out\\
a. in household plumbing.\\
b. in the clear well.\\
c. in the distribution system.\\
*d. on the filters.\\
  \item Which is the best type of salt to use in the regeneration of ion exchange softener resin?\\
a. Fine-grained salt\\
b. Block salt\\
c. Block or road salt\\
*d. Rock salt or pellet-type salt\\
  \item Powdered activated carbon is primarily used to control\\
a. disinfectant by-products.\\
*b. organic compounds responsible for tastes and odors.\\
c. synthetic organic chemicals.\\
d. humic and fulvic acids.\\
  \item Ion exchange will remove\\
*a. all hardness.\\
b. all hardness down to 7.4 mg/l, as CaCO3.\\
c. all hardness down to 17.2 mg/l, as CaCO3.\\
d. all hardness down to about 25.0 mg/l, as CaCO3.\\
\end{enumerate}
\section{Sample Questions for Level III, Answers on Page 164}
\begin{enumerate}[label=TIII-\arabic*]
  \item It is impossible to produce waters with a hardness of less than \_\_\_ when using the lime-soda ash process.\\
a. 9 mg/l\\
b. 17 mg/l \\
*c. 25 mg/l \\
d. 50 mg/l \\
  \item When added to water for softening purposes, soda ash will do which of the following?\\
*a. Disinfect the water and kill the vast majority of protozoans, viruses, bacteria, and other multicellular organisms\\
b. Raise the pH of water to between 8.0 and 9.8 pH units\\
c. Add CO2 to the water\\
d. Add calcium alkalinity to the water\\
  \item Magnetic ion exchange resin has been developed to remove\\
*a. total organic carbon.\\
b. chlorides.\\
c. iron and magnesium.\\
d. sulfates and sulfides. 
  \item Approximately how much carbon is lost during the reactivation process for granular activated carbon?\\
*a. 5 \%\\
b. 7 \%\\
c. 10 \%\\
d. 15 \%\\
  \item Which is the most advantageous application point for powdered activated carbon?\\
*a. Raw water intake\\
b. After coagulation\\
c. After oxidation with chlorine\\
d. In the filters\\
\end{enumerate}
\section{Sample Questions for Level IV, Answers on Page 165}
\begin{enumerate}[label=TIV-\arabic*]
  \item Which is the most effective method for removing tastes and odors?\\
a. Coagulation, sedimentation, and filtration\\
*b. Granular activated carbon\\
c. Anion exchange\\
d. Lime softening\\
  \item Backwashing rate procedures should be reassessed to determine the cause of granular activated carbon loss if the loss per year exceeds\\
*a. 2 inches.\\
b. 4 inches.\\
c. 6 inches.\\
d. 8 inches.\\
  \item Which is the most efficient process for the removal of nitrite and nitrate?\\
a. Powdered activated carbon\\
b. Granular activated carbon\\
*c. Anion exchange\\
d. Cation exchange\\
  \item Which is the main problem if particle agglomeration is occurring in a filter for iron and manganese removal at the interface of the coal layer and the layer below?\\
a. Oxidant is too weak\\
b. Coagulant dosage is excessive\\
c. Coal layer is too fine\\
*d. Coal layer is too coarse\\
  \item Which is the most effective method for the removal of disinfection by-products?\\
a. Reverse osmosis\\
b. Lime softening\\
c. Ultrafiltration\\
*d. Granular activated carbon\\
\end{enumerate}
\begin{enumerate}
  \item are used to cause particles to become destabilized and begin to clump together.\\
a) coagulant aids\\
b) nonsettable solids\\
c) zeta particles\\
*d) primary coagulants\\
 \item The purpose of stabilization is\\
a) to prevent floc from rising in the basin\\
b) to prevent sludge from entering the filters\\
*c) to prevent corrosion or excessive scale from entering the distribution system\\
d) to prevent excessive turbidity at the top of the filters\\
  \item Core sampling is a viable way to check the condition of your\\
a) raw water\\
b) coagulation process\\
c) finished water\\
*d) filters\\
 \item In solid contact units, the three main operational fundamentals are\\
a) sedimentation, slurry, \& suspended solids\\
b) mixing, clarifying, \& filtration\\
*c) chemical dosage, recirculation rate, \& sludge control\\
d) weighing agents, alkalinity \& pac\\
  \item Particle counters can be used in place of \rule{1.5cm}{0.5pt} in the treatment process to obtain better control.\\
a) flash mixers\\
b) variable drives\\
c) filter coring\\
*d) turbidimeters\\
  \item In their soluble or reduced state, iron and manganese are\\
a) alkalinity enhancers\\
*b) colorless\\
c) negatively charged\\
d) won't dissolve in water\\
  \item The maximum filtration rate allowable, without special permission, for dual and mixed media filters is\\
a) $2 \mathrm{gal} / \mathrm{min} / \mathrm{sq} \mathrm{ft}$\\
*b) $5 \mathrm{gal} / \mathrm{min} / \mathrm{sq} \mathrm{ft}$\\
c) there is no maximum\\
d) $9 \mathrm{gal} / \mathrm{min} / \mathrm{sq} \mathrm{ft}$\\

  \item During the coagulation/flocculation process, particulate impurities can be divided into two classifications.\\
a) primary coagulants and coagulant aids\\
*b) settleable and nonsettleable solids\\
c) hydraulic and mechanical\\
d) paddlewheel and walking beam\\
  \item In conventional rectangular sedimentation basins, $50 \%$ of the sludge should settle out in the of the basin.\\
*a) first third\\
b) last half\\
c) very beginning\\
d) tail end\\
  \item Generally, the more uniform the media, the the rate of headloss.\\
*a) slower\\
b) same\\
c) smaller\\
d) larger\\
  \item polymers are positively charged.\\
a) nonionic\\
b) anionic\\
*c) cationic\\
d) platonic\\
  \item The Van der Waals principle refers to\\
*a) oppositely charged particles attract\\
b) the settling rate of suspended solids\\
c) the benefits of early oxidation of raw water\\
d) the backwash rates of multi media filters\\
  \item The time necessary to perform the coagulation, flocculation, and settling processes in treatment are correctly listed in what order, starting with coagulation?\\
a) days, weeks, months\\
b) hours, minutes, seconds\\
c) weeks, months, years\\
*d) seconds, minutes, hours\\
  \item Overdosing of potassium permanganate will likely cause\\
a) an extremely high $\mathrm{pH}$\\
*b) pink water\\
c) taste and odor\\
d) inadequate settling\\
  \item Which of the following is most likely to be used as a primary coagulant?\\
a) brine\\
b) ammonious hydroxide\\
*c) ferric sulfate\\
d) sodium thiosulfate\\
\item Desirable media characteristics include\\
a) permeability\\
b) solubility in water\\
c) full of impurities\\
*d) hard and durable\\
  \item The two types of removal mechanisms for gravity filters are\\
a) redundant and repetitive\\
*b) mechanical and adsorption\\
c) coagulation and flocculation\\
d) regeneration and renewal\\
  \item Fluoride is added to water to\\
a) create a nuisance\\
*b) aid in the development of teeth and bones\\
c) so there is something that has both a primary and secondary MCL\\
d) aid in the protective coating of pipes\\
\item The best pH level for coagulation usually falls in the range of\\
a) 4-6\\
*b) $5-7$\\
c) $7-9$\\
d) $1-3$\\
  \item The mixing of coagulant chemicals and raw water is called\\
a) flocculation\\
b) aeration\\
c) reverse osmosis\\
*d) flash mixing\\
  \item Sedimentation basins have zones.\\
a) five\\
*b) four\\
c) three\\
d) two\\
  \item A jumbled mass or collection of floc, solids, and filter media that could grow into a larger mass and reduce filter efficiency is\\
a) turbidity mass\\
b) tuberculation\\
*c) a mudball\\
d) a media crack\\
  \item When backwashing filters, bed expansion should be between\\

*a) $15-30 \%$ percent.\\
b) $10-20 \%$\\
c) $20-40 \%$\\
d) $30-50 \%$\\
  \item The two main softening methods used by treatment facilities are\\
a) reverse osmosis and oxidation\\
b) distillation and disinfection\\
c) ultraviolet radiation and electrodialysis\\
*d) ion-exchange and lime-soda ash\\
  \item The effective way to combat taste and odor problems is\\
a) aeration and tube settlers\\
b) settling out by particle counting\\
*c) prevent them from occurring\\
d) coagulation and flocculation\\
  \item If a filter exceeds NTU at any time the system must arrange for the State to conduct a Comprehensive Performance Evaluation within thirty days.\\
*a) $2.0 \mathrm{NTU}$\\
b) $3.0 \mathrm{NTU}$\\
c) $5.0 \mathrm{NTU}$\\
d) $10.0 \mathrm{NTU}$\\
  \item Most water treatment facilities will run more effectively if\\
a) the mayor lends a hand\\
*b) it runs $24 \mathrm{hrs}$ a day\\
c) it runs $12 \mathrm{hrs}$ on and $12 \mathrm{hrs}$ off\\
d) it runs 16 hours a day\\
  \item Turbidity is used as a process control measurement because\\
a) everyone has a turbidimeter around\\
b) the results are foolproof\\
*c) the number of pathogens increase as turbidity increases\\
d) turbidity removal is an extremely easy task\\
 \item Solids contact units (clarifiers) generally demand a higher level of operator knowledge and skill than conventional treatment techniques and processes.\\
a) true\\
b) false\\
  \item Filtration actually particulates.\\
a) destroys\\
*b) stores\\
c) dissolves\\
d) suspends\\
 \item is a concentrated accumulation of chemicals and contaminants and pollutants that we attempt to remove from raw water.\\
a) pathogens\\
*b) sludge\\
c) coagulants\\
d) fluoride\\

  \item Iron and manganese removal can be accomplished by\\
a) oxidation with chlorine followed by filtration\\
b) oxidation by aeration followed by filtration\\
c) oxidation by potassium permanganate followed by filtration\\
*d) all of the above

  \item Short circuiting refers to\\
a) pumps running backwards which stops treatment\\
b) a movie made in the 80 's\\
c) inadequate voltage applied water treated by electrodialysis\\
*d) uneven flows which result in decreased treatment efficiency\\

\item Solids removed from a filter are most commonly removed by what method?\\
a.	Adsorption\\
b.	Straining\\
c.	Deactivation\\
d.	Flocculation\\
\item What is a typical filtration rate for slow sand filters?\\
a.	2.0-6.0 GPM/sq. ft\\
b.	6.0-10.0 GPM/sq. ft\\
c.	1.0-2.0 GPM/sq. ft\\
d.	0.5-0.10 GPM/sq. ft\\
\item In a typical conventional treatment plant, the finished water turbidity for an individual filter should be less than \rule{1.5cm}{0.5pt}.\\
a.	1.0 NTUs\\
b.	0.3 NTUs\\
c.	5.0 NTUs\\
d.	3.0 NTUs\\
\item A filter running under normal conditions will see head loss in a filter \rule{1.5cm}{0.5pt}.\\
a.	Remain constant\\
b.	Increase slowly\\
c.	Rapidly increase\\
d.	Decrease slowly\\
\item A filter must be washed if this condition is met:\\
a.	Head loss\\
b.	Turbidity breakthrough\\
c.	Maximum filter run time\\
d.	All of the above\\
\item Filter performance is measured by the removal of \rule{1.5cm}{0.5pt}.\\
a.	Oxygen\\
b.	Head loss\\
c.	Turbidity\\
d.	Chlorine\\
\item What is the biologically active layer of a slow sand filter called?\\
a.	Mixed media\\
b.	Duel media\\
c.	Sludge layer\\
d.	Schmutzdecke\\
\item The pressure drop in a filter is called \rule{1.5cm}{0.5pt}.\\
a.	Turbidity breakthrough\\
b.	Head Loss\\
c.	Filtration\\
d.	Backwash\\
\item What is the most common reason for putting a filter into the wash cycle?\\
a.	Head loss\\
b.	Filter run time\\
c.	Turbidity breakthrough\\
d.	Water level decrease\\
\item Formation of mud balls and excessive boiling during a wash is an indicator of \rule{1.5cm}{0.5pt}.\\
a.	Proper backwash rate\\
b.	Too low backwash rate\\
c.	Excessive backwash rate\\
d.	Improper chemical dose\\
\item Important processes which occur during filtration are \rule{1.5cm}{0.5pt}.\\
a.	Sedimentation\\
b.	Adsorption\\
c.	Straining\\
d.	All of the above\\
\item Typical filtration rates for a conventional treatment plant are \rule{1.5cm}{0.5pt}.\\
a.	0.2-0.6 GPM/sq.ft\\
b.	2.0-10.0 GPM/sq.ft\\
c.	10.0-20.0 GPM/sq.ft\\
d.	200-400 GPM/sq.ft\\

\item Which will occur if dry alum and quicklime are mixed together\\
a.  Create tremendous heat and hydrogen gas will be released\\
b.  Dust may be released which may cause an explosion\\
c.  A coagulated gel will form  a\\
d.  Nothing as they are neutral towards each other\\
\item {X.  }The quantity of dissolved oxygen in water is a function of\\
a.  Ph alkalinity temperature and total dissolved solids\\
b.  Temperature and alkalinity\\
c.  Ph and temperature  d\\
d.  Temperature pressure and salinity\\
\item When chlorine has destroyed all reducing compounds any chlorine remaining will react with\\
a.  Nitrite and form chloramines\\
b.  Nitrates and form chloramines\\
c.  Ammonia and form chloramines\\
d.  Organics and form aromatics\\
\item When used with alum which chemical improves coagulation\\
a.  Ferric chloride\\
b.  Ferric sulfate\\
c.  Sodium aluminate\\
d.  Aluminum sulfate\\
\item In the ion exchange softening process once the resin can no longer soften water it must be\\
a.  Renewed\\
b.  Re-catalyzed\\
c.  Regenerated\\
d.  Recharged\\
\item Low values for which characteristic may require the addition of lime caustic soda or sodium bicarbonate\\
a.  Turbidity\\
b.  Water temperature\\
c.  pH \\
d.  Alkalinity\\
\item Ion exchange process can typically be used for direct groundwater treatment as long as turbidity and \rule{1.5cm}{0.5pt} levels are not excessive\\
a.  Calcium carbonate\\
b.  Iron\\
c.  Carbon dioxide\\
d.  Sodium sulfate\\
\item Which statement is true concerning colloidal particles\\
a.  Colloidal particles are so small that gravity has little effect on them\\
b.  The zeta potential between colloidal is balanced by covalent bonding\\
c.  Electrical phenomenon of colloidal particles predominate and control their behavior a\\
d.  The surface area of colloidal particles is very small compared to theirs mass\\
\item The zeta potential measures the number of excess found on the surface of all particulate matter\\
a.  Electrons\\
b.  Ions\\
c.  Cations\\
d.  Protons\\
\item Which natural electrical force keeps colloidal particles apart in water treatment\\
a.  Van der waals force\\
b.  Ionic forces\\
c.  Zeta potential\\
d.  Quantum forces\\
\item Which forces will pull particles together once they have destabilized in the coagulation flocculation process\\
a.  Van der waals forces\\
b.  Zeta potential\\
c.  Ionic forces\\
d.  Quantum forces\\
\item Ferric sulfate has which advantage over aluminum sulfate (alum)\\
a.  Less staining characteristics\\
b.  Less cost\\
c.  More dense floc\\
d.  Not as corrosive\\
\item Which is the major cause of short circuiting in a sedimentation basin\\
a.  Open basins that are subject to algal growths and thick slime growths\\
b.  Basins without a wind break\\
*c.  Poor inlet baffling\\
d.  Density currents\\
\item Algae blooms may be responsible for which of the following problems\\
a.  An increased in chlorine demand\\
b.  Elevated disinfection by products\\
c.  Filter clogging d\\
*d.  All ofthe above\\
\item Strategies for taste and odor treatment generally fall under two broad categories\\
a.  Oxidation and chlorination\\
b.  Aeration and degasification\\
*c.  Removal and destruction\\
d.  Sedimentation and adsorption\\
\item A problem in filters that occurs when suspended solids pass through the filter media is called\\
a.  Head loss\\
*b.  Breakthrough\\
c.  Turbidity\\
d.  Air binding\\
\item What is a typical filtration rate for slow sand filters?\\
a.  2.0-6.0 GPM/sq. ft\\
b.  6.0-10.0 GPM/sq. ft\\
c.  1.0-2.0 GPM/sq. ft\\
d.  0.5-0.10 GPM/sq. ft\\
\item The branch of science which deals with water or fluids at rest or in motion is\\
a.  conductivity\\
b.  geology\\
c.  oceanography
*d.  hydraulics\\

\item What is the ratio of lime to copper sulfate for controlling algae growth on basin walls?\\
*a.	l part lime to 1 part copper sulfate\\
b.	1 part lime to 2·parts copper sulfate\\
c.	1 part lime to 3 parts copper sulfate\\
d.	2 parts lime to 3 parts copper sulfate\\

\item Copper sulfate is used in surface water reservoirs to control\\
a.	Emergent  weeds\\
*b.	Algae\\
c.	Mosquito larvae\\
d.	Snails\\

\item Which of the following best defines adsorption?\\
a.	Assimilation of one substance into the body of another by molecular or chemical action\\
*b.	Adhesion of a gas, liquid, or dissolved substance onto the surface or interface zone of another substance\\
c.	Converting small particles of suspended solids into larger particles by the use of chemicals\\
d.	Chemical complexing of metallic cations with certain inorganic compounds\\

\item About how much alkalinity is required for each milligram per liter of alum added to raw water?\\
*a.	0.5mg/L\\
b.	l.0mg/L\\
c.	1.5 mg/L\\
d.	2.0 mg/L\\

\item A water treatment plant's flocculation-coagulation and sedimentation processes should be checked if which of the following changes?\\
*a.	Turbidity\\
b.	Chlorine feed rate\\
c.	Fluoride feed rate\\
d.	Total trihalomethanes\\

\item	Which of the following is an example of a weighting agent?\\
a.	Polyelectrolytes\\
*b.	Bentonite clay\\
c.	Calcium carbonate\\
d.	Sodium bicarbonate\\

\item Algae can shorten filter runs by\\
*a.	Clogging the filters\\
b.	Increasing chlorine demand\\
c.	Lowering the pH\\
d.	Increasing turbidity\\

\item Coagulation is a chemical and physical reaction that converts\\
a.	Settleable solids into nonsettleable solids\\
*b.	Nonsettleable solids into settleable solids\\
c.	Dissolved solids into settleable solids\\
d.	Dissolved solids into a precipitate\\

\item Water, deposition in metal piping will increase if\\
a.	The pH is above 7.0 and has low dissolved oxygen levels\\
b.	The alkalinity is high and water temperature is low\\
c.	Total dissolved solids are high and the pH is below 7.0\\
*d.	The pH and alkalinity increase\\

\item The filtration unit process usually\\
a. is located at the beginning of a filtration plant\\
*b. follows the coagulation/flocculation/sedimentation processes\\
c. is located after the clear well area\\

  \item Filters are generally backwashed when the loss-of-head indicator registers a certain set value, such as 6-ft, or upon a certain time, say 48-hours, or upon a rise in\\
a. alkalinity\\
b. a jar-test result\\
*c. turbidity\\
d. temperature\\

  \item Which is the safe dosage for most species of fish when using copper sulfate to control algae in a body of water?\\
a.  0.3 mg/l\\
*b. 0.5 mg/l\\
c.  0.8 mg/l\\
d. 1.0 mg/l\\
  \item Manganese greensand filters can be regenerated by using\\
a. a surface wash and an air-water backwash.\\
b. brine water during backwashing.\\
*c. potassium permanganate solution during backwashing.\\
d. first a brine solution during the first backwashing cycle followed by potassium permanganate solution for the second backwash cycle.\\

 \item Which material is manganese greensand and which is the coating?\\
a. Quartz sand coated with manganese hydroxide $\left[\mathrm{Mn}(\mathrm{OH})_{2}\right]$\\
b. Garnet sand coated with manganese dioxide $\left[\mathrm{MnO}_{2}\right]$\\
c. Ilmenite sand coated with manganese hydroxide\\
*d. Glauconite sand coated with manganese dioxide\\
  \item Which is the layer of solids and biological growth that forms on the top of a slow sand filter?\\
a. Biosolids film\\
b. Bio-carbonated scale layer\\
*c. Schmutzdecke\\
d. Saprophytic layer 

  \item Below are four membrane technologies. Which is the correct sequence from larger to smaller pore sizes?\\
a. Microfiltration, reverse osmosis, ultrafiltration, and nanofiltration\\
*b. Microfiltration, ultrafiltration, nanofiltration, and reverse osmosis\\
c. Reverse osmosis, ultrafiltration, microfiltration, and nanofiltration\\
d. Ultrafiltration, microfiltration, reverse osmosis, and nanofiltration\\


  \item Which filter media material is given an abrasive number?\\
a. Garnet\\
*b. Activated carbon\\
c. Sand\\
d. Greensand\\


  \item Direct filtration is used to treat raw water that has average turbidities\\
a. below $10 \mathrm{NTU}$.\\
*b. up to $25 \mathrm{NTU}$.\\
c. 40 to $50 \mathrm{NTU}$.\\
d. above $50 \mathrm{NTU}$.\\

  \item Diatomaceous earth filters\\
a. have a relatively high installation cost.\\
b. have relatively high operating costs.\\
*c. are used only for water with low turbidity.\\
d. produce very little backwash sludge.\\


 \item A multi-barrier water filtration plant that contains a flash mix, a coagulation/flocculation zone, sedimentation, filtration and a clear well is considered to be a\\
a. community special treatment plant\\
b. direct filtration plant\\
c. reverse osmosis plant\\
d. *conventional filtration plant\\
e. traditional plant 

 \item Baffling inside a storage facility may be needed to prevent "hot spots" and\\
*a. straight through flows\\
b. long contact times with disinfectants\\
c. high chlorine residuals throughout the facility\\
d. high temperatures throughout the storage facility\\

  \item Which chemical, when it contacts moisture, will produce a cake that is both difficult and hazardous to handle?\\
a. Quicklime\\
b. Calcium oxide\\
c. Calcium hydroxide\\
*d. Soda ash\\

item Which of the following conditions is favorable for the rapid growth of algal?\\
*a. *moderate to high dissolved oxygen and nutrients\\
b. high $\mathrm{pH}$ and water hardness\\
c. low temperatures and low dissolved oxygen\\
d. high alkalinity and water hardness\\

  \item Flocculation is defined as\\
*a. the gathering of fine particles after coagulation by gentle mixing\\
b. clumps of bacteria\\
c. the capacity of water to neutralize acids\\
d. a high molecular weight of compounds that have negative charges\\




\item When handling fluoride chemicals, personnel should wear a respirator or mask approved by\\
a.  OSHA\\
b.  MSA\\
c.  EPA\\
*d.  NIOSH

 \item are used to cause particles to become destabilized and begin to clump together.\\
a) coagulant aids\\
b) nonsettable solids\\
c) zeta particles\\
*d) primary coagulants\\

  \item Which of the following are commonly used coagulation chemicals?\\
a. hypochlorites and free chlorine\\
b. sodium and potassium chlorides\\
c. *alum and polymers\\
d. bleach and HTH\\

  \item Chlorine gas is \rule{1.5cm}{0.5pt} times heavier than breathing air\\
a. 2.5\\
b. 20\\
c. 60\\
d. 460\\

  \item Filters are generally backwashed when the loss-of-head indicator registers a certain set value, such as 6-ft, or upon a certain time, say 48-hours, or upon a rise in\\
a. alkalinity\\
b. a jar-test result\\
c. *turbidity\\
d. temperature\\

  \item Why should a filter be drained if it is going to be out-of-service for a prolonged period?\\
a. to allow the media to dry out\\
b. to save water\\
c. to prevent the filter from floating on groundwater levels\\
d. *to avoid algal growth\\

\item The proper concentration for fluoride in drinking water is determined by:\\
*a. average annual air temperature\\ 
b.  average annual water  temperature\\
c.  average alkalinity\\
d.	average iron concentration\\

\item When operating a filter, one of the operational concerns is the difference between the pressure or head on top of the filter and the pressure or head at the bottom of the filter. This difference is called \rule{1.5cm}{0.5pt} pressure.\\
a. Different\\
*b. Differential\\
c. High\\
d. Low\\
\item List the basic processes, in the proper order, for a conventional treatment plant.\\
*a. Coagulation, flocculation, sedimentation, filtration\\
b. Flocculation, coagulation, sedimentation, filtration\\
c. Filtration, coagulation, flocculation, sedimentation\\
d. Coagulation, sedimentation, flocculation, filtration\\
\item The four most common oxidants include:\\
*a. Chlorine, potassium permanganate, ozone, chlorine dioxide\\
b. Chlorides, soap, air, coagulants\\
c. Air, chemicals, sodium, chloride\\
d. Flocculants, coagulants, sediments, granules\\
\item When comparing conventional treatment with direct filtration, what process unit, \rule{1.5cm}{0.5pt} is in the conventional treatment plant that is not in the direct filtration plant?\\
a. Filter\\
b. Clarifier\\
c. Mixer\\
d. Detention\\

 \item How thick should the layer of sodium fluoride crystals be maintained in a saturator tank for flows of less than 100 gpm?\\
*a. 6 inches\\
b. 10 inches\\
c. 1 foot\\
d. 2 feet\\



  \item How much alkalinity as CaCO$_{3}$ will dry-basis alum consume?\\
*a. 0.5 mg/l\\
b. 0.8 mg/l\\
c. 1.2 mg/l\\
d. 1.5 mg/l\\
  \item Natural zeolites that have become exhausted with use are regenerated by immersing them in a strong solution of which chemical?\\
*a. NaCl\\
b. NaOH\\
c. HCl\\
d. H$_2$SO$_4$ \\
\item The zeta potential on a particular sample of water is -2 . The degree of coagulation is best described as\\
a. poor.\\
b. fair.\\
*c. excellent.\\
d. maximum.\\
  \item Which is a disadvantage of using static mixers?\\
a. They do not provide good mixing\\
b. They are not economical\\
*c. They increase head loss\\
d. They require too much maintenance\\
  \item Which is the usual effective pH range of iron salt coagulants?\\
*a. 3.5 to 9.0\\
b. 6.5 to 8.8\\
c. 3.0 to 9.5\\
d. 4.2 to 9.0\\

  \item Which determines whether or not colloidal-sized particles in suspension repel each other, stay in suspension, or agglomerate and eventually settle?\\
a. Number of collisions\\
b. Flow and temperature of the water\\
c. Types of chemical bonding\\
*d. Magnitude of the charges\\
  \item If the sludge in a sedimentation basin becomes too thick, which could happen?\\
a. Gases from decomposition will rise through the settled sludge accelerating normal floc settling\\
b. Abundant trihalomethanes and haloacetic acids will form\\
*c. Solids can become resuspended or taste and odors can develop\\
d. The sludge will compact at the bottom of the basin making it very difficult to remove\\
  \item In basins using tube and plate settlers, which parameter must be much better than conventional treatment basins?\\
a. Metals must be oxidized before reaching the tubes and plates\\
b. Floc rate must be 2 to 3 times slower than conventional basins\\
c. Floc rate must be 3 to 4 times faster than conventional basins\\
*d. Floc must have good settling characteristics \\
\item Which type of sedimentation basins have the flow of water admitted at an angle?\\
a. Rectangular settling basins\\
b. Square settling basins\\
c. Center-feed settling basins\\
*d. Spiral-flow basins\\
  \item At which minimum angle must self-cleaning tube settlers be placed?\\
*a. 50 degrees\\
b. 60 degrees\\
c. 65 degrees\\
d. 70 degrees\\

  \item If crustaceans have clogged the water treatment plant's filters, which is the most probable solution?\\
a. Shut down the filters and physically remove them\\
b. Shut down one filter at a time and drain; once the crustaceans have died, physically remove them and then repeat process on the other filters\\
c. Backwash filters using a very high concentration of ozone in the water\\
*d. Use a disinfectant that targets the specific organisms in question\\
  \item Which organism can escape coagulation and thus pass through a granular filter?\\
a. Giardia\\
b. Entamoeba\\
*c. Cryptosporidium\\
d. Naegleria\\
  \item Reverse osmosis membranes will compact faster with\\
a. higher iron content.\\
b. higher chlorine contact.\\
*c. higher pressure.\\
d. higher pH.\\
  \item Which would immediately occur if newly installed manganese greensand was not skimmed of the fines after backwashing and stratification steps were completed?\\
a. Uneven flow through the bed\\
b. Cracks would develop in the bed\\
c. Mudball formation\\
*d. Shorter filter runs \\
\item When a filter is operated at normal flow rates, its ability to trap flocculated particles in suspension is a function of\\
a. effective size multiplied by uniformity coefficient.\\
b. effective size multiplied by uniformity coefficient divided by media size.\\
*c. media depth and media size.\\
d. media depth and uniformity coefficient.\\

  \item Which is the total concentration of dissolved solids in the wastewater from the regeneration of ion exchange units?\\
a. 10,000 to 20,000 mg/l\\
b. 20,000 to 30,000 mg/l\\
*c. 35,000 to 45,000 mg/l\\
d. 45,000 to 60,000 mg/l\\
  \item Which is the usual range of percent sludge solids, if the sludge is allowed to accumulate and compact at the bottom of a sedimentation basin?\\
*a. 2-4 \%\\
b. 4-7 \%\\
c. 7-12 \%\\
d. 12-15 \% \\
\item Which should be determined first before an in-ground sedimentation tank is drained?\\
a. The solids content\\
b. The hazardous metals content\\
c. Sludge volume and volume of process area to make sure it will be large enough\\
*d. Water table level\\
  \item Which sludge dewatering process is best for alum sludges (which are difficult to dewater) when the cakes are very dry, filtrate is clear, and solids capture is very high?\\
a. Centrifuge\\
b. Vacuum filters\\
*c. Filter press\\
d. Belt filter press\\
  \item Which sludge dewatering process requires a precoat of diatomaceous earth and its use has declined due to other newer methods?\\
a. Centrifuge\\
*b. Vacuum filters\\
c. Filter press\\
d. Belt filter press\\

 \item It is impossible to produce waters with a hardness of less than \rule{01.5cm}{0.5pt} when using the lime-soda ash process.\\
a. 9 mg/l\\
b. 17 mg/l \\
*c. 25 mg/l \\
d. 50 mg/l \\
  \item When added to water for softening purposes, soda ash will do which of the following?\\
*a. Disinfect the water and kill the vast majority of protozoans, viruses, bacteria, and other multicellular organisms\\
b. Raise the pH of water to between 8.0 and 9.8 pH units\\
c. Add CO2 to the water\\
d. Add calcium alkalinity to the water\\
  \item Magnetic ion exchange resin has been developed to remove\\
*a. total organic carbon.\\
b. chlorides.\\
c. iron and magnesium.\\
d. sulfates and sulfides. 
  \item Approximately how much carbon is lost during the reactivation process for granular activated carbon?\\
*a. 5 \%\\
b. 7 \%\\
c. 10 \%\\
d. 15 \%\\
  \item Which is the most advantageous application point for powdered activated carbon?\\
*a. Raw water intake\\
b. After coagulation\\
c. After oxidation with chlorine\\
d. In the filters\\


  \item To obtain the best and most lasting control of algae, which is the optimum alkalinity range of the water being treated when using copper sulfate for algae control?\\
*a. <50 mg/l\\
b.  50-75 mg/l\\
c.  76-100 mg/l\\
d.  >101 mg/l\\

\item Trout can be killed if the copper sulfate application exceeds what dosage level?\\
a. 0.08 mg/l\\
b. 0.10 mg/l\\
c. 0.12 mg/l\\
*d. 0.14 mg/l\\

  \item The copper sulfate dose for controlling algae in lakes is predominately based on a lake's\\
a. temperature.\\
b. pH.\\
c. turbidity.\\
*d. alkalinity.\\

  \item Which is the typical filtration rate for high-rate filters?\\
a. $\quad 0.5$ to $2.0 \mathrm{gpm} / \mathrm{ft}^{2}$\\
*b. $\quad 3.0$ to $12.0 \mathrm{gpm} / \mathrm{ft}^{2}$\\
c. $\quad 15.0$ to $20.0 \mathrm{gpm} / \mathrm{ft}^{2}$\\
d. $>25.0 \mathrm{gpm} / \mathrm{ft}^{2}$\\

 \item Particle counters use the principle of light\\
a. scattering.\\
b. reflection.\\
c. refraction.\\
*d. blockage.\\

  \item The pressure-reducing and shutoff valve on a vaporizer will shut off when there is a/an\\
*a. loss of electrical power.\\
b. high water level.\\
c. high water temperature.\\
d. over-pressurization of the vaporization system. 
 
  \item How much time does it usually take to slake lime in the detention-time lime slaker?\\
*a. 20 to 30 minutes\\
b. 30 to 45 minutes\\
c. 45 to 60 minutes\\
d. 60 to 75 minutes\\
  \item At which temperature should the slaking process be maintained?\\
a. 120Deg. F\\
b. 135Deg. F\\
c. 150Deg. F\\
*d. 160Deg. F or higher\\

  \item Which would be the most probable solution to control algae in the source water if the algae were clogging the filters at the water plant?\\
a. Use activated carbon\\
b. Decrease oxygen levels\\
c. Backwash filters more frequently\\
*d. Control nutrients\\

  \item A diatom is a type of\\
a. bacterium.\\
*b. algae.\\
c. virus.\\
d. protozoan.\\ 

\item Over which water quality indicator do operators have the greatest control?\\
a. alkalinity\\
b. $\mathrm{pH}$\\
c. temperature\\
d. *turbidity\\

  \item Which method would be the most effective treatment for zebra mussels at the inlet?\\
a. Continuous dosing of chloramines at 0.5 to 1.0 mg/l\\
b. Continuous dosing of chlorine at 0.5 to 1.0 mg/l\\
c. Shock dosages of potassium permanganate at 1.0 to 2.0 mg/l for at least 15 minutes\\
*d. Shock treatment using chlorine at a dosage of 10.0 mg/l for 30 minutes\\
  \item The bulk of synthetic organic compounds (SOCs) are\\
a. solvents.\\
b. PAHs, PCBs, and polynuclear aromatic hydrocarbons.\\
*c. pesticides.\\
d. volatiles.\\


  \item Which one of the following would be best to use for controlling algae in large water bodies?\\
a. Hydrogen peroxide\\
b. Granular activated carbon\\
*c. Powdered activated carbon\\
d. Magnesium sulfate 

  \item For operational corrosion control, the total alkalinity concentration should be measured every\\
*a. 8 hours.\\
b. 12 hours.\\
c. 4 hours.\\
d. Week.\\

\item When manganese greensand filters are run in the continuous regeneration mode [continuous regeneration of the $\mathrm{MnO}_{2}(\mathrm{~s})$ surfaces], the free chlorine residual in the filter effluent should be kept at\\
*a. 0.50 mg/l.\\
b. 1.00 mg/l.\\
c. 1.20 mg/l.\\
d. 1.75 mg/l.\\

\item The chemical that is used most for raising pH is:\\
a. Calgon\\
*b. Lime\\
c. Alum\\
d. Calcium chloride\\

\item Proper coagulant dosage can be determined by:\\
*a. Performing jar test\\
b. The break-point chlorination\\
c. Performing total solids tests\\
d. Observing the pilot filter\\

\item Which will occur if dry alum and quicklime are mixed together\\
a.	Create tremendous heat and hydrogen gas will be released\\
b.	Dust may be released which may cause an explosion\\
c.	A coagulated gel will form	a\\
d.	Nothing as they are neutral towards each other\\

\item Which device collects the settled water as it leaves the sedimentation basins\\
a.	Effluent weir\\
b.	Effluent flow box\\
c. Effluent baffle\\
*d.  Effluent launder\\

\item List three water treatment coagulant chemicals\\
a.	Turbidity, alkalinity, temperature\\
*b.	Alum, ferric chloride, polymers\\
c.	Chlorine, potassium permanganate, ozone\\
d.	Water, pH, alum\\

\item Low values for which water characteristic usually cause poor coagulation flocculation and settling characteristics\\
a.	Water temperature\\
b.	Turbidity\\
c.	Alkalinity\\
d.	pH\\

\item Which of the following is a required treatment technique for the control of lead?\\
a.	Ion exchange\\
*b.	Corrosion control\\
c.	Lime softening\\
d.	Activated carbon\\

\item Recarbonation basins are used to stabilize water after\\
a.	Filtration\\
b.	Disinfection\\
*c.  Softening\\
d.	Coagulation\\
\end{enumerate}

