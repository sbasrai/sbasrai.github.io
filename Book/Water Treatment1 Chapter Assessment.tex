\chapterimage{QuizCover} % Chapter heading image

\chapter*{Chapter 5 Assessment}
% \textbf{Multiple Choice}
\section*{Chapter 5 Assessment}
\begin{enumerate}[1.]
\item What is the purpose of coagulation and flocculation?\\
a. control corrosion\\
b. to kill disease causing organisms\\
c. to remove leaves, sticks, and fish debris\\
d. to remove particulate impurities and suspended matter\\
\item How are filter production (capacity) rates measured?\\
a. Mgd/sq.ft.\\
b. Gpm/sq.ft.\\
c. Gpm\\
d. Mgd\\
\item Why should a filter be drained if it is going to be out-of-service for a prolonged period?\\
a. to allow the media to dry out\\
b. to save water\\
c. to prevent the filter from floating on groundwater levels\\
d. to avoid algal growth\\
\item Which of the following are commonly used coagulation chemicals?\\
a. hypochlorites and free chlorine\\
b. sodium and potassium chlorides\\
c. alum and polymers\\
d. bleach and HTH\\
\item How can an operator tell if a filter is NOT completely cleaned after backwashing?\\
a. the initial headloss is on the high side\\
b. the backwash rate was too slow\\
c. mudballs are NOT present\\
d. backwashing pumping rate is too low\\
\item Flocculation is defined as\\
a. the gathering of fine particles after coagulation by gentle mixing\\
b. clumps of bacteria\\
c. the capacity of water to neutralize acids\\
d. a high molecular weight of compounds that have negative charges\\
\item A multi-barrier water filtration plant that contains a flash mix, a coagulation/flocculation zone, sedimentation, filtration and a clear well is considered to be a\\
a. community special treatment plant\\
b. direct filtration plant\\
c. reverse osmosis plant\\
d. conventional filtration plant\\
e. traditional plant\\
\item The filtration unit process usually\\
a. is located at the beginning of a filtration plant\\
b. follows the coagulation/flocculation/sedimentation processes\\
c. is located after the clear well area\\
d. is located on the plant effluent line after the clearwell\\
\item Filters are generally backwashed when the loss-of-head indicator registers a certain set value, such as 6-ft, or upon a certain time, say 48-hours, or upon a rise in\\
a. alkalinity\\
b. a jar-test result\\
c. turbidity\\
d. temperature\\
\item What is a method of reducing hardness?\\
a. Softening\\
b. Hardening\\
c. Lightning\\
d. Flashing\\
\item The solid that adsorbs a contaminant is called the:\\
a. Adsorbent\\
b. Adsorbate\\
c. Sorbet\\
d. Rock\\
\item The adsorption process is used to remove:\\
a. Organics or inorganics\\
b. Bugs or salts\\
c. Organisms or dirt\\
d. Color or particles\\
\item Describe two primary methods used to control taste and odor?\\
a. Oxidation and adsorption\\
b. Filtration and sedimentation\\
c. Mixing and coagulation\\
d. Sedimentation and clarification\\
\item What is the recommended loading rate for copper sulfate for algae control at an alkalinity greater than $50 \mathrm{mg} / \mathrm{L}$ ?\\
a. 0.9 of copper sulfate per acre of surface area\\
b. 1.9 of copper sulfate per acre of surface area\\
c. 2-4 lb of copper sulfate per acre of surface area\\
d. 5.4 of copper sulfate per acre of surface area\\
\item The basic goal for water treatment is to\\
a. Protect public health\\
b. Make it clear\\
c. Make it taste good\\
d. Get stuff out\\
\item Greensand can be operated in either \rule{1.5cm}{0.5pt} regeneration or \rule{1.5cm}{0.5pt} regeneration modes.\\
a. Continuous or intermittent\\
b. Fast or slow\\
c. Hot or cold\\
d. Constant or unusual\\

\end{enumerate}



