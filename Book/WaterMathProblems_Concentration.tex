\documentclass{article}
%\usepackage[english]{babel}%
\usepackage{graphicx}
\usepackage{tabulary}
\usepackage{tabularx}
\usepackage[table,xcdraw]{xcolor}
\usepackage{pdflscape}
%\usepackage{gensymb}
\usepackage{lastpage}
\usepackage{multirow}
\usepackage{xcolor}
\usepackage{cancel}
\usepackage{amsmath}
\usepackage[table]{xcolor}
\usepackage{fixltx2e}
\usepackage[T1]{fontenc}
\usepackage[utf8]{inputenc}
\usepackage{ifthen}
\usepackage{fancyhdr}
\usepackage[utf8]{inputenc}
\usepackage{tikz}
\usepackage[document]{ragged2e}
\usepackage[margin=1in,top=1.2in,headheight=57pt,headsep=0.1in]
{geometry}
\usepackage{ifthen}
\usepackage{fancyhdr}
\everymath{\displaystyle}
\usepackage[document]{ragged2e}
\usepackage{fancyhdr}
\usepackage{mathabx}
\usepackage{textcomp,mathcomp}
\usepackage[shortlabels]{enumitem}
\everymath{\displaystyle}
\linespread{2}%controls the spacing between lines. Bigger fractions means crowded lines%
\linespread{1.3}%controls the spacing between lines. Bigger fractions means crowded lines%
\pagestyle{fancy}
\setlength{\headheight}{56.2pt}
\usepackage{soul}
\usepackage{siunitx}

%\usepackage{textcomp}
\usetikzlibrary{shapes.multipart, shapes.geometric, arrows}
\usetikzlibrary{calc, decorations.markings}
\usetikzlibrary{arrows.meta}
\usetikzlibrary{shapes,snakes}
\usetikzlibrary{quotes,angles, positioning}
%\chead{\ifthenelse{\value{page}=1}{\includegraphics[scale=0.3]{BassettCTCLogo}}}
%\rhead{\ifthenelse{\value{page}=1}{Final Exam}{}}
%\lhead{\ifthenelse{\value{page}=1}{Water Treatment - Oct-Dec 2022}{\textbf Final Exam}}
%\rfoot{\ifthenelse{\value{page}=1}{}{}}
%
%\cfoot{}
%\lfoot{Page \thepage\ of \pageref{LastPage}}
%\renewcommand{\headrulewidth}{2pt}
%\renewcommand{\footrulewidth}{1pt}
\begin{document}
\begin{enumerate}

\item What is the solids content in mg/l of a 2.5\% sludge?\\
Answer:  25,000 mg/l\\

\item How many lbs of salt needs to be dissolved in water to make 1 liter of 5\% salt solution?\\
Solution:\\
$5\% \enspace salt \enspace solution \enspace \implies 50,000 \enspace mg/l \enspace salt$\\
To prepare 1 litre of salt solution need to dissolve 50,000 mg or:\\
$50,000 \enspace mg*\dfrac{lb}{453.6 \enspace gms}*\dfrac{gm}{1,000 \enspace mg}=\boxed{0.11 \enspace lb \enspace salt}\enspace$in enough  water to make 1 liter of solution.\\

\item What is the concentration in mg/l of  4.5\% solution of that substance.

\item How many lbs of salt is needed to make 5 gallons of a 2500mg/l salt solution

$2500mg/l = 2500ppm = \dfrac{2500 \enspace lbs \enspace salt}{1,000,000 \enspace lbs \enspace salt \enspace solution}*5*8.34 \enspace salt \enspace solution=\boxed{0.1 \enspace lbs \enspace salt }$


\item An operator mixes 40 lb of lime in a 100-gal tank containing 80 gal of water. What is the percent of lime in the slurry?
\vspace{0.2cm}
Solution:\\
\vspace{0.2cm}
$\Bigg(\dfrac{40 \enspace lbs \enspace lime}{80 \enspace gal \enspace water*8.34\dfrac{lbs}{gal \enspace water}+40\enspace lbs \enspace lime}*\dfrac{1,000,000 \enspace lbs}{million \enspace lbs}\Bigg)*\dfrac{\%}{10,000 \enspace ppm}=\boxed{5.7\%}$
\end{enumerate}



\end{document}