\documentclass{article}
%\usepackage[english]{babel}%
\usepackage{graphicx}
\usepackage{tabulary}
\usepackage{tabularx}
\usepackage[table,xcdraw]{xcolor}
\usepackage{pdflscape}
%\usepackage{gensymb}
\usepackage{lastpage}
\usepackage{multirow}
\usepackage{xcolor}
\usepackage{cancel}
\usepackage{amsmath}
\usepackage[table]{xcolor}
\usepackage{fixltx2e}
\usepackage[T1]{fontenc}
\usepackage[utf8]{inputenc}
\usepackage{ifthen}
\usepackage{fancyhdr}
\usepackage[utf8]{inputenc}
\usepackage{tikz}
\usepackage[document]{ragged2e}
\usepackage[margin=1in,top=1.2in,headheight=57pt,headsep=0.1in]
{geometry}
\usepackage{ifthen}
\usepackage{fancyhdr}
\everymath{\displaystyle}
\usepackage[document]{ragged2e}
\usepackage{fancyhdr}
\usepackage{mathabx}
\usepackage{textcomp,mathcomp}
\usepackage[shortlabels]{enumitem}
\everymath{\displaystyle}
\linespread{2}%controls the spacing between lines. Bigger fractions means crowded lines%
\linespread{1.3}%controls the spacing between lines. Bigger fractions means crowded lines%
\pagestyle{fancy}
\setlength{\headheight}{56.2pt}
\usepackage{soul}
\usepackage{siunitx}

%\usepackage{textcomp}
\usetikzlibrary{shapes.multipart, shapes.geometric, arrows}
\usetikzlibrary{calc, decorations.markings}
\usetikzlibrary{arrows.meta}
\usetikzlibrary{shapes,snakes}
\usetikzlibrary{quotes,angles, positioning}
%\chead{\ifthenelse{\value{page}=1}{\includegraphics[scale=0.3]{BassettCTCLogo}}}
%\rhead{\ifthenelse{\value{page}=1}{Final Exam}{}}
%\lhead{\ifthenelse{\value{page}=1}{Water Treatment - Oct-Dec 2022}{\textbf Final Exam}}
%\rfoot{\ifthenelse{\value{page}=1}{}{}}
%
%\cfoot{}
%\lfoot{Page \thepage\ of \pageref{LastPage}}
%\renewcommand{\headrulewidth}{2pt}
%\renewcommand{\footrulewidth}{1pt}
\begin{document}

\begin{enumerate}

\item There are four filters at a water treatment plant. The filters measure 20 feet wide by 30 feet in length. What is the filtration rate if the plant processes 8.0 MGD?\\
1.	1.51 GPM/sq.ft\\
2.	2.31 GPM/sq.ft\\
3.	2.61 GPM/sq.ft\\
4.	2.91 GPM/sq.ft\\

Filtration Rate:  $ \dfrac{8,000,000\dfrac{gal}{day}*\dfrac{day}{24hrs}*\dfrac{hr}{60min}}{20*30*4 \enspace sq.ft} = \boxed{2.31 \enspace GPM/sq.ft}$

\item A water treatment plant treats 6.0 MGD with four filters. Each filter use 60,000 gallons per wash. What is the percent backwash at the plant?\\
1.	10%\\
2.	8%\\
3.	6%\\
4.	4%\\

Backwash water, $\%=\dfrac{60,000*4 \text { gal }}{6,000,000 \text { gal }} \times 100 = \boxed{4 \%}$
\item A treatment plant filter washes at a rate of 10,000 GPM. The filter measures 18ft. wide by 24ft. long. What is the rate of rise expressed in inches per minute?\\
1.	17 inch/min\\
2.	27 inch/min\\
3.	37 inch/min\\
4.	47 inch/min\\

$$\text{ Backwash rinse rate, in/} \mathrm{min}=\frac{\text { Backwash rate, } \mathrm{gpm} / \mathrm{ft}^{2} \times 12 \mathrm{in} / \mathrm{ft}}{7.48 \mathrm{gal} / \mathrm{ft}^{3}}$$

\textit{Based upon the above formula, the Backwash tate in $gpm/ft^2$ needs to be calculated by dividing the gpm flow by the surface area}

$\text{Backwash Rinse Rate, in/} \mathrm{min}=\dfrac{
\Biggl(\dfrac{10,000 \mathrm{gpm}}{18 \mathrm{ft} \times 24 \mathrm{ft}}\Biggr) \times \dfrac{12 \mathrm{in}}{\mathrm{ft}
}
{
7.48 \mathrm{gal} / \mathrm{ft}^{3}
}=\boxed{37in/min}$
\item If a filter measures 20 feet by 30 feet by 7 foot deep and the backwash flow is $3.5 \mathrm{cuft} / \mathrm{sec}$, what is the backwash rate?\\
a) $1.1 \mathrm{gpm} / \mathrm{sqft}$\\
b) $3.3 \mathrm{gpm} / \mathrm{sqft}$\\
*c) $2.6 \mathrm{gpm} / \mathrm{sqft}$\\
d) $1.7 \mathrm{gpm} / \mathrm{sqft}$\\
\vspace{0.2cm}
Backwash Rate (gpm/sq.ft)=$\dfrac{\dfrac{3.5 \enspace ft^3}{sec}*\dfrac{7.48 \enspace gal}{ft^3}*\dfrac{60 \enspace sec}{min}}{(20*30) \enspace ft^2}=\boxed{\dfrac{2.6 \enspace gpm}{ft}}$
 \item A treatment facility treats 9.5 MGD through the use of six (6) filters, each measuring $20 \mathrm{ft}$ wide by $20 \mathrm{ft}$ long. What is their filtration rate?\\
a) $16.50 \mathrm{gpm} / \mathrm{sqft}$\\
b) $1.77 \mathrm{gpm} / \mathrm{sqft}$\\
*c) $2.75 \mathrm{gpm} / \mathrm{sqft}$\\
d) $4.76 \mathrm{gpm} / \mathrm{q} / \mathrm{ft}$\\
\vspace{0.2cm}
Filtration Rate:  $ \dfrac{9,500,000\dfrac{gal}{day}*\dfrac{day}{24hrs}*\dfrac{hr}{60min}}{(20*20)*6 \enspace ft^2} = \boxed{2.75 \enspace gpm/sq.ft}$
      \item The filters in the treatment plant are 40 feet by 20 feet by 7 feet deep. The flow is $1500 \mathrm{gpm}$. What is the filtration rate?\\
a) $.26 \mathrm{gpm} / \mathrm{sq} \mathrm{ft}$\\
*b) $1.9 \mathrm{gpm} / \mathrm{sq} \mathrm{ft}$\\
c) $2.6 \mathrm{gpm} / \mathrm{sq} \mathrm{ft}$\\
d) $3.7 \mathrm{gpm} / \mathrm{sq} \mathrm{ft}$\\
\vspace{0.2cm}
Filtration Rate:  $\dfrac{1,500\dfrac{gal}{min}}{(40*20) \enspace ft^2} = \boxed{1.9 \enspace gpm/sq.ft}$

  \item Calculate the weir overflow rate if your flow is $3.1 \mathrm{cuft} / \mathrm{sec}$ and the diameter of the weir is $28 \mathrm{ft}$..\\
a) $1391.28 \mathrm{gpm} / \mathrm{ft}$ of weir\\
*b) $15.8 \mathrm{gpm} / \mathrm{ft}$ of weir\\
c) $.035 \mathrm{gpm} / \mathrm{ft}$ of weir\\
d) $296 \mathrm{gpm} / \mathrm{ft}$ of weir\\

\item A filter box is 20 ft by 30 ft (including the sand area). If the influent valve is shut, the water drops 3 inches per minute. What is the rate of filtration in MGD?\\
\vspace{0.2cm}
Water passing through the filter - Rate of Filtration (ft$^3$/min) = $600ft^2*\dfrac{3in}{min}*\dfrac{ft}{12in}=\dfrac{150ft^3}{min}$\\
\vspace{0.2cm}
$\implies Rate \enspace of Filtration(MGD)=\dfrac{150\cancel{ft^3}}{\cancel{min}}*\dfrac{7.48\cancel{gal}}{\cancel{ft^3}}*\dfrac{MG}{1,000,000\cancel{gal}}*\dfrac{1440\cancel{min}}{day}=\boxed{1.62MGD}$


 

 

\item The flow rate through a filter is 4.25 MGD. What is this flow rate expressed as gpm?\\

\vspace{0.2cm}

$Flow rate, gpm=\dfrac{Flow \enspace rate, \enspace gpd}{1440 \enspace min/day}$\\

\vspace{0.2cm}

Note:  We are assuming that the filter operated uniformly over that 24 hour period.\\

\vspace{0.3cm}

$Flow rate, gpm=\dfrac{4.25 \enspace \dfrac{\cancel{MG}}{\cancel{day}} *1,000,000 \enspace \dfrac{gal}{\cancel{MG}}}{1440\dfrac{min}{\cancel{day}}}=\boxed{2,951 \enspace gpm}$

 

\vspace{0.3cm}

\item At an average flow rate of 4000 gpm, how long of a filter run, in hours, would be required to produce 25 MG of filtered water?\\

\vspace{0.2cm}

$Flow \enspace rate \enspace (gpm)=\dfrac{Total \enspace flow \enspace (gal)}{Filter \enspace run \enspace time \enspace (min)}$

\vspace{0.3cm}

$\implies Filter \enspace run \enspace time \enspace (min)=\dfrac{Total \enspace flow \enspace (gal)}{Flow \enspace rate \enspace (gpm)}$\\

\vspace{0.3cm}

$\implies Filter \enspace run \enspace time \enspace (hr)=25 \enspace MG*\dfrac{1,000,000 \enspace \cancel{gal}}{MG}*\dfrac{\cancel{min}}{4,000 \enspace \cancel{gal}}*60 \enspace \dfrac{hr}{\cancel{min}}=\boxed{104 \enspace hrs}$

\item A filter $28 \mathrm{ft}$ long by $18 \mathrm{ft}$ wide treats a flow of $3.5 \mathrm{MGD}$. What is the filtration rate in gpm/ft ${ }^{2}$ ?\\

\vspace{0.2cm}
\textit{Approach:  The flow will need to be converted to gpm and the surface area calculated in feet.}\\

$\text{Filtration rate, } \mathrm{gpm} / \mathrm{ft}^{2} = 
\dfrac{
\dfrac{3.5 \enspace \cancel{MG}}{ \cancel{day}} * \dfrac{1,000,000 \enspace gal}{\cancel{MG}}
*\dfrac{\cancel{day}}{1440 \mathrm{ min}}}
{28 \enspace ft * 18 \enspace feet}= \boxed{4.8 \enspace gpm/ft^2}$\\
\vspace{0.2cm}
\item A filter is $40 \mathrm{ft}$ long by $20 \mathrm{ft}$ wide. During a test of flow rate, the influent valve to the filter is closed for 6 minutes. The water level drop during this period is 16 inches. What is the filtration rate for the filter in $\mathrm{gpm} / \mathrm{ft}^{2}$ ?\\
\vspace{0.2cm}
\textit{Note:  The volume of the water dropped after the inlet valve was closed would be the filter flow rate.  Since the dimensions to calculate are in feet and inches, the volume needs to be converted from ft$^3$ to gallons}\\
\vspace{0.2cm}
$\text{Filtration rate, } \mathrm{gpm} / \mathrm{ft}^{2} = 
\dfrac{(
40 \mathrm{ ft}*20 \mathrm{ ft} * 16 \mathrm{\cancel{in}}*
\dfrac{ft}{12 \enspace \cancel{in}}
)
\cancel{ft^3}*7.48 \enspace 
\dfrac
{gal}
{\cancel{ft^3}}}
{40 \enspace ft * 20 \enspace feet}= \boxed{1.7\enspace gpm/ft^2}$\\

\item A filter has the following dimensions: $30 \mathrm{ft}$ long by $20 \mathrm{ft}$ wide with a depth of 24 inches of filter media. Assuming that a backwash rate of $15 \mathrm{gal} / \mathrm{ft}^{2} / \mathrm{min}$ is recommended and 10 minutes of backwash is required, calculate the amount of water, in gallons, required for each backwash.

\textit{The backwashing rate given in $gal/ft^2/min$ will need to be converted into gallons by multiplying it with the area (to eliminate $ft^2$ and by the backwash time in minutes}

$ \text{Backwashing rate (gal)} = 15\dfrac{gal}{\cancel{ft^2}-\cancel{min}}*(30 \mathrm{ ft} \times 20 \mathrm{ ft})\cancel{ft^2}*10 \enspace \cancel{min}=\boxed{90,000 \enspace \text{gal}}$

\item A filter $22 \mathrm{ft}$ long by $12 \mathrm{ft}$ wide has a backwash rate of $3260 \mathrm{gpm}$. What is this backwash rate expressed as a in/min rise?

$$\text{ Backwash rinse rate, in/} \mathrm{min}=\frac{\text { Backwash rate, } \mathrm{gpm} / \mathrm{ft}^{2} \times 12 \mathrm{in} / \mathrm{ft}}{7.48 \mathrm{gal} / \mathrm{ft}^{3}}$$

\textit{Based upon the above formula, the Backwash tate in $gpm/ft^2$ needs to be calculated by dividing the gpm flow by the surface area}

$\text{Backwash Rinse Rate, in/} \mathrm{min}=\dfrac{
\Biggl(\dfrac{3260 \mathrm{gpm}}{22 \mathrm{ft} \times 12 \mathrm{ft}}\Biggr) \mathrm{gpm} / \mathrm{ft}^{2} \times 12 \mathrm{in} / \mathrm{ft}
}
{
7.48 \mathrm{gal} / \mathrm{ft}^{3}
}=\boxed{19.7in/min}$

\item A total of $11,400,000$ gal of water was filtered during a filter run. If backwashing used 48,500 gal of this product water, what percent of the product water is used for backwashing?

Backwash water, $\%=\dfrac{48,500 \text { gal }}{11,400,000 \text { gal }} \times 100 = \boxed{0.43 \%}$

\end{enumerate}


\end{document}