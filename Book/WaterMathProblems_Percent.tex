\documentclass{article}
%\usepackage[english]{babel}%
\usepackage{graphicx}
\usepackage{tabulary}
\usepackage{tabularx}
\usepackage[table,xcdraw]{xcolor}
\usepackage{pdflscape}
%\usepackage{gensymb}
\usepackage{lastpage}
\usepackage{multirow}
\usepackage{xcolor}
\usepackage{cancel}
\usepackage{amsmath}
\usepackage[table]{xcolor}
\usepackage{fixltx2e}
\usepackage[T1]{fontenc}
\usepackage[utf8]{inputenc}
\usepackage{ifthen}
\usepackage{fancyhdr}
\usepackage[utf8]{inputenc}
\usepackage{tikz}
\usepackage[document]{ragged2e}
\usepackage[margin=1in,top=1.2in,headheight=57pt,headsep=0.1in]
{geometry}
\usepackage{ifthen}
\usepackage{fancyhdr}
\everymath{\displaystyle}
\usepackage[document]{ragged2e}
\usepackage{fancyhdr}
\usepackage{mathabx}
\usepackage{textcomp,mathcomp}
\usepackage[shortlabels]{enumitem}
\everymath{\displaystyle}
\linespread{2}%controls the spacing between lines. Bigger fractions means crowded lines%
\linespread{1.3}%controls the spacing between lines. Bigger fractions means crowded lines%
\pagestyle{fancy}
\setlength{\headheight}{56.2pt}
\usepackage{soul}
\usepackage{siunitx}

%\usepackage{textcomp}
\usetikzlibrary{shapes.multipart, shapes.geometric, arrows}
\usetikzlibrary{calc, decorations.markings}
\usetikzlibrary{arrows.meta}
\usetikzlibrary{shapes,snakes}
\usetikzlibrary{quotes,angles, positioning}
%\chead{\ifthenelse{\value{page}=1}{\includegraphics[scale=0.3]{BassettCTCLogo}}}
%\rhead{\ifthenelse{\value{page}=1}{Final Exam}{}}
%\lhead{\ifthenelse{\value{page}=1}{Water Treatment - Oct-Dec 2022}{\textbf Final Exam}}
%\rfoot{\ifthenelse{\value{page}=1}{}{}}
%
%\cfoot{}
%\lfoot{Page \thepage\ of \pageref{LastPage}}
%\renewcommand{\headrulewidth}{2pt}
%\renewcommand{\footrulewidth}{1pt}
\begin{document}


\section{Percent}\index{Percent}
\textbf{Example:}\\
What is $28 \%$ of $286 ?$\\

\begin{enumerate}[Step 1.]
\item Change the $28 \%$ to a decimal equivalent:  $$28 \% \div 100=0.28$$
\item Multiply $286 \times 0.28=80$\\
Thus $28 \%$ of 286 is 80.
\end{enumerate}

\textbf{Example:} A filter bed will expand $25 \%$ during backwash. If the filter bed is 36 inches deep, how deep will it be during backwash?\\

\begin{enumerate}[Step 1.]
\item Change the percent to a decimal.
$$
25 \% \div 100=0.25
$$
\item Add the whole number 1 to this value.
$$
1+0.25=1.25
$$
\item Multiply times the value.
$$
36 \text { in } \times 1.25=45 \text { inches }
$$
\end{enumerate}

\begin{enumerate}

\item What is 20\% of 250?\\
Solution:\\
$20\%=\frac{20}{100}=0.2 \implies 20\% \enspace of \enspace 250=0.2*250=\boxed{50}$

\item What percent is 0.4 of 4?\\
Solution:\\
$x\%=\frac{x}{100} \implies 0.4=\frac{x}{100}*4 \implies x=\frac{0.4*100}{4}=\boxed{10\%}$

\end{enumerate}


\subsection{Percentage Concentrations}

\textbf{Example 1:} A chlorine solution was made to have a $4 \%$ concentration. It is often desirable to determine this concentration in $\mathrm{mg} / \mathrm{L}$. This is relatively simple: the $4 \%$ is four percent of a million.

To find the concentration in $\mathrm{mg} / \mathrm{L}$ when it is expressed in percent, do the following:

\begin{enumerate}
  \item Change the percent to a decimal.
\end{enumerate}
$$
4 \% \div 100=0.04
$$

\begin{enumerate}
  \setcounter{enumi}{2}
  \item Multiply times a million.
\end{enumerate}
$$
0.04 \times 1,000,000=40,000 \mathrm{mg} / \mathrm{L}
$$
We get the million because a liter of water weighs $1,000,000 \mathrm{mg} .1 \mathrm{mg}$ in 1 liter is 1 part in a million parts ( $\mathrm{ppm}) .1 \%=10,000 \mathrm{mg} / \mathrm{L}$.


\textbf{Example 2:} How much $65 \%$ calcium hypochlorite is required to obtain 7 pounds of pure chlorine?\\
$65 \%$ implies that in every lb of calcium hypochlorite has $65 \%$ lbs of available chlorine.\\
\vspace{0.2cm}
Therefore, $\dfrac{0.65 \textrm{ lbs available chlorine}}{\textrm{lb of calcium hypochlorite}} $ or conversely $\dfrac{\textrm{lb of calcium hypochlorite}}{0.65 \textrm{ lbs available chlorine}}$\\
\vspace{0.2cm}
$\implies{\textrm{lbs calcium hypchlorite required}}=\dfrac{\textrm{lb of calcium hypochlorite}}{0.65 \cancel{\textrm{ lbs available chlorine}}}*\dfrac{7\cancel{\textrm{ lb of available chlorine}}}{}$\\
\vspace{0.2cm}
$=\boxed{10.8 \textrm{ lbs of calcium hypochlorite with } 65\%\textrm{available chlorine is required}}$


\begin{enumerate}

\item $25 \%$ of the chlorine in a 30-gallon vat has been used. How many gallons are remaining in the vat?\\
Solution:\\
Amount of chlorine remaining in the vat is 100\%-25\%=75\%\\

Gallons of chlorine remaining in the vat: $30*0.75=\boxed{22.5 \enspace gallons}$

\item The annual public works budget is $\$ 147,450$. If $75 \%$ of the budget should be spent by the end of September, how many dollars are to be spent? How many dollars will be remaining?\\
\vspace{0.2cm}
Solution:\\
Amount to be spent = \$147,450*0.75 = $\boxed{\$110,812.50}$\\
\vspace{0.2cm}
Amount remaining = \$ 147,450 - 110,812.50 = $\boxed{\$36,367.50}$

\item A 75 pound container of calcium hypochlorite has a purity of $67 \%$. What is the actual weight of the calcium hypochlorite in the container? \\
\vspace{0.2cm}
Solution:\\
Note: Calcium Hypochlorite can be written as Ca(OCl)$_2$\\
$75 \enspace lbs \enspace Ca(OCl)_2 \enspace - \enspace product \enspace in \enspace container*\dfrac{0.67 \enspace lbs \enspace Ca(OCl)_2 }{lb \enspace Ca(OCl)_2  \enspace - \enspace product \enspace in \enspace container} = \boxed{50.25 \enspace lbs \enspace Ca(OCl)_2}$\\
\vspace{0.2cm}


\item $3 / 4$ is the same as what percentage?\\
\vspace{0.2cm}
$\dfrac{3}{4}=0.75 \enspace which \enspace is \enspace \dfrac{75}{100} = \boxed{75\%}$\\
\vspace{0.2cm}

\item An operator mixes 40 lb of lime in a 100-gal tank containing 80 gal of water. What is the percent of lime in the slurry?
\vspace{0.2cm}
Solution:\\
\vspace{0.2cm}
\end{enumerate}

\end{document}