\documentclass{article}
%\usepackage[english]{babel}%
\usepackage{graphicx}
\usepackage{tabulary}
\usepackage{tabularx}
\usepackage[table,xcdraw]{xcolor}
\usepackage{pdflscape}
%\usepackage{gensymb}
\usepackage{lastpage}
\usepackage{multirow}
\usepackage{xcolor}
\usepackage{cancel}
\usepackage{amsmath}
\usepackage[table]{xcolor}
\usepackage{fixltx2e}
\usepackage[T1]{fontenc}
\usepackage[utf8]{inputenc}
\usepackage{ifthen}
\usepackage{fancyhdr}
\usepackage[utf8]{inputenc}
\usepackage{tikz}
\usepackage[document]{ragged2e}
\usepackage[margin=1in,top=1.2in,headheight=57pt,headsep=0.1in]
{geometry}
\usepackage{ifthen}
\usepackage{fancyhdr}
\everymath{\displaystyle}
\usepackage[document]{ragged2e}
\usepackage{fancyhdr}
\usepackage{mathabx}
\usepackage{textcomp,mathcomp}
\usepackage[shortlabels]{enumitem}
\everymath{\displaystyle}
\linespread{2}%controls the spacing between lines. Bigger fractions means crowded lines%
\linespread{1.3}%controls the spacing between lines. Bigger fractions means crowded lines%
\pagestyle{fancy}
\setlength{\headheight}{56.2pt}
\usepackage{soul}
\usepackage{siunitx}

%\usepackage{textcomp}
\usetikzlibrary{shapes.multipart, shapes.geometric, arrows}
\usetikzlibrary{calc, decorations.markings}
\usetikzlibrary{arrows.meta}
\usetikzlibrary{shapes,snakes}
\usetikzlibrary{quotes,angles, positioning}
%\chead{\ifthenelse{\value{page}=1}{\includegraphics[scale=0.3]{BassettCTCLogo}}}
%\rhead{\ifthenelse{\value{page}=1}{Final Exam}{}}
%\lhead{\ifthenelse{\value{page}=1}{Water Treatment - Oct-Dec 2022}{\textbf Final Exam}}
%\rfoot{\ifthenelse{\value{page}=1}{}{}}
%
%\cfoot{}
%\lfoot{Page \thepage\ of \pageref{LastPage}}
%\renewcommand{\headrulewidth}{2pt}
%\renewcommand{\footrulewidth}{1pt}
\begin{document}
\begin{enumerate}
\item It takes 6 gallons of chlorine solution to obtain a proper residual when the flow is 45,000 gpd. How many gallons will it take when the flow is 62,000 gpd?

\item A motor is rated at 41 amps average draw per leg at $30 \mathrm{Hp}$. What is the actual $\mathrm{Hp}$ when the draw is 36 amps? C. 

\item If it takes 2 operators $4.5$ days to clean an aeration basin, how long will it take three operators to do the same job?

\item It takes 3 hours to clean 400 feet of collection system using a sewer ball. How long will it take to clean 250 feet?

\item It takes 14 cups of $\mathrm{HTH}$ to make a $12 \%$ solution, and each cup holds 300 grams. How many cups will it take to make a $5 \%$ solution?
\end{enumerate}
\vspace{1cm}

\textbf{Solution}
\begin{enumerate}
\item The gallons chlorine and flow are directly related. 

Thus,

$\dfrac{6}{45,000}=\dfrac{X}{62,000} \implies X=\dfrac{6*62,000}{45,000}=8.3 \mathrm{gallons}$


\vspace{0.5cm}

\item The amp draw and Hp are directly related.

This

$\dfrac{30}{41}=\dfrac{X}{36} \implies X=\dfrac{30*36}{41}=26.3 \mathrm{Hp}$

\vspace{0.5cm}

\item The number of operators and the days to clean are inversely related.

Thus,

$2 * 4.5 = 3*X \implies X = \dfrac{2*4.5}{3} = 3 \mathrm{days}$



\vspace{0.5cm}

\item The hours to clean and the length of system cleaned are directly proportional.

Thus,

$\dfrac{3}{400}=\dfrac{X}{250} \implies X=\dfrac{3*250}{400}=1.9 \mathrm{hours}$

\vspace{0.5cm}

\item The cups of HTH and percentage HTH solution are directly propoirtional.

Thus,

$\dfrac{14}{12}=\dfrac{X}{5} \implies X=\dfrac{14*5}{12}=5.8 \mathrm{cups}$

\item It takes 6 gallons of chlorine solution to obtain a proper residual when the flow is 45,000 gpd. How many gallons will it take when the flow is 62,000 gpd?\\
\vspace{0.2cm}
Solution:\\
\vspace{0.2cm}
Required gallons of chlorine is directly proportional to the flow being treated.\\
\vspace{0.2cm}
Thus, $\dfrac{6 \enspace gallons}{45,000 \enspace gpd }=\dfrac{X \enspace gallons}{62,000 \enspace gpd}$
\vspace{0.2cm}
Solving for X:\\
\vspace{0.2cm}
$\implies \enspace X=\dfrac{6*62,000}{45,000}=\boxed{8.3 \enspace lbs \enspace bleach}$
\vspace{0.2cm}

\item A motor is rated at 41 amps average draw per leg at $30 \mathrm{Hp}$. What is the actual $\mathrm{Hp}$ when the draw is 36 amps? C. 
\vspace{0.2cm}
Solution:\\
\vspace{0.2cm}
Ampere draw and horsepower (Hp) are directly proportional - when Hp goes up, the ampere draw goes up\\
\vspace{0.2cm}
Thus, $\dfrac{30 \enspace Hp}{41 \enspace Amperes }=\dfrac{X \enspace Hp}{36 \enspace amperes}$
\vspace{0.2cm}
Solving for X:\\
\vspace{0.2cm}
$\implies \enspace X=\dfrac{6*62,000}{45,000}=\boxed{8.3 \enspace lbs \enspace bleach}$
\vspace{0.2cm}
\item If it takes 2 operators $4.5$ days to clean an aeration basin, how long will it take three operators to do the same job?
\vspace{0.2cm}
Solution:\\
\vspace{0.2cm}
Number of operators and the time required to accomplish a certain task are inversely proportional - when more operators are involved, the task will take less time.\\
\vspace{0.2cm}
$(2 \enspace \mathrm{Operators} * 4.5 \enspace \mathrm{days})=(3 \enspace \mathrm{Operators} * X \enspace \mathrm{days})$
\vspace{0.2cm}
Solving for X:\\
\vspace{0.2cm}
$\implies \enspace X=\dfrac{2*4.5}{3}=\boxed{3 \enspace days}$
\vspace{0.2cm}

\item It takes 6 gallons of chlorine solution to obtain a proper residual when the flow is 45,000 gpd. How many gallons will it take when the flow is 62,000 gpd?\\
\vspace{0.2cm}
Solution:\\
\vspace{0.2cm}
Required gallons of chlorine is directly proportional to the flow being treated.\\
\vspace{0.2cm}
Thus, $\dfrac{6 \enspace gallons}{45,000 \enspace gpd }=\dfrac{X \enspace gallons}{62,000 \enspace gpd}$
\vspace{0.2cm}
Solving for X:\\
\vspace{0.2cm}
$\implies \enspace X=\dfrac{6*62,000}{45,000}=\boxed{8.3 \enspace lbs \enspace bleach}$
\vspace{0.2cm}

\item A motor is rated at 41 amps average draw per leg at $30 \mathrm{Hp}$. What is the actual $\mathrm{Hp}$ when the draw is 36 amps? C. 
\vspace{0.2cm}
Solution:\\
\vspace{0.2cm}
Ampere draw and horsepower (Hp) are directly proportional - when Hp goes up, the ampere draw goes up\\
\vspace{0.2cm}
Thus, $\dfrac{30 \enspace Hp}{41 \enspace Amperes }=\dfrac{X \enspace Hp}{36 \enspace amperes}$
\vspace{0.2cm}
Solving for X:\\
\vspace{0.2cm}
$\implies \enspace X=\dfrac{30*36}{41}=\boxed{26.3 \enspace Hp}$
\vspace{0.2cm}
\item If it takes 2 operators $4.5$ days to clean an aeration basin, how long will it take three operators to do the same job?
\vspace{0.2cm}
Solution:\\
\vspace{0.2cm}
Number of operators and the time required to accomplish a certain task are inversely proportional - when more operators are involved, the task will take less time.\\
\vspace{0.2cm}
$(2 \enspace \mathrm{Operators} * 4.5 \enspace \mathrm{days})=(3 \enspace \mathrm{Operators} * X \enspace \mathrm{days})$
\vspace{0.2cm}
Solving for X:\\
\vspace{0.2cm}
$\implies \enspace X=\dfrac{2*4.5}{3}=\boxed{3 \enspace days}$
\vspace{0.2cm}
\end{enumerate}



\end{document}