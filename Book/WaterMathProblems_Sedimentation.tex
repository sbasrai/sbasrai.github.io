\documentclass{article}
%\usepackage[english]{babel}%
\usepackage{graphicx}
\usepackage{tabulary}
\usepackage{tabularx}
\usepackage[table,xcdraw]{xcolor}
\usepackage{pdflscape}
%\usepackage{gensymb}
\usepackage{lastpage}
\usepackage{multirow}
\usepackage{xcolor}
\usepackage{cancel}
\usepackage{amsmath}
\usepackage[table]{xcolor}
\usepackage{fixltx2e}
\usepackage[T1]{fontenc}
\usepackage[utf8]{inputenc}
\usepackage{ifthen}
\usepackage{fancyhdr}
\usepackage[utf8]{inputenc}
\usepackage{tikz}
\usepackage[document]{ragged2e}
\usepackage[margin=1in,top=1.2in,headheight=57pt,headsep=0.1in]
{geometry}
\usepackage{ifthen}
\usepackage{fancyhdr}
\everymath{\displaystyle}
\usepackage[document]{ragged2e}
\usepackage{fancyhdr}
\usepackage{mathabx}
\usepackage{textcomp,mathcomp}
\usepackage[shortlabels]{enumitem}
\everymath{\displaystyle}
\linespread{2}%controls the spacing between lines. Bigger fractions means crowded lines%
\linespread{1.3}%controls the spacing between lines. Bigger fractions means crowded lines%
\pagestyle{fancy}
\setlength{\headheight}{56.2pt}
\usepackage{soul}
\usepackage{siunitx}

%\usepackage{textcomp}
\usetikzlibrary{shapes.multipart, shapes.geometric, arrows}
\usetikzlibrary{calc, decorations.markings}
\usetikzlibrary{arrows.meta}
\usetikzlibrary{shapes,snakes}
\usetikzlibrary{quotes,angles, positioning}
%\chead{\ifthenelse{\value{page}=1}{\includegraphics[scale=0.3]{BassettCTCLogo}}}
%\rhead{\ifthenelse{\value{page}=1}{Final Exam}{}}
%\lhead{\ifthenelse{\value{page}=1}{Water Treatment - Oct-Dec 2022}{\textbf Final Exam}}
%\rfoot{\ifthenelse{\value{page}=1}{}{}}
%
%\cfoot{}
%\lfoot{Page \thepage\ of \pageref{LastPage}}
%\renewcommand{\headrulewidth}{2pt}
%\renewcommand{\footrulewidth}{1pt}
\begin{document}

\begin{enumerate}

	  \item The basin in a water plant measure 60 feet long by 40 feet wide by 8 feet deep. The flow through this plant is 4.1 cuft/sec. What is the detention time?\\
a) 1 hour 18 minutes\\
b) 144 minutes\\
*c) 449 minutes\\
d) 2 hours 24 minutes\\


  \item Calculate the weir overflow rate if the flow is $2.3 \mathrm{cuft} / \mathrm{sec}$ and the radius of the weir is 29 feet.\\
*a) $5.67 \mathrm{gpm} / \mathrm{ft}$ of weir\\
b) $8.50 \mathrm{gpm} / \mathrm{ft}$ of weir\\
c) $11.34 \mathrm{gpm} / \mathrm{ft}$ of weir\\
d) $17.01 \mathrm{gpm} / \mathrm{ft}$ of weir\\

\item A circular clarifier receives a flow of 5 MGD.  If the clarifier is 90 ft. in diameter and is 12 ft. deep, what is: a) the hydraulic/surface loading rate, b) clarifier detention time in hours, and c) weir overflow rate?\\
		\vspace{0.2cm}
a) Hydraulic/surface loading rate:\\
$Clarifier \enspace hydraulic \enspace loading \enspace 	\Big(\dfrac{gpd}{ft^2}\Big) ==\dfrac{\dfrac{5\cancel{MG}}{{day}}*\dfrac{10^6gal}{\cancel{MG}}}{0.785*90^2 ft^2}=\boxed{786gpd/ft^2}$\\
		\vspace{0.5cm}
b) Clarifier detention time:\\
$Clarifier \enspace detention \enspace time \enspace (hr) = 	\dfrac{ Clarifier \enspace volume (cu.ft \enspace or \enspace gal)}{Influent \enspace flow \enspace (cu.ft \enspace or \enspace gal)/hr)}$\\
		\vspace{0.2cm}
$Clarifier \enspace detention \enspace time \enspace (hr) = 	\dfrac{(0.785*90^2*12)\cancel{ft^3}}{\dfrac{5\cancel{MG}}{\cancel{day}}*\dfrac{10^6\cancel{gal}}{\cancel{MG}}*\dfrac{\cancel{ft^3}}{7.48\cancel{gal}}*\dfrac{\cancel{day}}{24hrs}}=\boxed{2.7hrs}$\\
		\vspace{0.5cm}
c) Overflow rate:\\
		\vspace{0.2cm} 
$Weir \enspace overflow \enspace rate \Big(\dfrac{gpd}{ft}\Big) =\dfrac{\dfrac{5\cancel{MG}}{{day}}*\dfrac{10^6gal}{\cancel{MG}}}{3.14*90 ft}=\boxed{17,692 \mathrm{gpd}/\mathrm{ft}}$\\


\item Calculate the weir loading for a sedimentation tank that has an outlet weir 480 ft long and a flow of 5 MGD.\\

\vspace{0.5cm}

a. 9,220 gpd/ ft \\
b. 9,600 gpd/ ft \\
c. 9,920 gpd/ ft \\
*d. 10,420 gpd/ft \\

\vspace{0.5cm}

Solution:\\
$Weir \enspace overflow \enspace rate \Big(\dfrac{gpd}{ft}\Big) =\dfrac{\dfrac{5\cancel{MG}}{{day}}*\dfrac{10^6gal}{\cancel{MG}}}{480 ft}=\boxed{10,417gpd/ft}$\\ 

\item A rectangular sedimentation tank is 85 feet long, 35 feet wide, and 14 feet deep including 3 feet of freeboard. Flow to this tank is 2.3 MGD. Calculate the surface loading to this tank in gpd per $ft^2$. \\

a. 318 gpd/$ft^2$ \\
*b. 773 gpd/$ft^2$ \\
c. 845 gpd/$ft^2$ \\
d. 1932 gpd/$ft^2$ \\

\end{enumerate}
\end{document}