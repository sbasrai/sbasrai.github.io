\documentclass{article}
%\usepackage[english]{babel}%
\usepackage{graphicx}
\usepackage{tabulary}
\usepackage{tabularx}
\usepackage[table,xcdraw]{xcolor}
\usepackage{pdflscape}
%\usepackage{gensymb}
\usepackage{lastpage}
\usepackage{multirow}
\usepackage{xcolor}
\usepackage{cancel}
\usepackage{amsmath}
\usepackage[table]{xcolor}
\usepackage{fixltx2e}
\usepackage[T1]{fontenc}
\usepackage[utf8]{inputenc}
\usepackage{ifthen}
\usepackage{fancyhdr}
\usepackage[utf8]{inputenc}
\usepackage{tikz}
\usepackage[document]{ragged2e}
\usepackage[margin=1in,top=1.2in,headheight=57pt,headsep=0.1in]
{geometry}
\usepackage{ifthen}
\usepackage{fancyhdr}
\everymath{\displaystyle}
\usepackage[document]{ragged2e}
\usepackage{fancyhdr}
\usepackage{mathabx}
\usepackage{textcomp,mathcomp}
\usepackage[shortlabels]{enumitem}
\everymath{\displaystyle}
\linespread{2}%controls the spacing between lines. Bigger fractions means crowded lines%
\linespread{1.3}%controls the spacing between lines. Bigger fractions means crowded lines%
\pagestyle{fancy}
\setlength{\headheight}{56.2pt}
\usepackage{soul}
\usepackage{siunitx}

%\usepackage{textcomp}
\usetikzlibrary{shapes.multipart, shapes.geometric, arrows}
\usetikzlibrary{calc, decorations.markings}
\usetikzlibrary{arrows.meta}
\usetikzlibrary{shapes,snakes}
\usetikzlibrary{quotes,angles, positioning}
%\chead{\ifthenelse{\value{page}=1}{\includegraphics[scale=0.3]{BassettCTCLogo}}}
%\rhead{\ifthenelse{\value{page}=1}{Final Exam}{}}
%\lhead{\ifthenelse{\value{page}=1}{Water Treatment - Oct-Dec 2022}{\textbf Final Exam}}
%\rfoot{\ifthenelse{\value{page}=1}{}{}}
%
%\cfoot{}
%\lfoot{Page \thepage\ of \pageref{LastPage}}
%\renewcommand{\headrulewidth}{2pt}
%\renewcommand{\footrulewidth}{1pt}
\begin{document}

\begin{enumerate}
\item Before pumping, the static water level in a well is 15 feet. During pumping, the water lewel drops to 45 feet. What is the drawdown? $45-15=30$\\
a. 15\\
b. 30\\
c. 45\\
d. 60\\
e. 90\\

\item A well produces 365 gpm with a drawdown of 22.5 ft.  What is	the specific yield in gallons per minute per foot?\\
a.	16.2\\
b.	22.5\\
c.	32.4\\
d.	86.5\\

\item A well is located in an aquifer with a water table elevation 20 feet below the ground surface. After operating for three hours, the water level in the well stabilizes at 50 feet below the ground surface. The pumping water level is:\\
a.	20 feet\\
b.  30 feet\\
c.	50 feet\\
d.	70 feet\\
e.	100 feet\\

\item Calculate drawdown, in feet, using the following data:\\
The water level in a well is 20 feet below the ground surface when the pump is not in operation, and the water level is 35 feet below the ground surface when the pump is in operation.\\
a.	15 feet\\
b.	20 feet\\
c.	35 feet\\
d.	55 feet\\

\item A well is producing 0.00125 MGD. Its static water level was 35 ft and its current pumping water level is 115 ft. What is the specific capacity of this well? \\
a. 0.016 gpm/ft\\
b. 4.7 gpm/ft\\
c. 10.9 gpm/ft\\
d. 15.6 gpm/ft\\
e. 100 gpm/ft\\

\item Determine the drawdown from a well measuring a static water level of 120 feet and a pumping water level of 205 feet?\\
a. 105 ft\\
b. 320 feet\\
c. 85 feet\\
d. 310 feet\\


\item Find the specific yield in gpm/ft if a well produces 105 gpm and the drawdown for the well is 16.3 ft.\\
a. 6.00 gpm / ft\\
*b. 6.44 gpm / ft\\
c. 7.20 gpm / ft\\
d. 7.28 gpm / ft\\

\item Find the drawdown of a well that has a specific yield of 28.4 , if the well
yields 325 gpm.\\
a. 9.8 ft\\
*b. 11.4 ft\\
c. 12.9 ft\\
d. 14.1 ft\\

\item Calculate the well yield in gpm, given a drawdown of 14.1 ft and a specific
yield of 31 gpm / ft.\\
a. 2.2 gpm\\
b. 7.3 gpm\\
c. 45.1 gpm\\
*d. 440 gpm\\
\end{enumerate}
\begin{enumerate}
\item A well yields 2,840 gallons in exactly 20 minutes. What is the well yield in gpm?\\
a. 140 gpm\\
b. 142 gpm\\
c. 145 gpm\\
d. 150 gpm

\item Before pumping, the water level in a well is 15 ft. down. During pumping, the water level is 45 ft. down. The drawdown is:\\
a. 30 ft.\\
b. 60 ft.\\
c. 45 ft.\\
d. 15 ft.\\

\item A well produces 365 gpm with a drawdown of 22.5 ft.  What is	the specific yield in gallons per minute per foot?\\
a.	16.2\\
b.	22.5\\
c.	32.4\\
d.	86.5\\

\item A well is located in an aquifer with a water table elevation 20 feet below the ground surface. After operating for three hours, the water level in the well stabilizes at 50 feet below the ground surface. The pumping water level is:\\
a.	20 feet\\
b.  30 feet\\
c.	50 feet\\
d.	70 feet\\
e.	100 feet\\

\item Calculate drawdown, in feet, using the following data:\\
The water level in a well is 20 feet below the ground surface when the pump is not in operation, and the water level is 35 feet below the ground surface when the pump is in operation.\\
a.	15 feet\\
b.	20 feet\\
c.	35 feet\\
d.	55 feet\\

\item Calculate the well yield in gpm, given a drawdown of 14.1 ft and a specific yield of 31
gpm/ft.\\
a. 2.2 gpm\\
b. 7.3 gpm\\
c. 45.1 gpm \\
d. 440 gpm\\

\item A well is producing 0.00125 MGD. Its static water level was 35 ft and its current pumping water level is 115 ft. What is the specific capacity of this well? \\
a. 0.016 gpm/ft\\
b. 4.7 gpm/ft\\
c. 10.9 gpm/ft\\
d. 15.6 gpm/ft\\
e. 100 gpm/ft\\

\item Determine the drawdown from a well measuring a static water level of 120 feet and a pumping water level of 205 feet?\\
a. 105 ft\\
b. 320 feet\\
c. 85 feet\\
d. 310 feet\\

\item Before pumping, the static water level in a well is 15 feet. During pumping, the water
level drops to 45 feet. What is the drawdown?\\
a. 15 ft\\
b. 30 ft\\
c. 45 ft\\
d. 60 ft\\
e. 90 ft\\

\item The specific capacity for a well is 10 gpm-ft. If the well produces 550 gallons per minute, what is the drawdown?

\item The distance between the ground surface to the water level in a well when the pump is not operating is 98 ft.  Distance from the ground surface to the water in the well when the pump is operating is 116 feet. Calculate the drawdown in the well under these conditions.

\item What is the specific capacity in gpm/feet a well that is pumping 495 gpm and has a
static level of 55 feet and a pumping level of 110 feet?

\item During a test for well yield, a well produced 760 gallons per minute. The drawdown for the test is 22 feet What is the specific capacity in gallons per min-ft/?

\item The pumped water level of a well is 400 feet below the surface. The well produces  250 gpm.  If the aquifer level 50 feet below the surface, what is the specific capacity for the well

\end{enumerate}

Solution:\\

\begin{enumerate}
\item $\mathrm{Yield \enspace (gpm)}=\dfrac{2,840 gallons}{20 min}=\boxed{142 \enspace \mathrm{gpm}}$

\item 
$\mathrm{Drawdown} = \mathrm{initial-pumping} = 15 - 45 = \boxed{30 ft}$
\item 
$\mathrm{Specific \enspace yield} = \dfrac{Yield}{Drawdown}=\dfrac{365 gpm}{22.5 ft}=\boxed{16.2 gpm/ft}$
\item 
50 feet
\item 
$\mathrm{Drawdown} = \mathrm{initial-pumping} = 20 - 35 = \boxed{15 ft}$
\item 
$\mathrm{Specific \enspace  yield} = \dfrac{Yield}{Drawdown}$\\
\vspace{0.2cm}
$\implies 31 gpm/ft=\dfrac{Yield}{14.1 \enspace ft} \implies Yield (gpm)=31*14.5=\boxed{437 ft \enspace \approx \enspace 440 ft}$
\item 
$\mathrm{Specific \enspace yield} = \dfrac{Yield}{Drawdown}$ where $\mathrm{Drawdown} = \mathrm{initial-pumping}$\\
\vspace{0.2cm}
$\implies \mathrm{Specific \enspace Yield} = \dfrac{Yield}{\mathrm{initial-pumping}}=\dfrac{0.00125 MGD * \dfrac{1,000,000 gpm}{MGD}}{35-115}=\boxed{15.6 gpm/ft}$
\item 
$\mathrm{Drawdown} = \mathrm{initial-pumping} = 120 - 205 = \boxed{85 ft}$
\item 
$\mathrm{Drawdown} = \mathrm{initial-pumping} = 15 - 45 = \boxed{30 ft}$
\vspace{0.2cm}
\item 
$\mathrm{Specific \enspace  Yield} = \dfrac{Yield}{Drawdown}$\\
\vspace{0.2cm}
$\implies 10 gpm/ft=\dfrac{550 gpm}{Drawdown (ft)} \implies Drawdown \enspace ft=\dfrac{550}{10}=\boxed{55 ft}$
\vspace{0.2cm}
\item 
$\mathrm{Drawdown} = \mathrm{initial-pumping} = 98 - 116 = \boxed{18 ft}$
\item 
$\mathrm{Specific \enspace Yield} = \dfrac{Yield}{Drawdown}$ where $\mathrm{Drawdown} = \mathrm{initial-pumping}$\\
\vspace{0.2cm}
$\implies \mathrm{Specific \enspace Yield} = \dfrac{Yield}{\mathrm{initial-pumping}}=\dfrac{495 gpm}{55-110}=\boxed{9 gpm/ft}$
\item 
$\mathrm{Specific \enspace yield} = \dfrac{Yield}{Drawdown}=\dfrac{760 gpm}{22 ft}=\boxed{34.5 gpm/ft}$
\vspace{0.2cm}
\item 
Drawdown = 400-50 = 350 ft\\
\vspace{0.2cm}
$\mathrm{Specific \enspace yield} = \dfrac{Yield}{Drawdown}=\dfrac{250 gpm}{350 ft}=\boxed{0.7 gpm/ft}$

\end{enumerate}


\end{document}