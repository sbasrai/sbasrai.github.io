\chapterimage{QuizCover} % Chapter heading image

\chapter{Week 1 Assessment}
% \textbf{Multiple Choice}
\section*{Week 1 Assessment}
\begin{enumerate}[1.]
\item Groundwaters generally have consistent water quality that include\\
a. having a higher total dissolved solids content than surface water\\
b. having a lower mineral content than surface waters\\
c. having lower $\mathrm{pH}$ values than surface waters\\
d. having a higher amount of bacteria than surface waters\\
\item When underground water is under pressure greater than atmospheric pressure and could rise above the its confining space and above the ground level is referred to as a(n)\\
a. aquifer\\
b. anaerobic condition\\
c. artesian effect\\
d. drawdown\\
e. pressure gradient\\
\item The gradual flow or movement of water into and through the pores of the soil is called\\
a. percolation\\
b. run-off\\
c. precipitation\\
d. impermeable flow\\
e. evapotranspiration\\
\item Water that has been used to carry solids away from a home or office into a treatment facility is referred to as\\
a. wastewater or sewage\\
b. potable\\
c. seawater intrusion injection water\\
d. riparian water\\
\item The water right to put it to beneficial use of the surface water adjacent to your land is called water.\\
a. wastewater\\
b. riparian\\
c. filter ripening\\
d. infiltration\\
e. run-off\\
\item The difference between static level and pumping level in a well is called:\\
a. drawdown.\\
b. cone of depression\\
c. zone of saturation\\
d. radius of influence\\
\item Which one of the following best defines the term aquifer?\\
 a. A low lying area where water pools\\
 b. Water-bearing stratum of rock, sand, or gravel\\
 c. Impervious stratum near the ground surface\\
 d. Treated water leaving the water system\\
 \item The height to which water will rise in wells located in an artesian aquifer is called the\\
 a. Pumping water level\\
 b. Water table\\
 c. Piezometric surface\\
 d. Drawdown\\
 e. Radius of influence\\
 \item What percentage of all the earth's water is readily available as a potential drinking water supply in the form of lakes, rivers, and near-surface groundwater?\\
 a. 97\\
 b. 50\\
 c. 2\\
 d. 1\\
 e. 0.34\\
 \item To prevent the entry of surface contamination into a well is the purpose of\\
 a. The well casing\\
 b. The water table\\
 c. The louvers or slots\\
 d. Well development\\
 e. The annular grout seal\\
 \item An aquifer that is located underneath an aquiclude is called\\
 a. An unconfined aquifer\\
 b. A confined aquifer\\
 c. A water table\\
 d. Unreachable groundwater\\
 e. An Artesian spring\\
 \item The process by which water changes from the gas to the liquid phase is termed\\
 a. Condensation ·\\
 b. Evaporation\\
 c. Percolation\\
 d. Precipitation\\
 e. Runoff\\
 \item The free surface of the water in an unconfined aquifer is known as the\\
 a. Pumping water level\\
 b. Artesian spring\\
 c. Water table\\
 d. Drawdown\\
 e. Percolation\\
 \item The transfer of liquid water from plants and animals on the surface of the earth into water vapor in the atmosphere is called\\
 a. Transpiration\\
 b. Evaporation\\
 c. Condensation\\
 d. Runoff\\
 e. Percolation\\
 \item The elevation of water in the casing of an operating well is called the\\
 a. Piezometric surface\\
 b. Water table\\
 c. Pumping water level\\
 d. Drawdown\\
 e. Radius of influence\\
 \item An aquifer under pressure is often termed\\
 a. Unconfined\\
 b. Pacific\\
 c. Artesian\\
 d. Alluvial\\
 e. Elevated\\
% \item An aquifer is usually composed of\\
% a. Sand and gravel\\
% b. Clays and silts\\
% c. Bedrock\\
% d. Large voids in the soil, resembling underground lakes\\
% e. None of the above\\
% \item Which of the following best defines the term specific capacity?\\
% a. Amount of water a given volume of saturated rock or sediment will yield to gravity\\
% b. Amount of water a given volume of saturated rock or sediment will yield to pumping\\
% c. Rate at which water would flow in an aquifer if the aquifer were an open conduit\\
% d. Amount of water a well will produce for each foot of drawdown\\
% \item The most common type of well used for public water supply systems is a\\
% a. Jetted well\\
% b. Driven well\\
% c. Drilled well\\
% d. Bored well 

% \item Which one of the following best defines the term aquifer?\\
% a. A low lying area where water pools\\
% b. Water-bearing stratum of rock, sand, or gravel\\
% c. Impervious stratum near the ground surface\\
% d. Treated water leaving the water system\\

% \item Which of the following best defines the term static water level?\\
% a. Water level in a well after a pump has operated for a period of time\\
% b. Water level in a well when the well is not in operation\\
% c. Water level in a well measured from the ground surface to the drawdown water level\\
% d. Waterlevel in a well measured from the natural water level to the drawdown water level\\

% \item The residual drawdown of a well is defined as\\
% a. Water level in a well after a pump has operated over a period of time\\
% b. Measured distance from the ground to the pumping level\\
% c. Water level below the normal level that persists after a well pump has been off for a period of time\\
% d. Measured distance between the water level and the top of the screen\\

% \item A well is located in an aquifer with a water table elevation 20 feet below the ground surface. After operating for three hours, the water level in the well stabilizes at 50 feet below the ground surface. The pumping water level is:\\
% a. 20 feet\\
% b. 30 feet\\
% c. 50 feet\\
% d. 70 feet\\
% e. 100 feet\\

% \item What percentage of all the earth's water is readily available as a potential drinking water supply in the form of lakes, rivers, and near-surface groundwater?\\
% a. $97 \%$\\
% b. $50 \%$\\
% c. $2 \%$\\
% d. $1 \%$\\
% e. $0.34 \%$\\

% \item To prevent the entry of surface contamination into a well is the purpose of\\
% a. The well casing\\
% b. The water table\\
% c. The louvers or slots\\
% d. Well development\\
% e. The annular grout seal\\

% \item The process by which water changes from the gas to the liquid phase is termed\\
% a. Condensation\\
% b. Evaporation\\
% c. Percolation\\
% d. Precipitation\\
% e. Runoff\\

% \item The free surface of the water in an unconfined aquifer is known as the\\
% a. Pumping water level\\
% b. Artesian spring\\
% c. Water table\\
% d. Drawdown\\
% e. Percolation\\
% \item The transfer of liquid water from plants and animals on the surface of the earth into water vapor in the atmosphere is called\\
% a. Transpiration\\
% b. Evaporation\\
% c. Condensation\\
% d. Runoff\\
% e. Percolation\\
% \item The term for the combined processes which transfer liquid water on the earth's surface into water in the gas phase in the atmosphere is\\
% a. Percolation\\
% b. Evapotranspiration\\
% c. Sublimation\\
% d. Overdraft\\
% e. Precipitation\\
% \item A primary advantage of using surface water as a water source includes:\\
% a. Usually higher in turbidity\\
% b. Generally softer than groundwater\\
% c. Easily contaminated with microorganisms\\
% d. Can be variable in quality\\
% \item Which source of water has the greatest natural protection from bacterial contamination?\\
% a. Shallow well\\
% b. Deep well\\
% c. Surface water\\
% d. Spring\\
% \item A water-bearing formation in the soil is referred to as\\
% a. An aquitard or aquiclude\\
% b. An aquifer\\
% c. An aqueduct\\
% d. The drawdown\\
% e. The static water level 34. An operating well will drain the water from a volume of soil around the well during pumping. This volume is referred to as the\\
% a. Pumping water level\\
% b. Radius of influence\\
% c. Drawdown\\
% d. Cone of depression\\
% e. Recharge zone\\
% \item One acre is 43,560 square feet. If this acre is covered with one foot of water, it contains\\
% a. 1 acre-foot\\
% b. 43,560 cubic feet\\
% c. 325,829 gallons\\
% d. All of the above\\
% e. None of the above\\
% \item The safe yield of an aquifer is\\
% a. Determined by the Department of Health Services\\
% b. Variable, depending on rainfall\\
% c. The average amount of water that can be withdrawn each year without causing a long-term drop in the water table\\
% d. The difference between the static water level and the pumping water level\\
% e. All of the above\\

% \item The movement of water from the surface of the earth into the soil is called\\
% a. Condensation\\
% b. Evaporation\\
% c. Evapotranspiration\\
% d. Runoff\\
% e. None of the above\\

% \item The freezing point of water is\\
% a. $0^{\circ} \mathrm{F}$\\
% b. $32^{\circ} \mathrm{C}$\\
% c. $32^{\circ} \mathrm{F}$\\
% d. $0^{\circ} \mathrm{C}$\\
% e. $100^{\circ} \mathrm{F}$\\

% \item The movement of water from the atmosphere to the surface of the earth is called\\
% a. Condensation\\
% b. Evaporation\\
% c. Evapotranspiration\\
% d. Runoff\\
% e. Precipitation\\
% \item The movement of water on the surface of the earth is called\\
% a. Percolation\\
% b. Evaporation\\
% c. Evapotranspiration\\
% d. Runoff\\
% e. Infiltration\\

% \item A formation in the soil that resists water movement (such as a clay layer) is called\\
% a. An aquitard or aquiclude\\
% b. An aquifer\\
% c. An aqueduct\\
% d. The drawdown\\
% \item Another term for the percolation that transports water from the surface into an aquifer is\\
% a. Artesian springs\\
% b. Recharge\\
% c. Extraction\\
% d. Overdraft\\
% e. Runoff\\
% \item Water that is safe to drink is called water.\\
% a. Potable\\
% b. Palatable\\
% c. Good\\
% d. Clear\\
% \item Groundwaters generally have consistent water quality that include\\
% a. having a higher total dissolved solids content than surface water*\\
% b. having a lower mineral content than surface waters\\
% c. having lower $\mathrm{pH}$ values than surface waters\\
% d. having a higher amount of bacteria than surface waters\\
% \item What is the middle layer of a stratified lake called?\\
% a. Thermocline\\
% b. Benthic Zone\\
% c. Epilimnion\\
% d. Hypolimnion\\
% \item What is the conversion of liquid water to gaseous water known as?\\
% a. Advection\\
% b. Condensation\\
% c. Precipitation\\
% d. Evaporation\\
% \item Water weighs\\
% a. $7.481 \mathrm{bs} / \mathrm{gal}$\\
% b. $8.34 \mathrm{lbs} / \mathrm{gal}$\\
% c. $62.4 \mathrm{lbs} / \mathrm{ft}^{3}$\\
% d. Both B. and C. 48. What is the static level of an unconfined aquifer also known as?\\
% a. Drawdown\\
% b. Water Table\\
% c. Pumping Water Level\\
% d. Aquitard\\
% \item A water bearing geologic formation that accumulates water due to its porousness\\
% a. Aquifer\\
% b. Lake\\
% c. Aquiclude\\
% d. Well\\
% \item What kind of stream flows continuously throughout the year?\\
% a. Ephemeral\\
% b. Perennial\\
% c. Intermittent\\
% d. Stratified\\
% \item The surface to atmosphere movement of water is known as\\
% a. Precipitation\\
% b. Percolation\\
% c. Stratification\\
% d. Evapotranspiration\\
% \item An aquifer that is underneath a layer of low permeability is known as\\
% a. Confined aquifer\\
% b. Water Table aquifer\\
% c. Unconfined aquifer\\
% d. Unreachable groundwater\\
% \item What is the middle layer of a stratified lake known as?\\
% a. Hypolimnion\\
% b. Benthic Zone\\
% c. Thermocline\\
% d. Epilimnion\\
% \item The amount of water that can be pulled from a aquifer without depleting\\
% a. Drawdown\\
% b. Safe yield\\
% c. Overdraft\\
% d. Subsidence\\
% \item The primary origin of coliforms in water supplies is\\
% a. Natural algae growth\\
% b. Industrial solvents\\
% c. Fecal contamination by warm-blooded animals\\
% d. Acid raid 56. A primary source of volatile organic chemical (VOC) contamination of water supplies is\\
% a. Agricultural pesticides\\
% b.Industrial solvents\\
% c. Acid rain\\
% d. Agricultural fertilizers\\
% \item The term "surface runoff" refers to\\
% a. Rainwater that soaks into the ground\\
% b. Rain that returns to the atmosphere from the earth's surface\\
% c. Surface water that overflows the banks of rivers\\
% d. Water that flows into rivers after a rainfall\\
% \item A disease that can be transferred by water is\\
% a. Gonorrhea\\
% b. Malaria\\
% c. Mumps\\
% d. Typhoid\\
% \item To prevent the entry of surface contamination into a well is the purpose of\\
% a. The well casing\\
% b. The water table\\
% c. The louvers or slots\\
% d. Well development\\
% e. The annular grout seal\\

% \item Potable water may be defined as\\
% a. Water high in organic content\\
% b. Any water that occasionally may be polluted from another source\\
% c. Any water that, according to recognized standards, is safe for consumption\\
% a. Water that indicates a septic condition\\
% e. Water that has been transported from outside the service area\\
% \item An operating well will drain the water from a volume of soil around the well during pumping. This volume is referred to as the\\
% a. Pumping water level\\
% b. Radius of influence\\
% c. Drawdown\\
% d. Cone of depression\\
% e. Recharge zone\\
% \item A well screen must be installed in\\
% a. deep wells\\
% b. consolidated materials\\
% c. shallow wells\\
% d. unconsolidated materials\\
% \item A well is acidified in order to\\
% a disinfect\\
% b. increase yield\\
% c. remove objectionable gases\\
% d. remove disinfection by-products\\
% \item The amount of water that a well will produce for each foot of drawdown is called:\\
% a. specific head\\
% b. static yield\\
% c. yield/feet\\
% d. specific capacity\\
% \item Surging a well to loosen scale deposits on the screen refers to:\\
% a. turning the pumps on and off as fast as possible to cause a water hammer\\
% b. pumping the water in and out of a well\\
% c. sending shock waves through the aquifer to cause a surge of water\\
% d. using a water jet to surge around the well casing.\\
% \item A well is acidized in order to\\
% a. Disinfect the water\\
% b. Increase yield\\
% c. Remove objectionable gasses\\
% d. Remove disinfection by-products\\
% \item To prevent the entry of surface contamination into a well is the purpose of\\
% a, The well casing\\
% b. The water table\\
% c. The louvers or slots\\
% d. Well development\\
% e. The annular grout seal\\
% \item The variation in water demand during the course of a day is termed\\
% a. Seasonal variation\\
% b. Fire flow requirements\\
% c. Emergency storage variation\\
% d. The straight line equalization method\\
% e. Diurnal variation\\
% \item The maximum momentary load placed on a water supply system is known as\\
% a. Average daily flow\\
% b. Average daily demand\\
% c. Rated capacity\\
% d. A System float\\
% d. Peak demand\\
% \item The term aquifer refers to:\\
% a. A special type of aqueduct.\\
% b. A natural source of water.\\
% c. A potable water.\\
% d. Water bearing strata.\\
% \item The use of a well supply as a source normally results in: a. Water that is high in nitrates\\
% b. Water of consistent quality\\
% c. Water very high in mineral content\\
% d. Water that is considered "soft".\\
% \item Maximum Safe Yield of a water source is defined as:\\
% a) Where the state health department has approved the source of use.\\
% b) The quantity of water that can be taken from a source of supply over a period of years without depleting the source permanently - beyond it's ability to replenish in wet years.\\
% c) Water that is free of bacteria.\\
% d) Quantity of water that may be treated in the plant.\\
% \item Movement of water through the ground is called:\\
% a) Hydraulic subsidence\\
% b) Runoff\\
% c. Percolation\\
% d. Infiltration\\
% \item A primary source of volatile organic chemical (VOC) contamination of water supplies is\\
% a. Agricultural pesticides\\
% b. Industrial solvents\\
% c. Acld rain\\
% d. Agricultural fertilizers\\
% \item Surging a well to loosen scale deposits on the screen refers to:\\
% a. turning the pumps on and off as fast as possible to cause a water hammer\\
% b. pumping the water in and out of a well\\
% c. sending shock waves through the aquifer to cause a surge of water\\
% d. using a water jet to surge around the well casing.\\
% \item A sanitary well seal is used to:\\
% a. seal the clear well\\
% b. seal the top of the well casing\\
% c. seal the water tower\\
% d. seal a break in the distribution system\\
% \item The amount of water that a well will produce for each foot of drawdown is called:\\
% a. specific head\\
% b. static yield\\
% c. yield/feet\\
% d. specific capacity\\
% \item After replacing a repaired pump back into a well, the operator should first:\\
% a put the seal on tight to avoid contamination\\
% b. add chlorine to disinfect the well and surrounding aquifer c. start the pump to make sure that it will pump water\\
% d. open the valve to let the pressure off the line\\
% \item The amount of water in a water-bearing formation depends on the\\
% a. Depth of the well\\
% b. Size of the pump\\
% c. Thickness and permeability of the formation\\
% d. Type of well casing 
% \item The water table is defined as the\\
% a. Pumping water level in a well\\
% b. Upper surface of the groundwater\\
% c. Water level in a reservoir\\
% d. Bottom of the aquifer\\

\item Convert 22$\dfrac{1}{4}$ into a fraction
\item Express 10ft, 6in as a fraction
\item Express 10ft, 6in as decimal
\item Add $\dfrac{3}{4}+\dfrac{1}{7}$
\item Multiply $\dfrac{4}{9}*\dfrac{3}{16}$

\item Write the equivalent of 10,000,000 as a power of ten
\item Find the product of $3.4564*10^2$
\item Find the product of $534.567*10^{-2}$
\vspace{0.2cm}
\item Find the value of $\dfrac{165.93}{10^{-2}}$
\vspace{0.2cm}
\item Find the value of $0.023*10^4$
\item Find the average of the following set of numbers:\\
$
\begin{aligned}
&0.2 \\
&0.2 \\
&0.1 \\
&0.3 \\
&0.2 \\
&0.4 \\
&0.6 \\
&0.1 \\
&0.3
\end{aligned}
$

\item The chemical used for each day during a week is given below. Based on these data, what was the average lb/day chemical used during the week?\\

\begin{tabular}{|l|l|}
\hline
Monday & 92 lb/day\\
\hline
Tuesday & 93 lb/day \\
\hline
Wednesday & 98 lb/day\\
\hline
Thursday & 93 lb/day \\
\hline
Friday & 89 lb/day\\
\hline
Saturday & 93 lb/day \\
\hline
Sunday & 97 lb/day\\
\hline
\end{tabular}
\vspace{0.3cm}

\item The average chemical use at a plant is 77 lb/day. If the chemical inventory is 2800 lbs, how many days supply is this?
\end{enumerate}


