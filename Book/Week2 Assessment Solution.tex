\chapterimage{QuizCover} % Chapter heading image

\chapter{Week 2 Assessment}
% \textbf{Multiple Choice}
\section{Week 2 Assessment}
\begin{enumerate}[1.]

\item Hard water contains an abundance of\\
a. sodium\\
b. iron\\
c. lead\\
d. calcium carbonate\\
\item A specific class of bacteria that only inhibit the intestines of warm-blooded animals is referred to as?\\
a. Eutrophic\\
b. Grazing\\
c. Salmonella\\
d. Fecal coliform\\
e. pathogenic\\
\item Water with a $\mathrm{pH}$ of 8.0 is considered to be\\
a. acidic\\
b. basic or alkaline\\
c. neutral\\
d. undrinkable\\
\item Over which water quality indicator do operators have the greatest control?\\
a. alkalinity\\
b. $\mathrm{pH}$\\
c. temperature\\
d. turbidity\\
\item Which piece of laboratory equipment is used to titrate a chemical reagent?\\
a. graduated cylinder\\
b. burette\\
c. pipet\\
d. Buchner funnel\\
\item Which $\mathrm{pH}$ range is generally accepted as most palatable (drinkable)?\\
a. $ 6.5$ to 8.5\\
b. 4.5 to 6.5\\
c. 8.5 to 9.5\\
d. 9.5 and above\\
e. all of the above\\
\item Which of the following conditions is favorable for the rapid growth of algal?\\
a. moderate to high dissolved oxygen and nutrients\\
b. high $\mathrm{pH}$ and water hardness\\
c. low temperatures and low dissolved oxygen\\
d. high alkalinity and water hardness\\
\item Which of the following is the name given for a turbidity meter that has reflected or scattered light off suspended particles as a measurement?\\
a. $\mathrm{HACH}$ colorimeter\\
b. spectrophotometer\\
c. Wheaton bridge\\
d. Nephelometer\\
\item Water hardness is the measure of the concentrations of and dissolved in the water sample.\\
a. iron, manganese\\
b. nitrates, nitrites\\
c. sulfates, bicarbonates\\
d. calcium \& magnesium carbonates\\
e. ferric chlorides and polymers\\
\item The electrical potential required to transfer electrons from one compound or element to another is commonly referred to as\\
a. oxidation-reduction potential (ORP)\\
b. voltage potential $(\mathrm{OHM} / \mathrm{P})$\\
c. resistance-impedance potential\\
d. microMho differential\\
\item Water has physical, chemical, and biological characteristics. Which of the following is a physical characteristic?\\
a. Coliform\\
b. Turbidity\\ 
c. Hardness\\
d. All the above\\
\item Tastes and odors in surface water are most often caused by:\\
a. clays\\
b. hardness\\
c. algae\\
d. coliform bacteria\\
\item Which of the following elements cause hardness in water?\\
a. sodium and potassium\\
b. calcium and magnesium\\
c. iron and manganese\\
d. turbidity and suspended solids\\
\item When measuring for free chlorine residual, which method is the quickest and simplest?\\
a. DPD color comparater\\
b. Orthotolidine method\\
c. Amperometric titration\\
d. 1, 2 nitrotoluene di-amine method\\
% \item Which water quality parameter requires a grab sample because it cannot be collected as a composite sample?\\
% a. $\mathrm{pH}$\\
% b. Iron\\
% c. Nitrate\\
% d. Zinc\\
% \item If a water sample is not analyzed immediately for chlorine residual, it is acceptable if it is analyzed within\\
% a. 10 minutes.\\
% b. 15 minutes.\\
% c. 20 minutes.\\
% d. 30 minutes.\\
% \item The volume of a sample for coliform compliance is\\
% a. $100 \mathrm{~mL}$.\\
% b. $200 \mathrm{~mL}$.\\
% c. $300 \mathrm{~mL}$.\\
% d. 0 ; there is no volume compliance for coliforms.\\
% \item Which of the following is an indicator organism?\\
% a. Giardia\\
% b. Cryptosporidium\\
% c. Hepatitis\\
% d. E. Coli\\
% \item What is the primary origin of coliform bacteria in water supplies?\\
% a. Natural algae growth\\
% b. Industrial solvents\\
% c. Animal or human feces\\
% d. Acid rain\\
% \item What ls the term for water samples collected at regular intervals and combined in equal volume with each other?\\
% a. Time grab samples\\
% b. Time flow samples\\
% c. Proportional time composite samples\\
% \item What is the basis for the number of samples that must be collected for utilities monitoring for lead and copper that are in compliance or have installed corrosion control'?\\
% a. Size of distribution system\\
% b. Population\\
% c. Amount of water produced\\
% d. Number of raw water sources\\
% \item Where should bacteriological samples be collected in the distribution system?\\
% a. Uniformly distributed throughout the system based on area\\
% b. At locations that are representative of conditions within the system\\
% c. Always from extreme locations in the system but occasionally at other locations\\
% d. Uniformly throughout the system based on population density\\
% \item The quantity of oxygen. that can remain dissolved in water is related to\\
% *a. Temperature\\
% b. $\mathrm{pH}$\\
% c. Turbidity\\
% d. Alkalinity\\

% \item In coliform analysis using the presence-absence test, a sample should be incubated for\\
% a. 24 hours at $25^{\circ} \mathrm{C}$\\
% b. 36 hours at $35^{\circ} \mathrm{C}$\\
% c. 24 and 36 hours at $25^{\circ} 0$\\
% *d. 24 and 48 hours at $35^{\circ} \mathrm{C}$\\
% \item A major source of error when obtaining water quality information is improper:\\
% *a. Sampling\\
% b. Preservation\\
% c. Tests of samples\\
% d. Reporting of data\\

% \item What is commonly used as an indicator of potential contamination in drinking water samples?\\
% a. Viruses\\
% *b. Coliform bacteria\\
% c. Intestinal parasites\\
% d. Pathogenic organisms\\
% \item The type of organisms that can cause disease are said to be microorganisms.\\
% a. Bad\\
% *b. Pathogenic\\
% c. Undesirable\\
% d. Sick\\
% \item Four types of aesthetic contaminants in water include the following:\\
% a. Odor, turbidity, color, hydrogen sulfide gas\\
% b. Pathogens, microorganisms, arsenic, disinfection by-products\\
% *c. Odor, color, turbidity, hardness\\
% d. Color, pathogens, metals, organics\\
% \item What is the purpose of adding fluoride to drinking water?\\
% a. Increase tooth decay\\
% *b. Reduce tooth decay\\
% c. Make teeth white\\
% d. Government conspiracy\\
% \item The test used to determine the effectiveness of disinfection is called the:\\
% a. Coliform bacteria test\\
% b. Color test\\
% c. Turbidity test\\
% d. Particle test\\
% \item Turbidity is measured as:\\
% a. $\mathrm{mg} / \mathrm{L}$\\
% b. $\mathrm{mL}$\\
% c. $\mathrm{gpm}$\\
% d. NTU\\
% \item Giardia and cryptosporidium are a type of:\\
% a. Mineral\\
% b. Organism\\
% c. Color\\
% d. Bird\\
% \item Chronic contaminants are those that can cause sickness after:\\
% a. Prolonged exposure\\
% b. Low levels or low exposure\\
% \item A positive total coliform test indicates that:\\
% a. Disease-causing organisms may be present in the water supply\\
% b. The water is safe to consume\\
% c. The water supply has high iron levels\\
% d. There is nothing to be concerned about\\
% \item What is the purpose of the bacteriological site sampling plan?\\
% a. To have a map showing where BacT samples are drawn\\
% b. In case of a positive Bac $\mathrm{T}$ sample, the operator will know where to take the four repeat samples\\
% c. The state will know where you are taking your repeat samples\\
% d. All of the above\\
% \item To ensure that the water supplied by a public water system meets state requirements, the water system operator must regularly collect samples and:\\
% a. Have water analyzed at an approved water testing laboratory\\
% b. Determine a sampling schedule based on state requirements\\
% c. Send all analyses results to the state\\
% d. All of the above\\
% \item Samples taken for routine bacteriological testing should be preserved by:\\
% a. Freezing\\
% b. Boiling\\
% c. DPD preservative\\
% d. Refrigeration\\
% \item How many coliform samples are required per month for a water system serving a population between 25 and 100 ?\\
% a. 1\\
% b. 2\\
% c. 3\\
% d. 4\\
% \item Before taking a bacteriological (BacT) water sample from a faucet, you should:\\
% a. Wash hands thoroughly b. Remove the faucet aerator\\
% c. Flush water until you're sure water is from the main, not the service line\\
% d. All of the above\\
% \item Monthly BacT samples should be taken from:\\
% a. The well pump house\\
% b. The distribution system\\
% c. The treatment plant\\
% d. An outside hose spigot\\
% \item If your BacT sample test is positive, how long do you have to collect four repeat samples and deliver them to the lab?\\
% a. 12 hours\\
% b. 24 hours\\
% c. 48 hours\\
% d. 72 hours\\
% \item \_\_ is a measure of the capacity of water to neutralize acids.\\
% a. Concentration\\
% b. Alkalinity\\
% c. $\mathrm{pH}$\\
% d. Conductivity\\
% \item The DPD method is used to determine the of a water sample.\\
% a. Dissolved oxygen content\\
% b. Conductivity\\
% c. $\mathrm{pH}$\\
% d. Free chlorine residual\\
% \item What color does N,N-diethyl-p-phenylenediamine (DPD) turn in the presence of chlorine?\\
% a. Brown\\
% b. Green\\
% c. Blue\\
% d. Pink\\
% \item The presence-absence ( $\mathrm{P}-\mathrm{A})$ test used for microbiological testing is also commonly referred to as\\
% a. Multiple Tube Fermentation\\
% b. Membrane Filtration\\
% c. Confirmed Test\\
% d. Colilert\\
% \item When testing for coliform bacteria with the multiple tube fermentation (MFT) method what is the best indicator for a positive test?\\
% a. Color change\\
% b. Gas bubble formation\\
% c. Formation of a cyst d. Formation of turbidity\\
% \item Coliform bacteria share many characteristics with pathogenic organisms. Which of the following is not true?\\
% a. They survive longer in water\\
% b. They grow in the intestines\\
% c. There are less coliform than pathogenic organisms\\
% d. They are still present in water without fecal contamination\\
% \item What is the second step in the multiple tube fermentation test?\\
% a. Presumptive test\\
% b. Negative test\\
% c. Completed\\
% d. Confirmed\\
% \item What is the removal and deactivation requirement for Giardia?\\
% a. $2 \log$\\
% b. $3 \log$\\
% c. $4 \log$\\
% d. There is no requirement\\
% \item The multiple barrier approach to water treatment includes removal through which method?\\
% a. Filtration\\
% b. Coagulation\\
% c. Disinfection\\
% d. a and c\\
% \item A pH reading of 7 is considered\\
% a. Slightly acidic\\
% b. Acidic\\
% c. Basic\\
% d. Neutral\\
% \item EDTA titration is used to determine the of a water sample.\\
% a. Hardness\\
% b. Conductivity\\
% c. Alkalinity\\
% d. Free chlorine residual\\
% \item A higher than normal turbidity reading could signify\\
% a. A change in water quality\\
% b. Nothing. Keep operating as normal\\
% c. Microbiological contamination\\
% d. Both $A \& C$\\
% \item What is the ingredient used during the second multiple tube fermentation test?\\
% a. Colilert\\
% b. MMO/MUG\\
% c. Brilliant Green Bile \\
% d. Chlorine\\
% \item When collecting a distribution system sample for bacteriological testing, the person collecting the sample should allow the water to run before filling the sample bottle.\\
% a. A minimum of five minutes.\\
% b. $1 \mathrm{hr}$.\\
% c. $30 \mathrm{~min}$\\
% d. only a few seconds\\
% \item Black stains on plumbing fixtures might be attributed to\\
% a. calcium.\\
% b. copper.\\
% c. magnesium.\\
% d. manganese.\\
% \item The multiple tube fermentation test consists of three distinct tests. These tests, in the order performed, are the:\\
% a. preliminary, confirmed, and completed tests.\\
% b. preliminary, presumptive and confirmed tests.\\
% c. presumptive, confirmed, and completed tests.\\
% d. prespumtive, preliminary, and completed tests.\\
% \item What should the sample volume be when testing for total coliform bacteria?\\
% a. $100 \mathrm{~mL}$\\
% b. $250 \mathrm{~mL}$\\
% c. $500 \mathrm{~mL}$\\
% d. $1,000 \mathrm{~mL}$\\
% \item $\mathrm{pH}$ is a measure of :\\
% a. conductivity\\
% b. water's ability to neutralize acid\\
% c. hydrogen ion activity\\
% d. dissolved solids\\
% \item Sodium Thiosulfate is used to\\
% a. Buffer chlorine solutions\\
% b. Neutralize chlorine residuals\\
% c. Detect chlorine leaks\\
% d. Sterilize sample bottles\\
% \item The presence of total coliforms in drinking water indicates\\
% a. The presence of pathogens.\\
% b. The absence of an adequate chlorine residual\\
% c. The existence of an urgent public health problem\\
% d. The potential presence of pathogens\\
% \item A primary health risk associated with microorganisms in drinking water is\\
% a. Cancer\\
% b. Acute gastrointestinal diseases\\
% c. Birth defects\\
% d. Nervous system disorders\\
% \item After 5 years use, a portion of cast iron pipe shows a white scale about $1 / 2$ inch thick lining the inside. This means\\
% a. Red water will soon become a problem\\
% b. The water has been corrosive\\
% c. The water is chemically unstable and is depositing\\
% d. Water should flow easier since the lining is smooth\\
% \item Hardness in water is caused by\\
% a. Dissolved minerals\\
% b. High $\mathrm{pH}$.\\
% c. Low turbidity\\
% d. Alkalinity\\
% \item An unknown substance is found on the bottom of the water within a drinking water reservoir. Which of the following statements is true of this substance?\\
% a. It has a specific gravity less than 1.0\\
% b. It has a specific gravity equal to 1.0\\
% c. It has a specific gravity greater than 1.0\\
% d. It has no specific gravity\\
% e. None of the above\\
% \item The term "Chain of Custody" refers to\\
% a. A large accessory to a come-along\\
% b. An attachment to a pipe-cutter\\
% c. Employee labor laws\\
% d. Procedures and documentation required for water quality sampling\\
% e. Procedures and documentation required for chemical application\\
% \item Water samples to be analyzed for taste and odor must be\\
% a. Analyzed in the field\\
% b. Collected in glass sample containers\\
% c. Dechlorinated with sodium thiosulfate\\
% d. Preserved with dilute hydrochloric acid e. None of the above\\
% \item Bacteriological samples for a distribution system must be collected in accordance with\\
% a. The Surface Water Treatment Rule\\
% b. OSHA requirements\\
% c. An approved sample siting plan\\
% d. FLSA requirements\\
% e. ANSI/NSF Standard 61\\
% \item Trihalomethanes are classified as\\
% a. Metals\\
% b. Inorganic constituents\\
% c. Secondary drinking water standards\\
% d. Radiological contaminants\\
% e. Volatile organic compounds\\
% \item The multiple tube fermentation analysis consists of\\
% a. Positive, negative, and neutral tests\\
% b. Presumptive, confirmed, and completed tests\\
% c. Preliminary, presumptive, and confirmed tests\\
% d. Preliminary, confirmed, and completed tests\\
% e. Presence or absence testing\\
% \item A bacteriological test that measures only the presence or absence of coliforms is\\
% *a. ColiLert (MMO/MUG)\\
% b. Multiple tube fermentation\\
% c. Most probable number (MPN)\\
% d. Membrane filtration\\
% e. Presumptive test\\
% \item After collection, if stored at $4^{\circ} \mathrm{C}$, bacteriological samples must be processed within\\
% a. 1 hour\\
% b. 6 hours\\
% *c. 24 hours\\
% d. 48 hours\\
% e. 72 hours\\
% \item Sample bottles which are furnished by a certified laboratory for collection of bacteriological samples\\
% a. Should be rinsed with the water to be sampled before use b. Should be placed in boiling water for at least 10 minutes before use\\
% c. Should be rinsed with a chlorine solution before use\\
% d. Should be rinsed with distilled water before use\\
% e. Are ready to use\\
% \item The standard indicator of potential fecal contamination of a water supply is\\
% a. Cryptosporidium\\
% b. $\mathrm{pH}$\\
% c. Alkalinity\\
% d. Hardness\\
% e. Coliform Presence - Absence\\
% \item Where should bacteriological samples be collected?\\
% a. At different locations on each sampling cycle, to make sure the entire system is sampled\\
% b. Only from public locations, such as drinking fountains and restrooms\\
% c. Only from locations owned by consumers\\
% d. Only from specially constructed sampling stations\\
% e. From several sampling locations around the entire distribution system, in accordance with a DHS-approved sample siting plan\\
% \item Storage of bacteriological samples during transport to a laboratory is best accomplished using\\
% a. A clean storage box specifically designed to hold sample containers\\
% b. An ice chest packed with ice\\
% c. An insulated storage box with "blue ice".\\
% d. An insulated storage box with "dry ice"\\
% e. No particular sample storage requirements apply, as long as the samples can be delivered to a laboratory prior to the end of the work day\\
% \item Sodium thiosulfate is added in the laboratory to bacteriological sample bottles to:\\
% a. Thoroughly disinfect the sample bottle\\
% b. -Complete the cleaning and sterilization process\\
% c. Neutralize any residual chlorine present in the sample at the time of collection\\
% d. Counteract the effects of sunlight on the water sample\\
% e. Prevent further growth of bacteria in water samples following collection\\

% \item Radiological contaminant concentrations in drinking water are measured in\\
% a. Milligrams per liter\\
% b. Micrograms per liter\\
% c. Nanograms per liter\\
% d. Picograms per liter\\
% e. None of the above\\
% \item Which of the following is NOT a characteristic of coliform organisms?\\
% a. Intestinal origin\\
% b. Will produce carbon dioxide from lactose\\
% c. Heartier in a water environment than pathogenic organisms\\
% d. Far less numerous than pathogenic organisms\\
% e. Able to survive with or without oxygen\\
% \item A water supply is found to have a calcium carbonate concentration of $50 \mathrm{mg} / \mathrm{L}$. This water would be considered\\
% a. soft water\\
% b. hard water\\
% c. potable water\\
% d. non-potable water\\
% \item Cathodic protection refers to protection against\\
% a. contamination\\
% b. corrosion\\
% c. hardness\\
% d. alkalinity\\
% \item An operator uses to test for residual chlorine\\
% a. DPD\\
% b. Cresol red\\
% c. Methyl orange\\
% d. Sulfuric acid\\
% \item The meniscus on calibrated glassware is read at the:\\
% a. Bottom of curvature for mercury but the top for water\\
% b. Extreme point of contact between the liquid and glass, i.e., where gas, liquid, and air all meet at one point\\
% c. Mid-height of the curvature so that beginning and ending readings will results in zero error\\
% d. Top of curvature for mercury but at the bottom for most other liquids including water\\
% \item The type of corrosion caused by the use of dissimilar metal in a water system is\\
% a. Caustic corrosion\\
% b. Galvanic corrosion\\
% c. Oxygen corrosion\\
% d. Tubercular corrosion\\
% \item Which of the following can cause tastes and odors in a water supply?\\
% a. Dissolved zinc\\
% b. Algae\\
% c. High $\mathrm{pH}$\\
% d. Low $\mathrm{pH}$\\
% \item The primary health risk associated with volatile organic chemicals (VOCs) is\\
% a. Cancer\\
% b. Acute respiratory diseases\\
% c. "Blue baby" syndrome d. Reduced IQ. in children\\
% \item Lead in drinking water can result in\\
% a. Impaired mental functioning in children\\
% b. Prostate cancer in men\\
% c. Stomach and intestinal disorders\\
% d. Reduced white blood cell count\\
% Sodium thiosulfate is used to\\
% a. Buffer chlorine solutions\\
% b. Neutralize chlorine residuals\\
% c. Raise pH d. Sterilize sample bottles\\
% \item Cathodic protection means protection against\\
% a. contamination\\
% b. corrosion\\
% c. hardness\\
% d. infiltration\\
% \item A water supply is found to have a calcium carbonate concentration of $50 \mathrm{mg} / \mathrm{l}$. This water would be considered\\
% a. soft water\\
% b. hard water\\
% c. potable water\\
% d. non-potable water\\

\item Flow in an 8-inch pipe is 500 gpm. What is the average velocity in ft/sec? (Assume pipe is flowing full)

\item A pipeline is 18” in diameter and flowing at a velocity of 125 ft. per minute. What is the flow in gallons per minute?

\item The velocity in a pipeline is 2 ft./sec. and the flow is 3,000 gpm. What is the diameter of the pipe in inches?



\item Find the flow in a 4-inch pipe when the velocity is $1.5$ feet per second.

  \item A 42-inch diameter pipe transfers 35 cubic feet of water per second. Find the velocity in $\mathrm{ft} / \mathrm{sec}$. 
  
  \item A plastic float is dropped into a channel and is found to travel 10 feet in $4.2$ seconds. The channel is $2.4$ feet wide and $1.8$ feet deep. Calculate the flow rate of water in cfs.
  
\item $25 \%$ of the chlorine in a 30-gallon vat has been used. How many gallons are remaining in the vat?

\item The annual public works budget is $\$ 147,450$. If $75 \%$ of the budget should be spent by the end of September, how many dollars are to be spent? How many dollars will be remaining?

\item A 75 pound container of calcium hypochlorite has a purity of $67 \%$. What is the total weight of the calcium hypochlorite? 

\item $3 / 4$ is the same as what percentage?

\item A 60-foot diameter tank contains 422,000 gallons of water. Calculate the height of water in the storage tank.

\item What is the volume of water in ft$^3$, of a sedimentation basin that is 22 feet long, and 15 feet wide, and filled to 10 feet?

\item What is the volume in ft$^3$ of an elevated clear well that is 17.5 feet in diameter, and filled to 14 feet?

\item What is the area of the top of a storage tank that is 75 feet in diameter?\\

\item  What is the area of a wall $175 \mathrm{ft}$. in length and $20 \mathrm{ft}$. wide?\\

\item  You are tasked with filling an area with rock near some of your equipment. 1 Bag of rock covers 250 square feet. The area that needs rock cover is 400 feet in length and 30 feet wide. How many bags do you need to purchase?\\



\end{enumerate}


