\chapterimage{QuizCover} % Chapter heading image

\chapter{Week 3 Assessment}

\section*{Week 3 Assessment}
% \textbf{Multiple Choice}
\begin{enumerate}[1.]
\item Primary drinking water standards are set to protect the public from illnesses as a direct result in drinking water that exceeds maximum set levels. Secondary standards were set to alert the public to\\
a. the incidences of local cancer numbers\\
b. dissolved solids in water\\
c. immediate health concerns\\
d. radiological conditions concerning drinking water\\
e. aesthetic issues with drinking water\\
\item A positive fecal coliform test must be reported to the primacy agency within\\
a. 8 hours.\\
b. 12 hours.\\
c. 24 hours.\\
d. 48 hours.\\
\item Which agency sets legal limits on the concentration levels of harmful contaminants in potable water distributed to customers?\\
a. National Primary Drinking Water Regulations\\
b. United States Environmental Protection Agency\\
c. United States Public Health Service\\
d. Occupational Health and Safety Organization\\
\item Which may be substituted for the analysis of residual disinfectant concentration, when total coliforms are also sampled at the same sampling point?\\
a. Heterotrophic plate count (HPC)\\
b. Fecal coliforms\\
c. Giardia lamblia\\
d. Combined chlorine\\
\item What does the acronym MCL stand for?\\
a. Minimum contaminant level\\
b. Micron contaminant level\\
c. Maximum contaminant level\\
d. Milligrams counted last\\
\item How long do sanitary surveys have to be retained for records?\\
a. 3 years\\
b. 5 years\\
c. 7 years\\
d. 10 years\\
\item The most severe water system violation that requires the fastest public notification\\
a. Tier I\\
b. Tier II\\
c. Tier III\\
d. Tier IV
\item The primacy agency may grant a variance or exemption as long as\\
a. The agency is using the Best Available Technology\\
b. There is no threat to public health\\
c. There is never a scenario for a variance or exemption\\
d. Both A. and B.\\
\item A public water system that serves at least 25 people six months out of the year\\
a. Nontransient noncommunity\\
b. Transient noncommunity\\
c. Community public water system\\
d. None of the above\\
\item Regulations based on the aesthetic quality of drinking water\\
a. Primary Standards\\
b. Secondary Standards\\
c. Microbiological Standards\\
d. Radiological Standards\\
\item The lowest reportable limit for a water sample\\
a. $0.5 \mathrm{mg} / 1$\\
b. Zero\\
c. Public health goal\\
d. Detection Level for reporting\\
\item Primary Standards are based on\\
a. Color and Taste\\
b. Aesthetic quality\\
c. Public Health\\
d. Odor\\
\item A disease causing microorganism\\
a. Pathogen\\
b. Colilert\\
c. Pathological\\
d. Turbidity\\
\item According to Surface Water Treatment Rule, what is the combined inactivation and removal for Giardia?\\
a. $1.0 \log s$\\
b. $2.0 \log \mathrm{s}$\\
c. $3.0 \log s$\\
d. 4.0 Logs\\
\item What is the equivalency expressed as a percentage for the SWTR inactivation and removal of viruses?\\
a. $99.9 \%$\\
b. $99.99 \%$\\
c. $99.0 \%$\\
d. $99.999 \%$\\
\item A water agency that takes more than 40 coliform samples must fall under what percentile?\\
a. $10 \%$\\
b. $7 \%$\\
c. $5 \%$\\
d. No positive samples allowable\\
\item The National Primary Drinking Water Regulations apply to drinking water contaminants that may have adverse effects on\\
a. Water color\\
b. Water taste\\
c. Water odor\\
d. Human health\\
\item Which of the following is considered an acute risk to health?\\
a. Two Tier 2 violations\\
b. One Tier 2 violation\\
c. Two Tier 1 violations\\
d. One Tier 1 violation\\
\item Records on turbidity analyses should be kept for a minimum of\\
a. 5 years\\
b. 7 years\\
c. 10 years\\
d. 25 years\\
\item Records on bacteriological analyses should be kept for a minimum of\\
a. 5 years\\
b. 7 years\\
c. 10 years\\
d. 25 years\\
% \item Difference between primary and secondary standard substances:\\
% a. Primary standards refer to substances that are carcinogenic, secondary standards do not.\\
% b. Primary standards refer to substances that are thought to pose a threat to human health, secondary standards do not.\\
% c. Primary standards refer to substances that, if not.put in check, will eventually kill humans, secondary standards do not.\\
% d. Secondary qualities are aesthetic qualities and will only make some people sick, while primary standards refer to substances that will make everyone sick and may possibly cause death.\\
% \item The SDWA defines a public water system that supplies piped water for human consumption as one that has\\
% a. 10 service connection or serves 20 or more people for 60 or more days per year b. 15 service connections or serves 20 or more people for 90 or more days per year\\
% c. 10 service connections or serves 25 or more people for 30 or more days per year\\
% d. 15 service connections or serves 25 or more people for 60 or more days per year\\
% \item According to the USEPA regulations, the owner or operator of a public water system that fails to comply with applicable\\
% monitoring requirements shall give notice to the public within\\
% a. 1 week of the violation in a letter hand-delivered to customers\\
% b. 45 days of the violation by posting a notice at the town hall\\
% c. 3 months of the violation in a daily newspaper in the area served by the system d. 1 year of the violation by including the notice with the water-bill .\\
% \item What US agency establishes drinking water standards?\\
% a. AWWA\\
% b. USEPA\\
% c. NIOSH\\
% d. NSF\\
% \item If a water supply exceeds the MCL, whose responsibility is it to notify the consumer?\\
% a. the testing lab\\
% b. the supplier\\
% c. the DOH\\
% d. the USEPA\\
% \item According to the Lead and Copper Rule. the action for the 90th percentile lead level is:\\
% a. $0.005 \mathrm{mg} / 1$\\
% b. $0.015 \mathrm{mg} / \mathrm{l}$\\
% c. $0.030 \mathrm{mg} / 1$\\
% d. $0.050 \mathrm{mg} / \mathrm{l}$\\
% \item The term "maximum contaminant level goal (MCLG)" means the:\\
% a. Maximum allowable level of a given contaminant in drinking water\\
% b. Level of a contaminant .in drinking water below which there are no known or suspected adverse health effects with a margin of safety\\
% c. Level of a contaminant in drinking water that will trigger a Tier 1 violation\\
% d. Minimum detectable level of a given contaminant\\
% \item The maximum contaminant level goal (MCLG) of known or probable carcinogens is:\\
% a. Set by the state\\
% b. The same number as the maximum contaminant level (MCL)\\
% c. Zero\\
% d. The minimum detectable level of a given contaminant\\
% \item The difference between Tier 1 and Tier 2 violations is:\\
% a. Tier 1 violations-potentially impose-direct and adverse health effects;-Tier 2 violations do not pose a a direct threat to public health. $b$. Tier 1 violations require public notification; Tier 2 violations do not require public notification c. Tier 1 violations are acute; Tier 2 violations are not acute\\
% d. Tier 1 violations have legal consequences; Tier 2 violations do not\\
% \item The Safe Drinking Water Act requires to develop a comprehensive coliform monitoring plan\\
% a. Large public water systems (serving $>50,000$ people)\\
% b. Large and medium public water systems (serving $>3,300$ people)\\
% c. Small and medium public water systems (serving $>25$ and $<3,300$ people)\\
% d. All public water systems\\
% \item The most important factor to consider in locating a well site from the health point of view is\\
% a. Anticipated yield\\
% b. Availability of electric power\\
% c. Distance from other wells\\
% d. Vulnerability\\
% \item Trihalomethanes are classified as:\\
% a. Metals\\
% b. Inorganic constituents\\
% c. Secondary drinking water standards\\
% d. Radiological contaminants\\
% e. Volatile organic compounds\\
% \item The term "primacy" means the\\
% a. Authority by the states to supersede USEPA drinking water regulations\\
% b. Authority by the USEPA to supersede state drinking water regulations\\
% c. Requirements for states to maintain drinking water regulations more stringent than USEPA regulations\\
% d. Primary authority for implementation and enforcement of drinking water regulations\\
% \item The Safe Drinking Water Act requires to develop a comprehensive coliform monitoring plan\\
% a. Large public water systems (serving $>50,000$ people)\\
% b. Large and medium public water systems (serving $>3,300$ people)\\
% c. Small and medium public water systems (serving $>25$ and $<3,300$ people)\\
% d. All public water systems\\
% \item Contaminant monitoring requirements can depend on\\
% a. The results of a vulnerability assessment\\
% b. The size of the water system c. Previous maximum contaminant level (MCL) violations\\
% d. All of the above\\
% \item For public water systems using surface water and groundwater under the influence of surface water, turbidity must be measure at least\\
% a. Every 4 hours\\
% b. Daily\\
% c. Weekly\\
% d. Monthly\\
% \item The difference between Tier 1 and Tier 2 violations is\\
% a. Tier1-violations potentially impose-direct and adverse health effects; Tier 2 violations do not pose a direct threat to public health\\
% b. Tier 1 violations require public notification; Tier 2 violations do not require public notification\\
% c. Tier 1 violations are acute; Tier 2 violations are not acute\\
% d. Tier 1 violations have legal consequences; Tier 2 violations do not\\
% \item The maximum contaminant level goal (MCLG) of known or probable carcinogens is\\
% a. Set by the state\\
% b. The same number as the maximum contaminant level (MCL)\\
% c. Zero\\
% d. The minimum detectable level of a given contaminant\\
% \item All of the following diseases may be transmitted by contaminated water, except for:\\
% a. Cryptosporidiosis\\
% b. Giardiasis\\
% c. Cholera\\
% d. Typhoid\\
% e. Tuberculosis\\
% \item The maximum disinfectant residual allowed in a distribution system is\\
% a. $\quad 0.2 \mathrm{mg} / \mathrm{L}$\\
% b. $\quad 2.0 \mathrm{mg} / \mathrm{L}$\\
% c. $\quad 2.0 \mu \mathrm{g} / \mathrm{L}$\\
% d. $4.0 \mathrm{mg} / \mathrm{L}$\\
% e. There is no maximum disinfectant residual standard\\
% \item What steps must be taken when a single routine sample tests positive for total coliform? a. Immediately notify the Department of Health Services\\
% b. Immediately notify customers\\
% c. Re-test a new sample taken from the original sample point\\
% d. Re-test a new sample taken from the original sample point, plus at points immediately upstream and downstream\\
% e. Flush the system around the original sample point to re-establish disinfectant levels\\
% \item For drinking water distribution systems with over 40 routine coliform samples per month, the maximum amount of coliform-positive samples permitted is\\
% a. 2\\
% b. $2 \%$\\
% c. 5\\
% d. $5 \%$\\
% e. variable, depending on the size of the system\\
% \item Final determination of vulnerability is made by\\
% a. Private contractor/consultants\\
% b. The primacy agency\\
% c. The water supplier\\
% d. All of the above\\
% \item The regulation that establishes standards for microbiological quality in drinking water is a. The Disinfection By-Product Rule\\
% b. Secondary Drinking Water Standards\\
% c. The Total Coliform Rule\\
% d. The Lead and Copper Rule\\
% e. Maximum Contaminant Level\\
% \item Primary and secondary drinking water standards are normally established with a\\
% a. Maximum contaminant level\\
% b. Minimum contaminant level\\
% c. Public health goal\\
% d. Maximum contaminant level goal\\
% e. Minimum contaminant level goal\\
% \item The presence of coliform bacteria in a distribution system\\
% a. Is positive proof that pathogenic organisms are present\\
% b. Indicates that chlorine demand has increased dramatically c. Indicates that pathogenic organisms may be present also\\
% d. Requires the use of brilliant green bile as a secondary disinfectant\\
% e. Has no particular significance\\
% \item The regulation that establishes standards for microbiological quality in drinking water is\\
% a. The Disinfection By-Product Rule\\
% b. Secondary Drinking Water Standards\\
% c. The Total Coliform Rule\\
% d. The Lead and Copper Rule\\
% e. Maximum Contaminant Level\\
% \item For public water systems using surface water and groundwater under the influence of surface water, turbidity must be measured at least\\
% a. Every 4 hours\\
% b. Daily\\
% c. Weekly.\\
% d. Monthly\\
% \item Contaminant monitoring requirements can depend on\\
% a. The results of a vulnerability assessment\\
% b. The size of the water system\\
% c. Previous maximum contaminant level (MCL) violations\\
% d. All of the above\\
% \item According to the Lead and Copper Rule. the action for the 90th percentile lead level is:\\
% a. $0.005 \mathrm{mg} / \mathrm{l}$\\
% b. $0.015 \mathrm{mg} / \mathrm{l}$\\
% c. $0.030 \mathrm{mg} / \mathrm{l}$\\
% d. $0.050 \mathrm{mg} / \mathrm{l}$\\
% \item The difference between Tier 1 and Tier 2 violations is\\
% a. Tier 1-violations potentially impose-direct and adverse health effects; Tier 2 violations do not pose a direct threat to public health\\
% b. Tier 1 violations require public notification; Tier 2 violations do not require public notification\\
% c. Tier 1 violations are acute; Tier 2 violations are not acute\\
% d. Tier 1 violations have legal consequences; Tier 2 violations do not\\
% \item For public water systems using surface water and groundwater under the influence of surface water, turbidity must be measure at least\\
% a. Every 4 hours\\
% b. Daily\\
% c. Weekly.\\
% d. Monthly\\
% \item Contaminant monitoring requirements can depend on\\
% a. The results of a vuilnerability assessment\\
% b. The size of the water system\\
% c. Previous maximum contaminant level (MCL) violations\\
% d. All of the above\\
% \item The Safe Drinking Water Act requires to develop a comprehensive coliform monitoring plan a. Large public water systems (serving $>50,000$ people)\\
% b. Large and medium public water systems (serving $>3,300$ people)\\
% c. Small and medium public water systems (serving $>25$ and $<3,300$ people)\\
% d. All public water systems 




\item It takes 6 gallons of chlorine solution to obtain a proper residual when the flow is 45,000 gpd. How many gallons will it take when the flow is 62,000 gpd?

\item A motor is rated at 41 amps average draw per leg at $30 \mathrm{Hp}$. What is the actual $\mathrm{Hp}$ when the draw is 36 amps? C. 

\item If it takes 2 operators $4.5$ days to clean an aeration basin, how long will it take three operators to do the same job?

% \item It takes 3 hours to clean 400 feet of collection system using a sewer ball. How long will it take to clean 250 feet?

% \item It takes 14 cups of $\mathrm{HTH}$ to make a $12 \%$ solution, and each cup holds 300 grams. How many cups will it take to make a $5 \%$ solution?

% \item A bike travelling at 5 miles/hr completes a journey in 40 minutes. How long would the same journey take if the speed was increased to 8 miles/hr?

\item Convert 1000 $ft^3$ to cu. yards\\

\item Convert 10 gallons/min to $ft^3$/hr\\

% \item Convert 100,000 $ft^3$ to acre-ft.\\

% \item Find the flow in gpm when the total flow for the day is 65,000 gpd.

% \item Find the flow in gpm when the flow is $1.3 \mathrm{cfs}$.

\item Find the flow in gpm when the flow is $0.25 \mathrm{cfs}$.

\item The flow rate through a filter is 4.25 MGD. What is this flow rate expressed as gpm?\\

\item After calibrating a chemical feed pump, you've determined that the maximum feed rate is $178 \mathrm{~mL} /$ minute. If this pump ran continuously, how many gallons will it pump in a full day?

\item A plant produces 2,000 cubic foot of water per hour. How many gallons of water is produced in an 8-hour shift?

\item Change 70 °F to °C
% \item Change 140 °F to °C
% \item Change 20 °C to °F
% \item Change 85 °C to °F
\item Change 4 °C to °F

\item A well yields 2,840 gallons in exactly 20 minutes. What is the well yield in gpm?\\

\item Before pumping, the water level in a well is 15 ft. down. During pumping, the water level is 45 ft. down. The drawdown is:\\

\item A well is located in an aquifer with a water table elevation 20 feet below the ground surface. After operating for three hours, the water level in the well stabilizes at 50 feet below the ground surface. Calculate tge pumping water level.\\

\item Calculate drawdown, in feet, using the following data:\\
The water level in a well is 20 feet below the ground surface when the pump is not in operation, and the water level is 35 feet below the ground surface when the pump is in operation.\\


\item Calculate the well yield in gpm, given a drawdown of 14.1 ft and a specific yield of 31
gpm/ft.\\


% \item A well is producing 0.00125 MGD. Its static water level was 35 ft and its current pumping water level is 115 ft. What is the specific capacity of this well? \\


% \item The specific capacity for a well is 10 gpm-ft. If the well produces 550 gallons per minute, what is the drawdown?

% \item The distance between the ground surface to the water level in a well when the pump is not operating is 98 ft.  Distance from the ground surface to the water in the well when the pump is operating is 116 feet. Calculate the drawdown in the well under these conditions.

\end{enumerate}


