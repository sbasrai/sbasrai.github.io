\chapterimage{QuizCover} % Chapter heading image

\chapter{Week 4 Assessment}
% \textbf{Multiple Choice}
\section{Week 4 Assessment}
\begin{enumerate}[1.]
\item What is the purpose of coagulation and flocculation?\\
a. control corrosion\\
b. to kill disease causing organisms\\
c. to remove leaves, sticks, and fish debris\\
d. to remove particulate impurities and suspended matter\\
\item How are filter production (capacity) rates measured?\\
a. Mgd/sq.ft.\\
b. Gpm/sq.ft.\\
c. Gpm\\
d. Mgd\\
\item Why should a filter be drained if it is going to be out-of-service for a prolonged period?\\
a. to allow the media to dry out\\
b. to save water\\
c. to prevent the filter from floating on groundwater levels\\
d. to avoid algal growth\\
\item Which of the following are commonly used coagulation chemicals?\\
a. hypochlorites and free chlorine\\
b. sodium and potassium chlorides\\
c. alum and polymers\\
d. bleach and HTH\\
\item How can an operator tell if a filter is NOT completely cleaned after backwashing?\\
a. the initial headloss is on the high side\\
b. the backwash rate was too slow\\
c. mudballs are NOT present\\
d. backwashing pumping rate is too low\\
\item Flocculation is defined as\\
a. the gathering of fine particles after coagulation by gentle mixing\\
b. clumps of bacteria\\
c. the capacity of water to neutralize acids\\
d. a high molecular weight of compounds that have negative charges\\
\item A multi-barrier water filtration plant that contains a flash mix, a coagulation/flocculation zone, sedimentation, filtration and a clear well is considered to be a\\
a. community special treatment plant\\
b. direct filtration plant\\
c. reverse osmosis plant\\
d. conventional filtration plant\\
e. traditional plant\\
\item The filtration unit process usually\\
a. is located at the beginning of a filtration plant\\
b. follows the coagulation/flocculation/sedimentation processes\\
c. is located after the clear well area\\
d. is located on the plant effluent line after the clearwell\\
\item Filters are generally backwashed when the loss-of-head indicator registers a certain set value, such as 6-ft, or upon a certain time, say 48-hours, or upon a rise in\\
a. alkalinity\\
b. a jar-test result\\
c. turbidity\\
d. temperature\\
\item What is a method of reducing hardness?\\
a. Softening\\
b. Hardening\\
c. Lightning\\
d. Flashing\\
\item The solid that adsorbs a contaminant is called the:\\
a. Adsorbent\\
b. Adsorbate\\
c. Sorbet\\
d. Rock\\
\item The adsorption process is used to remove:\\
a. Organics or inorganics\\
b. Bugs or salts\\
c. Organisms or dirt\\
d. Color or particles\\
\item Describe two primary methods used to control taste and odor?\\
a. Oxidation and adsorption\\
b. Filtration and sedimentation\\
c. Mixing and coagulation\\
d. Sedimentation and clarification\\
\item What is the recommended loading rate for copper sulfate for algae control at an alkalinity greater than $50 \mathrm{mg} / \mathrm{L}$ ?\\
a. 0.9 of copper sulfate per acre of surface area\\
b. 1.9 of copper sulfate per acre of surface area\\
c. 2-4 lb of copper sulfate per acre of surface area\\
d. 5.4 of copper sulfate per acre of surface area\\
\item The basic goal for water treatment is to\\
a. Protect public health\\
b. Make it clear\\
c. Make it taste good\\
d. Get stuff out\\
\item Greensand can be operated in either \rule{1.5cm}{0.5pt} regeneration or \rule{1.5cm}{0.5pt} regeneration modes.\\
a. Continuous or intermittent\\
b. Fast or slow\\
c. Hot or cold\\
d. Constant or unusual\\
\item The two most common types of chlorine disinfection by-products include:\\
a. TTHM and HAA5\\
b. TTHA of HMM5\\
c. Turbidity and color\\
d. Chloride and fluoride\\
\item Chlorine gas is times heavier than breathing air\\
a. 2.5\\
b. 20\\
c. 60\\
d. 460\\
\item A commonly used method to test for chlorine residual in water is called the method.\\
a. HTH\\
b. THM\\
c. VOC\\
d. DPD\\
\item When chlorine gas is added to water the pH goes down due to\\
a. chlorine gas producing caustic substances\\
b. two base materials that form\\
c. two acids that form\\
d. caustic soda being formed in the water\\
\item Disinfection by-products are a product of:\\
a. Filtration\\
b. Disinfection\\
c. Sedimentation\\
d. Adsorption\\
\item Chloramine is most effective as a disinfectant.\\
a. Primary\\
b. Secondary\\
c. Third\\
d. First\\
\item Name two methods commonly used to disinfect drinking water other than chlorination.\\
a. Ozone and ultraviolet light\\
b. Soap and agitation\\
c. Filtration and adsorption\\
d. Salt and vinegar\\
\item In order to determine the effectiveness of disinfection, it is desirable to maintain a disinfectant residual of at least $\mathrm{mg} / \mathrm{L}$ entering the distribution system.\\
a. 0.10\\
b. 0.5\\
c. 0.3\\
d. 0.2\\
\item Secondary disinfectants are used to provide a in the distribution system.\\
a. Color\\
b. Chemical\\
c. Smell\\
d. Residual\\
\item Primary disinfectants are used to microorganisms.\\
a. Hurt\\
b. Inactivate\\
c. Burn up\\
d. Evaporate\\
\item The quantity of chlorine remaining after primary disinfection is called a residual.\\
a. Chlorine\\
b. Permaganate\\
c. Hot\\
d. Cold\\
\item The two most common types of chlorine disinfection by-products include:\\
a. TTHM and HAA5\\
b. TTHA of HMM5\\
c. Turbidity and color\\
d. Chloride and fluoride\\
\item In order to determine the effectiveness of disinfection, it is desirable to maintain a disinfectant residual of at least $\mathrm{mg} / \mathrm{L}$ entering the distribution system.\\
a. 0.10\\
b. 0.5\\
c. 0.3\\
d. 0.2\\
\item A \rule{1.5cm}{0.5pt} residual of chlorine is required throughout the distribution system.\\
a. Large\\
b. High\\
c. Trace\\
d. Hot\\
\item The test used to determine the effectiveness of disinfection is called the:\\
a. Coliform bacteria test\\
b. Color test\\
c. Turbidity test\\
d. Particle test\\
\item Name two methods commonly used to disinfect drinking water other than chlorination.\\
a. Ozone and ultraviolet light\\
b. Soap and agitation\\
c. Filtration and adsorption\\
d. Salt and vinegar\\
\item Name the two types of hypochlorites used to disinfect water.\\
a. Chloride and monochloride\\
b. Sodium and calcium\\
c. Ozone and hydroxide\\
d. Arsenic and manganese\\
\item Free chlorine can only be obtained after \rule{1.5cm}{0.5pt} chlorination has been achieved.\\
a. Breakpoint\\
b. Fastpoint\\
c. Softpoint\\
d. Onpoint\\
\item The meaning of the " C" and the " T " in the term CT stands for:\\
a. Concentration and time\\
b. Color and turbidity\\
c. Calcium and tortellini\\
d. Chlorine and turbidity\\
\item Chloramine is most affective as a disinfectant.\\
a. Primary\\
b. Secondary\\
c. Third\\
d. First\\
\item TTHMs and HAA5s can affect:\\
a. Health\\
b. Aesthetics\\
c. Color\\
d. Odor\\

\item What is the concentration in mg/l of  4.5\% solution of that substance.\\

\item How much does each gallon of zinc orthophosphate weigh (pounds) if it has a specific gravity of 1.46?\\

\item How much does a 55 gallon drum of 25\% caustic soda weigh (pounds) if the specific gravity is 1.28?\\

\item A water treatment plant operates at the rate of 75 gallons per minute. They dose soda ash at 14 mg/L. How many pounds of soda ash will they use in a day?

\item A water treatment plant is producing 1.5 million gallons per day of potable water, and uses 38 pounds of soda ash for pH adjustment. What is the dose of soda ash at that plant?\\

\item A water treatment plant produces 150,000 gallons of water every day. It uses an average of 2 pounds of permanganate for iron and manganese removal. What is the dose of the permanganate? 


\item A water treatment plant uses 8 pounds of chlorine daily and the dose is 17 mg/l. How many gallons are they producing?\\


\vspace{0.2cm}
\item Ferric chloride is being added as a coagulant to the raw water entering a plant. Sampling shows that the concentration of ferric in the raw water is 25 ppm. A quick check of the chemical metering pump shows that it is operating at a flow rate of 4.3 gpm. If the flow through the water plant is 800 gpm, what is the concentration of raw chemical in the dosing tank?\\

\item A water plant is fed by two different wells. The first well produces water at a rate of 600 gpm and contains arsenic at 0.5 mg/L. The second well produces water at a rate of 350 gpm and contains arsenic at 12.5 mg/L. What is the arsenic concentration of the blended water?\\

\end{enumerate}



