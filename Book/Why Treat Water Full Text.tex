\documentclass{article}
%\usepackage[english]{babel}%
\usepackage{graphicx}
\usepackage{tabulary}
\usepackage{tabularx}
\usepackage[table,xcdraw]{xcolor}
\usepackage{pdflscape}
\usepackage{lastpage}
\usepackage{multirow}
\usepackage{cancel}
\usepackage{amsmath}
\usepackage[table]{xcolor}
\usepackage{fixltx2e}
\usepackage[T1]{fontenc}
\usepackage[utf8]{inputenc}
\usepackage{ifthen}
\usepackage{fancyhdr}
\usepackage[document]{ragged2e}
\usepackage[margin=1in,top=1.2in,headheight=57pt,headsep=0.1in]
{geometry}
\usepackage{ifthen}
\usepackage{fancyhdr}
\everymath{\displaystyle}
\usepackage[document]{ragged2e}
\usepackage{fancyhdr}
\everymath{\displaystyle}
\linespread{2}%controls the spacing between lines. Bigger fractions means crowded lines%
%\pagestyle{fancy}
%\usepackage[margin=1 in, top=1in, includefoot]{geometry}
%\everymath{\displaystyle}
\linespread{1.3}%controls the spacing between lines. Bigger fractions means crowded lines%
%\pagestyle{fancy}
\pagestyle{fancy}
\setlength{\headheight}{56.2pt}


\chead{\ifthenelse{\value{page}=1}{\includegraphics[scale=0.3]{BassettCTCLogo}\\ \textbf \textbf Why Treat Water}}
\rhead{\ifthenelse{\value{page}=1}{Shabbir Basrai}{Shabbir Basrai}}
\lhead{\ifthenelse{\value{page}=1}{}{\textbf Why Treat Water}}


\cfoot{}
\lfoot{Page \thepage\ of \pageref{LastPage}}
\rfoot{Module 8}
\renewcommand{\headrulewidth}{2pt}
\renewcommand{\footrulewidth}{1pt}
\begin{document}

Water is the most crucial compound for life on Earth, and having drinkable water is a key worldwide concern for the twenty-first century. All living things require clean, uncontaminated water as a basic requirement. Water covers more than 71 percent of the earth’s surface, but only around 1\% of it is drinkable according to international standards due to various contaminations . Waste water discharge from industries, agricultural pollution, municipal wastewater, environmental and global changes are the main sources of water contamination. Even trace levels of heavy metals, dyes, and microbes are hazardous to human health, aquatic systems, and the environment. According to a United Nations Sustainable Development Group report from 2021, 2.3 billion people now live in water-stressed countries, and 733 million people live in high and critically water-stressed countries.\\

The natural water cycle or hydrologic cycle (the journey water takes during its constant,
inevitable motion) is the means by which water in all three forms—solid,
liquid, and vapor—circulates through the biosphere. Water lost from the Earth’s surface
to the atmosphere, either by evaporation from the surface of lakes, rivers, and
oceans or through the transpiration of plants, forms clouds that condense to deposit
moisture on the land and sea. A drop of water may travel thousands of miles between
the time it evaporates and the time it falls to Earth again as rain, sleet, or snow. The
water that collects on land flows to the ocean in streams and rivers or seeps into the
Earth, joining groundwater. Even groundwater eventually flows toward the ocean for
recycling.\\

Water is the basic resource for guaranteeing the life of all living beings on the planet. Access to water, sanitation and hygiene is a fundamental right and yet billions of people throughout the world are battling daily against enormous difficulties accessing the most basic services.\\

the decreasing availability of quality drinking water is an important problem afflicting all continents. It is estimated that at least 1 out of every 4 people will be affected by  water scarcity by 2050.\\

Water scarcity is a problem affecting over 40\% of the global population and it is predicted that this percentage could increase due in part to global warming and desertification\\

Human beings are the main cause of water pollution More than 80\% of the waste water resulting from human activity is dumped in rivers or the sea without any kind of water treatment, producing contamination.\\

Water treatment is a process involving different types of operations (physical, chemical, physicochemical and biological), the aim of which is to eliminate and/or reduce contamination or non-desirable characteristics of water.\\

The objective of this process is to obtain water with the right features for the use intended for it. This is why the water treatment process varies as a function of the properties of the water being supplied and its final use.\\

Water treatment is increasingly necessary due to drinking water shortages and the growing needs of the global population. Of the planet’s total water reserves, only 2.5\% is freshwater - and of this amount only 0.4\% is water fit for human consumption.\\


Potable water is scarece vital resoirce,  Ut us estimated that only 0.4\% of water on the planet is fit for human consumption.  This is why it is essential to invest in water purification, to ensure that everyone has access to this vital resource. \\

Water systems today are finding themselves with ever increasing responsibilities in the area of proper treatment and protection of the water supply. The impact on small systems can be substantial. It is more important than ever to make sure the people who run these systems have better
understanding of their system’s operation.\\

The basic responsibility of the system is to provide each individual with an adequate supply of safe, potable drinking water. This responsibility extends to all employees; whether on a managerial, supervisory, operational, or clerical level, in some direct or indirect manner. Each employee should be aware of their duties and call to their supervisor’s attentionany any condition that might impair water quality or cause service interruption to any part of the system.\\
These responsibilities can be broken down into three major areas of concern:\\
1. Providing enough water to meet the total demands of the system.\\
2. Providing water that is both safe and palatable to the customers.\\
3. Providing that water to the customer when it is needed.\\

MEETING WATER SYSTEM DEMANDS\\
The amount of water used by the entire system is known as the demand placed on that system. This demand may come from several different sources.
DOMESTIC WATER USAGE\\
Domestic water usage is any water that is used directly by people in their daily activities. These activities include bathing, drinking, cooking, sanitation and other miscellaneous activities like watering lawns, washing the car and laundry. Two major factors that determine the domestic water demands placed on a system are one, the number of individuals the system serves and secondly the amount of water each person on the system will need per day. On a nationwide basis, the average daily consumption of water (total gallons used divided by the total population) is about 130 gallons per person per day. However, this figure will vary depending on the geographic location involved and the population density of that area. Rural areas tend to have a daily consumption rate lower than the national average, while urban areas may have a higher rate.\\
INDUSTRIAL WATER USAGE\\
Industrial usage of water is considered to be water used for production of goods for marketing. The primary sources of industrial demands in rural areas is dairies, food processing, wood products, and textiles. A single industrial user may require as much water as the entire domestic demand on the system.
AGRICULTURAL WATER USAGE\\
Agricultural usage of water is considered to be water used in irrigating crops, watering livestock, and in cooling and cleanup of dairies and farm equipment. Agricultural demand will generally represent a larger portion of the total water usage than the industries in rural areas.\\ 
PUBLIC WATER USAGE\\
Public water usage may be defined as any community service that requires potable water. Services may include fire protection, recreation (swimming pools, golf courses, etc.) and street maintenance. In rural communities where these services are limited, public water usage may not be a consideration.\\
Present conditions and future increases in water production should be considered when designing the system. Operators may not be concerned with the original design of the system but should be aware of the impact of new additions and extensions to the system. This is especially true in areas where present water mains are handling maximum capacities.\\
SOURCES OF SUPPLY\\
Finding enough water to satisfy the demand on the system is the certainly a major concern for the water supplier. The legal and financial considerations that arise when trying to procure water rights or finance the capital funding required to construct treatment facilities can limit the options available in many cases.\\
Systems are faces with essentially two choices when selecting a source of supply. They can drill wells and use ground water or, if a suitable river or lake is present, they may choose to use a surface water supply.
MEETING WATER QUALITY STANDARDS\\
Prior to 1976, water quality was regulated by individual state standards. In many cases these standards were only recommendations rather than enforceable regulations. The U.S. Congress passed the Safe Drinking Water Act (P.L.
93-523) in 1976. The law sets permissible levels of\\
substances found in water which could be hazardous to public health. These regulations include Maximum Contaminant Levels or MCL, for inorganic and organic chemicals, radioactivity, turbidity and microbiological levels. Testing and monitoring requirements, reporting and record keeping schedules, and public notification are enforced by individual state agencies.\\
MEETING WATER CONSUMPTION AND PEAK DEMANDS\\
Peak water consumption periods will vary daily according
to seasons and geographic locations. The major
responsibility of the operator is to make sure the water is
available when it is needed.\\
SEASONAL DEMANDS\\
The amount of water used each day will generally vary according to the time of year. Higher daily demands occur during the hot summer months while the demand will tend to drop off during the winter months. Fluctuations in temperature and rainfall may also dictate a rise or fall in daily water consumption.\\
DAILY PEAK DEMANDS\\
Ninety percent of the daily water usage will occur during a 16-hour period. The peak demand periods occur between the early hours of the morning, (5am to 10am) and the early evening hours, (5pm to 10pm.) Demand will usually increase earlier in rural areas. In urban areas, peak demands will be higher during the week while in rural areas the highest peak demands occur on weekends.\\
COMPONENTS OF A WATER SYSTEM\\
Water systems are made up of a number of devices that are
used to deliver water to the customer. The major
components can be divided into the areas of the source of
water, its treatment, and its distribution.\\
WATER SOURCES AND TREATMENT\\
The source of water can be from groundwater, surface water, or water purchased from another water system. Usually the source of your water will determine the type of treatment necessary. In most circumstances, groundwater requires little treatment. Groundwater quality problems include: minerals, hardness, and dissolved gasses. Surface water typically requires extensive treatment. Surface water quality issues are: turbidity, taste \& odor, and color. Surface water usually requires chemical treatment and filtration.\\
DISTRIBUTION AND TRANSMISSION WATER MAINS\\
Main lines transport water from the source or from the treatment facility to the area to be served. These pipes are usually the largest in the system. They also serve as feeder lines for those users who are located along them.\\

SERVICES
Services are small lines (usually l” or 3/4") that carry water from the main line to the water user. The service connection includes:\\
1. Some means of tapping the main line or feeder line.\\
2. A corporation stop for turning the water off at the main or feeder line.\\
3. Service pipe or tubing.\\
4. Some type of meter setter which includes a meter stop.\\
5. Water meter.\\
6. A fitting for the water customer’s connection.\\
PUMPS\\
Pumps are used to move raw water from the source into the treatment facility or from the well into the system. They are also used to move treated water from the treatment facility into the system. Pumps are used to create pressure for the system and to fill the water storage facilities.\\
STORAGE TANKS\\
Storage tanks hold a reserve of water for those times when the demand for water is greater than can be supplied by the trunk line or by the pumps. They also provide water for fire protection and for those times when the supply might be interrupted.\\
CONTROLS\\
Automatic pump controls sense pressure on the system and turn the pump on when the pressure falls below a desired point or when the water level in the water tower drops below a certain level. When the water level in the tower has been restored or when the system’s pressure has been raised to normal, the controls automatically turn the pump off. Pump controls can also turn the pump off, if alarm conditions occur. These types of alarm conditions include; high discharge pressure, motor overload, high motor or bearing temperatures, or low suction pressure.\\
ISOLATION VALVES\\Isolation valves are used throughout the system to stop the flow of water. They are usually gate valves or butterfly valves. The trunk line would have at least one isolation valve per mile of line in small rural systems and in large municipal systems they may need to be located every 300-600 feet. Each branch line should have an isolation valve at the point of connection to the trunk line. The proper location of these valves is important in order to isolate small sections of line for repair. This minimizes the number of customers that are out of water during times when repairs are being made.\\
CONTROL VALVES\\
Control valves are designed to control flows or pressures in the system. There are a number of different applications for the control valves that may be used in the system. The yare usually diaphragm operated globe valves that are controlled by hydraulic pilot systems. Here are some of the applications for control valves in a water system:
1. Pressure reducing valves - These valves are used to drop the pressure in a distribution zone in order to avoid damage to the system.\\
2. Pressure relief valves - These valves are used to “bleed” water from the system when the pressures reach a point that could result in damage to the system.\\
3. Altitude valves - These valves are used when two or more storage tanks are on the same main line. An altitude valve will isolate the lower tanks and prevent them from overflowing while the other tanks are filling.\\
4. Pressure sustain/flow limiting valves - These valves are used to limit the amount of flow to a certain portion of the system when it drops the pressure in other sections below a certain point.\\
5. Pump control valves - These valves are designed to replace check valves on booster pumps. They are closed when the pump starts and open slowly to minimize surges in the system. The also close before the pump stops.\\
SURGE TANKS\\
Surge tanks act as pressurized shock absorbers in the system. They dissipate the pressure spikes caused by water hammer. Water hammer occurs when waves of high/low pressure occur, usually by opening or closing valves too fast. Water hammer can damage piping.\\
FIRE HYDRANTS\\
Fire hydrants allow fire-fighting equipment to draw a large volume of water from the system quickly. They may also be used as sampling sites, flushing stations, and vent points for filling drained lines.\\
OPERATOR RESPONSIBILITIES\\
An operator is the person who is, in whole or part, responsible for the operation of a water system. At times, he/she may be a manager, laboratory technician, mechanic, meter reader, public relations person, troubleshooter, or inspector. Becoming a competent operator requires the development of many skills. To become a competent operator one must have an interest in his/her work, be dependable, be willing to learn, and be willing to assume responsibility and work without supervision. Each water utility represents a large financial investment in facilities and equipment and improper operation and maintenance can quickly damage both. Although much of


Water treatment is a broad term that covers a wide range of techniques and processes that are applied to water sources. The definition of water treatment is: ‘Any process that makes water more acceptable for a specific end-use’. That means water treatment can cover a huge range of applications, including drinking water, water for industrial use (such as in producing paper, chemicals, and cars), and ultra-pure water, which is used for semiconductor and pharmaceutical purposes.

But what happens when water isn’t treated properly? When it comes to domestic use water (also known as potable water), it can have disastrous results if the water hasn’t been appropriately treated at the plant. There are many dangerous waterborne diseases that pose a serious risk to human and animal health if ingested. In the Industrial Revolution, many major cities like London, Paris, and Frankfurt began the construction of large-scale public sewer works to help effectively remove wastewater and sewage, and treat it in a much safer way than had been attempted previously, following several outbreaks of disease that became linked to contaminated water.Diseases, microbes, bacteria
Waterborne diseases from untreated water

Untreated water is a breeding ground for several dangerous waterborne diseases, and were a real danger to public health for many centuries. The modernisation of sewer systems and properly treated drinking water have mostly eradicated these in the West, but they are still present and should not be underestimated for their severity. Waterborne diseases still pose a serious threat to health in underdeveloped countries where properly treated water may not be readily available.

These diseases are all caused by improper sanitation and unsafe water being ingested by people, and were once very common around the world:

Hepatitis A. Causes nausea, vomiting, jaundice, fever, diarrhoea and can result in acute liver failure.
Typhoid. Causes fever, abdominal pain, rash, and headaches.
Dysentery. Causes severe diarrhoea, fever, and abdominal pain.
Cholera. Causes severe diarrhoea and dehydration, fever, abdominal pain, and vomiting.
Leptospirosis. Symptoms include fever, rash and body pains. Can turn into Weil’s Disease which results in meningitis, kidney failure, and jaundice. Animals are also susceptible.
Legionnaire’s Disease. Legionnaire’s is a form of atypical pneumonia. It causes coughing, shortness of breath, fever, muscle pains, and headaches.
Giardiasis. Causes weakness, stomach cramps, vomiting, and diarrhoea.
Water Treatment Processes

While there are many waterborne diseases, there are a number of processes that are used to treat water, many of which are used together in order to produce the best and most suitable results. Processes include:

Sedimentation – use of gravity to remove solid particles within the water.
Disinfection – use of chemicals such as chlorine to kill bacteria, viruses, and harmful substances in the water. Disinfection also uses filtration to help remove these from the water.
Filtration – filtering of water through sand or membranous filters.
Coagulation – adding certain chemicals to encourage particles to clump together making them easier to remove through sedimentation.
Boiling and distillation – heating of water to boiling temperatures to kill microbes and pathogens, although this method cannot remove particles or chemical toxins. Distillation is more effective due to the evaporation process and can be up to 99\% pure.

\end{document}