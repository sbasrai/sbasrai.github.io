\documentclass[10pt]{article}
\usepackage[utf8]{inputenc}
\usepackage[T1]{fontenc}
\usepackage{amsmath}
\usepackage{amsfonts}
\usepackage{amssymb}
\usepackage{mhchem}
\usepackage{stmaryrd}

\title{Chapter Review }

\author{}
\date{}


\begin{document}
\maketitle
\begin{enumerate}
  \item What is the middle layer of a stratified lake called?\\
a. Thermocline\\
b. Benthic Zone\\
c. Epilimnion\\
d. Hypolimnion

  \item What is the conversion of liquid water to gaseous water known as?\\
a. Advection\\
b. Condensation\\
c. Precipitation\\
d. Evaporation

  \item Water weighs\\
a. $7.48 \mathrm{lbs} / \mathrm{gal}$\\
b. $8.34 \mathrm{lbs} / \mathrm{gal}$\\
c. $62.4 \mathrm{lbs} / \mathrm{ft}^{3}$\\
d. Both B. and C.

  \item What is the static level of an unconfined aquifer also known as?\\
a. Drawdown\\
b. Water Table\\
c. Pumping Water Level\\
d. Aquitard

  \item What is the cause of taste and odor problems in raw surface water?\\
a. Copper sulfate\\
b. Blue-green algae\\
c. Oxygen\\
d. Lake turnover

  \item What chemical reduces blue-green algae growth?\\
a. Chlorine\\
b. Caustic Soda\\
c. Copper Sulfate\\
d. Alum 7. A water bearing geologic formation that accumulates water due to its porousness\\
a. Aquifer\\
b. Lake\\
c. Aquiclude\\
d. Well

  \item What kind of stream flows continuously throughout the year?\\
a. Ephemeral\\
b. Perennial\\
c. Intermittent\\
d. Stratified

  \item The surface to atmosphere movement of water is known as\\
a. Precipitation\\
b. Percolation\\
c. Stratification\\
d. Evapotranspiration

  \item An aquifer that is underneath a layer of low permeability is known as\\
a. Confined aquifer\\
b. Water Table aquifer\\
c. Unconfined aquifer\\
d. Unreachable groundwater

  \item What is the middle layer of a stratified lake known as?\\
a. Hypolimnion\\
b. Benthic Zone\\
c. Thermocline\\
d. Epilimnion

  \item The amount of water that can be pulled from a aquifer without depleting\\
a. Drawdown\\
b. Safe yield\\
c. Overdraft\\
d. Subsidence

\end{enumerate}
\section{Math Questions}
Please show all work. On the State exams you will not get credit if work is not shown.

\begin{enumerate}
  \item What is the area of the top of a storage tank that is 75 feet in diameter?\\
a. $4,000 \mathrm{ft}^{2}$\\
b. $4416 \mathrm{ft}^{2}$\\
c. $1104 \mathrm{ft}^{2}$\\
d. $17,663 \mathrm{ft}^{2}$

  \item What is the area of a wall $175 \mathrm{ft}$. in length and $20 \mathrm{ft}$. wide?\\
a. $3,000 \mathrm{ft}^{2}$\\
b. $2,500 \mathrm{ft}^{2}$\\
c. $3,500 \mathrm{ft}^{2}$\\
d. $4,000 \mathrm{ft}^{2}$

  \item You are tasked with filling an area with rock near some of your equipment. 1 Bag of rock covers 250 square feet. The area that needs rock cover is 400 feet in length and 30 feet wide. How many bags do you need to purchase?\\
a. 40 Bags\\
b. 42 Bags\\
c. 45 Bags\\
d. 48 Bags

\end{enumerate}
\section{Chapter Review}
\begin{enumerate}
  \item The smallest part of an element is known as what?\\
a. Proton\\
b. Neutron\\
c. Atom\\
d. Nucleus

  \item The atomic weight is comprised of?\\
a. Neutrons and electrons\\
b. Protons and neutrons\\
c. Nucleus and neutrons\\
d. Atom and nucleus

  \item Atom with a negative charge\\
a. Proton\\
b. Neutron\\
c. Nucleus\\
d. Electron

  \item What does the symbol $\mathrm{mg} / \mathrm{L}$ stand for?\\
a. Micrograms per liter\\
b. Milligrams per/L\\
c. parts per million\\
d. Both B \& C

  \item What does the acronym $\mathrm{MCL}$ stand for?\\
a. Minimum contaminant level\\
b. Micron contaminant level\\
c. Maximum contaminant Level\\
d. Milligrams counted last

  \item How long do sanitary surveys have to be retained for records?\\
a. 3 years\\
b. 5 years\\
c. 7 years\\
d. 10 years 7. The most severe water system violation that requires the fastest public notification\\
a. Tier I\\
b. Tier II\\
c. Tier III\\
d. Tier IV

  \item The primacy agency may grant a variance or exemption as long as\\
a. The agency is using the Best Available Technology\\
b. There is no threat to public health\\
c. There is never a scenario for a variance or exemption\\
d. Both A. and B.

  \item A public water system that serves at least 25 people six months out of the year\\
a. Nontransient noncommunity\\
b. Transient noncommunity\\
c. Community public water system\\
d. None of the above

  \item Regulations based on the aesthetic quality of drinking water\\
a. Primary Standards\\
b. Secondary Standards\\
c. Microbiological Standards\\
d. Radiological Standards

  \item The lowest reportable limit for a water sample\\
a. $0.5 \mathrm{mg} / \mathrm{L}$\\
b. Zero\\
c. Public health goal\\
d. Detection Level for reporting

  \item Primary Standards are based on\\
a. Color and Taste\\
b. Aesthetic quality\\
c. Public Health\\
d. Odor 13. A circular clearwell is 150 feet in diameter and 40 feet tall. The Clearwell has an overflow at 35 feet. What is the maximum amount of water the clearwell can hold in Million gallons rounded to the nearest hundredth?\\
a. $M G$\\
b. $4.62 \mathrm{MG}$\\
c. $18.50 \mathrm{MG}$\\
d. $7.50 \mathrm{MG}$

  \item A sedimentation basin is 400 feet length, 50 feet in width, and 15 feet deep. What is the volume expressed in cubic feet?\\
a. $100,000 \mathrm{ft}^{3}$\\
b. $200,000 \mathrm{ft}^{3}$\\
c. $300,000 \mathrm{ft}^{3}$\\
d. $400,000 \mathrm{ft}^{3}$

  \item A clearwell holds $314,000 \mathrm{ft}^{3}$ of water. It is $100 \mathrm{ft}$ in diameter. What is the height of the clearwell?\\
a. $25 \mathrm{ft}$\\
b. $30 \mathrm{ft}$\\
c. $35 \mathrm{ft}$\\
d. $40 \mathrm{ft}$

\end{enumerate}
\section{Chapter Review}
\begin{enumerate}
  \item A disease causing microorganism\\
a. Pathogen\\
b. Colilert\\
c. Pathological\\
d. Turbidity

  \item According to Surface Water Treatment Rule, what is the combined inactivation and removal for Giardia?\\
a. 1.0 Logs\\
b. $2.0$ Logs\\
c. 3.0 Logs\\
d. 4.0 Logs

  \item What is the equivalency expressed as a percentage for the SWTR inactivation and removal of viruses?\\
a. $99.9 \%$\\
b. $99.99 \%$\\
c. $99.0 \%$\\
d. $99.999 \%$

  \item A water agency that takes more than 40 coliform samples must fall under what percentile?\\
a. $10 \%$\\
b. $7 \%$\\
c. $5 \%$\\
d. No positive samples allowable

  \item The multiple barrier treatment approach includes\\
a. Sterilization and filtration\\
b. Disinfection and filtration\\
c. Disinfection and sterilization\\
d. Infection and filtration 6. The maximum disinfectant residual allowed for chlorine in a water system is\\
a. $.02 \mathrm{mg} / \mathrm{L}$\\
b. $2.0 \mathrm{mg} / \mathrm{L}$\\
c. $3.0 \mathrm{mg} / \mathrm{L}$\\
d. $4.0 \mathrm{mg} / \mathrm{L}$

  \item How do water agencies monitor the effectiveness of their filtration process?\\
a. Alkalinity\\
b. Conductivity\\
c. Turbidity\\
d. $\mathrm{pH}$

  \item What is the disinfectant byproduct caused by ozonation?\\
a. Trihalomethanes\\
b. Bromate\\
c. Chlorite\\
d. No DBP formation

  \item Haloacitic Acids are also known as\\
a. TTHM\\
b. $\mathrm{HOCL}$\\
c. Chlorite\\
d. HAA5

  \item What is the MCL for trihalomethanes?\\
a. $.10 \mathrm{mg} / \mathrm{L}$\\
b. $.06 \mathrm{mg} / \mathrm{L}$\\
c. $.08 \mathrm{mg} / \mathrm{L}$\\
d. $.12 \mathrm{mg} / \mathrm{L}$

  \item What is the MCL for Haloacitic Acids?\\
a. $100 \mathrm{ppb}$\\
b. $60 \mathrm{ppb}$\\
c. $80 \mathrm{ppb}$\\
d. $120 \mathrm{ppb}$ 12. What is the $\mathrm{MCL}$ for bromate?\\
a. $.010 \mathrm{mg} / \mathrm{L}$\\
b. $.020 \mathrm{mg} / \mathrm{L}$\\
c. $.030 \mathrm{mg} / \mathrm{L}$\\
d. $.040 \mathrm{mg} / \mathrm{L}$

  \item A treatment plant operator must fill a clearwell with $10,000 \mathrm{ft}^{3}$ of water in 90 minutes. What is the rate of flow expressed in GPM?\\
a. $111 \mathrm{GPM}$\\
b. $831 \mathrm{GPM}$\\
c. $181 \mathrm{GPM}$\\
d. $900 \mathrm{GPM}$

  \item A water tank has a capacity of 6MG. It is currently half full. It will take 6 hours to fill. What is the flow rate of the pump?\\
a. $3,333 \mathrm{GPM}$\\
b. $6,333 \mathrm{GPM}$\\
c. $8,333 \mathrm{GPM}$\\
d. $16,666 \mathrm{GPM}$

  \item A clearwell with the capacity of $2.5 \mathrm{MG}$ is being filled after a maintenance period. The flow rate is 2,500 GPM. The operator begins filling at 7 AM. At what time will the clearwell be full?\\
a. 10:00 PM\\
b. 10:40 PM\\
c. 11:00 PM\\
d. 11:40 PM

\end{enumerate}
\section{Chapter Review}
\begin{enumerate}
  \item The optimal coagulant dose is determined by a\\
a. Chlorine Test\\
b. Flocculation test\\
c. Jar Test\\
d. Coagulation test

  \item The most common primary coagulant is\\
a. Alum\\
b. Cationic polymer\\
c. Fluoride\\
d. Anionic polymer

  \item Bacteria and Viruses belong to a particle size known as\\
a. Suspended\\
b. Dissolved\\
c. Strained\\
d. Colloidal

  \item The purpose of coagulation is to\\
a. Increase filter run times\\
b. Increase sludge\\
c. Increase particle size\\
d. Destabilize colloidal particles

  \item The purpose of flocculation\\
a. Destabilize colloidal particles\\
b. Increase particle size\\
c. Decrease sludge\\
d. Decrease filter run times

  \item Primary coagulant aids used in treatment process are\\
a. Poly-aluminum chloride\\
b. Aluminum sulfate\\
c. Ferric chloride\\
d. All of the Above 7. Flocculation is used to enhance\\
a. Number of particle collisions to increase floc\\
b. Charge neutralization\\
c. Dispersion of chemicals in water\\
d. Settling speed of floc

  \item If there is a problem with floc formation, what would you consider changing?\\
a. Adjust coagulant dose\\
b. Stay the course\\
c. Adjust mixing intensity\\
d. Both $A$ \& $C$

  \item Which step in the treatment process is the shortest?\\
a. Filtration\\
b. Sedimentation\\
c. Flocculation\\
d. Coagulation

  \item To lower the $\mathrm{pH}$ for enhanced coagulation the operator will add\\
a. Chlorine\\
b. Sulfuric acid\\
c. Lime\\
d. Caustic Soda

  \item The flocculation process lasts how long?\\
a. Seconds\\
b. 5-10 minutes\\
c. 15-45 minutes\\
d. Over an hour

  \item The function of a flocculation basin is to\\
a. Settle colloidal particles\\
b. Destabilize colloidal particles\\
c. Mix chemicals\\
d. Allow suspended particles to grow 13. A treatment plant has a maximum output of $30 \mathrm{MGD}$ and doses ferric chloride at 75 $\mathrm{mg} / \mathrm{L}$. How many pounds of Ferric Chloride does the plant use in a day?\\
a. 18,765\\
b. 17,765\\
c. 19,765\\
d. 16,765

  \item A treatment plant uses 750 pounds of alum a day as it treats $15 \mathrm{MGD}$. What was the dose rate?\\
a. $4 \mathrm{mg} / \mathrm{L}$\\
b. $5 \mathrm{mg} / \mathrm{L}$\\
c. $6 \mathrm{mg} / \mathrm{L}$\\
d. $7 \mathrm{mg} / \mathrm{L}$

  \item A treatment plant operates at 1,500 gallons a minute and uses 500 pounds of alum a day. What is the alum dose?\\
a. $18 \mathrm{mg} / \mathrm{L}$\\
b. $28 \mathrm{mg} / \mathrm{L}$\\
c. $8 \mathrm{mg} / \mathrm{L}$\\
d. $38 \mathrm{mg} / \mathrm{L}$

\end{enumerate}
\section{Chapter Review}
\begin{enumerate}
  \item The treatment process that involves coagulation, flocculation, sedimentation, and filtration is known as\\
a. Direct filtration\\
b. Slow sand Filtration\\
c. Conventional treatment\\
d. Pressure filtration

  \item Sedimentation produces waste known as\\
a. Backwash water\\
b. Sludge\\
c. Waste water\\
d. Mud

  \item What kind of process is the sedimentation step?\\
a. Physical\\
b. Chemical\\
c. Biological\\
d. Direct

  \item The weirs at the effluent of a sedimentation basin are also called\\
a. Effluent weirs\\
b. Baffling\\
c. Launders\\
d. Spokes

  \item Sedimentation is used in water treatment plants to\\
a. Settle pathogenic material\\
b. Destabilize particles\\
c. Disinfect water\\
d. Reduce loading on Filters

  \item Scouring is a term that describes conditions in a sedimentation tank which\\
a. Could impact the rest of treatment process\\
b. Higher flow rates in the sludge zone\\
c. Re-suspends settle sludge\\
d. All of the above 7. The four zones in a Sedimentation basin include\\
a. Inlet, sedimentation, sludge, outlet\\
b. Inlet, filter, waste, outlet\\
c. Inlet, top, bottom, outlet\\
d. Surface, sedimentation, sludge, outlet

  \item 
  \begin{enumerate}
    \setcounter{enumii}{8}
    \item Short circuiting in a sedimentation basin could be caused by\\
a. Surface wind\\
b. Ineffective weir placement, or weirs covered in algae\\
c. Poor baffling in sedimentation inlet zone\\
d. All of the Above
  \end{enumerate}
  \item 
  \begin{enumerate}
    \setcounter{enumii}{9}
    \item How much solids should be removed during sedimentation?\\
a. $95 \%$ or more\\
b. $80-95 \%$\\
c. $70-80 \%$\\
d. $60-70 \%$
  \end{enumerate}
  \item The type of basin that includes coagulation and flocculation is\\
a. Rectangular\\
b. Triangular\\
c. Up-Flow\\
d. None of the above

\end{enumerate}
\section{Chapter Review}
\begin{enumerate}
  \item What is residual Chlorine?\\
a. Chlorine used to disinfect\\
b. The amount of chlorine after the demand has been satisfied\\
c. The amount of chlorine added before disinfection\\
d. Film left on DPD kit to measure residual

  \item When Chlorine reacts with natural organic matter in water it can create\\
a. Disinfectant by-products\\
b. Coliform bacteria\\
c. Chloroform\\
d. Calcium

  \item What are trihalomenthanes classified as\\
a. Salts\\
b. Inorganic compounds\\
c. Volatile organic compounds\\
d. Radio

  \item What disinfectant is used for emergency purposes and not utilized in the water treatment industry?\\
a. Chlorine\\
b. Iodine\\
c. Ozone\\
d. Chlorine Dioxide

  \item What is the disinfectant with the least killing power but that has the longest lasting residual?\\
a. Chlorine\\
b. Ozone\\
c. Chlorine Dioxide\\
d. Chloramines

  \item The active ingredient in household bleach is\\
a. Calcium hypochlorite\\
b. Calcium hydroxide\\
c. Sodium hypochlorite\\
d. Sodium hydroxide

\end{enumerate}
89 Water Treatment Plant Operation Processes I 7. Cryptosporidium is not resistant to this chemical\\
a. Ozone\\
b. Chlorine Dioxide\\
c. Chlorine\\
d. Both $A$ \& $B$

\begin{enumerate}
  \setcounter{enumi}{8}
  \item The Removal and inactivation requirement for Giardia is?\\
a. $99.9 \%$\\
b. $99.99 \%$\\
c. $99.00 \%$\\
d. $90 \%$

  \item If a coliform test is positive, how many repeat samples are required at a minimum?\\
a. None\\
b. 1\\
c. 3\\
d. Depends on the severity of the positive sample

  \item Your water system takes 75 coliform tests per month. This month there were 6 positive samples. What is the percentage of samples which tested positive? Did your system violate regulations?\\
a. $3 \%$ Yes\\
b. $5 \% \mathrm{No}$\\
c. $8 \%$ Yes\\
d. $10 \%$ No

\end{enumerate}
\section{Chapter Review}
\begin{enumerate}
  \item The form of Chlorine which is $100 \%$ available chlorine is?\\
a. Sodium Hypochlorite\\
b. Calcium Hypochlorite\\
c. Calcium Hydroxide\\
d. Gaseous Chlorine

  \item What is the minimum amount of chlorine residual required in the distribution system?\\
a. There is no minimum\\
b. $\mathrm{mg} / \mathrm{L}$\\
c. $0.2 \mathrm{mg} / \mathrm{L}$\\
d. $\mathrm{mg} / \mathrm{L}$

  \item What is the approximate $\mathrm{pH}$ range of sodium hypochlorite?\\
a. 4-5\\
b. 6-7\\
c. $9-11$\\
d. $12-14$

  \item What is the typical concentration of sodium hypochlorite utilized by water treatment professionals?\\
a. $5 \%$\\
b. $65 \%$\\
c. $100 \%$\\
d. $12.5 \%$

  \item Chlorine demand refers to\\
a. Chlorine in the system for a given time\\
b. The difference between chlorine applied and chlorine residual-usually caused by inorganics, organics, bacteria, algae, ammonia, etc.\\
c. Chlorine needed to produce a higher $\mathrm{pH}$\\
d. None of the above

  \item What is the most effective chlorine disinfectant?\\
a. Dichloramine\\
b. Trichloramine\\
c. Hypochlorite Ion\\
d. Hypochlorous acid 7. What can form when chlorine reacts with natural organic matter in source water?\\
a. Disinfectant by-products\\
b. Sulfur\\
c. Algae\\
d. Coliform bacteria

  \item What is the maximum withdrawal rate per day for a 150-pound chlorine cylinder?\\
a. There is no maximum\\
b. 20 pounds\\
c. 40 pounds\\
d. 50 pounds

  \item What kind of solution is used to check for a gas chlorine leak?\\
a. Sodium hydroxide\\
b. Ozone\\
c. Ammonia\\
d. Calcium hypochlorite

  \item Chlorine is\\
a. Heavier than air\\
b. Lighter than air\\
c. Brown in color\\
d. not harmful to your health

  \item Chlorine demand may vary due to\\
a. Chlorine demand always stays the same\\
b. Temperature\\
c. $\mathrm{pH}$\\
d. Both B and C

  \item What effect does high turbidity have on disinfection?\\
a. It can increase chlorine demand\\
b. It has no effect\\
c. It gives the water a milky appearance that will clear out after some time\\
d. You must increase the temperature of the water

\end{enumerate}
\section{Chapter Review}
\begin{enumerate}
  \item What is the target chlorine:ammonia ratio?\\
a. $2: 1$\\
b. $3: 1$\\
c. $4: 1$\\
d. $5: 1$

  \item What is the MCL for Nitrates?\\
a. $1 \mathrm{ppm}$\\
b. $10 \mathrm{ppm}$\\
c. $5 \mathrm{ppm}$\\
d. None of the above

  \item What is the atomic weight of Chlorine?\\
a. 70\\
b. 14\\
c. 65\\
d. 20

  \item What disinfectant has the longest lasting residual?\\
a. Ozone\\
b. Chlorine\\
c. Chloramine\\
d. Chlorine Dioxide

  \item What are some of the early indicators of Nitrification?\\
a. Lowering chlorine residual\\
b. Excess ammonia in treated water\\
c. Raise in bacterial heterotrophic plate counts\\
d. All of the above

  \item What are THMs classified as?\\
a. Turbidity\\
b. Radiological\\
c. Volatile Organic Chemicals\\
d. Salts 7. What method can operators employ to combat nitrification?\\
a. Lower residual chlorine target\\
b. Keep reservoir levels static\\
c. Minimize free ammonia in treated water\\
d. Increase water age

  \item How many times stronger is Chlorine compared to monochloramine?\\
a. 25 times\\
b. 20 times\\
c. 15 times\\
d. 5 times

\end{enumerate}
\section{Chapter Review}
\begin{enumerate}
  \item An organism used to indicate the possible presence of E. coli contamination is\\
a. Cryptosporidium\\
b. Giardia\\
c. Coliform\\
d. Brilliant Green Vile

  \item The presence-absence (P-A) test used for microbiological testing is also commonly referred to as\\
a. Multiple Tube Fermentation\\
b. Membrane Filtration\\
c. Confirmed Test\\
d. Colilert

  \item When testing for coliform bacteria with the multiple tube fermentation (MFT) method what is the best indicator for a positive test?\\
a. Color change\\
b. Gas bubble formation\\
c. Formation of a cyst\\
d. Formation of turbidity

  \item Coliform bacteria share many characteristics with pathogenic organisms. Which of the following is not true?\\
a. They survive longer in water\\
b. They grow in the intestines\\
c. There are less coliform than pathogenic organisms\\
d. They are still present in water without fecal contamination

  \item What is the second step in the multiple tube fermentation test?\\
a. Presumptive test\\
b. Negative test\\
c. Completed\\
d. Confirmed 6. What is the removal and deactivation requirement for Giardia?\\
a. $2 \log$\\
b. $3 \log$\\
c. $4 \mathrm{Log}$\\
d. There is no requirement

  \item The multiple barrier approach to water treatment includes removal through which method?\\
a. Filtration\\
b. Coagulation\\
c. Disinfection\\
d. a and c

  \item A pH reading of 7 is considered\\
a. Slightly acidic\\
b. Acidic\\
c. Basic\\
d. Neutral

  \item A higher than normal turbidity reading could signify\\
a. A change in water quality\\
b. Nothing. Keep operating as normal\\
c. Microbiological contamination\\
d. Both $A$ \& $C$

  \item What is the ingredient used during the second multiple tube fermentation test?\\
a. Colilert\\
b. MMO/MUG\\
c. Brilliant Green Vile\\
d. Chlorine

\end{enumerate}

\end{document}